\chapter{Products in dagger categories with complete ordered Hom-sets}

\fxwarning{This is a rough draft. It is not yet checked for errors.}

\begin{note}
  What I previously denoted $\prod F$ is now denoted $\bigodot^{\text{proj}}_{\sqcap} F$ (and
  likewise for $\mathord{\coprod}$). The other draft chapters referring to
  this chapter may be not yet updated.
\end{note}

\begin{prop}
  ~
  \fxwarning{Should we move this to volume 1?}
  \begin{enumerate}
    \item Every entirely defined monovalued morphism is metamonovalued and metacomplete.
    \item Every surjective injective morphism is metainjective and co-metacomplete.
  \end{enumerate}

\end{prop}

\begin{proof}
Let's prove the first (the second follows from duality):
  
Let $f$ be an entirely defined monovalued morphism.

$\left( \bigsqcap G \right) \circ f \sqsubseteq \bigsqcap_{g \in G} (g \circ
f)$ by monotonicity of composition.

Using the fact that $f$ is monovalued and entirely defined:

$\left( \bigsqcap_{g \in G} (g \circ f) \right) \circ f^{\dagger} \sqsubseteq
\bigsqcap_{g \in G} (g \circ f \circ f^{\dagger}) \sqsubseteq \bigsqcap G$;

$\bigsqcap_{g \in G} (g \circ f) \sqsubseteq \left( \bigsqcap_{g \in G} (g
\circ f) \right) \circ f^{\dagger} \circ f \sqsubseteq \left( \bigsqcap G
\right) \circ f$.

So $\left( \bigsqcap G \right) \circ f = \bigsqcap_{g \in G} (g \circ f)$.

Let $f$ be a entirely defined monovalued morphism.

$f \circ \left( \bigsqcup G \right) \sqsupseteq \bigsqcup_{g \in G} (f \circ
g)$ by monotonicity of composition.

Using the fact that $f$ is entirely defined and monovalued:

$f^{\dagger} \circ \left( \bigsqcup_{g \in G} (f \circ g) \right) \sqsupseteq
\bigsqcup_{g \in G} (f^{\dagger} \circ f \circ g) \sqsupseteq \bigsqcap G$;

$\bigsqcup_{g \in G} (f \circ g) \sqsupseteq f \circ f^{\dagger} \circ
\bigsqcup_{g \in G} (f \circ g) \sqsupseteq f \circ \left( \bigsqcup G
\right)$.

So $f \circ \left( \bigsqcup G \right) = \bigsqcup_{g \in G} (f \circ g)$.
\end{proof}

\section{General product in partially ordered dagger category}

To understand the below better, you can restrict your imagination to the case
when $\mathcal{C}$ is the category $\mathbf{Rel}$.

\subsection{Products}

Let $\mathcal{C}$ be a dagger category, each Hom-set of which is a complete
lattice (having order agreed with the dagger).

We will designate some morphisms as \emph{principal} and require that
principal morphisms are both metacomplete and co-metacomplete. (For a
particular example of the category $\mathbf{Rel}$, all morphisms are
considered principal.)

Let $\prod^{(Q)} X$ be an object for each indexed family $X$ of objects.

Let $\pi$ be a partial function mapping elements $X \in \dom \pi$ (which
consists of small indexed families of objects of $\mathcal{C}$) to indexed
families $\prod^{(Q)} X \rightarrow X_i$ of principal morphisms (called
\emph{projections}) for every $i \in \dom X$.

We will denote particular projections as $\pi^X_i$.

\begin{defn}
  If $\pi$ is defined at $\lambda
  j \in n : \Src F_j$ and $\lambda j \in n : \Dst F_j$, then
  \[ \bigodot^{\text{proj}}_{\sqcap} F = \bigsqcap_{i \in \dom F} ((\pi^{\Dst \circ
   F_{}}_i)^{\dagger} \circ F_i \circ \pi^{\Src \circ F}_i) . \]
\end{defn}

If $F_i:Y\to X_i$ for all~$i$ for some object~$Y$:
\[
  \prod^{\text{proj}}_{\sqcap} F = \bigsqcap_{i \in \dom F} ((\pi^{\Dst\circ F}_i)^{\dagger} \circ F_i) .
\]

If $F_i:X_i\to Y$ for all~$i$ for some object~$Y$:
\[
  \coprod^{\text{proj}}_{\sqcup} F = \bigsqcup_{i \in \dom F} (F_i \circ \pi^{\Src\circ F}_i) .
\]

\begin{rem}
  \begin{align*}
  (\pi^{\Dst \circ
   F_{}}_i)^{\dagger} \circ F_i \circ \pi^{\Src \circ F}_i \in \Hom \left(
  \prod^{(Q)}_{j \in n} \Src F_j , \prod^{(Q)}_{j \in n} \Dst F_j
  \right);\\
  (\pi^{\Dst \circ
   F_{}}_i)^{\dagger} \circ F_i \in \Hom \left(
  Y , \prod^{(Q)}_{j \in n} \Dst F_j
  \right);\\
F_i \circ \pi^{\Src \circ F}_i \in \Hom \left(
  \prod^{(Q)}_{j \in n} \Src F_j , Y \right).
  \end{align*}
  are properly defined and have the same sources and destination
  (whenever $i \in \dom F$ is), thus the meet in the formulas is
  properly defined.
\end{rem}

\begin{rem}
  Thus, for example,
\begin{multline*}
  F_0 \odot^{\text{proj}}_{\sqcap} F_1 = ((\pi^{(\Dst F_0 , \Dst
  F_1)}_0)^{\dagger} \circ F_0 \circ \pi^{(\Src F_0 , \Src
  F_1)}_0) \sqcap\\ ((\pi^{(\Dst F_0 , \Dst F_1)}_1)^{\dagger}
  \circ F_1 \circ \pi^{(\Src F_0 , \Src F_1)}_1)
\end{multline*}
  that is product is defined by a pure algebraic formula.
\end{rem}

\begin{lem}
$F\mapsto\bigsqcup_{i\in\dom F}\phi(F_i)$ for ordinal variadic~$F$ is infinitely
associative for any function~$\phi$ defined on all values~$F_i$.
\end{lem}

\begin{proof}
I will denote $t(F)=\bigsqcup_{i\in\dom F}\phi(F_i)$. We need
to prove:
\begin{widedisorder}
\item[$t(t\circ S)=t(\concat S)$]
$t(\concat S)=
\bigsqcup_{i\in\dom(\concat S)}\phi((\concat S)_i)=
\bigsqcup_{i\in\dom(\uncurry(S))}\phi((\uncurry(S))_i)$.

$t(t\circ S)=
\bigsqcup_{i\in\dom S}\phi(tS_i)=
\bigsqcup_{i\in\dom S}\bigsqcup_{j\in\dom F}\phi((S_i)_j)$.

So, obviously $t(t\circ S)=t(\concat S)$.

\item[$t(\llbracket x\rrbracket)=x$] Obvious.
\end{widedisorder}
\end{proof}

\begin{cor}
All three above defined products are infinitely associative
for ordinal variadic families~$F$.
\end{cor}

\begin{proof}
An obvious consequence taking into account duality.
\end{proof}

\begin{prop}
  $\bigodot^{\text{proj}}_{\sqcap} F = \max \setcond{ \Phi \in \Hom \left( \prod^{(Q)}_{j \in
  n} \Src F_j , \prod^{(Q)}_{j \in n} \Dst F_j \right)
  }{ \forall i \in n : \Phi \sqsubseteq (\pi^{\Dst\circ F}_i)^{\dagger} \circ F_i \circ \pi^{\Src\circ F}_i }$.
\end{prop}

\begin{proof}
  By definition of meet on a complete lattice.
\end{proof}

\begin{thm}
  Let $\pi^X_i$ be metamonovalued morphisms. Let~$I$ be an index set. If $S \in \subsets \prod_{i\in I}\Hom(A_i, B_i)$ for some objects $A_i$, $B_i$ where $i\in I$ then
  \begin{gather*}
    \bigsqcap_{f\in S} \bigodot^{\operatorname{proj}}_{\sqcap}f =
    \bigodot_{i\in I}\bigsqcap_{f\in S}f_i=
    \bigodot_{i\in I}\bigsqcap\Pr_i S;\\
    \bigsqcap_{f\in S} \prod^{\operatorname{proj}}_{\sqcap} f =
    \prod_{i\in I}\bigsqcap_{f\in S}f_i=
    \prod_{i\in I}\bigsqcap\Pr_i S;\\
    \bigsqcap_{f\in S} \coprod^{\operatorname{proj}}_{\sqcap} f =
    \coprod_{i\in I}\bigsqcap_{f\in S}f_i=
    \coprod_{i\in I}\bigsqcap\Pr_i S.
  \end{gather*}
\end{thm}

\begin{proof}
Let us consider for example the first formula (two others
are similar):
  \begin{align*}
  \bigsqcap_{f\in S} \bigodot^{\operatorname{proj}}_{\sqcap} & = \\
  \bigsqcap_{f\in S} \bigsqcap_{i\in I}((\pi^{\Dst\circ f_i}_i)^{\dagger} \circ f_i \circ \pi^{\Src\circ f_i}_i) & = \\
  \bigsqcap_{i\in I}\bigsqcap_{f\in S}((\pi^{\Dst\circ f_i}_i)^{\dagger} \circ f_i \circ \pi^{\Src\circ f_i}_i) & = \\
  \bigsqcap_{i\in I}((\pi^{\Dst\circ f_i}_i)^{\dagger} \circ \bigsqcap_{f\in S}f_i \circ \pi^{\Src\circ f_i}_i) & = \\
  \bigodot_{i\in I}\bigsqcap_{f\in S}f_i & = \\
  \bigodot_{i\in I}\bigsqcap\Pr_i S.
  \end{align*}
\end{proof}

\begin{cor}
~
\begin{enumerate}
\item
  $(a_0 \odot^{\operatorname{proj}}_{\sqcap} b_0) \sqcap (a_1 \odot^{\operatorname{proj}}_{\sqcap} b_1) = (a_0 \sqcap a_1)
  \odot^{\operatorname{proj}}_{\sqcap} (b_0 \sqcap b_1)$;
\item
  $(a_0 \times^{\operatorname{proj}}_{\sqcap} b_0) \sqcap (a_1 \times^{\operatorname{proj}}_{\sqcap} b_1) = (a_0 \sqcap a_1)
\times^{\operatorname{proj}}_{\sqcap} (b_0 \sqcap b_1)$;
\item
  $(a_0 \amalg^{\operatorname{proj}}_{\sqcap} b_0) \sqcap (a_1 \amalg^{\operatorname{proj}}_{\sqcap} b_1) = (a_0 \sqcap a_1)
  \amalg^{\operatorname{proj}}_{\sqcap} (b_0 \sqcap b_1)$;
\end{enumerate}
\end{cor}

\subsection{Product for endomorphisms}

Let $F$ is an indexed family of endomorphisms of $\mathcal{C}$.

I will denote $\Ob f$ the object (source and destination) of an
endomorphism $f$.

Let also $\pi^X_i$ be a monovalued entirely defined morphism (for each $i \in
\dom F$).

Then
\begin{gather*}
\bigodot^{\text{proj}}_{\sqcap} F = \bigsqcap_{i \in \dom F} ((\pi^{\lambda j \in n :
\Ob F_j}_i)^{\dagger} \circ F_i \circ \pi^{\lambda j \in n : \Ob
F_j}_i);\\
\bigodot^{\text{proj}}_{\sqcup} F = \bigsqcup_{i \in \dom F} ((\pi^{\lambda j \in n :
\Ob F_j}_i)^{\dagger} \circ F_i \circ \pi^{\lambda j \in n : \Ob
F_j}_i)
\end{gather*}
(if $\pi$ is defined at $\lambda j \in n : \Ob F_j$).

Abbreviate $\pi_i = \pi^{\lambda j \in n : \Ob F_j}_i$.

So
\begin{gather*}
\bigodot^{\text{proj}}_{\sqcap} F = \bigsqcap_{i \in \dom F} ((\pi_i)^{\dagger} \circ
F_i \circ \pi_i);\\
\bigodot^{\text{proj}}_{\sqcup} F = \bigsqcup_{i \in \dom F} ((\pi_i)^{\dagger} \circ
F_i \circ \pi_i).
\end{gather*}

$\bigodot^{\text{proj}}_{\sqcap} F = \max \setcond{ \Phi \in \End \left( \prod^{(Q)}_{j \in n}
\Ob F_j \right) }{ \forall i \in n : \Phi
\sqsubseteq (\pi_i)^{\dagger} \circ F_i \circ \pi_i }$.

$\bigodot^{\text{proj}}_{\sqcup} F = \min \setcond{ \Phi \in \End \left( \prod^{(Q)}_{j \in n}
\Ob F_j \right) }{ \forall i \in n : \Phi
\sqsupseteq (\pi_i)^{\dagger} \circ F_i \circ \pi_i }$.

Taking into account that $\pi_i$ is a monovalued entirely defined morphism, we
get:

\begin{obvious}
$\bigodot^{\text{proj}}_{\sqcap} F = \max \setcond{ \Phi \in \End \left( \prod^{(Q)}_{j \in
n} \Ob F_j \right) }{ \forall i \in n : \pi_i
\in \mathrm{C} (\Phi , F_i) }$.
\end{obvious}

\begin{obvious}
$\bigodot^{\text{proj}}_{\sqcup} F = \min \setcond{ \Phi \in \End \left( \prod^{(Q)}_{j \in
n} \Ob F_j \right) }{ \forall i \in n : \pi_i
\in \mathrm{C}_{\ast} (\Phi , F_i) }$.
\end{obvious}

\begin{rem}
  The above formulas may allow to define the product for non-dagger categories
  (but only for endomorphisms). In this writing I don't introduce a notation
  for this, however.
\end{rem}

\begin{cor}
  $\pi_i \in \mathrm{C} \left( \bigodot^{\text{proj}}_{\sqcap} F , F_i \right)$ and
  $\pi_i \in \mathrm{C}_{\ast} \left( \bigodot^{\text{proj}}_{\sqcup} F , F_i \right)$
  for every $i \in \dom F$.
\end{cor}

\subsection{Category of continuous morphisms}

\begin{defn}
  The category $\cont (\mathcal{C})$ is defined as follows:
  \begin{itemize}
    \item Objects are endomorphisms of the category $\mathcal{C}$.
    
    \item Morphisms are triples $(f , a , b)$ where $a$ and $b$ are objects
    and $f : \Ob a \rightarrow \Ob b$ is an entirely defined
    monovalue principal morphism of the category $\mathcal{C}$ such that $f
    \in \mathrm{C} (a , b)$ (in other words, $f \circ a \sqsubseteq b \circ
    f$).
    
    \item Composition of morphisms is defined by the formula $(g , b , c)
    \circ (f , a , b) = (g \circ f , a , c)$.
    
    \item Identity morphisms are $(a , a , 1^{\mathcal{C}}_a)$.
  \end{itemize}
\end{defn}

It is really a category:

\begin{proof}
  We need to prove that: composition of morphisms is a morphism, composition
  is associative, and identity morphisms can be canceled on the left and on
  the right.
  
  That composition of morphisms is a morphism by properties of generalized
  continuity.
  
  That composition is associative is obvious.
  
  That identity morphisms can be canceled on the left and on the right is
  obvious.
\end{proof}

\begin{rem}
  The ``physical'' meaning of this category is:
  \begin{itemize}
    \item Objects (endomorphisms of $\mathcal{C}$) are spaces.
    
    \item Morphisms are continuous functions between spaces.
    
    \item $f \circ a \sqsubseteq b \circ f$ intuitively means that $f$
    combined with an infinitely small is less than infinitely small combined
    with $f$ (that is $f$ is continuous).
  \end{itemize}
\end{rem}

\begin{defn}
  $\pi^{\cont (\mathcal{C})}_i = \left( \bigodot^{\text{proj}}_{\sqcap} F , F_i ,
  \pi_i \right)$.
\end{defn}

\begin{prop}
  $\pi_i$ are continuous, that is $\pi^{\cont}
  (\mathcal{C})_i$ are morphisms.
\end{prop}

\begin{proof}
  We need to prove $\pi_i \in \mathrm{C} \left( \bigodot^{\text{proj}}_{\sqcap} F , F_i \right)$
  but that was proved above.
\end{proof}

Let further $\mathcal{C}$ have sets as objects and $\prod^{(Q)}X=\prod X$ for an indexed family~$X$ of sets and $\pi_i = \Pr_i$ (for $i \in \dom F$).

\begin{lem}
  $f \in \Hom_{\cont (\mathcal{C})} \left( Y ,
  \bigodot^{\text{proj}}_{\sqcap} F \right)$ is continuous iff all $\pi_i \circ f$ are continuous.
\end{lem}

\begin{proof}
  ~
  \begin{description}
    \item[$\Rightarrow$] Let $f \in \Hom_{\cont
    (\mathcal{C})} \left( Y , \bigodot^{\text{proj}}_{\sqcap} F \right)$. Then $f \circ Y
    \sqsubseteq \left( \bigodot^{\text{proj}}_{\sqcap} F \right) \circ f$; $\pi_i \circ f \circ Y
    \sqsubseteq \pi_i \circ \left( \bigodot^{\text{proj}}_{\sqcap} F \right) \circ f$; $\pi_i
    \circ f \circ Y \sqsubseteq \left( \bigodot^{\text{proj}}_{\sqcap} F \right) \circ \pi_i \circ
    f$. Thus $\pi_i \circ f$ is continuous.
    
    \item[$\Leftarrow$] Let all $\pi_i \circ f$ be continuous. Then
    $\pi^{\cont (\mathcal{C})}_i \circ f \in
    \Hom_{\cont (\mathcal{C})} (Y , F_i)$;
    $\pi^{\cont (\mathcal{C})}_i \circ f \circ Y \sqsubseteq
    F_i \circ \pi^{\cont (\mathcal{C})}_i \circ f$. We need
    to prove $Y \sqsubseteq f^{\dagger} \circ \left( \bigodot^{\text{proj}}_{\sqcap} F \right)
    \circ f$ that is
    \[ Y \sqsubseteq f^{\dagger} \circ \bigsqcap_{i \in n} ((\pi_i)^{\dagger}
       \circ F_i \circ \pi_i) \circ f \]
    for what is enough (because $f$ is metamonovalued)
    \[ Y \sqsubseteq \bigsqcap_{i \in n} (f^{\dagger} \circ (\pi_i)^{\dagger}
       \circ F_i \circ \pi_i \circ f) \]
    what follows from $Y \sqsubseteq \bigsqcap_{i \in n} (f^{\dagger} \circ
    (\pi_i)^{\dagger} \circ \pi_i \circ f \circ Y)$ what is obvious.
  \end{description}
\end{proof}

\begin{thm}\label{cont-pr-pr}
$\prod^{\text{proj}}_{\sqcap}$ together with $\pi$ is a categorical product in the category $\cont (\mathcal{C})$.
\end{thm}

\begin{proof}
Check
\url{http://math.stackexchange.com/questions/102632/how-to-check-whether-it-is-a-direct-product/102677\#102677}

I will denote $(\prod f)x=\prod_{i\in\dom f}f_i x$ for an
indexed family~$f$ of functions.

We need to prove:
\begin{enumerate}
\item $\pi_k\circ\prod f=f_k$;
\item $\prod_{i\in\dom f}(\pi_i\circ f)=f$.
\end{enumerate}
But it follows from the fact that $\pi_i=\Pr_i$.
\end{proof}

\section{On duality}

We will consider duality where both the category $\mathcal{C}$ and orders on
Hom-sets are replaced with their dual. I will denote $A
\xleftrightarrow{\dual} B$ when two formulas $A$ and $B$ are dual with
this duality.

\begin{prop}
  $f \in \mathrm{C} (\mu, \nu) \xleftrightarrow{\dual} f^{\dagger}
  \in \mathrm{C} (\nu^{\dagger} , \mu^{\dagger})$.
\end{prop}

\begin{proof}
  $f \in \mathrm{C} (\mu, \nu) \Leftrightarrow f \circ \mu
  \sqsubseteq \nu \circ f \xleftrightarrow{\dual} \mu^{\dagger}
  \circ f^{\dagger} \sqsupseteq f^{\dagger} \circ^{\dagger} \nu^{- 1}
  \Leftrightarrow f^{\dagger} \in \mathrm{C} (\nu^{\dagger} ,
  \mu^{\dagger})$.
\end{proof}

$f \text{ is entirely defined} \Leftrightarrow f^{\dagger} \circ f \sqsupseteq
1_{\Src f} \xleftrightarrow{\dual} f^{\dagger} \circ f \sqsubseteq
1_{\Src f} \Leftrightarrow f \text{ is injective} \Leftrightarrow
f^{\dagger} \text{ is monovalued}$.

$f \text{ is monovalued} \Leftrightarrow f \circ f^{\dagger} \sqsubseteq
1_{\Dst f} \xleftrightarrow{\dual} f \circ f^{\dagger} \sqsupseteq
1_{\Dst f} \Leftrightarrow f \text{ is surjective} \Leftrightarrow
f^{\dagger} \text{ is entirely defined}$.

\section{Dual products}

The below is the dual of the above, proofs are omitted as they are dual.

Let $\coprod^{(Q)}X$ be an object for each indexed family~$X$ of objects.

I will denote \emph{coprincipal} morphisms~$f^{\dagger}$ where $f$~is principal. (Usually there is no distinction between
principal and coprincipal.)

Let $\iota$ be a partial function mapping elements $X\in\dom\iota$ (which consist of small indexed families of objects of~$\mathcal{C}$) to indexed families $X_i\to\coprod^{(Q)}X$ of coprincipal morphisms (called \emph{injections}) for every $i\in\dom X$.

We will denote particular morphisms as $\iota_i^X$.

If (but we won't assume this below) $\iota_i =
(\pi_i)^{\dagger}$. We have the above equivalent to $\pi_i$ being principal.

We will define $\bigodot^{\operatorname{inj}}$, $\prod^{\operatorname{inj}}$, $\coprod^{\operatorname{inj}}$
by analogy with $\operatorname{proj}$ counterparts replacing~$\pi$ by~$\iota^{\dagger}$.

We will also define $\bigodot^{\operatorname{proj}}_{\sqcup}$, $\bigodot^{\operatorname{inj}}_{\sqcup}$, etc.\ by replacing~$\bigsqcap$ by~$\bigsqcup$.

\subsection{Dual products for endomorphisms}

\begin{prop}
  $\bigodot^{\operatorname{inj}}_{\sqcup} F = \min \setcond{ \Phi \in \End \left( \coprod^{(Q)}_{j
  \in n} \Ob F_j \right) }{ \forall i \in n :
  \Phi \sqsupseteq \iota^{\lambda j \in n : \Src F_j}_i \circ
  F_i^{\dagger} \circ (\iota^{\lambda j \in n : \Dst F_j}_i)^{\dagger} }$.
\end{prop}

\begin{proof}
  By duality.
\end{proof}

Let $F$ be an indexed family of endomorphisms of $\mathcal{C}$.

\begin{defn}
  $\bigodot^{\operatorname{inj}}_{\sqcup} F = \bigsqcup_{i \in \dom F} (\iota^{\lambda j \in n :
  \Ob F_j}_i \circ F_i^{\dagger} \circ (\iota^{\lambda j \in n :
  \Ob F_j}_i)^{\dagger})$.
\end{defn}

Abbreviate $\iota_i = \iota^{\lambda j \in n : \Ob F_j}_i$.

So $\bigodot^{\operatorname{inj}}_{\sqcup} F = \bigsqcup_{i \in \dom F} (\iota_i \circ F_i^{\dagger}
\circ (\iota_i)^{\dagger})$.

$\bigodot^{\operatorname{inj}}_{\sqcup} F = \min \setcond{ \Phi \in \End \left( \coprod^{(Q)}_{j \in n}
\Ob F_j \right) }{ \forall i \in n : \Phi
\sqsupseteq \iota_i \circ F_i^{\dagger} \circ (\iota_i)^{\dagger} }$.

Taking into account that $\iota_i$ is a monovalued entirely defined morphism,
we get:

\begin{obvious}
$\coprod^{(L)} = \min \setcond{ \Phi \in \End \left( \coprod^{(Q)}_{j \in
n} \Ob F_j \right) }{ \forall i \in n : \iota_i
\in \mathrm{C} (F_i^{\dagger} , \Phi) }$.{\hspace*{\fill}}{\medskip}
\end{obvious}

\begin{cor}
  $\iota_i \in \mathrm{C} \left( F_i , \coprod^{(L)} F \right)$ for every $i
  \in \dom F$.
\end{cor}

\begin{rem}
The last two theorems don't require that our category is dagger. I omit the proof.
\end{rem}

\subsection{Category of continuous morphisms}

Let $\iota_i$ be canonical injections.

\begin{defn}
$\iota^{\cont(\mathcal{C})}_i = \left(F_i,\bigodot^{\operatorname{inj}}_{\sqcup}F,\iota_i\right)$.
\end{defn}

\begin{obvious}
$\iota_i$ are continuous that is $\iota^{\cont(\mathcal{C})}_i$ are morphisms.
\end{obvious}

\begin{thm}
$\coprod^{\operatorname{inj}}_{\sqcup}$ together with $\iota$ is a categorical coproduct in the category~$\cont(\mathcal{C})$.
\end{thm}

\begin{proof}
Dual to theorem~\ref{cont-pr-pr}.
\end{proof}

\section{Applying this to the theory of funcoids and reloids}

\subsection{Funcoids}

\begin{defn}
  $\mathbf{Fcd} \eqdef \cont \mathsf{FCD}$.
\end{defn}

Let $F$ be a family of endofuncoids.

\subsection{Reloids}

\begin{defn}
  $\mathbf{Rld} \eqdef \cont \mathsf{RLD}$.
\end{defn}

Let $F$ be a family of endoreloids.

It is trivial?? that for uniform spaces infimum product of reloids coincides
with product uniformilty.

\section{Initial and terminal objects}

\subsection{Of category~$\mathbf{Fcd}$}

Initial object of $\mathbf{Fcd}$ is the endofuncoid
$\bot^{\mathsf{FCD} (\emptyset , \emptyset)}$. It is
initial because it has precisely one morphism $o$ (the empty set considered as
a function) to any object $Y$. $o$ is a morphism because $o \circ
\bot^{\mathsf{FCD} (\emptyset , \emptyset)} \sqsubseteq Y
\circ o$.

\begin{prop}
  Terminal objects of $\mathbf{Fcd}$ are exactly
  $\uparrow^{\mathscr{F}} \{ \ast \} \times^{\mathsf{FCD}}
  \uparrow^{\mathscr{F}} \{ \ast \} = \uparrow^{\mathsf{FCD}} \{ (\ast
  , \ast) \}$ where $\ast$ is an arbitrary point.
\end{prop}

\begin{proof}
  In order for a function $f : X \rightarrow \uparrow^{\mathsf{FCD}} \{
  (\ast , \ast) \}$ be a morphism, it is required exactly $f \circ X
  \sqsubseteq \uparrow^{\mathsf{FCD}} \{ (\ast , \ast) \} \circ f$
  
  $f \circ X \sqsubseteq (f^{- 1} \circ \uparrow^{\mathsf{FCD}} \{
  (\ast , \ast) \})^{- 1}$; $f \circ X \sqsubseteq (\{ \ast \}
  \times^{\mathsf{FCD}} \langle f^{- 1} \rangle \{ \ast \})^{- 1}$; $f
  \circ X \sqsubseteq \langle f^{- 1} \rangle \{ \ast \}
  \times^{\mathsf{FCD}} \{ \ast \}$ what true exactly when $f$ is a
  constant function with the value $\ast$.
\end{proof}

If $n = \emptyset$ then; $\bigodot^{\text{proj}}_{\sqcap} \emptyset = \prod^{\text{proj}}_{\sqcap} \emptyset = \coprod^{\text{proj}}_{\sqcap} \emptyset = \max
\mathsf{FCD} (\{\emptyset\}, \{\emptyset\}) = \uparrow^{\mathscr{F}} \{ \emptyset \}
\times^{\mathsf{FCD}} \uparrow^{\mathscr{F}} \{ \emptyset \} =
\uparrow^{\mathsf{FCD}} \{ (\emptyset , \emptyset) \}$.

\subsection{Of category~$\mathbf{Rld}$}

Initial object of $\mathbf{Rld}$ is the endofuncoid
$\bot^{\mathsf{RLD} (\emptyset , \emptyset)}$. It is
initial because it has precisely one morphism $o$ (the empty set considered as
a function) to any object $Y$. $o$ is a morphism because $o \circ
\bot^{\mathsf{RLD} (\emptyset , \emptyset)} \emptyset \sqsubseteq Y
\circ o$.

\begin{prop}
  Terminal objects of $\mathbf{Rld}$ are exactly
  $\uparrow^{\mathscr{F}} \{ \ast \} \times^{\mathsf{RLD}}
  \uparrow^{\mathscr{F}} \{ \ast \} = \uparrow^{\mathsf{RLD}} \{ (\ast
  , \ast) \}$ where $\ast$ is an arbitrary point.
\end{prop}

\begin{proof}
  In order for a function $f : X \rightarrow \uparrow^{\mathsf{RLD}} \{
  (\ast , \ast) \}$ be a morphism, it is required exactly $f \circ X
  \sqsubseteq \uparrow^{\mathsf{RLD}} \{ (\ast , \ast) \} \circ f$
  
  $f \circ X \sqsubseteq (f^{- 1} \circ \uparrow^{\mathsf{RLD}} \{
  (\ast , \ast) \})^{- 1}$; $f \circ X \sqsubseteq (\{ \ast \}
  \times^{\mathsf{RLD}} \langle f^{- 1} \rangle \{ \ast \})^{- 1}$; $f
  \circ X \sqsubseteq \langle f^{- 1} \rangle \{ \ast \}
  \times^{\mathsf{RLD}} \{ \ast \}$ what true exactly when $f$ is a
  constant function with the value $\ast$.
\end{proof}

If $n = \emptyset$ then; $\bigodot^{\text{proj}}_{\sqcap} \emptyset = \prod^{\text{proj}}_{\sqcap} \emptyset = \coprod^{\text{proj}}_{\sqcap} \emptyset = \max
\mathsf{RLD} (\{\emptyset\}, \{\emptyset\}) = \uparrow^{\mathscr{F}} \{ \emptyset \}
\times^{\mathsf{RLD}} \uparrow^{\mathscr{F}} \{ \emptyset \} =
\uparrow^{\mathsf{RLD}} \{ (\emptyset , \emptyset) \}$.

\section{Canonical product and subatomic product}

\fxwarning{Confusion between filters on products and multireloids.}

\begin{prop}
  $\Pr^{\mathsf{RLD}}_i |_{\mathfrak{F} (Z)} = \langle \pi_i \rangle$
  for every index $i$ of a cartesian product $Z$.
\end{prop}

\begin{proof}
  If $\mathcal{X} \in \mathfrak{F} (Z)$ then $(\Pr^{\mathsf{RLD}}_i
  |_{\mathfrak{F} (Z)}) \mathcal{X} = \Pr^{\mathsf{RLD}}_i  \mathcal{X}
  = \bigsqcap^{\mathscr{F}} \rsupfun{\Pr_i} \mathcal{X} =
  \bigsqcap \langle \pi_i \rangle \up \mathcal{X} = \langle \pi_i
  \rangle \mathcal{X}$.
\end{proof}

\begin{prop}
  $\prod^{(A)} F = \bigsqcap_{i \in n} \left( \left( \pi^{\mathsf{FCD}
  \left( \prod_{i \in n} \Dst F \right)}_i \right)^{- 1} \circ F_i \circ
  \pi^{\mathsf{FCD} \left( \prod_{i \in n} \Src F \right)}_i
  \right)$.
\end{prop}

\begin{proof}
  $a \mathrel{\left[ \prod^{(A)} F \right]} b \Leftrightarrow \forall i \in
  \dom F : \Pr^{\mathsf{RLD}}_i a \mathrel{[F_i]}
  \Pr^{\mathsf{RLD}}_i b \Leftrightarrow \forall i \in \dom F :
  \left\langle \pi^{\mathsf{FCD} \left( \prod_{i \in n}
  \Dst F \right)}_i \right\rangle a \mathrel{[F_i]}
  \left\langle \pi^{\mathsf{FCD} \left( \prod_{i \in n} \Src F
  \right)}_i \right\rangle b \Leftrightarrow \forall i \in \dom F : a
  \mathrel{\left[ \left( \pi^{\mathsf{FCD} \left( \prod_{i \in n}
  \Dst F \right)}_i \right)^{- 1} \circ F_i \circ
  \pi^{\mathsf{FCD} \left( \prod_{i \in n} \Src F \right)}_i
  \right]} b \Leftrightarrow a \mathrel{\left[ \bigsqcap_{i \in n} \left(
  \left( \pi^{\mathsf{FCD} \left( \prod_{i \in n} \Dst F
  \right)}_i \right)^{- 1} \circ F_i \circ \pi^{\mathsf{FCD} \left(
  \prod_{i \in n} \Src F \right)}_i \right) \right]} b$ for ultrafilters
  $a$ and $b$.
\end{proof}

\begin{cor}
  $\bigodot^{\text{proj}}_{\sqcap} F = \prod^{(A)} F$ is $F$ is a small indexed family of
  funcoids.
\end{cor}

\section{Further plans}

Coordinate-wise continuity.

\section{Cartesian closedness}

We are not only to prove (or maybe disprove) that our categories are cartesian closed, but also to find (if any) explicit formulas for exponential transpose and evaluation.

''Definition'' A category is //cartesian closed// iff:
\begin{enumerate}
\item It has finite products.
\item For each objects $A$, $B$ is given an object $\operatorname{HOM} ( A , B)$ (//exponentiation//) and a morphism $\varepsilon_{A, B} : \operatorname{HOM} ( A , B) \times A \rightarrow B$.
\item For each morphism $f : Z \times A \rightarrow B$ there is given a morphism (//exponential transpose//) $\sim f : Z \rightarrow \operatorname{HOM} ( A , B)$.
\item $\varepsilon_{B,C} \circ ( \sim f \times 1_A) = f$ for $f : A \rightarrow B \times C$.
\item $\sim ( \varepsilon_{B,C} \circ ( g \times 1_A)) = g$ for $g : A \rightarrow \operatorname{HOM} ( B , C)$.
\end{enumerate}

We will also denote $f\mapsto (-f)$ the reverse of the bijection $f\mapsto (\sim f)$.

Our purpose is to prove (or disprove) that categories $\mathbf{Dig}$, $\mathbf{Fcd}$, and $\mathbf{Rld}$ are cartesian closed. Note that they have finite (and even infinite) products is already proved.

Alternative way to prove:
you can prove that the functor $-\times B$ is left adjoint to the exponentiation $-^B$ where the counit is given by the evaluation map.

\subsection{Definitions}

Categories $\mathbf{Dig}$, $\mathbf{Fcd}$, and $\mathbf{Rld}$ are respectively categories of:
\begin{enumerate}
\item discretely continuous maps between digraphs;
\item (proximally) continuous maps between endofuncoids;
\item (uniformly) continuous maps between endoreloids.
\end{enumerate}

''Definition'' //Digraph// is an endomorphism of the category $\mathbf{Rel}$.

For a digraph $A$ we denote $\operatorname{Ob} A$ the set of vertexes or $A$ and $\operatorname{GR} A$ the set of edges or $A$.

''Definition'' Category $\mathbf{Dig}$ of digraphs is the category whose objects are digraphs and morphisms are discretely continuous maps between digraphs. That is morphisms from a digraph $\mu$ to a digraph $\nu$ are functions (or more precisely morphisms of $\mathbf{Set}$) $f$ such that $f \circ \mu \sqsubseteq \nu \circ f$ (or equivalently $\mu \sqsubseteq f^{- 1} \circ \nu \circ f$ or equivalently $f \circ \mu \circ f^{- 1} \sqsubseteq \nu$).

''Remark'' Category of digraphs is sometimes defined in an other (non equivalent) way, allowing multiple edges between two given vertices.

\subsection{Conjectures}

\begin{conjecture}
  The categories $\mathbf{Fcd}$ and $\mathbf{Rld}$ are
  cartesian closed (actually two conjectures).
\end{conjecture}

\url{http://mathoverflow.net/questions/141615/how-to-prove-that-there-are-no-exponential-object-in-a-category}
suggests to investigate colimits to prove that there are no exponential
object.

Our purpose is to prove (or disprove) that categories $\mathbf{Dig}$, $\mathbf{Fcd}$, and $\mathbf{Rld}$ are cartesian closed. Note that they have finite (and even infinite) products is already proved.

Alternative way to prove:
you can prove that the functor $-\times B$ is left adjoint to the exponentiation $-^B$ where the counit is given by the evaluation map.

See \url{http://www.springer.com/us/book/9780387977102} for another way to prove Cartesian closedness.

\subsection{Category Dig is cartesian closed}

Category of digraphs is the simplest of our three categories and it is easy to demonstrate that it is cartesian closed. I demonstrate cartesian closedness of $\mathbf{Dig}$ mainly with the purpose to show a pattern similarly to which we may probably demonstrate our two other categories are cartesian closed.

Let $G$ and $H$ be digraphs:
\begin{itemize}
\item $\operatorname{Ob} \operatorname{HOM} ( G , H) = ( \operatorname{Ob} H)^{\operatorname{Ob} G}$;
\item $( f , g) \in \operatorname{GR} \operatorname{HOM} ( G , H) \Leftrightarrow \forall ( v , w) \in \operatorname{GR} G : ( f ( v) , g ( w)) \in \operatorname{GR} H$ for every $f, g \in \operatorname{Ob} \operatorname{HOM} ( G , H) = ( \operatorname{Ob} H)^{\operatorname{Ob} G}$;
\end{itemize}

$\operatorname{GR} 1_{\operatorname{HOM} ( B , C)} = \operatorname{id}_{\operatorname{Ob} \operatorname{HOM} ( B , C)} = \operatorname{id}_{( \operatorname{Ob} H)^{\operatorname{Ob} G}}$

Equivalently

$( f , g) \in \operatorname{GR} \operatorname{HOM} ( G , H) \Leftrightarrow \forall ( v , w) \in \operatorname{GR} G : g \circ \{ ( v , w) \} \circ f^{- 1} \subseteq \operatorname{GR} H$

$( f , g) \in \operatorname{GR} \operatorname{HOM} ( G , H) \Leftrightarrow g \circ ( \operatorname{GR} G) \circ f^{- 1} \subseteq \operatorname{GR} H$

$( f , g) \in \operatorname{GR} \operatorname{HOM} ( G , H) \Leftrightarrow \langle f \times^{( C)} g \rangle \operatorname{GR} G \subseteq \operatorname{GR} H$

The transposition (the isomorphism) is uncurrying.

$\sim f = \lambda a \in Z \lambda y \in A : f ( a , y)$ that is $( \sim f) ( a) ( y) = f ( a , y)$.

$( - f) ( a , y) = f ( a) ( y)$

If $f : A \times B \rightarrow C$ then $\sim f : A \rightarrow \operatorname{HOM} ( B , C)$

''Proposition'' Transposition and its inverse are morphisms of $\mathbf{Dig}$.

''Proof'' It follows from the equivalence $\sim f : A \rightarrow \operatorname{HOM} ( B , C) \Leftrightarrow \forall x, y : ( x A y \Rightarrow ( \sim f) x ( \operatorname{HOM} ( B , C))  ( \sim f) y) \Leftrightarrow \\ \forall x, y : ( x A y \Rightarrow \forall ( v , w) \in B : ( ( \sim f) x v , ( \sim f) y w) \in C) \Leftrightarrow \\ \forall x, y, v, w : ( x A y \wedge v B w \Rightarrow ( ( \sim f) x v , ( \sim f) y w) \in C) \Leftrightarrow \\ \forall x, y, v, w : ( ( x , v)  ( A \times B)  ( y , w) \Rightarrow ( f ( x , v) , f ( y , w)) \in C) \Leftrightarrow f : A \times B \rightarrow C$.

Evaluation $\varepsilon : \operatorname{HOM} ( G , H) \times G \rightarrow H$ is defined by the formula:

Then evaluation is $\varepsilon_{B, C} = \mathop{\sim} 1_{\operatorname{HOM}(B,C)}$.

So $\varepsilon_{B, C} ( p , q) = ( \mathop{\sim} 1_{\operatorname{HOM}(B,C)}) ( p , q) = 1_{\operatorname{HOM}(B,C)} ( p) ( q) = p ( q)$.

''Proposition'' Evaluation is a morphism of $\mathbf{Dig}$.

''Proof'' Because $\varepsilon_{B, C} ( p , q) = \mathop{\sim} 1_{\operatorname{HOM}(B,C)}$.

It remains to prove:
\begin{itemize}
\item $\varepsilon_{B, C} \circ ( \sim f \times 1_{A}) = f$ for $f : A \rightarrow B \times C$;
\item $\sim ( \varepsilon_{B, C} \circ ( g \times 1_{A})) = g$ for $g : A \rightarrow \operatorname{HOM} ( B , C)$.
\end{itemize}

''Proof'' $\varepsilon_{B, C} ( \sim f \times 1_{A}) ( a , p) = \varepsilon_{B, C} ( ( \sim f) a , p) = ( \sim f) a p = f ( a , p)$. So $\varepsilon_{B, C} \circ ( \sim f \times 1_{A}) = f$.

  $(\sim ( \varepsilon_{B, C} \circ ( g \times 1_{A}))) ( p) ( q) = ( \varepsilon_{B, C} \circ ( g \times 1_{A})) ( p , q) = \varepsilon_{B, C} ( g \times 1_{A}) ( p , q) = \varepsilon_{B, C} ( g p , q) = g ( p) ( q)$. So $\sim ( \varepsilon_{B, C} \circ ( g \times 1_{A})) = g$.

\subsection{By analogy with the proof that Dig is cartesian closed}

The most obvious way for proof attempt that $\mathbf{Fcd}$ is cartesian closed is an analogy with the proof that
$\mathbf{Dig}$ is cartesian closed.

Consider the long formula above. The proof would arise if we replace $x$ and $y$ in this formula with filters and operations and relations on set element with operations and relations on filters.

This proof could be simplified in either of two ways:
\begin{itemize}
\item replace $x$ and $y$ with ultrafilters, see [[Proof for Fcd using ultrafilters]];
\item replace $x$ and $y$ with sets (principal filter), see [[Proof for Fcd using sets]].
\end{itemize}

This is not quite easy however, because we need to calculate uncurrying for a entirely defined monovalued principal funcoid (what is essentially the same as a function of a $\mathbf{Set}$-morphisms) taking either ultrafilters or principal filters as arguments. Such (generalized) uncurrying is not quite easy.

To sum what we need to prove:
\begin{itemize}
\item Transposition is a morphism.
\item Evaluation is a morphism.
\item $\varepsilon_{B,C} \circ ( \sim f \times 1_A) = f$ for $f : A \rightarrow B \times C$.
\item $\sim ( \varepsilon_{B,C} \circ ( g \times 1_A)) = g$ for $g : A \rightarrow \operatorname{HOM} ( B , C)$.
\end{itemize}

\subsection{Attempt to describe exponentials in Fcd}

\begin{itemize}
\item Exponential object $\operatorname{HOM}(A,B)$ is the following endofuncoid:
\item\begin{itemize}
\item Object $\operatorname{Ob}\operatorname{HOM}(A,B) = (\operatorname{Ob} B)^{\operatorname{Ob} A}$;
\item Graph is $\operatorname{GR} \operatorname{HOM} ( A , B) = \uparrow^{\mathsf{FCD}} \setcond{ ( f , g) }{ f, g \in (\operatorname{Ob}B)^{\operatorname{Ob} A} \wedge \uparrow^{\mathsf{FCD}}g \circ A \circ \uparrow^{\mathsf{FCD}}f^{- 1} \sqsubseteq B }$.
\end{itemize}
\item Transposition is uncurrying.
\item Evaluation is $\varepsilon_{A, B} x = \langle \dom x \rangle \im x$.
\end{itemize}

We need to prove that the above defined are really an exponential and an evaluation.

Possible ways to prove that $\mathbf{Fcd}$ is cartesian closed follow:

NEW IDEA: Instead take $\GR\operatorname{HOM}(A,B) =
\uparrow^{\mathsf{FCD}}\setcond{(\dom p,\im p)}{
p\in\End(B)^{\End(A)}\land\supfun{p}A\sqsubseteq B}$ (what's about other
kinds of projections?)

\subsection{Proof for Fcd using sets}

Currying for sets is $\langle f \rangle ( X \times Y) = \bigcup \langle \langle \sim f \rangle X
\rangle Y$ (as it's easy to prove). This simple formula gives hope, but...

It does not work with sets because an analogy for sets of the last equality of the above mentioned long formula would be:

$\forall X, Y, V, W \in \mathscr{P} \operatorname{Ob} A : \left( X \times V \mathrel{[
A \times B]^{\ast}} Y \times W \Rightarrow \langle f \rangle ( X \times V)
\mathrel{[ C]^{\ast}} \langle f \rangle ( Y \times W) \right) \Rightarrow \\ f : A
\times B \rightarrow C$

but this implication seems false.

The most obvious way for proof attempt that $\mathbf{Fcd}$ is cartesian closed is an analogy with the proof that Dig is cartesian closed.

Use the exponential object, transposition, and evaluation as defined in [[this page|Is category Fcd cartesian closed?]]

\subsection{Reducing to the fact that Dig is cartesian closed}
It is probably a simpler way to prove that $\mathbf{Fcd}$ is cartesian closed by embedding it into $\mathbf{Dig}$ (which is [[already known to be cartesian closed|Category Dig is cartesian closed]]).

$\mathbf{Fcd}$ can be embedded into $\mathbf{Dig}$ by the formulas:
\begin{itemize}
\item $A \mapsto \langle A \rangle$;
\item $f \mapsto \langle f \rangle$.
\end{itemize}

That this really maps a morphism of $\mathbf{Fcd}$ into a morphism of $\mathbf{Dig}$ follows from the fact that $\langle g\circ f\rangle = \langle g\rangle\circ\langle f\rangle$.

Obviously this embedding (denote it $T$) is an injective (both on objects and morphisms) functor.

We will define:
\begin{itemize}
\item $\varepsilon^{\mathbf{Fcd}}_{A, B} = T^{-1} \varepsilon^{\mathbf{Dig}}_{T A, T B}$;
\item $\sim^{\mathbf{Fcd}} f = T^{-1} \sim^{\mathbf{Dig}} T f$.
\end{itemize}

Due to functoriality and injectivity of $T$ it is enough to prove that above defined $\varepsilon^{\mathbf{Fcd}}_{A, B}$ and $\sim^{\mathbf{Fcd}} f$ exist and are morphisms of $\mathbf{Fcd}$.

$\varepsilon^{\mathbf{Dig}}_{T A, T B} \ne T\varepsilon^{\mathbf{Fcd}}_{A, B}$ because $\varepsilon^{\mathbf{Dig}}_{T A, T B}$ accepts ordered pairs as the argument and $T \varepsilon^{\mathbf{Fcd}}_{A, B}$ accepts sets as the argument. So this is a dead end. Can the proof idea be salvaged?

\section{Is category Rld cartesian closed?}

We may attempt to prove that $\mathbf{Rld}$ is cartesian closed by embedding it into supposedly cartesian closed category $\mathbf{Fcd}$ by the function $\rho$:

$\langle \rho f \rangle x = f \circ x \quad \text{and} \quad \langle \rho f^{- 1} \rangle y = f^{- 1} \circ y$.

TODO: More to write on this topic.
