
\chapter{More on order theory}


\section{Straight maps and separation subsets}


\subsection{Straight maps}
\begin{defn}
\index{order reflecting}An \emph{order reflecting} map from a poset~$\mathfrak{A}$ to a poset~$\mathfrak{B}$
is such a function~$f$ that (for every $x,y\in\mathfrak{A}$)
\[ fx\sqsubseteq fy \Rightarrow x\sqsubseteq y. \]
\end{defn}

\begin{obvious}
Order embeddings are exactly the same as monotone and order reflecting maps.
\end{obvious}

\begin{defn}
\index{straight map}Let $f$ be a monotone map from a meet-semilattice
$\mathfrak{A}$ to some poset $\mathfrak{B}$. I call $f$ a \emph{straight}
map when
\[
\forall a,b\in\mathfrak{A}:(fa\sqsubseteq fb\Rightarrow fa=f(a\sqcap b)).
\]
\end{defn}
\begin{prop}
The following statements are equivalent for a monotone map~$f$:
\begin{enumerate}
\item \label{str-def}$f$ is a straight map.
\item \label{str-le-le}$\forall a,b\in\mathfrak{A}:(fa\sqsubseteq fb\Rightarrow fa\sqsubseteq f(a\sqcap b))$.
\item \label{str-le-ng}$\forall a,b\in\mathfrak{A}:(fa\sqsubseteq fb\Rightarrow fa\nsqsupset f(a\sqcap b))$.
\item \label{str-g-nle}$\forall a,b\in\mathfrak{A}:(fa\sqsupset f(a\sqcap b)\Rightarrow fa\nsqsubseteq fb)$.
\end{enumerate}
\end{prop}
\begin{proof}
~
\begin{description}
\item [{\ref{str-def}$\Leftrightarrow$\ref{str-le-le}$\Leftrightarrow$\ref{str-le-ng}}] Due
$fa\sqsupseteq f(a\sqcap b)$.
\item [{\ref{str-le-ng}$\Leftrightarrow$\ref{str-g-nle}}] Obvious.
\end{description}
\end{proof}
\begin{rem}
The definition of straight map can be generalized for any poset $\mathfrak{A}$
by the formula
\[
\forall a,b\in\mathfrak{A}:(fa\sqsubseteq fb\Rightarrow\exists c\in\mathfrak{A}:(c\sqsubseteq a\land c\sqsubseteq b\land fa=fc)).
\]


This generalization is not yet researched however.\end{rem}
\begin{prop}
Let $f$ be a monotone map from a meet-semilattice $\mathfrak{A}$
to a meet-semilattice $\mathfrak{B}$. If
\[
\forall a,b\in\mathfrak{A}:f(a\sqcap b)=fa\sqcap fb
\]
then $f$ is a straight map.\end{prop}
\begin{proof}
Let $fa\sqsubseteq fb$. Then $f(a\sqcap b)=fa\sqcap fb=fa$.\end{proof}
\begin{prop}
Let $f$ be a monotone map from a meet-semilattice $\mathfrak{A}$
to some poset $\mathfrak{B}$. If $f$ is order reflecting,
then $f$ is a straight map.\end{prop}
\begin{proof}
$fa\sqsubseteq fb\Rightarrow a\sqsubseteq b\Rightarrow a=a\sqcap b\Rightarrow fa=f(a\sqcap b)$.\end{proof}

The following theorem is the main reason of why we are interested in straight maps:
\begin{thm}
If $f$ is a straight monotone map from a meet-semilattice $\mathfrak{A}$
then the following statements are equivalent:
\begin{enumerate}
\item \label{stra-inj}$f$ is an injection.
\item \label{stra-sqe-sqe}$f$ is order reflecting.
\item \label{stra-sq-sq}$\forall a,b\in\mathfrak{A}:(a\sqsubset b\Rightarrow fa\sqsubset fb)$.
\item \label{stra-sq-ne}$\forall a,b\in\mathfrak{A}:(a\sqsubset b\Rightarrow fa\neq fb)$.
\item \label{stra-sq-nsqe}$\forall a,b\in\mathfrak{A}:(a\sqsubset b\Rightarrow fa\nsqsupseteq fb)$.
\item \label{stra-sqe-nsq}$\forall a,b\in\mathfrak{A}:(fa\sqsubseteq fb\Rightarrow a\nsqsupset b)$.
\end{enumerate}
\end{thm}
\begin{proof}
~
\begin{description}
\item [{\ref{stra-inj}$\Rightarrow$\ref{stra-sq-sq}}] Let $a,b\in\mathfrak{A}$.
Let $fa=fb\Rightarrow a=b$. Let $a\sqsubset b$. $fa\neq fb$ because
$a\neq b$. $fa\sqsubseteq fb$ because $a\sqsubseteq b$. So $fa\sqsubset fb$.
\item [{\ref{stra-sqe-sqe}$\Rightarrow$\ref{stra-inj}}] Let $a,b\in\mathfrak{A}$.
Let $fa\sqsubseteq fb\Rightarrow a\sqsubseteq b$. Let $fa=fb$. Then
$a\sqsubseteq b$ and $b\sqsubseteq a$ and consequently $a=b$.
\item [{\ref{stra-sq-sq}$\Rightarrow$\ref{stra-sqe-sqe}}] Let $\forall a,b\in\mathfrak{A}:(a\sqsubset b\Rightarrow fa\sqsubset fb)$.
Let $a\nsqsubseteq b$. Then $a\sqsupset a\sqcap b$. So $fa\sqsupset f(a\sqcap b)$.
If $fa\sqsubseteq fb$ then $fa\sqsubseteq f(a\sqcap b)$ what is
a contradiction.
\item [{\ref{stra-sq-sq}$\Rightarrow$\ref{stra-sq-nsqe}$\Rightarrow$\ref{stra-sq-ne}}] Obvious.
\item [{\ref{stra-sq-ne}$\Rightarrow$\ref{stra-sq-sq}}] Because $a\sqsubset b\Rightarrow a\sqsubseteq b\Rightarrow fa\sqsubseteq fb$.
\item [{\ref{stra-sq-nsqe}$\Leftrightarrow$\ref{stra-sqe-nsq}}] Obvious.
\end{description}
\end{proof}

\subsection{\label{sep-and-full}Separation subsets and full stars}
\begin{defn}
$\corestar_{Y}a=\setcond{x\in Y}{x\nasymp a}$ for an element $a$
of a poset $\mathfrak{A}$ and $Y\in\subsets\mathfrak{A}$.
\end{defn}

\begin{defn}
\index{star!full}\emph{Full star} of $a\in\mathfrak{A}$ is $\fullstar a=\corestar_{\mathfrak{A}}a$.\end{defn}
\begin{prop}
If $\mathfrak{A}$ is a meet-semilattice, then $\fullstar$ is a straight
monotone map.\end{prop}
\begin{proof}
Monotonicity is obvious. Let $\fullstar a\nsqsubseteq\fullstar(a\sqcap b)$.
Then it exists $x\in\fullstar a$ such that $x\notin\fullstar(a\sqcap b)$.
So $x\sqcap a\notin\fullstar b$ but $x\sqcap a\in\fullstar a$ and
consequently $\fullstar a\nsqsubseteq\fullstar b$.\end{proof}
\begin{defn}
\index{separation subset}A \emph{separation subset} of a poset $\mathfrak{A}$
is such its subset $Y$ that
\[
\forall a,b\in\mathfrak{A}:(\corestar_{Y}a=\corestar_{Y}b\Rightarrow a=b).
\]

\end{defn}

\begin{defn}
\index{separable}\index{poset!separable}I call \emph{separable} such poset
that $\fullstar$ is an injection.\end{defn}
\begin{defn}
\index{separable}\index{poset!strongly separable}I call \emph{strongly separable} such poset
that $\fullstar$ is order reflecting.\end{defn}
\begin{obvious}
A poset is separable iff it has a separation subset.\end{obvious}
\begin{obvious}
A poset is strongly separable iff $\star$ is order embedding.\end{obvious}
\begin{obvious}
Strong separability implies separability.\end{obvious}
\begin{defn}
\index{disjunction property of Wallman}A poset $\mathfrak{A}$ has
\emph{disjunction property of Wallman} iff for any $a,b\in\mathfrak{A}$
either $b\sqsubseteq a$ or there exists a non-least element $c\sqsubseteq b$
such that $a\asymp c$.\end{defn}
\begin{thm}
\label{msl-sep-conds}For a meet-semilattice with least element the
following statements are equivalent:
\begin{enumerate}
\item \label{la1}$\mathfrak{A}$ is separable.
\item \label{la2}$\mathfrak{A}$ is strongly separable.
\item \label{la3}$\forall a,b\in\mathfrak{A}:(a\sqsubset b\Rightarrow\fullstar a\sqsubset\fullstar b)$.
\item \label{la4}$\forall a,b\in\mathfrak{A}:(a\sqsubset b\Rightarrow\fullstar a\neq\fullstar b)$.
\item \label{la5}$\forall a,b\in\mathfrak{A}:(a\sqsubset b\Rightarrow\fullstar a\nsqsupseteq\fullstar b)$.
\item \label{la6}$\forall a,b\in\mathfrak{A}:(\fullstar a\sqsubseteq\fullstar b\Rightarrow a\nsqsupset b)$.
\item \label{la7}$\mathfrak{A}$ conforms to Wallman's disjunction property.
\item \label{la8}$\forall a,b\in\mathfrak{A}:(a\sqsubset b\Rightarrow\exists c\in\mathfrak{A}\setminus\{\bot\}:(c\asymp a\land c\sqsubseteq b))$.
\end{enumerate}
\end{thm}
\begin{proof}
~
\begin{description}
\item [{\ref{la1}$\Leftrightarrow$\ref{la2}$\Leftrightarrow$\ref{la3}$\Leftrightarrow$\ref{la4}$\Leftrightarrow$\ref{la5}$\Leftrightarrow$\ref{la6}}] By
the above theorem.
\item [{\ref{la8}$\Rightarrow$\ref{la4}}] Let property~\ref{la8} hold.
Let $a\sqsubset b$. Then it exists element $c\sqsubseteq b$ such
that $c\neq\bot$ and $c\sqcap a=\bot$. But $c\sqcap b\neq\bot$.
So $\fullstar a\neq\fullstar b$.
\item [{\ref{la2}$\Rightarrow$\ref{la7}}] Let property~\ref{la2} hold.
Let $a\nsqsubseteq b$. Then $\fullstar a\nsqsubseteq\fullstar b$
that is it there exists $c\in\fullstar a$ such that $c\notin\fullstar b$,
in other words $c\sqcap a\neq\bot$ and $c\sqcap b=\bot$. Let $d=c\sqcap a$.
Then $d\sqsubseteq a$ and $d\neq\bot$ and $d\sqcap b=\bot$. So
disjunction property of Wallman holds.
\item [{\ref{la7}$\Rightarrow$\ref{la8}}] Obvious.
\item [{\ref{la8}$\Rightarrow$\ref{la7}}] Let $b\nsqsubseteq a$. Then
$a\sqcap b\sqsubset b$ that is $a'\sqsubset b$ where $a'=a\sqcap b$.
Consequently $\exists c\in\mathfrak{A}\setminus\{\bot\}:(c\asymp a'\land c\sqsubseteq b)$.
We have $c\sqcap a=c\sqcap b\sqcap a=c\sqcap a'=\bot$. So $c\sqsubseteq b$
and $c\sqcap a=\bot$. Thus Wallman's disjunction property holds.
\end{description}
\end{proof}
\begin{prop}\label{bool-sep}
Every boolean lattice is strongly separable.\end{prop}
\begin{proof}
Let $a,b\in\mathfrak{A}$ where $\mathfrak{A}$ is a boolean lattice
an $a\neq b$. Then $a\sqcap\bar{b}\neq\bot$ or $\bar{a}\sqcap b\neq\bot$
because otherwise $a\sqcap\bar{b}=\bot$ and $a\sqcup\bar{b}=\top$
and thus $a=b$. Without loss of generality assume $a\sqcap\bar{b}\neq\bot$.
Then $a\sqcap c\neq\bot$ and $b\sqcap c=\bot$ for $c=a\sqcap\bar{b}\neq\bot$,
that is our lattice is separable.

It is strongly separable by theorem~\ref{msl-sep-conds}.
\end{proof}

\subsection{\label{atm-sep}Atomically Separable Lattices}
\begin{prop}
``$\atoms$'' is a straight monotone map (for any meet-semilattice).\end{prop}
\begin{proof}
Monotonicity is obvious. The rest follows from the formula
\[
\atoms(a\sqcap b)=\atoms a\cap\atoms b
\]
(corollary \ref{atoms-meet}).\end{proof}
\begin{defn}
\index{separable!atomically}I will call \emph{atomically separable}
such a poset that ``$\atoms$'' is an injection.\end{defn}
\begin{prop}
$\forall a,b\in\mathfrak{A}:(a\sqsubset b\Rightarrow\atoms a\subset\atoms b)$
iff $\mathfrak{A}$ is atomically separable for a poset $\mathfrak{A}$.\end{prop}
\begin{proof}
~
\begin{description}
\item [{$\Leftarrow$}] Obvious.
\item [{$\Rightarrow$}] Let $a\neq b$ for example $a\nsqsubseteq b$.
Then $a\sqcap b\sqsubset a$; $\atoms a\supset\atoms(a\sqcap b)=\atoms a\cap\atoms b$
and thus $\atoms a\neq\atoms b$.
\end{description}
\end{proof}
\begin{prop}
\label{atms-is-asep}Any atomistic poset is atomically separable.\end{prop}
\begin{proof}
We need to prove that $\atoms a=\atoms b\Rightarrow a=b$. But it
is obvious because
\[
a=\bigsqcup\atoms a\quad\text{and}\quad b=\bigsqcup\atoms b.
\]
\end{proof}
\begin{thm}
\label{amstc-sep}A complete lattice is atomistic iff it is atomically
separable.\end{thm}
\begin{proof}
Direct implication is the above proposition. Let's prove the reverse
implication.

Let ``$\atoms$'' be injective. Consider an element $a$ of our
poset. Let $b=\bigsqcup\atoms a$. Obviously $b\sqsubseteq a$ and
thus $\atoms b\subseteq\atoms a$. But if $x\in\atoms a$ then $x\sqsubseteq b$
and thus $x\in\atoms b$. So $\atoms a=\atoms b$. By injectivity
$a=b$ that is $a=\bigsqcup\atoms a$.\end{proof}
\begin{thm}
\label{atomistic-enough}If a lattice with least element is atomic
and separable then it is atomistic.\end{thm}
\begin{proof}
Suppose the contrary that is $a\sqsupset\bigsqcup\atoms a$. Then,
because our lattice is separable, there exists $c\in\mathfrak{A}$
such that $c\sqcap a\neq\bot$ and $c\sqcap\bigsqcup\atoms a=\bot$.
There exists atom $d\sqsubseteq c$ such that $d\sqsubseteq c\sqcap a$.
$d\sqcap\bigsqcup\atoms a\sqsubseteq c\sqcap\bigsqcup\atoms a=\bot$.
But $d\in\atoms a$. Contradiction.\end{proof}
\begin{thm}
\label{sep-conds}Let $\mathfrak{A}$ be an atomic meet-semilattice
with least element. Then the following statements are equivalent:
\begin{enumerate}
\item \label{sc-sep}$\mathfrak{A}$ is separable.
\item \label{sc-stsep}$\mathfrak{A}$ is strongly separable.
\item \label{sc-at-sep}$\mathfrak{A}$ is atomically separable.
\item \label{sc-wall}$\mathfrak{A}$ conforms to Wallman's disjunction
property.
\item \label{sc-other}$\forall a,b\in\mathfrak{A}:(a\sqsubset b\Rightarrow\exists c\in\mathfrak{A}\setminus\{\bot\}:(c\asymp a\land c\sqsubseteq b))$.
\end{enumerate}
\end{thm}
\begin{proof}
~
\begin{description}
\item [{\ref{sc-sep}$\Leftrightarrow$\ref{sc-stsep}$\Leftrightarrow$\ref{sc-wall}$\Leftrightarrow$\ref{sc-other}}] Proved
above.
\item [{\ref{sc-at-sep}$\Rightarrow$\ref{sc-other}}] Let our semilattice
be atomically separable. Let $a\sqsubset b$. Then $\atoms a\subset\atoms b$
and there exists $c\in\atoms b$ such that $c\notin\atoms a$. $c\neq\bot$
and $c\sqsubseteq b$, from which (taking into account that $c$ is
an atom) $c\sqsubseteq b$ and $c\sqcap a=\bot$. So our semilattice
conforms to the formula~\ref{sc-other}.
\item [{\ref{sc-other}$\Rightarrow$\ref{sc-at-sep}}] Let formula~\ref{sc-other}
hold. Then for any elements $a\sqsubset b$ there exists $c\neq\bot$
such that $c\sqsubseteq b$ and $c\sqcap a=\bot$. Because $\mathfrak{A}$
is atomic there exists atom $d\sqsubseteq c$. $d\in\atoms b$ and
$d\notin\atoms a$. So $\atoms a\neq\atoms b$ and $\atoms a\subseteq\atoms b$.
Consequently $\atoms a\subset\atoms b$.
\end{description}
\end{proof}
\begin{thm}
\label{atom-is-sep}Any atomistic meet-semilattice with least element
is separable.\end{thm}
\begin{proof}
From the above.
\end{proof}

\section{Quasidifference and Quasicomplement}

I've got quasidifference and quasicomplement (and dual quasicomplement)
replacing $\max$ and $\min$ in the definition of pseudodifference
and pseudocomplement (and dual pseudocomplement) with $\bigsqcup$
and $\bigsqcap$. Thus quasidifference and (dual) quasicomplement
are generalizations of their pseudo- counterparts.
\begin{rem}
\emph{Pseudocomplements} and \emph{pseudodifferences} are standard
terminology. \emph{Quasi-} counterparts are my neologisms.\end{rem}
\begin{defn}
\index{quasicomplement}Let $\mathfrak{A}$ be a poset, $a\in\mathfrak{A}$.
\emph{Quasicomplement} of $a$ is
\[
a^{\ast}=\bigsqcup\setcond{c\in\mathfrak{A}}{c\asymp a}.
\]

\end{defn}

\begin{defn}
\index{quasicomplement!dual}Let $\mathfrak{A}$ be a poset, $a\in\mathfrak{A}$.
\emph{Dual quasicomplement} of $a$ is
\[
a^{+}=\bigsqcap\setcond{c\in\mathfrak{A}}{c\equiv a}.
\]

\end{defn}
I will denote quasicomplement and dual quasicomplement for a specific
poset~$\mathfrak{A}$ as $a^{\ast(\mathfrak{A)}}$ and $a^{+(\mathfrak{A)}}$.
\begin{defn}
\index{quasidifference}Let $a,b\in\mathfrak{A}$ where $\mathfrak{A}$
is a distributive lattice. \emph{Quasidifference} of $a$ and $b$
is
\[
a\psetminus b=\bigsqcap\setcond{z\in\mathfrak{A}}{a\sqsubseteq b\sqcup z}.
\]

\end{defn}

\begin{defn}
\index{quasidifference}Let $a,b\in\mathfrak{A}$ where $\mathfrak{A}$
is a distributive lattice. \emph{Second quasidifference} of $a$ and
$b$ is
\[
a\mathop\#b=\bigsqcup\setcond{z\in\mathfrak{A}}{z\sqsubseteq a\land z\asymp b}.
\]
\end{defn}
\begin{thm}
$a\psetminus b=\bigsqcap\setcond{z\in\mathfrak{A}}{z\sqsubseteq a\land a\sqsubseteq b\sqcup z}$
where $\mathfrak{A}$ is a distributive lattice and $a,b\in\mathfrak{A}$.\end{thm}
\begin{proof}
Obviously $\setcond{z\in\mathfrak{A}}{z\sqsubseteq a\land a\sqsubseteq b\sqcup z}\subseteq\setcond{z\in\mathfrak{A}}{a\sqsubseteq b\sqcup z}$.
Thus $\bigsqcap\setcond{z\in\mathfrak{A}}{z\sqsubseteq a\land a\sqsubseteq b\sqcup z}\sqsupseteq a\psetminus b$.

Let $z\in\mathfrak{A}$ and $z'=z\sqcap a$.

$a\sqsubseteq b\sqcup z\Rightarrow a\sqsubseteq(b\sqcup z)\sqcap a\Leftrightarrow a\sqsubseteq(b\sqcap a)\sqcup(z\sqcap a)\Leftrightarrow a\sqsubseteq(b\sqcap a)\sqcup z'\Rightarrow a\sqsubseteq b\sqcup z'$
and $a\sqsubseteq b\sqcup z\Leftarrow a\sqsubseteq b\sqcup z'$. Thus
$a\sqsubseteq b\sqcup z\Leftrightarrow a\sqsubseteq b\sqcup z'$.

If $z\in\setcond{z\in\mathfrak{A}}{a\sqsubseteq b\sqcup z}$ then
$a\sqsubseteq b\sqcup z$ and thus
\[
z'\in\setcond{z\in\mathfrak{A}}{z\sqsubseteq a\land a\sqsubseteq b\sqcup z}.
\]


But $z'\sqsubseteq z$ thus having $\bigsqcap\setcond{z\in\mathfrak{A}}{z\sqsubseteq a\land a\sqsubseteq b\sqcup z}\sqsubseteq\bigsqcap\setcond{z\in\mathfrak{A}}{a\sqsubseteq b\sqcup z}$.\end{proof}
\begin{rem}
If we drop the requirement that $\mathfrak{A}$ is distributive, two
formulas for quasidifference (the definition and the last theorem)
fork.\end{rem}
\begin{obvious}
Dual quasicomplement is the dual of quasicomplement.
\end{obvious}

\begin{obvious}
~
\begin{itemize}
\item Every pseudocomplement is quasicomplement.
\item Every dual pseudocomplement is dual quasicomplement.
\item Every pseudodifference is quasidifference.
\end{itemize}
\end{obvious}
Below we will stick to the more general quasies than pseudos. If needed,
one can check that a quasicomplement $a^{\ast}$ is a pseudocomplement
by the equation $a^{\ast}\asymp a$ (and analogously with other quasies).

Next we will express quasidifference through quasicomplement.
\begin{prop}
~
\begin{enumerate}
\item \label{minus-meet}$a\psetminus b=a\psetminus(a\sqcap b)$ for any
distributive lattice;
\item \label{minus-meet2}$a\mathop\#b=a\mathop\#(a\sqcap b)$ for any distributive
lattice with least element.
\end{enumerate}
\end{prop}
\begin{proof}
~
\begin{widedisorder}
\item [{\ref{minus-meet}}] $a\sqsubseteq(a\sqcap b)\sqcup z\Leftrightarrow a\sqsubseteq(a\sqcup z)\sqcap(b\sqcup z)\Leftrightarrow a\sqsubseteq a\sqcup z\land a\sqsubseteq b\sqcup z\Leftrightarrow a\sqsubseteq b\sqcup z$.
Thus $a\psetminus(a\sqcap b)=\bigsqcap\setcond{z\in\mathfrak{A}}{a\sqsubseteq(a\sqcap b)\sqcup z}=\bigsqcap\setcond{z\in\mathfrak{A}}{a\sqsubseteq b\sqcup z}=a\psetminus b$.
\item [{\ref{minus-meet2}}] ~
\begin{align*}
a\mathop\#(a\sqcap b) & =\\
\bigsqcup\setcond{z\in\mathfrak{A}}{z\sqsubseteq a\land z\sqcap a\sqcap b=\bot} & =\\
\bigsqcup\setcond{z\in\mathfrak{A}}{z\sqsubseteq a\land(z\sqcap a)\sqcap a\sqcap b=\bot} & =\\
\bigsqcup\setcond{z\sqcap a}{z\in\mathfrak{A},z\sqcap a\sqcap b=\bot} & =\\
\bigsqcup\setcond{z\in\mathfrak{A}}{z\sqsubseteq a,z\sqcap b=\bot} & =\\
a\mathop\#b.
\end{align*}

\end{widedisorder}
\end{proof}
I will denote $Da$ the lattice $\setcond{x\in\mathfrak{A}}{x\sqsubseteq a}$.
\begin{thm}
For $a,b\in\mathfrak{A}$ where $\mathfrak{A}$ is a distributive
lattice
\begin{enumerate}
\item \label{pdiff-comp}$a\psetminus b=(a\sqcap b)^{+(Da)}$;
\item \label{sec-pdiff-comp}$a\mathop\#b=(a\sqcap b)^{\ast(Da)}$ if $\mathfrak{A}$
has least element.
\end{enumerate}
\end{thm}
\begin{proof}
~
\begin{disorder}
\item [{\ref{pdiff-comp}}] ~
\begin{align*}
(a\sqcap b)^{+(Da)} & =\\
\bigsqcap\setcond{c\in Da}{c\sqcup(a\sqcap b)=a} & =\\
\bigsqcap\setcond{c\in Da}{c\sqcup(a\sqcap b)\sqsupseteq a} & =\\
\bigsqcap\setcond{c\in Da}{(c\sqcup a)\sqcap(c\sqcup b)\sqsupseteq a} & =\\
\bigsqcap\setcond{c\in\mathfrak{A}}{c\sqsubseteq a\land c\sqcup b\sqsupseteq a} & =\\
a\psetminus b.
\end{align*}

\item [{\ref{sec-pdiff-comp}}] ~
\begin{align*}
(a\sqcap b)^{\ast(Da)} & =\\
\bigsqcup\setcond{c\in Da}{c\sqcap a\sqcap b=\bot} & =\\
\bigsqcup\setcond{c\in\mathfrak{A}}{c\sqsubseteq a\land c\sqcap a\sqcap b=\bot} & =\\
\bigsqcup\setcond{c\in\mathfrak{A}}{c\sqsubseteq a\land c\sqcap b=\bot} & =\\
a\mathop\#b.
\end{align*}

\end{disorder}
\end{proof}
\begin{prop}
$(a\sqcup b)\psetminus b\sqsubseteq a$ for an arbitrary complete
lattice.\end{prop}
\begin{proof}
$(a\sqcup b)\psetminus b=\bigsqcap\setcond{z\in\mathfrak{A}}{a\sqcup b\sqsubseteq b\sqcup z}$.

But $a\sqsubseteq z\Rightarrow a\sqcup b\sqsubseteq b\sqcup z$. So
$\setcond{z\in\mathfrak{A}}{a\sqcup b\sqsubseteq b\sqcup z}\supseteq\setcond{z\in\mathfrak{A}}{a\sqsubseteq z}$.

Consequently, $(a\sqcup b)\psetminus b\sqsubseteq\bigsqcap\setcond{z\in\mathfrak{A}}{a\sqsubseteq z}=a$.
\end{proof}

\section{Several equal ways to express pseudodifference}
\begin{thm}
\label{pdiff-eq1}For an atomistic co-brouwerian lattice $\mathfrak{A}$
and $a,b\in\mathfrak{A}$ the following expressions are always equal:
\begin{enumerate}
\item \label{pdiff-pdiff}$a\psetminus b=\bigsqcap\setcond{z\in\mathfrak{A}}{a\sqsubseteq b\sqcup z}$
(quasidifference of $a$ and $b$);
\item \label{pdiff-sec}$a\mathop\#b=\bigsqcup\setcond{z\in\mathfrak{A}}{z\sqsubseteq a\land z\sqcap b=\bot}$
(second quasidifference of $a$ and $b$);
\item \label{pdiff-atm}$\bigsqcup(\atoms a\setminus\atoms b)$.
\end{enumerate}
\end{thm}
\begin{proof}
~
\begin{description}
\item [{Proof~of~\ref{pdiff-pdiff}=\ref{pdiff-atm}}] 
\begin{align*}
a\psetminus b & =\\
\left(\bigsqcup\atoms a\right)\psetminus b & =\text{ (theorem \ref{cup-pdiff})}\\
\bigsqcup_{A\in\atoms a}(A\psetminus b) & =\\
\bigsqcup_{A\in\atoms a}\left(\begin{cases}
A & \text{if }A\notin\atoms b\\
\bot & \text{if }A\in\atoms b
\end{cases}\right) & =\\
\bigsqcup\setcond A{A\in\atoms a,A\notin\atoms b} & =\\
\bigsqcup(\atoms a\setminus\atoms b).
\end{align*}

\item [{Proof~of~\ref{pdiff-sec}=\ref{pdiff-atm}}] $a\psetminus b$
is defined because our lattice is co-brouwerian. Taking the above
into account, we have
\begin{align*}
a\psetminus b & =\\
\bigsqcup(\atoms a\setminus\atoms b) & =\\
\bigsqcup\setcond{z\in\atoms a}{z\sqcap b=\bot}.
\end{align*}



So $\bigsqcup\setcond{z\in\atoms a}{z\sqcap b=\bot}$ is defined.


If $z\sqsubseteq a\land z\sqcap b=\bot$ then $z'=\bigsqcup\setcond{x\in\atoms z}{x\sqcap b=\bot}$
is defined because $z'=z\psetminus b$ (atomisticity taken into account).
$z'$ is a lower bound for $\setcond{z\in\atoms a}{x\sqcap b=\bot}$.


Thus $z'\in\setcond{z\in\mathfrak{A}}{z\sqsubseteq a\land z\sqcap b=\bot}$
and so $\bigsqcup\setcond{z\in\atoms a}{z\sqcap b=\bot}$ is an upper
bound of $\setcond{z\in\mathfrak{A}}{z\sqsubseteq a\land z\sqcap b=\bot}$.


If $y$ is above every $z'\in\setcond{z\in\mathfrak{A}}{z\sqsubseteq a\land z\sqcap b=\bot}$
then $y$ is above every $z\in\atoms a$ such that $z\sqcap b=\bot$
and thus $y$ is above $\bigsqcup\setcond{z\in\atoms a}{z\sqcap b=\bot}$.


Thus $\bigsqcup\setcond{z\in\atoms a}{z\sqcap b=\bot}$ is least upper
bound of
\[
\setcond{z\in\mathfrak{A}}{z\sqsubseteq a\land z\sqcap b=\bot},
\]
that is
\begin{align*}
\bigsqcup\setcond{z\in\mathfrak{A}}{z\sqsubseteq a\land z\sqcap b=\bot} & =\\
\bigsqcup\setcond{z\in\atoms a}{z\sqcap b=\bot} & =\\
\bigsqcup(\atoms a\setminus\atoms b).
\end{align*}


\end{description}
\end{proof}

\section{Partially ordered categories}


\subsection{Definition}
\begin{defn}
\index{category!partially ordered}\index{precategory!partially ordered}I
will call a partially ordered (pre)category a (pre)category together
with partial order $\sqsubseteq$ on each of its Mor-sets with the
additional requirement that
\[
f_{1}\sqsubseteq f_{2}\land g_{1}\sqsubseteq g_{2}\Rightarrow g_{1}\circ f_{1}\sqsubseteq g_{2}\circ f_{2}
\]
for every morphisms $f_{1}$, $g_{1}$, $f_{2}$, $g_{2}$ such that
$\Src f_{1}=\Src f_{2}$ and $\Dst f_{1}=\Dst f_{2}=\Src g_{1}=\Src g_{2}$
and $\Dst g_{1}=\Dst g_{2}$.
\end{defn}
I will denote lattice operations on a $\Hom$-set $C(A;B)$ of a category
(or any directed multigraph) like $\sqcup^{C}$ instead of writing
$\sqcup^{C(A;B)}$ explicitly.


\subsection{Dagger categories}
\begin{defn}
\index{precategory!dagger}I will call a \emph{dagger precategory}
a precategory together with an involutive contravariant identity-on-objects
prefunctor $x\mapsto x^{\dagger}$.

In other words, a dagger precategory is a precategory equipped with
a function $x\mapsto x^{\dagger}$ on its set of morphisms which reverses
the source and the destination and is subject to the following identities
for every morphisms $f$ and~$g$:
\begin{enumerate}
\item $f^{\dagger\dagger}=f$;
\item $(g\circ f)^{\dagger}=f^{\dagger}\circ g^{\dagger}$.
\end{enumerate}
\end{defn}

\begin{defn}
\index{category!dagger}I will call a \emph{dagger category} a category
together with an involutive contravariant identity-on-objects functor
$x\mapsto x^{\dagger}$.

In other words, a dagger category is a category equipped with a function
$x\mapsto x^{\dagger}$ on its set of morphisms which reverses the
source and the destination and is subject to the following identities
for every morphisms $f$ and $g$ and object~$A$:
\begin{enumerate}
\item $f^{\dagger\dagger}=f$;
\item $(g\circ f)^{\dagger}=f^{\dagger}\circ g^{\dagger}$;
\item $(1_{A})^{\dagger}=1_{A}$.
\end{enumerate}
\end{defn}
\begin{thm}
If a category is a dagger precategory then it is a dagger category.\end{thm}
\begin{proof}
We need to prove only that $(1_{A})^{\dagger}=1_{A}$. Really,
\[
(1_{A})^{\dagger}=(1_{A})^{\dagger}\circ1_{A}=(1_{A})^{\dagger}\circ(1_{A})^{\dagger\dagger}=((1_{A})^{\dagger}\circ1_{A})^{\dagger}=(1_{A})^{\dagger\dagger}=1_{A}.
\]

\end{proof}
For a partially ordered dagger (pre)category I will additionally require
(for every morphisms $f$ and $g$ with the same source and destination)
\[
f^{\dagger}\sqsubseteq g^{\dagger}\Leftrightarrow f\sqsubseteq g.
\]


An example of dagger category is the category $\mathbf{Rel}$ whose
objects are sets and whose morphisms are binary relations between
these sets with usual composition of binary relations and with $f^{\dagger}=f^{-1}$.
\begin{defn}
\index{morphism!unitary}A morphism $f$ of a dagger category is called
\emph{unitary} when it is an isomorphism and $f^{\dagger}=f^{-1}$.
\end{defn}

\begin{defn}
\index{morphism!symmetric}\emph{Symmetric} (endo)morphism of a dagger
precategory is such a morphism $f$ that $f=f^{\dagger}$.
\end{defn}

\begin{defn}
\index{morphism!transitive}\emph{Transitive} (endo)morphism of a
precategory is such a morphism $f$ that $f=f\circ f$.\end{defn}
\begin{thm}
\label{sym-trans}The following conditions are equivalent for a morphism
$f$ of a dagger precategory:
\begin{enumerate}
\item \label{sym-trans-both}$f$ is symmetric and transitive.
\item \label{f-df-f}$f=f^{\dagger}\circ f$.
\end{enumerate}
\end{thm}
\begin{proof}
~
\begin{description}
\item [{\ref{sym-trans-both}$\Rightarrow$\ref{f-df-f}}] If $f$ is symmetric
and transitive then $f^{\dagger}\circ f=f\circ f=f$.
\item [{\ref{f-df-f}$\Rightarrow$\ref{sym-trans-both}}] $f^{\dagger}=(f^{\dagger}\circ f)^{\dagger}=f^{\dagger}\circ f^{\dagger\dagger}=f^{\dagger}\circ f=f$,
so $f$ is symmetric. $f=f^{\dagger}\circ f=f\circ f$, so $f$ is
transitive.
\end{description}
\end{proof}

\subsubsection{Some special classes of morphisms}
\begin{defn}
\index{morphism!monovalued}For a partially ordered dagger category
I will call \emph{monovalued} morphism such a morphism $f$ that $f\circ f^{\dagger}\sqsubseteq1_{\Dst f}$.
\end{defn}

\begin{defn}
\index{morphism!entirely defined}For a partially ordered dagger category
I will call \emph{entirely defined} morphism such a morphism $f$
that $f^{\dagger}\circ f\sqsupseteq1_{\Src f}$.
\end{defn}

\begin{defn}
\index{morphism!injective}For a partially ordered dagger category
I will call \emph{injective} morphism such a morphism $f$ that $f^{\dagger}\circ f\sqsubseteq1_{\Src f}$.
\end{defn}

\begin{defn}
\index{morphism!surjective}For a partially ordered dagger category
I will call \emph{surjective} morphism such a morphism f that $f\circ f^{\dagger}\sqsupseteq1_{\Dst f}$.\end{defn}
\begin{rem}
It is easy to show that this is a generalization of monovalued, entirely
defined, injective, and surjective functions as morphisms of the category
$\mathbf{Rel}$.\end{rem}
\begin{obvious}
``Injective morphism'' is a dual of ``monovalued morphism'' and
``surjective morphism'' is a dual of ``entirely defined morphism''.\end{obvious}
\begin{defn}
For a given partially ordered dagger category $C$ the \emph{category
of monovalued} (\emph{entirely defined}, \emph{injective}, \emph{surjective})
morphisms of $C$ is the category with the same set of objects as
of $C$ and the set of morphisms being the set of monovalued (entirely
defined, injective, surjective) morphisms of $C$ with the composition
of morphisms the same as in $C$.
\end{defn}
We need to prove that these are really categories, that is that composition
of monovalued (entirely defined, injective, surjective) morphisms
is monovalued (entirely defined, injective, surjective) and that identity
morphisms are monovalued, entirely defined, injective, and surjective.
\begin{proof}
We will prove only for monovalued morphisms and entirely defined morphisms,
as injective and surjective morphisms are their duals.
\begin{description}
\item [{Monovalued}] Let $f$ and $g$ be monovalued morphisms, $\Dst f=\Src g$.
Then
\begin{align*}
(g\circ f)\circ(g\circ f)^{\dagger} & =\\
g\circ f\circ f^{\dagger}\circ g^{\dagger} & \sqsubseteq\\
g\circ1_{\Src g}\circ g^{\dagger} & =\\
g\circ g^{\dagger} & \sqsubseteq\\
1_{\Dst g}&=1_{\Dst(g\circ f)}.
\end{align*}



So $g\circ f$ is monovalued.


That identity morphisms are monovalued follows from the following:
\[
1_{A}\circ(1_{A})^{\dagger}=1_{A}\circ1_{A}=1_{A}=1_{\Dst1_{A}}\sqsubseteq1_{\Dst1_{A}}.
\]


\item [{Entirely~defined}] Let $f$ and $g$ be entirely defined morphisms,
$\Dst f=\Src g$. Then
\begin{align*}
(g\circ f)^{\dagger}\circ(g\circ f) & =\\
f^{\dagger}\circ g^{\dagger}\circ g\circ f & \sqsupseteq\\
f^{\dagger}\circ1_{\Src g}\circ f & =\\
f^{\dagger}\circ1_{\Dst f}\circ f & =\\
f^{\dagger}\circ f & \sqsupseteq\\
1_{\Src f}&=1_{\Src(g\circ f)}.
\end{align*}



So $g\circ f$ is entirely defined.


That identity morphisms are entirely defined follows from the following:
\[
(1_{A})^{\dagger}\circ1_{A}=1_{A}\circ1_{A}=1_{A}=1_{\Src1_{A}}\sqsupseteq1_{\Src1_{A}}.
\]


\end{description}
\end{proof}
\begin{defn}
\index{morphism!bijective}I will call a \emph{bijective} morphism
a morphism which is entirely defined, monovalued, injective, and surjective.\end{defn}
\begin{prop}
If a morphism is bijective then it is an isomorphism.\end{prop}
\begin{proof}
Let $f$ be bijective. Then $f\circ f^{\dagger}\sqsubseteq1_{\Dst f}$,
$f^{\dagger}\circ f\sqsupseteq1_{\Src f}$, $f^{\dagger}\circ f\sqsubseteq1_{\Src f}$,
$f\circ f^{\dagger}\sqsupseteq1_{\Dst f}$. Thus $f\circ f^{\dagger}=1_{\Dst f}$
and $f^{\dagger}\circ f=1_{\Src f}$ that is $f^{\dagger}$ is an
inverse of $f$.
\end{proof}
Let $\Hom$-sets be complete lattices.
\begin{defn}
\index{morphism!metamonovalued}A morphism $f$ of a partially ordered
category is \emph{metamonovalued} when $\left(\bigsqcap G\right)\circ f=\bigsqcap_{g\in G}(g\circ f)$
whenever $G$ is a set of morphisms with a suitable domain and image.
\end{defn}

\begin{defn}
\index{morphism!metainjective}A morphism $f$ of a partially ordered
category is \emph{metainjective} when $f\circ\left(\bigsqcap G\right)=\bigsqcap_{g\in G}(f\circ g)$
whenever $G$ is a set of morphisms with a suitable domain and image.\end{defn}
\begin{obvious}
Metamonovaluedness and metainjectivity are dual to each other.\end{obvious}
\begin{defn}
\index{morphism!metacomplete}A morphism $f$ of a partially ordered
category is \emph{metacomplete} when $f\circ\left(\bigsqcup G\right)=\bigsqcup_{g\in G}(f\circ g)$
whenever $G$ is a set of morphisms with a suitable domain and image.
\end{defn}

\begin{defn}
\index{morphism!co-metacomplete}A morphism $f$ of a partially ordered
category is \emph{co-metacomplete} when $\left(\bigsqcup G\right)\circ f=\bigsqcup_{g\in G}(g\circ f)$
whenever $G$ is a set of morphisms with a suitable domain and image.
\end{defn}
Let now $\Hom$-sets be meet-semilattices.
\begin{defn}
\index{morphism!weakly metamonovalued}A morphism $f$ of a partially
ordered category is \emph{weakly metamonovalued} when $(g\sqcap h)\circ f=(g\circ f)\sqcap(h\circ f)$
whenever $g$ and $h$ are morphisms with a suitable domain and image.
\end{defn}

\begin{defn}
\index{morphism!weakly metainjective}A morphism $f$ of a partially
ordered category is \emph{weakly metainjective} when $f\circ(g\sqcap h)=(f\circ g)\sqcap(f\circ h)$
whenever $g$ and $h$ are morphisms with a suitable domain and image.
\end{defn}
Let now $\Hom$-sets be join-semilattices.
\begin{defn}
\index{morphism!weakly metacomplete}A morphism $f$ of a partially
ordered category is \emph{weakly metacomplete} when $f\circ(g\sqcup h)=(f\circ g)\sqcup(f\circ h)$
whenever $g$ and $h$ are morphisms with a suitable domain and image.
\end{defn}

\begin{defn}
\index{morphism!weakly co-metacomplete}A morphism $f$ of a partially
ordered category is \emph{weakly co-metacomplete} when $(g\sqcup h)\circ f=(g\circ f)\sqcup(h\circ f)$
whenever $g$ and $h$ are morphisms with a suitable domain and image.\end{defn}
\begin{obvious}
~
\begin{enumerate}
\item Metamonovalued morphisms are weakly metamonovalued.
\item Metainjective morphisms are weakly metainjective.
\item Metacomplete morphisms are weakly metacomplete.
\item Co-metacomplete morphisms are weakly co-metacomplete.
\end{enumerate}
\end{obvious}

\section{Partitioning}
\begin{defn}
\index{torning}Let $\mathfrak{A}$ be a complete lattice. \emph{Torning}
of an element $a\in\mathfrak{A}$ is a set $S\in\subsets\mathfrak{A}\setminus\{\bot\}$
such that
\[
\bigsqcup S=a\quad\text{and}\quad\forall x,y\in S:(x\neq y\Rightarrow x\asymp y).
\]

\end{defn}

\begin{defn}
\index{partition!weak}Let $\mathfrak{A}$ be a complete lattice.
\emph{Weak partition} of an element $a\in\mathfrak{A}$ is a set $S\in\subsets\mathfrak{A}\setminus\{\bot\}$
such that
\[
\bigsqcup S=a\quad\text{and}\quad\forall x\in S:x\asymp\bigsqcup(S\setminus\{x\}).
\]

\end{defn}

\begin{defn}
\index{partition!strong}Let $\mathfrak{A}$ be a complete lattice.
\emph{Strong partition} of an element $a\in\mathfrak{A}$ is a set
$S\in\subsets\mathfrak{A}\setminus\{\bot\}$ such that
\[
\bigsqcup S=a\quad\text{and}\quad\forall A,B\in\subsets S:(A\asymp B\Rightarrow\bigsqcup A\asymp\bigsqcup B).
\]
\end{defn}
\begin{obvious}
~
\begin{enumerate}
\item Every strong partition is a weak partition.
\item Every weak partition is a torning.
\end{enumerate}
\end{obvious}

\section{A proposition about binary relations}
\begin{prop}
\label{rel-cross}Let $f$, $g$, $h$ be binary relations. Then $g\circ f\nasymp h\Leftrightarrow g\nasymp h\circ f^{-1}$.\end{prop}
\begin{proof}
~
\begin{align*}
g\circ f\nasymp h & \Leftrightarrow\\
\exists a,c:a\mathrel{((g\circ f)\cap h)}c & \Leftrightarrow\\
\exists a,c:(a\mathrel{(g\circ f)}c\land a\mathrel{h}c) & \Leftrightarrow\\
\exists a,b,c:(a\mathrel{f}b\land b\mathrel{g}c\land a\mathrel{h}c) & \Leftrightarrow\\
\exists b,c:(b\mathrel{g}c\land b\mathrel{(h\circ f^{-1})}c) & \Leftrightarrow\\
\exists b,c:b\mathrel{(g\cap(h\circ f^{-1}))}c & \Leftrightarrow\\
g\nasymp h\circ f^{-1}.
\end{align*}

\end{proof}

\section{Infinite associativity and ordinated product}


\subsection{Introduction}

We will consider some function $f$ which takes an arbitrary ordinal
number of arguments. That is $f$ can be taken for arbitrary (small,
if to be precise) ordinal number of arguments. More formally: Let
$x=x_{i\in n}$ be a family indexed by an ordinal $n$. Then $f(x)$
can be taken. The same function $f$ can take different number of
arguments. (See below for the exact definition.)

Some of such functions $f$ are associative in the sense defined below.
If a function is associative in the below defined sense, then the
binary operation induced by this function is associative in the usual
meaning of the word ``associativity'' as defined in basic algebra.

I also introduce and research an important example of infinitely associative
function, which I call \emph{ordinated product}.

Note that my searching about infinite associativity and ordinals in
Internet has provided no useful results. As such there is a reason
to assume that my research of generalized associativity in terms of
ordinals is novel.


\subsection{Used notation}

\index{ordinal}We identify natural numbers with finite Von Neumann's
ordinals (further just \emph{ordinals} or \emph{ordinal numbers}).

For simplicity we will deal with small sets (members of a Grothendieck
universe). We will denote the Grothendieck universe (aka \emph{universal
set}) as $\mho$.

I will denote a tuple of $n$ elements like $\left\llbracket a_{0};\ldots;a_{n-1}\right\rrbracket $.
By definition
\[
\left\llbracket a_{0};\ldots;a_{n-1}\right\rrbracket =\{(0;a_{0}),\ldots,(n-1;a_{n-1})\}.
\]


Note that an ordered pair $(a;b)$ is not the same as the tuple $\left\llbracket a;b\right\rrbracket $
of two elements. (However, we will use them interchangeably.)
\begin{defn}
\index{relation!anchored}An \emph{anchored relation} is a tuple $\left\llbracket n;r\right\rrbracket $
where $n$ is an index set and $r$ is an $n$-ary relation.
\end{defn}
\index{graph!of anchored relation}For an anchored relation $\arity\left\llbracket n;r\right\rrbracket =n$.
The graph\footnote{It is unrelated with graph theory.} of $\left\llbracket n;r\right\rrbracket $
is defined as follows: $\GR\left\llbracket n;r\right\rrbracket =r$.
\begin{defn}
$\Pr_{i}f$ is a function defined by the formula
\[
\Pr_{i}f=\setcond{x_{i}}{x\in f}
\]
for every small $n$-ary relation $f$ where $n$ is an ordinal number
and $i\in n$. Particularly for every $n$-ary relation $f$ and $i\in n$
where $n\in\mathbb{N}$
\[
\Pr_{i}f=\setcond{x_{i}}{\left\llbracket x_{0};\ldots;x_{n-1}\right\rrbracket \in f}.
\]

\end{defn}
\index{product!cartesian}Recall that Cartesian product is defined
as follows:
\[
\prod a=\setcond{z\in\left(\bigcup\im a\right)^{\dom a}}{\forall i\in\dom a:z(i)\in a_{i}}.
\]

\begin{obvious}
If $a$ is a small function, then $\prod a=\setcond{z\in\mho^{\dom a}}{\forall i\in\dom a:z(i)\in a_{i}}.$
\end{obvious}

\subsubsection{Currying and uncurrying}


\paragraph{The customary definition}

Let $X$, $Y$, $Z$ be sets.

We will consider variables $x\in X$ and $y\in Y$.

\index{currying}Let a function $f\in Z^{X\times Y}$. Then $\curry(f)\in(Z^{Y})^{X}$
is the function defined by the formula $(\curry(f)x)y=f(x;y)$.

\index{uncurrying}Let now $f\in(Z^{Y})^{X}$. Then $\uncurry(f)\in Z^{X\times Y}$
is the function defined by the formula $\uncurry(f)(x;y)=(fx)y$.
\begin{obvious}
~
\begin{enumerate}
\item $\uncurry(\curry(f))=f$ for every $f\in Z^{X\times Y}$.
\item $\curry(\uncurry(f))=f$ for every $f\in(Z^{Y})^{X}$.
\end{enumerate}
\end{obvious}

\paragraph{Currying and uncurrying with a dependent variable}

Let $X$, $Z$ be sets and $Y$ be a function with the domain $X$.
(Vaguely saying, $Y$ is a variable dependent on $X$.)

The disjoint union $\coprod Y=\bigcup_{i\in\dom Y}(\{i\}\times Y_{i})=\setcond{(i;x)}{i\in\dom Y,x\in Y_{i}}$.

We will consider variables $x\in X$ and $y\in Y_{x}$.

\index{currying}Let a function $f\in Z^{\coprod_{i\in X}Y_{i}}$
(or equivalently $f\in Z^{\coprod Y}$). Then $\curry(f)\in\prod_{i\in X}Z^{Y_{i}}$
is the function defined by the formula $(\curry(f)x)y=f(x;y)$.

\index{uncurrying}Let now $f\in\prod_{i\in X}Z^{Y_{i}}$. Then $\uncurry(f)\in Z^{\coprod_{i\in X}Y_{i}}$
is the function defined by the formula $\uncurry(f)(x;y)=(fx)y$.
\begin{obvious}
~
\begin{enumerate}
\item $\uncurry(\curry(f))=f$ for every $f\in Z^{\coprod_{i\in X}Y_{i}}$.
\item $\curry(\uncurry(f))=f$ for every $f\in\prod_{i\in X}Z^{Y_{i}}$.
\end{enumerate}
\end{obvious}

\subsubsection{Functions with ordinal numbers of arguments}

Let $\mathrm{Ord}$ be the set of small ordinal numbers.

If $X$ and $Y$ are sets and $n$ is an ordinal number, the set of
functions taking $n$ arguments on the set $X$ and returning a value
in $Y$ is $Y^{X^{n}}$.

The set of all small functions taking ordinal numbers of arguments
is $Y^{\bigcup_{n\in\mathrm{Ord}}X^{n}}$.

\index{ordinal variadic}I will denote $\mathrm{OrdVar}(X)=\mho^{\bigcup_{n\in\mathrm{Ord}}X^{n}}$
and call it \emph{ordinal variadic}. (``Var'' in this notation is
taken from the word \emph{variadic} in the collocation \emph{variadic
function} used in computer science.)


\subsection{On sums of ordinals}

Let $a$ be an ordinal-indexed family of ordinals.
\begin{prop}
$\coprod a$ with lexicographic order is a well-ordered set.\end{prop}
\begin{proof}
Let $S$ be non-empty subset of $\coprod a$.

Take $i_{0}=\min\Pr_{0}S$ and $x_{0}=\min\setcond{\Pr_{1}y}{y\in S,y(0)=i_{0}}$
(these exist by properties of ordinals). Then $(i_{0};x_{0})$ is
the least element of $S$.\end{proof}
\begin{defn}
$\sum a$ is the unique ordinal order-isomorphic to $\coprod a$.\end{defn}
\begin{xca}
Prove that for finite ordinals it is just a sum of natural numbers.
\end{xca}
This ordinal exists and is unique because our set is well-ordered.
\begin{rem}
An infinite sum of ordinals is not customary defined.
\end{rem}
\index{sum!structured}The \emph{structured sum} $\bigoplus a$ of
$a$ is an order isomorphism from lexicographically ordered set $\coprod a$
into $\sum a$.

There exists (for a given $a$) exactly one structured sum, by properties
of well-ordered sets.
\begin{obvious}
$\sum a=\im\bigoplus a$.\end{obvious}
\begin{thm}
$\left(\bigoplus a\right)(n;x)=\sum_{i\in n}a_{i}+x$.\end{thm}
\begin{proof}
We need to prove that it is an order isomorphism. Let's prove it is
an injection that is $m>n\Rightarrow\sum_{i\in m}a_{i}+x>\sum_{i\in n}a_{i}+x$
and $y>x\Rightarrow\sum_{i\in n}a_{i}+y>\sum_{i\in n}a_{i}+x$.

Really, if $m>n$ then $\sum_{i\in m}a_{i}+x\geq\sum_{i\in n+1}a_{i}+x>\sum_{i\in n}a_{i}+x$.
The second formula is true by properties of ordinals.

Let's prove that it is a surjection. Let $r\in\sum a$. There exist
$n\in\dom a$ and $x\in a_{n}$ such that $r=\left(\bigoplus a\right)(n;x)$.
Thus $r=\left(\bigoplus a\right)(n;0)+x=\sum_{i\in n}a_{i}+x$ because
$\left(\bigoplus a\right)(n;0)=\sum_{i\in n}a_{i}$ since $(n;0)$
has $\sum_{i\in n}a_{i}$ predecessors.
\end{proof}

\subsection{\label{ordinated-prod}Ordinated product}


\subsubsection{Introduction}

\emph{Ordinated product} defined below is a variation of Cartesian
product, but is associative unlike Cartesian product. However, ordinated
product unlike Cartesian product is defined not for arbitrary sets,
but only for relations having ordinal numbers of arguments.

Let $F$ indexed by an ordinal number be a small family of anchored
relations.


\subsubsection{Concatenation}
\begin{defn}
\index{concatenation}Let $z$ be an indexed by an ordinal number
family of functions each taking an ordinal number of arguments. The
\emph{concatenation} of $z$ is
\[
\concat z=\uncurry(z)\circ\left(\bigoplus(\dom\circ z)\right)^{-1}.
\]
\end{defn}
\begin{xca}
Prove, that if $z$ is a finite family of finitary tuples, it is concatenation
of $\dom z$ tuples in the usual sense (as it is commonly used in
computer science).\end{xca}
\begin{prop}
If $z\in\prod(\GR\circ F)$ then $\concat z=\uncurry(z)\circ\left(\bigoplus(\arity\circ F)\right)^{-1}.$\end{prop}
\begin{proof}
If $z\in\prod(\GR\circ F)$ then $\dom z(i)=\dom(\GR\circ F)_{i}=\arity F_{i}$
for every $i\in\dom F$. Thus $\dom\circ z=\arity\circ F$.\end{proof}
\begin{prop}
$\dom\concat z=\sum_{i\in\dom z}\dom z_{i}$.\end{prop}
\begin{proof}
Because $\dom\left(\bigoplus(\dom\circ z)\right)^{-1}=\sum_{i\in\dom f}(\dom\circ z)$,
it is enough to prove that
\[
\dom\uncurry(z)=\dom\bigoplus(\dom\circ z).
\]


Really,
\begin{align*}
\sum_{i\in\dom f}(\dom\circ z) & =\\
\setcond{(i;x)}{i\in\dom(\dom\circ z),x\in\dom z_{i}} & =\\
\setcond{(i;x)}{i\in\dom z,x\in\dom z_{i}} & =\\
\coprod z
\end{align*}
and $\dom\uncurry(z)=\coprod_{i\in X}z_{i}=\coprod z$.
\end{proof}

\subsubsection{Finite example}

If $F$ is a finite family (indexed by a natural number $\dom F$)
of anchored finitary relations, then by definition \[
\GR\prod^\mathrm{(ord)} =
\setcond{
\llbracket a_{0, 0} ; \ldots ; a_{0, \arity F_0 - 1} ; \ldots ; a_{\dom F - 1, 0} ; \ldots ; a_{\dom F - 1, \arity F_{\dom F - 1} - 1} \rrbracket
}{
\llbracket a_{0, 0} ; \ldots ; a_{0, \arity F_0 - 1} \rrbracket \in \GR F_0 \land \ldots \land \llbracket a_{\dom F - 1, \arity F_{\dom F - 1} - 1} \rrbracket \in \GR F_{\dom F - 1}
}
\]and 
\[
\arity\prod^{\mathrm{(ord)}}F=\arity F_{0}+\ldots+\arity F_{\dom F-1}.
\]


The above formula can be shortened to
\[
\GR\prod^{\mathrm{(ord)}}F=\setcond{\concat z}{z\in\prod(\GR\circ F)}.
\]



\subsubsection{The definition}
\begin{defn}
\index{product!ordinated}The anchored relation (which I call \emph{ordinated
product}) $\prod^{\mathrm{(ord)}}F$ is defined by the formulas:
\begin{gather*}
\arity\prod^{\mathrm{(ord)}}F=\sum(\arity\circ f);\\
\GR\prod^{\mathrm{(ord)}}F=\setcond{\concat z}{z\in\prod(\GR\circ F)}.
\end{gather*}
\end{defn}
\begin{prop}
$\prod^{\mathrm{(ord)}}F$ is a properly defined anchored relation.\end{prop}
\begin{proof}
$\dom\concat z=\sum_{i\in\dom F}\dom z_{i}=\sum_{i\in\dom F}\arity f_{i}=\sum(\arity\circ F)$.
\end{proof}

\subsubsection{Definition with composition for every multiplier}

$q(F)_{i}\eqdef\left(\curry\left(\bigoplus(\arity\circ F)\right)\right)i$.
\begin{prop}
$\prod^{\mathrm{(ord)}}F=\setcond{L\in\mho^{\sum(\arity\circ F)}}{\forall i\in\dom F:L\circ q(F)_{i}\in\GR F_{i}}$.\end{prop}
\begin{proof}
$\GR\prod^{\mathrm{(ord)}}F=\setcond{\concat z}{z\in\prod(\GR\circ F)}$;

$\GR\prod^{\mathrm{(ord)}}F=\setcond{\uncurry(z)\circ\left(\bigoplus(\arity\circ f)\right)^{-1}}{z\in\prod_{i\in\dom F}\mho^{\arity F_{i}},\forall i\in\dom F:z(i)\in\GR F_{i}}$.

Let $L=\uncurry(z)$. Then $z=\curry(L)$.

$\GR\prod^{\mathrm{(ord)}}F=\setcond{L\circ\left(\bigoplus(\arity\circ f)\right)^{-1}}{\curry(L)\in\prod_{i\in\dom F}\mho^{\arity F_{i}},\forall i\in\dom F:\curry(L)i\in\GR F_{i}}$;

$\GR\prod^{\mathrm{(ord)}}F=\setcond{L\circ\left(\bigoplus(\arity\circ f)\right)^{-1}}{L\in\mho^{\coprod_{i\in\dom F}\arity F_{i}},\forall i\in\dom F:\curry(L)i\in\GR F_{i}}$;

$\GR\prod^{\mathrm{(ord)}}F=\setcond{L\in\mho^{\sum(\arity\circ f)}}{\forall i\in\dom F:\curry\left(L\circ\bigoplus(\arity\circ F)\right)i\in\GR F_{i}}$;

$\left(\curry\left(L\circ\bigoplus(\arity\circ F)\right)i\right)x=L\left(\left(\curry\left(\bigoplus(\arity\circ F)\right)i\right)x\right)=L(q(F)_{i}x)=(L\circ q(F)_{i})x$;

$\curry\left(L\circ\bigoplus(\arity\circ F)\right)i=L\circ q(F)_{i}$;

$\prod^{\mathrm{(ord)}}F=\setcond{L\in\mho^{\sum(\arity\circ F)}}{\forall i\in\dom F:L\circ q(F)_{i}\in\GR F_{i}}$.\end{proof}
\begin{cor}
$\prod^{\mathrm{(ord)}}F=\setcond{L\in\left(\bigcup\im(\GR\circ F)\right)^{\sum(\arity\circ F)}}{\forall i\in\dom F:L\circ q(F)_{i}\in\GR F_{i}}$.
\end{cor}

\begin{cor}
$\prod^{\mathrm{(ord)}}F$ is small if $F$ is small.
\end{cor}

\subsubsection{Definition with shifting arguments}

Let $F'_{i}=\setcond{L\circ\Pr_{1}|_{\{i\}\times\arity F_{i}}}{L\in\GR F_{i}}$.
\begin{prop}
$F'_{i}=\setcond{L\circ\Pr_{1}|_{\{i\}\times\mho}}{L\in\GR F_{i}}$.\end{prop}
\begin{proof}
If $L\in\GR F_{i}$ then $\dom L=\arity F_{i}$. Thus
\[
L\circ\Pr_{1}|_{\{i\}\times\arity F_{i}}=L\circ\Pr_{1}|_{\{i\}\times\dom L}=L\circ\Pr_{1}|_{\{i\}\times\mho}.
\]
\end{proof}
\begin{prop}
$F'_{i}$ is an $(\{i\}\times\arity F_{i})$-ary relation.\end{prop}
\begin{proof}
We need to prove that $\dom\left(L\circ\Pr_{1}|_{\{i\}\times\arity F_{i}}\right)=\{i\}\times\arity F_{i}$
for $L\in\GR F_{i}$, but that's obvious.\end{proof}
\begin{obvious}
$\coprod(\arity\circ F)=\bigcup_{i\in\dom F}(\{i\}\times\arity F_{i})=\bigcup_{i\in\dom F}\dom F'_{i}$.\end{obvious}
\begin{lem}
$P\in\prod_{i\in\dom F}F'_{i}\Leftrightarrow\curry\left(\bigcup\im P\right)\in\prod(\GR\circ F)$
for a ($\dom F$)-indexed family $P$ where $P_{i}\in\mho^{\{i\}\times\arity F_{i}}$
for every $i\in\dom F$, that is for $P\in\coprod_{i\in\dom F}\mho^{\{i\}\times\arity F_{i}}$.\end{lem}
\begin{proof}
For every $P\in\coprod_{i\in\dom F}\mho^{\{i\}\times\arity F_{i}}$
we have:
\begin{align*}
P\in\prod_{i\in\dom F}F'_{i} & \Leftrightarrow\\
P\in\setcond{z\in\mho^{\dom F}}{\forall i\in\dom F:z(i)\in F'_{i}} & \Leftrightarrow\\
P\in\mho^{\dom F}\land\forall i\in\dom F:P(i)\in F'_{i} & \Leftrightarrow\\
P\in\mho^{\dom F}\land\forall i\in\dom F\exists L\in\GR F_{i}:P_{i}=L\circ(\Pr_{1}|_{\{i\}\times\mho}) & \Leftrightarrow\\
P\in\mho^{\dom F}\land\forall i\in\dom F\exists L\in\GR F_{i}:(P_{i}\in\mho^{\{i\}\times\arity F_{i}}\land\forall x\in\arity F_{i}:P_{i}(i;x)=Lx) & \Leftrightarrow\\
P\in\mho^{\dom F}\land\forall i\in\dom F\exists L\in\GR F_{i}:(P_{i}\in\mho^{\{i\}\times\arity F_{i}}\land\curry(P_{i})i=L) & \Leftrightarrow\\
P\in\mho^{\dom F}\land\forall i\in\dom F:(P_{i}\in\mho^{\{i\}\times\arity F_{i}}\land\curry(P_{i})i\in\GR F_{i}) & \Leftrightarrow\\
\forall i\in\dom F\exists Q_{i}\in(\mho^{\arity F_{i}})^{\{i\}}:(P_{i}=\uncurry(Q_{i})\land(Q_{i})i\in\mho^{\arity F_{i}}\land Q_{i}i\in\GR F_{i}) & \Leftrightarrow\\
\forall i\in\dom F\exists Q_{i}\in(\mho^{\arity F_{i}})^{\{i\}}:\left(P_{i}=\uncurry(Q_{i})\land\left(\bigcup_{i\in\dom F}Q_{i}\right)i\in\GR F_{i}\right) & \Leftrightarrow\\
\forall i\in\dom F\exists Q_{i}\in(\mho^{\arity F_{i}})^{\{i\}}:\left(P_{i}=\uncurry(Q_{i})\land\bigcup_{i\in\dom F}Q_{i}\in\prod(\GR\circ F)\right) & \Leftrightarrow\\
\forall i\in\dom F:\bigcup_{i\in\dom F}\curry(P_{i})\in\prod(\GR\circ F) & \Leftrightarrow\\
\curry\left(\bigcup_{i\in\dom F}P_{i}\right)\in\prod(\GR\circ F) & \Leftrightarrow\\
\curry\left(\bigcup\im P\right)\in\prod(\GR\circ F).
\end{align*}
\end{proof}
\begin{lem}
$\setcond{\curry(f)\circ\bigoplus(\arity\circ F)}{f\in\GR\prod^{\mathrm{(ord)}}F}=\prod(\GR\circ F)$.\end{lem}
\begin{proof}
First $\GR\prod^{\mathrm{(ord)}}F=\setcond{\uncurry(z)\circ\left(\bigoplus(\dom\circ z)\right)^{-1}}{z\in\prod(\GR\circ F)}$,
that is

$\setcond f{f\in\GR\prod^{\mathrm{(ord)}}F}=\setcond{\uncurry(z)\circ\left(\bigoplus(\arity\circ F)\right)^{-1}}{z\in\prod(\GR\circ F)}$.

Since $\bigoplus(\arity\circ F)$ is a bijection, we have

$\setcond{f\circ\bigoplus(\arity\circ F)}{f\in\GR\prod^{\mathrm{(ord)}}F}=\setcond{\uncurry(z)}{z\in\prod(\GR\circ F)}$
what is equivalent to

$\setcond{\curry(f)\circ\bigoplus(\arity\circ F)}{f\in\GR\prod^{\mathrm{(ord)}}F}=\setcond z{z\in\prod(\GR\circ F)}$
that is $\setcond{\curry(f)\circ\bigoplus(\arity\circ F)}{f\in\GR\prod^{\mathrm{(ord)}}F}=\prod(\GR\circ F)$.\end{proof}
\begin{lem}
$\setcond{\bigcup\im P}{P\in\coprod_{i\in\dom F}\mho^{\{i\}\times\arity F_{i}}\land\curry(\bigcup\im P)\in\prod(\GR\circ F)}=\setcond{L\in\mho^{\coprod_{i\in\dom F}\arity F_{i}}}{\curry(L)\in\prod(\GR\circ F)}$.\end{lem}
\begin{proof}
Let $L'\in\setcond{L\in\mho^{\coprod_{i\in\dom F}\arity F_{i}}}{\curry(L)\in\prod(\GR\circ F)}$.
Then $L'\in\mho^{\coprod_{i\in\dom F}\arity F_{i}}$ and $\curry(L')\in\prod(\GR\circ F)$.

Let $P=\mylambda i{\dom F}{L'|_{\{i\}\times\arity F_{i}}}$. Then
$P\in\coprod_{i\in\dom F}\mho^{\{i\}\times\arity F_{i}}$ and $\bigcup\im P=L'$.
So $L'\in\setcond{\bigcup\im P}{P\in\coprod_{i\in\dom F}\mho^{\{i\}\times\arity F_{i}}\land\curry(\bigcup\im P)\in\prod(\GR\circ F)}$.

Let now $L'\in\setcond{\bigcup\im P}{P\in\coprod_{i\in\dom F}\mho^{\{i\}\times\arity F_{i}}\land\curry(\bigcup\im P)\in\prod(\GR\circ F)}$.
Then there exists $P\in\coprod_{i\in\dom F}\mho^{\{i\}\times\arity F_{i}}$
such that $L'=\bigcup\im P$ and $\curry(L')\in\prod(\GR\circ F)$.
Evidently $L'\in\mho^{\coprod_{i\in\dom F}\arity F_{i}}$. So
$L'\in\setcond{L\in\mho^{\coprod_{i\in\dom F}\arity F_{i}}}{\curry(L)\in\prod(\GR\circ F)}$.
\end{proof}
\begin{lem}
$\setcond{f\circ\bigoplus(\arity\circ F)}{f\in\GR\prod^{\mathrm{(ord)}}F}=\setcond{\bigcup\im P}{P\in\prod_{i\in\dom F}F'_{i}}$.\end{lem}
\begin{proof}
~
\begin{align*}
L\in\setcond{\bigcup\im P}{P\in\prod_{i\in\dom F}F'_{i}} & \Leftrightarrow\\
L\in\setcond{\bigcup\im P}{P\in\coprod_{i\in\dom F}\mho^{\{i\}\times\arity F_{i}}\land\curry\left(\bigcup\im P\right)\in\prod(\GR\circ F)} & \Leftrightarrow\\
L\in\mho^{\coprod_{i\in\dom F}\arity F_{i}}\land\curry(L)\in\prod(\GR\circ F) & \Leftrightarrow\\
L\in\mho^{\coprod_{i\in\dom F}\arity F_{i}}\land\curry(L)\in\setcond{\curry(f)\circ\bigoplus(\arity\circ F)}{f\in\GR\prod^{\mathrm{(ord)}}F} & \Leftrightarrow\\
\text{(because \ensuremath{\bigoplus(\arity\circ F)} is a bijection)}\\
\curry(L)\circ\left(\bigoplus(\arity\circ F)\right)^{-1}\in\setcond{\curry(f)}{f\in\GR\prod^{\mathrm{(ord)}}F} & \Leftrightarrow\\
L\circ\left(\bigoplus(\arity\circ F)\right)^{-1}\in\setcond f{f\in\GR\prod^{\mathrm{(ord)}}F} & \Leftrightarrow\\
\text{(because \ensuremath{\bigoplus(\arity\circ F)} is a bijection)}\\
L\in\setcond{f\circ\bigoplus(\arity\circ F)}{f\in\GR\prod^{\mathrm{(ord)}}F}.
\end{align*}
\end{proof}
\begin{thm}
$\GR\prod^{\mathrm{(ord)}}F=\setcond{\left(\bigcup\im P\right)\circ\left(\bigoplus(\arity\circ F)\right)^{-1}}{P\in\prod_{i\in\dom F}F'_{i}}$.\end{thm}
\begin{proof}
From the lemma, because $\bigoplus(\arity\circ F)$ is a bijection.\end{proof}
\begin{thm}
$\GR\prod^{\mathrm{(ord)}}F=\setcond{\bigcup_{i\in\dom F}\left(P_{i}\circ\left(\bigoplus(\arity\circ f)\right)^{-1}\right)}{P\in\prod_{i\in\dom F}F'_{i}}$.\end{thm}
\begin{proof}
From the previous theorem.\end{proof}
\begin{thm}
$\GR\prod^{\mathrm{(ord)}}F=\setcond{\bigcup\im P}{P\in\prod_{i\in\dom F}\setcond{f\circ\left(\bigoplus(\arity\circ f)\right)^{-1}}{f\in F'_{i}}}$.\end{thm}
\begin{proof}
From the previous.\end{proof}
\begin{rem}
Note that the above formulas contain both $\bigcup_{i\in\dom F}\dom F'_{i}$
and $\bigcup_{i\in\dom F}F'_{i}$. These forms are similar but different.
\end{rem}

\subsubsection{Associativity of ordinated product}

Let $f$ be an ordinal variadic function.

Let $S$ be an ordinal indexed family of functions of ordinal indexed
families of functions each taking an ordinal number of arguments in
a set $X$.

\index{associative!infinite}I call $f$ \emph{infinite associative}
when
\begin{enumerate}
\item $f(f\circ S)=f(\concat S)$ for every $S$;
\item $f(\llbracket x\rrbracket)=x$ for $x\in X$.
\end{enumerate}

\paragraph{Infinite associativity implies associativity}
\begin{prop}
Let $f$ be an infinitely associative function taking an ordinal number
of arguments in a set $X$. Define $x\star y=f\llbracket x;y\rrbracket$
for $x,y\in X$. Then the binary operation $\star$ is associative.\end{prop}
\begin{proof}
Let $x,y,z\in X$. Then $(x\star y)\star z=f\llbracket f\llbracket x;y\rrbracket;z\rrbracket=f(f\llbracket x;y\rrbracket;f\llbracket z\rrbracket)=f\llbracket x;y;z\rrbracket$.
Similarly $x\star(y\star z)=f\llbracket x;y;z\rrbracket$. So $(x\star y)\star z=x\star(y\star z)$.
\end{proof}

\paragraph{Concatenation is associative}

First we will prove some lemmas.

Let $a$ and $b$ be functions on a poset. Let $a\sim b$ iff there
exist an order isomorphism $f$ such that $a=b\circ f$. Evidently
$\sim$ is an equivalence relation.
\begin{obvious}
$\concat a=\concat b\Leftrightarrow\uncurry(a)\sim\uncurry(b)$ for
every ordinal indexed families $a$ and $b$ of functions taking an
ordinal number of arguments.
\end{obvious}
Thank to the above, we can reduce properties of $\concat$ to properties
of $\uncurry$.
\begin{lem}
$a\sim b\Rightarrow\uncurry a\sim\uncurry b$ for every ordinal indexed
families $a$ and $b$ of functions taking an ordinal number of arguments.\end{lem}
\begin{proof}
There exist an order isomorphism $f$ such that $a=b\circ f$.

$\uncurry(a)(x;y)=(ax)y=(bfx)y=\uncurry(b)(fx;y)=\uncurry(b)g(x;y)$
where $g(x;y)=(fx;y)$.

$g$ is an order isomorphism because $g(x_{0};y_{0})\ge g(x_{1};y_{1})\Leftrightarrow(x_{0};y_{0})\ge(x_{1};y_{1})$.
(Injectivity and surjectivity are obvious.)\end{proof}
\begin{lem}
\label{by-member}Let $a_{i}\sim b_{i}$ for every $i$. Then $\uncurry a\sim\uncurry b$
for every ordinal indexed families $a$ and $b$ of ordinal indexed
families of functions taking an ordinal number of arguments.\end{lem}
\begin{proof}
Let $a_{i}=b_{i}\circ f_{i}$ where $f_{i}$ is an order isomorphism
for every $i$.

$\uncurry(a)(i;y)=a_{i}y=b_{i}f_{i}y=\uncurry(b)(i;f_{i}y)=\uncurry(b)g(i;y)=(\uncurry(b)\circ g)(i;y)$
where $g(i;y)=(i;f_{i}y)$.

$g$ is an order isomorphism because $g(i;y_{0})\ge g(i;y_{1})\Leftrightarrow f_{i}y_{0}\ge f_{i}y_{1}\Leftrightarrow y_{0}\ge y_{1}$
and $i_{0}>i_{1}\Rightarrow g(i;y_{0})>g(i;y_{1})$. (Injectivity
and surjectivity are obvious.)
\end{proof}
Let now $S$ be an ordinal indexed family of ordinal indexed families
of functions taking an ordinal number of arguments.
\begin{lem}
$\uncurry(\uncurry\circ S)\sim\uncurry(\uncurry S)$.\end{lem}
\begin{proof}
$\uncurry\circ S=\mylambda iS{\uncurry(S_{i})}$;

$(\uncurry(\uncurry\circ S))((i;x);y)=(\uncurry S_{i})(x;y)=(S_{i}x)y$;

$(\uncurry(\uncurry S))((i;x);y)=((\uncurry S)(i;x))y=(S_{i}x)y$.

Thus $(\uncurry(\uncurry\circ S))((i;x);y)=(\uncurry(\uncurry S))((i;x);y)$
and thus evidently $\uncurry(\uncurry\circ S)\sim\uncurry(\uncurry S)$.\end{proof}
\begin{thm}
$\concat$ is an infinitely associative function.\end{thm}
\begin{proof}
$\concat(\llbracket x\rrbracket)=x$ for a function $x$ taking an
ordinal number of argument is obvious. It is remained to prove
\[
\concat(\concat\circ S)=\concat(\concat S);
\]


We have, using the lemmas,
\begin{align*}
\concat(\concat\circ S) & \sim\\
\uncurry(\concat\circ S) & \sim\\
\text{(by lemma \ref{by-member})}\\
\uncurry(\uncurry\circ S) & \sim\\
\uncurry(\uncurry S) & \sim\\
\uncurry(\concat S) & \sim\\
\concat(\concat S).
\end{align*}


Consequently $\concat(\concat\circ S)=\concat(\concat S)$.\end{proof}
\begin{cor}
Ordinated product is an infinitely associative function.
\end{cor}

\section{Galois surjections}

\begin{defn}\index{Galois surjection}
  \emph{Galois surjection} is the special case of Galois connection such
  that $f^{\ast} \circ f_{\ast} $ is identity.
\end{defn}

\begin{prop}\label{gal-eq}
  For Galois surjection $\mathfrak{A} \rightarrow \mathfrak{B}$ such that
  $\mathfrak{A}$ is a join-semilattice we have (for every $y \in
  \mathfrak{B}$)
  \[ f_{\ast} y = \max \setcond{ x \in \mathfrak{A} }{
     f^{\ast} x = y } . \]
\end{prop}

\begin{proof}
  We need to prove (theorem~\ref{adj-max})
  \[ \max \setcond{ x \in \mathfrak{A} }{ f^{\ast} x = y } =
     \max \setcond{ x \in \mathfrak{A} }{ f^{\ast} x \sqsubseteq y } . \]
  To prove it, it's enough to show that for each $f^{\ast} x \sqsubseteq y$
  there exists an $x' \sqsupseteq x$ such that $f^{\ast} x' = y$.
  
  Really, $y = f^{\ast} f_{\ast} y$. It's enough to prove $f^{\ast} (x \sqcup
  f_{\ast} y) = y$.
  
  Indeed (because lower adjoints preserve joins),
  $f^{\ast} (x \sqcup f_{\ast} y) = f^{\ast} x \sqcup f^{\ast} f_{\ast} y = f^{\ast} x \sqcup y = y$.
\end{proof}

\section{Some properties of frames}\label{some-frames}

This section is based on a \noun{Todd Trimble}'s proof. A shorter
but less elementary proof (also by \noun{Todd Trimble}) is available
at\\
\href{http://ncatlab.org/toddtrimble/published/topogeny}{http://ncatlab.org/toddtrimble/published/topogeny}

I will abbreviate \emph{join-semilattice with least element} as JSWLE.
\begin{obvious}
JSWLEs are the same as finitely join-closed posets (with nullary joins
included).
\end{obvious}

\begin{defn}
It is said that a function $f$ from a poset $\mathfrak{A}$ to a
poset $\mathfrak{B}$ \emph{preserves finite joins}, when for every
finite set $S\in\subsets\mathfrak{A}$ such that $\bigsqcup^{\mathfrak{A}}S$
exists we have $\bigsqcup^{\mathfrak{B}}\supfun f^{\ast}S=f\bigsqcup^{\mathfrak{A}}S$.\end{defn}
\begin{obvious}
A function between JSWLEs preserves finite joins iff it preserves
binary joins ($f(x\sqcup y)=fx\sqcup fy$) and nullary joins ($f(\bot^{\mathfrak{A}})=\bot^{\mathfrak{B}}$).\end{obvious}
\begin{defn}
A \emph{fixed point} of a function $F$ is such $x$ that $F(x)=x$.
We will denote $\Fix(F)$ the set of all fixed points of a function
$F$.
\end{defn}

\begin{defn}
Let $\mathfrak{A}$ be a JSWLE. A \emph{co-nucleus} is a function
$F:\mathfrak{A}\rightarrow\mathfrak{A}$ such that for every $p,q\in\mathfrak{A}$
we have:
\begin{enumerate}
\item \label{co-nucleus-less}$F(p)\sqsubseteq p$;
\item $F(F(p))=F(p)$;
\item $F(p\sqcup q)=F(p)\sqcup F(q)$.
\end{enumerate}
\end{defn}
\begin{prop}
Every co-nucleus is a monotone function.\end{prop}
\begin{proof}
It follows from $F(p\sqcup q)=F(p)\sqcup F(q)$.\end{proof}
\begin{lem}
$\bigsqcup^{\Fix(F)}S=\bigsqcup S$ for every $S\in\subsets\Fix(F)$
for every co-nucleus~$F$ on a complete lattice.\end{lem}
\begin{proof}
Obviously $\bigsqcup S\sqsupseteq x$ for every $x\in S$.

Suppose $z\sqsupseteq x$ for every $x\in S$ for a $z\in\Fix(F)$.
Then $z\sqsupseteq\bigsqcup S$.

$F\left(\bigsqcup S\right)\sqsupseteq F(x)$ for every $x\in S$.
Thus $F\left(\bigsqcup S\right)\sqsupseteq\bigsqcup_{x\in S}F(x)=\bigsqcup S$.
But $F\left(\bigsqcup S\right)\sqsubseteq\bigsqcup S$. Thus $F\left(\bigsqcup S\right)=\bigsqcup S$
that is $\bigsqcup S\in\Fix(F)$.

So $\bigsqcup^{\Fix(F)}S=\bigsqcup S$ by the definition of join.\end{proof}
\begin{cor}
$\bigsqcup^{\Fix(F)}S$ is defined for every $S\in\subsets\Fix(F)$.\end{cor}
\begin{lem}
$\bigsqcap^{\Fix(F)}S=F\left(\bigsqcap S\right)$ for every $S\in\subsets\Fix(F)$
for every co-nucleus~$F$ on a complete lattice.\end{lem}
\begin{proof}
Obviously $F\left(\bigsqcap S\right)\sqsubseteq x$ for every $x\in S$.

Suppose $z\sqsubseteq x$ for every $x\in S$ for a $z\in\Fix(F)$.
Then $z\sqsubseteq\bigsqcap S$ and thus $z\sqsubseteq F\left(\bigsqcap S\right)$.

So $\bigsqcap^{\Fix(F)}S=F\left(\bigsqcap S\right)$ by the definition
of meet.\end{proof}
\begin{cor}
$\bigsqcap^{\Fix(F)}S$ is defined for every $S\in\subsets\Fix(F)$.\end{cor}
\begin{obvious}
$\Fix(F)$ with induced order is a complete lattice.\end{obvious}
\begin{lem}
\label{fix-is-co-frame}If $F$ is a co-nucleus on a co-frame $\mathfrak{A}$,
then the poset $\Fix(F)$ of fixed points of $F$, with order inherited
from $\mathfrak{A}$, is also a co-frame.\end{lem}
\begin{proof}
Let $b\in\Fix(F)$, $S\in\subsets\Fix(F)$. Then 
\begin{align*}
b\sqcup^{\Fix(F)}\bigsqcap^{\Fix(F)}S & =\\
b\sqcup^{\Fix(F)}F\left(\bigsqcap S\right) & =\\
F(b)\sqcup F\left(\bigsqcap S\right) & =\\
F\left(b\sqcup\bigsqcap S\right) & =\\
F\left(\bigsqcap\langle b\sqcup\rangle^{\ast}S\right) & =\\
\bigsqcap^{\Fix(F)}\langle b\sqcup\rangle^{\ast}S & =\\
\bigsqcap^{\Fix(F)}\langle b\sqcup^{\Fix(F)}\rangle^{\ast}S.
\end{align*}
\end{proof}
\begin{defn}
Denote $\Upper(\mathfrak{A})$ the set of upper sets on $\mathfrak{A}$
ordered \emph{reverse} to set theoretic inclusion.\end{defn}
\begin{defn}
Denote $\uparrow a = \setcond{x\in\mathfrak{A}}{x\sqsupseteq a} \in \Upper(\mathfrak{A})$.
\end{defn}
\begin{lem}
The set $\Upper(\mathfrak{A})$ is closed under arbitrary meets and
joins.\end{lem}
\begin{proof}
Let $S\in\subsets\Upper(\mathfrak{A})$.

Let $X\in\bigcup S$ and $Y\sqsupseteq X$ for an $Y\in\mathfrak{A}$.
Then there is $P\in S$ such that $X\in P$ and thus $Y\in P$ and
so $Y\in\bigcup S$. So $\bigcup S\in\Upper(\mathfrak{A})$.

Let now $X\in\bigcap S$ and $Y\sqsupseteq X$ for an $Y\in\mathfrak{A}$.
Then $\forall T\in S:X\in T$ and so $\forall T\in S:Y\in T$, thus
$Y\in\bigcap S$. So $\bigcap S\in\Upper(\mathfrak{A})$.\end{proof}
\begin{thm}
\label{compl-via-down}A poset $\mathfrak{A}$ is a complete lattice
iff there is a antitone map $s:\Upper(\mathfrak{A})\rightarrow\mathfrak{A}$
such that
\begin{enumerate}
\item $s(\uparrow p)=p$ for every $p\in\mathfrak{A}$;
\item $D\subseteq\uparrow s(D)$ for every $D\in\Upper(\mathfrak{A})$.
\end{enumerate}
Moreover, in this case $s(D)=\bigsqcap D$ for every $D\in\Upper(\mathfrak{A})$.\end{thm}
\begin{proof}
~
\begin{description}
\item [{$\Rightarrow$}] Take $s(D)=\bigsqcap D$.
\item [{$\Leftarrow$}] $\forall x\in D:x\sqsupseteq s(D)$ from the second
formula.


Let $\forall x\in D:y\sqsubseteq x$. Then $x\in\uparrow y$, $D\subseteq\uparrow y$;
because $s$ is an antitone map, thus follows $s(D)\sqsupseteq s(\uparrow y)=y$.
So $\forall x\in D:y\sqsubseteq s(D)$.


That $s$ is the meet follows from the definition of meets.


It remains to prove that $\mathfrak{A}$ is a complete lattice.


Take any subset~$S$ of $\mathfrak{A}$. Let $D$ be the smallest
upper set containing~$S$. (It exists because $\Upper(\mathfrak{A})$
is closed under arbitrary joins.) This is 
\[
D=\setcond{x\in\mathfrak{A}}{\exists s\in S:x\sqsupseteq s}.
\]
Any lower bound of $D$ is clearly a lower bound of $S$ since $D\supseteq S$.
Conversely any lower bound of $S$ is a lower bound of $D$.~ Thus
$S$ and $D$ have the same set of lower bounds, hence have the same
greatest lower bound.

\end{description}
\end{proof}
\begin{prop}
\label{down-is-homo}For any poset $\mathfrak{A}$ the following are
mutually reverse order isomorphisms between upper sets $F$ (ordered
reverse to set-theoretic inclusion) on $\mathfrak{A}$ and order homomorphisms
$\varphi:\mathfrak{A}^{\op}\rightarrow2$ (here $2$ is the partially
ordered set of two elements: $0$ and $1$ where $0\sqsubseteq1$),
defined by the formulas 
\begin{enumerate}
\item \label{phi-01}$\varphi(a)=\left\{ \begin{array}{ll}
1 & \text{if }a\in F\\
0 & \text{if }a\notin F
\end{array}\right.$ for every $a\in\mathfrak{A}$;
\item \label{phi-inv}$F=\varphi^{-1}(1)$. 
\end{enumerate}
\end{prop}
\begin{proof}
Let $X\in\varphi^{-1}(1)$ and $Y\sqsupseteq X$. Then $\varphi(X)=1$
and thus $\varphi(Y)=1$. Thus $\varphi^{-1}(1)$ is a upper set.

It is easy to show that $\varphi$ defined by the formula~\ref{phi-01}
is an order homomorphism $\mathfrak{A}^{\op}\rightarrow2$ whenever
$F$ is a upper set.

Finally we need to prove that they are mutually inverse. Really: Let
$\varphi$ be defined by the formula~\ref{phi-01}. Then take $F'=\varphi^{-1}(1)$
and define $\varphi'(a)$ by the formula~\ref{phi-01}. We have 
\[
\varphi'(a)=\left\{ \begin{array}{ll}
1 & \text{if }a\in\varphi^{-1}(1)\\
0 & \text{if }a\notin\varphi^{-1}(1)
\end{array}\right.=\left\{ \begin{array}{ll}
1 & \text{if }\varphi(a)=1\\
0 & \text{if }\varphi(a)\neq1
\end{array}\right.=\varphi(a).
\]
Let now $F$ be defined by the formula~\ref{phi-inv}. Then take
$\varphi'(a)=\left\{ \begin{array}{ll}
1 & \text{if }a\in F\\
0 & \text{if }a\notin F
\end{array}\right.$ as defined by the formula~\ref{phi-01} and define $F'=\varphi'^{-1}(1)$.
Then 
\[
F'=\varphi'^{-1}(1)=F.
\]
\end{proof}
\begin{lem}
For a complete lattice $\mathfrak{A}$, the map $\bigsqcap:\Upper(\mathfrak{A})\rightarrow\mathfrak{A}$
preserves arbitrary meets.\end{lem}
\begin{proof}
Let $S\in\subsets\Upper(\mathfrak{A})$ . We have $\bigsqcap S\in\Upper(\mathfrak{A})$.

$\bigsqcap\bigsqcap S=\bigsqcap\bigsqcap_{X\in S}X=\bigsqcap_{X\in S}\bigsqcap X$
is what we needed to prove.\end{proof}
\begin{lem}
A complete lattice $\mathfrak{A}$ is a co-frame iff $\bigsqcap:\Upper(\mathfrak{A})\rightarrow\mathfrak{A}$
preserves finite joins.\end{lem}
\begin{proof}
~
\begin{description}
\item [{$\Rightarrow$}] Let $\mathfrak{A}$ be a co-frame. Let $D,D'\in\Upper(\mathfrak{A})$.
Obviously $\bigsqcap(D\sqcup D')\sqsupseteq\bigsqcap D$ and $\bigsqcap(D\sqcup D')\sqsupseteq\bigsqcap D'$,
so $\bigsqcap(D\sqcup D')\sqsupseteq\bigsqcap D\sqcup\bigsqcap D'$.


Also 
\begin{multline*}
\bigsqcap D\sqcup\bigsqcap D'=\bigcup D\sqcup\bigcup D'=\text{(because \ensuremath{\mathfrak{A}} is a co-frame)}=\\
\bigcup\setcond{d\sqcup d'}{d\in D,d'\in D'}.
\end{multline*}
Obviously $d\sqcup d'\in D\cap D'$, thus $\bigsqcap D\sqcup\bigsqcap D'\subseteq\bigcup(D\cap D')=\bigsqcap(D\cap D')$
that is $\bigsqcap D\sqcup\bigsqcap D'\sqsupseteq\bigsqcap(D\cap D')$.
So $\bigsqcap(D\sqcup D')=\bigsqcap D\sqcup\bigsqcap D'$ that is
$\bigsqcap:\Upper(A)\rightarrow A$ preserves binary joins.


It preserves nullary joins since $\bigsqcap^{\Upper(\mathfrak{A})}\bot^{\Upper(\mathfrak{A})}=\bigsqcap^{\Upper(\mathfrak{A})}\mathfrak{A}=\bot^{\mathfrak{A}}$.

\item [{$\Leftarrow$}] Suppose $\bigsqcap:\Upper(\mathfrak{A})\rightarrow\mathfrak{A}$
preserves finite joins. Let $b\in\mathfrak{A}$, $S\in\subsets\mathfrak{A}$.
Let $D$ be the smallest upper set containing $S$ (so $D=\bigcup\rsupfun{\uparrow}S$).
Then
\begin{align*}
b\sqcup\bigsqcap S & =\\
\bigsqcap\uparrow b\sqcup\bigcup\bigsqcap\rsupfun{\uparrow}S & =\\
\bigsqcap\uparrow b\sqcup\bigsqcap\bigcup\rsupfun{\uparrow}S & =\text{(since \ensuremath{\bigsqcap} preserves finite joins)}\\
\bigsqcap\left(\uparrow b\sqcup\bigcup\rsupfun{\uparrow}S\right) & =\\
\bigcup\left(\uparrow b\cap\bigcup\rsupfun{\uparrow}S\right) & =\\
\bigsqcap\bigcup_{a\in S}(\uparrow b\cap\uparrow a) & =\\
\bigsqcap\bigcup_{a\in S}\uparrow(b\sqcup a) & =\text{(since \ensuremath{\bigsqcap} preserves all meets)}\\
\bigcup_{a\in S}\bigsqcap\uparrow(b\sqcup a) & =\\
\bigcup_{a\in S}(b\sqcup a) & =\\
\bigsqcap_{a\in S}(b\sqcup a).
\end{align*}

\end{description}
\end{proof}
\begin{cor}
\label{down-meet-co-nucleus}If $\mathfrak{A}$ is a co-frame, then
the composition $F=\uparrow\circ\bigsqcap:\Upper(\mathfrak{A})\rightarrow\Upper(\mathfrak{A})$
is a co-nucleus. The embedding $\uparrow:\mathfrak{A}\rightarrow\Upper(\mathfrak{A})$
is an isomorphism of $\mathfrak{A}$ onto the co-frame $\Fix(F)$.\end{cor}
\begin{proof}
$D\sqsupseteq F(D)$ follows from theorem~\ref{compl-via-down}.

We have $F(F(D))=F(D)$ for all $D\in\Upper(\mathfrak{A})$ since
$F(F(D))=\uparrow\bigsqcap\uparrow\bigsqcap D=\text{(because \ensuremath{\bigsqcap\uparrow s=s} for any \ensuremath{s})}=\uparrow\bigsqcap D=F(D)$.

And since both $\bigsqcap:\Upper(\mathfrak{A})\rightarrow\mathfrak{A}$
and $\uparrow$ preserve finite joins, $F$ preserves finite joins.
Thus $F$ is a co-nucleus.

Finally, we have $a\sqsupseteq a'$ if and only if $\uparrow a\subseteq\uparrow a'$,
so that $\uparrow:\mathfrak{A}\rightarrow\Upper(\mathfrak{A})$ maps
$\mathfrak{A}$ isomorphically onto its image $\rsupfun{\uparrow}\mathfrak{A}$.
This image is $\Fix(F)$ because if $D$ is any fixed point (i.e.
if $D=\uparrow\bigsqcap D$), then $D$ clearly belongs to $\rsupfun{\uparrow}\mathfrak{A}$;
and conversely $\uparrow a$ is always a fixed point of $F=\uparrow\circ\bigsqcap$
since $F(\uparrow a)=\uparrow\bigsqcap\uparrow a=\uparrow a$.\end{proof}
\begin{defn}
If $\mathfrak{A}$, $\mathfrak{B}$ are two JSWLEs, then $\operatorname{Join}(\mathfrak{A};\mathfrak{B})$
is the (ordered pointwise) set of finite joins preserving maps $\mathfrak{A}\rightarrow\mathfrak{B}$.\end{defn}
\begin{obvious}
$\operatorname{Join}(\mathfrak{A};\mathfrak{B})$ is a JSWLE, where~$f\sqcup g$
is given by the formula $(f\sqcup g)(p)=f(p)\sqcup g(p)$, $\bot^{\operatorname{Join}(\mathfrak{A};\mathfrak{B})}$
is given by the formula $\bot^{\operatorname{Join}(\mathfrak{A};\mathfrak{B})}(p)=\bot^{\mathfrak{B}}$.\end{obvious}
\begin{defn}
Let $h:Q\rightarrow R$ be a finite joins preserving map. Then by
definition $\operatorname{Join}(P,h):\operatorname{Join}(P;Q)\rightarrow\operatorname{Join}(P;R)$
takes $f\in\operatorname{Join}(P;Q)$ into the composition $h\circ f\in\operatorname{Join}(P;R)$.\end{defn}
\begin{lem}
Above defined $\operatorname{Join}(P,h)$ is a finite joins preserving
map.\end{lem}
\begin{proof}
~

\begin{multline*}
(h\circ(f\sqcup f'))x=h(f\sqcup f')x=h(fx\sqcup f'x)=\\
hfx\sqcup hf'x=(h\circ f)x\sqcup(h\circ f')x=((h\circ f)\sqcup(h\circ f'))x.
\end{multline*}
Thus $h\circ(f\sqcup f')=(h\circ f)\sqcup(h\circ f')$.

$(h\circ\bot^{\operatorname{Join}(P;Q)})x=h\bot^{\operatorname{Join}(P;Q)}x=h\bot^Q=\bot^R$.\end{proof}
\begin{prop}
If $h,h':Q\rightarrow R$ are finite join preserving maps and $h\sqsupseteq h'$,
then $\operatorname{Join}(P,h)\sqsupseteq\operatorname{Join}(P,h')$.\end{prop}
\begin{proof}
$\operatorname{Join}(P,h)(f)(x)=(h\circ f)(x)=hfx\sqsupseteq h'fx=(h'\circ f)(x)=\operatorname{Join}(P,h')(f)(x)$.\end{proof}
\begin{lem}
If $g:Q\rightarrow R$ and $h:R\rightarrow S$ are finite joins preserving,
then the composition $\operatorname{Join}(P;h)\circ\operatorname{Join}(P;g)$
is equal to $\operatorname{Join}(P;h\circ g)$. Also $\operatorname{Join}(P;\id_{Q})$
for identity map $\id_{Q}$ on $Q$ is the identity map $\id_{\operatorname{Join}(P;Q)}$
on $\operatorname{Join}(P;Q)$.\end{lem}
\begin{proof}
$\operatorname{Join}(P;h)\operatorname{Join}(P;g)f=\operatorname{Join}(P;h)(g\circ f)=h\circ g\circ f=\operatorname{Join}(P;h\circ g)f$.

$\operatorname{Join}(P;\id_{Q})f=\id_{Q}\circ f=f$.\end{proof}
\begin{cor}
\label{join-map-co-nucleus}If $Q$ is a JSWLE and $F:Q\rightarrow Q$
is a co-nucleus, then for any JSWLE $P$ we have that 
\[
\operatorname{Join}(P;F):\operatorname{Join}(P;Q)\rightarrow\operatorname{Join}(P;Q)
\]
 is also a co-nucleus.\end{cor}
\begin{proof}
From $\id_{Q}\sqsupseteq F$ (co-nucleus axiom \ref{co-nucleus-less})
we have $\operatorname{Join}(P;\id_{Q})\sqsupseteq\operatorname{Join}(P;F)$
and since by the last lemma the left side is the identity on $\operatorname{Join}(P;Q)$,
we see that $\operatorname{Join}(P;F)$ also satisfies co-nucleus
axiom \ref{co-nucleus-less}.

$\operatorname{Join}(P;F)\circ\operatorname{Join}(P;F)=\operatorname{Join}(P;F\circ F)$
by the same lemma and thus $\operatorname{Join}(P;F)\circ\operatorname{Join}(P;F)=\operatorname{Join}(P;F)$
by the second co-nucleus axiom for $F$, showing that $\operatorname{Join}(P;F)$
satisfies the second co-nucleus axiom.

By an other lemma, we have that $\operatorname{Join}(P;F)$ preserves
binary joins, given that $F$ preserves binary joins, which is the
third co-nucleus axiom.\end{proof}
\begin{lem}
\label{join-fix-inter}$\Fix(\operatorname{Join}(P;F))=\operatorname{Join}(P;\Fix(F))$
for every JSWLEs $P$, $Q$ and a join preserving function $F:Q\rightarrow Q$.\end{lem}
\begin{proof}
$a\in\Fix(\operatorname{Join}(P;F))\Leftrightarrow a\in F^{P}\wedge F\circ a=a\Leftrightarrow a\in F^{P}\wedge\forall x\in P:F(a(x))=a(x)$.

$a\in\operatorname{Join}(P;\Fix(F))\Leftrightarrow a\in\Fix(F)^{P}\Leftrightarrow a\in F^{P}\wedge\forall x\in P:F(a(x))=a(x)$.

Thus $\Fix(\operatorname{Join}(P;F))=\operatorname{Join}(P;\Fix(F))$.
That the order of the left and right sides of the equality agrees
is obvious.\end{proof}
\begin{defn}
$\mathbf{Pos}(\mathfrak{A};\mathfrak{B})$ is the pointwise ordered
poset of monotone maps from a poset $\mathfrak{A}$ to a poset $\mathfrak{B}$.\end{defn}
\begin{lem}
\label{join-pos-interch}If $Q$, $R$ are JSWLEs and $P$ is a poset,
then $\mathbf{Pos}(P;R)$ is a JSWLE and $\mathbf{Pos}(P;\operatorname{Join}(Q;R))$
is isomorphic to $\operatorname{Join}\left(Q;\mathbf{Pos}(P;R)\right)$.
If $R$ is a co-frame, then also $\mathbf{Pos}(P;R)$ is a co-frame.
\end{lem}
\begin{proof}
Let $f,g\in\mathbf{Pos}(P;R)$. Then $\lambda x\in P:(fx\sqcup gx)$
is obviously monotone and then it is evident that $f\sqcup^{\mathbf{Pos}(P;R)}g=\lambda x\in P:(fx\sqcup gx)$.
$\lambda x\in P:\bot^{R}$ is also obviously monotone and it is evident
that $\bot^{\mathbf{Pos}(P;R)}=\lambda x\in P:\bot^{R}$.

Obviously both $\mathbf{Pos}(P;\operatorname{Join}(Q;R))$ and $\operatorname{Join}\left(Q;\mathbf{Pos}(P;R)\right)$
are sets of order preserving maps.

Let $f$ be a monotone map.

$f\in\mathbf{Pos}(P;\operatorname{Join}(Q;R))$ iff $f\in\operatorname{Join}(Q;R)^{P}$
iff $f\in\setcond{ g\in R^{Q} }{ g\text{ preserves finite joins}} ^{P}$
iff $f\in(R^{Q})^{P}$ and every $g=f(x)$ (for $x\in P$) preserving
finite joins. This is bijectively equivalent ($f\mapsto f'$) to $f'\in(R^{P})^{Q}$
preserving finite joins.

$f'\in\operatorname{Join}\left(Q;\mathbf{Pos}(P;R)\right)$ iff $f'$
preserves finite joins and $f'\in\mathbf{Pos}(P;R)^{Q}$ iff $f'$
preserves finite joins and $f'\in\setcond{g\in(R^{P})^{Q}}{g(x)\text{ is monotone}}$
iff $f'$ preserves finite joins and $f'\in(R^{P})^{Q}$.

So we have proved that $f\mapsto f'$ is a bijection between $\mathbf{Pos}(P;\operatorname{Join}(Q;R))$
and $\operatorname{Join}\left(Q;\mathbf{Pos}(P;R)\right)$. That it
preserves order is obvious.

It remains to prove that if $R$ is a co-frame, then also $\mathbf{Pos}(P;R)$
is a co-frame.

First, we need to prove that $\mathbf{Pos}(P;R)$ is a complete lattice.
But it is easy to prove that for every set $S\in\subsets\mathbf{Pos}(P;R)$
we have $\lambda x\in P:\bigsqcup_{f\in S}f(x)$ and $\lambda x\in P:\bigsqcap_{f\in S}f(x)$
are monotone and thus are the joins and meets on $\mathbf{Pos}(P;R)$.

Next we need to prove that 
\[
b\sqcup^{\mathbf{Pos}(P;R)}\bigsqcap^{\mathbf{Pos}(P;R)}S=\bigsqcap^{\mathbf{Pos}(P;R)}\left\langle b\sqcup^{\mathbf{Pos}(P;R)}\right\rangle ^{\ast}S.
\]
Really (for every $x\in P$),

\begin{multline*}
\left(b\sqcup^{\mathbf{Pos}(P;R)}\bigsqcap^{\mathbf{Pos}(P;R)}S\right)x=b(x)\sqcup\left(\bigsqcap^{\mathbf{Pos}(P;R)}S\right)x=\\
b(x)\sqcup\bigsqcap_{f\in S}f(x)=\bigsqcap_{f\in S}(b(x)\sqcup f(x))=\bigsqcap_{f\in S}\left(b\sqcup^{\mathbf{Pos}(P;R)}f\right)x=\\
\left(\bigsqcap_{f\in S}^{\mathbf{Pos}(P;R)}\left(b\sqcup^{\mathbf{Pos}(P;R)}f\right)\right)x.
\end{multline*}


Thus $b\sqcup^{\mathbf{Pos}(P;R)}\bigsqcap^{\mathbf{Pos}(P;R)}S=\bigsqcap_{f\in S}^{\mathbf{Pos}(P;R)}\left(b\sqcup^{\mathbf{Pos}(P;R)}f\right)=\bigsqcap^{\mathbf{Pos}(P;R)}\left\langle b\sqcup^{\mathbf{Pos}(P;R)}\right\rangle ^{\ast}S$.
\end{proof}
\begin{defn}
$P\cong Q$ means that posets $P$ and $Q$ are isomorphic.\end{defn}