\chapter{Applications of algebraic general topology}

\fxnote{Should also Consider ``one-side topologies'' like $T\sqcap\setcond{(x;y)}{x,y\in\mathbb{R},y\ge x}$ where
$T$ is the usual topology on~$\mathbb{R}$. Wrap a $\pi$-length segment of this line into a circle
and get one-side circle: We can traverse continuously counter-clockwise but not clockwise.}

\section{``Hybrid'' objects}

Algebraic general topology allows to construct ``hybrid'' objects of ``continuous'' (as topological spaces)
and discrete (as graphs).

Consider for example $D\sqcup T$ where $D$ is a digraph and $T$ is a topological space.
I am unsure whether there are important applications of such hybrid objects.

\section{Field of directions}

\fxwarning{Define \emph{curves} for beginning students.}

Compare with \url{https://en.wikipedia.org/wiki/Integral_curve}.

Consider the following class of problems:

Consider a plane (or $\mathbb{R}^n$ in general).

Let to any point of the plane corresponds set of directions (set of unit vectors, I mean). Remark: Sometimes these sets are of exactly one element.

Does there exists a smooth curve from point $A$ to point $B$ such that in every point of the curve its direction lies in the set of directions for this point?

It seems that this problem can be reformulated in the language of funcoids: From the sets of directions we easily construct certain funcoid.
Existence of such smooth curve is equivalent to existence of a continuous function from $[0;1]$ to our funcoid (it seems that we also need to add the
condition that it smooth, I have not yet defined it quite formally).

Explicitly construct this funcoid as (now, for simplicity only for $\mathbb{R}^2$):

$R(p;d;\phi) = \setcond{(p_0+x;p_1+y)}{\lvert\arctan\frac{y}{x}-\arctan{d}\rvert<\phi, x\ne 0\lor y\ne 0}$ \fxerror{Wrong sign of $\arctan$ argument} for a plane point~$p$, direction (unit vector)~$d$ and an angle~$\phi$.

$W(p;d) = \bigsqcap^{\mathscr{F}}\setcond{R(d;\phi)}{\phi\in\mathbb{R},\phi>0}$.

Finally our funcoid $Q=\bigsqcup_{p\in\mathbb{R}, d\in D(p)} (W(p;d) \sqcap ((\Delta+p_0) \times^{\mathsf{FCD}} (\Delta+p_1))$. ($\Delta$ is taken from the ``counter-examples'' section.)
\fxwarning{Rewrite this formula to add $p$ once.}

Here $D(p)$ is the given set of directions for a point~$p$.

One advantage to use funcoids in this case, that when describing the curves we don't need to require that the curve is smooth. I think, it's a big advantage.

\fxwarning{Add proof, rewrite in a more understandable way, $\mathbb{R}^n$ in general.}

\begin{conjecture}
If the curve goes only through points with exactly one direction, then the curve is smooth (if it is continuous regarding the above funcoid).
\end{conjecture}

\section{A way to construct directed topological spaces}

\fxnote{Should include definition of directed topological space.}

Directed topological spaces are defined at\\
\url{http://ncatlab.org/nlab/show/directed+topological+space}

I propose a new way to construct a directed topological space. My way is more geometric/topological as it does not involve dealing with particular paths.

\begin{conjecture}
Every directed topological space can be constructed in the below described way.
\end{conjecture}

Consider topological space $T$ and its subfuncoid $F$ (that is $F$ is a funcoid which is less that $T$ in the order of funcoids).
Note that in our consideration $F$ is an endofuncoid (its source and destination are the same).

Then a directed path from point $A$ to point $B$ is defined as a continuous function $f$ from $[0;1]$ to $F$ such that $f(0)=A$ and $f(1)=B$.
\fxwarning{Specify whether the interval $[0;1]$ is treated as a proximity, pretopology, or preclosure.}

Because $F$ is less that $T$, we have that every directed path is a path.

\begin{conjecture}
The two directed topological spaces, constructed from a fixed topological space and two different funcoids,
are different.
\end{conjecture}

For a counter-example of (which of the two?) the conjecture consider funcoid $T\sqcap(\mathbb{Q}\times^{\mathsf{FCD}}\mathbb{Q})$
where $T$ is the usual topology on real line.

We need to consider stability of existence and uniqueness of a path under transformations of our funcoid and
under transformations of the vector field. Can this be a step to solve Navier-Stokes existence and smoothness problems?

It seems (check!) that solutions not only of differential equations but also of difference equations can be
expressed as paths in funcoids.