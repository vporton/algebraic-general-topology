\chapter{Applications of algebraic general topology}

\section{``Hybrid'' objects}

Algebraic general topology allows to construct ``hybrid'' objects of ``continuous'' (as topological spaces)
and discrete (as graphs).

Consider for example $D\sqcup T$ where $D$ is a digraph and $T$ is a topological space.

The $n$-th power $(D\sqcup T)^n$ yields an expression with $2^n$ terms.
So treating $D\sqcup T$ as one object (what becomes possible using algebraic general topology)
rather than the join of two objects may have an exponential benefit for simplicity of formulas.

\section{A way to construct directed topological spaces}

\subsection{Some notation}

I use~$\mathcal{E}$ and $\iota$ notations from {\tt volume-2.pdf}. \fxwarning{Reorder document fragments to describe it before use.}

I remind that $f|_X = f\circ \id_X$ for binary relations, funcoids, and reloid.

$f\parallel_X = f\circ(\mathcal{E}^X)^{-1}$.

$f\square X = \id_X\circ f\circ\id_X^{-1}$.

As proved in {\tt volume-2.pdf}, the following are bijections and moreover isomorphisms (for $R$ being either funcoids or reloids or binary relations):
\begin{enumerate}
\item $\setcond{(f|_X,f\parallel_X)}{f\in R}$;
\item $\setcond{(f\square X,\iota_X f)}{f\in R}$.
\end{enumerate}

As easily follows from these isomorphisms and theorem~\bookref{rect-cont}:

\begin{prop}
For funcoids, reloids, and binary relations:
\begin{enumerate}
\item $f\in\continuous(\mu,\nu)\Rightarrow f\parallel_{A}\in\continuous(\iota_A\mu,\nu)$;
\item $f\in\continuous'(\mu,\nu)\Rightarrow f\parallel_{A}\in\continuous'(\iota_A\mu,\nu)$;
\item $f\in\continuous''(\mu,\nu)\Rightarrow f\parallel_{A}\in\continuous''(\iota_A\mu,\nu)$. 
\end{enumerate}
\end{prop}

\subsection{Directed line and directed intervals}

Let $\mathfrak{A}$ be a poset. We will denote $\overline{\mathfrak{A}}=\mathfrak{A}\cup\{-\infty,+\infty\}$ the poset
with two added elements $-\infty$ and $+\infty$, such that $+\infty$ is strictly greater than every element of~$\mathfrak{A}$
and $-\infty$ is strictly less.

\begin{defn}
For an element~$a$ of a poset~$\mathfrak{A}$
\begin{enumerate}
\item $J_{\geq}(a) = \setcond{x\in\mathfrak{A}}{x\geq a}$;
\item $J_{>}(a) = \setcond{x\in\mathfrak{A}}{x>a}$;
\item $J_{\leq}(a) = \setcond{x\in\mathfrak{A}}{x\leq a}$;
\item $J_{<}(a) = \setcond{x\in\mathfrak{A}}{x<a}$.
\end{enumerate}
\end{defn}

\begin{defn}
Let $a$ be an element of a poset~$\mathfrak{A}$.
\begin{enumerate}
\item $\Delta(a) = \bigsqcap^{\mathscr{F}} \setcond{]x;y[}{x,y\in\overline{\mathfrak{A}}, x<a\land y>a}$;
\item $\Delta_{\geq}(a) = \bigsqcap^{\mathscr{F}} \setcond{[a;y[}{y\in\overline{\mathfrak{A}}, y>a}$;
\item $\Delta_{>}(a) = \bigsqcap^{\mathscr{F}} \setcond{]a;y[}{y\in\overline{\mathfrak{A}}, x<a\land y>a}$;
\item $\Delta_{\leq}(a) = \bigsqcap^{\mathscr{F}} \setcond{]x;a]}{x\in\overline{\mathfrak{A}}, x<a}$;
\item $\Delta_{<}(a) = \bigsqcap^{\mathscr{F}} \setcond{]x;a[}{x\in\overline{\mathfrak{A}}, x<a}$.
\end{enumerate}
\end{defn}

\begin{obvious}
~
\begin{enumerate}
\item $\Delta_{\geq}(a) = \Delta(a)\sqcap^{\mathscr{F}} @J_{\geq a}$;
\item $\Delta_{>}(a) = \Delta(a)\sqcap^{\mathscr{F}} @J_{>a}$;
\item $\Delta_{\leq}(a) = \Delta(a)\sqcap^{\mathscr{F}} @J_{\leq a}$;
\item $\Delta_{<}(a) = \Delta(a)\sqcap^{\mathscr{F}} @J_{<a}$.
\end{enumerate}
\end{obvious}

\begin{defn}
~
Given a partial order~$\mathfrak{A}$ and~$x\in\mathfrak{A}$, the following defines complete funcoids:
\begin{enumerate}
\item $\rsupfun{|\mathfrak{A}|}\{x\} = \Delta(x)$;
\item $\rsupfun{|\mathfrak{A}|_{\geq}}\{x\} = \Delta_{\geq}(x)$;
\item $\rsupfun{|\mathfrak{A}|_{>}}\{x\} = \Delta_{>}(x)$;
\item $\rsupfun{|\mathfrak{A}|_{\leq}}\{x\} = \Delta_{\leq}(x)$;
\item $\rsupfun{|\mathfrak{A}|_{<}}\{x\} = \Delta_{<}(x)$.
\end{enumerate}
\end{defn}

\begin{prop}
The complete funcoid corresponding to the order topology\footnote{See Wikipedia for a definition of ``Order topology''.}
is equal to $|\mathfrak{A}|$.
\end{prop}

\begin{proof}
Because every open set is a finite union of open intervals, the complete funcoid~$f$ corresponding to the order topology
is described by the formula: $\rsupfun{f}\{x\} = \bigsqcap^{\mathscr{F}}\setcond{]a;b[}{a,b\in\overline{\mathfrak{A}}, a<x\land b>x} =
\Delta(x) = \rsupfun{|\mathfrak{A}|}\{x\}$. Thus $f=|\mathfrak{A}|$.
\end{proof}

\begin{xca}
Show that $|\mathfrak{A}|_{\geq}$ (in general) is not the same as ``right order topology''\footnote{See Wikipedia}.
\end{xca}

\begin{prop}
~
\begin{enumerate}
\item $\rsupfun{|\mathfrak{A}|_{\geq}^{-1}}@X = @\setcond{a\in\mathfrak{A}}{\forall y\in\overline{\mathfrak{A}}:(y>a \Rightarrow X\cap[a;y[\ne\emptyset)}$;
\item $\rsupfun{|\mathfrak{A}|_{>}^{-1}}@X = @\setcond{a\in\mathfrak{A}}{\forall y\in\overline{\mathfrak{A}}:(y>a \Rightarrow X\cap]a;y[\ne\emptyset)}$;
\item $\rsupfun{|\mathfrak{A}|_{\leq}^{-1}}@X = @\setcond{a\in\mathfrak{A}}{\forall x\in\overline{\mathfrak{A}}:(x<a \Rightarrow X\cap]x;a]\ne\emptyset)}$;
\item $\rsupfun{|\mathfrak{A}|_{<}^{-1}}@X = @\setcond{a\in\mathfrak{A}}{\forall x\in\overline{\mathfrak{A}}:(x<a \Rightarrow X\cap]x;a[\ne\emptyset)}$.
\end{enumerate}
\end{prop}

\begin{proof}
$a\in\rsupfun{|\mathfrak{A}|_{\geq}^{-1}}@X \Leftrightarrow
@\{a\} \nasymp \rsupfun{|\mathfrak{A}|_{\geq}^{-1}}@X \Leftrightarrow
\rsupfun{|\mathfrak{A}|_{\geq}}@\{a\} \nasymp @X \Leftrightarrow
\Delta_{\geq}(a) \nasymp @X \Leftrightarrow
\forall y\in\overline{\mathfrak{A}}:(y>a \Rightarrow X\cap[a;y[\ne\emptyset)$.

$a\in\rsupfun{|\mathfrak{A}|_{>}^{-1}}@X \Leftrightarrow
@\{a\} \nasymp \rsupfun{|\mathfrak{A}|_{>}^{-1}}@X \Leftrightarrow
\rsupfun{|\mathfrak{A}|_{>}}@\{a\} \nasymp @X \Leftrightarrow
\Delta_{>}(a) \nasymp @X \Leftrightarrow
\forall y\in\overline{\mathfrak{A}}:(y>a \Rightarrow X\cap]a;y[\ne\emptyset)$.

The rest follows from duality.
\end{proof}

\begin{rem}
On trivial ultrafilters these obviously agree:
\begin{enumerate}
\item $\rsupfun{|\mathbb{R}|_{\geq}}\{x\} = \rsupfun{|\mathbb{R}| \sqcap \geq}\{x\}$;
\item $\rsupfun{|\mathbb{R}|_{>}}\{x\} = \rsupfun{|\mathbb{R}| \sqcap >}\{x\}$;
\item $\rsupfun{|\mathbb{R}|_{\leq}}\{x\} = \rsupfun{|\mathbb{R}| \sqcap \leq}\{x\}$;
\item $\rsupfun{|\mathbb{R}|_{<}}\{x\} = \rsupfun{|\mathbb{R}| \sqcap <}\{x\}$.
\end{enumerate}
\end{rem}

\begin{cor}
~
\begin{enumerate}
\item $|\mathbb{R}|_{\geq} = \Compl(|\mathbb{R}| \sqcap \geq)$;
\item $|\mathbb{R}|_{>} = \Compl(|\mathbb{R}| \sqcap >)$;
\item $|\mathbb{R}|_{\leq} = \Compl(|\mathbb{R}| \sqcap \leq)$;
\item $|\mathbb{R}|_{<} = \Compl(|\mathbb{R}| \sqcap <)$.
\end{enumerate}
\end{cor}

\section{Some inequalities}

We will prove

\begin{thm}
~
\begin{enumerate}
\item $|\mathbb{R}|_{\geq} \sqsubset |\mathbb{R}| \sqcap \geq$;
\item $|\mathbb{R}|_{>} \sqsubset |\mathbb{R}| \sqcap >$;
\item $|\mathbb{R}|_{\leq} \sqsubset |\mathbb{R}| \sqcap \leq$;
\item $|\mathbb{R}|_{<} \sqsubset |\mathbb{R}| \sqcap <$.
\end{enumerate}
\end{thm}

From the above it easily follows that it's enough to prove
\[ |\mathbb{R}|_{\geq} \ne |\mathbb{R}| \sqcap \geq \quad \text{and} \quad
|\mathbb{R}|_{>} \ne |\mathbb{R}| \sqcap >. \]

\fxwarning{Below there is a proof only of $|\mathbb{R}|_{\geq} \ne |\mathbb{R}| \sqcap \geq$ not of
$|\mathbb{R}|_{>} \ne |\mathbb{R}| \sqcap >$.}

To check the inequalities, we will calculate $\supfun{|\mathbb{R}|_{\geq}} x$ and
$\supfun{|\mathbb{R}| \sqcap \geq} x$ for ultrafilters~$x$.

For simplicity, we will calculate it only for convergent ultrafilters (these
of them which are required for disproving the equalities and for completeness
these which are not required). (Exercise: Show that for non-convergent
ultrafilters there are no simple formulas like below.)

Note that on trivial ultrafilters they are already calculated above.

\fxwarning{Define the ultrafilter ``at the left'' and ``at the right'' of a real number.
Also define ``convergent ultrafilter''.}

\begin{prop}
~
\begin{enumerate}
\item\label{comp-ord-ge} $| \mathbb{R} |_{\geq} \circ | \mathbb{R} |_{\geq} = | \mathbb{R} |_{\geq}$;
\item\label{comp-ord-gt} $| \mathbb{R} |_{>} \circ | \mathbb{R} |_{>} = | \mathbb{R} |_{>}$;
\item\label{comp-ord-gt} $| \mathbb{R} |_{\geq} \circ | \mathbb{R} |_{>} = | \mathbb{R} |_{>}$;
\item\label{comp-ord-gt} $| \mathbb{R} |_{>} \circ | \mathbb{R} |_{\geq} = | \mathbb{R} |_{>}$.
\end{enumerate}  
\end{prop}

\begin{proof}
~
\begin{widedisorder}
\item[\ref{comp-ord-ge}]
  $| \mathbb{R} |_{\geq} \circ | \mathbb{R} |_{\geq} \sqsupseteq |
  \mathbb{R} |_{\geq}$ because $| \mathbb{R} |_{\geq} \sqsupseteq
  {(\mathord{=})}$.
  
  $| \mathbb{R} |_{\geq} \circ | \mathbb{R} |_{\geq} \sqsubseteq \setcond{ (x,
  y) }{ x \geq y \wedge | x - y | < \varepsilon
  } \circ \setcond{ (x, y) }{ x \geq y \wedge |
  x - y | < \varepsilon } = \setcond{ (x, y) }{
  x \geq y \wedge | x - y | < 2 \varepsilon }$.
  
  So $\supfun{ | \mathbb{R} |_{\geq} \circ | \mathbb{R} |_{\geq} } z
  \sqsubseteq \supfun{ \setcond{ (x, y) }{ x \geq y
  \wedge | x - y | < 2 \varepsilon } } z \sqsubseteq [0 ; 2
  \varepsilon [+ z$ for every ultrafilter $z$ and $\varepsilon > 0$. Thus
  $\supfun{ | \mathbb{R} |_{\geq} \circ | \mathbb{R} |_{\geq} } z
  \sqsubseteq \Delta_{\geq} + z$ and so $\supfun{ | \mathbb{R} |_{\geq} \circ |
  \mathbb{R} |_{\geq} } z \sqsubseteq \supfun{ | \mathbb{R} |_{\geq}
  } z$.
\end{widedisorder}
??
\end{proof}

\begin{prop}
?? Most likely, there are counterexamples.
\begin{enumerate}
\item $(|\mathbb{R}| \sqcap {\geq}) \circ (|\mathbb{R}| \sqcap {\geq}) = (|\mathbb{R}| \sqcap {\geq})$.
\item $(|\mathbb{R}| \sqcap {>}) \circ (|\mathbb{R}| \sqcap {>}) = (|\mathbb{R}| \sqcap {>})$.
\end{enumerate}
\end{prop}

\begin{proof}
??
\end{proof}

\begin{lem}
  If $x$ is a convergent ultrafilter at the right of its limit point $\alpha$,
  $\supfun{ | \mathbb{R} |_{\geq} } x = \Delta_{>} + \alpha$.
\end{lem}

\begin{proof}
  $x \sqsubseteq \Delta_{>} + \alpha$, thus $\supfun{ | \mathbb{R} |_{\geq}
  } x \sqsubseteq \supfun{ | \mathbb{R} |_{\geq} } (\Delta_{>} +
  \alpha) = \supfun{ | \mathbb{R} |_{\geq} } \supfun{ | \mathbb{R}
  |_{\geq} } \{ \alpha \} = \supfun{ | \mathbb{R} |_{\geq} } \{
  \alpha \} = \Delta_{\geq} + \alpha$ and obviously $\supfun{ | \mathbb{R}
  |_{\geq} } x \sqsubseteq] \alpha ; + \infty [$. So $\supfun{ |
  \mathbb{R} |_{\geq} } x \sqsubseteq \Delta_{>} + \alpha$.
  
  $\supfun{ | \mathbb{R} |_{\geq} } x = \bigsqcap_{X \in \up x}
  \supfun{ | \mathbb{R} |_{\geq} } X$.
  
  For every $Y$ such that $Y \nasymp \Delta_{>} + \alpha$ we have $Y \nasymp
  \supfun{ | \mathbb{R} |_{\geq} } X$ for some
  \fxerror{Not enough ``for some'', need ``for every''!}
  $X \in \up x$ and
  thus by properties of generalized filter bases $Y \nasymp \bigsqcap_{X \in
  \up x} \supfun{ | \mathbb{R} |_{\geq} } X$. So $\Delta_{>} +
  \alpha \sqsubseteq \supfun{ | \mathbb{R} |_{\geq} } x$.
\end{proof}

\begin{lem}
  If $x$ is a convergent ultrafilter at the left of its
  limit point $\alpha$, $\supfun{ | \mathbb{R} |_{\geq} } x = \bot$.
\end{lem}

\begin{proof}
  Take a strictly increasing sequence $z_i$ where $\im z \sqsupseteq x$
  and $\lim z = \alpha$.
  
  Let $X_0 = \bigcup_{i \in \mathbb{N}} [x_{2 i} ; x_{2 i + 1} [$, $X_1 =
  \bigcup_{i \in \mathbb{N}} [x_{2 i + 1} ; x_{2 i + 2} [$.
  
  $\supfun{ | \mathbb{R} |_{\geq} } x \sqsubseteq \supfun{ | \mathbb{R}
  |_{\geq} } X = \bigsqcup_{\alpha \in X} (\Delta_{\geq} + \alpha)
  \sqsubseteq \bigsqcup_{\alpha \in X} (\Delta_{\geq} + \alpha) \sqsubseteq
  X_0$ if we take $X = i \mapsto x_{2 i}$ and likewise $\supfun{ | \mathbb{R}
  |_{\geq} } x \sqsubseteq X_1$ (take $X = i \mapsto x_{2 i + 1}$).
  
  Thus $\supfun{ | \mathbb{R} |_{\geq} } x \sqsubseteq X_0 \cap X_1 =
  \bot$.
\end{proof}

\begin{lem}
  If $x$ is a convergent ultrafilter at the right of its limit point $\alpha$,
  $\supfun{| \mathbb{R} |} x = \Delta_{>} + \alpha$.
\end{lem}

\begin{proof}
  $\supfun{| \mathbb{R} |} x \sqsubseteq \Delta_{>} + \alpha$ is
  obvious.
  
  For every $Y$ such that $Y \nasymp \Delta_{>} + \alpha$ we have $Y \nasymp
  \supfun{| \mathbb{R} |_{>}} X$ for some $X \in \up x$ and thus
  by properties of generalized filter bases $Y \nasymp \bigsqcap_{X \in
  \up x} \supfun{| \mathbb{R} |_{>}} X$. So $\Delta_{>} + \alpha
  \sqsubseteq \supfun{| \mathbb{R} |_{>}} x$.
\end{proof}

\begin{lem}
  If $x$ is a convergent ultrafilter at the right of its limit point $\alpha$,
  $\supfun{\geq} x = \Delta_{>} + \alpha$.
\end{lem}

\begin{proof}
  $\supfun{\geq} x \sqsubseteq \Delta_{>} + \alpha$ is obvious.
  Obviously $\supfun{\geq} x \sqsupseteq \supfun{ | \mathbb{R} |_{\geq}
  } x = \Delta_{>} + \alpha$. So $\supfun{\geq} x = \Delta_{>} +
  \alpha$.
\end{proof}

\begin{lem}
  If $x$ is a convergent ultrafilter at the right of its limit point $\alpha$,
  $\supfun{ | \mathbb{R} | \sqcap \geq } x = \Delta_{>} + \alpha$.
\end{lem}

\begin{proof}
  From two previous lemmas and $\supfun{ | \mathbb{R} | \sqcap \geq } x
  = \supfun{| \mathbb{R} |} \sqcap \supfun{\geq} x$.
\end{proof}

\begin{lem}
  If $x$ is a convergent ultrafilter at the left of its limit point $\alpha$,
  $\supfun{| \mathbb{R} |} x = \Delta_{<} + \alpha$.
\end{lem}

\begin{proof}
  By symmetry from an above lemma.
\end{proof}

\begin{lem}
  If $x$ is a convergent ultrafilter at the left of its limit point $\alpha$,
  $\supfun{\geq} x \sqsupseteq \Delta_{<} + \alpha$.
\end{lem}

\begin{proof}
  $\supfun{\geq} X \sqsupseteq] - \varepsilon ; 0 [$ for some
  $\varepsilon > 0$ for every $X \in \up x$. Therefore $\supfun{ \geq
  } X = \bigsqcap^{\mathscr{F}}_{X \in \up x} \rsupfun{ \geq
  } X \sqsupseteq \Delta_{<} + \alpha$.
\end{proof}

\begin{lem}
  If $x$ is a convergent ultrafilter at the left of its
  limit point $\alpha$, $\supfun{ | \mathbb{R} | \sqcap \geq } x =
  \Delta_{<} + \alpha$.
\end{lem}

\begin{proof}
From above.
\end{proof}

\begin{proof}
  (of the theorem above) $| \mathbb{R} |_{\geq} \neq | \mathbb{R} | \sqcap
  \geq$ follows from $\supfun{ | \mathbb{R} | \sqcap \geq } x =
  \Delta_{<} + \alpha \neq \bot = \supfun{ | \mathbb{R} |_{\geq} } x$
  for a convergent ultrafilter at the left of its limit point $\alpha$.
\end{proof}

\section{Continuity}

I will say that a property holds on a filter~$\mathcal{A}$ iff there is $A\in\up\mathcal{A}$ on which the property holds.

\fxnote{$f\in\continuous(A,B)\land f\in\continuous(\iota_A|\mathbb{R}|_{\geq},\iota_B|\mathbb{R}|_{\geq}) \Leftrightarrow
(f,f)\in\continuous((A,\iota_A|\mathbb{R}|_{\geq}),(B,\iota_B|\mathbb{R}|_{\geq}))$}

\begin{lem}
Let function~$f:A\rightarrow B$ where $A,B\in\subsets\mathbb{R}$ and $A$ is connected.
\begin{enumerate}
\item $f$ is monotone and $f\in\continuous(A,B)$ iff
$f\in\continuous(A,B)\cap\continuous(\iota_A|\mathbb{R}|_{\geq},\iota_B|\mathbb{R}|_{\geq})$ iff
$f\in\continuous(A,B)\cap\continuous(\iota_A|\mathbb{R}|_{>},\iota_B|\mathbb{R}|_{\geq})$ iff
$f\in\continuous(\iota_A|\mathbb{R}|_{\geq},\iota_B|\mathbb{R}|_{\geq})\cap
\continuous(\iota_A|\mathbb{R}|_{\leq},\iota_B|\mathbb{R}|_{\leq})$.
\item $f$ is strictly monotone and $\in\continuous(A,B)$ iff
$f\in\continuous(A,B)\cap\continuous(\iota_A|\mathbb{R}|_{>},\iota_B|\mathbb{R}|_{>})$ iff
$f\in\continuous(\iota_A|\mathbb{R}|_{>},\iota_B|\mathbb{R}|_{>})\cap
\continuous(\iota_A|\mathbb{R}|_{<},\iota_B|\mathbb{R}|_{<})$.
\end{enumerate}
\fxnote{Generalize for arbitrary posets.}
\fxnote{Generalize for $f$ being a funcoid.}
\end{lem}

\begin{proof}
Because $f$ is continuous, we have $\rsupfun{f\circ\iota_A|\mathbb{R}|}\{x\} \sqsubseteq \rsupfun{\iota_B|\mathbb{R}|\circ f}\{x\}$
that is $\rsupfun{f} \Delta(x) \sqsubseteq \Delta(f(x))$ for every~$x$.

If $f$ is monotone, we have $\rsupfun{f} \Delta_{\geq}(x) \sqsubseteq [f(x);\infty[$.
Thus $\rsupfun{f} \Delta_{\geq}(x) \sqsubseteq \Delta_{\geq}(f(x))$, that is
$\rsupfun{f\circ \iota_A|\mathbb{R}|_{\geq}}\{x\} \sqsubseteq \rsupfun{\iota_B|\mathbb{R}|_{\geq}\circ f}\{x\}$, thus
$f\in\continuous(\iota_A|\mathbb{R}|_{\geq},\iota_B|\mathbb{R}|_{\geq})$.

If $f$ is strictly monotone, we have $\rsupfun{f} \Delta_{>}(x) \sqsubseteq ]f(x);\infty[$.
Thus $\rsupfun{f} \Delta_{>}(x) \sqsubseteq \Delta_{>}(f(x))$, that is
$\rsupfun{f\circ \iota_A|\mathbb{R}|_{>}}\{x\} \sqsubseteq \rsupfun{\iota_B|\mathbb{R}|_{>}\circ f}\{x\}$, thus
$f\in\continuous(\iota_A|\mathbb{R}|_{>},\iota_B|\mathbb{R}|_{>})$.

Let now $f\in\continuous(\iota_A|\mathbb{R}|_{\geq},\iota_B|\mathbb{R}|_{\geq})$.

Take any~$a\in A$ and let $c=\setcond{b\in B}{b\geq a, \forall x\in[a;b[: f(x)\geq f(a)}$.
It's enough to prove that $c$ is the right endpoint (finite or infinite) of~$A$.

Indeed by continuity $f(a)\leq f(c)$ and if $c$ is not already the right endpoint of~$A$, then
there is $b'>c$ such that $\forall x\in[c;b'[: f(x)\geq f(c)$.
So we have $\forall x\in[a;b'[: f(x)\geq f(c)$ what contradicts to the above.

So $f$ is monotone on the entire~$A$.

$f\in\continuous(\iota_A|\mathbb{R}|_{\geq},\iota_B|\mathbb{R}|_{\geq}) \Rightarrow f\in\continuous(\iota_A|\mathbb{R}|_{>},\iota_B|\mathbb{R}|_{\geq})$ is obvious. Reversely
$f\in\continuous(\iota_A|\mathbb{R}|_{>},\iota_B|\mathbb{R}|_{\geq}) \Rightarrow
f\circ \iota_A|\mathbb{R}|_{>} \sqsubseteq \iota_B|\mathbb{R}|_{\geq}\circ f \Leftrightarrow
\forall x\in\mathbb{R}: \supfun{f}\rsupfun{\iota_A|\mathbb{R}|_{>}}\{x\} \sqsubseteq \rsupfun{\iota_B|\mathbb{R}|_{\geq}}\rsupfun{f}\{x\} \Leftrightarrow
\forall x\in\mathbb{R}: \supfun{f}\Delta_{>}(x) \sqsubseteq \Delta_{\geq}f(x) \Leftrightarrow
\forall x\in\mathbb{R}: \supfun{f}\Delta_{>}(x) \sqcup \{f(x)\} \sqsubseteq \Delta_{\geq}f(x) \Leftrightarrow
\forall x\in\mathbb{R}: \supfun{f}\Delta_{>}(x) \sqcup \{x\} \sqsubseteq \Delta_{\geq}f(x) \Leftrightarrow
\forall x\in\mathbb{R}: \supfun{f}\Delta_{\geq}(x) \sqsubseteq \Delta_{\geq}f(x) \Leftrightarrow
\forall x\in\mathbb{R}: \supfun{f}\rsupfun{\iota_A|\mathbb{R}|_{\geq}}\{x\} \sqsubseteq \rsupfun{\iota_B|\mathbb{R}|_{\geq}}\rsupfun{f}\{x\} \Leftrightarrow
\forall x\in\mathbb{R}: f\circ \iota_A|\mathbb{R}|_{\geq} \sqsubseteq \iota_B|\mathbb{R}|_{\geq}\circ f \Leftrightarrow
f\in\continuous(\iota_A|\mathbb{R}|_{\geq},\iota_B|\mathbb{R}|_{\geq})$.

Let $f\in\continuous(\iota_A|\mathbb{R}|_{>},\iota_B|\mathbb{R}|_{>})$. Then $f\in\continuous(\iota_A|\mathbb{R}|_{>},\iota_B|\mathbb{R}|_{\geq})$ and thus it is monotone.
We need to prove that $f$ is strictly monotone.
Suppose the contrary. Then there is a nonempty interval $[p;q]\subseteq A$ such that $f$ is constant on this interval.
But this is impossible because $f\in\continuous(\iota_A|\mathbb{R}|_{>},\iota_B|\mathbb{R}|_{>})$.

Prove that $f\in\continuous(\iota_A|\mathbb{R}|_{\geq},\iota_B|\mathbb{R}|_{\geq})\cap
\continuous(\iota_A|\mathbb{R}|_{\leq},\iota_B|\mathbb{R}|_{\leq})$ implies
$f\in\continuous(A,B)$. Really, it implies
$\supfun{f}\Delta_{\leq}(x)\sqsubseteq\Delta_{\leq}(fx)$ and $\supfun{f}\Delta_{\geq}(x)\sqsubseteq\Delta_{\geq}(fx)$
thus $\supfun{f}\Delta(x) = \supfun{f}(\Delta_{\leq}(x)\sqcup\{x\}\sqcup\Delta_{\geq}(x)) \subseteq
\Delta_{\leq}f(x)\sqcup\{f(x)\}\sqcup\Delta_{\geq}f(x) =
\Delta(f(x))$.

Prove that $f\in\continuous(\iota_A|\mathbb{R}|_{>},\iota_B|\mathbb{R}|_{>})\cap
\continuous(\iota_A|\mathbb{R}|_{<},\iota_B|\mathbb{R}|_{<})$
$f\in\continuous(A,B)$. Really, it implies
$\supfun{f}\Delta_{<}(x)\sqsubseteq\Delta_{<}(fx)$ and $\supfun{f}\Delta_{>}(x)\sqsubseteq\Delta_{>}(fx)$
thus $\supfun{f}\Delta(x) = \supfun{f}(\Delta_{<}(x)\sqcup\{x\}\sqcup\Delta_{>}(x)) \subseteq
\Delta_{<}f(x)\sqcup\{f(x)\}\sqcup\Delta_{>}f(x) = \Delta(f(x))$.
\end{proof}

\begin{thm}
Let function~$f:A\rightarrow B$ where $A,B\in\subsets\mathbb{R}$.
\begin{enumerate}
\item $f$ is locally monotone and $f\in\continuous(A,B)$ iff
$f\in\continuous(A,B)\cap\continuous(\iota_A|\mathbb{R}|_{\geq},\iota_B|\mathbb{R}|_{\geq})$ iff
$f\in\continuous(A,B)\cap\continuous(\iota_A|\mathbb{R}|_{>},\iota_B|\mathbb{R}|_{\geq})$ iff
$f\in\continuous(\iota_A|\mathbb{R}|_{\geq},\iota_B|\mathbb{R}|_{\geq})\cap
\continuous(\iota_A|\mathbb{R}|_{\leq},\iota_B|\mathbb{R}|_{\leq})$.
\item $f$ is locally strictly monotone and $\in\continuous(A,B)$ iff
$f\in\continuous(A,B)\cap\continuous(\iota_A|\mathbb{R}|_{>},\iota_B|\mathbb{R}|_{>})$ iff
$f\in\continuous(\iota_A|\mathbb{R}|_{>},\iota_B|\mathbb{R}|_{>})\cap
\continuous(\iota_A|\mathbb{R}|_{<},\iota_B|\mathbb{R}|_{<})$.
\end{enumerate}
\end{thm}

\begin{proof}
By the lemma it is (strictly) monotone on each connected component.
\end{proof}

See also related math.SE questions:
\begin{enumerate}
\item \url{http://math.stackexchange.com/q/1473668/4876}
\item \url{http://math.stackexchange.com/a/1872906/4876}
\item \url{http://math.stackexchange.com/q/1875975/4876}
\end{enumerate}

\subsection{Directed topological spaces}

Directed topological spaces are defined at\\
\url{http://ncatlab.org/nlab/show/directed+topological+space}

\begin{defn}
A \emph{directed topological space} (or \emph{d-space} for short) is a pair $(X,d)$ of a topological space~$X$ and
a set $d\subseteq\continuous([0;1],X)$ (called \emph{directed paths} or \emph{d-paths}) of paths in~$X$ such that
\begin{enumerate}
\item (constant paths) every constant map $[0;1]\to X$ is directed;
\item (reparameterization) $d$ is closed under composition with increasing continuous maps $[0;1]\to [0;1]$;
\item (concatenation) $d$ is closed under path-concatenation: if the d-paths $a$, $b$ are consecutive in $X$ ($a(1)=b(0)$), then their ordinary concatenation $a+b$ is also a d-path
\begin{gather*}
(a+b)(t) = a(2t),\,\text{if}\, 0\le t\le \frac{1}{2}, \\
(a+b)(t) = b(2t-1),\,\text{if}\, \frac{1}{2}\le t\le 1.
\end{gather*}
\end{enumerate}
\end{defn}

I propose a new way to construct a directed topological space. My way is more geometric/topological as it does not involve dealing with particular paths.

\begin{defn}
Let $ T$ be the complete endofuncoid corresponding to a topological space
and $\nu\sqsubseteq T$ be its ``subfuncoid''. The $\mathrm{d}$-space $\operatorname{(dir)}(T,\nu)$ induced by the pair $(T,\nu)$
consists of~$ T$ and paths $f\in\continuous([0;1], T) \cap \continuous(|[0;1]|_{\geq}, \nu)$
such that $f(0)=f(1)$.
\end{defn}

\begin{prop}
It is really a $\mathrm{d}$-space.
\end{prop}

\begin{proof}
Every $\mathrm{d}$-path is continuous.

Constant path are $\mathrm{d}$-paths because $\nu$ is reflexive.

Every reparameterization is a $\mathrm{d}$-path because they are $\continuous(|[0;1]|_{\geq}, \nu)$ and we can apply the theorem about
composition of continuous functions.

Every concatenation is a $\mathrm{d}$-path. Denote
$f_0 = \mylambda{t}{[0;\frac{1}{2}]}{a(2t)}$ and $f_1 = \mylambda{t}{[\frac{1}{2};1]}{b(2t-1)}$.
Obviously $f_0,f_1 \in \continuous([0;1],\mu) \cap \continuous(|[0;1]|_{\geq}, \nu)$.
Then we conclude that $a+b = f_1\sqcup f_1$ is in $f_0,f_1 \in \continuous([0;1],\mu) \cap \continuous(|[0;1]|_{\geq}, \nu)$
using the fact that the operation $\circ$ is distributive over $\sqcup$.
\end{proof}

Below we show that not every $\mathrm{d}$-space is induced by a pair of an endofuncoid and its subfuncoid.
But are $\mathrm{d}$-spaces not represented this way good anything except counterexamples?

Let now we have a $\mathrm{d}$-space $(X,d)$. Define funcoid~$\nu$ corresponding to the $\mathrm{d}$-space by the formula
$\nu = \bigsqcup_{a\in d}(a\circ |\mathbb{R}|_{\geq}\circ a^{-1})$.

\begin{example}
The two directed topological spaces, constructed from a fixed topological space and two different reflexive funcoids,
are the same.
\end{example}

\begin{proof}
Consider the indiscrete topology~$T$ on $\mathbb{R}$ and the funcoids~$1^{\mathsf{FCD}(\mathbb{R},\mathbb{R})}$
and $1^{\mathsf{FCD}(\mathbb{R},\mathbb{R})}\sqcup(\{0\}\times^{\mathsf{FCD}} \Delta_{\geq})$.
The only $\mathrm{d}$-paths in both these settings are constant functions.
\end{proof}

\begin{example}
A $\mathrm{d}$-space is not determined by the induced funcoid.
\end{example}

\begin{proof}
The following a $\mathrm{d}$-space induces the same funcoid as the $\mathrm{d}$-space of all paths on the plane.

Consider a plane $\mathbb{R}^2$ with the usual topology. Let $\mathrm{d}$-paths be paths lying inside a polygonal chain (in the plane).
\end{proof}

\begin{conjecture}
A $\mathrm{d}$-path~$a$ is determined by the funcoids (where $x$ spans $[0;1]$)
\[ (\mylambda{t}{\mathbb{R}}{a(x+t)})|_{\Delta(0)}. \]
\end{conjecture}

\section{A way to construct directed topological spaces}

\fxnote{Should include definition of directed topological space.}

Directed topological spaces are defined at\\
\url{http://ncatlab.org/nlab/show/directed+topological+space}

I propose a new way to construct a directed topological space. My way is more geometric/topological as it does not involve dealing with particular paths.

\begin{conjecture}
Every directed topological space can be constructed in the below described way.
\end{conjecture}

Consider topological space $T$ and its subfuncoid $F$ (that is $F$ is a funcoid which is less that $T$ in the order of funcoids).
Note that in our consideration $F$ is an endofuncoid (its source and destination are the same).

Then a directed path from point $A$ to point $B$ is defined as a continuous function $f$ from $[0;1]$ to $F$ such that $f(0)=A$ and $f(1)=B$.
\fxwarning{Specify whether the interval $[0;1]$ is treated as a proximity, pretopology, or preclosure.}

Because $F$ is less that $T$, we have that every directed path is a path.

\begin{conjecture}
The two directed topological spaces, constructed from a fixed topological space and two different funcoids,
are different.
\end{conjecture}

For a counter-example of (which of the two?) the conjecture consider funcoid $T\sqcap(\mathbb{Q}\times^{\mathsf{FCD}}\mathbb{Q})$
where $T$ is the usual topology on real line.We need to consider stability of existence and uniqueness of a path under transformations of our funcoid and
under transformations of the vector field. Can this be a step to solve Navier-Stokes existence and smoothness problems?

\section{Integral curves}

We will consider paths in a normed vector space~$V$.

\begin{defn}
Let $D$ be a connected subset of~$\mathbb{R}$. A \emph{path} is a function $D\rightarrow V$.
\end{defn}

Let $d$ be a vector field in a normed vector space~$V$.

\begin{defn}
\emph{Integral curve} of a vector field~$d$ is a differentiable function $f:D\rightarrow V$ such that $f'(t)=d(f(t))$ for every $t\in D$.
\end{defn}

\begin{defn}
The definition of \emph{right side integral curve} is the above definition with right derivative of~$f$ instead of derivative~$f'$.
\emph{Left side integral curve} is defined similarly.
\end{defn}

\subsection{Path reparameterization}

$C^1$~is a function which has continuous derivative on every point of the domain.

By $D^1$ I will denote a $C^1$ function whose derivative is either nonzero at every point or is zero everywhere.

\begin{defn}
A \emph{reparameterization} of a $C^1$ path is a bijective $C^1$ function $\phi:D\rightarrow D$ such that
$\phi'(t)>0$. A curve $f_2$ is called a reparametrized curve $f_1$ if there is a reparameterization~$\phi$ such that
$f_2=f_1\circ\phi$.
\end{defn}

It is well known that this defines an equivalence relation of functions.

\begin{prop}
Reparameterization of $D^1$ function is $D^1$.
\end{prop}

\begin{proof}
If the function has zero derivative, it is obvious.

Let $f_1$ has everywhere nonzero derivative. Then $f_2'(t) = f_1'(\phi(t))\phi'(t)$ what is trivially nonzero.
\end{proof}

\begin{defn}
Vectors~$p$ and~$q$ have the \emph{same direction} ($p\upuparrows q$) iff there exists a strictly positive real~$c$ such that $p=cq$.
\end{defn}

\begin{obvious}
Being same direction is an equivalence relation.
\end{obvious}

\begin{obvious}
The only vector with the same direction as the zero vector is zero vector.
\end{obvious}

\begin{thm}
A $D^1$ function~$y$ is some reparameterization of a $D^1$ integral curve~$x$ of a continuous vector field~$d$ iff
$y'(t)\upuparrows d(y(t))$ that is vectors~$y'(t)$ and $d(y(t))$ have the same direction (for every $t$).
\end{thm}

\begin{proof}
If $y$ is a reparameterization of~$x$, then $y(t)=x(\phi(t))$. Thus $y'(t)=x'(\phi(t))\phi'(t)=d(x(\phi(t)))\phi'(t)=d(y(t))\phi'(t)$.
So $y'(t)\upuparrows d(y(t))$ because $\phi'(t)>0$.

Let now $x'(t)\upuparrows d(x(t))$ that is that is there is a strictly positive function $c(t)$ such that
$x'(t) = c(t) d(x(t))$.

If $x'(t)$ is zero everywhere, then $d(x(t))=0$ and thus $x'(t)=d(x(t))$ that is $x$ is an Integral curve.
Note that $y$ is a reparameterization of itself.

We can assume that $x'(t)\ne 0$ everywhere. Then $F(x(t))\ne 0$. We have
that $c(t) = \frac{||x'(t)||}{||d(x(t))||}$ is a continuous function. \fxnote{Does it work for non-normed spaces?}

Let $y(u(t)) = x(t)$, where \[ u(t)=\int_0^t c(s)ds, \]
which is defined and finite because $c$ is continuous and monotone (thus having inverse defined on its image)
because $c$ is positive.

Then
\begin{align*}
y'(u(t)) u'(t) &= x'(t) \\
&= c(t) d(x(t)) \text{, so}\\
y'(u(t)) c(t) &= c(t) d(y(u(t))) \\
y'(u(t))  &=  d(y(u(t)))
\end{align*}
and letting $s=u(t)$ we have $y'(s)=d(y(s))$ for a reparameterization~$y$ of~$x$.
\end{proof}

\subsection{Vector space with additional coordinate}

Consider the normed vector space with additional coordinate~$t$.

Our vector field~$d(t)$ induces vector field~$\hat{d}(t,v)=(1,d(v))$ in this space. Also $\hat{f}(t)=(t,f(t))$.

\begin{prop}
Let $f$ be a $D^1$ function. $f$ is an integral curve of~$d$ iff $\hat{f}$ is a reparametrized integral curve of~$\hat{d}$.
\end{prop}

\begin{proof}
First note that $\hat{f}$ always has a nonzero derivative.
$\hat{f}'(t)\upuparrows \hat{d}(\hat{f}(t)) \Leftrightarrow (1,f'(t))\upuparrows (1,d(f(t))) \Leftrightarrow
f'(t)=d(f(t))$.
\end{proof}

Thus we have reduced (for $D^1$ paths) being an integral curve to being a reparametrized integral curve.
We will also describe being a reparametrized integral curve topologically (through funcoids).

\subsection{Topological description of $C^1$ curves}

Explicitly construct this funcoid as follows:

$R(d,\phi) = \setcond{v\in V}{\widehat{vd}<\phi, v\ne 0}$ for $d\ne 0$ and $R(0,\phi) = \{0\}$,
where $\widehat{ab}$ is the angle between the vectors $a$ and $b$,
for a direction~$d$ and an angle~$\phi$.

\begin{defn}
$W(d) = \bigsqcap^{\mathsf{RLD}}\setcond{R(d,\phi)}{\phi\in\mathbb{R},\phi>0} \sqcap \bigsqcap^{\mathsf{RLD}}_{r>0}B_r(0)$.
\fxnote{This is defined for infinite dimensional case.}
\fxnote{Consider also $\mathsf{FCD}$ instead of $\mathsf{RLD}$.}
\end{defn}

\begin{prop}
For finite dimensional case~$\mathbb{R}^n$ we have
$W(d) = \bigsqcap^{\mathsf{RLD}}\setcond{R(d,\phi)}{\phi\in\mathbb{R},\phi>0} \sqcap \Delta^{(\mathsf{RLD})n}$
where \[ \Delta^{(\mathsf{RLD})n} = \underbrace{\Delta\times^{\mathsf{RLD}}\dots\times^{\mathsf{RLD}}\Delta}_{n\text{ times}}. \]
\end{prop}

\begin{proof}
??
\end{proof}

Finally our funcoids are the complete funcoids~$Q_+$ and~$Q_-$ described by the formulas
\[
\rsupfun{Q_+}@\{p\} = \supfun{p+} W(d(p)) \quad\text{and}\quad \rsupfun{Q_-}@\{p\} = \supfun{p+} W(-d(p)).
\]
Here $\Delta$ is taken from the ``counter-examples'' section.

In other words,
\[
Q_+ = \bigsqcup_{p\in\mathbb{R}} (@\{p\}\times^{\mathsf{FCD}}\supfun{p+}{W(d(p))});
\quad
Q_- = \bigsqcup_{p\in\mathbb{R}} (@\{p\}\times^{\mathsf{FCD}}\supfun{p+}{W(-d(p))}).
\]

That is $\rsupfun{Q_+}@\{p\}$ and $\rsupfun{Q_-}@\{p\}$ are something like infinitely small spherical sectors
(with infinitely small aperture and infinitely small radius).

\fxnote{Describe the co-complete funcoids reverse to these complete funcoids.}

\begin{thm}
A $D^1$ curve~$f$ is an reparametrized integral curve for a direction field~$d$ iff
$f\in\continuous(\iota_D|\mathbb{R}|_{>},Q_+)\cap\continuous(\iota_D|\mathbb{R}|_{<},Q_-)$.
\end{thm}

\begin{proof}
Equivalently transform $f\in\continuous(\iota_D|\mathbb{R}|,Q_+)$; $f\circ \iota_D|\mathbb{R}|\sqsubseteq Q_+\circ f$;
$\rsupfun{f\circ \iota_D|\mathbb{R}|}@\{t\}\sqsubseteq \rsupfun{Q_+\circ f}@\{t\}$;
$\rsupfun{f}\Delta_{>}(t)\sqcap D\sqsubseteq \rsupfun{Q_+}f(t)$;
$\rsupfun{f}\Delta_{>}(t)\sqsubseteq \rsupfun{Q_+}f(t)$;
$\rsupfun{f}\Delta_{>}(t)\sqsubseteq f(t)+W(D(f(t)))$;
$\rsupfun{f}\Delta_{>}(t)-f(t)\sqsubseteq W(D(f(t)))$;
\[ \forall r>0, \phi>0 \exists\delta>0: \rsupfun{f}(]t;t+\delta[)-f(t)\subseteq R(d(f(t)),\phi) \cap B_r(f(t)); \]
\[ \forall r>0, \phi>0 \exists\delta>0 \forall 0<\gamma<\delta: \rsupfun{f}(]t;t+\gamma[)-f(t)\subseteq R(d(f(t)),\phi) \cap B_r(f(t)); \]
\[ \forall r>0, \phi>0 \exists\delta>0 \forall 0<\gamma<\delta: \frac{\rsupfun{f}(]t;t+\gamma[)-f(t)}{\gamma}\subseteq R(d(f(t)),\phi) \cap B_{r/\delta}(f(t)); \]
\[ \forall r>0, \phi>0 \exists\delta>0: \partial_+ f(t)\subseteq R(d(f(t)),\phi) \cap B_{r/\delta}(f(t)); \]
\[ \forall r>0, \phi>0: \partial_+ f(t)\subseteq R(d(f(t)),\phi); \]
\[ \partial_+ f(t) \upuparrows d(f(t)) \]
where $\partial_+$ is the right derivative.

In the same way we derive that $\continuous(|\mathbb{R}|_{<},Q_-)\Leftrightarrow \partial_- f(t) \upuparrows d(f(t))$.

Thus $f'(t) \upuparrows d(f(t))$ iff $f\in\continuous(|\mathbb{R}|_{>},Q_+)\cap\continuous(|\mathbb{R}|_{<},Q_-)$.
\end{proof}

The following idea seems wrong. I grayed it out as a candidate for deletion from the text:

\begin{grayed}

\subsection{$C^n$ curves}

\fxnote{Related questions:
\url{http://math.stackexchange.com/q/1884856/4876}
\url{http://math.stackexchange.com/q/107460/4876}
\url{http://mathoverflow.net/q/88501}}

Define $R^n(d) = \setcond{v\in V}{\widehat{vd}<o(|v|^n), v\ne 0}$ for $d\ne 0$ and $R^n(0) = \{0\}$.

\begin{defn}
$W^n(d) = R^n(d) \sqcap \bigsqcap^{\mathsf{RLD}}_{r>0}B_r(0)$.
\end{defn}

Finally our funcoids are the complete funcoids~$Q_+^n$ and~$Q_-^n$ described by the formulas
\[
\rsupfun{Q_+^n}@\{p\} = \supfun{p+} W^n(d(p)) \quad\text{and}\quad \rsupfun{Q_-^n}@\{p\} = \supfun{p+} W^n(-d(p)).
\]

\begin{lem}
Let for every $x$ in the domain of the path for small $t > 0$ we have $f (x + t) \in R^n (F (f (x)))$ and $f (x - t) \in R^n (- F (f (x)))$.
Then $f$ is $C^n$ smooth.
\end{lem}

\begin{proof}
\fxerror{Not yet proved!}\\
See also \url{http://math.stackexchange.com/q/1884930/4876}.
\end{proof}

\begin{conjecture}
A path~$f$ is $C^n$ smooth iff $f\in\continuous(\iota_D|\mathbb{R}|_{>},Q_+^n)\cap\continuous(\iota_D|\mathbb{R}|_{<},Q_-^n)$.
\end{conjecture}

\begin{proof}
Reverse implication follows from the lemma.

Let now a path~$f$ is $C^n$. Then
\[
f(x+t) = \sum_{i=0}^n f^{(i)}(x)\frac{t^i}{i!} + o(t^i) =
f(x)+f'(x)t + \sum_{i=2}^n f^{(i)}(x)\frac{t^i}{i!} + o(t^i)
\]
\end{proof}

\end{grayed}

\subsection{Plural funcoids}

Take $I_+$ and $Q_+$ as described above in forward direction and $I_-$ and $Q_-$ in backward direction. Then
\[ f\in\continuous(I_+,Q_+)\land f\in\continuous(I_-,Q_-) \Leftrightarrow f\times f\in\continuous(I_+\times^{(A)}I_-,Q_+\times^{(A)}Q_-)? \]

To describe the above we can introduce new term \emph{plural funcoids}. This is simply a map
from an index set to funcoids. Composition is defined component-wise. Order is defined as product order.
Well, do we need this? Isn't it the same as infimum product of funcoids?

\subsection{Multiple allowed directions per point}

\[ \rsupfun{Q}@\{p\} = \bigsqcup_{d\in d(p)} \supfun{p+} W(d). \]

It seems (check!) that solutions not only of differential equations but also of difference equations can be
expressed as paths in funcoids.