\chapter{Applications of algebraic general topology}

\section{``Hybrid'' objects}

Algebraic general topology allows to construct ``hybrid'' objects of ``continuous'' (as topological spaces)
and discrete (as graphs).

Consider for example $D\sqcup T$ where $D$ is a digraph and $T$ is a topological space.

The $n$-th power $(D\sqcup T)^n$ yields an expression with $2^n$ terms.
So treating $D\sqcup T$ as one object (what becomes possible using algebraic general topology)
rather than the join of two objects may have an exponential benefit for simplicity of formulas.

\section{A way to construct directed topological spaces}

\subsection{Some notation}

I use~$\mathcal{E}$ and $\iota$ notations from {\tt volume-2.pdf}. \fxwarning{Reorder document fragments to describe it before use.}

I remind that $f|_X = f\circ \id_X$ for binary relations, funcoids, and reloid.

$f\parallel_X = f\circ(\mathcal{E}^X)^{-1}$.

$f\square X = \id_X\circ f\circ\id_X^{-1}$.

As proved in {\tt volume-2.pdf}, the following are bijections and moreover isomorphisms (for $R$ being either funcoids or reloids or binary relations):
\begin{enumerate}
\item $\setcond{(f|_X;f\parallel_X)}{f\in R}$;
\item $\setcond{(f\square X;\iota_X f)}{f\in R}$.
\end{enumerate}

As easily follows from these isomorphisms and theorem~\bookref{rect-cont}:

\begin{prop}
For funcoids, reloids, and binary relations:
\begin{enumerate}
\item $f\in\continuous(\mu;\nu)\Rightarrow f\parallel_{A}\in\continuous(\iota_A\mu;\nu)$;
\item $f\in\continuous'(\mu;\nu)\Rightarrow f\parallel_{A}\in\continuous'(\iota_A\mu;\nu)$;
\item $f\in\continuous''(\mu;\nu)\Rightarrow f\parallel_{A}\in\continuous''(\iota_A\mu;\nu)$. 
\end{enumerate}
\end{prop}

\subsection{Directed line and directed intervals}

Let $\mathfrak{A}$ be a poset. We will denote $\overline{\mathfrak{A}}=\mathfrak{A}\cup\{-\infty,+\infty\}$ the poset
with two added elements $-\infty$ and $+\infty$, such that $+\infty$ is strictly greater than every element of~$\mathfrak{A}$
and $-\infty$ is strictly less.

\begin{defn}
For an element~$a$ of a poset~$\mathfrak{A}$
\begin{enumerate}
\item $J_{\geq}(a) = \setcond{x\in\mathfrak{A}}{x\geq a}$;
\item $J_{>}(a) = \setcond{x\in\mathfrak{A}}{x>a}$;
\item $J_{\leq}(a) = \setcond{x\in\mathfrak{A}}{x\leq a}$;
\item $J_{<}(a) = \setcond{x\in\mathfrak{A}}{x<a}$.
\end{enumerate}
\end{defn}

\begin{defn}
Let $a$ be an element of a poset~$\mathfrak{A}$.
\begin{enumerate}
\item $\Delta(a) = \bigsqcap^{\mathscr{F}} \setcond{]x;y[}{x,y\in\overline{\mathfrak{A}}, x<a\land y>a}$;
\item $\Delta_{\geq}(a) = \bigsqcap^{\mathscr{F}} \setcond{[a;y[}{y\in\overline{\mathfrak{A}}, y>a}$;
\item $\Delta_{>}(a) = \bigsqcap^{\mathscr{F}} \setcond{]a;y[}{y\in\overline{\mathfrak{A}}, x<a\land y>a}$;
\item $\Delta_{\leq}(a) = \bigsqcap^{\mathscr{F}} \setcond{]x;a]}{x\in\overline{\mathfrak{A}}, x<a}$.
\item $\Delta_{<}(a) = \bigsqcap^{\mathscr{F}} \setcond{]x;a[}{x\in\overline{\mathfrak{A}}, x<a}$;
\end{enumerate}
\end{defn}

\begin{obvious}
~
\begin{enumerate}
\item $\Delta_{\geq}(a) = \Delta(a)\sqcap^{\mathscr{F}} @J_{\geq a}$;
\item $\Delta_{>}(a) = \Delta(a)\sqcap^{\mathscr{F}} @J_{>a}$;
\item $\Delta_{\leq}(a) = \Delta(a)\sqcap^{\mathscr{F}} @J_{\leq a}$;
\item $\Delta_{<}(a) = \Delta(a)\sqcap^{\mathscr{F}} @J_{<a}$.
\end{enumerate}
\end{obvious}

\begin{defn}
~
Given a partial order~$\mathfrak{A}$ and~$x\in\mathfrak{A}$, the following defines complete funcoids:
\begin{enumerate}
\item $\rsupfun{|\mathfrak{A}|}\{x\} = \Delta(x)$;
\item $\rsupfun{|\mathfrak{A}|_{\geq}}\{x\} = \Delta_{\geq}(x)$;
\item $\rsupfun{|\mathfrak{A}|_{>}}\{x\} = \Delta_{>}(x)$;
\item $\rsupfun{|\mathfrak{A}|_{\leq}}\{x\} = \Delta_{\leq}(x)$;
\item $\rsupfun{|\mathfrak{A}|_{<}}\{x\} = \Delta_{<}(x)$.
\end{enumerate}
\end{defn}

\begin{prop}
The complete funcoid corresponding to the order topology\footnote{See Wikipedia for a definition of ``Order topology''.}
is equal to $|\mathfrak{A}|$.
\end{prop}

\begin{proof}
Because every open set is a finite union of open intervals, the complete funcoid~$f$ corresponding to the order topology
is described by the formula: $\rsupfun{f}\{x\} = \bigsqcap^{\mathscr{F}}\setcond{]a;b[}{a,b\in\overline{\mathfrak{A}}, a<x\land b>x} =
\Delta(x) = \rsupfun{|\mathfrak{A}|}\{x\}$. Thus $f=|\mathfrak{A}|$.
\end{proof}

\begin{xca}
Show that $|\mathfrak{A}|_{\geq}$ (in general) is not the same as ``right order topology''\footnote{See Wikipedia}.
\end{xca}

\begin{prop}
~
\begin{enumerate}
\item $\rsupfun{|\mathfrak{A}|_{\geq}^{-1}}@X = @\setcond{a\in\mathfrak{A}}{\forall y\in\overline{\mathfrak{A}}:(y>a \Rightarrow X\cap[a;y[\ne\emptyset)}$;
\item $\rsupfun{|\mathfrak{A}|_{>}^{-1}}@X = @\setcond{a\in\mathfrak{A}}{\forall y\in\overline{\mathfrak{A}}:(y>a \Rightarrow X\cap]a;y[\ne\emptyset)}$;
\item $\rsupfun{|\mathfrak{A}|_{\leq}^{-1}}@X = @\setcond{a\in\mathfrak{A}}{\forall \in\overline{\mathfrak{A}}:(x<a \Rightarrow X\cap]x;a]\ne\emptyset)}$;
\item $\rsupfun{|\mathfrak{A}|_{<}^{-1}}@X = @\setcond{a\in\mathfrak{A}}{\forall \in\overline{\mathfrak{A}}:(x<a \Rightarrow X\cap]x;a[\ne\emptyset)}$.
\end{enumerate}
\end{prop}

\begin{proof}
$a\in\rsupfun{|\mathfrak{A}|_{\geq}^{-1}}@X \Leftrightarrow
@\{a\} \nasymp \rsupfun{|\mathfrak{A}|_{\geq}^{-1}}@X \Leftrightarrow
\rsupfun{|\mathfrak{A}|_{\geq}}@\{a\} \nasymp @X \Leftrightarrow
\Delta_{\geq}(a) \nasymp @X \Leftrightarrow
\forall y\in\overline{\mathfrak{A}}:(y>a \Rightarrow X\cap[a;y[\ne\emptyset)$.

$a\in\rsupfun{|\mathfrak{A}|_{>}^{-1}}@X \Leftrightarrow
@\{a\} \nasymp \rsupfun{|\mathfrak{A}|_{>}^{-1}}@X \Leftrightarrow
\rsupfun{|\mathfrak{A}|_{>}}@\{a\} \nasymp @X \Leftrightarrow
\Delta_{>}(a) \nasymp @X \Leftrightarrow
\forall y\in\overline{\mathfrak{A}}:(y>a \Rightarrow X\cap]a;y[\ne\emptyset)$.

The rest follows from duality.
\end{proof}

\begin{conjecture}
~
\begin{enumerate}
\item $|\mathbb{R}|_{\geq} = |\mathbb{R}| \sqcap \geq$;
\item $|\mathbb{R}|_{>} = |\mathbb{R}| \sqcap >$;
\item $|\mathbb{R}|_{\leq} = |\mathbb{R}| \sqcap \leq$;
\item $|\mathbb{R}|_{<} = |\mathbb{R}| \sqcap <$.
\end{enumerate}
\end{conjecture}

\begin{rem}
On trivial ultrafilters these obviously agree:
\begin{enumerate}
\item $\rsupfun{|\mathbb{R}|_{\geq}}\{x\} = \rsupfun{|\mathbb{R}| \sqcap \geq}\{x\}$;
\item $\rsupfun{|\mathbb{R}|_{>}}\{x\} = \rsupfun{|\mathbb{R}| \sqcap >}\{x\}$;
\item $\rsupfun{|\mathbb{R}|_{\leq}}\{x\} = \rsupfun{|\mathbb{R}| \sqcap \leq}\{x\}$;
\item $\rsupfun{|\mathbb{R}|_{<}}\{x\} = \rsupfun{|\mathbb{R}| \sqcap <}\{x\}$.
\end{enumerate}
\end{rem}

I will say that a property holds on a filter~$\mathcal{A}$ iff there is $A\in\up\mathcal{A}$ on which the property holds.

\fxnote{$f\in\continuous(A;B)\land f\in\continuous(\iota_A|\mathbb{R}|_{\geq};\iota_B|\mathbb{R}|_{\geq}) \Leftrightarrow
(f;f)\in\continuous((A;\iota_A|\mathbb{R}|_{\geq});(B;\iota_B|\mathbb{R}|_{\geq}))$}

\begin{lem}
Let function~$f\in\continuous(A;B)$ where $A,B\in\subsets\mathbb{R}$ and $A$ is connected.
\begin{enumerate}
\item $f$ is monotone iff $f\in\continuous(\iota_A|\mathbb{R}|_{\geq};\iota_B|\mathbb{R}|_{\geq})$ iff
$f\in\continuous(\iota_A|\mathbb{R}|_{>};\iota_B|\mathbb{R}|_{\geq})$.
\item $f$ is strictly monotone iff $f\in\continuous(\iota_A|\mathbb{R}|_{>};\iota_B|\mathbb{R}|_{>})$.
\end{enumerate}
\fxnote{Generalize for arbitrary posets.}
\fxnote{Generalize for $f$ being a funcoid.}
\end{lem}

\fxnote{Isn't it equivalent to $f\in\continuous(\iota_A|\mathbb{R}|_{\geq};\iota_B|\mathbb{R}|_{\geq})\cap
\continuous(\iota_A|\mathbb{R}|_{\leq};\iota_B|\mathbb{R}|_{\leq})$ and
$f\in\continuous(\iota_A|\mathbb{R}|_{>};\iota_B|\mathbb{R}|_{>})\cap
\continuous(\iota_A|\mathbb{R}|_{<};\iota_B|\mathbb{R}|_{<})$?}

\begin{proof}
Because $f$ is continuous, we have $\rsupfun{f\circ\iota_A|\mathbb{R}|}\{x\} \sqsubseteq \rsupfun{\iota_B|\mathbb{R}|\circ f}\{x\}$
that is $\rsupfun{f} \Delta(x) \sqsubseteq \Delta(f(x))$ for every~$x$.

If $f$ is monotone, we have $\rsupfun{f} \Delta_{\geq}(x) \sqsubseteq [f(x);\infty[$.
Thus $\rsupfun{f} \Delta_{\geq}(x) \sqsubseteq \Delta_{\geq}(f(x))$, that is
$\rsupfun{f\circ \iota_A|\mathbb{R}|_{\geq}}\{x\} \sqsubseteq \rsupfun{\iota_B|\mathbb{R}|_{\geq}\circ f}\{x\}$, thus
$f\in\continuous(\iota_A|\mathbb{R}|_{\geq};\iota_B|\mathbb{R}|_{\geq})$.

If $f$ is strictly monotone, we have $\rsupfun{f} \Delta_{>}(x) \sqsubseteq ]f(x);\infty[$.
Thus $\rsupfun{f} \Delta_{>}(x) \sqsubseteq \Delta_{>}(f(x))$, that is
$\rsupfun{f\circ \iota_A|\mathbb{R}|_{>}}\{x\} \sqsubseteq \rsupfun{\iota_B|\mathbb{R}|_{>}\circ f}\{x\}$, thus
$f\in\continuous(\iota_A|\mathbb{R}|_{>};\iota_B|\mathbb{R}|_{>})$.

Let now $f\in\continuous(\iota_A|\mathbb{R}|_{\geq};\iota_B|\mathbb{R}|_{\geq})$.

Take any~$a\in A$ and let $c=\setcond{b\in B}{b\geq a, \forall x\in[a;b[: f(x)\geq f(a)}$.
It's enough to prove that $c$ is the right endpoint (finite or infinite) of~$A$.

Indeed by continuity $f(a)\leq f(c)$ and if $c$ is not already the right endpoint of~$A$, then
there is $b'>c$ such that $\forall x\in[c;b'[: f(x)\geq f(c)$.
So we have $\forall x\in[a;b'[: f(x)\geq f(c)$ what contradicts to the above.

So $f$ is monotone on the entire~$A$.

$f\in\continuous(\iota_A|\mathbb{R}|_{\geq};\iota_B|\mathbb{R}|_{\geq}) \Rightarrow f\in\continuous(\iota_A|\mathbb{R}|_{>};\iota_B|\mathbb{R}|_{\geq})$ is obvious. Reversely
$f\in\continuous(\iota_A|\mathbb{R}|_{>};\iota_B|\mathbb{R}|_{\geq}) \Rightarrow
f\circ \iota_A|\mathbb{R}|_{>} \sqsubseteq \iota_B|\mathbb{R}|_{\geq}\circ f \Leftrightarrow
\forall x\in\mathbb{R}: \supfun{f}\rsupfun{\iota_A|\mathbb{R}|_{>}}\{x\} \sqsubseteq \rsupfun{\iota_B|\mathbb{R}|_{\geq}}\rsupfun{f}\{x\} \Leftrightarrow
\forall x\in\mathbb{R}: \supfun{f}\Delta_{>}(x) \sqsubseteq \Delta_{\geq}f(x) \Leftrightarrow
\forall x\in\mathbb{R}: \supfun{f}\Delta_{>}(x) \sqcup \{f(x)\} \sqsubseteq \Delta_{\geq}f(x) \Leftrightarrow
\forall x\in\mathbb{R}: \supfun{f}\Delta_{>}(x) \sqcup \{x\} \sqsubseteq \Delta_{\geq}f(x) \Leftrightarrow
\forall x\in\mathbb{R}: \supfun{f}\Delta_{\geq}(x) \sqsubseteq \Delta_{\geq}f(x) \Leftrightarrow
\forall x\in\mathbb{R}: \supfun{f}\rsupfun{\iota_A|\mathbb{R}|_{\geq}}\{x\} \sqsubseteq \rsupfun{\iota_B|\mathbb{R}|_{\geq}}\rsupfun{f}\{x\} \Leftrightarrow
\forall x\in\mathbb{R}: f\circ \iota_A|\mathbb{R}|_{\geq} \sqsubseteq \iota_B|\mathbb{R}|_{\geq}\circ f \Leftrightarrow
f\in\continuous(\iota_A|\mathbb{R}|_{\geq};\iota_B|\mathbb{R}|_{\geq})$.

Let $f\in\continuous(\iota_A|\mathbb{R}|_{>};\iota_B|\mathbb{R}|_{>})$. Then $f\in\continuous(\iota_A|\mathbb{R}|_{>};\iota_B|\mathbb{R}|_{\geq})$ and thus it is monotone.
We need to prove that $f$ is strictly monotone.
Suppose the contrary. Then there is a nonempty interval $[p;q]\subseteq A$ such that $f$ is constant on this interval.
But this is impossible because $f\in\continuous(\iota_A|\mathbb{R}|_{>};\iota_B|\mathbb{R}|_{>})$.
\end{proof}

\begin{thm}
Let function~$f\in\continuous(A;B)$ where $A,B\in\subsets\mathbb{R}$.
\begin{enumerate}
\item $f$ is locally monotone iff $f\in\continuous(\iota_A|\mathbb{R}|_{\geq};\iota_B|\mathbb{R}|_{\geq})$.
\item $f$ is locally strictly monotone iff $f\in\continuous(\iota_A|\mathbb{R}|_{>};\iota_B|\mathbb{R}|_{>})$.
\end{enumerate}
\end{thm}

\begin{proof}
By the lemma it is (strictly) monotone on each connected component.
\end{proof}

See also related math.SE questions:
\begin{enumerate}
\item \url{http://math.stackexchange.com/q/1473668/4876}
\item \url{http://math.stackexchange.com/a/1872906/4876}
\item \url{http://math.stackexchange.com/q/1875975/4876}
\end{enumerate}

\subsection{Directed topological spaces}

Directed topological spaces are defined at\\
\url{http://ncatlab.org/nlab/show/directed+topological+space}

\begin{defn}
A \emph{directed topological space} (or \emph{d-space} for short) is a pair $(X;d)$ of a topological space~$X$ and
a set $d\subseteq\continuous([0;1];X)$ (called \emph{directed paths} or \emph{d-paths}) of paths in~$X$ such that
\begin{enumerate}
\item (constant paths) every constant map $[0;1]\to X$ is directed;
\item (reparameterization) $d$ is closed under composition with increasing continuous maps $[0;1]\to [0;1]$;
\item (concatenation) $d$ is closed under path-concatenation: if the d-paths $a$, $b$ are consecutive in $X$ ($a(1)=b(0)$), then their ordinary concatenation $a+b$ is also a d-path
\begin{gather*}
(a+b)(t) = a(2t),\,\text{if}\, 0\le t\le \frac{1}{2}, \\
(a+b)(t) = b(2t-1),\,\text{if}\, \frac{1}{2}\le t\le 1.
\end{gather*}
\end{enumerate}
\end{defn}

I propose a new way to construct a directed topological space. My way is more geometric/topological as it does not involve dealing with particular paths.

\begin{defn}
Let $ T$ be the complete endofuncoid corresponding to a topological space
and $\nu\sqsubseteq T$ be its ``subfuncoid''. The $\mathrm{d}$-space $\operatorname{(dir)}(T;\nu)$ induced by the pair $(T;\nu)$
consists of~$ T$ and paths $f\in\continuous([0;1]; T) \cap \continuous(|[0;1]|_{\geq}; \nu)$
such that $f(0)=f(1)$.
\end{defn}

\begin{prop}
It is really a $\mathrm{d}$-space.
\end{prop}

\begin{proof}
Every $\mathrm{d}$-path is continuous.

Constant path are $\mathrm{d}$-paths because $\nu$ is reflexive.

Every reparameterization is a $\mathrm{d}$-path because they are $\continuous(|[0;1]|_{\geq}; \nu)$ and we can apply the theorem about
composition of continuous functions.

Every concatenation is a $\mathrm{d}$-path. Denote
$f_0 = \mylambda{t}{[0;\frac{1}{2}]}{a(2t)}$ and $f_1 = \mylambda{t}{[\frac{1}{2};1]}{b(2t-1)}$.
Obviously $f_0,f_1 \in \continuous([0;1];\mu) \cap \continuous(|[0;1]|_{\geq}; \nu)$.
Then we conclude that $a+b = f_1\sqcup f_1$ is in $f_0,f_1 \in \continuous([0;1];\mu) \cap \continuous(|[0;1]|_{\geq}; \nu)$
using the fact that the operation $\circ$ is distributive over $\sqcup$.
\end{proof}

Below we show that not every $\mathrm{d}$-space is induced by a pair of an endofuncoid and its subfuncoid.
But are $\mathrm{d}$-spaces not represented this way good anything except counterexamples?

Let now we have a $\mathrm{d}$-space $(X;d)$. Define funcoid~$\nu$ corresponding to the $\mathrm{d}$-space by the formula
$\nu = \bigsqcup_{a\in d}(a\circ |\mathbb{R}|_{\geq}\circ a^{-1})$.

\begin{example}
The two directed topological spaces, constructed from a fixed topological space and two different reflexive funcoids,
are the same.
\end{example}

\begin{proof}
Consider the indiscrete topology~$T$ on $\mathbb{R}$ and the funcoids~$1^{\mathsf{FCD}(\mathbb{R};\mathbb{R})}$
and $1^{\mathsf{FCD}(\mathbb{R};\mathbb{R})}\sqcup(\{0\}\times^{\mathsf{FCD}} \Delta_{\geq})$.
The only $\mathrm{d}$-paths in both these settings are constant functions.
\end{proof}

\begin{example}
A $\mathrm{d}$-space is not determined by the induced funcoid.
\end{example}

\begin{proof}
The following a $\mathrm{d}$-space induces the same funcoid as the $\mathrm{d}$-space of all paths on the plane.

Consider a plane $\mathbb{R}^2$ with the usual topology. Let $\mathrm{d}$-paths be paths lying inside a polygonal chain (in the plane).
\end{proof}

\begin{conjecture}
A $\mathrm{d}$-path~$a$ is determined by the funcoids (where $x$ spans $[0;1]$)
\[ (\mylambda{t}{\mathbb{R}}{a(x+t)})|_{\Delta(0)}. \]
\end{conjecture}

\section{A way to construct directed topological spaces}

\fxnote{Should include definition of directed topological space.}

Directed topological spaces are defined at\\
\url{http://ncatlab.org/nlab/show/directed+topological+space}

I propose a new way to construct a directed topological space. My way is more geometric/topological as it does not involve dealing with particular paths.

\begin{conjecture}
Every directed topological space can be constructed in the below described way.
\end{conjecture}

Consider topological space $T$ and its subfuncoid $F$ (that is $F$ is a funcoid which is less that $T$ in the order of funcoids).
Note that in our consideration $F$ is an endofuncoid (its source and destination are the same).

Then a directed path from point $A$ to point $B$ is defined as a continuous function $f$ from $[0;1]$ to $F$ such that $f(0)=A$ and $f(1)=B$.
\fxwarning{Specify whether the interval $[0;1]$ is treated as a proximity, pretopology, or preclosure.}

Because $F$ is less that $T$, we have that every directed path is a path.

\begin{conjecture}
The two directed topological spaces, constructed from a fixed topological space and two different funcoids,
are different.
\end{conjecture}

For a counter-example of (which of the two?) the conjecture consider funcoid $T\sqcap(\mathbb{Q}\times^{\mathsf{FCD}}\mathbb{Q})$
where $T$ is the usual topology on real line.We need to consider stability of existence and uniqueness of a path under transformations of our funcoid and
under transformations of the vector field. Can this be a step to solve Navier-Stokes existence and smoothness problems?

\section{Integral curves}

\fxwarning{This section is a very rough draft.}

See also \url{https://en.wikipedia.org/wiki/Slope_field}

Let $d$ be a vector field in~$\mathbb{R}^n$ (for a finite~$n$).

\fxnote{$\mathbb{R}$ is made into a ``locally ordered'' space by a direction field. Then use properties of ordered spaces to define the funcoid.}

\begin{defn}
\emph{Integral curve} is a function $f:[0;1]\rightarrow\mathbb{R}^n$ such that $f'(t)=d(f(t))$ for every $t\in[0;1]$.
\end{defn}

\begin{defn}
The definition of \emph{right side integral curve} is the above definition with right derivative of~$f$ instead of derivative~$f'$.
\emph{Left side integral curve} is defined similarly.
\end{defn}

$R(d;\phi) = \setcond{v\in\mathbb{R}^n}{\widehat{vd}<\phi, v\ne 0}$ for $d\ne 0$ and $R(d;\phi) = 0$ if $d=0$,
where $\widehat{ab}$ is the angle between the vectors $a$ and $b$,
for a direction~$d$ and an angle~$\phi$.

$W(d) = \bigsqcap^{\mathsf{RLD}}\setcond{R(d;\phi)}{\phi\in\mathbb{R},\phi>0} \sqcap \Delta^{(\mathsf{RLD})n}$
where \[ \Delta^{(\mathsf{RLD})n} = \underbrace{\Delta\times^{\mathsf{RLD}}\dots\times^{\mathsf{RLD}}\Delta}_{n\text{ times}}. \]
\fxnote{Consider also $\mathsf{FCD}$ instead of $\mathsf{RLD}$. This would use pointfree funcoids.}

Finally our funcoid is the complete funcoid~$Q$ described by the formula
\[ \rsupfun{Q}@\{p\} = \supfun{p+} W(D(p)). \]
Here $\Delta$ is taken from the ``counter-examples'' section;
$D(p)$ is the given set of directions for a point~$p$.

In other words, $Q = \bigsqcup_{p\in\mathbb{R}} (@\{p\}\times^{\mathsf{FCD}}\supfun{p+}{W(D(p))})$.

\fxnote{Describe the co-complete funcoid reverse to this complete funcoid.}

\begin{thm}
$f$ is an image of integral curve for a direction field~$d$ iff $f\in\continuous(|\mathbb{R}|_{>};Q)$.
\end{thm}

\fxnote{The condition that left and right derivatives are the same is equal to continuity in $\mathbb{R}^{n+1}$.}

It seems that this problem can be reformulated in the language of funcoids: From the sets of directions we easily construct certain funcoid.
Existence of such smooth curve is equivalent to existence of a continuous function from $[0;1]$ to our funcoid (it seems that we also need to add the
condition that it smooth, I have not yet defined it quite formally).

Explicitly construct this funcoid as (now, for simplicity only for $\mathbb{R}^n$, there is only one direction $D(p)$ per point~$p$):

\begin{thm}
A $C^1$ integral curve (for our vector field~$D$) (having a start and end points) in $\mathbb{R}^n$ is exactly the same as
a function in $\continuous(\iota_{[0;1]}|\mathbb{R}|;\mathbb{R}^n)\cap\continuous(\iota_{[0;1]}|\mathbb{R}|_{>};Q)$.
\end{thm}

\begin{proof}
Equivalently transform $f\in\continuous(\iota_{[0;1]}|\mathbb{R}|;Q)$; $f\circ \iota_{[0;1]}|\mathbb{R}|\sqsubseteq Q\circ f$;
$\rsupfun{f\circ \iota_{[0;1]}|\mathbb{R}|}@\{t\}\sqsubseteq \rsupfun{Q\circ f}@\{t\}$;
$\rsupfun{f}\Delta_{>}(t)\sqcap[0;1])\sqsubseteq \rsupfun{Q}f(t)$;
$\rsupfun{f}\Delta_{>}(t)\sqsubseteq \rsupfun{Q}f(t)$;
$\rsupfun{f}\Delta_{>}(t)\sqsubseteq f(t)+W(D(f(t)))$;
$\rsupfun{f}\Delta_{>}(t)-f(t)\sqsubseteq W(D(f(t)))$;
\[ \forall r>0, \phi>0 \exists\delta>0: \rsupfun{f}(]t;t+\delta[)-f(t)\subseteq R(d;\phi) \cap B_r(f(t)) \]
\[ \forall r>0, \phi>0 \exists\delta>0: \frac{\rsupfun{f}(]t;t+\delta[)-f(t)}{\delta}\subseteq R(d;\phi) \cap B_{r/\delta}(f(t)) \]
??
\end{proof}

Alternatively consider $G(p) = \Delta_+\times^{\mathsf{RLD}}Q_+(p) \sqcup \Delta_-\times^{\mathsf{RLD}}Q_-(p)$.
\[ (\mylambda{t}{\mathbb{R}}{f(x+t)-f(x)})|_{\Delta} \sqsubseteq G(x). \]

\begin{conjecture}
If the curve goes only through points with exactly one direction, then the curve is smooth (if it is continuous regarding the above funcoid).
\end{conjecture}

\fxnote{This probably can be generalized for $C^n$ and $C^\infty$ curves. Consider also fractional~$n$. What about curves limited by an exponent?}

$f'(t)=d(f(t))$ is equivalent to $w(x,t)=(v(x),t)$ being a continuous function on~$\mathbb{R}^{n+1}$ with direction specified by
$F(v(t), 1)$.

\subsection{Plural funcoids}

Take $I_+$ and $Q_+$ as described above in forward direction and $I_-$ and $Q_-$ in backward direction. Then
\[ f\in\continuous(I_+;Q_+)\land f\in\continuous(I_-;Q_-) \Leftrightarrow f\times f\in\continuous(I_+\times^{(A)}I_-;Q_+\times^{(A)}Q_-)? \]

To describe the above we can introduce new term \emph{plural funcoids}. This is simply a map
from an index set to funcoids. Composition is defined component-wise. Order is defined as product order.
Well, do we need this? Isn't it the same as infimum product of funcoids?

\subsection{Multiple allowed directions per point}

\[ \rsupfun{Q}@\{p\} = \bigsqcup_{d\in D(p)} \supfun{p+} W(d). \]

One advantage to use funcoids in this case, that when describing the curves we don't need to require that the curve is smooth. I think, it's a big advantage.

It seems (check!) that solutions not only of differential equations but also of difference equations can be
expressed as paths in funcoids.