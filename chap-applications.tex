\chapter{Applications of algebraic general topology}

\fxnote{Should also Consider ``one-side topologies'' like $T\sqcap\setcond{(x;y)}{x,y\in\mathbb{R},y\ge x}$ where
$T$ is the usual topology on~$\mathbb{R}$. Wrap a $\pi$-length segment of this line into a circle
and get one-side circle: We can traverse continuously counter-clockwise but not clockwise.}

\section{``Hybrid'' objects}

Algebraic general topology allows to construct ``hybrid'' objects of ``continuous'' (as topological spaces)
and discrete (as graphs).

Consider for example $D\sqcup T$ where $D$ is a digraph and $T$ is a topological space.

The $n$-th power $(D\sqcup T)^n$ yields an expression with $2^n$ terms.
So treating $D\sqcup T$ as one object (what becomes possible using algebraic general topology)
rather than the join of two objects may have an exponential benefit for simplicity of formulas.

\section{A way to construct directed topological spaces}

\subsection{Directed line and directed intervals}

\begin{defn}
Let $a$ be an element of a poset~$\mathfrak{A}$.
\begin{enumerate}
\item $\Delta(a) = \bigsqcap^{\mathscr{F}} \setcond{]x;y[}{x,y\in\mathfrak{A}, x<a\land y>a}$;
\item $\Delta_{>}(a) = \bigsqcap^{\mathscr{F}} \setcond{]a;y[}{y\in\mathfrak{A}, x<a\land y>a}$;
\item $\Delta_{\geq}(a) = \bigsqcap^{\mathscr{F}} \setcond{[a;y[}{y\in\mathfrak{A}, y>a}$;
\item $\Delta_{<}(a) = \bigsqcap^{\mathscr{F}} \setcond{]x;a[}{x\in\mathfrak{A}, x<a}$;
\item $\Delta_{\leq}(a) = \bigsqcap^{\mathscr{F}} \setcond{]x;a]}{x\in\mathfrak{A}, x<a}$.
\end{enumerate}
\end{defn}

\begin{obvious}
~
\begin{enumerate}
\item $\Delta_{>}(a) = \Delta(a)\sqcap^{\mathscr{F}} \setcond{y\in\mathfrak{A}}{y>a}$;
\item $\Delta_{\geq}(a) = \Delta(a)\sqcap^{\mathscr{F}} \setcond{y\in\mathfrak{A}}{y\geq a}$;
\item $\Delta_{<}(a) = \Delta(a)\sqcap^{\mathscr{F}} \setcond{x\in\mathfrak{A}}{x<a}$;
\item $\Delta_{\leq}(a) = \Delta(a)\sqcap^{\mathscr{F}} \setcond{x\in\mathfrak{A}}{x\leq a}$.
\end{enumerate}
\end{obvious}

\begin{defn}~
\begin{enumerate}
\item \emph{Right directed (real) line} is the complete funcoid $\overrightarrow{\mathbb{R}}$ defined by the formula
$\rsupfun{\overrightarrow{\mathbb{R}}}@\{a\} = \Delta_{+} + a$ for every $a\in\mathbb{R}$;
\item \emph{Left directed (real) line} is the complete funcoid $\overleftarrow{\mathbb{R}}$ defined by the formula
$\rsupfun{\overleftarrow{\mathbb{R}}}@\{a\} = \Delta_{-} + a$ for every $a\in\mathbb{R}$;
\end{enumerate}
\end{defn}

\begin{xca}~
\begin{enumerate}
\item $\rsupfun{\overrightarrow{\mathbb{R}}^{-1}}@X = @\setcond{a\in\mathbb{R}}{\forall\epsilon>0:X\cap[a;a+\epsilon[\ne\emptyset}$;
\item $\rsupfun{\overleftarrow{\mathbb{R}}^{-1}}@X = @\setcond{a\in\mathbb{R}}{\forall\epsilon>0:X\cap]a-\epsilon;a]\ne\emptyset}$.
\end{enumerate}
\end{xca}

\begin{defn}
For every set $A\in\subsets\mathbb{R}$
\begin{enumerate}
\item $\overrightarrow{A} = \overrightarrow{\mathbb{R}} \sqcap(A\times A)$;
\item $\overleftarrow{A} = \overleftarrow{\mathbb{R}} \sqcap(A\times A)$.
\end{enumerate}
If $A$ is an interval, then $\overrightarrow{A}$ is called \emph{right directed interval} and $\overleftarrow{A}$ is called \emph{left directed interval}.
\end{defn}

We can generalize it for arbitrary ordered endofuncoids or endoreloids:

\begin{defn}
An \emph{ordered endofuncoid} is a pair of an endofuncoid~$\mu$ and a poset on~$\Ob\mu$.
\emph{Ordered endoreloids} are defined similarly.
\fxwarning{Should we also require that order and the structure agree?}
\end{defn}

\[
\overrightarrow{\mu} = \mu \sqcap @\setcond{(x;y)}{x,y\in\Ob\mu, y\geq x}
\quad\text{and}\quad
\overleftarrow{\mu} = \mu \sqcap @\setcond{(x;y)}{x,y\in\Ob\mu, y\leq x}.
\]

\fxwarning{Prove it is the same as above defined for real numbers.}

I will say that a property holds on a filter~$\mathcal{A}$ iff there is $A\in\up\mathcal{A}$ on which the property holds.

\begin{prop}
$\overrightarrow{\mu} \sqcup \overleftarrow{\mu} = \mu$ for both funcoids and reloids.
\end{prop}

\begin{proof}
Both funcoids and reloids are distributive lattices.
\end{proof}

\begin{prop}
$f\in\continuous(\overrightarrow{A};\overrightarrow{B})$ iff
$f$ is monotone on the filter~$\Delta_+$ and right-continuous.
\end{prop}

\begin{proof}
??
\end{proof}

\begin{conjecture}
A function~$f:A\rightarrow B$ where $A,B\in\subsets\mathbb{R}$ is continuous (regarding the usual topologies on~$A$ and~$B$) and monotone iff
both $f\in\continuous(\overrightarrow{A};\overrightarrow{B})$ and $f\in\continuous(A;B)$, provided that $A$ is a connected set (=interval).
\url{http://math.stackexchange.com/a/1872906/4876}
\fxnote{It is also locally monotone even if $A$ is not a connected set.}
\end{conjecture}

\begin{proof} (attempted)
\url{http://math.stackexchange.com/questions/1473668/locally-monotone-function-is-monotone}

The hard part is to prove that it is monotone. It's enough to prove that it is locally monotone.
$f\in\continuous(\overrightarrow{A};\overrightarrow{B}) \Leftrightarrow
f\circ\overrightarrow{A} \sqsubseteq \overrightarrow{B}\circ f \Leftrightarrow
\forall x\in\Src f:\rsupfun{f\circ\overrightarrow{A}}\{x\} \sqsubseteq \rsupfun{\overrightarrow{B}\circ f}\{x\}
$.

Without lost of generality, prove just that it is locally monotone at~$0$.


Suppose for the contrary
that there is an $\epsilon>0$ such that $f(x) > f(0)$ for $x\in]-\epsilon;0[$ (for $x>0$ it follows from ??).

$\rsupfun{f\circ\overrightarrow{[0;1]}\circ\overleftarrow{[0;1]}^{-1}} ]-\epsilon;0[ =
\rsupfun{f\circ\overrightarrow{[0;1]}} ]-\epsilon;0] \ni f(x), f(0)$

$\rsupfun{B\circ f} ]-\epsilon;0[ \not\ni ?? \not\ni f(0)$;

Consider for counter-examples $f(t) = t\sin(1/t)$ or
\[ f(t) = \begin{cases}t\sin(1/t)&\text{ if }x<0\\0&\text{ if }x\ge 0\end{cases}. \]

Old counter-example: $f(t) = \begin{cases}-1/x&\text{ if }x<0\\0&\text{ if }x\ge 0\end{cases}$.
??
\end{proof}

\subsection{Directed topological spaces}

Directed topological spaces are defined at\\
\url{http://ncatlab.org/nlab/show/directed+topological+space}

\begin{defn}
A \emph{directed topological space} (or \emph{d-space} for short) is a pair $(X;d)$ of a topological space~$X$ and
a set $d\subseteq\continuous([0;1];X)$ (called \emph{directed paths} or \emph{d-paths}) of paths in~$X$ such that
\begin{enumerate}
\item (constant paths) every constant map $[0;1]\to X$ is directed;
\item (reparameterization) $d$ is closed under composition with increasing continuous maps $[0;1]\to [0;1]$;
\item (concatenation) $d$ is closed under path-concatenation: if the d-paths $a$, $b$ are consecutive in $X$ ($a(1)=b(0)$), then their ordinary concatenation $a+b$ is also a d-path
\begin{gather*}
(a+b)(t) = a(2t),\,\text{if}\, 0\le t\le \frac{1}{2}, \\
(a+b)(t) = b(2t-1),\,\text{if}\, \frac{1}{2}\le t\le 1.
\end{gather*}
\end{enumerate}
\end{defn}

I propose a new way to construct a directed topological space. My way is more geometric/topological as it does not involve dealing with particular paths.

\begin{conjecture}
Every directed topological space can be constructed in the below described way.
\fxnote{And every one with complete funcoid~$F$ is directed topological space?}
\end{conjecture}

Consider topological space $T$ and its reflexive subfuncoid $F$ (that is $F$ is a funcoid which is less that $T$ in the order of funcoids).
Note that in our consideration $F$ is an endofuncoid (its source and destination are the same).

Then a directed path from point $A$ to point $B$ is defined as a continuous function $f$ from
$\overrightarrow{[0;1]}\circ\overleftarrow{[0;1]}^{-1}$ to $F$ such that $f(0)=A$ and $f(1)=B$.

Because $F$ is less that $T$, we have that every directed path is a path.

\begin{conjecture}
The two directed topological spaces, constructed from a fixed topological space and two different funcoids,
are different.
\end{conjecture}

For a counter-example of (which of the two?) the conjecture consider funcoid $T\sqcap(\mathbb{Q}\times^{\mathsf{FCD}}\mathbb{Q})$
where $T$ is the usual topology on real line.

We need to consider stability of existence and uniqueness of a path under transformations of our funcoid and
under transformations of the vector field. Can this be a step to solve Navier-Stokes existence and smoothness problems?

It seems (check!) that solutions not only of differential equations but also of difference equations can be
expressed as paths in funcoids.