\chapter{Applications of algebraic general topology}

\section{``Hybrid'' objects}

Algebraic general topology allows to construct ``hybrid'' objects of ``continuous'' (as topological spaces)
and discrete (as graphs).

Consider for example $D\sqcup T$ where $D$ is a digraph and $T$ is a topological space.

The $n$-th power $(D\sqcup T)^n$ yields an expression with $2^n$ terms.
So treating $D\sqcup T$ as one object (what becomes possible using algebraic general topology)
rather than the join of two objects may have an exponential benefit for simplicity of formulas.

\section{A way to construct directed topological spaces}

\subsection{Directed line and directed intervals}

\begin{defn}
Let $a$ be an element of a poset~$\mathfrak{A}$.
\begin{enumerate}
\item $\Delta(a) = \bigsqcap^{\mathscr{F}} \setcond{]x;y[}{x,y\in\mathfrak{A}, x<a\land y>a}$;
\item $\Delta_{>}(a) = \bigsqcap^{\mathscr{F}} \setcond{]a;y[}{y\in\mathfrak{A}, x<a\land y>a}$;
\item $\Delta_{\geq}(a) = \bigsqcap^{\mathscr{F}} \setcond{[a;y[}{y\in\mathfrak{A}, y>a}$;
\item $\Delta_{<}(a) = \bigsqcap^{\mathscr{F}} \setcond{]x;a[}{x\in\mathfrak{A}, x<a}$;
\item $\Delta_{\leq}(a) = \bigsqcap^{\mathscr{F}} \setcond{]x;a]}{x\in\mathfrak{A}, x<a}$.
\end{enumerate}
\end{defn}

\begin{obvious}
~
\begin{enumerate}
\item $\Delta_{>}(a) = \Delta(a)\sqcap^{\mathscr{F}} \setcond{y\in\mathfrak{A}}{y>a}$;
\item $\Delta_{\geq}(a) = \Delta(a)\sqcap^{\mathscr{F}} \setcond{y\in\mathfrak{A}}{y\geq a}$;
\item $\Delta_{<}(a) = \Delta(a)\sqcap^{\mathscr{F}} \setcond{x\in\mathfrak{A}}{x<a}$;
\item $\Delta_{\leq}(a) = \Delta(a)\sqcap^{\mathscr{F}} \setcond{x\in\mathfrak{A}}{x\leq a}$.
\end{enumerate}
\end{obvious}

\begin{defn}
~
Given a partial order~$\mathfrak{A}$ and~$x\in\mathfrak{A}$, the following defines complete funcoids:
\begin{enumerate}
\item $|\mathfrak{A}|\{x\} = \Delta(x)$;
\item $|\mathfrak{A}|_{>}\{x\} = \Delta_{>}(x)$;
\item $|\mathfrak{A}|_{\geq}\{x\} = \Delta_{\geq}(x)$;
\item $|\mathfrak{A}|_{<}\{x\} = \Delta_{<}(x)$;
\item $|\mathfrak{A}|_{\leq}\{x\} = \Delta_{\leq}(x)$;
\end{enumerate}
\end{defn}

\begin{xca}
\fxnote{Generalize for arbitrary posets.}
\begin{enumerate}
\item $\rsupfun{|\mathbb{R}|_{\geq}^{-1}}@X = @\setcond{a\in\mathbb{R}}{\forall\epsilon>0:X\cap[a;a+\epsilon[\ne\emptyset}$;
\item $\rsupfun{|\mathbb{R}|_{\leq}^{-1}}@X = @\setcond{a\in\mathbb{R}}{\forall\epsilon>0:X\cap]a-\epsilon;a]\ne\emptyset}$.
\end{enumerate}
\end{xca}

\begin{conjecture}
~
\begin{enumerate}
\item $|\mathbb{R}|_{\geq} = |\mathbb{R}| \sqcap \geq$;
\item $|\mathbb{R}|_{>} = |\mathbb{R}| \sqcap >$;
\item $|\mathbb{R}|_{\leq} = |\mathbb{R}| \sqcap \leq$;
\item $|\mathbb{R}|_{<} = |\mathbb{R}| \sqcap <$.
\end{enumerate}
\end{conjecture}

\begin{rem}
On trivial ultrafilters these obviously agree:
\begin{enumerate}
\item $\rsupfun{|\mathbb{R}|_{\geq}}\{x\} = \rsupfun{|\mathbb{R}| \sqcap \geq}\{x\}$;
\item $\rsupfun{|\mathbb{R}|_{>}}\{x\} = \rsupfun{|\mathbb{R}| \sqcap >}\{x\}$;
\item $\rsupfun{|\mathbb{R}|_{\leq}}\{x\} = \rsupfun{|\mathbb{R}| \sqcap \leq}\{x\}$;
\item $\rsupfun{|\mathbb{R}|_{<}}\{x\} = \rsupfun{|\mathbb{R}| \sqcap <}\{x\}$.
\end{enumerate}
\end{rem}

I will say that a property holds on a filter~$\mathcal{A}$ iff there is $A\in\up\mathcal{A}$ on which the property holds.

\fxnote{$f\in\continuous(A;B)\land f\in\continuous(|A|_{\geq};|B|_{\geq}) \Leftrightarrow
(f;f)\in\continuous((A;|A|_{\geq});(B;|B|_{\geq}))$}

\begin{lem}
Let function~$f\in\continuous(A;B)$ where $A,B\in\subsets\mathbb{R}$ and $A$ is connected.
\begin{enumerate}
\item $f$ is monotone iff $f\in\continuous(|A|_{\geq};|B|_{\geq})$.
\item $f$ is strictly monotone iff $f\in\continuous(|A|_{>};|B|_{>})$.
\end{enumerate}
\fxnote{Generalize for arbitrary posets.}
\fxnote{Generalize for $f$ being a funcoid.}
\end{lem}

\begin{proof}
Because $f$ is continuous, we have $\rsupfun{f\circ A}\{x\} \sqsubseteq \rsupfun{B\circ f}\{x\}$
that is $\rsupfun{f} \Delta(x) \sqsubseteq \Delta(f(x))$ for every~$x$.

If $f$ is monotone, we have $\rsupfun{f} \Delta_{\geq}(x) \sqsubseteq [f(x);\infty[$.
Thus $\rsupfun{f} \Delta_{\geq}(x) \sqsubseteq \Delta_{\geq}(f(x))$, that is
$\rsupfun{f\circ |A|_{\geq}}\{x\} \sqsubseteq \rsupfun{|B|_{\geq}\circ f}\{x\}$, thus
$f\in\continuous(|A|_{\geq};|B|_{\geq})$.

If $f$ is strictly monotone, we have $\rsupfun{f} \Delta_{>}(x) \sqsubseteq ]f(x);\infty[$.
Thus $\rsupfun{f} \Delta_{>}(x) \sqsubseteq \Delta_{>}(f(x))$, that is
$\rsupfun{f\circ |A|_{>}}\{x\} \sqsubseteq \rsupfun{|B|_{>}\circ f}\{x\}$, thus
$f\in\continuous(|A|_{>};|B|_{>})$.

Let now $f\in\continuous(|A|_{\geq};|B|_{\geq})$.

Take any~$a\in A$ and let $c=\setcond{b\in B}{b\geq a, \forall x\in[a;b[: f(x)\geq f(a)}$.
It's enough to prove that $c$ is the right endpoint (finite or infinite) of~$A$.

Indeed by continuity $f(a)\leq f(c)$ and if $c$ is not already the right endpoint of~$A$, then
there is $b'>c$ such that $\forall x\in[c;b'[: f(x)\geq f(c)$.
So we have $\forall x\in[a;b'[: f(x)\geq f(c)$ what contradicts to the above.

So $f$ is monotone on the entire~$A$.

Let now moreover $f\in\continuous(|A|_{>};|B|_{>})$. We need to prove that $f$ is strictly monotone.
Suppose the contrary. Then there is a nonempty interval $[p;q]\subseteq A$ such that $f$ is constant on this interval.
But this is impossible because $f\in\continuous(|A|_{>};|B|_{>})$.
\end{proof}

\begin{thm}
Let function~$f\in\continuous(A;B)$ where $A,B\in\subsets\mathbb{R}$.
\begin{enumerate}
\item $f$ is locally monotone iff $f\in\continuous(|A|_{\geq};|B|_{\geq})$.
\item $f$ is locally strictly monotone iff $f\in\continuous(|A|_{>};|B|_{>})$.
\end{enumerate}
\end{thm}

\begin{proof}
By the lemma it is (strictly) monotone on each connected component.
\end{proof}

See also related math.SE questions:
\begin{enumerate}
\item \url{http://math.stackexchange.com/q/1473668/4876}
\item \url{http://math.stackexchange.com/a/1872906/4876}
\item \url{http://math.stackexchange.com/q/1875975/4876}
\end{enumerate}

\subsection{Directed topological spaces}

Directed topological spaces are defined at\\
\url{http://ncatlab.org/nlab/show/directed+topological+space}

\begin{defn}
A \emph{directed topological space} (or \emph{d-space} for short) is a pair $(X;d)$ of a topological space~$X$ and
a set $d\subseteq\continuous([0;1];X)$ (called \emph{directed paths} or \emph{d-paths}) of paths in~$X$ such that
\begin{enumerate}
\item (constant paths) every constant map $[0;1]\to X$ is directed;
\item (reparameterization) $d$ is closed under composition with increasing continuous maps $[0;1]\to [0;1]$;
\item (concatenation) $d$ is closed under path-concatenation: if the d-paths $a$, $b$ are consecutive in $X$ ($a(1)=b(0)$), then their ordinary concatenation $a+b$ is also a d-path
\begin{gather*}
(a+b)(t) = a(2t),\,\text{if}\, 0\le t\le \frac{1}{2}, \\
(a+b)(t) = b(2t-1),\,\text{if}\, \frac{1}{2}\le t\le 1.
\end{gather*}
\end{enumerate}
\end{defn}

I propose a new way to construct a directed topological space. My way is more geometric/topological as it does not involve dealing with particular paths.

\begin{defn}
Let $ T$ be the complete endofuncoid corresponding to a topological space
and $\nu\sqsubseteq T$ be its ``subfuncoid''. The $\mathrm{d}$-space $\operatorname{(dir)}(T;\nu)$ induced by the pair $(T;\nu)$
consists of~$ T$ and paths $f\in\continuous([0;1]; T) \cap \continuous(|[0;1]|_{\geq}; \nu)$
such that $f(0)=f(1)$.
\end{defn}

\begin{prop}
It is really a $\mathrm{d}$-space.
\end{prop}

\begin{proof}
Every $\mathrm{d}$-path is continuous.

Constant path are $\mathrm{d}$-paths because $\nu$ is reflexive.

Every reparameterization is a $\mathrm{d}$-path because they are $\continuous(|[0;1]|_{\geq}; \nu)$ and we can apply the theorem about
composition of continuous functions.

Every concatenation is a $\mathrm{d}$-path. Denote
$f_0 = \mylambda{t}{[0;\frac{1}{2}]}{a(2t)}$ and $f_1 = \mylambda{t}{[\frac{1}{2};1]}{b(2t-1)}$.
Obviously $f_0,f_1 \in \continuous([0;1];\mu) \cap \continuous(|[0;1]|_{\geq}; \nu)$.
Then we conclude that $a+b = f_1\sqcup f_1$ is in $f_0,f_1 \in \continuous([0;1];\mu) \cap \continuous(|[0;1]|_{\geq}; \nu)$
using the fact that the operation $\circ$ is distributive over $\sqcup$.
\end{proof}

Let now we have a $\mathrm{d}$-space $(X;d)$. Define funcoid~$\nu$ corresponding to the $\mathrm{d}$-space by the formula
$\nu = \bigsqcup_{a\in d}(a\circ |\mathbb{R}|_{\geq}\circ a^{-1})$.

\begin{example}
There is a $\mathrm{d}$-space which cannot be represented as a pair of funcoids.
\end{example}

\begin{proof}
Let $\mathrm{d}$-paths are twice (and also four time, six times, eight times, etc.) clock-wise paths in the unit circle.
\fxerror{It is not a $\mathrm{d}$-space because there is a (nonsurjective) reparameterization which makes one-time clockwise path.
So the counter-example is wrong.}
\end{proof}

\begin{example}
The two directed topological spaces, constructed from a fixed topological space and two different reflexive funcoids,
are the same.
\end{example}

\begin{proof}
Consider the indiscrete topology~$T$ on $\mathbb{R}$ and the funcoids~$1^{\mathsf{FCD}(\mathbb{R};\mathbb{R})}$
and $1^{\mathsf{FCD}(\mathbb{R};\mathbb{R})}\sqcup(\{0\}\times^{\mathsf{FCD}} \Delta_{\geq})$.
The only $\mathrm{d}$-paths in both these settings are constant functions.
\end{proof}

\begin{conjecture}
Every $\mathrm{d}$-space $(X;d)$ is generated by the pair $(X;\nu)$ of funcoids where
$\nu = \bigsqcup_{a\in d}(a\circ |\mathbb{R}|_{\geq}\circ a^{-1})$.
\fxwarning{Counter-example: $d$ is all paths which are polygonal chains.}
\fxwarning{A simpler counterexample: Consider a plane with a fixed point $C$. $\mathrm{d}$-paths
are constant paths and paths which lie in some line passing through $C$.}
\end{conjecture}

\begin{proof}
$f\in\continuous(|[0;1]|_{\geq}; \bigsqcup_{a\in d}(a\circ |\mathbb{R}|_{\geq}\circ a^{-1})) \Leftrightarrow
|[0;1]|_{\geq}\sqsubseteq f^{-1}\circ\bigsqcup_{a\in d}(a\circ |\mathbb{R}|_{\geq}\circ a^{-1})\circ f \Leftrightarrow
|[0;1]|_{\geq}\sqsubseteq \bigsqcup_{a\in d}(f^{-1}\circ a\circ |\mathbb{R}|_{\geq}\circ a^{-1}\circ f)$.

If we take $a=f$ then $f^{-1}\circ a\circ |\mathbb{R}|_{\geq}\circ a^{-1}\circ f \sqsupseteq |\mathbb{R}|_{\geq}$ and thus
$|[0;1]|_{\geq}\sqsubseteq \bigsqcup_{a\in d}(f^{-1}\circ a\circ |\mathbb{R}|_{\geq}\circ a^{-1}\circ f)$.

Let now $|[0;1]|_{\geq}\sqsubseteq \bigsqcup_{a\in d}(f^{-1}\circ a\circ |\mathbb{R}|_{\geq}\circ a^{-1}\circ f)$. Then ??
\end{proof}

\begin{conjecture}
A $\mathrm{d}$-path~$a$ is determined by the funcoids (where $x$ spans $[0;1]$)
\[ (\mylambda{t}{\mathbb{R}}{a(x+t)})|_{\Delta(0)}. \]
\end{conjecture}

We need to consider stability of existence and uniqueness of a path under transformations of our funcoid and
under transformations of the vector field. Can this be a step to solve Navier-Stokes existence and smoothness problems?

It seems (check!) that solutions not only of differential equations but also of difference equations can be
expressed as paths in funcoids.