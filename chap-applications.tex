\chapter{Applications of algebraic general topology}

\fxnote{Should also Consider ``one-side topologies'' like $T\sqcap\setcond{(x;y)}{x,y\in\mathbb{R},y\ge x}$ where
$T$ is the usual topology on~$\mathbb{R}$. Wrap a $\pi$-length segment of this line into a circle
and get one-side circle: We can traverse continuously counter-clockwise but not clockwise.}

\section{``Hybrid'' objects}

Algebraic general topology allows to construct ``hybrid'' objects of ``continuous'' (as topological spaces)
and discrete (as graphs).

Consider for example $D\sqcup T$ where $D$ is a digraph and $T$ is a topological space.
I am unsure whether there are important applications of such hybrid objects.

\section{A way to construct directed topological spaces}

\subsection{Directed line and directed intervals}

\begin{defn}~
\begin{enumerate}
\item $\Delta_{+} + a = \bigsqcap^{\mathscr{F}} \setcond{[a;a+\epsilon[}{\epsilon\in\mathbb{R}, \epsilon>0} = \Delta\sqcap[a;+\infty]$;
\item $\Delta_{-} + a = \bigsqcap^{\mathscr{F}} \setcond{]a-\epsilon;a]}{\epsilon\in\mathbb{R}, \epsilon>0} = \Delta\sqcap[-\infty;a]$.
\end{enumerate}
\end{defn}

\begin{defn}~
\begin{enumerate}
\item \emph{Right directed (real) line} is the complete funcoid $\overleftarrow{\mathbb{R}}$ defined by the formula
$\rsupfun{\overrightarrow{\mathbb{R}}}@\{a\} = \Delta_{+} + a$ for every $a\in\mathbb{R}$;
\item \emph{Left directed (real) line} is the complete funcoid $\overleftarrow{\mathbb{R}}$ defined by the formula
$\rsupfun{\overrightarrow{\mathbb{R}}}@\{a\} = \Delta_{-} + a$ for every $a\in\mathbb{R}$;
\end{enumerate}
\end{defn}

\begin{xca}~
\begin{enumerate}
\item $\rsupfun{\overrightarrow{\mathbb{R}}^{-1}}@X = @\setcond{a\in\mathbb{R}}{\forall\epsilon>0:X\cap[a;a+\epsilon[\ne\emptyset}$;
\item $\rsupfun{\overleftarrow{\mathbb{R}}^{-1}}@X = @\setcond{a\in\mathbb{R}}{\forall\epsilon>0:X\cap]a-\epsilon;a]\ne\emptyset}$.
\end{enumerate}
\end{xca}

\begin{defn}
For every set $A\in\subsets\mathbb{R}$
\begin{enumerate}
\item $\overrightarrow{A} = \overrightarrow{\mathbb{R}} \sqcap(A\times A)$;
\item $\overleftarrow{A} = \overleftarrow{\mathbb{R}} \sqcap(A\times A)$.
\end{enumerate}
If $A$ is an interval, then $\overrightarrow{A}$ is called \emph{right directed interval} and $\overleftarrow{A}$ is called \emph{left directed interval}.
\end{defn}

First prove a special case of the below theorem: \fxwarning{Should we remove this?}

\begin{example}
Let $A = \overrightarrow{[0;1]}\circ\overleftarrow{[0;1]}^{-1}$.
Consider function $f (t) = \begin{cases}
  - 1 / x & \text{ if } t < 0\\
  0 & \text{ if } t \geqslant 0
\end{cases}$. Let us deduce $f \notin \mathrm{C} (A ; B)$ from the
fact that $f$ is not monotone.
\end{example}

\begin{proof}
$\rsupfun{f\circ\overrightarrow{[0;1]}\circ\overleftarrow{[0;1]}^{-1}} ]-\epsilon;0[ =
\rsupfun{f\circ\overrightarrow{[0;1]}} ]-\epsilon;0] = [0;1/\epsilon]$;

$\rsupfun{B\circ f} ]-\epsilon;0[ = \rsupfun{B} ]0;1/\epsilon[$;

Then $@\{0\}\nasymp\rsupfun{f\circ\overrightarrow{[0;1]}\circ\overleftarrow{[0;1]}^{-1}} ]-\epsilon;0[$;
but $@\{0\}\asymp\rsupfun{B} ]-\epsilon;0[$;
\end{proof}

\begin{thm}
A function~$f:A\rightarrow B$ where $A,B\in\subsets\mathbb{R}$ is continuous (regarding the usual topologies on~$A$ and~$B$) and monotone iff
$f\in\continuous(\overrightarrow{A}\circ\overleftarrow{A}^{-1};\overrightarrow{B})$, provided that $A$ is a connected set (=interval).
\end{thm}

\begin{proof}
\url{http://math.stackexchange.com/questions/1473668/locally-monotone-function-is-monotone}

Old counter-example: $f(t) = \begin{cases}-1/x&\text{ if }x<0\\0&\text{ if }x\ge 0\end{cases}$.
??
\end{proof}

\subsection{Directed topological spaces}

Directed topological spaces are defined at\\
\url{http://ncatlab.org/nlab/show/directed+topological+space}

\begin{defn}
A \emph{directed topological space} (or \emph{d-space} for short) is a pair $(X;d)$ of a topological space~$X$ and
a set $d\subseteq\continuous([0;1];X)$ (called \emph{directed paths} or \emph{d-paths}) of paths in~$X$ such that
\begin{enumerate}
\item (constant paths) every constant map $[0;1]\to X$ is directed;
\item (reparameterization) $d$ is closed under composition with increasing continuous maps $[0;1]\to [0;1]$;
\item (concatenation) $d$ is closed under path-concatenation: if the d-paths $a$, $b$ are consecutive in $X$ ($a(1)=b(0)$), then their ordinary concatenation $a+b$ is also a d-path
\begin{gather*}
(a+b)(t) = a(2t),\,\text{if}\, 0\le t\le \frac{1}{2}, \\
(a+b)(t) = b(2t-1),\,\text{if}\, \frac{1}{2}\le t\le 1.
\end{gather*}
\end{enumerate}
\end{defn}

I propose a new way to construct a directed topological space. My way is more geometric/topological as it does not involve dealing with particular paths.

\begin{conjecture}
Every directed topological space can be constructed in the below described way.
\end{conjecture}

Consider topological space $T$ and its reflexive subfuncoid $F$ (that is $F$ is a funcoid which is less that $T$ in the order of funcoids).
Note that in our consideration $F$ is an endofuncoid (its source and destination are the same).

Then a directed path from point $A$ to point $B$ is defined as a continuous function $f$ from $[0;1]$ to $F$ such that $f(0)=A$ and $f(1)=B$.
We can consider $[0;1]$ either as the funcoid corresponding to the closure operator on $[0;1]$ topology or the pretopology on $[0;1]$.
These two funcoids are reverse of each other (TODO: theorem reference). If we flip it reverse, then also flip $F$ to its reverse $F^{-1}$ and
this way the set $\continuous([0;1];F)$ remains the same (when both funcoids are reversed simultaneously). \fxwarning{Add proof.}

Because $F$ is less that $T$, we have that every directed path is a path.

\begin{conjecture}
The two directed topological spaces, constructed from a fixed topological space and two different funcoids,
are different.
\end{conjecture}

For a counter-example of (which of the two?) the conjecture consider funcoid $T\sqcap(\mathbb{Q}\times^{\mathsf{FCD}}\mathbb{Q})$
where $T$ is the usual topology on real line.

We need to consider stability of existence and uniqueness of a path under transformations of our funcoid and
under transformations of the vector field. Can this be a step to solve Navier-Stokes existence and smoothness problems?

It seems (check!) that solutions not only of differential equations but also of difference equations can be
expressed as paths in funcoids.