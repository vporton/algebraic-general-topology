\chapter{Story of the discovery}

I was a Protestant. (Now I have a new religion.\footnote{\url{https://www.smashwords.com/books/view/618525}}).

I deemed that I should openly proclaim my faith: (Luk.~9:26) ``For whoever shall be ashamed of me and of my words, of him shall the Son of man be ashamed, when he shall come in his own glory, and in his Father's, and of the holy angels.'' and Mrk.~8:38.

Moreover, I ``reduced'' my confession: ``I am a sectarian'', ``I am a religious fanatic.'' I considered the word ``sectarian'' as one of the Christ's words, because the Gospel, 2Cor.~6:17 contains the word ``separate'', the root of which has the same meaning as the roots of the words ``sectarian'' and ``holy''. I considered the word ``fanatic'' to be one of Christ's words, because the Bible says (Rev.~3:19) ``be zealous'' and ``jealous'' and ``fanatic'' are words with the root of a similar meaning.

My so-called ``confession of faith'' caused a sharply negative reaction of people and led to religious discrimination, refusal to talk to me, insults, and often beatings.
Moreover, realizing the hopelessness of my situation, I did not even try to improve my social status, since this was clearly impossible. In addition, with such my position, new opportunities would mean new problems for me.

When I was a first year student at Perm State University, I became interested in general topology and set a goal to discover algebraic general topology.

So I ended up on the street, without food. I began to eat grass and
drink from a puddle and wait for death from hunger (as you know, I still
survived).

From nothing to do, I continued my mathematical thoughts and
came up with a definition of funcoid. The biggest math
discovery in general topology since 1937 (when the filters were opened)
was made by a hungry homeless on the street.

I wrote a term paper at my first year opening in the university.

Understanding that a religious fanatic cannot find a job and for me it is threatening soon
again starvation and death, I decided to show humility:
become economically weaker (abandon my economic
goals and ambitions) in order to become richer. To become
economically weaker, I decided to leave the university in the 5th year
and filed a deduction.

My humility worked: I managed to get a second disability
group that provided the conditions for my survival. Besides
other things, I told psychiatrists that I have a strange object in my brain, a seraph (``genius'' in Greek mythology). Consider both
options: if I have a foreign object in my brain then I'm a disabled person
in the psyche, if not then disabled in the psyche, too.

As you know, I wrote a doctoral dissertation in mathematics (you read it) and I was not awarded the title of Doctor of Science for religious reasons as a punishment for practicing my religion.

I sued, demanding compensation for the unpaid salary of a professor of mathematics and other things, as well as 4~trillion dollars as compensation for not made due to poverty scientific discoveries. (I valued this book along with amendments, as well as my XML file processing method in 2~trillion dollars; well, how much is the limit of the discontinuous function?)

It was not that court, and after that I filed a lawsuit in the Tverskoy court of Moscow. This time without the requirement of 4~trillions and the title of hero of Russia.

But when she saw the word ``sectarian'', the chairman of the court, Olga Nikolaevna Solopova, went crazy with laughter and shame and, deciding that humor took precedence over the law, did not respond to my lawsuit. It is clear that Solopova cannot answer, therefore I demanded that the qualification collegium of judges recognize her as incompetent and insane as a result of exposure to her brain with information about an abnormal sectarian and transfer the case to another judge. Qualification board has not yet responded. Such should be the reaction of a judge to a suit of a subhuman, in accordance with humor.

Note: I'm not going to actually bankrupt Russia.

About \emph{mathematical} aspects of the story of my discoveries, see blog post:\\
\url{https://portonmath.wordpress.com/?p=2992}
