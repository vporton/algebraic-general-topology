\appendix

\chapter{\label{app:prim-exists}Using logic of generalizations}


\section{Logic of generalization}

In mathematics it is often encountered that a smaller set $S$ naturally
bijectively corresponds to a subset $R$ of a larger set $B$. (In
other words, there is specified an injection from $S$ to $B$.) It
is a widespread practice to equate $S$ with $R$.
\begin{rem}
I denote the first set $S$ from the first letter of the word ``small''
and the second set $B$ from the first letter of the word ``big'',
because $S$ is intuitively considered as smaller than $B$. (However
we do not require $\card S<\card B$.)
\end{rem}
The set $B$ is considered as a generalization of the set $S$, for
example: whole numbers generalizing natural numbers, rational numbers
generalizing whole numbers, real numbers generalizing rational numbers,
complex numbers generalizing real numbers, etc.

But strictly speaking this equating may contradict to the axioms of
ZF/ZFC because we are not insured against $S\cap B\neq\emptyset$
incidents. Not wonderful, as it is often labeled as ``without proof''.

To work around of this (and formulate things exactly what could benefit
computer proof assistants) we will replace the set $B$ with a new
set $B'$ having a bijection $M:B\rightarrow B'$ such that $S\subseteq B'$.
(I call this bijection $M$ from the first letter of the word ``move''
which signifies the move from the old set $B$ to a new set $B'$).

The following is a formal but rather silly formalization of this situation
in ZF. (A more natural formalization may be done in a smarter formalistic,
such as dependent type theory.)


\subsection{The formalistic}

Let $S$ and $B$ be sets. Let $E$ be an injection from $S$ to $B$.
Let $R=\im E$.

Let $t=\subsets\bigcup\bigcup S$.

Let $M(x)=\left\{ \begin{array}{ll}
E^{-1}x & \text{if }x\in R;\\
(t,x) & \text{if }x\notin R.
\end{array}\right.$

Recall that in standard ZF $(t,x)=\{t,\{t,x\}\}$ by definition.
\begin{thm}
$(t,x)\notin S$.\end{thm}
\begin{proof}
Suppose $(t,x)\in S$. Then $\{t,\{t,x\}\}\in S$. Consequently $\{t\}\in\bigcup S$;
$\{t\}\subseteq\bigcup\bigcup S$; $\{t\}\in\subsets\bigcup\bigcup S$;
$\{t\}\in t$ what contradicts to the axiom of foundation (aka axiom
of regularity).\end{proof}
\begin{defn}
Let $B'=\im M$.\end{defn}
\begin{thm}
$S\subseteq B'$.\end{thm}
\begin{proof}
Let $x\in S$. Then $Ex\in R$; $M(Ex)=E^{-1}Ex=x$; $x\in\im M=B'$.\end{proof}
\begin{obvious}
$E$ is a bijection from $S$ to $R$.\end{obvious}
\begin{thm}
$M$ is a bijection from $B$ to $B'$.\end{thm}
\begin{proof}
Surjectivity of $M$ is obvious. Let's prove injectivity. Let $a,b\in B$
and $M(a)=M(b)$. Consider all cases: \end{proof}
\begin{description}
\item [{$a,b\in R$}] $M(a)=E^{-1}a$; $M(b)=E^{-1}b$; $E^{-1}a=E^{-1}b$;
thus $a=b$ because $E^{-1}$ is a bijection.
\item [{$a\in R$,~$b\notin R$}] $M(a)=E^{-1}a$; $M(b)=(t,b)$; $M(a)\in S$;
$M(b)\notin S$. Thus $M(a)\neq M(b)$.
\item [{$a\notin R$,~$b\in R$}] Analogous.
\item [{$a,b\notin R$}] $M(a)=(t,a)$; $M(b)=(t,b)$. Thus $M(a)=M(b)$
implies $a=b$. \end{description}
\begin{thm}
$M\circ E=\id_{S}$.\end{thm}
\begin{proof}
Let $x\in S$. Then $Ex\in R$; $M(Ex)=E^{-1}Ex=x$.\end{proof}
\begin{obvious}
$E=M^{-1}|_{S}$.
\end{obvious}

\subsection{Existence of primary filtrator}
\begin{thm}
For every poset~$\mathfrak{Z}$ there exists a poset $\mathfrak{A}\supseteq\mathfrak{Z}$
such that $(\mathfrak{A},\mathfrak{Z})$ is a primary filtrator.\end{thm}
\begin{proof}
Take $S=\mathfrak{Z}$, $B=\mathfrak{F}$, $E=\uparrow$. By the above
there exists an injection $M$ defined on~$\mathfrak{F}$ such that
$M\circ\uparrow=\id_{\mathfrak{Z}}$.

Take $\mathfrak{A}=\im M$. Order ($\sqsubseteq'$) elements of~$\mathfrak{A}$
in such a way that $M:\mathfrak{F}(\mathfrak{Z})\rightarrow\mathfrak{A}$
become order isomorphism. If $x\in\mathfrak{Z}$ then $x=\id_{\mathfrak{Z}}x=M\uparrow x\in\im M=\mathfrak{A}$.
Thus $\mathfrak{A}\supseteq\mathfrak{Z}$.

If $x\sqsubseteq y$ for elements $x$, $y$ of $\mathfrak{Z}$, then
$\uparrow x\sqsubseteq\uparrow y$ and thus $M\uparrow x\sqsubseteq'M\uparrow y$
that is $x\sqsubseteq'y$, so $\mathfrak{Z}$ is a subposet of~$\mathfrak{A}$,
that is $(\mathfrak{A},\mathfrak{Z})$ is a filtrator.

It remains to prove that $M$ is an isomorphism between filtrators
$(\mathfrak{F}(\mathfrak{Z}),\mathfrak{P})$ and $(\mathfrak{A},\mathfrak{Z})$.
That $M$ is an order isomorphism from $\mathfrak{F}(\mathfrak{Z})$
to $\mathfrak{A}$ is already known. It remains to prove that $M$
maps $\mathfrak{P}$ to~$\mathfrak{Z}$. We will instead prove that
$M^{-1}$ maps $\mathfrak{Z}$ to~$\mathfrak{P}$. Really, $\uparrow x=M^{-1}x$ for every $x\in\mathfrak{Z}$.\end{proof}

