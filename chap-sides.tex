\chapter{Systems of sides}

Now we will consider a common generalization of (some of pointfree) funcoids and (some of) Galois connections.
The main purpose of this is general theorem~\ref{neg-prod} below.

First consider some properties of Galois connections:

\section{More on Galois connections}

Here I will denote $\supfun{f}$ the lower adjoint of a Galois connection~$f$. \fxnote{Switch to this notation in the book?}

Let $\mathsf{GAL}$ be the category of Galois connections.
\fxwarning{Need to decide whether use $\mathsf{GAL}(A;B)$ or $A\otimes B$.}

I will denote $(f;g)^{-1}=(g;f)$ for a Galois connection~$(f;g)$.

We will order Galois connections by the formula
\[ f\sqsubseteq g \Leftrightarrow \supfun{f}\sqsubseteq\supfun{g} \Leftrightarrow \supfun{f^{-1}}\sqsupseteq\supfun{g^{-1}}. \]

\begin{obvious}
This defines a partial order on the set of Galois connections between any two (fixed) posets.
\end{obvious}

\begin{prop}\label{gal-fjoin-x}
If $f$ and~$g$ are Galois connections (between a join-semilattice~$\mathfrak{A}$ and a meet-semilattice~$\mathfrak{B}$),
then there exists a Galois connection~$f\sqcup g$
determined by the formula~$\supfun{f\sqcup g}x = \supfun{f}x\sqcup\supfun{g}x$.
\end{prop}

\begin{proof}
It is enough to prove that
\[ (x\mapsto\supfun{f}x\sqcup\supfun{g}x; y\mapsto\supfun{f^{-1}}y\sqcap\supfun{g^{-1}}y) \]
is a Galois connection that is that
\[ \supfun{f}x\sqcup\supfun{g}x\sqsubseteq y\Leftrightarrow x\sqsubseteq\supfun{f^{-1}}y\sqcap\supfun{g^{-1}}y \] for all relevant~$x$ and~$y$.

Really,
\begin{multline*}
\supfun{f}x\sqcup\supfun{g}x\sqsubseteq y \Leftrightarrow
\supfun{f}x\sqsubseteq y\land\supfun{g}x\sqsubseteq y \Leftrightarrow \\
x\sqsubseteq\supfun{f^{-1}}y\land x\sqsubseteq\supfun{g^{-1}}y \Leftrightarrow
x\sqsubseteq\supfun{f^{-1}}y\sqcap\supfun{g^{-1}}y.
\end{multline*}
\end{proof}

\fxnote{Describe infinite join of Galois connections.}

\begin{prop}\label{a-bot}
If $\mathfrak{A}$ is a poset with least element, then $\supfun{a}\bot=\bot$.
\end{prop}

\begin{proof}
$\supfun{a}\bot\sqsubseteq y\Leftrightarrow \bot\sqsubseteq\supfun{a^{-1}}y\Leftrightarrow 1$.
Thus $\supfun{a}\bot$ is the least element.
\end{proof}

\begin{prop}
$(\mathfrak{A}\times\{\bot^\mathfrak{B}\}; \mathfrak{B}\times\{\top^\mathfrak{A}\})$ is the least Galois connection from
a poset~$\mathfrak{A}$ with greatest element to a poset~$\mathfrak{B}$ with least element.
\end{prop}

\begin{proof}
Let's prove that it is a Galois connection. We need to prove
\[ (\mathfrak{A}\times\{\bot^\mathfrak{B}\})x\sqsubseteq y \Leftrightarrow x\sqsubseteq (\mathfrak{B}\times\{\top^\mathfrak{A}\})y. \]
But this is trivially equivalent to $1\Leftrightarrow 1$. Thus it's a Galois connection.

That it the least is obvious.
\end{proof}

\begin{cor}
$\supfun{\bot}x=\bot$ for Galois connections from
a poset~$\mathfrak{A}$ with greatest element to a poset~$\mathfrak{B}$ with least element.
\fxwarning{Clarify.}
\end{cor}

\begin{thm}\label{gal-bound}
If $\mathfrak{A}$ and $\mathfrak{B}$ are bounded posets, then $\mathsf{GAL}(\mathfrak{A}; \mathfrak{B})$ is bounded.
\end{thm}

\begin{proof}
That $\mathsf{GAL} (\mathfrak{A}; \mathfrak{B})$ has least element was proved
above. I will demonstrate that $(\alpha ; \beta)$
is the greatest element of $\mathsf{pFCD} (\mathfrak{A}; \mathfrak{B})$ for
\[ \alpha X = \begin{cases}
     \bot^{\mathfrak{B}} & \text{if } X = \bot^{\mathfrak{A}}\\
     \top^{\mathfrak{B}} & \text{if } X \neq \bot^{\mathfrak{A}}
   \end{cases} ; \quad
   \beta Y = \begin{cases}
     \top^{\mathfrak{A}} & \text{if } Y = \top^{\mathfrak{B}}\\
     \bot^{\mathfrak{A}} & \text{if } Y \neq \top^{\mathfrak{B}}
   \end{cases} . \]
First prove $Y \sqsubseteq \alpha X \Leftrightarrow X \sqsubseteq \beta Y$.

Really $\alpha X\sqsubseteq Y \Leftrightarrow X=\bot^{\mathfrak{A}}\lor Y=\top^{\mathfrak{B}} \Leftrightarrow X \sqsubseteq \beta Y$.

That it is the greatest Galois connection between~$\mathfrak{A}$ and~$\mathfrak{B}$ easily follows from proposition~\ref{a-bot}.
\end{proof}

\begin{thm}\label{gal-id-ex}
  For every brouwerian lattice $x \mapsto c \sqcap x$ is a lower adjoint.
\end{thm}

\begin{proof}
  By dual of theorem~\bookref{cobrow-adj}.
\end{proof}

\begin{xca}
  Describe the corresponding upper adjoint, especially for the special case of
  boolean lattices.
\end{xca}

\section{Definition}

\begin{defn}
\emph{System of presides} is
a functor $\Upsilon = (f\mapsto\supfun{f})$ from an ordered category
to the category of functions between (small) bounded lattices,
such that (for all relevant variables):
\begin{enumerate}
  \item Every Hom-set of~$\Src\Upsilon$ is a bounded join-semilattice.

  \item $\supfun{a}\bot = \bot$.

  \item $\supfun{a \sqcup b} X = \supfun{a} X \sqcup \supfun{b} X$ (equivalent to $\Upsilon$ to be a join-semilattice homomorphism,
    if we order functions between small bounded lattices component-wise).
\end{enumerate}
I call morphisms of such categories \emph{sides}.\footnote{The idea for the name is that we consider one ``side''~$\supfun{f}$ of a funcoid instead of both sides~$\supfun{f}$ and~$\supfun{f^{-1}}$.}
\end{defn}

\begin{rem}
We could generalize to functions between small join-semilattices with least elements instead of bounded lattices only, but this is not really necessary.
\end{rem}

\begin{defn}
\emph{Bounded} system of presides is system of presides from an ordered category with bounded Hom-sets
such that $X,Y\in\Ob\Src\Upsilon$ the following additional axioms hold for all suitable~$a$:
\begin{enumerate}
  \item $\supfun{\bot^{\Hom(X;Y)}} a = \bot$.

  \item $\supfun{\top^{\Hom(X;Y)}} a = \top$ unless $a = \bot$
\end{enumerate}
\end{defn}

\begin{defn}
\emph{System of presides with identities} is a system of presides with
a morphism $\id_a\in\Src\Upsilon$ for every object $\mathfrak{A}$ of~$\Src\Upsilon$ and $a\in\mathfrak{A}$
and the following additional axioms:
\begin{enumerate}
  \item $\id_c \sqsubseteq 1_{\mathfrak{A}}$ for every $c \in \mathfrak{A}$
    where $\mathfrak{A}$ is an object of~$\Src\Upsilon$.

  \item $\supfun{\id_c} = (\lambda x \in \mathfrak{A}: x \sqcap c)$ for every $c \in \mathfrak{A}$
    where $\mathfrak{A}$ is an object of~$\Src\Upsilon$
\end{enumerate}
\end{defn}

\begin{defn}
\emph{System of sides} is a system of presides which is both bounded and with identities.
\end{defn}

\begin{prop}
$\supfun{1^{\Src\Upsilon}_\mathfrak{A}} a = a$ for every system of presides.
\end{prop}

\begin{proof}
By properties of functors.
\end{proof}

\begin{defn}
I call a system of \emph{monotone} presides a system of presides with additional axiom:
\begin{enumerate}
\item $\supfun{a}$ is monotone.
\end{enumerate}
\end{defn}

\begin{defn}
I call a system of \emph{distributive} presides a system of presides with additional axiom:
\begin{enumerate}
\item $\supfun{a}(X\sqcup Y) = \supfun{a}X\sqcup\supfun{a}Y$.
\end{enumerate}
\end{defn}

\begin{obvious}
Every distributive system of presides is monotone.
\end{obvious}

\begin{prop}
$\supfun{a \sqcap b} X \sqsubseteq \supfun{a} X \sqcap \supfun{b} X$ for monotone systems of sides
if Hom-sets are lattices.
\end{prop}

\begin{defn}
A system of presides \emph{with correct identities} is a system of presides with identities with additional axiom:
\begin{enumerate}
\item $\id_b\circ\id_a = \id_{a\sqcap b}$.
\end{enumerate}
\end{defn}

\begin{prop}
Every faithful system of presides with identities is with correct identities.
\end{prop}

\begin{proof}
$\supfun{\id_b\circ\id_a}x = (\supfun{\id_b}\circ\supfun{\id_a})x = \supfun{\id_b}\supfun{\id_a}x = b\sqcap a\sqcap x = \supfun{\id_{b\sqcap a}}x$.
Thus by faithfulness $\id_b\circ\id_a = \id_{b\sqcap a} = \id_{a\sqcap b}$.
\end{proof}

\begin{defn}
Protofuncoids \emph{over} a set~$X$ of functors is a protofuncoid~$f$
such that $\supfun{f}\in X\land\supfun{f^{-1}}\in X$.
\end{defn}

\section{Concrete examples of sides}

\begin{obvious}
The category~$\mathbf{Rel}$ with $\supfun{f}=\rsupfun{f}$ for $f\in\mathbf{Rel}$ and usual $\id_c$ defines a distributive system of sides with correct identities.
\end{obvious}

\subsection{Funcoids and pointfree funcoids}

\begin{prop}
The category of pointfree funcoids between starrish join-semilattices with usual~$\supfun{f}$ defines a system of presides.
\end{prop}

\begin{proof}
Theorem~\bookref{pf-fin-join}.
\end{proof}

\begin{prop}
The category of pointfree funcoids between bounded starrish join-semilattices with usual~$\supfun{f}$ defines a system of
bounded presides.
\end{prop}

\begin{proof}
Take the proof of theorem~\bookref{pfcd-bound} into account.
\end{proof}

\begin{prop}
The category of pointfree funcoids from a starrish join-semilattices to a separable starrish join-semilattices
defines a distributive system of presides.
\end{prop}

\begin{proof}
Theorem~\bookref{pf-dist-func}.
\end{proof}

\begin{prop}
The category of pointfree funcoids between starrish lattices with usual~$\supfun{f}$ defines a system of presides with correct identities.
\end{prop}

\begin{proof}
That it is with identities is obvious.

That it is with correct identities is obvious.
\end{proof}

\begin{obvious}
The category of pointfree funcoids between bounded starrish lattices with usual~$\supfun{f}$ defines a system of sides with correct identities.
\end{obvious}

\subsection{Galois connections}

\begin{prop}
The category of Galois connections between (small) lattices with least elements together with usual~$\supfun{f}$
defines a distributive system of presides.
\end{prop}

\begin{proof}
Propositions~\ref{gal-fjoin-x} and~\ref{a-bot} for a system of presides.

It is distributive because lower adjoints preserve all joins.
\end{proof}

\begin{prop}
The category of Galois connections between (small) bounded lattices together with usual~$\supfun{f}$
defines a bounded system of presides.
\end{prop}

\begin{proof}
Theorem~\ref{gal-bound}.
\end{proof}

\begin{prop}
The category of Galois connections between (small) Heyting lattices together with usual~$\supfun{f}$
defines a system of sides with correct identities.
\end{prop}

\begin{proof}
Theorem~\ref{gal-id-ex} ensures that they a system of sides with identities. The identities are correct due to faithfulness.
\end{proof}

\subsection{Reloids}

??

\fxnote{These examples works for (dagger) systems of sides with binary product.}

\subsection{Some subsides}

\begin{defn}
\emph{Full subsystem} of a system~$\Upsilon$ of presides is the functor~$\Upsilon$ restricted to a full subcategory of~$\Src\Upsilon$.
\end{defn}

\begin{obvious}
Full subsystem of a system of presides is always a system of presides.
\end{obvious}

\begin{obvious}
Full subsystem of a bounded system of presides is always a bounded subsystem of presides.
\end{obvious}

\begin{obvious}
~
\begin{enumerate}
\item Full subsystem of a system of presides with identities is always with identities.
\item Full subsystem of a system of presides with correct identities is always with correct identities.
\end{enumerate}
\end{obvious}

\begin{obvious}
Full subsystem of a distributive system of presides is always a distributive system of presides.
\end{obvious}

\begin{obvious}
Full subsystem of a system of sides is always a system of sides.
\end{obvious}

\fxnote{Derive that funcoids are sides from properties of pointfree funcoids.}

\section{Product}

\begin{defn}
A \emph{system of sides with binary product} is a system of sides together with functions $\mathord{\times}:\mathfrak{A}\rightarrow\mathfrak{B}$
\fxerror{Wrong source and destination.}
for every objects~$\mathfrak{A}$,~$\mathfrak{B}$, such that
\begin{enumerate}
\item \label{f-c-x-y}$f\sqcap(X\times Y) = \id_Y\circ f\circ\id_X$
\item \label{f-xy-x0y0}$(X_0\times Y_0)\sqcap(X_1\times Y_1) = (X_0\sqcap X_1)\times(Y_0\sqcap Y_1)$
\item $\supfun{f}\circ(X\times Y) = X\times\supfun{f}Y$
\end{enumerate}
\end{defn}

\begin{prop}
$\supfun{A\times B}X = \begin{cases}\bot&\text{ if }X\asymp A\\B&\text{ if }X\nasymp A\end{cases}$
\end{prop}

\begin{proof}
~
\begin{multline*}
\supfun{A\times B}X = \supfun{\top\sqcap(A\times B)}X = \supfun{\id_B\circ\top\circ\id_A}X = \\
\supfun{\id_B}\supfun{\top}\supfun{\id_A}X =
B\sqcap\supfun{\top}(X\sqcap A) = \\
B\sqcap\begin{cases}\bot&\text{ if }X\asymp A\\\top&\text{ if }X\nasymp A\end{cases} =
\begin{cases}\bot&\text{ if }X\asymp A\\B&\text{ if }X\nasymp A\end{cases}
\end{multline*}
\end{proof}

\begin{cor}
For faithful systems of sides binary product can be defined no more than in one way.
\fxnote{Give a definition of faithful.}
\end{cor}

\begin{prop}
For faithful systems of sides axiom~\ref{f-xy-x0y0} follows from axiom~\ref{f-c-x-y}.
\end{prop}

\begin{proof}
$(X_0\times Y_0)\sqcap(X_1\times Y_1) = \id_{Y_1}\circ(X_0\times Y_0)\circ\id_{X_1}$.
Thus
\begin{multline*}
\supfun{(X_0\times Y_0)\sqcap(X_1\times Y_1)} P = \supfun{\id_{Y_1}}\supfun{X_0\times Y_0}\supfun{\id_{X_1}} P = \\
\supfun{\id_{Y_1}}\begin{cases}\bot&\text{ if }X_0\asymp\supfun{\id_{X_1}}P\\Y_0&\text{ if }X_0\nasymp\supfun{\id_{X_1}}P\end{cases} =
\begin{cases}\bot&\text{ if }X_0\sqcap X_1\asymp P\\Y_0\sqcap Y_1&\text{ if }X_0\sqcap X_1\nasymp P\end{cases} = \\
\supfun{(X_0\sqcap X_1)\times(Y_0\sqcap Y_1)} P.
\end{multline*}
So $(X_0\times Y_0)\sqcap(X_1\times Y_1) = (X_0\sqcap X_1)\times(Y_0\sqcap Y_1)$ follows by full faithfulness.
\end{proof}

\fxnote{Prove existence of binary product for the system of sides for Galois connections.}

\fxnote{The category $\mathbf{Rel}$ as an example of a system of sides.}

\section{Negative results}

The following negative result generalizes theorem~3.8 in~\cite{tprod-dist-lat}.

\begin{thm}\label{neg-prod}
  The element $1^{(\Src\Upsilon)(\mathfrak{A}; \mathfrak{A})}$ is not
  complemented if $\mathfrak{A}$ is a non-atomic boolean lattice,
  for every monotone system of sides.
\end{thm}

\begin{proof}
  Let $T = 1^{(\Src\Upsilon)(\mathfrak{A}; \mathfrak{A})}$.
  
  Let's suppose $T \sqcup V = \top$ for $V \in (\Src\Upsilon) (\mathfrak{A};
  \mathfrak{A})$ and prove $T \sqcap V \neq \bot$.
  
  Then $\supfun{T \sqcup V} a = \top$ for all $a \neq \bot$ and thus $\supfun{V}
  a \sqcup a = \top$.
  
  Consequently $\supfun{V} a \sqsupseteq \neg a$ for all $a \neq \bot$.
  
  If $a$ isn't an atom, then there exists $b$ with $0 \sqsubset b \sqsubset a$
  and hence $\supfun{V} a \sqsupseteq \supfun{V} b \sqsupseteq \neg b \sqsupset \neg a$;
  thus $\supfun{V} a \sqsupset \neg a$.
  
  There is such $c\sqsubset\top$ that $a \sqsubseteq c$ for every atom $a$. (Really,
  suppose some element $p \neq \bot$ has no atoms. Thus all atoms are in $\neg
  p$.)
  
  For $a \nsqsubseteq c$ we have $\supfun{V} a \sqcap a \sqsupset \bot$
  for all $a \sqsubseteq \neg c$ thus $\supfun{T \sqcap V} a \sqsupseteq
  \supfun{V} a \sqcap a \sqsupset \bot$. Thus $\supfun{(T \sqcap V) \circ
  \id_{\neg c}} a \sqsupset \bot$
  
  So $T \sqcap V \sqsupseteq (T \sqcap V) \circ \id_{\neg c} \sqsupset
  \bot$. So $V$ is not a complement of $T$.
\end{proof}

\begin{cor}
  $(\Src\Upsilon)(\mathfrak{A}; \mathfrak{A})$ is not boolean if $\mathfrak{A}$
  is a non-atomic boolean lattice.
\end{cor}

\section{Dagger systems of sides}

\begin{defn}
\emph{Dagger system of sides} is system of sides with category~$\Src\Upsilon$ being a dagger category
and $(\id_X)^\dagger = \id_X$ for every~$X$.
\end{defn}

\begin{defn}
\emph{Dagger system of sides with binary products} is a system which is both a dagger system of sides and system of sides with binary products
such that $(X\times Y)^\dagger = Y\times X$.
\end{defn}