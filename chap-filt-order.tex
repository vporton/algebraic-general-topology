
\chapter{Orderings of filters in terms of reloids}

Whilst the other chapters of this book use filters to research funcoids
and reloids, here the opposite thing is discussed, the theory of reloids
is used to describe properties of filters.

In this chapter the word \emph{filter} is used to denote a filter
on a set (not on an arbitrary poset) only.


\section{Equivalent filters}
\begin{defn}
\index{equivalent!filters}Two filters $\mathcal{A}$ and $\mathcal{B}$
(with possibly different base sets) are equivalent ($\mathcal{A}\sim\mathcal{B}$)
iff there exists a set $X$ such that $X\in\mathcal{A}$ and $X\in\mathcal{B}$
and $\subsets X\cap\mathcal{A}=\subsets X\cap\mathcal{B}$.\end{defn}
\begin{prop}
If two filters with the same base are equivalent they are equal.\end{prop}
\begin{proof}
Let $\mathcal{A}$ and $\mathcal{B}$ be two filters and $\subsets X\cap\mathcal{A}=\subsets X\cap\mathcal{B}$
for some set $X$ such that $X\in\mathcal{A}$ and $X\in\mathcal{B}$,
and $\Base(\mathcal{A})=\Base(\mathcal{B})$. Then
\begin{multline*}
\mathcal{A}=(\subsets X\cap\mathcal{A})\cup\setcond{Y\in\subsets\Base(\mathcal{A})}{Y\supseteq X}=\\
(\subsets X\cap\mathcal{B})\cup\setcond{Y\in\subsets\Base(\mathcal{B})}{Y\supseteq X}=\mathcal{B}.
\end{multline*}
\end{proof}
\begin{prop}
$\sim$ restricted to small filters is an equivalence relation.\end{prop}
\begin{proof}
~
\begin{description}
\item [{Reflexivity}] Obvious.
\item [{Symmetry}] Obvious.
\item [{Transitivity}] Let $\mathcal{A}\sim\mathcal{B}$ and $\mathcal{B}\sim\mathcal{C}$
for some small filters $\mathcal{A}$, $\mathcal{B}$, and $\mathcal{C}$.
Then there exist a set $X$ such that $X\in\mathcal{A}$ and $X\in\mathcal{B}$
and $\subsets X\cap\mathcal{A}=\subsets X\cap\mathcal{B}$ and a set
$Y$ such that $Y\in\mathcal{B}$ and $Y\in\mathcal{C}$ and $\subsets Y\cap\mathcal{B}=\subsets Y\cap\mathcal{C}$.
So $X\cap Y\in\mathcal{A}$ because
\[
\subsets Y\cap\subsets X\cap\mathcal{A}=\subsets Y\cap\subsets X\cap\mathcal{B}=\subsets(X\cap Y)\cap\mathcal{B}\supseteq\{X\cap Y\}\cap\mathcal{B}\ni X\cap Y.
\]
Similarly we have $X\cap Y\in\mathcal{C}$. Finally
\begin{multline*}
\subsets(X\cap Y)\cap\mathcal{A}=\subsets Y\cap\subsets X\cap\mathcal{A}=\subsets Y\cap\subsets X\cap\mathcal{B}=\\
\subsets X\cap\subsets Y\cap\mathcal{B}=\subsets X\cap\subsets Y\cap\mathcal{C=}\subsets(X\cap Y)\cap\mathcal{C}.
\end{multline*}

\end{description}
\end{proof}
\begin{defn}
\index{filter!rebase}The \emph{rebase} $\mathcal{A}\div A$ for a
filter $\mathcal{A}$ and a set $A$ (base) such that $\exists X\in\mathcal{A}:X\subseteq A$
is defined by the formula
\[
\mathcal{A}\div A=\setcond{X\in\subsets A}{\exists Y\in\mathcal{A}:Y\subseteq X}.
\]
\end{defn}
\begin{prop}
If $\exists X\in\mathcal{A}:X\subseteq A$ then:
\begin{enumerate}
\item \label{div-filt}$\mathcal{A}\div A$ is a filter on $A$;
\item \label{div-eq}$\mathcal{A}\div A\sim\mathcal{A}$.
\end{enumerate}
\end{prop}
\begin{proof}
~
\begin{widedisorder}
\item [{\ref{div-filt}}] We need to prove that $\setcond{X\in\subsets A}{\exists Y\in\mathcal{A}:Y\subseteq X}$
is a filter on $A$. That it is an upper set is obvious. It is non-empty
because $\exists Y\in\mathcal{A}:Y\subseteq A$ and thus $A\in\setcond{X\in\subsets A}{\exists Y\in\mathcal{A}:Y\subseteq X}$.
Let $P,Q\in\setcond{X\in\subsets A}{\exists Y\in\mathcal{A}:Y\subseteq X}$.
Then $P,Q\subseteq A$ and $\exists P'\in\mathcal{A}:P'\subseteq P$
and $\exists Q'\in\mathcal{A}:Q'\subseteq Q$. So $P\cap Q\subseteq A$
and $P'\cap Q'\subseteq P\cap Q$. Thus $P\cap Q\in\setcond{X\in\subsets A}{\exists Y\in\mathcal{A}:Y\subseteq X}$.
\item [{\ref{div-eq}}] ~
\begin{align*}
(\mathcal{A}\div A)\cap\subsets(A\cap\Base(\mathcal{A})) & =\\
\setcond{X\in\subsets A}{\exists Y\in\mathcal{A}:Y\subseteq X}\cap\subsets(A\cap\Base(\mathcal{A})) & =\\
\setcond{X\in\subsets A}{\exists Y\in\mathcal{A}:Y\subseteq X}\cap\subsets\Base(\mathcal{A}) & =\\
\setcond{X\in\subsets(A\cap\Base(\mathcal{A}))}{\exists Y\in\mathcal{A}:Y\subseteq X} & =\\
\mathcal{A}\cap\subsets(A\cap\Base(\mathcal{A})).
\end{align*}
Thus $\mathcal{A}\div A\sim\mathcal{A}$ because $A\cap\Base(\mathcal{A})\supseteq X\in\mathcal{A}$
for some $X\in\mathcal{A}$ and 
\[
A\cap\Base(\mathcal{A})\supseteq X\cap\Base(\mathcal{A})\in\setcond{X\in\subsets A}{\exists Y\in\mathcal{A}:Y\subseteq X}=\mathcal{A}\div A.
\]

\end{widedisorder}
\end{proof}
\begin{prop}
$A\in\mathcal{A}\Rightarrow\mathcal{A}\div A=\subsets A\cap\mathcal{A}$.\end{prop}
\begin{proof}
Let $A\in\mathcal{A}$. Then $\mathcal{A}\div A=\setcond{X\in\subsets A}{\exists Y\in\mathcal{A}:Y\subseteq X}=\setcond{X\in\subsets A}{X\in\mathcal{A}}=\mathscr{P}A\cap\mathcal{A}$.\end{proof}
\begin{lem}
If $\mathcal{A}\sim\mathcal{B}$ then $\exists Y\in\mathcal{A}:Y\subseteq X\Leftrightarrow\exists Y\in\mathcal{B}:Y\subseteq X$
for every filters $\mathcal{A}$, $\mathcal{B}$, and a set $X$.\end{lem}
\begin{proof}
We will prove $\exists Y\in\mathcal{A}:Y\subseteq X\Rightarrow\exists Y\in\mathcal{B}:Y\subseteq X$
(the other direction is similar).

We have $\subsets K\cap\mathcal{A}=\subsets K\cap\mathcal{B}$ for
some set $K$ such that $K\in\mathcal{A}$, $K\in\mathcal{B}$.

\[
\exists Y\in\mathcal{A}:Y\subseteq X\Rightarrow\exists Y\in\subsets K\cap\mathcal{A}:Y\subseteq X\Rightarrow\exists Y\in\subsets K\cap\mathcal{B}:Y\subseteq X\Rightarrow\exists Y\in\mathcal{B}:Y\subseteq X.
\]
\end{proof}
\begin{prop}
If $\mathcal{A}\sim\mathcal{B}$ then $\mathcal{B}=\mathcal{A}\div\Base(\mathcal{B})$
for every filters $\mathcal{A}$,~$\mathcal{B}$.\end{prop}
\begin{proof}
$\subsets Y\cap\mathcal{A}=\subsets Y\cap\mathcal{B}$ for some set
$Y\in\mathcal{A}$, $Y\in\mathcal{B}$. There exists a set $X\in\mathcal{A}$
such that $X\in\mathcal{B}$. Thus $\exists X\in\mathcal{A}:X\subseteq\Base(\mathcal{B})$
and so $\mathcal{A}\div\Base(\mathcal{B})$ is a filter.
\begin{multline*}
X\in\mathcal{A}\div\Base(\mathcal{B})\Leftrightarrow X\in\subsets\Base(\mathcal{B})\land\exists Y\in\mathcal{A}:Y\subseteq X\Leftrightarrow\\
X\in\subsets\Base(\mathcal{B})\land\exists Y\in\mathcal{B}:Y\subseteq X\Leftrightarrow X\in\mathcal{B}
\end{multline*}
(the lemma used).
\end{proof}

\section{Ordering of filters}

Below I will define some categories having filters (with possibly
different bases) as their objects and some relations having two filters
(with possibly different bases) as arguments induced by these categories
(defined as existence of a morphism between these two filters).
\begin{thm}
$\card a=\card U$ for every ultrafilter $a$ on $U$ if $U$ is infinite.\end{thm}
\begin{proof}
Let $f(X)=X$ if $X\in a$ and $f(X)=U\setminus X$ if $X\notin a$.
Obviously $f$ is a surjection from $U$ to $a$.

Every $X\in a$ appears as a value of $f$ exactly twice, as $f(X)$
and $f(U\setminus X)$. So $\card a=(\card U)/2=\card U$.\end{proof}
\begin{cor}
Cardinality of every two ultrafilters on a set $U$ is the same.\end{cor}
\begin{proof}
For infinite $U$ it follows from the theorem. For finite case it
is obvious.\end{proof}
\begin{prop}
$\supfun{\uparrow^{\mathsf{FCD}}f}\mathcal{A}=\setcond{C\in\subsets(\Dst f)}{\rsupfun{f^{-1}}C\in\mathcal{A}}$
for every $\mathbf{Set}$-morphism $f:\Base(\mathcal{A})\rightarrow\Base(\mathcal{B})$.
(Here a funcoid is considered as a pair of functions $\mathfrak{F}(\Base(\mathcal{A}))\rightarrow\mathfrak{F}(\Base(\mathcal{B}))$,
$\mathfrak{F}(\Base(\mathcal{B}))\rightarrow\mathfrak{F}(\Base(\mathcal{A}))$
rather than as a pair of functions $\mathscr{F}(\Base(\mathcal{A}))\rightarrow\mathscr{F}(\Base(\mathcal{B}))$,
$\mathscr{F}(\Base(\mathcal{B}))\rightarrow\mathscr{F}(\Base(\mathcal{A}))$.)\end{prop}
\begin{proof}
For every set $C\in\subsets\Base(\mathcal{B})$ we have
\begin{align*}
\rsupfun{f^{-1}}C\in\mathcal{A} & \Rightarrow\\
\exists K\in\mathcal{A}:\rsupfun{f^{-1}}C=K & \Rightarrow\\
\exists K\in\mathcal{A}:\rsupfun f\rsupfun{f^{-1}}C=\rsupfun fK & \Rightarrow\\
\exists K\in\mathcal{A}:C\supseteq\rsupfun fK & \Leftrightarrow\\
\exists K\in\mathcal{A}:C\in\rsupfun{\uparrow^{\mathsf{FCD}}f}K & \Rightarrow\\
C\in\supfun{\uparrow^{\mathsf{FCD}}f}\mathcal{A}.
\end{align*}
So $C\in \setcond{C\in\subsets(\Dst f)}{\rsupfun{f^{-1}}C\in\mathcal{A}}\Rightarrow C\in\supfun{\uparrow^{\mathsf{FCD}}f}\mathcal{A}$.

Let now $C\in\supfun{\uparrow^{\mathsf{FCD}}f}\mathcal{A}$. Then
$\uparrow\rsupfun{f^{-1}}C\sqsupseteq\supfun{\uparrow^{\mathsf{FCD}}f^{-1}}\supfun{\uparrow^{\mathsf{FCD}}f}\mathcal{A}\sqsupseteq\mathcal{A}$
and thus $\rsupfun{f^{-1}}C\in\mathcal{A}$.
\end{proof}
Below I'll define some directed multigraphs. By an abuse of notation,
I will denote these multigraphs the same as (below defined) categories
based on some of these directed multigraphs with added composition
of morphisms (of directed multigraphs edges). As such I will call
vertices of these multigraphs objects and edges morphisms.
\begin{defn}
I will denote $\mathbf{GreFunc}{}_{1}$ the multigraph whose objects
are filters and whose morphisms between objects $\mathcal{A}$ and
$\mathcal{B}$ are $\mathbf{Set}$-morphisms from $\Base(\mathcal{A})$
to $\Base(\mathcal{B})$ such that $\mathcal{B}\sqsubseteq\supfun{\uparrow^{\mathsf{FCD}}f}\mathcal{A}$.
\end{defn}

\begin{defn}
I will denote $\mathbf{GreFunc}{}_{2}$ the multigraph whose objects
are filters and whose morphisms between objects $\mathcal{A}$ and
$\mathcal{B}$ are $\mathbf{Set}$-morphisms from $\Base(\mathcal{A})$
to $\Base(\mathcal{B})$ such that $\mathcal{B}=\supfun{\uparrow^{\mathsf{FCD}}f}\mathcal{A}$.
\end{defn}

\begin{defn}
Let $\mathcal{A}$ be a filter on a set $X$ and $\mathcal{B}$ be
a filter on a set $Y$. $\mathcal{A}\ge_{1}\mathcal{B}$ iff $\Hom_{\mathbf{GreFunc}_{1}}(\mathcal{A};\mathcal{B})$
is not empty.
\end{defn}

\begin{defn}
Let $\mathcal{A}$ be a filter on a set $X$ and $\mathcal{B}$ be
a filter on a set $Y$. $\mathcal{A}\ge_{2}\mathcal{B}$ iff $\Hom_{\mathbf{GreFunc}_{2}}(\mathcal{A};\mathcal{B})$
is not empty.\end{defn}
\begin{prop}
~
\begin{enumerate}
\item \label{gre-imp}$f\in\Hom_{\mathbf{GreFunc}_{1}}(\mathcal{A};\mathcal{B})$
iff $f$ is a $\mathbf{Set}$-morphism from $\Base(\mathcal{A})$
to $\Base(\mathcal{B})$ such that
\[
C\in\mathcal{B}\Leftarrow\rsupfun{f^{-1}}C\in\mathcal{A}
\]
for every $C\in\subsets\Base(\mathcal{B})$.
\item \label{gre-eq}$f\in\Hom_{\mathbf{GreFunc}_{2}}(\mathcal{A};\mathcal{B})$
iff $f$ is a $\mathbf{Set}$-morphism from $\Base(\mathcal{A})$
to $\Base(\mathcal{B})$ such that
\[
C\in\mathcal{B}\Leftrightarrow\rsupfun{f^{-1}}C\in\mathcal{A}
\]
for every $C\in\subsets\Base(\mathcal{B})$.
\end{enumerate}
\end{prop}
\begin{proof}
~
\begin{widedisorder}
\item [{\ref{gre-imp}}] ~
\begin{multline*}
f\in\Hom_{\mathbf{GreFunc}_{1}}(\mathcal{A};\mathcal{B})\Leftrightarrow\mathcal{B}\sqsubseteq\supfun{\uparrow^{\mathsf{FCD}}f}\mathcal{A}\Leftrightarrow\\
\forall C\in\supfun{\uparrow^{\mathsf{FCD}}f}\mathcal{A}:C\in\mathcal{B}\Leftrightarrow\forall C\in\subsets\Base(\mathcal{B}):(\rsupfun{f^{-1}}C\in\mathcal{A}\Rightarrow C\in\mathcal{B}).
\end{multline*}

\item [{\ref{gre-eq}}] ~
\begin{multline*}
f\in\Hom_{\mathbf{GreFunc}_{2}}(\mathcal{A};\mathcal{B})\Leftrightarrow\mathcal{B}=\supfun{\uparrow^{\mathsf{FCD}}f}\mathcal{A}\Leftrightarrow\forall C:(C\in\mathcal{B}\Leftrightarrow C\in\supfun{\uparrow^{\mathsf{FCD}}f}\mathcal{A})\Leftrightarrow\\
\forall C\in\subsets\Base(\mathcal{B}):(C\in\mathcal{B}\Leftrightarrow C\in\supfun{\uparrow^{\mathsf{FCD}}f}\mathcal{A})\Leftrightarrow\\
\forall C\in\subsets\Base(\mathcal{B}):(\rsupfun{f^{-1}}C\in\mathcal{A}\Leftrightarrow C\in\mathcal{B}).
\end{multline*}

\end{widedisorder}
\end{proof}
\begin{defn}
The directed multigraph $\mathbf{FuncBij}$ is the directed multigraph
got from $\mathbf{GreFunc}_{2}$ by restricting to only bijective
morphisms.
\end{defn}

\begin{defn}
\index{directly isomorphic}A filter $\mathcal{A}$ is \emph{directly
isomorphic} to a filter $\mathcal{B}$ iff there is a morphism $f\in\Hom_{\mathbf{FuncBij}}(\mathcal{A};\mathcal{B})$.\end{defn}
\begin{obvious}
$f\in\Hom_{\mathbf{GreFunc}_{1}}(\mathcal{A};\mathcal{B})\Leftrightarrow\mathcal{B}\sqsubseteq\supfun{\uparrow^{\mathsf{FCD}}f}\mathcal{A}$
for every $\mathbf{Set}$-morphism from $\Base(\mathcal{A})$ to $\Base(\mathcal{B})$.
\end{obvious}

\begin{obvious}
$f\in\Hom_{\mathbf{GreFunc}_{2}}(\mathcal{A};\mathcal{B})\Leftrightarrow\mathcal{B}=\supfun{\uparrow^{\mathsf{FCD}}f}\mathcal{A}$
for every $\mathbf{Set}$-morphism from $\Base(\mathcal{A})$ to $\Base(\mathcal{B})$.
\end{obvious}

\begin{cor}
$\mathcal{A}\ge_{1}\mathcal{B}$ iff it exists a $\mathbf{Set}$-morphism
$f:\Base(\mathcal{A})\rightarrow\Base(\mathcal{B})$ such that $\mathcal{B}\sqsubseteq\supfun{\uparrow^{\mathsf{FCD}}f}\mathcal{A}$.
\end{cor}

\begin{cor}
$\mathcal{A}\ge_{2}\mathcal{B}$ iff it exists a $\mathbf{Set}$-morphism
$f:\Base(\mathcal{A})\rightarrow\Base(\mathcal{B})$ such that $\mathcal{B}=\supfun{\uparrow^{\mathsf{FCD}}f}\mathcal{A}$.\end{cor}
\begin{prop}
For a bijective $\mathbf{Set}$-morphism $f:\Base(\mathcal{A})\rightarrow\Base(\mathcal{B})$
the following are equivalent:
\begin{enumerate}
\item \label{fbij-star}$\mathcal{B}=\setcond{C\in\subsets\Base(\mathcal{B})}{\rsupfun{f^{-1}}C\in\mathcal{A}}$.
\item \label{fbij-eback}$\forall C\in\Base(\mathcal{B}):(C\in\mathcal{B}\Leftrightarrow\rsupfun{f^{-1}}C\in\mathcal{A})$.
\item \label{fbij-eforw}$\forall C\in\Base(\mathcal{A}):(C\in\rsupfun f\mathcal{B}\Leftrightarrow C\in\mathcal{A})$.
\item \label{fbij-rbij}$\supfun{\uparrow^{\mathsf{FCD}}f}|_{\mathcal{A}}$
is a bijection from $\mathcal{A}$ to~$\mathcal{B}$.
\item \label{fbij-rsurj}$\supfun{\uparrow^{\mathsf{FCD}}f}|_{\mathcal{A}}$
is a function onto~$\mathcal{B}$.
\item \label{fbij-feq}$\mathcal{B}=\supfun{\uparrow^{\mathsf{FCD}}f}\mathcal{A}$.
\item \label{fbij-gre}$f\in\Hom_{\mathbf{GreFunc}_{2}}(\mathcal{A};\mathcal{B})$.
\item \label{fbij-grp}$f\in\Hom_{\mathbf{FuncBij}}(\mathcal{A};\mathcal{B})$.
\end{enumerate}
\end{prop}
\begin{proof}
~
\begin{description}
\item [{\ref{fbij-star}$\Leftrightarrow$\ref{fbij-eback}}] ~
\[
\mathcal{B}=\setcond{C\in\subsets\Base(\mathcal{B})}{\rsupfun{f^{-1}}C\in\mathcal{A}}\Leftrightarrow\\
\forall C\in\subsets\Base(\mathcal{B}):(C\in\mathcal{B}\Leftrightarrow\rsupfun{f^{-1}}C\in\mathcal{A}).
\]

\item [{\ref{fbij-eback}$\Leftrightarrow$\ref{fbij-eforw}}] Because
$f$ is a bijection.
\item [{\ref{fbij-eback}$\Rightarrow$\ref{fbij-rsurj}}] For every $C\in\mathcal{B}$
we have $\rsupfun{f^{-1}}C\in\mathcal{A}$ and thus $\supfun{\uparrow^{\mathsf{FCD}}f}|_{\mathcal{A}}\supfun{\uparrow^{\mathsf{FCD}}f^{-1}}C=\rsupfun f\rsupfun{f^{-1}}C=C$.
Thus $\supfun{\uparrow^{\mathsf{FCD}}f}|_{\mathcal{A}}$ is onto $\mathcal{B}$.
\item [{\ref{fbij-rbij}$\Rightarrow$\ref{fbij-rsurj}}] Obvious.
\item [{\ref{fbij-rsurj}$\Rightarrow$\ref{fbij-rbij}}] We need to prove
only that $\supfun{\uparrow^{\mathsf{FCD}}f}|_{\mathcal{A}}$ is an
injection. But this follows from the fact that $f$ is a bijection.
\item [{\ref{fbij-rbij}$\Rightarrow$\ref{fbij-eforw}}] We have $\forall C\in\Base(\mathcal{A}):((\supfun{\uparrow^{\mathsf{FCD}}f}|_{\mathcal{A}})C\in\mathcal{B}\Leftrightarrow C\in\mathcal{A})$
and consequently $\forall C\in\Base(\mathcal{A}):(\rsupfun fC\in\mathcal{B}\Leftrightarrow C\in\mathcal{A})$.
\item [{\ref{fbij-feq}$\Leftrightarrow$\ref{fbij-star}}] From the last
corollary.
\item [{\ref{fbij-star}$\Leftrightarrow$\ref{fbij-gre}}] Obvious.
\item [{\ref{fbij-gre}$\Leftrightarrow$\ref{fbij-grp}}] Obvious.
\end{description}
\end{proof}
\begin{cor}
The following are equivalent for every filters $\mathcal{A}$ and
$\mathcal{B}$:
\begin{enumerate}
\item $\mathcal{A}$ is directly isomorphic to $\mathcal{B}$.
\item There is a bijective $\mathbf{Set}$-morphism $f:\Base(\mathcal{A})\rightarrow\Base(\mathcal{B})$
such that for every $C\in\mathscr{P}\Base(\mathcal{B})$
\[
C\in\mathcal{B}\Leftrightarrow\rsupfun{f^{-1}}C\in\mathcal{A}.
\]

\item There is a bijective $\mathbf{Set}$-morphism $f:\Base(\mathcal{A})\rightarrow\Base(\mathcal{B})$
such that for every $C\in\mathscr{P}\Base(\mathcal{B})$
\[
\rsupfun fC\in\mathcal{B}\Leftrightarrow C\in\mathcal{A}.
\]

\item There is a bijective $\mathbf{Set}$-morphism $f:\Base(\mathcal{A})\rightarrow\Base(\mathcal{B})$
such that $\supfun{\uparrow^{\mathsf{FCD}}f}|_{\mathcal{A}}$ is a
bijection from $\mathcal{A}$ to $\mathcal{B}$.
\item There is a bijective $\mathbf{Set}$-morphism $f:\Base(\mathcal{A})\rightarrow\Base(\mathcal{B})$
such that $\supfun{\uparrow^{\mathsf{FCD}}f}|_{\mathcal{A}}$ is a
function onto $\mathcal{B}$.
\item There is a bijective $\mathbf{Set}$-morphism $f:\Base(\mathcal{A})\rightarrow\Base(\mathcal{B})$
such that $\mathcal{B}=\supfun{\uparrow^{\mathsf{FCD}}f}\mathcal{A}$.
\item There is a bijective morphism $f\in\Hom_{\mathbf{GreFunc}_{2}}(\mathcal{A};\mathcal{B})$.
\item There is a bijective morphism $f\in\Hom_{\mathbf{FuncBij}}(\mathcal{A};\mathcal{B})$.
\end{enumerate}
\end{cor}
\begin{prop}
$\mathbf{GreFunc}_{1}$ and $\mathbf{GreFunc}_{2}$ with function
composition are categories.\end{prop}
\begin{proof}
Let $f:\mathcal{A}\rightarrow\mathcal{B}$ and $g:\mathcal{B}\rightarrow\mathcal{C}$
be morphisms of $\mathbf{GreFunc}_{1}$. Then $\mathcal{B}\sqsubseteq\supfun{\uparrow^{\mathsf{FCD}}f}\mathcal{A}$
and $\mathcal{C}\sqsubseteq\supfun{\uparrow^{\mathsf{FCD}}g}\mathcal{B}$.
So 
\[
\supfun{\uparrow^{\mathsf{FCD}}(g\circ f)}\mathcal{A}=\supfun{\uparrow^{\mathsf{FCD}}g}\supfun{\uparrow^{\mathsf{FCD}}f}\mathcal{A}\sqsupseteq\supfun{\uparrow^{\mathsf{FCD}}g}\mathcal{B}\sqsupseteq\mathcal{C}.
\]
Thus $g\circ f$ is a morphism of $\mathbf{GreFunc}_{1}$. Associativity
law is evident. $\id_{\Base(\mathcal{A})}$ is the identity morphism
of $\mathbf{GreFunc}_{1}$ for every filter~$\mathcal{A}$.

Let $f:\mathcal{A}\rightarrow\mathcal{B}$ and $g:\mathcal{B}\rightarrow\mathcal{C}$
be morphisms of $\mathbf{GreFunc}_{2}$. Then $\mathcal{B}=\supfun{\uparrow^{\mathsf{FCD}}f}\mathcal{A}$
and $\mathcal{C}=\supfun{\uparrow^{\mathsf{FCD}}g}\mathcal{B}$. So
\[
\supfun{\uparrow^{\mathsf{FCD}}(g\circ f)}\mathcal{A}=\supfun{\uparrow^{\mathsf{FCD}}g}\supfun{\uparrow^{\mathsf{FCD}}f}\mathcal{A}=\supfun{\uparrow^{\mathsf{FCD}}g}\mathcal{B}=\mathcal{C}.
\]
Thus $g\circ f$ is a morphism of $\mathbf{GreFunc}_{2}$. Associativity
law is evident. $\id_{\Base(\mathcal{A})}$ is the identity morphism
of $\mathbf{GreFunc}_{2}$ for every filter~$\mathcal{A}$.\end{proof}
\begin{cor}
$\le_{1}$ and $\le_{2}$ are preorders. \end{cor}
\begin{thm}
$\mathbf{FuncBij}$ is a groupoid.\end{thm}
\begin{proof}
First let's prove it is a category. Let $f:\mathcal{A}\rightarrow\mathcal{B}$
and $g:\mathcal{B}\rightarrow\mathcal{C}$ be morphisms of $\mathbf{FuncBij}$.
Then $f:\Base(\mathcal{A})\rightarrow\Base(\mathcal{B})$ and $g:\Base(\mathcal{B})\rightarrow\Base(\mathcal{C})$
are bijections and $\mathcal{B}=\supfun{\uparrow^{\mathsf{FCD}}f}\mathcal{A}$
and $\mathcal{C}=\supfun{\uparrow^{\mathsf{FCD}}g}\mathcal{B}$. Thus
$g\circ f:\Base(\mathcal{A})\rightarrow\Base(\mathcal{C})$ is a bijection
and $\mathcal{C}=\supfun{\uparrow^{\mathsf{FCD}}(g\circ f)}\mathcal{A}$.
Thus $g\circ f$ is a morphism of $\mathbf{FuncBij}$. $\id_{\Base(\mathcal{A})}$
is the identity morphism of $\mathbf{FuncBij}$ for every filter $\mathcal{A}$.
Thus it is a category.

It remains to prove only that every morphism $f\in\Hom_{\mathbf{FuncBij}}(\mathcal{A};\mathcal{B})$
has a reverse (for every filters $\mathcal{A}$, $\mathcal{B}$).
We have $f$ is a bijection $\Base(\mathcal{A})\rightarrow\Base(\mathcal{B})$
such that for every $C\in\subsets\Base(\mathcal{A})$
\[
\rsupfun fC\in\mathcal{B}\Leftrightarrow C\in\mathcal{A}.
\]
Then $f^{-1}:\Base(\mathcal{B})\rightarrow\Base(\mathcal{A})$ is
a bijection such that for every $C\in\subsets\Base(\mathcal{B})$
\[
\rsupfun{f^{-1}}C\in\mathcal{A}\Leftrightarrow C\in\mathcal{B}.
\]
 Thus $f^{-1}\in\Hom_{\mathbf{FuncBij}}(\mathcal{B};\mathcal{A})$.\end{proof}
\begin{cor}
Being directly isomorphic is an equivalence relation.
\end{cor}
\index{order!Rudin-Keisler}Rudin-Keisler order of ultrafilters is
considered in such a book as \cite{comfort-ultra}.
\begin{obvious}
For the case of ultrafilters being directly isomorphic is the same
as being Rudin-Keisler equivalent.\end{obvious}
\begin{defn}
\index{isomorphic!filters}A filter $\mathcal{A}$ is \emph{isomorphic}
to a filter $\mathcal{B}$ iff there exist sets $A\in\mathcal{A}$
and $B\in\mathcal{B}$ such that $\mathcal{A}\div A$ is directly
isomorphic to $\mathcal{B}\div B$.\end{defn}
\begin{obvious}
Equivalent filters are isomorphic.\end{obvious}
\begin{thm}
Being isomorphic (for small filters) is an equivalence relation.\end{thm}
\begin{proof}
~
\begin{description}
\item [{Reflexivity}] Because every filter is directly isomorphic to itself.
\item [{Symmetry}] If filter $\mathcal{A}$ is isomorphic to $\mathcal{B}$
then there exist sets $A\in\mathcal{A}$ and $B\in\mathcal{B}$ such
that $\mathcal{A}\div A$ is directly isomorphic to $\mathcal{B}\div B$
and thus $\mathcal{B}\div B$ is directly isomorphic to $\mathcal{A}\div A$.
So $\mathcal{B}$ is isomorphic to $\mathcal{A}$.
\item [{Transitivity}] Let $\mathcal{A}$ be isomorphic to $\mathcal{B}$
and $\mathcal{B}$ be isomorphic to $\mathcal{C}$. Then exist $A\in\mathcal{A}$,
$B_{1}\in\mathcal{B}$, $B_{2}\in\mathcal{B}$, $C\in\mathcal{C}$
such that there are bijections $f:A\rightarrow B_{1}$ and $g:B_{2}\rightarrow C$
such that
\[
\forall X\in\subsets A:(X\in\mathcal{B}\Leftrightarrow\rsupfun{f^{-1}}X\in\mathcal{A})\quad\text{and}\quad\forall X\in\subsets B_{1}:(X\in\mathcal{A}\Leftrightarrow\rsupfun fX\in\mathcal{B})
\]
and also $\forall X\in\subsets B_{2}:(X\in\mathcal{B}\Leftrightarrow\rsupfun gX\in\mathcal{C})$.


So $g\circ f$ is a bijection from $\rsupfun{f^{-1}}(B_{1}\cap B_{2})\in\mathcal{A}$
to $\rsupfun g(B_{1}\cap B_{2})\in\mathcal{C}$ such that
\[
X\in\mathcal{A}\Leftrightarrow\rsupfun fX\in\mathcal{B}\Leftrightarrow\rsupfun g\rsupfun fX\in\mathcal{C}\Leftrightarrow\rsupfun{g\circ f}X\in\mathcal{C}.
\]
Thus $g\circ f$ establishes a bijection which proves that $\mathcal{A}$
is isomorphic to~$\mathcal{C}$.

\end{description}
\end{proof}
\begin{lem}
Let $\card X=\card Y$, $u$ be an ultrafilter on $X$ and $v$ be
an ultrafilter on $Y$; let $A\in u$ and $B\in v$. Let $u\div A$
and $v\div B$ be directly isomorphic. Then if $\card(X\setminus A)=\card(Y\setminus B)$
we have $u$ and $v$ directly isomorphic.\end{lem}
\begin{proof}
Arbitrary extend the bijection witnessing being directly isomorphic
to the sets $X\setminus A$ and $X\setminus B$.\end{proof}
\begin{thm}
If $\card X=\card Y$ then being isomorphic and being directly isomorphic
are the same for ultrafilters $u$ on $X$ and $v$ on $Y$.\end{thm}
\begin{proof}
That if two filters are isomorphic then they are directly isomorphic
is obvious.

Let ultrafilters $u$ and $v$ be isomorphic that is there is a bijection
$f:A\rightarrow B$ where $A\in u$, $B\in v$ witnessing isomorphism
of $u$ and $v$.

If one of the filters $u$ or $v$ is a trivial ultrafilter then the
other is also a trivial ultrafilter and as it is easy to show they
are directly isomorphic. So we can assume $u$ and $v$ are not trivial
ultrafilters.

If $\card(X\setminus A)=\card(Y\setminus B)$ our statement follows
from the last lemma.

Now assume without loss of generality $\card(X\setminus A)>\card(Y\setminus B)$.

$\card B=\card Y$ because $\card(Y\setminus B)<\card X=\card Y$.

It is easy to show that there exists $B'\supset B$ such that $\card(X\setminus A)=\card(Y\setminus B')$
and $\card B'=\card B$.

We will find a bijection $g$ from $B$ to $B'$ which witnesses direct
isomorphism of $v$ to $v$ itself. Then the composition $g\circ f$
witnesses a direct isomorphism of $u\div A$ and $v\div B'$ and by
the lemma $u$ and $v$ are directly isomorphic.

Let $D=B'\setminus B$. We have $D\notin v$.

There exists a set $E\subseteq B$ such that $\card E\ge\card D$
and $E\notin v$.

We have $\card E=\card(D\cup E)$ and thus there exists a bijection
$h:E\rightarrow D\cup E$.

Let
\[
g(x)=\begin{cases}
x & \text{if }x\in B\setminus E;\\
h(x) & \text{if }x\in E.
\end{cases}
\]


$g|_{B\setminus E}$ and $g|_{E}$ are bijections.

$\im(g|_{B\setminus E})=B\setminus E$; $\im(g|_{E})=\im h=D\cup E$;
\[
(D\cup E)\cap(B\setminus E)=(D\cap(B\setminus E))\cup(E\cap(B\setminus E))=\emptyset\cup\emptyset=\emptyset.
\]
Thus $g$ is a bijection from $B$ to $(B\setminus E)\cup(D\cup E)=B\cup D=B'$.

To finish the proof it's enough to show that $\rsupfun gv=v$. Indeed
it follows from $B\setminus E\in v$.\end{proof}
\begin{prop}
~
\begin{enumerate}
\item \label{ge2-restr}For every $A\in\mathcal{A}$ and $B\in\mathcal{B}$
we have $\mathcal{A}\ge_{2}\mathcal{B}$ iff $\mathcal{A}\div A\ge_{2}\mathcal{B}\div B$.
\item \label{ge1-restr}For every $A\in\mathcal{A}$ and $B\in\mathcal{B}$
we have $\mathcal{A}\ge_{1}\mathcal{B}$ iff $\mathcal{A}\div A\ge_{1}\mathcal{B}\div B$.
\end{enumerate}
\end{prop}
\begin{proof}
~
\begin{widedisorder}
\item [{\ref{ge2-restr}}] $\mathcal{A}\ge_{2}\mathcal{B}$ iff there exist
a bijective $\mathbf{Set}$-morphism $f$ such that $\mathcal{B}=\supfun{\uparrow^{\mathsf{FCD}}f}\mathcal{A}$.
The equality is obviously preserved replacing $\mathcal{A}$ with
$\mathcal{A}\div A$ and $\mathcal{B}$ with $\mathcal{B}\div B$.
\item [{\ref{ge1-restr}}] $\mathcal{A}\ge_{1}\mathcal{B}$ iff there exist
a bijective $\mathbf{Set}$-morphism $f$ such that $\mathcal{B}\subseteq\supfun{\uparrow^{\mathsf{FCD}}f}\mathcal{A}$.
The equality is obviously preserved replacing $\mathcal{A}$ with
$\mathcal{A}\div A$ and $\mathcal{B}$ with $\mathcal{B}\div B$.
\end{widedisorder}
\end{proof}
\begin{prop}
For ultrafilters $\ge_{2}$ is the same as Rudin-Keisler ordering
(as defined in \cite{comfort-ultra}).\end{prop}
\begin{proof}
$x\ge_{2}y$ iff there exist sets $A\in x$ and $B\in y$ a bijective
$\mathbf{Set}$-morphism $f:X\rightarrow Y$ such that
\[
y\div B=\setcond{C\in\subsets Y}{\rsupfun{f^{-1}}C\in x\div A}
\]
 that is when $C\in y\div B\Leftrightarrow\rsupfun{f^{-1}}C\in x\div A$
what is equivalent to~$C\in y\Leftrightarrow\rsupfun{f^{-1}}C\in x$
what is the definition of Rudin-Keisler ordering.\end{proof}
\begin{rem}
\index{Rudin-Keisler equivalence}The relation of being isomorphic
for ultrafilters is traditionally called \emph{Rudin-Keisler equivalence}.\end{rem}
\begin{obvious}
$(\ge_{1})\supseteq(\ge_{2})$.\end{obvious}
\begin{defn}
Let $Q$ and $R$ be binary relations on the set of filters. I will
denote $\mathbf{MonRld}_{Q,R}$ the directed multigraph with objects
being filters and morphisms such monovalued reloids $f$ that $(\dom f)\mathrel Q\mathcal{A}$
and $(\im f)\mathrel R\mathcal{B}$.

I will also denote $\mathbf{CoMonRld}_{Q,R}$ the directed multigraph
with objects being filters and morphisms such injective reloids $f$
that $(\im f)\mathrel Q\mathcal{A}$ and $(\dom f)\mathrel R\mathcal{B}$.
These are essentially the duals.
\end{defn}
Some of these directed multigraphs are categories with reloid composition
(see below). By abuse of notation I will denote these categories the
same as these directed multigraphs.
\begin{thm}
For every filters $\mathcal{A}$ and $\mathcal{B}$ the following
are equivalent:
\begin{enumerate}
\item \label{ge1-ineq}$\mathcal{A}\ge_{1}\mathcal{B}$.
\item \label{ge1-eq-ge}$\Hom_{\mathbf{MonRld}_{=,\sqsupseteq}}(\mathcal{A};\mathcal{B})\ne\emptyset$.
\item \label{ge1-le-ge}$\Hom_{\mathbf{MonRld}_{\sqsubseteq,\sqsupseteq}}(\mathcal{A};\mathcal{B})\ne\emptyset$.
\item \label{ge1-le-eq}$\Hom_{\mathbf{MonRld}_{\sqsubseteq,=}}(\mathcal{A};\mathcal{B})\ne\emptyset$.
\item \label{g1-c-eq-ge}$\Hom_{\mathbf{CoMonRld}_{=,\sqsupseteq}}(\mathcal{A};\mathcal{B})\ne\emptyset$.
\item \label{g1-c-le-ge}$\Hom_{\mathbf{CoMonRld}_{\sqsubseteq,\sqsupseteq}}(\mathcal{A};\mathcal{B})\ne\emptyset$.
\item \label{g1-c-le-eq}$\Hom_{\mathbf{CoMonRld}_{\sqsubseteq,=}}(\mathcal{A};\mathcal{B})\ne\emptyset$.
\end{enumerate}
\end{thm}
\begin{proof}
~
\begin{description}
\item [{\ref{ge1-ineq}$\Rightarrow$\ref{ge1-eq-ge}}] There exists a
$\mathbf{Set}$-morphism $f:\Base(\mathcal{A})\rightarrow\Base(\mathcal{B})$
such that $\mathcal{B}\sqsubseteq\supfun{\uparrow^{\mathsf{FCD}}f}\mathcal{A}$.
We have
\[
\dom(\uparrow^{\mathsf{RLD}}f)|_{\mathcal{A}}=\mathcal{A}\sqcap\top^{\mathscr{F}(\Base(\mathcal{A}))}=\mathcal{A}
\]
and
\[
\im(\uparrow^{\mathsf{RLD}}f)|_{\mathcal{A}}=\im\tofcd(\uparrow^{\mathsf{RLD}}f)|_{\mathcal{A}}=\im(\uparrow^{\mathsf{FCD}}f)|_{\mathcal{A}}=\supfun{\uparrow^{\mathsf{FCD}}f}\mathcal{A}\sqsupseteq\mathcal{B}.
\]
Thus $(\uparrow^{\mathsf{RLD}}f)|_{\mathcal{A}}$ is a monovalued
reloid such that $\dom(\uparrow^{\mathsf{RLD}}f)|_{\mathcal{A}}=\mathcal{A}$
and $\im(\uparrow^{\mathsf{RLD}}f)|_{\mathcal{A}}\sqsupseteq\mathcal{B}$.
\item [{\ref{ge1-eq-ge}$\Rightarrow$\ref{ge1-le-ge},~\ref{ge1-le-eq}$\Rightarrow$\ref{ge1-le-ge},~\ref{g1-c-eq-ge}$\Rightarrow$\ref{g1-c-le-ge},~\ref{g1-c-le-eq}$\Rightarrow$\ref{g1-c-le-ge}}] Obvious.
\item [{\ref{ge1-le-ge}$\Rightarrow$\ref{ge1-ineq}}] We have $\mathcal{B}\sqsubseteq\supfun{\tofcd f}\mathcal{A}$
for a monovalued reloid $f\in\mathsf{RLD}(\Base(\mathcal{A});\Base(\mathcal{B}))$.
Then there exists a $\mathbf{Set}$-morphism $F:\Base(\mathcal{A})\rightarrow\Base(\mathcal{B})$
such that $\mathcal{B}\sqsubseteq\supfun{\uparrow^{\mathsf{FCD}}F}\mathcal{A}$
that is $\mathcal{A}\ge_{1}\mathcal{B}$.
\item [{\ref{g1-c-le-ge}$\Rightarrow$\ref{g1-c-le-eq}}] $\dom f|_{\mathcal{B}}=\mathcal{B}$
and $\im f|_{\mathcal{B}}\sqsupseteq\mathcal{A}$.
\item [{\ref{ge1-eq-ge}$\Leftrightarrow$\ref{g1-c-eq-ge},~\ref{ge1-le-ge}$\Leftrightarrow$\ref{g1-c-le-ge},~\ref{ge1-le-eq}$\Leftrightarrow$\ref{g1-c-le-eq}}] By
duality.
\end{description}
\end{proof}
\begin{thm}
For every filters $\mathcal{A}$ and $\mathcal{B}$ the following
are equivalent:
\begin{enumerate}
\item \label{ge2-in}$\mathcal{A}\ge_{2}\mathcal{B}$.
\item \label{ge2-mon}$\Hom_{\mathbf{MonRld}_{=,=}}(\mathcal{A};\mathcal{B})\ne\emptyset$.
\item \label{ge2-comon}$\Hom_{\mathbf{CoMonRld}_{=,=}}(\mathcal{A};\mathcal{B})\ne\emptyset$.
\end{enumerate}
\end{thm}
\begin{proof}
~
\begin{description}
\item [{\ref{ge2-in}$\Rightarrow$\ref{ge2-mon}}] Let $\mathcal{A}\ge_{2}\mathcal{B}$
that is $\mathcal{B}=\supfun{\uparrow^{\mathsf{FCD}}f}\mathcal{A}$
for some $\mathbf{Set}$-morphism $f:\Base(\mathcal{A})\rightarrow\Base(\mathcal{B})$.
Then $\dom(\uparrow^{\mathsf{RLD}}f)|_{\mathcal{A}}=\mathcal{A}$
and 
\[
\im(\uparrow^{\mathsf{RLD}}f)|_{\mathcal{A}}=\im\tofcd(\uparrow^{\mathsf{RLD}}f)|_{\mathcal{A}}=\im(\uparrow^{\mathsf{FCD}}f)|_{\mathcal{A}}=\supfun{\uparrow^{\mathsf{FCD}}f}\mathcal{A}=\mathcal{B}.
\]
So $(\uparrow^{\mathsf{RLD}}f)|_{\mathcal{A}}$ is a sought for reloid.
\item [{\ref{ge2-mon}$\Rightarrow$\ref{ge2-in}}] By corollary \ref{mv-is-restr}
below, there exists a $\mathbf{Set}$-morphism $F:\Base(\mathcal{A})\rightarrow\Base(\mathcal{B})$
such that $f=(\uparrow^{\mathsf{RLD}}F)|_{\mathcal{A}}$. Thus
\[
\supfun{\uparrow^{\mathsf{FCD}}F}\mathcal{A}=\im(\uparrow^{\mathsf{FCD}}F)|_{\mathcal{A}}=\im\tofcd(\uparrow^{\mathsf{RLD}}F)|_{\mathcal{A}}=\im\tofcd f=\im f=\mathcal{B}.
\]
Thus $\mathcal{A}\ge_{2}\mathcal{B}$ is testified by the morphism
$F$.
\item [{\ref{ge2-mon}$\Leftrightarrow$\ref{ge2-comon}}] By duality.
\end{description}
\end{proof}
\begin{thm}
The following are categories (with reloid composition):
\begin{enumerate}
\item \label{monrld-le-ge}$\mathbf{MonRld}_{\sqsubseteq,\sqsupseteq}$;
\item \label{monrld-le-eq}$\mathbf{MonRld}_{\sqsubseteq,=}$;
\item \label{monrld-eq-eq}$\mathbf{MonRld}_{=,=}$;
\item $\mathbf{CoMonRld}_{\sqsubseteq,\sqsupseteq}$;
\item $\mathbf{CoMonRld}_{\sqsubseteq,=}$;
\item $\mathbf{CoMonRld}_{=,=}$.
\end{enumerate}
\end{thm}
\begin{proof}
We will prove only the first three. The rest follow from duality.
We need to prove only that composition of morphisms is a morphism,
because associativity and existence of identity morphism are evident.
We have:
\begin{widedisorder}
\item [{\ref{monrld-le-ge}}] Let $f\in\Hom_{\mathbf{MonRld}_{\sqsubseteq,\sqsupseteq}}(\mathcal{A};\mathcal{B})$,
$g\in\Hom_{\mathbf{MonRld}_{\sqsubseteq,\sqsupseteq}}(\mathcal{B};\mathcal{C})$.
Then $\dom f\sqsubseteq\mathcal{A}$, $\im f\sqsupseteq\mathcal{B}$,
$\dom g\sqsubseteq\mathcal{B}$, $\im g\sqsupseteq\mathcal{C}$. So
$\dom(g\circ f)\sqsubseteq\mathcal{A}$, $\im(g\circ f)\sqsupseteq\mathcal{C}$
that is $g\circ f\in\Hom_{\mathbf{MonRld}_{\sqsubseteq,\sqsupseteq}}(\mathcal{A};\mathcal{C})$.
\item [{\ref{monrld-le-eq}}] Let $f\in\Hom_{\mathbf{MonRld}_{\sqsubseteq,=}}(\mathcal{A};\mathcal{B})$,
$g\in\Hom_{\mathbf{MonRld}_{\sqsubseteq,=}}(\mathcal{B};\mathcal{C})$.
Then $\dom f\sqsubseteq\mathcal{A}$, $\im f=\mathcal{B}$, $\dom g\sqsubseteq\mathcal{B}$,
$\im g=\mathcal{C}$. So $\dom(g\circ f)\sqsubseteq\mathcal{A}$,
$\im(g\circ f)=\mathcal{C}$ that is $g\circ f\in\Hom_{\mathbf{MonRld}_{\sqsubseteq,=}}(\mathcal{A};\mathcal{C})$.
\item [{\ref{monrld-eq-eq}}] Let $f\in\Hom_{\mathbf{MonRld}_{=,=}}(\mathcal{A};\mathcal{B})$,
$g\in\Hom_{\mathbf{MonRld}_{=,=}}(\mathcal{B};\mathcal{C})$. Then
$\dom f=\mathcal{A}$, $\im f=\mathcal{B}$, $\dom g=\mathcal{B}$,
$\im g=\mathcal{C}$. So $\dom(g\circ f)=\mathcal{A}$, $\im(g\circ f)=\mathcal{C}$
that is $g\circ f\in\Hom_{\mathbf{MonRld}_{=,=}}(\mathcal{A};\mathcal{C})$.
\end{widedisorder}
\end{proof}
\begin{defn}
Let $\mathbf{BijRld}$ be the groupoid of all bijections of the category
of reloid triples. Its objects are filters and its morphisms from
a filter $\mathcal{A}$ to filter $\mathcal{B}$ are monovalued injective
reloids $f$ such that $\dom f=\mathcal{A}$ and $\im f=\mathcal{B}$.\end{defn}
\begin{thm}
Filters $\mathcal{A}$ and $\mathcal{B}$ are isomorphic iff $\Hom_{\mathbf{BijRld}}(\mathcal{A};\mathcal{B})\neq\emptyset$.\end{thm}
\begin{proof}
~
\begin{description}
\item [{$\Rightarrow$}] Let $\mathcal{A}$ and $\mathcal{B}$ be isomorphic.
Then there are sets $A\in\mathcal{A}$, $B\in\mathcal{B}$ and a bijective
$\mathbf{Set}$-morphism $F:A\rightarrow B$ such that $\rsupfun F:\subsets A\cap\mathcal{A}\rightarrow\subsets B\cap\mathcal{B}$
is a bijection.


Obviously $f=(\uparrow^{\mathsf{RLD}}F)|_{\mathcal{A}}$ is monovalued
and injective.
\begin{align*}
\im f & =\\
\bigsqcap\setcond{\uparrow \im G}{G\in(\uparrow^{\mathsf{RLD}}F)|_{\mathcal{A}}} & =\\
\bigsqcap\setcond{\uparrow \im(H\cap F|_{X})}{H\in(\uparrow^{\mathsf{RLD}}F)|_{\mathcal{A}},X\in\mathcal{A}} & =\\
\bigsqcap\setcond{\uparrow \im F|_{P}}{P\in\mathcal{A}} & =\\
\bigsqcap\setcond{\uparrow \rsupfun FP}{P\in\mathcal{A}} & =\\
\bigsqcap\setcond{\uparrow \rsupfun FP}{P\in\subsets A\cap\mathcal{A}} & =\\
\bigsqcap\rsupfun{\uparrow }(\subsets B\cap\mathcal{B}) & =\\
\bigsqcap\rsupfun{\uparrow }\mathcal{B}=\mathcal{B}.
\end{align*}
Thus $\dom f=\mathcal{A}$ and $\im f=\mathcal{B}$.

\item [{$\Leftarrow$}] Let $f$ be a monovalued injective reloid such
that $\dom f=\mathcal{A}$ and $\im f=\mathcal{B}$. Then there exist
a function $F'$ and an injective binary relation $F''$ such that
$F',F''\in f$. Thus $F=F'\cap F''$ is an injection such that $F\in f$.
The function $F$ is a bijection from $A=\dom F$ to $B=\im F$. The
function $\rsupfun F$ is an injection on $\subsets A\cap\mathcal{A}$
(and moreover on $\subsets A$). It's simple to show that $\forall X\in\subsets A\cap\mathcal{A}:\rsupfun FX\in\subsets B\cap\mathcal{B}$
and similarly 
\[
\forall Y\in\subsets B\cap\mathcal{B}:(\rsupfun F)^{-1}Y=\rsupfun{F^{-1}}Y\in\subsets A\cap\mathcal{A}.
\]
Thus $\rsupfun F|_{\subsets A\cap\mathcal{A}}$ is a bijection $\subsets A\cap\mathcal{A}\rightarrow\subsets B\cap\mathcal{B}$.
So filters $\mathcal{A}$ and $\mathcal{B}$ are isomorphic.
\end{description}
\end{proof}
\begin{prop}
$(\ge_{1})=(\sqsupseteq)\circ(\ge_{2})$ (when we limit to small filters).\end{prop}
\begin{proof}
$\mathcal{A}\ge_{1}\mathcal{B}$ iff exists a function $f:\Base(\mathcal{A})\rightarrow\Base(\mathcal{B})$
such that $\mathcal{B}\sqsubseteq\supfun{\uparrow^{\mathsf{FCD}}f}\mathcal{A}$.
But $\mathcal{B}\sqsubseteq\supfun{\uparrow^{\mathsf{FCD}}f}\mathcal{A}$
is equivalent to $\exists\mathcal{B}'\in\mathscr{F}:(\mathcal{B}'\sqsupseteq\mathcal{B}\land\mathcal{B}'=\supfun{\uparrow^{\mathsf{FCD}}f}\mathcal{A})$.
So $\mathcal{A}\ge_{1}\mathcal{B}$ is equivalent to existence of
$\mathcal{B}'\in\mathscr{F}$ such that $\mathcal{B}'\sqsupseteq\mathcal{B}$
and existence of a function $f:\Base(\mathcal{A})\rightarrow\Base(\mathcal{B})$
such that $\mathcal{B}'=\supfun{\uparrow^{\mathsf{FCD}}f}\mathcal{A}$.
This is equivalent to $\mathcal{A}\mathrel{((\sqsupseteq)\circ(\ge_{2}))}\mathcal{B}$.\end{proof}
\begin{prop}
If $a$ and $b$ are ultrafilters then $b\ge_{1}a\Leftrightarrow b\ge_{2}a$.\end{prop}
\begin{proof}
We need to prove only $b\ge_{1}a\Rightarrow b\ge_{2}a$. If $b\ge_{1}a$
then there exists a monovalued reloid $f:\Base(b)\rightarrow\Base(a)$
such that $\dom f=b$ and $\im f\sqsupseteq a$. Then $\im f=\im\tofcd f\in\{\bot^{\mathscr{F}(\Base(a))}\}\cup\atoms^{\mathscr{F}(\Base(a))}$
because $\tofcd f$ is a monovalued funcoid. So $\im f=a$ (taken
into account $a\ne\bot^{\mathscr{F}(\Base(a))}$) and thus $b\ge_{2}a$\@.\end{proof}
\begin{cor}
For atomic filters $\ge_{1}$ is the same as $\ge_{2}$.
\end{cor}
Thus I will write simply $\ge$ for atomic filters.


\subsection{Existence of no more than one monovalued injective reloid for a given
pair of ultrafilters}


\subsubsection{The lemmas}

The lemmas in this section were provided to me by Robert Martin Solovay
in \cite{solovay-on-identity}. They are based on Wistar Comfort's
work.

In this section we will assume $\mu$ is an ultrafilter on a set $I$
and function $f:I\rightarrow I$ has the property $X\in\mu\Leftrightarrow\rsupfun{f^{-1}}X\in\mu$.
\begin{lem}
\label{lem:one-reloid-first}If $X\in\mu$ then $X\cap\rsupfun fX\in\mu$.\end{lem}
\begin{proof}
If $\rsupfun fX\notin\mu$ then $X\subseteq\rsupfun{f^{-1}}\rsupfun fX\notin\mu$
and so $X\notin\mu$. Thus $X\in\mu\land\rsupfun fX\in\mu$ and consequently
$X\cap\rsupfun fX\in\mu$.
\end{proof}
We will say that $x$ is \emph{periodic} when $f^{n}(x)=x$ for some
positive integer $x$. The least such $n$ is called \emph{the period}
of $x$.

Let's define $x\sim y$ iff there exist $i,j\in\mathbb{N}$ such that
$f^{i}(x)=f^{j}(y)$. Trivially it is an equivalence relation. If
$x$ and $y$ are periodic, then $x\sim y$ iff exists $n\in\mathbb{N}$
such that $f^{n}(y)=x$.

Let $A=\setcond{x\in I}{x\text{ is periodic with period}>1}$.

We will show $A\notin\mu$. Let's assume $A\in\mu$.

Let a set $D\subseteq A$ contains (by the axiom of choice) exactly
one element from each equivalence class of $A$ defined by the relation
$\sim$.

Let $\alpha$ be a function $A\rightarrow\mathbb{N}$ defined as follows.
Let $x\in A$. Let $y$ be the unique element of $D$ such that $x\sim y$.
Let $\alpha(x)$ be the least $n\in\mathbb{N}$ such that $f^{n}(y)=x$.

Let $B_{0}=\setcond{x\in A}{\alpha(x)\text{ is even}}$ and $B_{1}=\setcond{x\in A}{\alpha(x)\text{ is odd}}$.

Let $B_{2}=\setcond{x\in A}{\alpha(x)=0}$.
\begin{lem}
$B_{0}\cap\rsupfun fB_{0}\subseteq B_{2}$.\end{lem}
\begin{proof}
If $x\in B_{0}\cap\rsupfun fB_{0}$ then for a minimal even $n$ and
$x=f(x')$ where $f^{m}(y')=x'$ for a minimal even $m$. Thus $f^{n}(y)=f(x')$
thus $y$ and $x'$ laying in the same equivalence class and thus
$y=y'$. So we have $f^{n}(y)=f^{m+1}(y)$. Thus $n\le m+1$ by minimality.

$x'$ lies on an orbit and thus $x'=f^{-1}(x)$ where by $f^{-1}$
I mean step backward on our orbit; $f^{m}(y)=f^{-1}(x)$ and thus
$x'=f^{n-1}(y)$ thus $n-1\ge m$ by minimality or $n=0$.

Thus $n=m+1$ what is impossible for even $n$ and $m$. We have a
contradiction what proves $B_{0}\cap\rsupfun fB_{0}\subseteq\emptyset$.

Remained the case $n=0$, then $x=f^{0}(y)$ and thus $\alpha(x)=0$.\end{proof}
\begin{lem}
$B_{1}\cap\rsupfun fB_{1}=\emptyset$.\end{lem}
\begin{proof}
Let $x\in B_{1}\cap\rsupfun fB_{1}$. Then $f^{n}(y)=x$ for an odd
$n$ and $x=f(x')$ where $f^{m}(y')=x'$ for an odd $m$. Thus $f^{n}(y)=f(x')$
thus $y$ and $x'$ laying in the same equivalence class and thus
$y=y'$. So we have $f^{n}(y)=f^{m+1}(y)$. Thus $n\le m+1$ by minimality.

$x'$ lies on an orbit and thus $x'=f^{-1}(x)$ where by $f^{-1}$
I mean step backward on our orbit;

$f^{m}(y)=f^{-1}(x)$ and thus $x'=f^{n-1}(y)$ thus $n-1\ge m$ by
minimality ($n=0$ is impossible because $n$ is odd).

Thus $n=m+1$ what is impossible for odd $n$ and $m$. We have a
contradiction what proves $B_{1}\cap\rsupfun fB_{1}=\emptyset$.\end{proof}
\begin{lem}
$B_{2}\cap\rsupfun fB_{2}=\emptyset$.\end{lem}
\begin{proof}
Let $x\in B_{2}\cap\rsupfun fB_{2}$. Then $x=y$ and $x'=y$ where
$x=f(x')$. Thus $x=f(x)$ and so $x\notin A$ what is impossible.\end{proof}
\begin{lem}
$A\notin\mu$.\end{lem}
\begin{proof}
Suppose $A\in\mu$.

Since $A\in\mu$ we have $B_{0}\in\mu$ or $B_{1}\in\mu$.

So either $B_{0}\cap\rsupfun fB_{0}\subseteq B_{2}$ or $B_{1}\cap\rsupfun fB_{1}\subseteq B_{2}$.
As such by the lemma \ref{lem:one-reloid-first} we have $B_{2}\in\mu$.
This is incompatible with $B_{2}\cap\rsupfun fB_{2}=\emptyset$. So
we got a contradiction.
\end{proof}
Let $C$ be the set of points $x$ which are not periodic but $f^{n}(x)$
is periodic for some positive $n$.
\begin{lem}
$C\notin\mu$.\end{lem}
\begin{proof}
Let $\beta$ be a function $C\rightarrow\mathbb{N}$ such that $\beta(x)$
is the least $n\in\mathbb{N}$ such that $f^{n}(x)$ is periodic.

Let $C_{0}=\setcond{x\in C}{\beta(x)\text{ is even}}$ and $C_{0}=\setcond{x\in C}{\beta(x)\text{ is odd}}$.

Obviously $C_{j}\cap\rsupfun fC_{j}=\emptyset$ for $j=0,1$. Hence
by lemma \ref{lem:one-reloid-first} we have $C_{0},C_{1}\notin\mu$
and thus $C=C_{0}\cup C_{1}\notin\mu$.
\end{proof}
Let $E$ be the set of $x\in I$ such that for no $n\in\mathbb{N}$
we have $f^{n}(x)$ periodic.
\begin{lem}
Let $x,y\in E$ be such that $f^{i}(x)=f^{j}(y)$ and $f^{i'}(x)=f^{j'}(y)$
for some $i,j,i',j'\in\mathbb{N}$. Then $i-j=i'-j'$.\end{lem}
\begin{proof}
$i\mapsto f^{i}(x)$ is a bijection.

So $y=f^{i-j}(y)$ and $y=f^{i'-j'}(y)$. Thus $f^{i-j}(y)=f^{i'-j'}(y)$
and so $i-j=i'-j'$.\end{proof}
\begin{lem}
$E\notin\mu$.\end{lem}
\begin{proof}
Let $D'\subseteq E$ be a subset of $E$ with exactly one element
from each equivalence class of the relation $\sim$ on $E$.

Define the function $\gamma:E\rightarrow\mathbb{Z}$ as follows. Let
$x\in E$. Let $y$ be the unique element of $D'$ such that $x\sim y$.
Choose $i,j\in\mathbb{N}$ such that $f^{i}(y)=f^{j}(x)$. Let $\gamma(x)=i-j$.
By the last lemma, $\gamma$ is well-defined.

It is clear that if $x\in E$ then $f(x)\in E$ and moreover $\gamma(f(x))=\gamma(x)+1$.

Let $E_{0}=\setcond{x\in E}{\gamma(x)\text{ is even}}$ and $E_{1}=\setcond{x\in E}{\gamma(x)\text{ is odd}}$.

We have $E_{0}\cap\rsupfun fE_{0}=\emptyset\notin\mu$ and hence $E_{0}\notin\mu$.

Similarly $E_{1}\notin\mu$.

Thus $E=E_{0}\cup E_{1}\notin\mu$.\end{proof}
\begin{lem}
$f$ is the identity function on a set in $\mu$.\end{lem}
\begin{proof}
We have shown $A,C,E\notin\mu$. But the points which lie in none
of these sets are exactly points periodic with period $1$ that is
fixed points of $f$. Thus the set of fixed points of $f$ belongs
to the filter $\mu$.
\end{proof}

\subsubsection{The main theorem and its consequences}
\begin{thm}
For every ultrafilter $a$ the morphism $(a;a;\id_{a}^{\mathsf{FCD}})$
is the only
\begin{enumerate}
\item \label{atom-oneiso}monovalued morphism of the category of reloid
triples from $a$ to $a$;
\item injective morphism of the category of reloid triples from $a$ to
$a$;
\item bijective morphism of the category of reloid triples from $a$ to
$a$.
\end{enumerate}
\end{thm}
\begin{proof}
We will prove only \ref{atom-oneiso} because the rest follow from
it.

Let $f$ be a monovalued morphism from $\Base(a)$ to $\Base(a)$.
Then it exists a $\mathbf{Set}$-morphism $F$ such that $F\in f$.
Trivially $\supfun{\uparrow^{\mathsf{FCD}}F}a\sqsupseteq a$ and thus
$\rsupfun FA\in a$ for every $A\in a$. Thus by the lemma we have
that $F$ is the identity function on a set in $a$ and so obviously
$f$ is an identity.\end{proof}
\begin{cor}
For every two atomic filters (with possibly different bases) $\mathcal{A}$
and $\mathcal{B}$ there exists at most one bijective reloid triple
from $\mathcal{A}$ to $\mathcal{B}$.\end{cor}
\begin{proof}
Suppose that $f$ and $g$ are two different bijective reloids from
$\mathcal{A}$ to $\mathcal{B}$. Then $g^{-1}\circ f$ is not the
identity reloid (otherwise $g^{-1}\circ f=\id_{\dom f}^{\mathsf{RLD}}$
and so $f=g$). But $g^{-1}\circ f$ is a bijective reloid (as a composition
of bijective reloids) from $\mathcal{A}$ to $\mathcal{A}$ what is
impossible.
\end{proof}

\section{Rudin-Keisler equivalence and Rudin-Keisler order}
\begin{thm}
Atomic filters $a$ and $b$ (with possibly different bases) are isomorphic
iff $a\ge b\land b\ge a$.\end{thm}
\begin{proof}
Let $a\ge b\land b\ge a$. Then there are a monovalued reloids $f$
and $g$ such that $\dom f=a$ and $\im f=b$ and $\dom g=b$ and
$\im g=a$. Thus $g\circ f$ and $f\circ g$ are monovalued morphisms
from $a$ to $a$ and from $b$ to $b$. By the above we have $g\circ f=\id_{a}^{\mathsf{RLD}}$
and $f\circ g=\id_{b}^{\mathsf{RLD}}$ so $g=f^{-1}$ and $f^{-1}\circ f=\id_{a}^{\mathsf{RLD}}$
and $f\circ f^{-1}=\id_{b}^{\mathsf{RLD}}$. Thus $f$ is an injective
monovalued reloid from $a$ to $b$ and thus $a$ and $b$ are isomorphic.
\end{proof}
The last theorem cannot be generalized from atomic filters to arbitrary
filters, as it's shown by the following example:
\begin{example}
$\mathcal{A}\ge_{1}\mathcal{B}\wedge\mathcal{B}\ge_{1}\mathcal{A}$
but $\mathcal{A}$ is not isomorphic to $\mathcal{B}$ for some filters
$\mathcal{A}$ and $\mathcal{B}$.\end{example}
\begin{proof}
Consider $\mathcal{A}=\uparrow^{\mathbb{R}}[0;1]$ and $\mathcal{B}=\bigsqcap\setcond{\uparrow^{\mathbb{R}}[0;1+\epsilon)}{\epsilon>0}$.
Then the function $f=\mylamdba x{\mathbb{R}}{x/2}$ witnesses both
inequalities $\mathcal{A}\ge_{1}\mathcal{B}$ and $\mathcal{B}\ge_{1}\mathcal{A}$.
But these filters cannot be isomorphic because only one of them is
principal.\end{proof}
\begin{lem}
Let $f_{0}$ and $f_{1}$ be $\mathbf{Set}$-morphisms. Let $f(x;y)=(f_{0}x;f_{1}y)$
for a function $f$. Then
\[
\supfun{\uparrow^{\mathsf{FCD}(\Src f_{0}\times\Src f_{1};\Dst f_{0}\times\Dst f_{1})}}(\mathcal{A}\times^{\mathsf{RLD}}\mathcal{B})=\supfun{\uparrow^{\mathsf{FCD}}f_{0}}\mathcal{A}\times^{\mathsf{RLD}}\supfun{\uparrow^{\mathsf{FCD}}f_{1}}\mathcal{B}.
\]
\end{lem}
\begin{proof}
~
\begin{align*}
\supfun{\uparrow^{\mathsf{FCD}(\Src f_{0}\times\Src f_{1};\Dst f_{0}\times\Dst f_{1})}}(\mathcal{A}\times^{\mathsf{RLD}}\mathcal{B}) & =\\
\supfun{\uparrow^{\mathsf{FCD}(\Src f_{0}\times\Src f_{1};\Dst f_{0}\times\Dst f_{1})}}\bigsqcap\setcond{\uparrow^{\Src f_{0}\times\Src f_{1}}(A\times B)}{A\in\mathcal{A},B\in\mathcal{B}} & =\\
\bigsqcap\setcond{\uparrow^{\Dst f_{0}\times\Dst f_{1}}\rsupfun f(A\times B)}{A\in\mathcal{A},B\in\mathcal{B}} & =\\
\bigsqcap\setcond{\uparrow^{\Dst f_{0}\times\Dst f_{1}}(\rsupfun{f_{0}}A\times\rsupfun{f_{1}}B)}{A\in\mathcal{A},B\in\mathcal{B}} & =\\
\bigsqcap\setcond{\uparrow^{\Dst f_{0}}\rsupfun{f_{0}}A\times\uparrow^{\Dst f_{1}}\rsupfun{f_{1}}B)}{A\in\mathcal{A},B\in\mathcal{B}} & =\text{ (theorem \ref{meet-prod-fcd})}\\
\bigsqcap\setcond{\uparrow^{\Dst f_{0}}\rsupfun{f_{0}}A}{A\in\mathcal{A}}\times^{\mathsf{RLD}}\bigsqcap\setcond{\uparrow^{\Dst f_{1}}\rsupfun{f_{1}}B}{B\in\mathcal{B}} & =\\
\supfun{\uparrow^{\mathsf{FCD}}f_{0}}\mathcal{A}\times^{\mathsf{RLD}}\supfun{\uparrow^{\mathsf{FCD}}f_{1}}\mathcal{B}.
\end{align*}
\end{proof}
\begin{thm}
\label{inj-iso-dom}Let $f$ be a monovalued reloid. Then $\GR f$
is isomorphic to the filter $\dom f$.\end{thm}
\begin{proof}
Let $f$ be a monovalued reloid. There exists a function $F\in\GR f$.
Consider the bijective function $p=\mylamdba x{\dom F}{(x;Fx)}$.

$\rsupfun p\dom F=F$ and consequently
\begin{align*}
\supfun p\dom f & =\\
\bigsqcap_{K\in\up f}^{\mathsf{RLD}}\rsupfun p\dom K & =\\
\bigsqcap_{K\in\up f}^{\mathsf{RLD}}\rsupfun p\dom(K\cap F) & =\\
\bigsqcap_{K\in\up f}^{\mathsf{RLD}}(K\cap F) & =\\
\bigsqcap_{K\in\up f}^{\mathsf{RLD}}K & =f.
\end{align*}
Thus $p$ witnesses that $f$ is isomorphic to the filter $\dom f$.\end{proof}
\begin{cor}
The graph of a monovalued reloid with atomic domain is atomic.
\end{cor}

\begin{cor}
$\id_{\mathcal{A}}^{\mathsf{RLD}}$ is isomorphic to $\mathcal{A}$
for every filter $\mathcal{A}$.\end{cor}
\begin{thm}
There are atomic filters incomparable by Rudin-Keisler order. (Elements~$a$
and~$b$ are \emph{incomparable} when $a\nsqsubseteq b\land b\nsqsubseteq a$.)\end{thm}
\begin{proof}
See \cite{Gryzlov1997151}.\end{proof}
\begin{thm}
$\ge_{1}$ and $\ge_{2}$ are different relations.\end{thm}
\begin{proof}
Consider $a$ is an arbitrary non-empty filter. Then $a\ge_{1}\bot^{\mathscr{F}(\Base(a))}$
but not $a\ge_{2}\bot^{\mathscr{F}(\Base(a))}$.\end{proof}
\begin{prop}
If $a\ge_{2}b$ where $a$ is an ultrafilter then $b$ is also an
ultrafilter.\end{prop}
\begin{proof}
$b=\supfun{\uparrow^{\mathsf{FCD}}f}a$ for some $f:\Base(a)\rightarrow\Base(b)$.
So $b$ is an ultrafilter since $f$ is monovalued.\end{proof}
\begin{cor}
If $a\ge_{1}b$ where $a$ is an ultrafilter then $b$ is also an
ultrafilter or $\bot^{\mathscr{F}(\Base(a))}$.\end{cor}
\begin{proof}
$b\sqsubseteq\supfun{\uparrow^{\mathsf{FCD}}f}a$ for some $f:\Base(a)\rightarrow\Base(b)$.
Therefore $b'=\supfun{\uparrow^{\mathsf{FCD}}f}a$ is an ultrafilter.
From this our statement follows.\end{proof}
\begin{prop}
Principal filters, generated by sets of the same cardinality, are
isomorphic.\end{prop}
\begin{proof}
Let $A$ and $B$ be sets of the same cardinality. Then there are
a bijection $f$ from $A$ to $B$. We have $\rsupfun fA=B$ and thus
$A$ and $B$ are isomorphic.\end{proof}
\begin{prop}
If a filter is isomorphic to a principal filter, then it is also a
principal filter induced by a set with the same cardinality.\end{prop}
\begin{proof}
Let $A$ be a principal filter and $B$ is a filter isomorphic to
$A$. Then there are sets $X\in A$ and $Y\in B$ such that there
are a bijection $f:X\rightarrow Y$ such that $\rsupfun fA=B$.

So $\min B$ exists and $\min B=\rsupfun f\min A$ and thus $B$ is
a principal filter (of the same cardinality as $A$).\end{proof}
\begin{prop}
A filter isomorphic to a non-trivial ultrafilter is a non-trivial
ultrafilter.\end{prop}
\begin{proof}
Let $a$ be a non-trivial ultrafilter and $a$ is isomorphic to $b$.
Then $a\ge_{2}b$ and thus $b$ is an ultrafilter. The filter $b$
cannot be trivial because otherwise $a$ would be also trivial.\end{proof}
\begin{thm}
For an infinite set $U$ there exist $2^{2^{\card U}}$ equivalence
classes of isomorphic ultrafilters.\end{thm}
\begin{proof}
The number of bijections between any two given subsets of $U$ is
no more than $(\card U)^{\card U}=2^{\card U}$. The number of bijections
between all pairs of subsets of $U$ is no more than $2^{\card U}\cdot2^{\card U}=2^{\card U}$.
Therefore each isomorphism class contains at most $2^{\card U}$ ultrafilters.
But there are $2^{2^{\card U}}$ ultrafilters. So there are $2^{2^{\card U}}$
classes.\end{proof}
\begin{rem}
One of the above mentioned equivalence classes contains trivial ultrafilters.\end{rem}
\begin{cor}
There exist non-isomorphic nontrivial ultrafilters on any infinite
set.
\end{cor}

\section{Consequences}
\begin{thm}
\label{triv-atom-prod}The graph of reloid $\mathcal{F}\times^{\mathsf{RLD}}\uparrow^{A}\{a\}$
is isomorphic to the filter $\mathcal{F}$ for every set $A$ and
$a\in A$.\end{thm}
\begin{proof}
From \ref{inj-iso-dom}.\end{proof}
\begin{thm}
If $f$, $g$ are reloids, $f\sqsubseteq g$ and $g$ is monovalued
then $g|_{\dom f}=f$.\end{thm}
\begin{proof}
It's simple to show that $f=\bigsqcup\setcond{f|_{a}}{a\in\atoms^{\mathscr{F}(\Src f)}}$
(use the fact that $k\sqsubseteq f|_{a}$ for some $a\in\atoms^{\mathscr{F}(\Src f)}$
for every $k\in\atoms f$ and the fact that $\mathsf{RLD}(\Src f;\Dst f)$
is atomistic).

Suppose that $g|_{\dom f}\neq f$. Then there exists $a\in\atoms\dom f$
such that $g|_{a}\neq f|_{a}$.

Obviously $g|_{a}\sqsupseteq f|_{a}$.

If $g|_{a}\sqsupset f|_{a}$ then $g|_{a}$ is not atomic (because
$f|_{a}\ne\bot^{\mathsf{RLD}(\Src f;\Dst f)}$) what contradicts to
a theorem above. So $g|_{a}=f|_{a}$ what is a contradiction and thus
$g|_{\dom f}=f$.\end{proof}
\begin{cor}
\label{mv-is-restr}Every monovalued reloid is a restricted principal
monovalued reloid.\end{cor}
\begin{proof}
Let $f$ be a monovalued reloid. Then there exists a function $F\in\GR f$.
So we have
\[
(\uparrow^{\mathsf{RLD}(\Src f;\Dst f)}F)|_{\dom f}=f.
\]
\end{proof}
\begin{cor}
Every monovalued injective reloid is a restricted injective monovalued
principal reloid.\end{cor}
\begin{proof}
Let $f$ be a monovalued injective reloid. There exists a function
$F$ such that $f=(\uparrow^{\mathsf{RLD}(\Src f;\Dst f)}F)|_{\dom f}$.
Also there exists an injection $G\in\up f$.

Thus
\begin{multline*}
f=f\sqcap(\uparrow^{\mathsf{RLD}(\Src f;\Dst f)}G)|_{\dom f}=\\
(\uparrow^{\mathsf{RLD}(\Src f;\Dst f)}F)|_{\dom f}\sqcap(\uparrow^{\mathsf{RLD}(\Src f;\Dst f)}G)|_{\dom f}=\\
(\uparrow^{\mathsf{RLD}(\Src f;\Dst f)}(F\sqcap G))|_{\dom f}.
\end{multline*}
Obviously $F\sqcap G$ is an injection.\end{proof}
\begin{thm}
If a reloid $f$ is monovalued and $\dom f$ is an principal filter
then $f$ is principal.\end{thm}
\begin{proof}
$f$ is a restricted principal monovalued reloid. Thus $f=F|_{\dom f}$
where $F$ is a principal monovalued reloid. Thus $f$ is principal.\end{proof}
\begin{lem}
If a filter $\mathcal{A}$ is isomorphic to a filter $\mathcal{B}$
then if $X$ is a set then there exists a set $Y$ such that $\uparrow^{\Base(\mathcal{A})}X\sqcap\mathcal{A}$
is a filter isomorphic to $\uparrow^{\Base(\mathcal{B})}Y\sqcap\mathcal{B}$.\end{lem}
\begin{proof}
Let $f$ be a monovalued injective reloid such that $\dom f=\mathcal{A}$,
$\im f=\mathcal{B}$.

By proposition \ref{factor-isomor} we have: $\uparrow^{\Base(\mathcal{A})}X\sqcap\mathcal{A}=\mathcal{X}$
where $\mathcal{X}$ is a filter complementive to $\mathcal{A}$.
Let $\mathcal{Y}=\mathcal{A}\setminus\mathcal{X}$.

$\supfun{\tofcd f}\mathcal{X}\sqcap\supfun{\tofcd f}\mathcal{Y}=\bot^{\mathscr{F}(\Base(\mathcal{B}))}$
by injectivity of $f$.

$\supfun{\tofcd f}\mathcal{X}\sqcup\supfun{\tofcd f}\mathcal{Y}=\supfun{\tofcd f}(\mathcal{X}\sqcup\mathcal{Y})=\supfun{\tofcd f}\mathcal{A}=\mathcal{B}$.
So $\supfun{\tofcd f}\mathcal{X}$ is a filter complementive to $\mathcal{B}$.
So by proposition \ref{factor-isomor} there exists a set $Y$ such
that $\supfun{\tofcd f}\mathcal{X}=\uparrow^{\Base(\mathcal{B})}Y\sqcap\mathcal{B}$.

$f|_{\mathcal{X}}$ is obviously a monovalued injective reloid with
$\dom(f|_{\mathcal{X}})=\uparrow^{\Base(\mathcal{A})}X\sqcap\mathcal{A}$
and $\im(f|_{\mathcal{X}})=\uparrow^{\Base(\mathcal{B})}Y\sqcap\mathcal{B}$.
So $\uparrow^{\Base(\mathcal{A})}X\sqcap\mathcal{A}$ is isomorphic
to $\uparrow^{\Base(\mathcal{B})}Y\sqcap\mathcal{B}$.\end{proof}
\begin{example}
$\mathcal{A}\ge_{2}\mathcal{B}\wedge\mathcal{B}\ge_{2}\mathcal{A}$
but $\mathcal{A}$ is not isomorphic to $\mathcal{B}$ for some filters
$\mathcal{A}$ and $\mathcal{B}$.\end{example}
\begin{proof}
(proof idea by \noun{Andreas Blass}, rewritten using reloids by me)

Let $u_{n}$, $h_{n}$ with $n$ ranging over the set $\mathbb{Z}$
be sequences of ultrafilters on $\mathbb{N}$ and functions $\mathbb{N}\rightarrow\mathbb{N}$
such that $\supfun{\uparrow^{\mathsf{FCD}(\mathbb{N};\mathbb{N})}h_{n}}u_{n+1}=u_{n}$
and $u_{n}$ are pairwise non-isomorphic. (See \cite{kleene-degrees}
for a proof that such ultrafilters and functions exist.)

$\mathcal{A}\eqdef\bigsqcup_{n\in\mathbb{Z}}(\uparrow^{\mathbb{Z}}\{n\}\times^{\mathsf{RLD}} u_{2n+1})$;
$\mathcal{B}\eqdef\bigsqcup_{n\in\mathbb{Z}}(\uparrow^{\mathbb{Z}}\{n\}\times^{\mathsf{RLD}} u_{2n})$.

Let the $\mathbf{Set}$-morphisms $f,g:\mathbb{Z}\times\mathbb{N}\rightarrow\mathbb{Z}\times\mathbb{N}$
be defined by the formulas $f(n;x)=(n;h_{2n}x)$ and $g(n;x)=(n-1;h_{2n-1}x)$.

Using the fact that every function induces a complete funcoid and
a lemma above we get:
\begin{align*}
\supfun{\uparrow^{\mathsf{FCD}}f}\mathcal{A} & =\\
\bigsqcup\rsupfun{\supfun{\uparrow^{\mathsf{FCD}}f}}\setcond{\uparrow^{\mathbb{Z}}\{n\}\times^{\mathsf{RLD}} u_{2n+1}}{n\in\mathbb{Z}} & =\\
\bigsqcup\setcond{\uparrow^{\mathbb{Z}}\{n\}\times^{\mathsf{RLD}} u_{2n}}{n\in\mathbb{Z}} & =\\
\mathcal{B}.\\
\supfun{\uparrow^{\mathsf{FCD}}f}\mathcal{B} & =\\
\bigsqcup\rsupfun{\supfun{\uparrow^{\mathsf{FCD}}f}}\setcond{\uparrow^{\mathbb{Z}}\{n\}\times^{\mathsf{RLD}} u_{2n}}{n\in\mathbb{Z}} & =\\
\bigsqcup\rsupfun{\supfun{\uparrow^{\mathsf{FCD}}f}}\setcond{\uparrow^{\mathbb{Z}}\{n-1\}\times^{\mathsf{RLD}} u_{2n-1}}{n\in\mathbb{Z}} & =\\
\bigsqcup\setcond{\uparrow^{\mathbb{Z}}\{n\}\times^{\mathsf{RLD}} u_{2n+1}}{n\in\mathbb{Z}} & =\\
\mathcal{A}.
\end{align*}


It remains to show that $\mathcal{A}$ and $\mathcal{B}$ are not
isomorphic.

Let $X\in\up(\uparrow^{\mathbb{Z}}\{n\}\times^{\mathsf{RLD}}u_{2n+1})$
for some $n\in\mathbb{Z}$. Then if $\uparrow^{\mathbb{Z}\times\mathbb{N}}X\sqcap\mathcal{A}$
is an ultrafilter we have $\uparrow^{\mathbb{Z}\times\mathbb{N}}X\sqcap\mathcal{A}=\uparrow^{\mathbb{Z}}\{n\}\times^{\mathsf{RLD}}u_{2n+1}$
and thus by the theorem \ref{triv-atom-prod} is isomorphic to $u_{2n+1}$.

If $X\notin\up(\uparrow^{\mathbb{Z}}\{n\}\times^{\mathsf{RLD}}u_{2n+1})$
for every $n\in\mathbb{Z}$ then $(\mathbb{Z}\times\mathbb{N})\setminus X\in\up(\uparrow^{\mathbb{Z}}\{n\}\times^{\mathsf{RLD}}u_{2n+1})$
and thus $(\mathbb{Z}\times\mathbb{N})\setminus X\in\up\mathcal{A}$
and thus $\uparrow^{\mathbb{Z}\times\mathbb{N}}X\sqcap\mathcal{A}=\bot^{\mathbb{Z}\times\mathbb{N}}$.

We have also
\begin{multline*}
(\uparrow^{\mathbb{Z}}\{0\}\times^{\mathsf{RLD}}\mathbb{N})\sqcap\mathcal{B}=(\uparrow^{\mathbb{Z}}\{0\}\times^{\mathsf{RLD}}\mathbb{N})\sqcap\bigsqcup\setcond{\uparrow^{\mathbb{Z}}\{n\}\times^{\mathsf{RLD}} u_{2n}}{n\in\mathbb{Z}}=\\
\bigsqcup\setcond{(\uparrow^{\mathbb{Z}}\{0\}\times^{\mathsf{RLD}}\mathbb{N})\sqcap(\uparrow^{\mathbb{Z}}\{n\}\times^{\mathsf{RLD}} u_{2n})}{n\in\mathbb{Z}}=\uparrow^{\mathbb{Z}}\{0\}\times^{\mathsf{RLD}}u_{0}\text{ (an ultrafilter).}
\end{multline*}


Thus every ultrafilter generated as intersecting $\mathcal{A}$ with
a principal filter $\uparrow^{\mathbb{Z}\times\mathbb{N}}X$ is isomorphic
to some $u_{2n+1}$ and thus is not isomorphic to $u_{0}$. By the
lemma it follows that $\mathcal{A}$ and $\mathcal{B}$ are non-isomorphic.
\end{proof}

\subsection{Metamonovalued reloids}
\begin{prop}
$\left(\bigcap G\right)\circ f=\bigcap_{g\in G}(g\circ f)$ for every
function $f$ and a set $G$ of binary relations.\end{prop}
\begin{proof}
~
\begin{align*}
(x;z)\in\left(\bigcap G\right)\circ f & \Leftrightarrow\\
\exists y:(fx=y\land(y;z)\in\bigcap G) & \Leftrightarrow\\
(fx;z)\in\bigcap G & \Leftrightarrow\\
\forall g\in G:(fx;z)\in g & \Leftrightarrow\\
\forall g\in G\exists y:(fx=y\land(y;z)\in g) & \Leftrightarrow\\
\forall g\in G:(x;z)\in g\circ f & \Leftrightarrow\\
(x;z)\in\bigcap_{g\in G}(g\circ f).
\end{align*}
\end{proof}
\begin{lem}
$\left(\bigsqcap G\right)\circ f=\bigsqcap_{g\in G}(g\circ f)$ if
$f$ is a monovalued principal reloid and $G$ is a set of reloids
(with matching sources and destinations).\end{lem}
\begin{proof}
Let $f=\uparrow^{\mathsf{RLD}}\varphi$ for some monovalued $\mathbf{Rel}$-morphism
$\varphi$.

$\left(\bigsqcap G\right)\circ f=\bigsqcap_{g\in\up\bigsqcap G}^{\mathsf{RLD}}(g\circ\varphi)$;
\begin{align*}
\up\bigsqcap_{g\in G}(g\circ f) & =\\
\up\bigsqcap_{g\in G}\bigsqcap_{\Gamma\in\up g}^{\mathsf{RLD}}(\Gamma\circ\varphi) & =\\
\up\bigsqcap\bigcup_{g\in G}\setcond{\uparrow^{\mathsf{RLD}}(\Gamma\circ\varphi)}{\Gamma\in\up g} & =\\
\up\bigsqcap_{\Gamma\in\up\bigsqcap G}^{\mathsf{RLD}}(\Gamma\circ\varphi) & =\\
\up\bigsqcap\setcond{(\Gamma_{0}\circ\varphi)\sqcap\dots\sqcap(\Gamma_{n}\circ\varphi)}{\Gamma_{i}\in\bigcup G\text{ where \ensuremath{i=0,\dots,n} for \ensuremath{n\in\mathbb{N}}}} & =\text{ (proposition above)}\\
\up\bigsqcap\setcond{(\Gamma_{0}\sqcap\dots\sqcap\Gamma_{n})\circ\varphi}{\Gamma_{i}\in\bigcup G\text{ where \ensuremath{i=0,\dots,n} for \ensuremath{n\in\mathbb{N}}}} & =\\
\up\bigsqcap\setcond{\Gamma\circ\varphi}{\Gamma\in\up\bigsqcap G}.
\end{align*}
Thus $\left(\bigsqcap G\right)\circ f=\bigsqcap_{g\in G}(g\circ f)$.\end{proof}
\begin{thm}\label{rld-meta}
~
\begin{enumerate}
\item Monovalued reloids are metamonovalued.
\item Injective reloids are metainjective.
\end{enumerate}
\end{thm}
\begin{proof}
We will prove only the first, as the second is dual.

Let $G$ be a set of reloids and $f$ be a monovalued reloid.

Let $f'$ be a principal monovalued continuation of $f$ (so that
$f=f'|_{\dom f}$).

By the lemma $\left(\bigsqcap G\right)\circ f'=\bigsqcap_{g\in G}(g\circ f')$.
Restricting this equality to $\dom f$ we get: $\left(\bigsqcap G\right)\circ f=\bigsqcap_{g\in G}(g\circ f)$.\end{proof}
\begin{conjecture}
Every metamonovalued reloid is monovalued.\end{conjecture}

