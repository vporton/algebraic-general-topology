\chapter{Boolean funcoids}

\section{Definitions}

I call \emph{boolean funcoids} pointfree funcoids between boolean lattices.

\begin{prop}
  Every boolean funcoid is complete and co-complete.
\end{prop}

\begin{proof}
  It's enough to prove $\supfun{f} \bigsqcup S = \bigsqcup \langle
  \supfun{f} \rangle^{\ast} S$ for a boolean funcoid $f$ for every $S
  \in \subsets \Src f$ whenever both $\bigsqcup S$ and $\bigsqcup
  \langle \supfun{f} \rangle^{\ast} S$ are defined.
  
  Really, $Y \nasymp \supfun{f} \bigsqcup S \Leftrightarrow \bigsqcup S
  \nasymp \supfun{f^{- 1}} Y \Leftrightarrow \exists X \in S : X
  \nasymp \supfun{f^{- 1}} Y \Leftrightarrow \exists X \in S : Y
  \nasymp \supfun{f} X \Leftrightarrow Y \nasymp \bigsqcup \langle
  \supfun{f} \rangle^{\ast} S$ for every $Y \in \Dst f$ and thus
  by proposition~\bookref{bool-sep} we have $\supfun{f} \bigsqcup S = \bigsqcup
  \langle \supfun{f} \rangle^{\ast} S$.
\end{proof}

\begin{thm}
  Let $\mathfrak{A}$, $\mathfrak{B}$ be complete boolean lattices.
  \fxnote{Generalize for complete (separable?) join infinite distributive lattices.}
  
  A function $\alpha \in \mathfrak{B}^{\mathfrak{A}}$ is equal to the
  component $\supfun{f}$ of a pointree funcoid $f \in \mathsf{pFCD}
  (\mathfrak{A}; \mathfrak{B})$ iff $\alpha$ is preserving all joins (= lower
  adjoint).
\end{thm}

\begin{proof}
  Let $\alpha \in \mathfrak{B}^{\mathfrak{A}}$ and preserves all joins. Then
  $\alpha \in \mathscr{F} (\mathfrak{B})^{\mathfrak{A}}$ (We equate principal
  filters of the set $\mathscr{F} \mathfrak{A}$ of filters on $\mathfrak{A}$
  with elements of $\mathfrak{A}$). Thus $\alpha = \rsupfun{g}$
  for some $g \in \mathsf{pFCD} \left( \mathscr{F} \mathfrak{A}; \mathscr{F}
  \mathfrak{B} \right)$.
  
  $\langle g^{- 1} \rangle \in \mathscr{F} (\mathfrak{A})^{\mathscr{F}
  (\mathfrak{B})}$.
  
  Let $y \in \mathfrak{B}$. We need to prove $\langle g^{- 1} \rangle y \in
  \mathfrak{A}$ that is $\bigsqcup S \nasymp \langle g^{- 1} \rangle y
  \Leftrightarrow \exists x \in S : \langle g^{- 1} \rangle y \nasymp x$ for
  every $S \in \subsets \mathfrak{A}$.
  
  Really, $\bigsqcup S \nasymp \langle g^{- 1} \rangle y \Leftrightarrow y
  \nasymp \supfun{g} \bigsqcup S \Leftrightarrow y \nasymp \bigsqcup
  \langle \supfun{g} \rangle^{\ast} S \Leftrightarrow \exists x \in S :
  y \nasymp \supfun{g} x \Leftrightarrow \exists x \in S : \langle g^{-
  1} \rangle y \nasymp x$.
  
  Take $\beta = \langle g^{- 1} \rangle^{\ast}$. We have $\beta \in
  \mathfrak{A}^{\mathfrak{B}}$.
  
  $x \nasymp \beta y \Leftrightarrow x \nasymp \langle g^{- 1} \rangle y
  \Leftrightarrow y \nasymp \supfun{g} x \Leftrightarrow y \nasymp
  \alpha x$.
  
  So $(\mathfrak{A}; \mathfrak{B}; \alpha ; \beta)$ is a pointfree funcoid.
  
  The other direction: Let now $f \in \mathsf{pFCD} (\mathfrak{A};
  \mathfrak{B})$. We need to prove that it preserves all joins. But it was
  proved above.
\end{proof}

\begin{conjecture}
  Let $\mathfrak{A}$, $\mathfrak{B}$ be boolean lattices.
  
  A function $\alpha \in \mathfrak{B}^{\mathfrak{A}}$ is equal to the
  component $\supfun{f}$ of a pointree funcoid $f \in \mathsf{pFCD}
  (\mathfrak{A}; \mathfrak{B})$ iff $\alpha$ is a lower adjoint.
\end{conjecture}

It is tempting to conclude that $\supfun{f}$ is a lower adjoint to
$\supfun{f^{- 1}}$. But that's false:

We should disprove that $Y \sqsubseteq \supfun{f} X \Leftrightarrow X
\sqsubseteq \supfun{f^{- 1}} Y$.

For a counter-example, take $f = \{ 0 \} \times \mathbb{N}$. Then our
condition takes form $1 \Leftrightarrow X \sqsubseteq \{ 0 \}$ for $Y \ni 0$
what obviously does not hold.

Let $p$ be a Galois connection between complete boolean lattices. Then
there exists a funcoid~$f$ such that $p^\ast = \supfun{f}$ and thus there exists
a Galois connection~$q$ such that $q^\ast = \supfun{f^{-1}}$. We have found that
to every Galois connection between two complete boolean lattices corresponds
another Galois connection. TODO: Research it in details.

Galois connections (particularly between boolean lattices) are studied in~\cite{tprod-dist-lat}
and~\cite{shmuely1974}. We should study them in details.

\section{One-element boolean lattice}

Let $\mathfrak{A}$ be a boolean lattice and $\mathfrak{B}= \subsets 0$.
It's sole element is $\bot$.

$f \in \mathsf{pFCD} (\mathfrak{A}; \mathfrak{B}) \Leftrightarrow \forall X \in
\mathfrak{A}: (\supfun{f} X \nasymp \bot \Leftrightarrow \langle f^{-
1} \rangle \bot \nasymp X) \Leftrightarrow \forall X \in \mathfrak{A}: (0
\Leftrightarrow \supfun{f^{- 1}} \bot \nasymp X) \Leftrightarrow
\forall X \in \mathfrak{A}: \supfun{f^{- 1}} \bot \asymp X
\Leftrightarrow \forall X \in \mathfrak{A}: \supfun{f^{- 1}} \bot =
\bot^{\mathfrak{A}} \Leftrightarrow \supfun{f^{- 1}} \bot =
\bot^{\mathfrak{A}} \Leftrightarrow \supfun{f^{- 1}} = \{ (\bot ;
\bot^{\mathfrak{A}}) \}$.

Thus $\card \mathsf{pFCD} (\mathfrak{A}; \subsets 0) = 1$.

\section{Two-element boolean lattice}

Consider the two-element boolean lattice $\mathfrak{B}= \subsets 1$.

Let $f$ be a pointfree protofuncoid from $\mathfrak{A}$ to $\mathfrak{B}$
(that is $(\mathfrak{A};\mathfrak{B};\alpha;\beta)$ where $\alpha\in\mathfrak{B}^{\mathfrak{A}}$, $\beta\in\mathfrak{A}^{\mathfrak{B}}$).

$f \in \mathsf{pFCD} (\mathfrak{A}; \mathfrak{B}) \Leftrightarrow \forall X \in
\mathfrak{A}, Y \in \mathfrak{B}: (\supfun{f} X \nasymp Y
\Leftrightarrow \supfun{f^{- 1}} Y \nasymp X) \Leftrightarrow \forall X
\in \mathfrak{A}, Y \in \mathfrak{B}: ((0 \in \supfun{f} X \wedge 0 \in
Y) \vee (1 \in \supfun{f} X \wedge 1 \in Y) \Leftrightarrow \langle
f^{- 1} \rangle Y \nasymp X)$.

$T = \setcond{ X \in \mathfrak{A} }{ 0 \in \langle f
\rangle X }$ is an ideal. Really: That it's an upper set is obvious.
Let $P \cup Q \in \setcond{ X \in \mathfrak{A} }{ 0 \in
\supfun{f} X }$. Then $0 \in \supfun{f} (P \cup Q) =
\supfun{f} P \cup \supfun{f} Q$; $0 \in \supfun{f} P \vee
0 \in \supfun{f} Q$.

Similarly $S = \setcond{ X \in \mathfrak{A} }{ 1 \in
\supfun{f} X }$ is an ideal.

Let now $T, S \in \subsets \mathfrak{A}$ be ideals. Can we restore $\langle
f \rangle$? Yes, because we know $0 \in \supfun{f} X$ and $1 \in
\supfun{f} X$ for every $X \in \mathfrak{A}$.

So it is equivalent to $\forall X \in \mathfrak{A}, Y \in \mathfrak{B}: ((X
\in T \wedge 0 \in Y) \vee (X \in S \wedge 1 \in Y) \Leftrightarrow \langle
f^{- 1} \rangle Y \nasymp X)$.

$f \in \mathsf{pFCD} (\mathfrak{A}; \mathfrak{B})$ is equivalent to conjunction
of all rows of this table:

\begin{tabular}{|l|l|}
  \hline
  $Y$ & equality \\ \hline\hline
  $\emptyset$ & $\supfun{f^{- 1}} \emptyset = \emptyset$\\
  \{0\} & $X \in T \Leftrightarrow \supfun{f^{- 1}} \{ 0 \} \nasymp
  X$\\ \hline
  \{1\} & $X \in S \Leftrightarrow \supfun{f^{- 1}} \{ 1 \} \nasymp
  X$\\ \hline
  \{0,1\} & $X \in T \vee X \in S \Leftrightarrow \supfun{f^{- 1}} \{
  0, 1 \} \nasymp X$ \\
  \hline
\end{tabular}

Simplified:

\begin{tabular}{|l|l|}
  \hline
  $Y$ & equality \\ \hline\hline
  $\emptyset$ & $\supfun{f^{- 1}} \emptyset = \emptyset$\\ \hline
  \{0\} & $T = \corestar \supfun{f^{- 1}} \{ 0 \}$\\ \hline
  \{1\} & $S = \corestar \supfun{f^{- 1}} \{ 1 \}$\\ \hline
  \{0,1\} & $T \cup S = \corestar \supfun{f^{- 1}} \{ 0, 1 \}$ \\
  \hline
\end{tabular}

From the last table it follows that $T$ and $S$ are principal ideals.

So we can take arbitrary either $\supfun{f^{- 1}} \{ 0 \}$, $\langle
f^{- 1} \rangle \{ 1 \}$ or principal ideals $T$ and $S$.

In other words, we take $\supfun{f^{- 1}} \{ 0 \}$, $\langle f^{- 1}
\rangle \{ 1 \}$ arbitrary and indepently. So we have $\mathsf{pFCD}
(\mathfrak{A}; \mathfrak{B})$ equivalent to disjoint union of two instances of
$\mathfrak{A}$. So it a boolean lattice.

\section{Finite boolean lattices}

We can assume $\mathfrak{B}= \subsets B$ for a set $B$, $\card B =
n$. Then

$f \in \mathsf{pFCD} (\mathfrak{A}; \mathfrak{B}) \Leftrightarrow \forall X \in
\mathfrak{A}, Y \in \mathfrak{B}: (\supfun{f} X \nasymp Y
\Leftrightarrow \supfun{f^{- 1}} Y \nasymp X) \Leftrightarrow \forall X
\in \mathfrak{A}, Y \in \mathfrak{B}: (\exists i \in Y : i \in \langle f
\rangle X \Leftrightarrow \supfun{f^{- 1}} Y \nasymp X)$.

Having values of $\supfun{f^{- 1}} \{ i \}$ we can restore all $\langle
f^{- 1} \rangle Y$. [need this paragraph?]

Let $T_i = \setcond{ X \in \mathfrak{A} }{ i \in
\supfun{f} X }$.

Let now $T_i \in \subsets \mathfrak{A}$ be ideals. Can we restore $\langle
f \rangle$? Yes, because we know $i \in \supfun{f} X$ for every $X \in
\mathfrak{A}$.

So, it is equivalent to:
\begin{equation}
  \label{pfcond-x} \forall X \in \mathfrak{A}, Y \in \mathfrak{B}: (\exists i
  \in Y : X \in T_i \Leftrightarrow \supfun{f^{- 1}} Y \nasymp X) .
\end{equation}
\begin{lem}
  The formula (\ref{pfcond-x}) is equivalent to:
  \begin{equation}
    \label{pfcond-i} \forall X \in \mathfrak{A}, i \in B : (X \in T_i
    \Leftrightarrow \supfun{f^{- 1}} \{ i \} \nasymp X) .
  \end{equation}
\end{lem}

\begin{proof}
  
  \begin{description}
    \item[(\ref{pfcond-x})$\Rightarrow$(\ref{pfcond-i})] Just take $Y = \{ i
    \}$.
    
    \item[(\ref{pfcond-i})$\Rightarrow$(\ref{pfcond-x})] Let (\ref{pfcond-i})
    holds. Let also $X \in \mathfrak{A}, Y \in \mathfrak{B}$. Then $\langle
    f^{- 1} \rangle Y \nasymp X \Leftrightarrow \bigcup_{i \in Y} \langle f^{-
    1} \rangle \{ i \} \nasymp X \Leftrightarrow \exists i \in Y : \langle
    f^{- 1} \rangle \{ i \} \nasymp X \Leftrightarrow \exists i \in Y : X \in
    T_i$.
  \end{description}
\end{proof}

Further transforming: $\forall i \in B : T_i = \corestar \langle f^{- 1}
\rangle \{ i \}$.

So $\supfun{f^{- 1}} \{ i \}$ are arbitary elements of $\mathfrak{B}$
and $T_i$ are corresponding arbitrary principal ideals.

In other words, $\mathsf{pFCD} (\mathfrak{A}; \mathfrak{B}) \cong \mathfrak{A}
\amalg \ldots \amalg \mathfrak{A}$ ($\card B$ times). Thus $\mathsf{pFCD}
(\mathfrak{A}; \mathfrak{B})$ is a boolean lattice.

\section{About infinite case}

Let $\mathfrak{A}$ be a complete boolean lattice, $\mathfrak{B}$ be an
atomistic boolean lattice.

$f \in \mathsf{pFCD} (\mathfrak{A}; \mathfrak{B}) \Leftrightarrow \forall X \in
\mathfrak{A}, Y \in \mathfrak{B}: (\supfun{f} X \nasymp Y
\Leftrightarrow \supfun{f^{- 1}} Y \nasymp X) \Leftrightarrow \forall X
\in \mathfrak{A}, Y \in \mathfrak{B}: (\exists i \in \atoms Y : i \in
\atoms \supfun{f} X \Leftrightarrow \supfun{f^{- 1}} Y
\nasymp X)$.

Let $T_i = \setcond{ X \in \mathfrak{A} }{ i \in
\atoms \supfun{f} X }$.

$T_i$ is an ideal: Really: That it's an upper set is obvious. Let $P \cup Q
\in \setcond{ X \in \mathfrak{A} }{ i \in \atoms
\supfun{f} X }$. Then $i \in \atoms \supfun{f} (P
\cup Q) = \atoms \supfun{f} P \cup \atoms \supfun{f}
Q$; $i \in \supfun{f} P \vee i \in \supfun{f} Q$.

Let now $T_i \in \subsets \mathfrak{A}$ be ideals. Can we restore $\langle
f \rangle$? Yes, because we know $i \in \atoms \supfun{f} X$ for
every $X \in \mathfrak{A}$ and $\mathfrak{B}$ is atomistic.

So, it is equivalent to:
\begin{equation}
  \label{mpfcond-x} \forall X \in \mathfrak{A}, Y \in \mathfrak{B}: (\exists i
  \in \atoms Y : X \in T_i \Leftrightarrow \supfun{f^{- 1}} Y
  \nasymp X) .
\end{equation}
\begin{lem}
  The formula (\ref{mpfcond-x}) is equivalent to:
  \begin{equation}
    \label{mpfcond-i} \forall X \in \mathfrak{A}, i \in
    \atoms^{\mathfrak{B}} : (X \in T_i \Leftrightarrow \langle f^{- 1}
    \rangle i \nasymp X) .
  \end{equation}
\end{lem}

\begin{proof}
  
  \begin{description}
    \item[(\ref{mpfcond-x})$\Rightarrow$(\ref{mpfcond-i})] Let
    (\ref{mpfcond-x}) holds. Take $Y = i$. Then $\atoms Y = \{ i \}$ and
    thus $X \in T_i \Leftrightarrow \exists i \in \atoms Y : X \in T_i
    \Leftrightarrow \supfun{f^{- 1}} Y \nasymp X \Leftrightarrow
    \supfun{f^{- 1}} i \nasymp X$.
    
    \item[(\ref{mpfcond-i})$\Rightarrow$(\ref{mpfcond-x})] Let
    (\ref{pfcond-i}) holds. Let also $X \in \mathfrak{A}$, $Y \in \mathfrak{B}$.
    Then $\supfun{f^{- 1}} Y \nasymp X \Leftrightarrow \langle f^{- 1}
    \rangle \bigsqcup \atoms Y \nasymp X \Leftrightarrow \bigsqcup_{i
    \in \atoms Y} \supfun{f^{- 1}} i \nasymp X \Leftrightarrow
    \exists i \in \atoms Y : \supfun{f^{- 1}} i \nasymp X
    \Leftrightarrow \exists i \in \atoms Y : X \in T_i$.
  \end{description}
\end{proof}

Further equivalently transforming: $\forall i \in \atoms^{\mathfrak{B}}
: T_i = \corestar \supfun{f^{- 1}} i$.

So $\supfun{f^{- 1}} i$ are arbitary elements of $\mathfrak{B}$ and
$T_i$ are corresponding arbitrary principal ideals.

In other words, $\mathsf{pFCD} (\mathfrak{A}; \mathfrak{B}) \cong \coprod_{i \in
\card \atoms^{\mathfrak{B}}} \mathfrak{A}$. Thus $\mathsf{pFCD}
(\mathfrak{A}; \mathfrak{B})$ is a boolean lattice.

So finally we have a very weird theorem, which is a partial solution for the
above open problem (The weirdness is in its partiality and asymmetry):

\begin{thm}
  If $\mathfrak{A}$ is a complete boolean lattice and $\mathfrak{B}$ is an
  atomistic boolean lattice (or vice versa), then $\mathsf{pFCD} (\mathfrak{A};
  \mathfrak{B})$ is a boolean lattice.
\end{thm}

\cite{shmuely1974}~proves
``THEOREM 4.6. Let $A$, $B$ be bounded posets. $A \otimes B$ is a completely
distributive complete Boolean lattice iff $A$ and $B$ are completely distributive
Boolean lattices.'' (where $A \otimes B$ is equivalent to the set of Galois connections between~$A$ and~$B$) and other interesting results.

\section{The diagram}

\begin{thm}
\label{fcd-diagram}Let $A$ and~$B$ be fixed sets. The diagram at the figure~\ref{rels-dia} is
a commutative diagram (in category $\mathbf{Set}$), every arrow in
this diagram is an isomorphism. Every cycle in this diagram is an
identity (therefore ``parallel'' arrows are mutually inverse). The
arrows preserve order, and reversal ($f\mapsto f^{-1}$).
All paths of the diagram (except of paths starting or ending with ``antitone Galois connections'' for which we do not define composition)
preserve composition. For a Galois connection~$f$ I denote $f_0$ the lower adjoint and $f_1$ the upper adjoint.

\begin{figure}[ht]
\caption{\label{rels-dia}}
\begin{tikzcd}[row sep=1cm, column sep=-1.0cm, ampersand replacement=\&]
\& \begin{tabular}{c}binary relations\\ between $A$ and $B$\end{tabular}
\arrow[rd, shift left, "\Psi_1^{-1}"]
\arrow[ld, shift left, "\Psi_2"] \\
\begin{tabular}{c}pointfree funcoids\\ between\\ $\subsets A$ and $\subsets B$\end{tabular}
\arrow[ru, shift left, "\Psi_2^{-1}"]
\arrow[rr, shift left, "\Psi_3"]
\arrow[rd, shift left, "\Psi_4"]
\& \& \begin{tabular}{c}antitone Galois\\ connections\\ between\\ $\subsets A$ and $\subsets B$\end{tabular}
\arrow[lu, shift left, "\Psi_1"]
\arrow[ll, shift left, "\Psi_3^{-1}"]
\arrow[ld, leftrightarrow, "\Psi_5=\Psi_5^{-1}"]
\\
\& \begin{tabular}{c}Galois connections\\ between $\subsets A$ and $\subsets B$\end{tabular}
\arrow[lu, shift left, "\Psi_4^{-1}"]
\end{tikzcd}

\begin{itemize}
\item $\Psi_1:f\mapsto\setcond{(x;y)}{y\in f_0\{x\}}=\setcond{(x;y)}{x\in f_1\{y\}}$
\item $\Psi_1^{-1}:r\mapsto\left(X\mapsto\setcond{y\in B}{\forall x\in X:x\mathrel{r}y}; Y\mapsto\setcond{x\in A}{\forall y\in Y:x\mathrel{r}y}\right)$
\item $\Psi_2:r\mapsto\left(\subsets A;\subsets B;\rsupfun{r};\rsupfun{r^{-1}}\right)$
\item $\Psi_2^{-1}:f\mapsto\suprel{f}$
\item $\Psi_3:f\mapsto\scriptstyle\left(X\mapsto\lnot\supfun{f}X;Y\mapsto\bigsqcap_{y\in\mathscr{T} Y\setminus\{\bot\}}\supfun{f^{-1}}y\right)=
\left(X\mapsto\lnot\supfun{f}X;Y\mapsto\bigsqcap_{y\in Y}\supfun{f^{-1}}\{y\}\right)$
\item $\Psi_3^{-1}:f\mapsto\scriptstyle\left(X\mapsto\lnot f_0 X;Y\mapsto\bigsqcup_{y\in\subsets Y}f_1 y\right)=
  \left(X\mapsto\lnot f_0 X;Y\mapsto\bigsqcup_{y\in Y}f_1\{y\}\right)$
\item $\Psi_4:f\mapsto\scriptstyle\left(\supfun{f};Y\mapsto\bigsqcup_{y\in\mathscr{T}Y\setminus\{\top\}}\supfun{f^{-1}}\lnot y\right)=
  \left(\supfun{f};Y\mapsto\bigsqcup_{y\in Y}\supfun{f^{-1}}\lnot\{y\}\right)$
\item $\Psi_4^{-1}:f\mapsto\left(f_0;Y\mapsto\bigsqcap_{y\in\subsets Y}f_1\lnot y\right)=\left(f_0;Y\mapsto\bigsqcap_{y\in Y}f_1\lnot\{y\}\right)$
\item $\Psi_5=\Psi_5^{-1}:f\mapsto(\mathord{\lnot}\circ f_0;f_1\circ\mathord{\lnot})$
\end{itemize}

\end{figure}
\end{thm}

\begin{proof}
TODO
\end{proof}