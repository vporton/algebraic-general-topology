\chapter{Introduction to funcoids}

In this chapter (and several following chapters) the word \emph{filter}
will refer to a filter (or equivalently any filter object) on a set
(rather than a filter on an arbitrary poset).


\section{\index{funcoid}Informal introduction into funcoids}

Funcoids are a generalization of proximity spaces and a generalization
of pretopological spaces. Also funcoids are a generalization of binary
relations.

That funcoids are a common generalization of ``spaces'' (proximity
spaces, (pre)topological spaces) and binary relations (including monovalued
functions) makes them smart for describing properties of functions
in regard of spaces. For example the statement ``$f$ is a continuous
function from a space $\mu$ to a space $\nu$'' can be described
in terms of funcoids as the formula $f\circ\mu\sqsubseteq\nu\circ f$
(see below for details).

Most naturally funcoids appear as a generalization of proximity spaces.\footnote{In fact I discovered funcoids pondering on topological spaces, not on proximity spaces, but this is only of a historic interest.}

Let $\delta$ be a proximity. We will extend the relation~$\delta$
from sets to filters by the formula:
\[
\mathcal{A}\mathrel\delta'\mathcal{B}\Leftrightarrow\forall
A\in\up\mathcal{A},B\in\up\mathcal{B}:A\mathrel\delta B.
\]


Then (as it will be proved below) there exist two functions
$\alpha,\beta\in\mathscr{F}^{\mathscr{F}}$
such that
\[
\mathcal{A}\mathrel\delta'\mathcal{B}\Leftrightarrow\mathcal{B}
\sqcap\alpha\mathcal{A}\ne\bot^{\mathscr{F}}\Leftrightarrow\mathcal{A}
\sqcap\beta\mathcal{B}\ne\bot^{\mathscr{F}}.
\]


The pair $(\alpha,\beta)$ is called \emph{funcoid} when
$\mathcal{B}\sqcap\alpha\mathcal{A}\ne\bot^{\mathscr{F}}\Leftrightarrow\mathcal{
A}\sqcap\beta\mathcal{B}\ne\bot^{\mathscr{F}}$.
So funcoids are a generalization of proximity spaces.

Funcoids consist of two components the first $\alpha$ and the second
$\beta$. The first component of a funcoid $f$ is denoted as $\supfun f$
and the second component is denoted as $\supfun{f^{-1}}$. (The similarity
of this notation with the notation for the image of a set under a
function is not a coincidence, we will see that in the case of principal
funcoids (see below) these coincide.)

One of the most important properties of a funcoid is that it is uniquely
determined by just one of its components. That is a funcoid $f$ is
uniquely determined by the function $\supfun f$. Moreover a funcoid
$f$ is uniquely determined by values of $\supfun f$ on principal
filters.

Next we will consider some examples of funcoids determined by specified
values of the first component on sets.

Funcoids as a generalization of pretopological spaces: Let $\alpha$
be a pretopological space that is a map \emph{$\alpha\in\mathscr{F}^{\mho}$}
for some set $\mho$. Then we define $\alpha'X=\bigsqcup_{x\in X}\alpha x$
for every set $X\in\subsets\mho$. We will prove that there exists
a unique funcoid $f$ such that $\alpha'=\supfun f|_{\mathfrak{P}}\circ\uparrow$
where $\mathfrak{P}$ is the set of principal filters on $\mho$.
So funcoids are a generalization of pretopological spaces. Funcoids
are also a generalization of preclosure operators: For every preclosure
operator $p$ on a set $\mho$ it exists a unique funcoid $f$ such
that $\supfun f|_{\mathfrak{P}}\circ\uparrow=\uparrow\circ p$.

For every binary relation $p$ on a set $\mho$ there exists unique
funcoid $f$ such that
\[
\forall X\in\subsets\mho:\supfun f\uparrow X=\uparrow\rsupfun pX
\]
(where $\rsupfun p$ is defined in the introduction), recall that
a funcoid is uniquely determined by the values of its first component
on sets. I will call such funcoids \emph{principal}. So funcoids are
a generalization of binary relations.

Composition of binary relations (i.e. of principal funcoids) complies
with the formulas:
\[
\rsupfun{g\circ f}=\rsupfun g\circ\rsupfun f\quad\text{and}\quad\rsupfun{(g\circ
f)^{-1}}=\rsupfun{f^{-1}}\circ\rsupfun{g^{-1}}.
\]
By similar formulas we can define composition of every two funcoids.
Funcoids with this composition form a category (\emph{the category
of funcoids}).

Also funcoids can be reversed (like reversal of $X$ and $Y$ in a
binary relation) by the formula $(\alpha,\beta)^{-1}=(\beta,\alpha)$.
In the particular case if $\mu$ is a proximity we have $\mu^{-1}=\mu$
because proximities are symmetric.

Funcoids behave similarly to (multivalued) functions but acting on
filters instead of acting on sets. Below there will be defined domain
and image of a funcoid (the domain and the image of a funcoid are
filters).


\section{Basic definitions}
\begin{defn}
\index{funcoid}Let us call a \emph{funcoid} from a set $A$ to a
set $B$ a quadruple $(A,B,\alpha,\beta)$ where
$\alpha\in\mathscr{F}(B)^{\mathscr{F}(A)}$,
$\alpha\in\mathscr{F}(A)^{\mathscr{F}(B)}$ such that
\[
\forall\mathcal{X}\in\mathscr{F}(A),\mathcal{Y}\in\mathscr{F}(B):(\mathcal{Y}
\nasymp\alpha\mathcal{X}\Leftrightarrow\mathcal{X}\nasymp\beta\mathcal{Y}).
\]

\end{defn}

\begin{defn}
\index{funcoid!source}\index{funcoid!destination}\emph{Source} and
\emph{destination} of every funcoid $(A,B,\alpha,\beta)$ are defined
as:
\[
\Src(A,B,\alpha,\beta)=A\quad\text{and}\quad\Dst(A,B,\alpha,\beta)=B.
\]

\end{defn}
I will denote $\mathsf{FCD}(A,B)$ the set of funcoids from $A$ to
$B$.

I will denote $\mathsf{FCD}$ the set of all funcoids (for small sets).
\begin{defn}
\index{endo-funcoid}I will call an \emph{endofuncoid} a funcoid whose source is the same as it's destination.
\end{defn}

\begin{defn}
$\supfun{(A,B,\alpha,\beta)}\eqdef\alpha$ for a funcoid $(A,B,\alpha,\beta)$.
\end{defn}

\begin{defn}
\index{funcoid!reverse}The \emph{reverse} funcoid
$(A,B,\alpha,\beta)^{-1}=(B,A,\beta,\alpha)$
for a funcoid $(A,B,\alpha,\beta)$.\end{defn}
\begin{note}
The reverse funcoid is \emph{not} an inverse in the sense of group
theory or category theory.\end{note}
\begin{prop}
If $f$ is a funcoid then $f^{-1}$ is also a funcoid.\end{prop}
\begin{proof}
It follows from symmetry in the definition of funcoid.\end{proof}
\begin{obvious}
$(f^{-1})^{-1}=f$ for a funcoid $f$.\end{obvious}
\begin{defn}
The relation $\mathord{\suprel f}\in\subsets(\mathscr{F}(\Src
f)\times\mathscr{F}(\Dst f))$
is defined (for every funcoid $f$ and $\mathcal{X}\in\mathscr{F}(\Src f)$,
$\mathcal{Y}\in\mathscr{F}(\Dst f)$ by the formula $\mathcal{X}\suprel
f\mathcal{Y}\Leftrightarrow\mathcal{Y}\nasymp\supfun f\mathcal{X}$.\end{defn}
\begin{obvious}
$\mathcal{X}\suprel f\mathcal{Y}\Leftrightarrow\mathcal{Y}\nasymp\supfun
f\mathcal{X}\Leftrightarrow\mathcal{X}\nasymp\supfun{f^{-1}}\mathcal{Y}$
for every funcoid $f$ and $\mathcal{X}\in\mathscr{F}(\Src f)$,
$\mathcal{Y}\in\mathscr{F}(\Dst f)$.
\end{obvious}

\begin{obvious}
$\suprel{f^{-1}}=\suprel f^{-1}$ for a funcoid $f$.\end{obvious}
\begin{thm}
Let $A$, $B$ be sets.
\begin{enumerate}
\item For given value of $\supfun f\in\mathscr{F}(B)^{\mathscr{F}(A)}$
there exists no more than one funcoid $f\in\mathsf{FCD}(A,B)$.
\item For given value of $\mathord{\suprel
f}\in\subsets(\mathscr{F}(A)\times\mathscr{F}(B))$
there exists no more than one funcoid $f\in\mathsf{FCD}(A,B)$.
\end{enumerate}
\end{thm}
\begin{proof}
Let $f,g\in\mathsf{FCD}(A,B)$.

Obviously, $\supfun f=\supfun g\Rightarrow\suprel f=\suprel g$ and
$\supfun{f^{-1}}=\supfun{g^{-1}}\Rightarrow\suprel f=\suprel g$.
So it's enough to prove that $\suprel f=\suprel g\Rightarrow\supfun f=\supfun
g$.

Provided that $\suprel f=\suprel g$ we have $\mathcal{Y}\nasymp\supfun
f\mathcal{X}\Leftrightarrow\mathcal{X}\suprel
f\mathcal{Y}\Leftrightarrow\mathcal{X}\suprel
g\mathcal{Y}\Leftrightarrow\mathcal{Y}\nasymp\supfun g\mathcal{X}$
and consequently $\supfun f\mathcal{X}=\supfun g\mathcal{X}$ for
every $\mathcal{X}\in\mathscr{F}(A)$, $\mathcal{Y}\in\mathscr{F}(B)$
because a set of filters is separable, thus $\supfun f=\supfun g$.\end{proof}
\begin{prop}
$\supfun f\bot=\bot$
for every funcoid $f$.\end{prop}
\begin{proof}
$\mathcal{Y}\nasymp\supfun f\bot\Leftrightarrow\bot\nasymp\supfun{f^{-1}}\mathcal{Y}\Leftrightarrow0\Leftrightarrow\mathcal{Y}
\nasymp\bot$.
Thus $\supfun f\bot=\bot$
by separability of filters.\end{proof}
\begin{prop}
$\supfun f(\mathcal{I}\sqcup\mathcal{J})=\supfun f\mathcal{I}\sqcup\supfun
f\mathcal{J}$
for every funcoid $f$ and $\mathcal{I},\mathcal{J}\in\mathscr{F}(\Src
f)$.\end{prop}
\begin{proof}
~
\begin{align*}
\fullstar\supfun f(\mathcal{I}\sqcup\mathcal{J}) & =\\
\setcond{\mathcal{Y}\in\mathscr{F}}{\mathcal{Y}\nasymp\supfun
f(\mathcal{I}\sqcup\mathcal{J})} & =\\
\setcond{\mathcal{Y}\in\mathscr{F}}{\mathcal{I}\sqcup\mathcal{J}\nasymp\supfun{
f^{-1}}\mathcal{Y}} & =\\
\setcond{\mathcal{Y}\in\mathscr{F}}{\mathcal{I}\nasymp\supfun{f^{-1}}\mathcal{Y}
\lor\mathcal{J}\nasymp\supfun{f^{-1}}\mathcal{Y}} & =\\
\setcond{\mathcal{Y}\in\mathscr{F}}{\mathcal{Y}\nasymp\supfun
f\mathcal{I}\lor\mathcal{Y}\nasymp\supfun f\mathcal{J}} & =\\
\setcond{\mathcal{Y}\in\mathscr{F}}{\mathcal{Y}\nasymp\supfun
f\mathcal{I}\sqcup\supfun f\mathcal{J}} & =\\
\fullstar(\supfun f\mathcal{I}\sqcup\supfun f\mathcal{J}).
\end{align*}


Thus $\supfun f(\mathcal{I}\sqcup\mathcal{J})=\supfun f\mathcal{I}\sqcup\supfun
f\mathcal{J}$
because $\mathscr{F}(\Dst f)$ is separable.\end{proof}
\begin{prop}
For every $f\in\mathsf{FCD}(A,B)$ for every sets $A$ and $B$ we
have:
\begin{enumerate}
\item \label{fcd-f-d1}$\mathcal{K}\suprel
f\mathcal{I}\sqcup\mathcal{J}\Leftrightarrow\mathcal{K}\suprel
f\mathcal{I}\lor\mathcal{K}\suprel f\mathcal{J}$
for every $\mathcal{I},\mathcal{J}\in\mathscr{F}(B)$,
$\mathcal{K}\in\mathscr{F}(A)$.
\item \label{fcd-f-d2}$\mathcal{I}\sqcup\mathcal{J}\suprel
f\mathcal{K}\Leftrightarrow\mathcal{I}\suprel f\mathcal{K}\lor\mathcal{J}\suprel
f\mathcal{K}$
for every $\mathcal{I},\mathcal{J}\in\mathscr{F}(A)$,
$\mathcal{K}\in\mathscr{F}(B)$.
\end{enumerate}
\end{prop}
\begin{proof}
~
\begin{disorder}
\item [{\ref{fcd-f-d1}}] ~
\begin{align*}
\mathcal{K}\suprel f\mathcal{I}\sqcup\mathcal{J} & \Leftrightarrow\\
(\mathcal{I}\sqcup\mathcal{J})\sqcap\supfun f\mathcal{K}\ne\bot^{\mathscr{F}(B)}
& \Leftrightarrow\\
\mathcal{I}\sqcap\supfun
f\mathcal{K}\ne\bot^{\mathscr{F}(B)}\lor\mathcal{J}\sqcap\supfun
f\mathcal{K}\ne\bot^{\mathscr{F}(B)} & \Leftrightarrow\\
\mathcal{K}\suprel f\mathcal{I}\lor\mathcal{K}\suprel f\mathcal{J}.
\end{align*}

\item [{\ref{fcd-f-d2}}] Similar.
\end{disorder}
\end{proof}

\subsection{Composition of funcoids}
\begin{defn}
\index{composable!funcoids}\index{funcoids!composable}Funcoids $f$
and $g$ are \emph{composable} when $\Dst f=\Src g$.
\end{defn}

\begin{defn}
\index{composition!funcoids}\index{funcoids!composition}\emph{Composition}
of composable funcoids is defined by the formula
\[
(B,C,\alpha_{2},\beta_{2})\circ(A,B,\alpha_{1},\beta_{1})=(A,C,\alpha_{2}
\circ\alpha_{1},\beta_{1}\circ\beta_{2}).
\]
\end{defn}
\begin{prop}
If $f$, $g$ are composable funcoids then $g\circ f$ is a funcoid.\end{prop}
\begin{proof}
Let $f=(A,B,\alpha_{1},\beta_{1})$, $g=(B,C,\alpha_{2},\beta_{2})$.
For every $\mathcal{X}\in\mathscr{F}(A)$, $\mathcal{Y}\in\mathscr{F}(C)$
we have
\[
\mathcal{Y}\nasymp(\alpha_{2}\circ\alpha_{1})\mathcal{X}\Leftrightarrow\mathcal{
Y}\nasymp\alpha_{2}\alpha_{1}\mathcal{X}\Leftrightarrow\alpha_{1}\mathcal{X}
\nasymp\beta_{2}\mathcal{Y}\Leftrightarrow\mathcal{X}\nasymp\beta_{1}\beta_{2}
\mathcal{Y}\Leftrightarrow\mathcal{X}\nasymp(\beta_{1}\circ\beta_{2})\mathcal{Y}
.
\]


So $(A,C,\alpha_{2}\circ\alpha_{1},\beta_{1}\circ\beta_{2})$ is a
funcoid.\end{proof}
\begin{obvious}
$\supfun{g\circ f}=\supfun g\circ\supfun f$ for every composable
funcoids $f$ and $g$.\end{obvious}
\begin{prop}
$(h\circ g)\circ f=h\circ(g\circ f)$ for every composable funcoids
$f$, $g$, $h$.\end{prop}
\begin{proof}
~
\begin{align*}
\supfun{(h\circ g)\circ f} & =\\
\supfun{h\circ g}\circ\supfun f & =\\
(\supfun h\circ\supfun g)\circ\supfun f & =\\
\supfun h\circ(\supfun g\circ\supfun f) & =\\
\supfun h\circ\supfun{g\circ f} & =\\
\supfun{h\circ(g\circ f)}.
\end{align*}
\end{proof}
\begin{thm}
$(g\circ f)^{-1}=f^{-1}\circ g^{-1}$ for every composable funcoids
$f$ and $g$.\end{thm}
\begin{proof}
$\supfun{(g\circ
f)^{-1}}=\supfun{f^{-1}}\circ\supfun{g^{-1}}=\supfun{f^{-1}\circ g^{-1}}$.
\end{proof}

\section{Identity funcoids}

\begin{defn}
\index{funcoid!identity}Let $A$ be a set. The \emph{identity funcoid}
$1_{A}^{\mathsf{FCD}}=(A,A,\id_{\mathscr{F}(A)},\id_{\mathscr{F}(A)})$.
\end{defn}
\begin{obvious}
The identity funcoid is a funcoid.\end{obvious}
\begin{prop}
$\mathord{\suprel f}=\mathord{\suprel{1_{\Dst f}}}\circ\supfun f$
for every funcoid $f$.\end{prop}
\begin{proof}
From proposition~\ref{comp-fcd-r}.\end{proof}
\begin{defn}
\index{funcoid!restricted identity}Let $A$ be a set,
$\mathcal{A}\in\mathscr{F}(A)$.
The \emph{restricted identity funcoid
\[
\id_{\mathcal{A}}^{\mathsf{FCD}}=(A,A,\mathcal{A}\sqcap,\mathcal{A}\sqcap).
\]
}\end{defn}
\begin{prop}
The restricted identity funcoid is a funcoid.\end{prop}
\begin{proof}
We need to prove that
$(\mathcal{A}\sqcap\mathcal{X})\sqcap\mathcal{Y}\ne\bot
\Leftrightarrow(\mathcal{A}\sqcap\mathcal{Y})\sqcap\mathcal{X}\ne\bot$
what is obvious.\end{proof}

\begin{obvious}
~
\begin{enumerate}
\item $(1_{A}^{\mathsf{FCD}})^{-1}=1_{A}^{\mathsf{FCD}}$;
\item
$(\id_{\mathcal{A}}^{\mathsf{FCD}})^{-1}=\id_{\mathcal{A}}^{\mathsf{FCD}}$.
\end{enumerate}
\end{obvious}

\begin{obvious}
For every $\mathcal{X},\mathcal{Y}\in\mathscr{F}(A)$
\begin{enumerate}
\item
$\mathcal{X}\suprel{1_{A}^{\mathsf{FCD}}}\mathcal{Y}\Leftrightarrow\mathcal{X}
\sqcap\mathcal{Y}\ne\bot$;
\item
$\mathcal{X}\suprel{\id_{\mathcal{A}}^{\mathsf{FCD}}}\mathcal{Y}
\Leftrightarrow\mathcal{A}\sqcap\mathcal{X}\sqcap\mathcal{Y}\ne\bot$.
\end{enumerate}
\end{obvious}
\begin{defn}
\index{restricting!funcoid}I will define \emph{restricting} of a
funcoid $f$ to a filter $\mathcal{A}\in\mathscr{F}(\Src f)$ by the
formula
\[
f|_{\mathcal{A}}=f\circ\id_{\mathcal{A}}^{\mathsf{FCD}}.
\]
\end{defn}

