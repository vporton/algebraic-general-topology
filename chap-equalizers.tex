\chapter{Equalizers and co-Equalizers in Certain Categories}

It is a rough draft. Errors are possible.

\fxwarning{Change notation $\prod$ $\rightarrow$ $\prod^{(L)}$.}

\section{Categories with embeddings}

\begin{note}
  This section in not used below, it is just to feed your intuition.
\end{note}

The following generalizes the well known concept of embedding function $A
\hookrightarrow B$ for from a set $A$ to a set $B$ where $A \subseteq B$.

I will set that the unique morphism from an object $A$ to an object $B$ of a
thin category is equal to the pair $(A ; B)$.

\begin{defn}
  A \emph{category with embeddings of objects} is a dagger category with a
  preorder of the set of objects together with a functor $\hookrightarrow$ (we
  will denote applying this functor to the object $(A ; B)$ as $A
  \hookrightarrow B$.) such that:
  \begin{itemize}
    \item $\hookrightarrow$ is an identity on objects.
    
    \item Every $A \hookrightarrow B$ is a monomorphism.
    
    \item $(A \hookrightarrow B)^{\dagger} \circ (A \hookrightarrow B) = 1_A$.
  \end{itemize}
\end{defn}

\begin{obvious}
$A\hookrightarrow B$ is defined when $(A ; B)$ is a morphism of the preorder
that is when $A \sqsubseteq B$.
\end{obvious}

\begin{obvious}
$A \hookrightarrow B : A \rightarrow B$ when $A \sqsubseteq B$.
\end{obvious}

\begin{prop}
  $A \hookrightarrow A = 1_A$.
\end{prop}

\begin{proof}
  Because $(A ; A)$ is an identity morphism and $\hookrightarrow$ preserves
  identities.
\end{proof}

\begin{prop}
  $(B \hookrightarrow C) \circ (A \hookrightarrow B) = A \hookrightarrow C$
  whenever $A \sqsubseteq B \sqsubseteq C$.
\end{prop}

\begin{proof}
  $(B \hookrightarrow C) \circ (A \hookrightarrow B) = \hookrightarrow (B ; C)
  \circ \hookrightarrow (A ; B) = \hookrightarrow ((B ; C) \circ (A ; B)) =
  \hookrightarrow (A ; C) = A \hookrightarrow C$.
\end{proof}

\section{\texorpdfstring{Categories under $\mathbf{Rel}$}{Categories under Rel}}

\begin{defn}
  The $\mathbf{Rel}$-morphism $\mathcal{E}^{A,B}$
  (\emph{restriction-embedding}) is defined by the formula: $\mathcal{E}^{A,B}
  = (A ; B ; \id_{A \cap B})$.
  
  When $A$ is clear from context, I will denote it just as $\mathcal{E}^B$.
\end{defn}

\begin{obvious}
If $A \subseteq B$ then $\mathcal{E}^{A,B}$ is an embedding $A \hookrightarrow B
= (A ; B ; \id_A)$.
\end{obvious}

\begin{obvious}
If $A \supseteq B$ then $\mathcal{E}^{A,B} = (A ; B ;
\id_B)$.
\end{obvious}

\begin{obvious}
$\mathcal{E}^{A,A} = 1^{\mathbf{Rel}}_A$.
\end{obvious}

\begin{obvious}
$(\mathcal{E}^{A,B})^{- 1} = \mathcal{E}^{B,A}$.
\end{obvious}

\begin{defn}
\emph{Dagger functor} between two dagger categories is a functor between
these categories, which commutes with the daggers.
\fxwarning{Clearer wording.}
\end{defn}

\begin{defn}
\emph{Category under $\mathbf{Rel}$} is a pair $(C ; \uparrow)$
where $C$ is a category whose objects are small sets and $\uparrow$ is an
identity-on-objects functor $\mathbf{Rel} \rightarrow C$. I call
$\uparrow$ \emph{up-arrow functor}.
\end{defn}

\begin{defn}
  \emph{Dagger category under $\mathbf{Rel}$} is a pair $(C ;
  \uparrow)$ where $C$ is a dagger category whose objects are small sets and
  $\uparrow$ is a dagger identity-on-objects functor $\mathbf{Rel}
  \rightarrow C$.
\end{defn}

\begin{defn}
  $\mathcal{E}_{\mathcal{C}}^{A,B} = \uparrow \mathcal{E}^{A,B}$. In
  other words, $\mathcal{E}_{\mathcal{C}} = \uparrow \circ \mathcal{E}$.
  
  When $A$ is clear from context, I will denote it just as $\mathcal{E}_{\mathcal{C}}^B$.
\end{defn}

\begin{prop}
  $\mathcal{E}_{\mathcal{C}}^{A,A} = 1_{\mathcal{C}}^A$.
\end{prop}

\begin{proof}
  $\mathcal{E}_{\mathcal{C}}^{A,A} = \uparrow \mathcal{E}^{A,A} =
  \uparrow 1_{\mathbf{Rel}} = 1_{\mathcal{C}}^A$.
\end{proof}

\begin{prop}
  If $f : X \rightarrow Y$ is a $\mathbf{Rel}$-morphism and
  $\im f = A \subseteq Y$ then
  \[ \mathcal{E}^{A,Y} \circ \mathcal{E}^{Y,A} \circ f = f. \]
\end{prop}

\begin{proof}
  $\mathcal{E}^{A,Y} \circ \mathcal{E}^{Y,A} \circ f = \id_A
  \circ f = f$.
\end{proof}

\begin{cor}
  If $f : X \rightarrow Y$ is a morphism of a category under
  $\mathbf{Rel}$ and $\im f = A \subseteq Y$, then
  \[ \mathcal{E}_{\mathcal{C}}^{A,Y} \circ \mathcal{E}_{\mathcal{C}}^{Y,A}
  \circ \uparrow f = \uparrow f. \]
\end{cor}

\begin{prop}
  ~  
  \begin{enumerate}
    \item If $A \subseteq B$ then $\mathcal{E}_{\mathcal{C}}^{A,B}$ is a
    monomorphism.
    
    \item If $A \supseteq B$ then $\mathcal{E}_{\mathcal{C}}^{A,B}$ is a
    epimorphism.
  \end{enumerate}
\end{prop}

\begin{proof}
  We'll prove only the first as the second is dual.
  
  Let $\mathcal{E}_{\mathcal{C}}^{A,B} \circ f = \mathcal{E}^{A,B} \circ g$. Then
  $\mathcal{E}^{B,A} \circ \mathcal{E}_{\mathcal{C}}^{A,B}
  \circ f = \mathcal{E}_{\mathcal{C}}^{B,A} \circ \mathcal{E}_{\mathcal{C}}^{A,B} \circ g$;
  $1^A \circ f = 1^A \circ g$; $f = g$.
\end{proof}

\begin{prop}
  $\mathcal{E}^{B,C} \circ \mathcal{E}^{A,B} = \mathcal{E}^{A,C}$
  iff $B \supseteq A \cap C$ (for every sets $A$, $B$, $C$).
\end{prop}

\begin{proof}
  $\mathcal{E}^{B,C} \circ \mathcal{E}^{A,B} = \mathcal{E}^{A,C}$
  is equivalent to:
  
  $(B ; C ; \id_{B \cap C}) \circ (A ; B ; \id_{A \cap B}) = (A ;
  C ; \id_{A \cap C})$;
  
  $(A ; C ; \id_{A \cap B \cap C}) = (A ; C ; \id_{A \cap C})$;
  
  $A \cap B \cap C = A \cap C$;
  
  $B \supseteq A \cap C$.
\end{proof}

\begin{cor}
  $\mathcal{E}^{B,C} \circ \mathcal{E}^{A,B} = \mathcal{E}_{\mathcal{C}}^{A,C}$ if
  $B \supseteq A \cap C$ (for every sets $A$, $B$, $C$).
\end{cor}

\begin{defn}
  \emph{Partially ordered dagger category under $\mathbf{Rel}$} is
  a category which is both a partially ordered dagger category and a category
  under $\mathbf{Rel}$ such that $\uparrow \circ f^{- 1} = (\uparrow
  \circ f)^{\dagger}$ and $A \sqsubseteq B \Rightarrow \uparrow A \sqsubseteq
  \uparrow B$.
\end{defn}

\begin{prop}
  $(\mathcal{E}_{\mathcal{C}}^{A,B})^{\dagger} = \mathcal{E}_{\mathcal{C}}^{B,A}$ for a dagger category under
  $\mathbf{Rel}$.
\end{prop}

\begin{proof}
  $(\mathcal{E}_{\mathcal{C}}^{A,B})^{\dagger} = (\uparrow \mathcal{E}^{A,B})^{\dagger} = \uparrow (\mathcal{E}^{A,B})^{- 1} =
  \uparrow \mathcal{E}^{B,A} = \mathcal{E}_{\mathcal{C}}^{B,A}$.
\end{proof}

\begin{prop}
  For a partially ordered dagger category $\mathcal{C}$ under
  $\mathbf{Rel}$ we have $\mathcal{E}_{\mathcal{C}}^{A,B}$ is:
  \begin{enumerate}
    \item monovalued;
    
    \item injective;
    
    \item entirely defined if $A \subseteq B$;
    
    \item surjective if $B \subseteq A$.
  \end{enumerate}
\end{prop}

\begin{proof}
  ~
  \begin{enumerate}
    \item $\mathcal{E}^{A,B} \circ \mathcal{E}^{B,A} \sqsubseteq
    1^{\mathbf{Rel}}_B$; $\mathcal{E}^{A,B} \circ (\mathcal{E}^{A,B})^{- 1} \sqsubseteq 1^{\mathbf{Rel}}_B$;
    $\mathcal{E}_{\mathcal{C}}^{A,B} \circ (\mathcal{E}_{\mathcal{C}}^{A,B})^{\dagger} \sqsubseteq 1^{\mathcal{C}}_B$.
    
    \item $\mathcal{E}^{B,A} \circ \mathcal{E}^{A,B} \sqsubseteq
    1^{\mathbf{Rel}}_A$; $(\mathcal{E}^{A,B})^{- 1} \circ \mathcal{E}^{A,B} \sqsubseteq 1^{\mathbf{Rel}}_A$;
    $(\mathcal{E}_{\mathcal{C}}^{A,B})^{\dagger} \circ \mathcal{E}^{A,B} \sqsubseteq 1^{\mathcal{C}}_A$.
    
    \item $\mathcal{E}^{B,A} \circ \mathcal{E}^{A,B} \sqsupseteq
    1^{\mathbf{Rel}}_A$; $(\mathcal{E}^{A,B})^{- 1} \circ \mathcal{E}^{A,B} \sqsupseteq 1^{\mathbf{Rel}}_A$;
    $(\mathcal{E}_{\mathcal{C}}^{A,B})^{\dagger} \circ \mathcal{E}_{\mathcal{C}}^{A,B} \sqsupseteq 1^{\mathcal{C}}_A$.
    
    \item $\mathcal{E}^{A,B} \circ \mathcal{E}^{B,A} \sqsupseteq
    1^{\mathbf{Rel}}_A$; $\mathcal{E}^{A,B} \circ (\mathcal{E}^{A,B})^{- 1} \sqsupseteq 1^{\mathbf{Rel}}_A$;
    $\mathcal{E}_{\mathcal{C}}^{A,B} \circ (\mathcal{E}_{\mathcal{C}}^{A,B})^{\dagger} \sqsupseteq 1^{\mathcal{C}}_A$.
  \end{enumerate}
\end{proof}

\section{Rectangular embedding-restriction}

\begin{defn}
  $\iota_{B_0, B_1} f = \mathcal{E}_{\mathcal{C}}^{B_1} \circ f \circ
  (\mathcal{E}_{\mathcal{C}}^{B_0})^{-1}$ for $f \in
  \Hom_{\mathcal{C}} (A_0 ; A_1)$.
\end{defn}

For brevity $\iota_B f = \iota_{B, B} f$.

\begin{prop}
  $\iota_{\Src f, \Dst f} f = f$.
\end{prop}

\begin{proof}
  $\iota_{\Src f, \Dst f} f = \mathcal{E}_\mathcal{C}^{\Dst f} \circ f \circ \mathcal{E}_\mathcal{C}^{\Src f} =
  1_{\mathcal{C}}^{\Dst f} \circ f \circ 1_{\mathcal{C}}^{\Src f} = f$.
\end{proof}

\begin{prop}
  The function $\iota_{B_0, B_1} |_{f \in \Hom_{\mathcal{C}} (A_0 ;
  A_1)}$ is injective, if $A_0 \subseteq B_0 \wedge A_1 \subseteq B_1$.
\end{prop}

\begin{proof}
  Because $\mathcal{E}_{\mathcal{C}}^{A_1,B_1}$ is a monomorphism and $\mathcal{E}_{\mathcal{C}}^{A_0,B_0}$ is an epimorphism.
\end{proof}

\begin{prop}
  $\iota_{C_0, C_1} \iota_{B_0, B_1} f = \iota_{C_0, C_1} f$ for $B_0
  \supseteq A_0 \cap C_0$, $B_1 \supseteq A_1 \cap C_1$ and $f : A_0
  \rightarrow A_1$.
\end{prop}

\begin{proof}
  $\iota_{C_0, C_1} \iota_{B_0, B_1} f = \mathcal{E}_{\mathcal{C}}^{B_1,C_1}
  \circ \mathcal{E}_{\mathcal{C}}^{A_1,B_1} \circ f \circ \mathcal{E}_{\mathcal{C}}^{B_0,A_0} \circ
  \mathcal{E}_{\mathcal{C}}^{C_0,B_0} = \mathcal{E}_{\mathcal{C}}^{A_1,C_1} \circ f \circ \mathcal{E}_{\mathcal{C}}^{C_0,A_0} =
  \iota_{C_0,C_1} f$.
\end{proof}

\begin{prop}
  Let $f : A_0 \rightarrow A_1$ and $g : A_1 \rightarrow A_2$ and $A_1
  \subseteq B_1$. Then $\iota_{B_0, B_2} (g \circ f) = \iota_{B_1, B_1} g
  \circ \iota_{B_0, B_1} f$.
\end{prop}

\begin{proof}
  $\iota_{B_0, B_2} (g \circ f) = \mathcal{E}_{\mathcal{C}}^{A_2,B_2}
  \circ (g \circ f) \circ \mathcal{E}_{\mathcal{C}}^{B_0,A_0} = \mathcal{E}_{\mathcal{C}}^{A_2,B_2} \circ g \circ \id_{A_1} \circ f
  \circ \mathcal{E}_{\mathcal{C}}^{B_0,A_0} = \mathcal{E}_{\mathcal{C}}^{A_2,B_2} \circ g \circ \mathcal{E}^{B_1,A_1}
  \circ \mathcal{E}^{A_1,B_1} \circ f \circ \mathcal{E}_{\mathcal{C}}^{B_0,A_0} = \iota_{B_1, B_1} g \circ \iota_{B_0,
  B_1} f$.
\end{proof}

\section{\texorpdfstring{Examples of partially ordered dagger categories under
$\mathbf{Rel}$}{Examples of partially ordered dagger categories under Rel}}

\subsection{Generalized rebase of filters}

In \cite{volume-1} I defined \emph{rebase} $\mathcal{A} \div A$ for a
set-theoretic filter $\mathcal{A}$ and a set $X$ such that $\exists X \in
\mathcal{A} : X \subseteq A$.

Now define a generalized rebase for every set-theoretic filter $\mathcal{A}$
and every set $A$:

\begin{defn}
  $\mathcal{A} \div A = \bigsqcap \setcond{ \uparrow^A  (X \cap A) }
  { X \in \mathcal{A} }$.
\end{defn}

\begin{prop}
  These two definitions coincide.
\end{prop}

\begin{proof}
  It is proved in
{\cite{volume-1}} $\setcond{ X \in \subsets A }
{\exists Y \in \mathcal{A} : Y\cap A \subseteq X }$ is a filter.

If $P \in \setcond{ X \in \subsets A }{ \exists Y
\in \mathcal{A} : Y\cap A \subseteq X }$ then $P \in \subsets A$ and $Y\cap A
\subseteq P$ for some $Y \in \mathcal{A}$. Thus $P \supseteq Y \cap A \in
\bigsqcap \setcond{ \uparrow^A  (Y \cap A) }{ Y \in \mathcal{A} }$.

If $P \in \bigsqcap \setcond{ \uparrow^A  (X \cap A) }{
X \in \mathcal{A} }$ then by properties of generalized filter bases,
there exists $X \in \mathcal{A}$ such that $P \supseteq X \cap A$. Also $P \in
\subsets A$. Thus $P \in \setcond{ X \in \subsets A
}{ \exists Y \in \mathcal{A} : Y\cap A \subseteq X }$.

\fxwarning{Clear this proof: wording, consistent use of letters.}
\end{proof}

\begin{prop}
  $(\mathcal{X} \div A) \div B = \mathcal{X} \div B$ if $B \subseteq A$.
\end{prop}

\begin{proof}
  $(\mathcal{X} \div A) \div B = \bigsqcap \setcond{ \uparrow^B  (Y \cap B)
  }{ Y \in \bigsqcap \setcond{ \uparrow^A  (X \cap A)}
  {X \in \mathcal{X} } \mathcal{} } =
  \bigsqcap \setcond{ \uparrow^B  (X \cap A) }{ X \in
  \mathcal{X} } \sqcap \uparrow^B B = \bigsqcap \setcond{ \uparrow^B  (X
  \cap A \cap B) }{ X \in \mathcal{X} } =
  \mathcal{X} \div (A \cap B) = \mathcal{X} \div B$.
\end{proof}

\subsection{\texorpdfstring{Category $\mathbf{Rel}$}{Category Rel}}

Category $\mathbf{Rel}$ with the identity up-arrow functor to itself
and ``reverse relation'' as the dagger is an obvious example of a partially
ordered dagger category under $\mathbf{Rel}$.

\begin{prop}
  $\iota_{A, B} f = (A ; B ; \GR f \cap (A \times B))$.
\end{prop}

\begin{proof}
  $\iota_{A, B} f = \mathcal{E}^{B} \circ f \circ (\mathcal{E}^{A})^{-1} = (A ; B ; \GR f \cap (A \times B))$.
\end{proof}

\subsection{\texorpdfstring{Category $\mathsf{FCD}$}{Category FCD}}

Category $\mathsf{FCD}$ with the up-arrow functor
$\uparrow^{\mathsf{FCD}}$ and ``reverse funcoid'' as the dagger is a
partially ordered dagger category under $\mathbf{Rel}$.

\begin{prop}
  $\mathcal{E}_{\mathsf{FCD}}^{A,B} = (A ; B ; \lambda \mathcal{X}
  \in \mathfrak{F} (A) : \mathcal{X} \div B ; \lambda \mathcal{Y} \in
  \mathfrak{F} (B) : \mathcal{Y} \div A)$ for objects $A \subseteq B$ of
  $\mathsf{FCD}$.
\end{prop}

\begin{proof}
  $\langle \mathcal{E}_{\mathsf{FCD}}^{A,B} \rangle \mathcal{X} =
  \bigsqcap \setcond{ \langle \mathcal{E}_{\mathsf{FCD}}^{A,B}
  \rangle^{\ast} X }{ X \in \mathcal{X} } =
  \bigsqcap \setcond{ \uparrow^B  \langle \mathcal{E}^{A,B} \rangle X
  }{ X \in \mathcal{X} } = \bigsqcap \setcond{
  \uparrow^B  (X \cap A \cap B) }{ X \in \mathcal{X}
  } = \bigsqcap \setcond{ \uparrow^B  (X \cap B) }{
  X \in \mathcal{X} } = \mathcal{X} \div B$.
  
  Rest follows from symmetry.
\end{proof}

\begin{prop}
  ~
  \begin{enumerate}
    \item $\langle \mathcal{E}_{\mathsf{FCD}}^{A,B} \rangle^{\ast} X
    = \uparrow^B X$ for every $X \in \subsets A$ if $A \subseteq B$.
    
    \item $\langle \mathcal{E}_{\mathsf{FCD}}^{B,A} \rangle^{\ast}
    Y = \uparrow^A (Y \cap A)$ for every $Y \in \subsets B$ if $A \subseteq
    B$.
  \end{enumerate}
\end{prop}

\begin{proof}
  By definition of principal funcoid.
\end{proof}

\subsection{\texorpdfstring{Category $\mathsf{RLD}$}{Category RLD}}

Category $\mathsf{RLD}$ with the up-arrow functor
$\uparrow^{\mathsf{RLD}}$ and ``reverse reloid'' as the dagger is a
partially ordered dagger category under $\mathbf{Rel}$.

\begin{obvious}
$\mathcal{E}_{\mathsf{RLD}}^{A,B} = \uparrow^{\mathsf{RLD} (A ;
B)} \id_{A \cap B}$.
\end{obvious}

\begin{defn}
  $f \div (A \times B) = (A ; B ; (\GR f) \div (A \times B))$ for every
  reloid $f$.
\end{defn}

\begin{prop}
  $\iota_{A, B} f = f \div (A \times B)$.
\end{prop}

\begin{proof}
  $\iota_{A, B} f = \mathcal{E}_{\mathsf{RLD}}^{B}
\circ f \circ (\mathcal{E}_{\mathsf{RLD}}^{A})^{-1} =
\bigsqcap \setcond{ \uparrow^{\mathsf{RLD}} (\mathcal{E}^{B} \circ F \circ (\mathcal{E}^{A})^{-1}
}{ F \in \GR f } = \bigsqcap \setcond{
\uparrow^{\mathsf{RLD}} (F \cap (A \times B))}
{F \in \GR f } = f \div (A \times B)$.

\fxwarning{Filters on cartesian products vs reloids.}
\end{proof}

\subsection{Some isomorphisms}

\begin{prop}
  $\setcond{ (\mathcal{A} \div A ; \mathcal{A} \sqcap A) }{
  \mathcal{A} \in \mathfrak{F} (U) }$ is a function and
  moreover is an order isomorphism for a set $A \subseteq U$.
\end{prop}

\begin{proof}
  $\mathcal{A} \div A$ and $\mathcal{A} \sqcap A$ are determined by each other
  by the following formulas:
  \[ \mathcal{A} \div A = (\mathcal{A} \sqcap A) \div A \quad
     \text{and} \quad \mathcal{A} \sqcap A = (\mathcal{A} \div A) \div
     \Base (\mathcal{A}) . \]
  Prove the formulas: $(\mathcal{A} \sqcap A) \div A = \bigsqcap \setcond{
  \uparrow^A (X \cap A) }{ X \in \mathcal{A} \sqcap A
  } = \bigsqcap \setcond{ \uparrow^A (X \cap A) }{
  X \in \mathcal{A} } = \mathcal{A} \div A$.
  
  $(\mathcal{A} \div A) \div \Base (\mathcal{A}) = \bigsqcap \setcond{
  \uparrow^A (X \cap A) }{ X \in \mathcal{A} }
  \div \Base (\mathcal{A}) = \bigsqcap \setcond{ \uparrow^{\Base
  (\mathcal{A})} (Y \cap \Base (\mathcal{A})) }{
  Y \in \bigsqcap \setcond{ \uparrow^A (X \cap A) }{
  X \in \mathcal{A} } } = \text{(by properties of
  filter bases)} = \bigsqcap \setcond{ \uparrow^{\Base (\mathcal{A})} (X
  \cap A \cap \Base (\mathcal{A})) }{ X \in
  } = \bigsqcap \setcond{ \uparrow^{\Base
  (\mathcal{A})} (X \cap A) }{ X \in \mathcal{A}
  } = \mathcal{A} \sqcap A$.
  
  That this defines a bijection, follows from $\mathcal{A} \div A \sim
  \mathcal{A} \sqcap A$ what easily follows from the above formulas.
\end{proof}

\begin{prop}
  $\setcond{ (\iota_{X, Y} f ; \id^{\mathbf{Rel}}_Y \circ f \circ
  \id^{\mathbf{Rel}}_X) }{ f \in
  \mathbf{Rel} (A ; B) }$ is a function and moreover is an
  (order and semigroup) isomorphism, for sets $X \subseteq \Src f$, $Y
  \subseteq \Dst f$.
\end{prop}

\begin{proof}
  $\iota_{X, Y} f = (X ; Y ; \GR f \cap (X \times Y))$;
  $\id^{\mathbf{Rel}}_Y \circ f \circ
  \id^{\mathbf{Rel}}_X = (\Src f ; \Dst f ;
  \GR f \cap (X \times Y))$. The isomorphism (both order and semigroup)
  is evident.
\end{proof}

\begin{prop}
  $\setcond{ (\iota_{X, Y} f ; \id^{\mathsf{FCD}}_Y
  \circ f \circ \id^{\mathsf{FCD}}_X) }{
  f \in \mathsf{FCD} (A ; B) }$ is a function and moreover is an
  (order and semigroup) isomorphism, for sets $X \subseteq \Src f$, $Y
  \subseteq \Dst f$.
\end{prop}

\begin{proof}
  From symmetry it follows that it's enough to prove that $\setcond{ \left(
  \mathcal{E}^Y \circ f ; \id^{\mathsf{FCD}}_Y \circ f \right)
  }{ f \in \mathsf{FCD} (A ; B) }$ is a
  function and moreover is an (order and semigroup) isomorphism, for a set $Y
  \subseteq \Dst f$.
  
  Really, $\setcond{ (\langle \mathcal{E}^Y \rangle x ; \langle
  \id^{\mathsf{FCD}}_Y \rangle x) }{ x
  \in \Dst f } = \setcond{ (x \div Y ; x \sqcap Y) }{
  x \in \Dst f }$ is an order isomorphism by proved
  above. This implies that $\setcond{ \left( \mathcal{E}^Y \circ f ;
  \id^{\mathsf{FCD}}_Y \circ f \right) }{
  f \in \mathsf{FCD} (A ; B) }$ is an isomorphism
  (both order and semigroup).
\end{proof}

\begin{prop}
  $\setcond{ (\iota_{X, Y} f ; \id^{\mathsf{RLD}}_Y \circ f \circ
  \id^{\mathsf{RLD}}_X) }{ f \in
  \mathsf{RLD} (A ; B) }$ is a function and moreover is an
  (order and semigroup) isomorphism, for sets $X \subseteq \Src f$, $Y
  \subseteq \Dst f$.
\end{prop}

\begin{proof}
  $\iota_{X, Y} f = (X ; Y ; (\up f) \div (X \times Y))$;
  $\id^{\mathsf{RLD}}_Y \circ f \circ
  \id^{\mathsf{RLD}}_X = (\Src f ; \Dst f ;
  (\up f) \sqcap (X \times Y))$. They are order isomorphic by proved
  above.
  
  $\iota_{Y, Z} g \circ \iota_{X, Y} f =\mathcal{E}^Z \circ g \circ
  (\mathcal{E}^Y)^{- 1} \circ \mathcal{E}^Y \circ f \circ (\mathcal{E}^X)^{-
  1} =\mathcal{E}^Z \circ g \circ \id^{\mathsf{RLD}}_Y \circ
  \id^{\mathsf{RLD}}_Y \circ f \circ (\mathcal{E}^X)^{- 1}$
  because $(\mathcal{E}^Y)^{- 1} \circ \mathcal{E}^Y =
  \id^{\mathbf{Rel}}_Y = \id^{\mathbf{Rel}}_Y
  \circ \id^{\mathbf{Rel}}_Y$. Thus by proved above
  \[ \setcond{ (\iota_{Y, Z} g \circ \iota_{X, Y} f ;
     \id^{\mathsf{RLD}}_Z \circ g \circ
     \id^{\mathsf{RLD}}_Y \circ \id^{\mathsf{RLD}}_Y
     \circ f \circ \id^{\mathsf{RLD}}_X) }{
     f \in \mathsf{RLD} (A ; B) } \]
  is a bijection.
\end{proof}

\section{Equalizers}

Categories $\cont (\mathcal{C})$ are defined above.

I will denote $W$ the forgetful functor from $\cont
(\mathcal{C})$ to $\mathcal{C}$.

In the definition of the category $\cont (\mathcal{C})$ take
values of $\uparrow$ as principal morphisms. \fxwarning{Wording.}

\begin{lem}
  Let $f : X \rightarrow Y$ be a morphism of the category
  $\cont (\mathcal{C})$ where $\mathcal{C}$ is a concrete
  category (so $W f = \uparrow \varphi$ for a $\mathbf{Rel}$-morphism
  $\varphi$ because $f$ is principal) and $\im \varphi = A \subseteq
  \Ob Y$. Factor it $\varphi = \mathcal{E}^{\Ob Y} \circ u$
  where $u : \Ob X \rightarrow A$ using properties of
  $\mathbf{Set}$. Then $u$ is a morphism of $\cont
  (\mathcal{C})$ (that is a continuous function $X \rightarrow \iota_A Y$).
\end{lem}

\begin{proof}
  $(\mathcal{E}^{\Ob Y})^{- 1} \circ \varphi = (\mathcal{E}^{\Ob Y})^{- 1} \circ \mathcal{E}^{\Ob Y} \circ u$;
  
  $(\mathcal{E}_{\mathcal{C}}^{\Ob Y})^{- 1} \circ \uparrow \varphi
  = (\mathcal{E}_{\mathcal{C}}^{\Ob Y})^{- 1} \circ \mathcal{E}_{\mathcal{C}}^{\Ob Y} \circ \uparrow u$;
  
  $(\mathcal{E}_{\mathcal{C}}^{\Ob Y})^{- 1} \circ \uparrow \varphi
  = \uparrow u$;
  
  $X \sqsubseteq (\uparrow u)^{- 1} \circ \pi_A Y \circ \uparrow u
  \Leftrightarrow X \sqsubseteq (\uparrow \varphi)^{- 1} \circ
  \mathcal{E}_{\mathcal{C}}^{\Ob Y} \circ \pi_A Y \circ
  (\mathcal{E}_{\mathcal{C}}^{\Ob Y})^{- 1} \circ \uparrow \varphi
  \Leftrightarrow X \sqsubseteq (\uparrow \varphi)^{- 1} \circ
  \mathcal{E}_{\mathcal{C}}^{\Ob Y} \circ
  (\mathcal{E}_{\mathcal{C}}^{\Ob Y})^{- 1} \circ Y \circ
  \mathcal{E}_{\mathcal{C}}^{\Ob Y} \circ
  (\mathcal{E}_{\mathcal{C}}^{\Ob Y})^{- 1} \circ \uparrow \varphi
  \Leftrightarrow X \sqsubseteq (\uparrow \varphi)^{- 1} \circ Y \circ
  \uparrow \varphi \Leftrightarrow X \sqsubseteq (W f)^{- 1} \circ Y \circ W
  f$ what is true by definition of continuity.
\end{proof}

Equational definition of equalizers:

\url{http://nforum.mathforge.org/comments.php?DiscussionID=5328/}

\begin{thm}
  The following is an equalizer of parallel morphisms $f, g : A \rightarrow B$
  of category $\cont (\mathcal{C})$:
  \begin{itemize}
    \item the object $X = \iota_{\setcond{ x \in \Ob A }{
    f x = g x }} A$;
    
    \item the morphism $\mathcal{E}^{\Ob X, \Ob A}$ considered
    as a morphism $X \rightarrow A$.
  \end{itemize}
\end{thm}

\begin{proof}
  Denote $e = \mathcal{E}^{\Ob X, \Ob A}$.
  
  Let $f \circ z = g \circ z$ for some morphism $z$.
  
  Let's prove $e \circ u = z$ for some $u : \Src z \rightarrow X$.
  Really, as a morphism of $\mathbf{Set}$ it exists and is unique.
  
  Consider $z$ as as a generalized element.
  
  $f (z) = g (z)$. So $z \in X$ (that is $\Dst z \in X$). Thus $z = e
  \circ u$ for some $u$ (by properties of $\mathbf{Set}$). The
  generalized element $u$ is a $\cont (\mathcal{C})$-morphism
  because of the lemma above. It is unique by properties of
  $\mathbf{Set}$.
\end{proof}

We can (over)simplify the above theorem by the obvious below:

\begin{obvious}
$\setcond{ x \in \Ob A }{ f x = g x } = \dom (f \cap g)$.
\end{obvious}

\section{Co-equalizers}

\url{http://math.stackexchange.com/questions/539717/how-to-construct-co-equalizers-in-mathbftop}	

Let $\sim$ be an equivalence relation. Let's denote $\pi$ its canonical
projection.

\begin{defn}
  $f / \sim = \uparrow \pi \circ f \circ \uparrow \pi^{- 1}$ for every
  morphism $f$.
\end{defn}

\begin{obvious}
$\Ob (f / \sim) = (\Ob f) / r$.
\end{obvious}

\begin{obvious}
$f / \sim = \langle \uparrow^{\mathsf{FCD}} \pi \times^{(C)}
\uparrow^{\mathsf{FCD}} \pi \rangle f$ for every morphism
$f$.
\end{obvious}

To define co-equalizers of morphisms $f$ and $g$ let $\sim$ be is the smallest
equivalence relation such that $f x = g x$.

\begin{lem}
  Let $f : X \rightarrow Y$ be a morphism of the category
  $\cont (\mathcal{C})$ where $\mathcal{C}$ is a concrete
  category (so $W f = \uparrow \varphi$ for a $\mathbf{Rel}$-morphism
  $\varphi$ because $f$ is principal) such that $\varphi$ respects $\sim$.
  Factor it $\varphi = u \circ \pi$ where $u : \Ob (X / \sim)
  \rightarrow \Ob Y$ using properties of $\mathbf{Set}$. Then
  $u$ is a morphism of $\cont (\mathcal{C})$ (that is a
  continuous function $X / \sim \rightarrow Y$).
\end{lem}

\begin{proof}
  $f \circ X \circ f^{- 1} \sqsubseteq Y$; $\uparrow u \circ \uparrow \pi
  \circ X \circ \uparrow \pi^{- 1} \circ \uparrow u^{- 1} \sqsubseteq Y$;
  $\uparrow u \in \mathrm{C} (\uparrow \pi \circ X \circ \uparrow \pi^{- 1} ;
  Y) = \mathrm{C} (X / \sim ; Y)$.
\end{proof}

\begin{thm}
  The following is a co-equalizer of parallel morphisms $f, g : A \rightarrow
  B$ of category $\cont (\mathcal{C})$:
  \begin{itemize}
    \item the object $Y = f / \sim$;
    
    \item the morphism $\pi$ considered as a morphism $B \rightarrow Y$.
  \end{itemize}
\end{thm}

\begin{proof}
  Let $z \circ f = z \circ g$ for some morphism $z$.
  
  Let's prove $u \circ \pi = z$ for some $u : Y \rightarrow \Dst z$.
  Really, as a morphism of $\mathbf{Set}$ it exists and is unique.
  
  $\Src z \in Y$. Thus $z = u \circ \pi$ for some $u$ (by properties of
  $\mathbf{Set}$). The function $u$ is a $\cont
  (\mathcal{C})$-morphism because of the lemma above. It is unique by
  properties of $\mathbf{Set}$ ($\pi$ obviously respects equivalence
  classes).
\end{proof}

\section{Rest}

\begin{thm}
  The categories $\cont (\mathcal{C})$ (for example in
  $\mathbf{Fcd}$ and $\mathbf{Rld}$) are complete.
\end{thm}

\begin{proof}
  They have products and equalizers.
\end{proof}

\begin{thm}
  The categories $\cont (\mathcal{C})$ (for example in
  $\mathbf{Fcd}$ and $\mathbf{Rld}$) are co-complete.
\end{thm}

\begin{proof}
  They have co-products and co-equalizers.
\end{proof}

\begin{defn}
  I call morphisms $f$ and $g$ of a category with embeddings
  \emph{equivalent} ($f \sim g$) when there exist a morphism $p$ such that
  $\Src p \sqsubseteq \Src f$, $\Src p \sqsubseteq
  \Src g$, $\Dst p \sqsubseteq \Dst f$, $\Dst p
  \sqsubseteq \Dst g$ and $\iota_{\Src f, \Dst f} p = f$ and
  $\iota_{\Src g, \Dst g} p = g$.
\end{defn}

\begin{problem}
  Find under which conditions:
  \begin{enumerate}
    \item Equivalence of morphisms is an equivalence relation.
    
    \item Equivalence of morphisms is a congruence for our category.
  \end{enumerate}
\end{problem}