\chapter{Micronization}

First read \url{https://portonmath.wordpress.com/2015/05/08/order-top/}.

\begin{defn}
\emph{Micronization} of a binary relation~$E$ is
$\mu(E) = \bigsqcap^{\mathsf{RLD}} \setcond{f\in\mathbf{Rel}}{S(f)\sqsupseteq E}$.
\fxnote{It directly generalized for a funcoid or reloid~$E$.}
\end{defn}

\begin{obvious}
Micronization is always reflexive.
\end{obvious}

\begin{defn}
\emph{Interval}~$I$ on a poset~$\mathfrak{A}$ is such a subset of~$\mathfrak{A}$ that
\[ \forall a,b\in I, x\in\mathfrak{A}:(a\leq x\leq b\Rightarrow x\in I). \]
\end{defn}

Let $L$ be a interval on~$\mathbb{Z}$ (with the standard order).

\begin{prop}
Consider $L$ as a poset (with the induced order~$\leq$). Then
\[ \mu(L) = (L\times L) \cap \bigcup_{i\in\mathbb{Z}} \{(i;i), (i;i+1)\}. \]
\end{prop}

\begin{proof}
Let's denote~$D$ the right part of the formula to be proved.

We will prove that $D \in \min \setcond{f\in\mathbf{Rel}}{S(f)\sqsupseteq E}$.

We have $S(D)=L\times L$ and thus $D\sqsupseteq L$; thus $D\in\setcond{f\in\mathbf{Rel}}{S(f)\sqsupseteq E}$.

It remains to prove that $S(f)\sqsupseteq E\Rightarrow D\sqsubseteq f$.

Suppose contrary that $D\nsqsubseteq f$. Then either $(i;i)\notin f$ for $i\in L$ or $(i;i+1)\notin f$ for $i,i+1\in L$.

In either case $S(f)\subset (L\times L) \cap \bigcup_{i\in\mathbb{Z}} \{(i;i), (i;i+1)\} = D$
and thus $(f)\nsqsupseteq E$.
\end{proof}

\begin{question}
$\mu(E) = \bigsqcap^{\mathsf{RLD}} \setcond{f\in\mathbf{Rel}}{S(f)=E}$, if $E$ is reflexive and transitive?
\end{question}

\begin{conjecture}
$S^{\ast}(\mu(E)) = E$, if $E$ is reflexive and transitive.
\end{conjecture}
