\chapter{Cauchy Filters on Reloids}

In this chapter I consider \emph{low filters} on reloids, generalizing
Cauchy filters on uniform spaces. Using low filters, I define Cauchy-complete
reloids, generalizing complete uniform spaces.

\section{Preface}

Replace {\tt \textbackslash langle \dots \textbackslash rangle} with {\tt \textbackslash supfun\{\dots\}} in \LaTeX{}.

This is a preliminary partial draft.

To understand this article you need first look into my book \cite{volume-1}.

\url{http://math.stackexchange.com/questions/401989/what-are-interesting-properties-of-totally-bounded-uniform-spaces}

\url{http://ncatlab.org/nlab/show/proximity+space\#uniform\_spaces} for a proof
sketch that proximities correspond to totally bounded uniformities.

\section{Low filters space}

\begin{defn}
A \emph{lower set}\footnote{Remember that our orders on filters is the
reverse to set theoretic inclusion. It could be called an \emph{upper} set
in other sources.} of proper filters on $U$ (a set) is a set $\mathscr{C}$
of proper filters on $U$, such that if $\bot \neq \mathcal{G} \sqsubseteq
\mathcal{F}$ and $\mathcal{F} \in \mathscr{C}$ then $\mathcal{G} \in
\mathscr{C}$. \fxwarning{Probably should include the improper filter.}
\end{defn}

\begin{defn}
  I call \emph{low filters space} a set together with a lower set of proper
  filters on this set.
\end{defn}

\begin{defn}
  $\GR \left( U ; \mathscr{C} \right) = \mathscr{C}$; $\Ob \left(
  U ; \mathscr{C} \right) = U$.
\end{defn}

\begin{defn}
  Introduce an order on low filters spaces: $\left( U ; \mathscr{C} \right)
  \sqsubseteq \left( U ; \mathscr{D} \right) \Leftrightarrow \mathscr{C}
  \sqsubseteq \mathscr{D}$.
\end{defn}

\section{Cauchy spaces}

\begin{defn}
  A \emph{Cauchy space} on a set $X$ is a low filters space $\left( U ;
  \mathscr{C} \right)$ (element of $\mathscr{C}$ are called \emph{Cauchy
  filters}) such that:
  \begin{enumerate}
    \item $\forall x \in U : \uparrow^X \{ x \} \in \mathscr{C}$;
    
    \item If $\mathcal{F}$, $\mathcal{G}$ are Cauchy filters and $\mathcal{F}
    \nasymp \mathcal{G}$ then $\mathcal{F} \sqcup \mathcal{G}$ is a Cauchy
    filter.
  \end{enumerate}
\end{defn}

\begin{defn}
  A \emph{completely Cauchy space} on a set $X$ is a low filters space
  $\left( U ; \mathscr{C} \right)$ (element of $\mathscr{C}$ are called
  \emph{Cauchy filters}) such that:
  \begin{enumerate}
    \item $\forall x \in X : \uparrow^X \{ x \} \in \mathscr{C}$;
    
    \item If $S$ is a nonempty set of Cauchy filters and $\bigsqcap S \neq
    \bot^{\mathscr{F} (X)}$ then $\bigsqcup S$ is a Cauchy filter.
  \end{enumerate}
\end{defn}

\begin{obvious}
Every completely Cauchy space is a Cauchy space.
\end{obvious}

\begin{prop}
  $\bigsqcup^{
    \setcond{\mathcal{X} \in \mathscr{C} }{\mathcal{X} \sqsupseteq \mathcal{F}}
  } S = \bigsqcup S$ for nonempty
  $S \in \subsets \setcond{\mathcal{X} \in \mathscr{C} }{\mathcal{X} \sqsupseteq \mathcal{F}}$,
  provided that $\mathcal{F}$ is a fixed Cauchy filter on a completely Cauchy space.
\end{prop}

\begin{proof}
  $\mathcal{F}$ is proper. So for every nonempty $S \in \subsets
  \setcond{\mathcal{X} \in \mathscr{C} }{\mathcal{X} \sqsupseteq \mathcal{F}}$
  we have $\bigsqcap S \sqsupseteq \mathcal{F} \neq \bot^{\mathscr{F} (X)}$. Thus $\bigsqcup S$ is a Cauchy
  filter and so $\bigsqcup S \in \setcond{\mathcal{X} \in \mathscr{C} }{\mathcal{X} \sqsupseteq \mathcal{F}}$.
\end{proof}

\section{Relationships with symmetric reloids}

\begin{defn}
  Denote $(\mathsf{RLD})_{\Low} \left( U ; \mathscr{C} \right) =
  \bigsqcup \setcond{ \mathcal{X} \times^{\mathsf{RLD}} \mathcal{X}}
  {\mathcal{X} \in \mathscr{C} }$.
\end{defn}

\begin{defn}
  $(\Low) \nu$ (\emph{low filters} for reloid $\nu$) is a low filters
  space on $U$ such that
  \[ \GR (\Low) \nu = \setcond{\mathcal{X} \in \mathscr{F}^U
     \setminus \{ \bot^{\mathscr{F}} \}} {\mathcal{X}
     \times^{\mathsf{RLD}} \mathcal{X} \sqsubseteq \nu } . \]
\end{defn}

\begin{thm}
  If $\left( U ; \mathscr{C} \right)$ is a low filters space, then $\left( U ;
  \mathscr{C} \right) = (\Low) (\mathsf{RLD})_{\Low} \left(
  U ; \mathscr{C} \right)$.
\end{thm}

\begin{proof}
  If $\mathcal{X} \in \mathscr{C}$ then $\mathcal{X}
  \times^{\mathsf{RLD}} \mathcal{X} \sqsubseteq
  (\mathsf{RLD})_{\Low} \left( U ; \mathscr{C} \right)$ and thus
  $\mathcal{X} \in \GR (\Low) (\mathsf{RLD})_{\Low}
  \left( U ; \mathscr{C} \right)$. Thus $\left( U ; \mathscr{C} \right)
  \sqsubseteq (\Low) (\mathsf{RLD})_{\Low} \left( U ;
  \mathscr{C} \right)$.
  
  Let's prove $\left( U ; \mathscr{C} \right) \sqsupseteq (\Low)
  (\mathsf{RLD})_{\Low} \left( U ; \mathscr{C} \right)$.
  
  Let $\mathcal{A} \in \GR (\Low)
  (\mathsf{RLD})_{\Low} \left( U ; \mathscr{C} \right)$. We need
  to prove $\mathcal{A} \in \mathscr{C}$.
  
  Really $\mathcal{A} \times^{\mathsf{RLD}} \mathcal{A} \sqsubseteq
  (\mathsf{RLD})_{\Low} \left( U ; \mathscr{C} \right)$. It is
  enough to prove that $\exists \mathcal{X} \in \mathscr{C} : \mathcal{A}
  \sqsubseteq \mathcal{X}$.
  
  Suppose $\nexists \mathcal{X} \in \mathscr{C} : \mathcal{A} \sqsubseteq
  \mathcal{X}$.
  
  For every $\mathcal{X} \in \mathscr{C}$ obtain $X_{\mathcal{X}} \in
  \mathcal{X}$ such that $X_{\mathcal{X}} \notin \mathcal{A}$ (if forall $X \in
  \mathcal{X}$ we have $X_{\mathcal{X}} \in \mathcal{A}$, then $\mathcal{X}
  \sqsupseteq \mathcal{A}$ what is contrary to our supposition).
  
  It is now enough to prove $\mathcal{A} \times^{\mathsf{RLD}}
  \mathcal{A} \mathrel{\not{\sqsubseteq}} \bigsqcup \setcond{ \uparrow^U
  X_{\mathcal{X}} \times^{\mathsf{RLD}} \uparrow^U X_{\mathcal{X}}
  }{\mathcal{X} \in \mathscr{C} }$.
  
  Really, $\bigsqcup \setcond{ \uparrow^U X_{\mathcal{X}}
  \times^{\mathsf{RLD}} \uparrow^U X_{\mathcal{X}} }{
  \mathcal{X} \in \mathscr{C} } =
  \uparrow^{\mathsf{RLD} (U ; U)} \bigcup \setcond{ \uparrow^U X_{\mathcal{X}}
  \times^{\mathsf{RLD}} \uparrow^U X_{\mathcal{X}} }{
  \mathcal{X} \in \mathscr{C} }$. So our claim takes the form $\bigcup \setcond{ \uparrow^U X_{\mathcal{X}}
  \times^{\mathsf{RLD}} \uparrow^U X_{\mathcal{X}} }{
  \mathcal{X} \in \mathscr{C} } \notin \GR (\mathcal{A}
  \times^{\mathsf{RLD}} \mathcal{A})$ that is $\forall A \in
  \mathcal{A} : \bigcup \setcond{ \uparrow^U X_{\mathcal{X}}
  \times^{\mathsf{RLD}} \uparrow^U X_{\mathcal{X}} }{
  \mathcal{X} \in \mathscr{C} } \nsupseteq
  A \times A$ what is true because $X_{\mathcal{X}} \nsupseteq A$ for every $A
  \in \mathcal{A}$.
\end{proof}

\begin{rem}
  The last theorem does not hold with $\mathcal{X}
  \times^{\mathsf{FCD}} \mathcal{X}$ instead of $\mathcal{X}
  \times^{\mathsf{RLD}} \mathcal{X}$ (take $\mathscr{C} = \setcond{ \{ x
  \} }{x \in U}$ for an infinite set $U$ as a counter-example).
\end{rem}

\begin{rem}
  Not every symmetric reloid is in the form
  $(\mathsf{RLD})_{\Low} \left( U ; \mathscr{C} \right)$ for some
  Cauchy space $\left( U ; \mathscr{C} \right)$. The same Cauchy space can be
  induced by different uniform spaces. See
  \url{http://math.stackexchange.com/questions/702182/different-uniform-spaces-having-the-same-set-of-cauchy-filters}
\end{rem}

\begin{conjecture}
There exist low filter spaces~$f$ and~$g$ such that $(\mathsf{RLD})_{\Low}g\circ(\mathsf{RLD})_{\Low}f \ne (\mathsf{RLD})_{\Low}p$
for \emph{all} low spaces~$p$.
\end{conjecture}

\section{Lattices of low filter spaces}

\begin{defn}
$\nu \sqsubseteq \mu \Leftrightarrow \forall \mathcal{X}\in\GR\mu \exists\mathcal{Y}\in\GR\nu: \mathcal{Y}\sqsubseteq\mathcal{X}$ for
low filter spaces (on the same set~$U$).
\end{defn}

\begin{prop}
The above defined relation is a partial order.
\end{prop}

\begin{proof}
??
\end{proof}

TODO

\fxnote{Do $(\mathsf{RLD})_{\Low}$ and $(\Low)$ preserve lattice operations?}

\fxnote{Also lattices of complete low filter spaces and of Cauchy spaces (and complete Cauchy spaces?).}

\fxnote{\url{https://en.wikipedia.org/wiki/Cauchy_space} says ``The category of Cauchy spaces and Cauchy continuous maps is cartesian closed.'' Generalize.}

\section{More on Cauchy filters}

\begin{obvious}
Low filter on an endoreloid $\nu$ is a filter $\mathcal{F}$ such that
\[ \forall U \in \GR f \exists A \in \mathcal{F} : A \times A \subseteq
   U. \]
\end{obvious}

\begin{rem}
  The above formula is the standard definition of Cauchy filters on uniform
  spaces.
\end{rem}

\begin{prop}
  If $\nu \sqsupseteq \nu \circ \nu^{- 1}$ then every neighborhood filter is a
  Cauchy filter, that it
  \[ \nu \sqsupseteq \rsupfun{\tofcd \nu} \{ x \}
     \times^{\mathsf{RLD}} \rsupfun{\tofcd \nu} \{ x \} \]
  for every point $x$.
\end{prop}

\begin{proof}
  $\rsupfun{\tofcd \nu} \{ x \}
  \times^{\mathsf{RLD}} \rsupfun{\tofcd \nu} \{ x \} = \supfun{\tofcd \nu}
  \uparrow^{\Ob \nu} \left\{ x \right\} \times^{\mathsf{RLD}}
  \supfun{\tofcd \nu} \uparrow^{\Ob \nu} \{ x \} =
  \nu \circ (\uparrow^{\Ob \nu} \{ x \} \times^{\mathsf{RLD}}
  \uparrow^{\Ob \nu} \{ x \}) \circ \nu^{- 1} = \nu \circ
  (\uparrow^{\mathsf{RLD} (\Ob \nu ; \Ob \nu)} \{ (x ; x)
  \}) \circ \nu^{- 1} \sqsubseteq \nu \circ \id^{\mathsf{RLD}
  (\Ob \nu ; \Ob \nu)} \circ \nu^{- 1} = \nu \circ \nu^{- 1}
  \sqsubseteq \nu$.
\end{proof}

\begin{prop}
  If a filter converges to a point, it is a low filter, provided that every
  neighborhood filter is a low filter.
\end{prop}

\begin{proof}
  Let $\mathcal{F} \sqsubseteq \rsupfun{\tofcd \nu} \{ x \}$. Then $\mathcal{F} \times^{\mathsf{RLD}}
  \mathcal{F} \sqsubseteq \rsupfun{\tofcd \nu} \{
  x \} \times^{\mathsf{RLD}} \rsupfun{\tofcd \nu} \{ x \} \sqsubseteq \nu$.
\end{proof}

\begin{cor}
  If a filter converges to a point, it is a low filter, provided that $\nu
  \sqsupseteq \nu \circ \nu^{- 1}$.
\end{cor}

\section{Maximal Cauchy filters}

\begin{lem}
  Let $S$ be a set of sets with $\bigsqcap \langle \uparrow^{\mathfrak{F}}
  \rangle S \neq 0^{\mathfrak{F}}$ (in other words, $S$ has finite
  intersection property). Let $T = \setcond{ X \times X}
  {X \in S }$. Then
  \[ \bigcup T \circ \bigcup T = \bigcup S \times \bigcup S. \]
\end{lem}

\begin{proof}
  Let $x \in \bigcup S$. Then $x \in X$ for some $X \in S$. $\left\langle
  \bigcup T \right\rangle \{ x \} \sqsupseteq \uparrow X \supseteq \bigcap S
  \neq \emptyset$. Thus
  
  $\left\langle \bigcup T \circ \bigcup T \right\rangle \{ x \} = \left\langle
  \bigcup T \right\rangle \left\langle \bigcup T \right\rangle \{ x \} \in
  \left\langle \uparrow^{\mathsf{FCD}} \bigcup T \right\rangle
  \bigsqcap \langle \uparrow^{\mathfrak{F}} \rangle S \sqsupseteq \bigsqcup
  \setcond{ \langle \uparrow^{\mathsf{FCD}} (X \times X) \rangle
  \bigsqcap \langle \uparrow^{\mathfrak{F}} \rangle S }{
  X \in S } = \bigsqcup \setcond{ \uparrow^{\mathfrak{F}} X}{X \in S } =
  \bigsqcup \langle
  \uparrow^{\mathfrak{F}} \rangle S$ that is $\left\langle \bigcup T \circ
  \bigcup T \right\rangle \{ x \} \supseteq \bigcup S$.
\end{proof}

\begin{cor}
  Let $S$ be a set of filters (on some fixed set) with nonempty meet. Let
  \[ T = \setcond{ \mathcal{X} \times^{\mathsf{RLD}} \mathcal{X} }
     {\mathcal{X} \in S} \]
  Then
  \[ \bigsqcup T \circ \bigsqcup T = \bigsqcup S \times^{\mathsf{RLD}}
     \bigsqcup S. \]
\end{cor}

\begin{proof}
  $\bigsqcup T \circ \bigsqcup T = \bigsqcap \setcond{ \uparrow^{\mathfrak{F}}
  (X \circ X) }{ X \in \bigsqcup T }$.
  
  If $X \in \bigsqcup T$ then $X = \bigcup_{Q \in T} (P_Q \times P_Q)$ where
  $P_Q \in Q$. Therefore by the lemma we have
  \[ \bigcup \setcond{ P_Q \times P_Q}{Q \in T} \circ \bigcup
  \setcond{ P_Q \times P_Q}{ Q \in T } = \bigcup_{Q \in T} P_Q \times \bigcup_{Q \in T} P_Q .
  \]
  Thus $X \circ X = \bigcup_{Q \in T} P_Q \times \bigcup_{Q \in T} P_Q$.
  
  Consequently $\bigsqcup T \circ \bigsqcup T = \bigsqcap \setcond{
  \uparrow^{\mathfrak{F}} \left( \bigcup_{Q \in T} P_Q \times \bigcup_{Q \in
  T} P_Q \right)}{X \in \bigsqcup T }
  \sqsupseteq \bigsqcup S \times^{\mathsf{RLD}} \bigsqcup S$.
  
  $\bigsqcup T \circ \bigsqcup T \sqsubseteq \bigsqcup S
  \times^{\mathsf{RLD}} \bigsqcup S$ is obvious.
\end{proof}

\begin{defn}
  I call an endoreloid $\nu$ \emph{symmetrically transitive} iff for every
  symmetric endofuncoid $f \in \mathsf{FCD} (\Ob \nu ; \Ob
  \nu)$ we have $f \sqsubseteq \nu \Rightarrow f \circ f \sqsubseteq \nu$.
\end{defn}

\begin{obvious}
  It is symmetrically transitive if at least one of the following holds:
  \begin{enumerate}
    \item $\nu \circ \nu \sqsubseteq \nu$;
    
    \item $\nu \circ \nu^{- 1} \sqsubseteq \nu$;
    
    \item $\nu^{- 1} \circ \nu \sqsubseteq \nu$.
    
    \item $\nu^{- 1} \circ \nu^{- 1} \sqsubseteq \nu$.
  \end{enumerate}
\end{obvious}

\begin{cor}
  Every uniform space is symmetrically transitive.
\end{cor}

\begin{prop}
  $(\Low) \nu$ is a completely Cauchy space for every symmetrically
  transitive endoreloid $\nu$.
\end{prop}

\begin{proof}
  Suppose $S \in \subsets \setcond{ \mathcal{X} \in \mathfrak{F} \setminus \{
  0^{\mathfrak{F}} \}}{ \mathcal{X}
  \times^{\mathsf{RLD}} \mathcal{X} \sqsubseteq \nu }$ and $S
  \neq \emptyset$.
  
  $\bigsqcup \setcond{ \mathcal{X} \times^{\mathsf{RLD}} \mathcal{X}}
  {\mathcal{X} \in S } \sqsubseteq \nu$;
  $\bigsqcup \setcond{ \mathcal{X} \times^{\mathsf{RLD}} \mathcal{X}}
  {\mathcal{X} \in S } \circ \bigsqcup
  \setcond{ \mathcal{X} \times^{\mathsf{RLD}} \mathcal{X} }
  {\mathcal{X} \in S} \sqsubseteq \nu$; $\bigsqcup S
  \times^{\mathsf{RLD}} \bigsqcup S \sqsubseteq \nu$ (taken into
  account that $S$ has nonempty meet). Thus $\bigsqcup S$ is Cauchy.
\end{proof}

\begin{prop}
  The neighbourhood filter $\langle \tofcd \nu \rangle^{\ast} \{
  x \}$ of a point $x \in \Ob \nu$ is a maximal Cauchy filter, if it is a
  Cauchy filter and $\nu$ is a reflexive reloid.
  \fxnote{Does it holds for all low filters?}
\end{prop}

\begin{proof}
  Let $\mathscr{N} = \langle \tofcd \nu \rangle^{\ast} \{ x
  \}$. Let $\mathscr{C \sqsupseteq N}$ be a Cauchy filter. We need to show
  $\mathscr{N \sqsupseteq C}$.
  
  Since $\mathscr{C}$ is Cauchy filter, $\mathscr{C}
  \times^{\mathsf{RLD}} \mathscr{C} \sqsubseteq \nu$. Since $\mathscr{C
  \sqsupseteq N}$ we have $\mathscr{C}$ is a neighborhood of $x$ and thus
  $\uparrow^{\Ob \nu} \{ x \} \sqsubseteq \mathscr{C}$ (reflexivity of
  $\nu$). Thus $\uparrow^{\Ob \nu} \{ x \} \times^{\mathsf{RLD}}
  \mathscr{C} \sqsubseteq \mathscr{C} \times^{\mathsf{RLD}}
  \mathscr{C}$ and hence $\uparrow^{\Ob \nu} \{ x \}
  \times^{\mathsf{RLD}} \mathscr{C} \sqsubseteq \nu$;
  \[ \mathscr{C} \sqsubseteq \im (\nu |_{\uparrow^{\Ob \nu} \{ x
     \}}) = \langle \tofcd \nu \rangle^{\ast} \{ x \} =
     \mathscr{N} . \]
\end{proof}

\section{Cauchy continuous functions}

\begin{defn}
  A function $f : U \rightarrow V$ is \emph{Cauchy continuous} from a low
  filters space $\left( U ; \mathscr{C} \right)$ to a low filters space
  $\left( V ; \mathscr{D} \right)$ when $\forall \mathcal{X} \in \mathscr{C} :
  \langle \uparrow^{\mathsf{FCD}} f \rangle \mathcal{X} \in
  \mathscr{D}$.
\end{defn}

\begin{prop}
  Let $f$ be a principal reloid. Then $f \in \mathrm{C} \left( 
  (\mathsf{RLD})_{\Low} \mathscr{C} ;
  (\mathsf{RLD})_{\Low} \mathscr{D} \right)$ iff $f$ is Cauchy
  continuous.
  \begin{eqnarray*}
    f \circ (\mathsf{RLD})_{\Low} \mathscr{C} \circ f^{- 1}
    \sqsubseteq (\mathsf{RLD})_{\Low} \mathscr{D} &
    \Leftrightarrow & \\
    \bigsqcup_{\mathcal{X} \in
    \mathscr{C}} (f \circ (\mathcal{X} \times^{\mathsf{RLD}}
    \mathcal{X} ) \circ f^{- 1} ) \sqsubseteq (\mathsf{RLD})_{\Low}
    \mathscr{D} & \Leftrightarrow & \\
    \bigsqcup_{\mathcal{X} \in \mathscr{C}} ( \langle \uparrow^{\mathsf{FCD}} f \rangle
    \mathcal{X} \times^{\mathsf{RLD}} \langle
    \uparrow^{\mathsf{FCD}} f \rangle \mathcal{X}) \sqsubseteq
    (\mathsf{RLD})_{\Low} \mathscr{D} & \Leftrightarrow & \\
    \forall \mathcal{X} \in \mathscr{C} : \langle
    \uparrow^{\mathsf{FCD}} f \rangle \mathcal{X}
    \times^{\mathsf{RLD}} \langle \uparrow^{\mathsf{FCD}} f
    \rangle \mathcal{X} \sqsubseteq (\mathsf{RLD})_{\Low}
    \mathscr{D} & \Leftrightarrow & \\
    \forall \mathcal{X} \in \mathscr{C} : \langle
    \uparrow^{\mathsf{FCD}} f \rangle \mathcal{X} \in \mathscr{D} . & 
    & 
  \end{eqnarray*}
\end{prop}

Thus we have expressed Cauchy properties through the algebra of reloids.

\section{Cauchy-complete reloids}

\begin{defn}
  An endoreloid $\nu$ is \emph{Cauchy-complete} iff every low filter for
  this reloid converges to a point.
\end{defn}

\begin{rem}
  In my book \cite{volume-1} \emph{complete reloid} means something
  different. I will always prepend the word ``Cauchy'' to the word
  ``complete'' when meaning is by the last definition.
\end{rem}

\url{https://en.wikipedia.org/wiki/Complete\_uniform\_space\#Completeness}

\section{Totally bounded}

\url{http://ncatlab.org/nlab/show/Cauchy+space}

\begin{defn}
  Cauchy space is called \emph{totally bounded} when every proper filter
  contains a Cauchy filter.
\end{defn}

\begin{obvious}
A reloid $\nu$ is totally bounded iff
\[ \forall X \in \subsets \Ob \nu \exists \mathcal{X} \in
   \mathfrak{F}^{\Ob \nu} : (\bot \neq \mathcal{X} \sqsubseteq
   \uparrow^{\Ob \nu} X \wedge \mathcal{X} \times^{\mathsf{RLD}}
   \mathcal{X} \sqsubseteq \nu) . \]
\end{obvious}

\begin{thm}
  A symmetric transitive reloid is totally bounded iff its Cauchy space is
  totally bounded.
\end{thm}

\begin{proof}~
  
  \begin{description}
    \item[$\Rightarrow$] Let $\mathcal{F}$ be a proper filter on $\Ob
    \nu$ and let $a \in \atoms \mathcal{F}$. It's enough to prove that
    $a$ is Cauchy.
    
    Let $D \in \GR \nu$. Let also $E \in \GR \nu$ is symmetric
    and $E \circ E \subseteq D$. There existsa finite subset $F \subseteq
    \Ob \nu$ such that $\langle E \rangle F = \Ob \nu$. Then
    obviously exists $x \in F$ such that $a \sqsubseteq \uparrow^{\Ob
    \nu} \langle E \rangle \{ x \}$, but $\langle E \rangle \{ x \} \times
    \langle E \rangle \{ x \} = E^{- 1} \circ (\{ x \} \times \{ x \}) \circ E
    \subseteq D$, thus $a \times^{\mathsf{RLD}} a \sqsubseteq
    \uparrow^{\mathsf{RLD} (\Ob \nu ; \Ob \nu)} D$.
    
    Because $D$ was taken arbitrary, we have $a \times^{\mathsf{RLD}} a
    \sqsubseteq \nu$ that is $a$ is Cauchy.
    
    \item[$\Leftarrow$] Suppose that Cauchy space associated with a reloid
    $\nu$ is totally bounded but the reloid $\nu$ isn't totally bounded. So
    there exists a $D \in \GR \nu$ such that $(\Ob \nu) \setminus
    \langle D \rangle F \neq \emptyset$ for every finite set $F$.
    
    Consider the filter base
    \[ S = \setcond{ (\Ob \nu) \setminus \langle D \rangle F}
       {F \in \subsets \Ob \nu, F \text{ is finite}
       } \]
    and the filter $\mathcal{F} = \bigsqcap \langle \uparrow^{\Ob \nu}
    \rangle S$ generated by this base. The filter $\mathcal{F}$ is proper
    because intersection $P \cap Q \in S$ for every $P, Q \in S$ and
    $\emptyset \notin S$. Thus there exists a Cauchy (for our Cauchy space)
    filter $\mathcal{X} \sqsubseteq \mathcal{F}$ that is $\mathcal{X}
    \times^{\mathsf{RLD}} \mathcal{X} \sqsubseteq \nu$.
    
    Thus there exists $M \in \mathcal{X}$ such that $M \times M \subseteq D$.
    Let $F$ be a finite subset of $\Ob \nu$. Then $(\Ob \nu)
    \setminus \langle D \rangle F \in \mathcal{F} \sqsupseteq \mathcal{X}$.
    Thus $M \nasymp (\Ob \nu) \setminus \langle D \rangle F$ and so
    there exists a point $x \in M \cap ((\Ob \nu) \setminus \langle D
    \rangle F)$.
    
    $\langle M \times M \rangle \{ p \} \subseteq \langle D \rangle \{ x \}$
    for every $p \in M$; thus $M \subseteq \langle D \rangle \{ x \}$.
    
    So $M \subseteq \langle D \rangle (F \cup \{ x \})$. But this means that
    $M \in \mathcal{X}$ does not intersect $(\Ob \nu) \setminus \langle
    D \rangle (F \cup \{ x \}) \in \mathcal{F} \sqsupseteq \mathcal{X}$, what
    is a contradiction (taken into account that $\mathcal{X}$ is proper).
  \end{description}
\end{proof}

\url{http://math.stackexchange.com/questions/104696/pre-compactness-total-boundedness-and-cauchy-sequential-compactness}

\section{Totally bounded funcoids}

\begin{defn}
  A funcoid $\nu$ is totally bounded iff
  \[ \forall X \in \Ob \nu \exists \mathcal{X} \in
     \mathfrak{F}^{\Ob \nu} : (0 \neq \mathcal{X} \sqsubseteq
     \uparrow^{\Ob \nu} X \wedge \mathcal{X}
     \times^{\mathsf{FCD}} \mathcal{X} \sqsubseteq \nu) . \]
\end{defn}

This can be rewritten in elementary terms (without using funcoidal product:

$\mathcal{X} \times^{\mathsf{FCD}} \mathcal{X} \sqsubseteq \nu
\Leftrightarrow \forall P \in \corestar \mathcal{X} : \mathcal{X} \sqsubseteq
\supfun{\nu} P \Leftrightarrow \forall P \in \corestar \mathcal{X}, Q
\in \corestar \mathcal{X} : P \mathrel{[\nu]^{\ast}} Q \Leftrightarrow \forall
P, Q \in \Ob \nu : \left( \forall E \in \mathcal{X} : (E \cap P \neq
\emptyset \wedge E \cap Q \neq \emptyset) \Rightarrow P \mathrel{[\nu]^{\ast}}
Q \right)$.

Note that probably I am the first person which has written the above formula
(for proximity spaces for instance) explicitly.

\section{On principal low filter spaces}

\begin{defn}
  A low filter space $\left( U ; \mathscr{C} \right)$ is \emph{principal}
  when all filters in $\mathscr{C}$ are principal.
\end{defn}

\begin{defn}
  A low filter space $\left( U ; \mathscr{C} \right)$ is \emph{reflexive}
  when $\forall x \in U : \uparrow^U \{ x \} \in \mathscr{C}$.
\end{defn}

\begin{prop}
  Having fixed a set $U$, principal reflexive low filter spaces on $U$
  bijectively correspond to principal reflexive symmertic endoreloids on $U$.
\end{prop}

\begin{proof}
  ??
  
  http://math.stackexchange.com/questions/701684/union-of-cartesian-squares
\end{proof}

\section{Rest}

\url{https://en.wikipedia.org/wiki/Cauchy\_filter\#Cauchy\_filters}

\url{https://en.wikipedia.org/wiki/Uniform\_space} ``Hausdorff completion of a
uniform space'' here)

\url{http://at.yorku.ca/z/a/a/b/13.htm} : the category $\mathbf{Prox}$ of proximity
spaces and proximally continuous maps (i.e. maps preserving nearness between
two sets) is isomorphic to the category of totally bounded uniform spaces (and
uniformly continuous maps).

\url{https://en.wikipedia.org/wiki/Cauchy\_space}
\url{http://ncatlab.org/nlab/show/Cauchy+space}

\url{http://arxiv.org/abs/1309.1748}

\url{http://projecteuclid.org/download/pdf\_1/euclid.pja/1195521991}

\url{http://www.emis.de/journals/HOA/IJMMS/Volume5\_3/404620.pdf}

\url{\~/math/books/Cauchy\_spaces.pdf}