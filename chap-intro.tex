
\chapter{Introduction}

The main purpose of this book is to record the current state of my
research. The book is however written in such a way that it can be
used as a textbook for studying my research.

For related materials, articles, research questions, and erratum consult
the Web page of the author of the book:\\
\url{http://www.mathematics21.org/algebraic-general-topology.html}


\section{License and editing}

This work is licensed under the Creative Commons Attribution 4.0 International
License. To view a copy of the license, visit\\
\href{http://creativecommons.org/licenses/by/4.0/}{http://creativecommons.org/licenses/by/4.0/}.

You can create your own copy of \LaTeX{} source of the book and edit it (to correct errors, add new results, generalize existing results).
The editable source of the book is presented at\\
\href{https://bitbucket.org/portonv/algebraic-general-topology}{https://bitbucket.org/portonv/algebraic-general-topology}

Please consider reviewing this book at\\
\href{http://www.euro-math-soc.eu/node/add/book-review}{http://www.euro-math-soc.eu/node/add/book-review}

\section{Draft status}

This is a draft.

It is considered a finished representation of the research on this
topic, but the research itself is a work in progress.

The book was checked for errors partially. It is expected that there
are errors in ``\nameref{pf-funcoids}'' and ``\nameref{multi}''
chapters, in fine-tuning theorem weakest enough theorem conditions.
The rest of the book is expected not to contain significant errors,
but may contain some typos (especially in formulas).


\section{Intended audience }

This book is suitable for any math student as well as for researchers.

To make this book be understandable even for first grade students,
I made a chapter about basic concepts (posets, lattices, topological
spaces, etc.), which an already knowledgeable person may skip reading.
It is assumed that the reader knows basic set theory.

But it is also valuable for mature researchers, as it contains much
original research which you could not find in any other source except
of my work.

Knowledge of the basic set theory is expected from the reader.

Despite that this book presents new research, it is well structured
and is suitable to be used as a textbook for a college course.

Your comments about this book are welcome to the email \href{mailto:porton@narod.ru}{porton@narod.ru}.


\section{Reading Order }

If you know basic order and lattice theory (including Galois connections
and brouwerian lattices) and basics of category theory, you may skip
reading the chapter ``\nameref{chap-common}''.

You are recommended to read the rest of this book by the order.


\section{Our topic and rationale}

From \cite{mw-gen-top}: \emph{Point-set topology, also called set-theoretic
topology or general topology, is the study of the general abstract
nature of continuity or ``closeness'' on spaces. Basic point-set
topological notions are ones like continuity, dimension, compactness,
and connectedness.}

In this work we study a new approach to point-set topology (and pointfree
topology).

Traditionally general topology is studied using topological spaces
(defined below in the section ``\nameref{sec-top}''). I however
argue that the theory of topological spaces is not the best method
of studying general topology and introduce an alternative theory,
the theory of \emph{funcoids}. Despite of popularity of the theory
of topological spaces it has some drawbacks and is in my opinion not
the most appropriate formalism to study most of general topology.
Because topological spaces are tailored for study of special sets,
so called open and closed sets, studying general topology with topological
spaces is a little anti-natural and ugly. In my opinion the theory
of funcoids is more elegant than the theory of topological spaces,
and it is better to study funcoids than topological spaces. One of
the main purposes of this work is to present an alternative General
Topology based on funcoids instead of being based on topological spaces
as it is customary. In order to study funcoids the prior knowledge
of topological spaces is not necessary. Nevertheless in this work
I will consider topological spaces and the topic of interrelation
of funcoids with topological spaces.

In fact funcoids are a generalization of topological spaces, so the
well known theory of topological spaces is a special case of the below
presented theory of funcoids.

But probably the most important reason to study funcoids is that funcoids
are a generalization of proximity spaces (see section ``\nameref{sec-prox}''
for the definition of proximity spaces). Before this work it was written
that the theory of proximity spaces was an example of a stalled research,
almost nothing interesting was discovered about this theory. It was
so because the proper way to research proximity spaces is to research
their generalization, funcoids. And so it was stalled until discovery
of funcoids. That generalized theory of proximity spaces will bring
us yet many interesting results.

In addition to \emph{funcoids} I research \emph{reloids}. Using below
defined terminology it may be said that reloids are (basically) filters
on Cartesian product of sets, and this is a special case of uniform
spaces.

Afterward we study some generalizations.

Somebody might ask, why to study it? My approach relates to traditional
general topology like complex numbers to real numbers theory. Be sure
this will find applications.

This book has a deficiency: It does not properly relate my theory
with previous research in general topology and does not consider deeper
category theory properties. It is however OK for now, as I am going
to do this study in later volumes (continuation of this book).

Many proofs in this book may seem too easy and thus this theory not
sophisticated enough. But it is largely a result of a well structured
digraph of proofs, where more difficult results are made easy by reducing
them to easier lemmas and propositions.


\section{Earlier works}

Some mathematicians were researching generalizations of proximities
and uniformities before me but they have failed to reach the right
degree of generalization which is presented in this work allowing
to represent properties of spaces with algebraic (or categorical)
formulas.

Proximity structures were introduced by Smirnov in \cite{geom-prox}.

Some references to predecessors:
\begin{itemize}
\item In \cite{gen-prox-uni1,gen-prox-uni2,some-props-prox-genunif,prox-gen-unif,compl-unif-prox}
generalized uniformities and proximities are studied.
\item Proximities and uniformities are also studied in \cite{complete-prox,prox-not-compl,compl-prox-acad,compl-prox1,compl-prox2}.
\item \cite{qunif,qunif2001} contains recent progress in quasi-uniform
spaces. \cite{qunif2001} has a very long list of related literature.
\end{itemize}
Some works (\cite{on-proximity-spaces}) about proximity spaces consider
relationships of proximities and compact topological spaces. In this
work the attempt to define or research their generalization, compactness
of funcoids or reloids is not done. It seems potentially productive
to attempt to borrow the definitions and procedures from the above
mentioned works. I hope to do this study in a separate volume.

\cite{mapping-prox} studies mappings between proximity structures.
(In this volume no attempt to research mappings between funcoids is
done.) \cite{quasi-unif-top} researches relationships of quasi-uniform
spaces and topological spaces. \cite{eq-prox-totbound-unif} studies
how proximity structures can be treated as uniform structures and
compactification regarding proximity and uniform spaces.

This book is based partially on my articles \cite{filters,funcoidsreloids,pointfree}.


\section{Kinds of continuity}

A research result based on this book but not fully included in this
book (and not yet published) is that the following kinds of continuity
are described by the same algebraic (or rather categorical) formulas
for different kinds of continuity and have common properties:
\begin{itemize}
\item discrete continuity (between digraphs);
\item (pre)topological continuity;
\item proximal continuity;
\item uniform continuity;
\item Cauchy continuity;
\item (probably other kinds of continuity).
\end{itemize}
Thus my research justifies using the same word ``continuity'' for
these diverse kinds of continuity.

See \url{http://www.mathematics21.org/algebraic-general-topology.html}


\section{Structure of this book}

In the chapter ``\nameref{chap-common}'' some well known definitions
and theories are considered. You may skip its reading if you already
know it. That chapter contains info about:
\begin{itemize}
\item posets;
\item lattices and complete lattices;
\item Galois connections;
\item co-brouwerian lattices;
\item a very short intro into category theory;
\item a very short introduction to group theory.
\end{itemize}
Afterward there are my little additions to poset/lattice and category
theory.

Afterward there is the theory of filters and filtrators.

Then there is ``\nameref{common-top}'', which considers briefly:
\begin{itemize}
\item metric spaces;
\item topological spaces;
\item pretopological spaces;
\item proximity spaces.
\end{itemize}
Despite of the name ``Common knowledge'' this second common knowledge
chapter is recommended to be read completely even if you know topology
well, because it contains some rare theorems not known to most mathematicians
and hard to find in literature.

Then the most interesting thing in this book, the theory of funcoids,
starts.

Afterwards there is the theory of reloids.

Then I show relationships between funcoids and reloids.

The last I research generalizations of funcoids, \emph{pointfree funcoids},
\emph{staroids}, and \emph{multifuncoids} and some different kinds
of products of morphisms.


\section{Basic notation}

I will denote a set definition like $\setcond{x\in A}{P(x)}$ instead
of customary $\{x\in A\mid P(x)\}$. I do this because otherwise formulas
don't fit horizontally in the available space.


\subsection{\index{Grothendieck universe}Grothendieck universes}

We will work in ZFC with an infinite and uncountable Grothendieck
universe.

A Grothendieck universe is just a set big enough to make all usual
set theory inside it. For example if $\mho$ is a Grothendieck universe,
and sets $X,Y\in\mho$, then also $X\cup Y\in\mho$, $X\cap Y\in\mho$,
$X\times Y\in\mho$, etc.

\index{small set}A set which is a member of a Grothendieck universe
is called a \emph{small set} (regarding this Grothendieck universe).
We can restrict our consideration to small sets in order to get rid
troubles with proper classes.
\begin{defn}
Grothendieck universe is a set $\mho$ such that:
\begin{enumerate}
\item If $x\in\mho$ and $y\in x$ then $y\in\mho$.
\item If $x,y\in\mho$ then $\{x,y\}\in\mho$.
\item If $x\in\mho$ then $\subsets x\in\mho$.
\item If $\setcond{x_{i}}{i\in I\in\mho}$ is a family of elements of $\mho$,
then $\bigcup_{i\in I}x_{i}\in\mho$.
\end{enumerate}
\end{defn}
One can deduce from this also:
\begin{enumerate}
\item If $x\in\mho$, then $\{x\}\in\mho$.
\item If $x$ is a subset of $y\in\mho$, then $x\in\mho$.
\item If $x,y\in\mho$ then the ordered pair $(x;y)=\{\{x,y\},x\}\in\mho$.
\item If $x,y\in\mho$ then $x\cup y$ and $x\times y$ are in $\mho$.
\item If $\setcond{x_{i}}{i\in I\in\mho}$ is a family of elements of $\mho$,
then the product $\prod_{i\in I}x_{i}\in\mho$.
\item If $x\in\mho$, then the cardinality of $x$ is strictly less than
the cardinality of~$\mho$.
\end{enumerate}

\subsection{Misc}

In this book quantifiers bind tightly. That is $\forall x\in A:P(x)\land Q$
and $\forall x\in A:P(x)\Rightarrow Q$ should be read $\left(\forall x\in A:P(x)\right)\land Q$
and $\left(\forall x\in A:P(x)\right)\Rightarrow Q$ not $\forall x\in A:\left(P(x)\land Q\right)$
and $\forall x\in A:\left(P(x)\Rightarrow Q\right)$.

The set of functions from a set $A$ to a set $B$ is denoted as $B^{A}$.

I will often skip parentheses and write $fx$ instead of $f(x)$ to
denote the result of a function $f$ acting on the argument $x$.

I will denote $\rsupfun fX=\setcond{\beta\in\im f}{\exists\alpha\in X:\alpha\mathrel f\beta}$
and $X\rsuprel fY\Leftrightarrow\exists x\in X,y\in Y:x\mathrel{f}y$
for sets $X$, $Y$ and a binary relation $f$. (Note that functions
are a special case of binary relations.)

By just $\rsupfun f$ and $\rsuprel f$ I will denote the corresponding
function and relation on small sets.
\begin{obvious}
$\{\alpha\}\rsuprel f\{\beta\}\Leftrightarrow\alpha\mathrel f\beta$
for every $\alpha$ and $\beta$.
\end{obvious}
$\mylamdba xD{f(x)}=\setcond{(x;f(x))}{x\in D}$ for a set $D$ and
and a form $f$ depending on the variable $x$.

I will denote source and destination of a morphism $f$ of any category
(See chapter~\ref{chap-common} chapter for a definition of a category.)
as $\Src f$ and $\Dst f$ correspondingly. Note that below defined
domain and image of a funcoid are not the same as it source and destination.

I will denote $\GR(A;B;f)=f$ for any morphism $(A;B;f)$ of either
$\mathbf{Set}$ or $\mathbf{Rel}$. (See definitions of $\mathbf{Set}$
and $\mathbf{Rel}$ below.)


\section{Unusual notation}

In the chapter ``\nameref{chap-common}'' (which you may skip reading
if you are already knowledgeable) some non-standard notation is defined.
I summarize here this notation for the case if you choose to skip
reading that chapter:

Partial order is denoted as $\sqsubseteq$.

Meets and joins are denoted as $\sqcap$, $\sqcup$, $\bigsqcap$,
$\bigsqcup$.

I call element $b$ \emph{substractive} from an elements $a$ (of a distributive
lattice $\mathfrak{A}$) when the difference $a\setminus b$ exists.
I call $b$ \emph{complementive} to $a$ when there exists $c\in\mathfrak{A}$
such that $b\sqcap c=\bot$ and $b\sqcup c=a$. We will prove that
$b$ is complementive to $a$ iff $b$ is substractive from $a$ and
$b\sqsubseteq a$.
\begin{defn}
Call $a$ and $b$ of a poset $\mathfrak{A}$ \emph{intersecting},
denoted $a\nasymp b$, when there exists a non-least element $c$
such that $c\sqsubseteq a\land c\sqsubseteq b$.
\end{defn}

\begin{defn}
$a\asymp b\eqdef\lnot(a\nasymp b)$.
\end{defn}

\begin{defn}
I call elements $a$ and $b$ of a poset $\mathfrak{A}$ \emph{joining}
and denote $a\equiv b$ when there are no non-greatest element $c$
such that $c\sqsupseteq a\land c\sqsupseteq b$.
\end{defn}

\begin{defn}
$a\nequiv b\eqdef\lnot(a\equiv b)$.\end{defn}
\begin{obvious}
$a\nasymp b$ iff $a\sqcap b$ is non-least, for every elements $a$,
$b$ of a meet-semilattice.
\end{obvious}

\begin{obvious}
$a\equiv b$ iff $a\sqcup b$ is the greatest element, for every elements
a, b of a join-semilattice.
\end{obvious}
I extend the definitions of pseudocomplement and dual pseudocomplement
to arbitrary posets (not just lattices as it is customary):
\begin{defn}
Let $\mathfrak{A}$ be a poset. \emph{Pseudocomplement} of $a$ is
\[
\max\setcond{c\in\mathfrak{A}}{c\asymp a}.
\]


If $z$ is the pseudocomplement of $a$ we will denote $z=a^{\ast}$.
\end{defn}

\begin{defn}
Let $\mathfrak{A}$ be a poset. \emph{Dual pseudocomplement} of $a$
is
\[
\min\setcond{c\in\mathfrak{A}}{c\equiv a}.
\]


If $z$ is the dual pseudocomplement of $a$ we will denote $z=a^{+}$.\end{defn}

