\chapter{Interior funcoids}

Having a funcoid $f$ let define \emph{interior funcoid} $f^{\circ}$.

\begin{defn}
  Let $f \in \mathsf{\mathsf{FCD}} (A , B) = \mathsf{\mathsf{pFCD}} \left(
  \mathscr{T} A , \mathscr{T} B \right)$ be a co-complete funcoid. Then
  $f^{\circ} \in \mathsf{\mathsf{pFCD}} \left( \dual \mathscr{T} A ,
  \dual \mathscr{T} B \right)$ is defined by the formula $\langle
  f^{\circ} \rangle^{\ast} X = \overline{\supfun{f}
  \overline{X}}$.
\end{defn}

\begin{prop}
  Pointfree funcoid $f^{\circ}$ exists and is unique.
\end{prop}

\begin{proof}
  $X \mapsto \overline{\supfun{f} \overline{X}}$ is a component
  of pointfree funcoid $\dual \mathscr{T} A \rightarrow \dual
  \mathscr{T} B$ iff $\supfun{f}$ is a component of the
  corresponding pointfree funcoid $\mathscr{T} A \rightarrow \mathscr{T} B$
  that is essentially component of the corresponding funcoid
  $\mathsf{\mathsf{FCD}} (A , B)$ what holds for a unique funcoid.
\end{proof}

It can be also defined for arbitrary funcoids by the formula $f^{\circ} =
(\CoCompl f)^{\circ}$.

\begin{obvious}
$f^{\circ}$ is co-complete.
\end{obvious}

\begin{thm}
  The following values are pairwise equal for a co-complete funcoid $f$ and $X
  \in \mathscr{T} \Src f$:
  \begin{enumerate}
    \item\label{int-simpl} $\rsupfun{f^{\circ}} X$;
    
    \item\label{int-set} $\setcond{ y \in \Dst f }{ \rsupfun{ f^{-
    1} } \{ y \} \sqsubseteq X }$
    
    \item\label{int-sset-set} $\bigsqcup \setcond{ Y \in \mathscr{T} \Dst f }{
    \rsupfun{f^{- 1}} Y \sqsubseteq X }$
    
    \item\label{int-sset-flt} $\bigsqcup \setcond{ \mathcal{Y} \in \mathscr{F} \Dst f
    }{ \supfun{f^{- 1}} \mathcal{Y}
    \sqsubseteq X }$
  \end{enumerate}
\end{thm}

\begin{proof}
  ~
  \begin{description}
    \item[\ref{int-simpl}=\ref{int-set}] $\setcond{ y \in \Dst f }{
    \rsupfun{f^{- 1}} \{ y \} \sqsubseteq X } = \setcond{ x \in
    \Dst f }{ \rsupfun{f^{- 1}} \{
    x \} \asymp \overline{X} } = \setcond{ x \in \Dst f
    }{ \{ x \} \asymp \supfun{f}
    \overline{X} } = \overline{\supfun{f} \overline{X}} =
    \rsupfun{f^{\circ}} X$.
    
    \item[\ref{int-set}=\ref{int-sset-set}] If $\rsupfun{f^{- 1}} Y \sqsubseteq X$ then (by
    completeness of $f^{- 1}$) $Y = \setcond{ y \in Y }{
    \rsupfun{f^{- 1}} \{ y \} \sqsubseteq X }$ and thus
    \[ \bigsqcup \setcond{ Y \in \mathscr{T} \Dst f }
       { \rsupfun{f^{- 1}} Y \sqsubseteq X }
       \sqsubseteq \setcond{ y \in \Dst f }{
       \rsupfun{f^{- 1}} \{ y \} \sqsubseteq X } . \]
    The reverse inequality is obvious.
    
    \item[\ref{int-sset-set}=\ref{int-sset-flt}] It's enough to prove that if $\supfun{f^{- 1}}
    \mathcal{Y} \sqsubseteq X$ for $\mathcal{Y} \in \mathscr{F} \Dst f$
    then exists $Y \in \up \mathcal{Y}$ such that $\langle f^{- 1}
    \rangle^{\ast} Y \sqsubseteq X$. Really let $\supfun{f^{- 1}}
    \mathcal{Y} \sqsubseteq X$. Then $\bigsqcap \langle \langle f^{- 1}
    \rangle^{\ast} \rangle^{\ast} \up \mathcal{Y} \sqsubseteq X$ and
    thus exists $Y \in \up \mathcal{Y}$ such that $\langle f^{- 1}
    \rangle^{\ast} Y \sqsubseteq X$ by properties of generalized filter bases.
  \end{description}
\end{proof}

This coincides with the customary definition of interior in topological
spaces.

\begin{prop}
  $f^{\circ \circ} = f$ for every funcoid $f$.
\end{prop}

\begin{proof}
  $\rsupfun{f^{\circ\circ}} X = \neg \neg \supfun{f}
  \neg \neg X = \supfun{f} X$.
\end{proof}

\begin{prop}\label{get-rid-interior}
  Let $g \in \mathsf{FCD} (A , B)$, $f \in \mathsf{FCD} (B ,
  C)$, $h \in \mathsf{FCD} (A , C)$ for some sets $A$. $B$, $C$.
  
  $g \sqsubseteq f^{\circ} \circ h \Leftrightarrow f^{- 1} \circ g \sqsubseteq
  h$, provided $f$ and $h$ are co-complete.
\end{prop}

\begin{proof}
  $g \sqsubseteq f^{\circ} \circ h \Leftrightarrow \forall X \in A : \rsupfun{ g
  } X \sqsubseteq \rsupfun{ f^{\circ} \circ h } X
  \Leftrightarrow \forall X \in A : \rsupfun{ g } X \sqsubseteq
  \rsupfun{ f^{\circ} } \rsupfun{ h } X \Leftrightarrow
  \forall X \in A : \rsupfun{ g } X \sqsubseteq \neg \rsupfun{ f
  } \neg \rsupfun{ h } X \Leftrightarrow
  \forall X \in A : \rsupfun{ g } X \asymp \rsupfun{ f } \neg \rsupfun{h} X \Leftrightarrow \forall X \in A : \rsupfun{ f^{- 1}
  } \rsupfun{ g } X \asymp \neg \rsupfun{ h
  } X \Leftrightarrow \forall X \in A : \rsupfun{ f^{- 1}
  } \rsupfun{ g } X \sqsubseteq \rsupfun{ h
  } X \Leftrightarrow \forall X \in A : \rsupfun{ f^{- 1} \circ g
  } X \sqsubseteq \rsupfun{ h } X \Leftrightarrow f^{-
  1} \circ g \sqsubseteq h$.
\end{proof}

\begin{rem}
The above theorem allows to get rid of interior funcoids (and use only ``regular'' funcoids) in some formulas.
\end{rem}
