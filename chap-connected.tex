
\chapter{Connectedness regarding funcoids and reloids}


\section{Some lemmas}
\begin{lem}
Let $U$ be a set, $A,B\in\mathscr{T}U$ be typed sets, $f$ be an
endo-funcoid on $U$. If $\neg(A\rsuprel fB)\wedge A\sqcup B\in\up(\dom f\sqcup\im f)$
then $f$ is closed on $A$.\end{lem}
\begin{proof}
Let $A\sqcup B\in\up(\dom f\sqcup\im f)$.
\begin{align*}
\lnot(A\rsuprel fB) & \Leftrightarrow\\
B\sqcap\rsupfun fA=\bot & \Rightarrow\\
(\dom f\sqcup\im f)\sqcap B\sqcap\rsupfun fA=\bot & \Rightarrow\\
((\dom f\sqcup\im f)\setminus A)\sqcap\rsupfun fA=\bot & \Leftrightarrow\\
\rsupfun fA\sqsubseteq A.
\end{align*}
\end{proof}
\begin{cor}
If $\neg(A\rsuprel fB)\wedge A\sqcup B\in\up(\dom f\sqcup\im f)$ then
$f$ is closed on $A\setminus B$ for a funcoid $f\in\mathsf{FCD}(U,U)$
for every sets $U$ and typed sets $A,B\in\mathscr{T}U$.\end{cor}
\begin{proof}
Let $\neg(A\rsuprel fB)\wedge A\sqcup B\in\up(\dom f\sqcup\im f)$. Then
\[
\lnot((A\setminus B)\rsuprel fB)\land(A\setminus B)\sqcup B\in\up(\dom f\sqcup\im f).
\]
\end{proof}
\begin{lem}
If $\neg(A\rsuprel fB)\wedge A\sqcup B\in\up(\dom f\sqcup\im f)$ then
$\lnot(A\rsuprel{f^{n}}B)$ for every whole positive $n$.\end{lem}
\begin{proof}
Let $\neg(A\rsuprel fB)\wedge A\sqcup B\in\up(\dom f\sqcup\im f)$. From
the above lemma $\rsupfun fA\sqsubseteq A$. $B\sqcap\supfun fA=\bot$,
consequently $\rsupfun fA\sqsubseteq A\setminus B$. Because (by the
above corollary) $f$ is closed on $A\setminus B$, then $\supfun f\supfun fA\sqsubseteq A\setminus B$,
$\supfun f\supfun f\supfun fA\sqsubseteq A\setminus B$, etc. So $\supfun{f^{n}}A\sqsubseteq A\setminus B$,
$B\asymp\supfun{f^{n}}A$, $\lnot(A\rsuprel{f^{n}}B)$.
\end{proof}

\section{Endomorphism series}
\begin{defn}
$S_{1}(\mu)=\mu\sqcup\mu^{2}\sqcup\mu^{3}\sqcup\ldots$ for an endomorphism
$\mu$ of a precategory with countable join of morphisms (that is
join defined for every countable set of morphisms).
\end{defn}

\begin{defn}
\index{endomorphism series}$S(\mu)=\mu^{0}\sqcup S_{1}(\mu)=\mu^{0}\sqcup\mu\sqcup\mu^{2}\sqcup\mu^{3}\sqcup\ldots$where
$\mu^{0}=1_{\Ob\mu}$ (identity morphism for the object $\Ob\mu$)
where $\Ob\mu$ is the object of endomorphism $\mu$ for an endomorphism
$\mu$ of a category with countable join of morphisms.
\end{defn}
I call $S_{1}$ and $S$ \emph{endomorphism series}.
\begin{prop}
The relation $S(\mu)$ is transitive for the category~$\mathbf{Rel}$.\end{prop}
\begin{proof}
~
\begin{multline*}
S(\mu)\circ S(\mu)=\mu^{0}\sqcup S(\mu)\sqcup\mu\circ S(\mu)\sqcup\mu^{2}\circ S(\mu)\sqcup\dots=\\
(\mu^{0}\sqcup\mu^{1}\sqcup\mu^{2}\sqcup\dots)\sqcup(\mu^{1}\sqcup\mu^{2}\sqcup\mu^{3}\sqcup\dots)\sqcup(\mu^{2}\sqcup\mu^{3}\sqcup\mu^{4}\sqcup\dots)=\\
\mu^{0}\sqcup\mu^{1}\sqcup\mu^{2}\sqcup\dots=S(\mu).
\end{multline*}

\end{proof}

\section{Connectedness regarding binary relations}

Before going to research connectedness for funcoids and reloids we
will excurse into the basic special case of connectedness regarding
binary relations on a set~$\mho$.

This is commonly studied in ``graph theory'' courses. \emph{Digraph}
as commonly defined is essentially the same as an endomorphism of
the category~$\mathbf{Rel}$.
\begin{defn}
\index{connectedness!regarding binary relation}A set~$A$ is called
\emph{(strongly) connected} regarding a binary relation~$\mu$ on~$U$
when
\[
\forall X,Y\in\subsets U\setminus\{\emptyset\}:(X\cup Y=A\Rightarrow X\rsuprel{\mu}Y).
\]

\end{defn}

\begin{defn}
\index{connectedness!regarding mathbf{Rel}
-endomorphism@regarding $\mathbf{Rel}$-endomorphism}A typed set~$A$ of type~$U$ is called \emph{(strongly) connected}
regarding a $\mathbf{Rel}$-endomorphism~$\mu$ on~$U$ when
\[
\forall X,Y\in\mathscr{T}(\Ob\mu)\setminus\{\bot^{\mathscr{T}(\Ob\mu)}\}:(X\sqcup Y=A\Rightarrow X\rsuprel{\mu}Y).
\]
\end{defn}
\begin{obvious}
A typed set~$A$ is connected regarding $\mathbf{Rel}$-endomorphism~$\mu$
on its type iff $\GR A$ is connected regarding $\GR\mu$.
\end{obvious}
Let $\mho$ be a set.
\begin{defn}
\index{path}\emph{Path} between two elements $a,b\in\mho$ in a set
$A\subseteq\mho$ through binary relation $\mu$ is the finite sequence
$x_{0}\ldots x_{n}$ where $x_{0}=a$, $x_{n}=b$ for $n\in\mathbb{N}$
and $x_{i}\mathrel{(\mu\cap A\times A)}x_{i+1}$ for every $i=0,\ldots,n-1$.
$n$ is called path \emph{length}.\end{defn}
\begin{prop}
There exists path between every element $a\in\mho$ and that element
itself.\end{prop}
\begin{proof}
It is the path consisting of one vertex (of length $0$).\end{proof}
\begin{prop}
There is a path from element $a$ to element $b$ in a set $A$ through
a binary relation $\mu$ iff $a\mathrel{(S(\mu\cap A\times A))}b$
(that is $(a,b)\in S(\mu\cap A\times A)$).\end{prop}
\begin{proof}
~
\begin{description}
\item [{$\Rightarrow$}] If a path from $a$ to $b$ exists, then $\{b\}\subseteq\rsupfun{(\mu\cap A\times A)^{n}}\{a\}$
where $n$ is the path length. Consequently $\{b\}\subseteq\rsupfun{S(\mu\cap A\times A)}\{a\}$;
$a\mathrel{(S(\mu\cap A\times A))}b$.
\item [{$\Leftarrow$}] If $a\mathrel{(S(\mu\cap A\times A))}b$ then there
exists $n\in\mathbb{N}$ such that $a\mathrel{(\mu\cap A\times A)^{n}}b$.
By definition of composition of binary relations this means that there
exist finite sequence $x_{0}\ldots x_{n}$ where $x_{0}=a$, $x_{n}=b$
for $n\in\mathbb{}{N}$ and $x_{i}\mathrel{(\mu\cap A\times A)}x_{i+1}$
for every $i=0,\ldots,n-1$. That is there is a path from $a$ to
$b$.
\end{description}
\end{proof}
\begin{prop}
There is a path from element $a$ to element $b$ in a set $A$ through
a binary relation $\mu$ iff $a\mathrel{(S_1(\mu\cap A\times A))}b$
(that is $(a,b)\in S_1(\mu\cap A\times A)$).\end{prop}
\begin{proof}
Similar to the previous proof.
\end{proof}

\begin{thm}
The following statements are equivalent for a binary relation $\mu$
and a set $A$:
\begin{enumerate}
\item \label{rconn-path}For every $a,b\in A$ there is
a nonzero-length path between $a$ and $b$ in $A$ through $\mu$.
\item \label{rconn-ge}$S_1(\mu\cap(A\times A))\supseteq A\times A$.
\item \label{rconn-eq}$S_1(\mu\cap(A\times A))=A\times A$.
\item \label{rconn-conn}$A$ is connected regarding $\mu$.
\end{enumerate}
\end{thm}
\begin{proof}
~
\begin{description}
\item [{\ref{rconn-path}$\Rightarrow$\ref{rconn-ge}}] Let for every
$a,b\in A$ there is a nonzero-length path between $a$ and $b$ in $A$ through
$\mu$. Then $a\mathrel{(S_1(\mu\cap A\times A))}b$ for every $a,b\in A$.
It is possible only when $S_1(\mu\cap(A\times A))\supseteq A\times A$.
\item [{\ref{rconn-eq}$\Rightarrow$\ref{rconn-path}}] For every two
vertices $a$ and $b$ we have $a\mathrel{(S_1(\mu\cap A\times A))}b$.
So (by the previous) for every two vertices $a$ and $b$
there exists a nonzero-length path from $a$ to $b$.
\item [{\ref{rconn-eq}$\Rightarrow$\ref{rconn-conn}}] Suppose $\lnot(X\rsuprel{\mu\cap(A\times A)}Y)$
for some $X,Y\in\mathscr{P}\mho\setminus\{\emptyset\}$ such that
$X\cup Y=A$. Then by a lemma $\lnot(X\rsuprel{(\mu\cap(A\times A))^{n}}Y)$
for every $n\in\mathbb{Z}_+$. Consequently $\lnot(X\rsuprel{S_1(\mu\cap(A\times A))}Y)$.
So $S_1(\mu\cap(A\times A))\ne A\times A$.
\item [{\ref{rconn-conn}$\Rightarrow$\ref{rconn-eq}}] If $\rsupfun{S_1(\mu\cap(A\times A))}\{v\}=A$
for every vertex $v$ then $S_1(\mu\cap(A\times A))=A\times A$. Consider
the remaining case when $V\eqdef\rsupfun{S_1(\mu\cap(A\times A))}\{v\}\subset A$
for some vertex $v$. Let $W=A\setminus V$. If $\card A=1$ then
$S_1(\mu\cap(A\times A))\supseteq\id_{A}=A\times A$; otherwise $W\neq\emptyset$.
Then $V\cup W=A$ and so $V\rsuprel{\mu}W$ what is equivalent to
$V\rsuprel{\mu\cap(A\times A)}W$ that is $\rsupfun{\mu\cap(A\times A)}V\cap W\ne\emptyset$.
This is impossible because
\begin{multline*}
\rsupfun{\mu\cap(A\times A)}V=\rsupfun{\mu\cap(A\times A)}\rsupfun{S_1(\mu\cap(A\times A))}V=\\
\rsupfun{(\mu\cap(A\times A))^2\cup (\mu\cap(A\times A))^3\cup\dots\cup}V \subseteq
\rsupfun{S_1(\mu\cap(A\times A))}V=V.
\end{multline*}

\item [{\ref{rconn-ge}$\Rightarrow$\ref{rconn-eq}}] Because $S_1(\mu\cap(A\times A))\subseteq A\times A$.
\end{description}
\end{proof}
\begin{cor}
A set $A$ is connected regarding a binary relation $\mu$ iff it
is connected regarding $\mu\cap(A\times A)$.\end{cor}
\begin{defn}
\index{connected component}A \emph{connected component} of a set
$A$ regarding a binary relation $F$ is a maximal connected subset
of $A$.\end{defn}
\begin{thm}
The set $A$ is partitioned into connected components (regarding every
binary relation $F$).\end{thm}
\begin{proof}
Consider the binary relation $a\sim b\Leftrightarrow a\mathrel{(S(F))}b\wedge b\mathrel{(S(F))}a$.
$\sim$ is a symmetric, reflexive, and transitive relation. So all
points of $A$ are partitioned into a collection of sets $Q$. Obviously
each component is (strongly) connected. If a set $R\subseteq A$ is
greater than one of that connected components $A$ then it contains
a point $b\in B$ where $B$ is some other connected component. Consequently
$R$ is disconnected.\end{proof}
\begin{prop}
A set is connected (regarding a binary relation) iff it has one connected
component.\end{prop}
\begin{proof}
Direct implication is obvious. Reverse is proved by contradiction.
\end{proof}

\section{Connectedness regarding funcoids and reloids}
\begin{defn}
\index{connectivity reloid}\emph{Connectivity reloid} $S_{1}^{\ast}(\mu)=\bigsqcap_{M\in\up\mu}^{\mathsf{RLD}}S_{1}(M)$
for an endoreloid~$\mu$.
\end{defn}

\begin{defn}
$S^{\ast}(\mu)$
for an endoreloid $\mu$ is defined as follows:
\[
S^{\ast}(\mu)=\bigsqcap_{M\in\up\mu}^{\mathsf{RLD}}S(M).
\]

\end{defn}
Do not mess the word \emph{connectivity} with the word \emph{connectedness}
which means being connected.\footnote{In some math literature these two words are used interchangeably.}
\begin{prop}
$S^{\ast}(\mu)=1_{\Ob\mu}^{\mathsf{RLD}}\sqcup S_{1}^{\ast}(\mu)$
for every endoreloid $\mu$.\end{prop}
\begin{proof}
By the proposition \ref{b-f-back-distr}.\end{proof}
\begin{prop}
$S^{\ast}(\mu)=S(\mu)$ and $S^{\ast}_1(\mu)=S_1(\mu)$ if $\mu$ is a principal reloid.\end{prop}
\begin{proof}
$S^{\ast}(\mu)=\bigsqcap\{S(\mu)\}=S(\mu)$;
$S^{\ast}_1(\mu)=\bigsqcap\{S_1(\mu)\}=S_1(\mu)$.
\end{proof}
\begin{defn}
\index{connected!regarding endoreloid}A filter $\mathcal{A}\in\mathscr{F}(\Ob\mu)$
is called \emph{connected} regarding an endoreloid $\mu$ when $S^{\ast}_1(\mu\sqcap(\mathcal{A}\times^{\mathsf{RLD}}\mathcal{A}))\sqsupseteq\mathcal{A}\times^{\mathsf{RLD}}\mathcal{A}$.\end{defn}
\begin{obvious}
A filter $\mathcal{A}\in\mathscr{F}(\Ob\mu)$ is connected regarding
an endoreloid $\mu$ iff $S^{\ast}_1(\mu\sqcap(\mathcal{A}\times^{\mathsf{RLD}}\mathcal{A}))=\mathcal{A}\times^{\mathsf{RLD}}\mathcal{A}$.\end{obvious}
\begin{defn}
\index{connected!regarding endofuncoid}A filter $\mathcal{A}\in\mathscr{F}(\Ob\mu)$
is called \emph{connected} regarding an endofuncoid $\mu$ when
\[
\forall\mathcal{X},\mathcal{Y}\in\mathscr{F}(\Ob\mu)\setminus\{\bot^{\mathscr{F}(\Ob\mu)}\}:(\mathcal{X}\sqcup\mathcal{Y}=\mathcal{A}\Rightarrow\mathcal{X}\suprel{\mu}\mathcal{Y}).
\]
\end{defn}
\begin{prop}
Let $A$ be a typed set of type~$U$. The filter $\uparrow A$ is
connected regarding an endofuncoid $\mu$ on~$U$ iff
\[
\forall X,Y\in\mathscr{T}(\Ob\mu)\setminus\{\bot^{\mathscr{T}(\Ob\mu)}\}:(X\sqcup Y=A\Rightarrow X\rsuprel{\mu}Y).
\]
\end{prop}
\begin{proof}
~
\begin{description}
\item [{$\Rightarrow$}] Obvious.
\item [{$\Leftarrow$}] It follows from co-separability of filters.
\end{description}
\end{proof}
\begin{thm}
The following are equivalent for every typed set $A$ of type~$U$
and $\mathbf{Rel}$-endomorphism~$\mu$ on a set $U$:
\begin{enumerate}
\item \label{princ-conn-rel}$A$ is connected regarding~$\mu$.
\item \label{princ-conn-rld}$\uparrow A$ is connected regarding $\uparrow^{\mathsf{RLD}}\mu$.
\item \label{princ-conn-fcd}$\uparrow A$ is connected regarding $\uparrow^{\mathsf{FCD}}\mu$.
\end{enumerate}
\end{thm}
\begin{proof}
~
\begin{description}
\item [{\ref{princ-conn-rel}$\Leftrightarrow$\ref{princ-conn-rld}}] ~
\begin{align*}
S^{\ast}_1(\uparrow^{\mathsf{RLD}}\mu\sqcap(A\times^{\mathsf{RLD}}A)) & =\\
S^{\ast}_1(\uparrow^{\mathsf{RLD}}(\mu\sqcap(A\times A))) & =\\
\uparrow^{\mathsf{RLD}}S_1(\mu\sqcap(A\times A)).
\end{align*}
So
\begin{align*}
S^{\ast}_1(\uparrow^{\mathsf{RLD}}\mu\sqcap(A\times^{\mathsf{RLD}}A))\sqsupseteq A\times^{\mathsf{RLD}}A & \Leftrightarrow\\
\uparrow^{\mathsf{RLD}}S_1(\mu\sqcap(A\times A))\sqsupseteq\uparrow^{\mathsf{RLD}}(A\times A)=A\times^{\mathsf{RLD}}A.
\end{align*}

\item [{\ref{princ-conn-rel}$\Leftrightarrow$\ref{princ-conn-fcd}}] It
follows from the previous proposition.
\end{description}
\end{proof}
Next is conjectured a statement more strong than the above theorem:
\begin{conjecture}
Let $\mathcal{A}$ be a filter on a set $U$ and $F$ be a $\mathbf{Rel}$-endomorphism
on $U$.

$\mathcal{A}$ is connected regarding $\uparrow^{\mathsf{FCD}}F$
iff $\mathcal{A}$ is connected regarding $\uparrow^{\mathsf{RLD}}F$.\end{conjecture}
\begin{obvious}
A filter $\mathcal{A}$ is connected regarding a reloid $\mu$ iff
it is connected regarding the reloid $\mu\sqcap(\mathcal{A}\times^{\mathsf{RLD}}\mathcal{A})$.
\end{obvious}

\begin{obvious}
A filter $\mathcal{A}$ is connected regarding a funcoid $\mu$ iff
it is connected regarding the funcoid $\mu\sqcap(\mathcal{A}\times^{\mathsf{FCD}}\mathcal{A})$.\end{obvious}
\begin{thm}
A filter $\mathcal{A}$ is connected regarding a reloid $f$ iff $\mathcal{A}$
is connected regarding every $F\in\rsupfun{\uparrow^{\mathsf{RLD}}}\up f$.\end{thm}
\begin{proof}
~
\begin{description}
\item [{$\Rightarrow$}] Obvious.
\item [{$\Leftarrow$}] $\mathcal{A}$ is connected regarding $\uparrow^{\mathsf{RLD}}F$
iff $S_1(F)=F^{1}\sqcup F^{2}\sqcup\dots\in\up(\mathcal{A}\times^{\mathsf{RLD}}\mathcal{A})$.


$S^{\ast}_1(f)=\bigsqcap_{F\in\up f}^{\mathsf{RLD}}S_1(F)\sqsupseteq\bigsqcap_{F\in\up f}(\mathcal{A}\times^{\mathsf{RLD}}\mathcal{A})=\mathcal{A}\times^{\mathsf{RLD}}\mathcal{A}$.

\end{description}
\end{proof}
\begin{conjecture}
A filter $\mathcal{A}$ is connected regarding a funcoid $f$ iff
$\mathcal{A}$ is connected regarding every $F\in\rsupfun{\uparrow^{\mathsf{FCD}}}\up f$.
\end{conjecture}
The above conjecture is open even for the case when $\mathcal{A}$
is a principal filter.
\begin{conjecture}
A filter $\mathcal{A}$ is connected regarding a reloid $f$ iff it
is connected regarding the funcoid $\tofcd f$.
\end{conjecture}
The above conjecture is true in the special case of principal filters:
\begin{prop}
A filter $\uparrow A$ (for a typed set $A$) is connected regarding
an endoreloid $f$ on the suitable object iff it is connected regarding the endofuncoid $\tofcd f$.\end{prop}
\begin{proof}
$\uparrow A$ is connected regarding a reloid $f$ iff $A$
is connected regarding every $F\in\up f$ that is when (taken into
account that connectedness for $\uparrow^{\mathsf{RLD}}F$ is the
same as connectedness of $\uparrow^{\mathsf{FCD}}F$)
\begin{align*}
\forall F\in\up f\forall\mathcal{X},\mathcal{Y}\in\mathscr{F}(\Ob f)\setminus\{\bot^{\mathscr{F}(\Ob f)}\}:(\mathcal{X}\sqcup\mathcal{Y}=\uparrow A\Rightarrow\mathcal{X}\suprel{\uparrow^{\mathsf{FCD}}F}\mathcal{Y}) & \Leftrightarrow\\
\forall\mathcal{X},\mathcal{Y}\in\mathscr{F}(\Ob f)\setminus\{\bot^{\mathscr{F}(\Ob f)}\}\forall F\in\up f:(\mathcal{X}\sqcup\mathcal{Y}=\uparrow A\Rightarrow\mathcal{X}\suprel{\uparrow^{\mathsf{FCD}}F}\mathcal{Y}) & \Leftrightarrow\\
\forall\mathcal{X},\mathcal{Y}\in\mathscr{F}(\Ob f)\setminus\{\bot^{\mathscr{F}(\Ob f)}\}(\mathcal{X}\sqcup\mathcal{Y}=\uparrow A\Rightarrow\forall F\in\up f:\mathcal{X}\suprel{\uparrow^{\mathsf{FCD}}F}\mathcal{Y}) & \Leftrightarrow\\
\forall\mathcal{X},\mathcal{Y}\in\mathscr{F}(\Ob f)\setminus\{\bot^{\mathscr{F}(\Ob f)}\}(\mathcal{X}\sqcup\mathcal{Y}=\uparrow A\Rightarrow\mathcal{X}\suprel{\tofcd f}\mathcal{Y})
\end{align*}
that is when the set $\uparrow A$ is connected regarding
the funcoid $\tofcd f$.\end{proof}
\begin{conjecture}
A set $A$ is connected regarding an endofuncoid $\mu$ iff for every
$a,b\in A$ there exists a totally ordered set $P\subseteq A$ such
that $\min P=a$, $\max P=b$ and
\[
\forall q\in P\setminus\{b\}:\setcond{x\in P}{x\le q}\rsuprel{\mu}\setcond{x\in P}{x>q}.
\]


Weaker condition:
\[
\forall q\in P\setminus\{b\}:\setcond{x\in P}{x\le q}\rsuprel{\mu}\setcond{x\in P}{x>q}\lor\forall q\in P\setminus\{a\}:\setcond{x\in P}{x<q}\rsuprel{\mu}\setcond{x\in P}{x\ge q}.
\]

\end{conjecture}

\section{\texorpdfstring{Algebraic properties of $S$ and $S^{\ast}$}%
{Algebraic properties of S and S*}}
\begin{thm}
$S^{\ast}(S^{\ast}(f))=S^{\ast}(f)$ for every endoreloid $f$.\end{thm}
\begin{proof}
~
\begin{align*}
S^{\ast}(S^{\ast}(f)) & =\\
\bigsqcap_{R\in\up S^{\ast}(f)}^{\mathsf{RLD}}S(R) & \sqsubseteq\\
\bigsqcap_{R\in\setcond{S(F)}{F\in\up f}}^{\mathsf{RLD}}S(R) & =\\
\bigsqcap_{R\in\up f}^{\mathsf{RLD}}S(S(R)) & =\\
\bigsqcap_{R\in\up f}^{\mathsf{RLD}}S(R) & =\\
S^{\ast}(f).
\end{align*}


So $S^{\ast}(S^{\ast}(f))\sqsubseteq S^{\ast}(f)$. That $S^{\ast}(S^{\ast}(f))\sqsupseteq S^{\ast}(f)$
is obvious.\end{proof}
\begin{cor}
$S^{\ast}(S(f))=S(S^{\ast}(f))=S^{\ast}(f)$ for every endoreloid
$f$.\end{cor}
\begin{proof}
Obviously $S^{\ast}(S(f))\sqsupseteq S^{\ast}(f)$ and $S(S^{\ast}(f))\sqsupseteq S^{\ast}(f)$.

But $S^{\ast}(S(f))\sqsubseteq S^{\ast}(S^{\ast}(f))=S^{\ast}(f)$
and $S(S^{\ast}(f))\sqsubseteq S^{\ast}(S^{\ast}(f))=S^{\ast}(f)$.\end{proof}
\begin{conjecture}
$S(S(f))=S(f)$ for
\begin{enumerate}
\item every endoreloid $f$;
\item every endofuncoid $f$.
\end{enumerate}
\end{conjecture}

\begin{conjecture}
$S(f)\circ S(f)=S(f)$ for every endoreloid $f$.
\end{conjecture}

\begin{thm}
$S^{\ast}(f)\circ S^{\ast}(f) = S(f)\circ S^{\ast}(f)=S^{\ast}(f)\circ S(f)=S^{\ast}(f)$ for every endoreloid $f$.
\end{thm}

\begin{proof}
\footnote{Can be more succintly proved considering $\mu \mapsto S^{\ast} (\mu)$ as a
pointfree funcoid?}

It is enough to prove $S^{\ast}(f)\circ S^{\ast}(f) = S^{\ast}(f)$ because
$S^{\ast}(f) \sqsubseteq S(f)\circ S^{\ast}(f) \sqsubseteq S^{\ast}(f)\circ S^{\ast}(f)$ and likewise for $S^{\ast}(f)\circ S(f)$.

$S^{\ast} (\mu) \circ S^{\ast} (\mu) = \bigsqcap^{\mathsf{RLD}}_{F
\in \up S^{\ast} (\mu)} (F \circ F) = \text{(see below)} =
\bigsqcap^{\mathsf{RLD}}_{X \in \up \mu} (S (X) \circ S (X)) =
\bigsqcap^{\mathsf{RLD}}_{X \in \up \mu} S (X) = S^{\ast}
(\mu)$.

$F \in \up S^{\ast} (\mu)
\Leftrightarrow
F \in \up \bigsqcap^{\mathscr{F}}_{F\in\up\mu} S(F)
\Rightarrow
\text{(by properties of filter bases)}
\Rightarrow \exists X \in \up \mu : F
\sqsupseteq S (X)
\Rightarrow \exists X \in \up \mu : F \circ F
\sqsupseteq S (X) \circ S (X)$ thus
\[ \bigsqcap^{\mathsf{RLD}}_{
    F \in \up S^{\ast} (\mu) } F \circ F \sqsupseteq
    \bigsqcap^{\mathsf{RLD}}_{ X \in \up \mu } (S (X) \circ S (X)) ; \]
$X \in \up \mu \Rightarrow S (X) \in \up S^{\ast} (\mu)
\Rightarrow \exists F \in \up S^{\ast} (\mu) : S (X) \circ S (X)
\sqsupseteq F \circ F$ thus
\[ \bigsqcap^{\mathsf{RLD}}_{
    F \in \up S^{\ast} (\mu) } F \circ F \sqsubseteq
    \bigsqcap^{\mathsf{RLD}}_{ X \in \up \mu } (S (X) \circ S (X)) . \]
\end{proof}

\begin{conjecture}
$S(f)\circ S(f)=S(f)$ for every endofuncoid $f$.\end{conjecture}

\section{Irreflexive reloids}

\begin{defn}
Endoreloid~$f$ is irreflexive iff $f\asymp 1^{\Ob f}$.
\end{defn}

\begin{prop}
Endoreloid~$f$ is irreflexive iff $f\sqsubseteq \top\setminus 1$.
\end{prop}

\begin{proof}
By theorem \ref{f-intrs-and-compl}.
\end{proof}

\begin{obvious}
$f \setminus 1$ is an irreflexive endoreloid if $f$ is an endoreloid.
\end{obvious}

\begin{prop}
$S (f) = S (f \sqcup 1)$ if~$f$ is an endoreloid,
endofuncoid, or endorelation.
\end{prop}

\begin{proof}
First prove $(f \sqcup 1)^n = 1 \sqcup f \sqcup \ldots \sqcup f^n$ for $n \in
\mathbb{N}$. For $n = 0$ it's obvious. By induction we have
\begin{align*}
(f\sqcup 1)^{n+1} &= \\
(f\sqcup 1)^n\circ(f\sqcup 1) &= \\
(1\sqcup f\sqcup\dots\sqcup f^n)\circ(f\sqcup 1) &= \\
(f\sqcup f^2\sqcup\dots\sqcup f^{n+1}) \sqcup (1\sqcup f\sqcup\dots\sqcup f^n) &= \\
1\sqcup f\sqcup\dots\sqcup f^{n+1}.
\end{align*}
So $S (f \sqcup 1) = 1 \sqcup (1 \sqcup f) \sqcup (1 \sqcup f \sqcup f^2) \sqcup
\ldots = 1 \sqcup f \sqcup f^2 \sqcup \ldots = S (f)$.
\end{proof}

\begin{cor}
$S (f) = S (f \sqcup 1) = S (f \setminus 1)$ if~$f$ is an endoreloid
(or just an endorelation).
\end{cor}

\begin{proof}
$S (f \setminus 1) = S ((f \setminus 1) \sqcup 1) \sqsupseteq S (f)$. But $S
(f \setminus 1) \sqsubseteq S (f)$ is obvious. So $S (f \setminus 1) = S (f)$.
\end{proof}

\section{Micronization}

``Micronization'' was a thoroughly wrong idea with several errors in
the proofs. This section is removed from the book.
