\chapter{Unfixed funcoids}

\section{More on composition}

\begin{lem}
For every sets $A$, $B$, $C$
\begin{enumerate}
\item $f\circ(g\sqcup h)=f\circ g\sqcup f\circ h$ for $g,h\in\mathsf{FCD}(A,B)$,
$f\in\mathsf{FCD}(B,C)$;
\item $(g\sqcup h)\circ f=g\circ f\sqcup h\circ f$ for
$g,h\in\mathsf{FCD}(B,C)$,
$f\in\mathsf{FCD}(A,B)$.
\end{enumerate}
\end{lem}

\begin{proof}
Theorem~\ref{pffcd-assc-j}.
\end{proof}

Another proof of the above lemma (without atomic filters):

\begin{proof}
~
\begin{align*}
\supfun{f\circ(g\sqcup h)}\mathcal{X} & =\\
\supfun f\supfun{g\sqcup h}\mathcal{X} & =\\
\supfun f(\supfun g\mathcal{X}\sqcup\supfun h\mathcal{X}) & =\\
\supfun f\supfun g\mathcal{X}\sqcup\supfun f\supfun h\mathcal{X} & =\\
\supfun{f\circ g}\mathcal{X}\sqcup\supfun{f\circ h}\mathcal{X} & =\\
\supfun{f\circ g\sqcup f\circ h}\mathcal{X}.
\end{align*}
\end{proof}

\begin{thm}
For unfixed funcoids:
\begin{enumerate}
\item $f\circ(g\sqcup h)=f\circ g\sqcup f\circ h$;
\item $(g\sqcup h)\circ f=g\circ f\sqcup h\circ f$.
\end{enumerate}
\end{thm}

\begin{proof}
Proposition~\ref{f-circcup-unfix}.
\end{proof}

\section{Domain and range of a funcoid}

\begin{thm}
$\operatorname{Im}f=\im f$ for every funcoid~$f$.
\end{thm}

\begin{proof}
For $Y\in\mathscr{T}\Dst f$ we have
$Y\in\im f\Leftrightarrow
Y\in\supfun{f}\top^{\mathscr{F}(\Src f)}\Leftrightarrow
\top\times Y\sqsupseteq f\Leftrightarrow
(\top\times Y)\sqcap f=f\Leftrightarrow
\id^{\mathcal{C}(\Dst f)}_Y \circ f = f
\Leftrightarrow
\id^{\mathcal{C}(\Dst f,\Dst f)}_{[Y]} \circ f = f
\Leftrightarrow
\id^{\mathcal{C}(\Dst f,\Dst f)}_{[Y]\sqcap[\Dst f]} \circ f = f \Leftrightarrow
Y\in\operatorname{Im}f$.
\end{proof}

\begin{defn}
Image of unfixed funcoid can be defined as:
$\im f = [\im F]$ for $F\in f$.
\end{defn}

Need to prove it does not depend on the choice of~$F$.

\begin{proof}
Need to prove $\im F_0\sim\im F_1$ if $F_0\sim F_1$. But it follows from
\begin{multline*}
\im F_0\sim\im\iota_{\dom F_0\sqcup\dom F_1,\im F_0\sqcup\im F_1}F_0=\\\im\iota_{\dom F_0\sqcup\dom F_1,\im F_0\sqcup\im F_1}F_1\sim\im F_1.
\end{multline*}
\end{proof}

\begin{thm}
$\im f = \operatorname{IM}f = \operatorname{IM}F = \im\mathscr{S}f$ for every unfixed funcoid~$f$.
\end{thm}

\begin{proof}
??
\end{proof}

\begin{obvious}
For every morphism $f\in\mathbf{Rel}(A,B)$ for sets~$A$ and~$B$
\begin{enumerate}
\item $\im\uparrow^{\mathsf{FCD}}f=\uparrow\im f$;
\item $\dom\uparrow^{\mathsf{FCD}}f=\uparrow\dom f$.
\end{enumerate}
\end{obvious}

\begin{prop}
For every binary relation~$f$
\begin{enumerate}
\item $\im\uparrow^{\mathsf{FCD}}f=\uparrow\im f$;
\item $\dom\uparrow^{\mathsf{FCD}}f=\uparrow\dom f$.
\end{enumerate}
\fxwarning{Define image of unfixed funcoid.}
\end{prop}

\begin{proof}
??
\end{proof}

\begin{prop}
$\supfun f\mathcal{X}=\supfun f(\mathcal{X}\sqcap\dom f)$ for every
funcoid $f$, $\mathcal{X}\in\mathscr{F}(\Src f)$.
\end{prop}

Easily follows from proposition~\ref{pf-cap-dom} and the above theorem, or can be proved directly:

\begin{proof}
For every $\mathcal{Y}\in\mathscr{F}(\Dst f)$ we have
\begin{align*}
\mathcal{Y}\sqcap\supfun f(\mathcal{X}\sqcap\dom f)\ne\bot
& \Leftrightarrow\\
\mathcal{X}\sqcap\dom f\sqcap\supfun{f^{-1}}\mathcal{Y}\ne\bot & \Leftrightarrow\\
\mathcal{X}\sqcap\im
f^{-1}\sqcap\supfun{f^{-1}}\mathcal{Y}\ne\bot &
\Leftrightarrow\\
\mathcal{X}\sqcap\supfun{f^{-1}}\mathcal{Y}\ne\bot &
\Leftrightarrow\\
\mathcal{Y}\sqcap\supfun f\mathcal{X}\ne\bot.
\end{align*}


Thus $\supfun f(\mathcal{X}\sqcap\dom f)=\supfun f\mathcal{X}$ because
the lattice of filters is separable.\end{proof}
\begin{prop}
$\supfun f\mathcal{X}=\im(f|_{\mathcal{X}})$ for every funcoid $f$,
$\mathcal{X}\in\mathscr{F}(\Src f)$.\end{prop}
\begin{proof}
~
\begin{align*}
\im(f|_{\mathcal{X}}) & =\\
\supfun{f\circ\id_{\mathcal{X}}^{\mathsf{FCD}}}\top & =\\
\supfun f\supfun{\id_{\mathcal{X}}^{\mathsf{FCD}}}\top &
=\\
\supfun f(\mathcal{X}\sqcap\top) & =\\
\supfun f\mathcal{X}.
\end{align*}
\end{proof}
\begin{prop}\label{x-dom-fcd}
$\mathcal{X}\sqcap\dom f\ne\bot\Leftrightarrow\supfun
f\mathcal{X}\ne\bot$
for every funcoid $f$ and $\mathcal{X}\in\mathscr{F}(\Src f)$.\end{prop}
\begin{proof}
~
\begin{align*}
\mathcal{X}\sqcap\dom f\ne\bot & \Leftrightarrow\\
\mathcal{X}\sqcap\supfun{f^{-1}}\top^{\mathscr{F}(\Dst
f)}\ne\bot & \Leftrightarrow\\
\top\sqcap\supfun f\mathcal{X}\ne\bot & \Leftrightarrow\\
\supfun f\mathcal{X}\ne\bot.
\end{align*}
\end{proof}
\begin{cor}\label{dom-fcd-at}
$\dom f=\bigsqcup\setcond{a\in\atoms^{\mathscr{F}(\Src f)}}{\supfun
fa\ne\bot}$ for a funcoid~$f$.
\end{cor}
\begin{proof}
This follows from the fact that $\mathscr{F}(\Src f)$ is an atomistic
lattice.\end{proof}
\begin{prop}
$\dom(f|_{\mathcal{A}})=\mathcal{A}\sqcap\dom f$ for every funcoid
$f$ and $\mathcal{A}\in\mathscr{F}(\Src f)$.\end{prop}
\begin{proof}
~

\begin{align*}
\dom(f|_{\mathcal{A}}) & =\\
\im(\id_{\mathcal{A}}^{\mathsf{FCD}}\circ f^{-1}) & =\\
\supfun{\id_{\mathcal{A}}^{\mathsf{FCD}}}\supfun{f^{-1}}\top & =\\
\mathcal{A}\sqcap\supfun{f^{-1}}\top & =\\
\mathcal{A}\sqcap\dom f.
\end{align*}
\end{proof}

\begin{cor}
$\dom(f|_{\mathcal{A}})=\mathcal{A}\sqcap\dom f$ for every unfixed funcoid~$f$
and unfixed filter~$\mathcal{A}$.
\end{cor}

\begin{proof}
??
\end{proof}

\begin{thm}
$\im f=\bigsqcap^{\mathscr{F}}\rsupfun{\im}\up f$ and $\dom
f=\bigsqcap^{\mathscr{F}}\rsupfun{\dom}\up f$
for every funcoid $f$.\end{thm}
\begin{proof}
~
\begin{align*}
\im f & =\\
\supfun f\top & =\\
\bigsqcap_{F\in\up f}^{\mathscr{F}}\supfun F\top & =\\
\bigsqcap_{F\in\up f}^{\mathscr{F}}\im F & =\\
\bigsqcap^{\mathscr{F}}\rsupfun{\im}\up f.
\end{align*}


The second formula follows from symmetry.\end{proof}

\section{Specifying funcoids by functions or relations on atomic filters}
\begin{cor}
~Let $f$ be a funcoid.
\begin{itemize}
\item The value of $f$ can be restored from the value of $\supfun
f|_{\atoms^{\mathscr{F}(\Src f)}}$.
\item The value of $f$ can be restored from the value of $\mathord{\suprel
f}|_{\atoms^{\mathscr{F}(\Src f)}\times\atoms^{\mathscr{F}(\Dst f)}}$.
\end{itemize}
\end{cor}
\begin{thm}
\label{cont-fcd-on-atoms}Let $A$ and $B$ be sets.
\begin{enumerate}
\item \label{at-restr-f}A function
$\alpha\in\mathscr{F}(B)^{\atoms^{\mathscr{F}(A)}}$
such that (for every $a\in\atoms^{\mathscr{F}(A)}$)
\begin{equation}
\alpha
a\sqsubseteq\bigsqcap\rsupfun{\bigsqcup\circ\rsupfun{\alpha}
\circ\atoms\circ\uparrow}\up a\label{at-func-cond}
\end{equation}
can be continued to the function $\supfun f$ for a unique
$f\in\mathsf{FCD}(A,B)$;
\begin{equation}
\supfun
f\mathcal{X}=\bigsqcup\rsupfun{\alpha}\atoms\mathcal{X}\label{at-func-eq}
\end{equation}
for every $\mathcal{X}\in\mathscr{F}(A)$.

\item \label{at-restr-r}A relation
$\delta\in\subsets(\atoms^{\mathscr{F}(A)}\times\atoms^{\mathscr{F}(B)})$
such that (for every $a\in\atoms^{\mathscr{F}(A)}$,
$b\in\atoms^{\mathscr{F}(B)}$)
\begin{equation}
\forall X\in\up a,Y\in\up b\exists x\in\atoms\uparrow X,y\in\atoms\uparrow
Y:x\mathrel\delta y\Rightarrow a\mathrel\delta b\label{at-rel-cond}
\end{equation}
can be continued to the relation $\suprel f$ for a unique
$f\in\mathsf{FCD}(A,B)$;
\begin{equation}
\mathcal{X}\suprel f\mathcal{Y}\Leftrightarrow\exists
x\in\atoms\mathcal{X},y\in\atoms\mathcal{Y}:x\mathrel\delta y\label{at-rel-eq}
\end{equation}
for every $\mathcal{X}\in\mathscr{F}(A)$, $\mathcal{Y}\in\mathscr{F}(B)$.

\end{enumerate}
\end{thm}
\begin{proof}
Existence of no more than one such funcoids and formulas (\ref{at-func-eq})
and (\ref{at-rel-eq}) follow from the previous theorem.
\begin{widedisorder}
\item [{\ref{at-restr-f}}] Consider the function
$\alpha'\in\mathscr{F}(B)^{\mathscr{T}A}$
defined by the formula (for every $X\in\mathscr{T}A$)
\[
\alpha'X=\bigsqcup\rsupfun{\alpha}\atoms\uparrow X.
\]



Obviously $\alpha'\bot^{\mathscr{T}A}=\bot^{\mathscr{F}(B)}$. For
every $I,J\in\mathscr{T}A$
\begin{align*}
\alpha'(I\sqcup J) & =\\
\bigsqcup\rsupfun{\alpha}\atoms\uparrow(I\sqcup J) & =\\
\bigsqcup\rsupfun{\alpha}(\atoms\uparrow\cup\atoms\uparrow J) & =\\
\bigsqcup(\rsupfun{\alpha}\atoms\uparrow I\cup\rsupfun{\alpha}\atoms\uparrow J)
& =\\
\bigsqcup\rsupfun{\alpha}\atoms\uparrow
I\sqcup\bigsqcup\rsupfun{\alpha}\atoms\uparrow J & =\\
\alpha'I\sqcup\alpha'J.
\end{align*}



Let continue $\alpha'$ till a funcoid $f$ (by the theorem \ref{fcd-as-cont}):
$\supfun f\mathcal{X}=\bigsqcap\rsupfun{\alpha'}\up\mathcal{X}$.


Let's prove the reverse of (\ref{at-func-cond}):
\begin{align*}
\bigsqcap\rsupfun{\bigsqcup\circ\rsupfun{\alpha}\circ\atoms\circ\uparrow}\up a &
=\\
\bigsqcap\rsupfun{\bigsqcup\circ\rsupfun{\alpha}}\rsupfun{\atoms}\rsupfun{
\uparrow}\up a & \sqsubseteq\\
\bigsqcap\rsupfun{\bigsqcup\circ\rsupfun{\alpha}}\{\{a\}\} & =\\
\bigsqcap\left\{ \left(\bigsqcup\circ\rsupfun{\alpha}\right)\{a\}\right\}  & =\\
\bigsqcap\left\{ \bigsqcup\rsupfun{\alpha}\{a\}\right\}  & =\\
\bigsqcap\left\{ \bigsqcup\{\alpha a\}\right\}  & =\\
\bigsqcap\{\alpha a\}  & =\\
\alpha a.
\end{align*}



Finally,
\[
\alpha
a=\bigsqcap\rsupfun{\bigsqcup\circ\rsupfun{\alpha}\circ\atoms\circ\uparrow}\up
a=\bigsqcap\rsupfun{\alpha'}\up a=\supfun fa,
\]



so $\supfun f$ is a continuation of $\alpha$.

\item [{\ref{at-restr-r}}] Consider the relation
$\delta'\in\subsets(\mathscr{T}A\times\mathscr{T}B)$
defined by the formula (for every $X\in\mathscr{T}A$, $Y\in\mathscr{T}B$)
\[
X\mathrel{\delta'}Y\Leftrightarrow\exists x\in\atoms\uparrow
X,y\in\atoms\uparrow Y:x\mathrel\delta y.
\]



Obviously $\lnot(X\mathrel{\delta'}\bot^{\mathscr{F}(B)})$ and
$\lnot(\bot^{\mathscr{F}(A)}\mathrel{\delta'}Y)$.


For suitable $I$ and $J$ we have:
\begin{align*}
I\sqcup J\mathrel{\delta'}Y & \Leftrightarrow\\
\exists x\in\atoms\uparrow(I\sqcup J),y\in\atoms\uparrow Y:x\mathrel\delta y &
\Leftrightarrow\\
\exists x\in\atoms\uparrow I\cup\atoms\uparrow J,y\in\atoms\uparrow
Y:x\mathrel\delta y & \Leftrightarrow\\
\exists x\in\atoms\uparrow I,y\in\atoms\uparrow Y:x\mathrel\delta y\lor\exists
x\in\atoms\uparrow J,y\in\atoms\uparrow Y:x\mathrel\delta y & \Leftrightarrow\\
I\mathrel{\delta'}Y\lor J\mathrel{\delta'}Y;
\end{align*}
similarly $X\mathrel{\delta'}I\sqcup J\Leftrightarrow X\mathrel{\delta'}I\lor
X\mathrel{\delta'}J$
for suitable $I$ and $J$. Let's continue $\delta'$ till a funcoid
$f$ (by the theorem \ref{fcd-as-cont}):
\[
\mathcal{X}\suprel f\mathcal{Y}\Leftrightarrow\forall
X\in\up\mathcal{X},Y\in\up\mathcal{Y}:X\mathrel{\delta'}Y.
\]



The reverse of (\ref{at-rel-cond}) implication is trivial, so
\[
\forall X\in\up a,Y\in\up b\exists x\in\atoms\uparrow X,y\in\atoms\uparrow
Y:x\mathrel\delta y\Leftrightarrow a\mathrel\delta b.
\]



Also
\begin{align*}
\forall X\in\up a,Y\in\up b\exists x\in\atoms\uparrow X,y\in\atoms\uparrow
Y:x\mathrel\delta y & \Leftrightarrow\\
\forall X\in\up a,Y\in\up b:X\mathrel{\delta'}Y & \Leftrightarrow\\
a\suprel fb.
\end{align*}



So $a\mathrel\delta b\Leftrightarrow a\suprel fb$, that is $\suprel f$
is a continuation of $\delta$.

\end{widedisorder}
\end{proof}
One of uses of the previous theorem is the proof of the following
theorem:
\begin{thm}
\label{fcd-intrs-atom}If $A$ and $B$ are sets, $R\in\subsets\mathsf{FCD}(A,B)$,
$x\in\atoms^{\mathscr{F}(A)}$, $y\in\atoms^{\mathscr{F}(B)}$, then
\begin{enumerate}
\item \label{meet-f-at}$\supfun{\bigsqcap R}x=\bigsqcap_{f\in R}\supfun fx$;
\item \label{meet-r-at}$x\suprel{\bigsqcap R}y\Leftrightarrow\forall f\in
R:x\suprel fy$.
\end{enumerate}
\end{thm}
\begin{proof}
~
\begin{widedisorder}
\item [{\ref{meet-r-at}}] Let denote $x\mathrel\delta y\Leftrightarrow\forall
f\in R:x\suprel fy$.
For every $a\in\atoms^{\mathscr{F}(A)}$, $b\in\atoms^{\mathscr{F}(B)}$
\begin{align*}
\forall X\in\up a,Y\in\up b\exists x\in\atoms\uparrow X,y\in\atoms\uparrow
Y:x\mathrel\delta y & \Rightarrow\\
\forall f\in R,X\in\up a,Y\in\up b\exists x\in\atoms\uparrow X,y\in\atoms\uparrow
Y:x\suprel fy & \Rightarrow\\
\forall f\in R,X\in\up a,Y\in\up b:X\rsuprel fY & \Rightarrow\\
\forall f\in R:a\suprel fb & \Leftrightarrow\\
a\mathrel\delta b.
\end{align*}



So by theorem~\ref{cont-fcd-on-atoms}, $\delta$ can be continued
till $\suprel p$ for some funcoid $p\in\mathsf{FCD}(A,B)$.


For every funcoid $q\in\mathsf{FCD}(A,B)$ such that $\forall f\in R:q\sqsubseteq
f$
we have
\[
x\suprel qy\Rightarrow\forall f\in R:x\suprel fy\Leftrightarrow x\mathrel\delta
y\Leftrightarrow x\suprel py,
\]



so $q\sqsubseteq p$. Consequently $p=\bigsqcap R$.


From this $x\suprel{\bigsqcap R}y\Leftrightarrow\forall f\in R:x\suprel fy$.

\item [{\ref{meet-f-at}}] From the former
\begin{align*}
y\in\atoms\supfun{\bigsqcap R}x & \Leftrightarrow\\
y\sqcap\supfun{\bigsqcap R}x\ne\bot & \Leftrightarrow\\
\forall f\in R:y\sqcap\supfun fx\ne\bot & \Leftrightarrow\\
y\in\bigsqcap\rsupfun{\atoms}\setcond{\supfun fx}{f\in R} & \Leftrightarrow\\
y\in\atoms\bigsqcap_{f\in R}\supfun fx
\end{align*}
for every $y\in\atoms^{\mathscr{F}(A)}$. From this it follows $\supfun{\bigsqcap
R}x=\bigsqcap_{f\in R}\supfun fx$.
\end{widedisorder}
\end{proof}
\begin{thm}
$g\circ f=\bigsqcap^{\mathsf{FCD}}\setcond{G\circ F}{F\in\up f.G\in\up g}$ for every
composable funcoids~$f$ and~$g$.\end{thm}
\begin{proof}
Let $x\in\atoms^{\mathscr{F}(\Src f)}$. Then
\begin{align*}
\supfun{g\circ f}x & =\\
\supfun g\supfun fx & =\text{ (theorem \ref{fcd-up-x})}\\
\bigsqcap_{G\in\up g}^{\mathscr{F}}\supfun G\supfun fx & =\text{ (theorem
\ref{fcd-up-x})}\\
\bigsqcap_{G\in\up g}^{\mathscr{F}}\supfun G\bigsqcap_{F\in\up
f}^{\mathscr{F}}\supfun Fx & =\text{ (theorem \ref{supfun-genbase})}\\
\bigsqcap_{G\in\up g}^{\mathscr{F}}\bigsqcap_{F\in\up f}^{\mathscr{F}}\supfun
G\supfun Fx & =\\
\bigsqcap^{\mathscr{F}}\setcond{\supfun G\supfun Fx}{F\in\up f,G\in\up g} & =\\
\bigsqcap^{\mathscr{F}}\setcond{\supfun{G\circ F}x}{F\in\up f,G\in\up g} & =\text{ (theorem
\ref{fcd-intrs-atom})}\\
\supfun{\bigsqcap^{\mathsf{FCD}}\setcond{G\circ F}{F\in\up f,G\in\up g}}x.
\end{align*}


Thus $g\circ f=\bigsqcap^{\mathsf{FCD}}\setcond{G\circ F}{F\in\up f.G\in\up g}$.\end{proof}

\begin{prop}
  For $f \in \mathsf{FCD} (A, B)$, a finite set $X \in \subsets A$
  and a function $t \in \mathscr{F} (B)^X$ there exists (obviously unique) $g
  \in \mathsf{FCD} (A, B)$ such that $\supfun{g} p = \langle f
  \rangle p$ for $p \in \atoms^{\mathscr{F} (A)} \setminus \atoms
  X$ and $\supfun{g} @\{ x \} = t (x)$ for $x \in X$.

  This funcoid $g$ is determined by the formula
  \[ g = (f \setminus (@X \times^{\mathsf{FCD}} \top)) \sqcup
     \bigsqcup_{x \in X} (@\{ x \} \times^{\mathsf{FCD}} t (x)) . \]
\end{prop}

\begin{proof}
  Take $g = (f \setminus (@X \times^{\mathsf{FCD}} \top)) \sqcup
  \bigsqcup_{q \in X} (@\{ q \} \times^{\mathsf{FCD}} t (x))$ that is $g
  = \left( f \sqcap \overline{X \times \top} \right) \sqcup \bigsqcup_{q \in
  X} (@\{ q \} \times^{\mathsf{FCD}} t (x)) = \left( f \sqcap \left(
  \overline{X} \times \top \right) \right) \sqcup \bigsqcup_{q \in X} (@\{ q \}
  \times^{\mathsf{FCD}} t (x))$.

  $\supfun{g} p = \text{(theorem~\ref{fcd-fin-join})} = \left\langle f
  \sqcap \left( \overline{X} \times \top \right) \right\rangle p \sqcup
  \bigsqcup_{q \in X} \langle @\{ q \} \times^{\mathsf{FCD}} t (x)
  \rangle p = \text{(theorem~\ref{fcd-intrs-atom})} = \left( \supfun{f} p \sqcap
  \left\langle \overline{X} \times \top \right\rangle p \right) \sqcup
  \bigsqcup_{q \in X} \langle @\{ q \} \times^{\mathsf{FCD}} t (x)
  \rangle p$.

  So $\supfun{g} @\{ x \} = (\supfun{f}^{\ast} @\{ x \} \sqcap
  \bot) \sqcup t (x) = t (x)$ for $x \in X$.

  If $p \in \atoms^{\mathscr{F} (A)} \setminus \atoms X$ then we
  have $\supfun{g} p = (\supfun{f} p \sqcap \top) \sqcup \bot =
  \supfun{f} p$.
\end{proof}

\begin{cor}
  If $f \in \mathsf{FCD} (A, B)$, $x \in A$, and $\mathcal{Y} \in
  \mathscr{F} (B)$, then there exists an (obviously unique) $g \in
  \mathsf{FCD} (A, B)$ such that $\supfun{g} p = \langle f
  \rangle p$ for all ultrafilters $p$ except of $p = @\{ x \}$ and $\langle g
  \rangle @\{ x \} = \mathcal{Y}$.

  This funcoid $g$ is determined by the formula
  \[ g = (f \setminus (@\{ x \} \times^{\mathsf{FCD}} \top)) \sqcup (\{
     x \} \times^{\mathsf{FCD}} \mathcal{Y}) . \]
\end{cor}

\begin{thm}
\label{fcd-cross}Let $A$, $B$, $C$ be sets, $f\in\mathsf{FCD}(A,B)$,
$g\in\mathsf{FCD}(B,C)$, $h\in\mathsf{FCD}(A,C)$. Then
\[
g\circ f\nasymp h\Leftrightarrow g\nasymp h\circ f^{-1}.
\]
\end{thm}
\begin{proof}
~
\begin{align*}
g\circ f\nasymp h & \Leftrightarrow\\
\exists a\in\atoms^{\mathscr{F}(A)},c\in\atoms^{\mathscr{F}(C)}:a\suprel{(g\circ
f)\sqcap h}c & \Leftrightarrow\\
\exists a\in\atoms^{\mathscr{F}(A)},c\in\atoms^{\mathscr{F}(C)}:(a\suprel{g\circ
f}c\land a\suprel hc) & \Leftrightarrow\\
\exists
a\in\atoms^{\mathscr{F}(A)},b\in\atoms^{\mathscr{F}(B)},c\in\atoms^{\mathscr{F}
(C)}:(a\suprel fb\land b\suprel gc\land a\suprel hc) & \Leftrightarrow\\
\exists b\in\atoms^{\mathscr{F}(B)},c\in\atoms^{\mathscr{F}(C)}:(b\suprel
gc\land b\suprel{h\circ f^{-1}}c) & \Leftrightarrow\\
\exists
b\in\atoms^{\mathscr{F}(B)},c\in\atoms^{\mathscr{F}(C)}:b\suprel{g\sqcap(h\circ
f^{-1})}c & \Leftrightarrow\\
g\nasymp h\circ f^{-1}.
\end{align*}

\end{proof}

\begin{thm}
Every filtrator of funcoids is star-separable.\end{thm}
\begin{proof}
The set of funcoidal products of principal filters is a separation
subset of the lattice of funcoids by corollary~\ref{pintrs-fcd}.
\end{proof}

\begin{thm}
\label{meet-prod-fcd}Let $A$, $B$ be sets. If
$S\in\subsets(\mathscr{F}(A)\times\mathscr{F}(B))$
then
\[
\bigsqcap_{(\mathcal{A},\mathcal{B})\in
S}(\mathcal{A}\times^{\mathsf{FCD}}\mathcal{B})=\bigsqcap\dom
S\times^{\mathsf{FCD}}\bigsqcap\im S.
\]
\end{thm}
\begin{proof}
If $x\in\atoms^{\mathscr{F}(A)}$ then by theorem \ref{fcd-intrs-atom}
\[
\supfun{\bigsqcap_{(\mathcal{A},\mathcal{B})\in
S}(\mathcal{A}\times^{\mathsf{FCD}}\mathcal{B})}x=\bigsqcap_{(\mathcal{A}
,\mathcal{B})\in S}\supfun{\mathcal{A}\times^{\mathsf{FCD}}\mathcal{B}}x.
\]


If $x\nasymp\bigsqcap\dom S$ then
\begin{gather*}
\forall(\mathcal{A},\mathcal{B})\in
S:(x\sqcap\mathcal{A}\ne\bot\land\supfun{\mathcal{A}\times^{
\mathsf{FCD}}\mathcal{B}}x=\mathcal{B});\\
\setcond{\supfun{\mathcal{A}\times^{\mathsf{FCD}}\mathcal{B}}x}{(\mathcal{A}
,\mathcal{B})\in S}=\im S;
\end{gather*}


if $x\asymp\bigsqcap\dom S$ then
\begin{gather*}
\exists(\mathcal{A},\mathcal{B})\in
S:(x\sqcap\mathcal{A}=\bot\land\supfun{\mathcal{A}\times^{
\mathsf{FCD}}\mathcal{B}}x=\bot);\\
\setcond{\supfun{\mathcal{A}\times^{\mathsf{FCD}}\mathcal{B}}x}{(\mathcal{A}
,\mathcal{B})\in S}\ni\bot.
\end{gather*}


So
\[
\supfun{\bigsqcap_{(\mathcal{A},\mathcal{B})\in
S}(\mathcal{A}\times^{\mathsf{FCD}}\mathcal{B})}x=\begin{cases}
\bigsqcap\im S & \text{if }x\nasymp\bigsqcap\dom S\\
\bot^{\mathscr{F}(B)} & \text{if }x\asymp\bigsqcap\dom S.
\end{cases}
\]


From this the statement of the theorem follows.\end{proof}
\begin{cor}
For every $\mathcal{A}_{0},\mathcal{A}_{1}\in\mathscr{F}(A)$,
$\mathcal{B}_{0},\mathcal{B}_{1}\in\mathscr{F}(B)$
(for every sets $A$,~$B$)
\[
(\mathcal{A}_{0}\times^{\mathsf{FCD}}\mathcal{B}_{0})\sqcap(\mathcal{A}_{1}
\times^{\mathsf{FCD}}\mathcal{B}_{1})=(\mathcal{A}_{0}\sqcap\mathcal{A}_{1}
)\times^{\mathsf{FCD}}(\mathcal{B}_{0}\sqcap\mathcal{B}_{1}).
\]
\end{cor}
\begin{proof}
$(\mathcal{A}_{0}\times^{\mathsf{FCD}}\mathcal{B}_{0})\sqcap(\mathcal{A}_{1}
\times^{\mathsf{FCD}}\mathcal{B}_{1})=\bigsqcap\{\mathcal{A}\times^{\mathsf{FCD
}}\mathcal{B}_{0},\mathcal{A}_{1}\times^{\mathsf{FCD}}\mathcal{B}_{1}\}$
what is by the last theorem equal to
$(\mathcal{A}_{0}\sqcap\mathcal{A}_{1})\times^{\mathsf{FCD}}(\mathcal{B}_{0}
\sqcap\mathcal{B}_{1})$.\end{proof}
\begin{thm}
If $A$, $B$ are sets and $\mathcal{A}\in\mathscr{F}(A)$ then
$\mathcal{A}\times^{\mathsf{FCD}}$
is a complete homomorphism from the lattice $\mathscr{F}(B)$ to the
lattice $\mathsf{FCD}(A,B)$, if also $\mathcal{A}\ne\bot^{\mathscr{F}(A)}$
then it is an order embedding.\end{thm}
\begin{proof}
Let $S\in\subsets\mathscr{F}(B)$, $X\in\mathscr{T}A$,
$x\in\atoms^{\mathscr{F}(A)}$.
\begin{align*}
\rsupfun{\bigsqcup\rsupfun{\mathcal{A}\times^{\mathsf{FCD}}}S}X & =\\
\bigsqcup_{\mathcal{B}\in
S}\rsupfun{\mathcal{A}\times^{\mathsf{FCD}}\mathcal{B}}X & =\\
\begin{cases}
\bigsqcup S & \text{if }X\in\corestar\mathcal{A}\\
\bot^{\mathscr{F}(B)} & \text{if }X\notin\corestar\mathcal{A}
\end{cases} & =\\
\rsupfun{\mathcal{A}\times^{\mathsf{FCD}}\bigsqcup S}X;\\
\supfun{\bigsqcap\rsupfun{\mathcal{A}\times^{\mathsf{FCD}}}S}x & =\\
\bigsqcap_{\mathcal{B}\in
S}\supfun{\mathcal{A}\times^{\mathsf{FCD}}\mathcal{B}}x & =\\
\begin{cases}
\bigsqcap S & \text{if }x\nasymp\mathcal{A}\\
\bot^{\mathscr{F}(B)} & \text{if }x\asymp\mathcal{A}.
\end{cases}
\end{align*}


Thus
$\bigsqcup\rsupfun{\mathcal{A}\times^{\mathsf{FCD}}}S=\mathcal{A}\times^{\mathsf
{FCD}}\bigsqcup S$
and
$\bigsqcap\rsupfun{\mathcal{A}\times^{\mathsf{FCD}}}S=\mathcal{A}\times^{\mathsf
{FCD}}\bigsqcap S$.

If $\mathcal{A}\ne\bot$ then obviously
$\mathcal{A}\times^{\mathsf{FCD}}\mathcal{X}\sqsubseteq\mathcal{A}\times^{\mathsf{FCD}}\mathcal{Y} \Leftrightarrow
\mathcal{X}\sqsubseteq\mathcal{Y}$.
\end{proof}
The following proposition states that cutting a rectangle of atomic
width from a funcoid always produces a rectangular (representable
as a funcoidal product of filters) funcoid (of atomic width).
\begin{prop}
If $f$ is a funcoid and $a$ is an atomic filter on $\Src f$ then
\[
f|_{a}=a\times^{\mathsf{FCD}}\supfun fa.
\]
\end{prop}
\begin{proof}
Let $\mathcal{X}\in\mathscr{F}(\Src f)$.
\[
\mathcal{X}\nasymp a\Rightarrow\supfun{f|_{a}}\mathcal{X}=\supfun
fa,\quad\mathcal{X}\asymp
a\Rightarrow\supfun{f|_{a}}\mathcal{X}=\bot^{\mathscr{F}(\Dst f)}.
\]

\end{proof}

\begin{lem}
$\mylambda{\mathcal{B}}{\mathscr{F}(B)}{\top^{\mathscr{F}}\times^{\mathsf{FCD}}
\mathcal{B}}$
is an upper adjoint of $\mylambda f{\mathsf{FCD}(A,B)}{\im f}$ (for
every sets $A$, $B$).\end{lem}
\begin{proof}
We need to prove $\im f\sqsubseteq\mathcal{B}\Leftrightarrow
f\sqsubseteq\top\times^{\mathsf{FCD}}\mathcal{B}$
what is obvious.\end{proof}
\begin{cor}
\label{fcd-dom-join}Image and domain of funcoids preserve joins.\end{cor}
\begin{proof}
By properties of Galois connections and duality.\end{proof}
\begin{prop}
$f\sqsubseteq\mathcal{A}\times^{\mathsf{FCD}}\mathcal{B}\Leftrightarrow\dom
f\sqsubseteq\mathcal{A}\wedge\im f\sqsubseteq\mathcal{B}$
for every funcoid $f$ and filters $\mathcal{A}\in\mathfrak{F}(\Src f)$,
$\mathcal{B}\in\mathfrak{F}(\Dst f)$.\end{prop}
\begin{proof}
$f\sqsubseteq\mathcal{A}\times^{\mathsf{FCD}}\mathcal{B}\Rightarrow\dom
f\sqsubseteq\mathcal{A}$
because
$\dom(\mathcal{A}\times^{\mathsf{FCD}}\mathcal{B})\sqsubseteq\mathcal{A}$.

Let now $\dom f\sqsubseteq\mathcal{A}\wedge\im f\sqsubseteq\mathcal{B}$.
Then $\supfun f\mathcal{X}\neq\bot\Rightarrow\mathcal{X}\nasymp\mathcal{A}$
that is $f\sqsubseteq\mathcal{A}\times^{\mathsf{FCD}}\top$. Similarly
$f\sqsubseteq\top\times^{\mathsf{FCD}}\mathcal{B}$. Thus
$f\sqsubseteq\mathcal{A}\times^{\mathsf{FCD}}\mathcal{B}$.
\end{proof}

\section{Atomic funcoids}
\begin{thm}
An $f\in\mathsf{FCD}(A,B)$ is an atom of the lattice $\mathsf{FCD}(A,B)$
(for some sets $A$, $B$) iff it is a funcoidal product of two atomic
filter objects.\end{thm}
\begin{proof}
~
\begin{description}
\item [{$\Rightarrow$}] Let $f\in\mathsf{FCD}(A,B)$ be an atom of the
lattice $\mathsf{FCD}(A,B)$. Let's get elements $a\in\atoms\dom f$
and $b\in\atoms\supfun fa$. Then for every $\mathcal{X}\in\mathscr{F}(A)$
\[
\mathcal{X}\asymp
a\Rightarrow\supfun{a\times^{\mathsf{FCD}}b}\mathcal{X}=\bot
\sqsubseteq\supfun f\mathcal{X},\quad\mathcal{X}\nasymp
a\Rightarrow\supfun{a\times^{\mathsf{FCD}}b}\mathcal{X}=b\sqsubseteq\supfun
f\mathcal{X}.
\]



So $a\times^{\mathsf{FCD}}b\sqsubseteq f$; because $f$ is atomic we
have $f=a\times^{\mathsf{FCD}}b$.

\item [{$\Leftarrow$}] Let $a\in\atoms^{\mathscr{F}(A)}$,
$b\in\atoms^{\mathscr{F}(B)}$,
$f\in\mathsf{FCD}(A,B)$. If $b\asymp\supfun fa$ then $\lnot(a\suprel fb)$,
$f\asymp a\times^{\mathsf{FCD}}b$; if $b\sqsubseteq\supfun fa$ then
$\forall\mathcal{X}\in\mathscr{F}(A):(\mathcal{X}\nasymp a\Rightarrow\supfun
f\mathcal{X}\sqsupseteq b)$,
$f\sqsupseteq a\times^{\mathsf{FCD}}b$. Consequently $f\asymp
a\times^{\mathsf{FCD}}b\lor f\sqsupseteq a\times^{\mathsf{FCD}}b$;
that is $a\times^{\mathsf{FCD}}b$ is an atom.
\end{description}
\end{proof}
\begin{thm}
The lattice $\mathsf{FCD}(A,B)$ is atomic (for every fixed sets $A$, $B$).\end{thm}
\begin{proof}
Let $f$ be a non-empty funcoid from $A$ to $B$. Then $\dom
f\ne\bot$,
thus by theorem~\ref{filt-atomic} there exists $a\in\atoms\dom f$.
So $\supfun fa\ne\bot$ thus it exists $b\in\atoms\supfun fa$.
Finally the atomic funcoid $a\times^{\mathsf{FCD}}b\sqsubseteq f$.\end{proof}
\begin{thm}
The lattice $\mathsf{FCD}(A,B)$ is separable (for every fixed sets $A$,
$B$).\end{thm}
\begin{proof}
Let $f,g\in\mathsf{FCD}(A,B)$, $f\sqsubset g$. Then there exists
$a\in\atoms^{\mathscr{F}(A)}$ such that $\supfun fa\sqsubset\supfun ga$.
So because the lattice $\mathscr{F}(B)$ is atomically separable,
there exists $b\in\atoms$ such that $\supfun fa\sqcap
b=\bot$
and $b\sqsubseteq\supfun ga$. For every $x\in\atoms^{\mathscr{F}(A)}$
\begin{gather*}
\supfun fa\sqcap\supfun{a\times^{\mathsf{FCD}}b}a=\supfun fa\sqcap
b=\bot,\\
x\ne a\Rightarrow\supfun fx\sqcap\supfun{a\times^{\mathsf{FCD}}b}x=\supfun
fx\sqcap\bot=\bot.
\end{gather*}


Thus $\supfun fx\sqcap\supfun{a\times^{\mathsf{FCD}}b}x=\bot$
and consequently $f\asymp a\times^{\mathsf{FCD}}b$.
\begin{gather*}
\supfun{a\times^{\mathsf{FCD}}b}a=b\sqsubseteq\supfun ga,\\
x\ne
a\Rightarrow\supfun{a\times^{\mathsf{FCD}}b}x=\bot
\sqsubseteq\supfun gx.
\end{gather*}


Thus $\supfun{a\times^{\mathsf{FCD}}b}x\sqsubseteq\supfun gx$ and
consequently $a\times^{\mathsf{FCD}}b\sqsubseteq g$.

So the lattice $\mathsf{FCD}(A,B)$ is separable by theorem
\ref{msl-sep-conds}.\end{proof}
\begin{cor}
\label{fcd-is-sep}The lattice $\mathsf{FCD}(A,B)$ is:
\begin{enumerate}
\item separable;
\item strongly separable;
\item atomically separable;
\item conforming to Wallman's disjunction property.
\end{enumerate}
\end{cor}
\begin{proof}
By theorem \ref{sep-conds}.\end{proof}
\begin{rem}
For more ways to characterize (atomic) separability of the lattice
of funcoids see subsections ``Separation subsets and full stars''
and ``Atomically separable lattices''.\end{rem}
\begin{cor}
The lattice $\mathsf{FCD}(A,B)$ is an atomistic lattice.\end{cor}
\begin{proof}
By theorem~\ref{amstc-sep}.\end{proof}
\begin{prop}
$\atoms(f\sqcup g)=\atoms f\cup\atoms g$ for every funcoids
$f,g\in\mathsf{FCD}(A,B)$
(for every sets $A$, $B$).\end{prop}
\begin{proof}
$a\times^{\mathsf{FCD}}b\nasymp f\sqcup g\Leftrightarrow a\suprel{f\sqcup
g}b\Leftrightarrow a\suprel fb\lor a\suprel gb\Leftrightarrow
a\times^{\mathsf{FCD}}b\nasymp f\lor a\times^{\mathsf{FCD}}b\nasymp g$
for every atomic filters $a$ and $b$.\end{proof}
\begin{thm}
The set of funcoids between sets~$A$ and~$B$ is a co-frame.\end{thm}
\begin{proof}
Theorems \ref{fcd-as-cont} and \ref{frame-main}.\end{proof}
\begin{rem}
The above proof does not use axiom of choice (unlike the below proof).
\end{rem}
See also an older proof of the set of funcoids being co-brouwerian:
\begin{thm}
For every $f,g,h\in\mathsf{FCD}(A,B)$, $R\in\subsets\mathsf{FCD}(A,B)$
(for every sets $A$ and $B$)
\begin{enumerate}
\item \label{fcd-dist-j}$f\sqcap(g\sqcup h)=(f\sqcap g)\sqcup(f\sqcap h)$;
\item \label{fcd-dist-m}$f\sqcup\bigsqcap R=\bigsqcap\rsupfun{f\sqcup}R$.
\end{enumerate}
\end{thm}
\begin{proof}
We will take into account that the lattice of funcoids is an atomistic
lattice.
\begin{widedisorder}
\item [{\ref{fcd-dist-j}}] ~
\begin{align*}
\atoms(f\sqcap(g\sqcup h)) & =\\
\atoms f\cap\atoms(g\sqcup h) & =\\
\atoms f\cap(\atoms g\cup\atoms h) & =\\
(\atoms f\cap\atoms g)\cup(\atoms f\cap\atoms h) & =\\
\atoms(f\sqcap g)\cup\atoms(f\sqcap h) & =\\
\atoms((f\sqcap g)\sqcup(f\sqcap h)).
\end{align*}

\item [{\ref{fcd-dist-m}}] ~
\begin{align*}
\atoms\left(f\sqcup\bigsqcap R\right) & =\\
\atoms f\cup\atoms\bigsqcap R & =\\
\atoms f\cup\bigcap\rsupfun{\atoms}R & =\\
\bigcap\rsupfun{(\atoms f)\cup}\rsupfun{\atoms}R & =\text{ (use the following
equality)}\\
\bigcap\rsupfun{\atoms}\rsupfun{f\sqcup}R & =\\
\atoms\bigsqcap\rsupfun{f\sqcup}R.\\
\rsupfun{(\atoms f)\cup}\rsupfun{\atoms}R & =\\
\setcond{(\atoms f)\cup A}{A\in\rsupfun{\atoms}R} & =\\
\setcond{(\atoms f)\cup A}{\exists C\in R:A=\atoms C} & =\\
\setcond{(\atoms f)\cup(\atoms C)}{C\in R} & =\\
\setcond{\atoms(f\sqcup C)}{C\in R} & =\\
\setcond{\atoms B}{\exists C\in R:B=f\sqcup C} & =\\
\setcond{\atoms B}{B\in\rsupfun{f\sqcup}C} & =\\
\rsupfun{\atoms}\rsupfun{f\sqcup}R.
\end{align*}

\end{widedisorder}
\end{proof}

\begin{conjecture}
$f \sqcap \bigsqcup S = \bigsqcup \langle f \sqcap \rangle^{\ast} S$ for principal funcoid~$f$ and a set~$S$ of
funcoids of appropriate sources and destinations.
\end{conjecture}

\begin{rem}
See also example~\ref{fcd-not-infdist} below.
\end{rem}

The next proposition is one more (among the theorem \ref{fcd-atom-middle})
generalization for funcoids of composition of relations.
\begin{prop}\label{fcd-at-comp}
For every composable funcoids $f$, $g$
\begin{multline*}
\atoms(g\circ f)=\\
\setcond{x\times^{\mathsf{FCD}}z}{\begin{array}{l}
x\in\atoms^{\mathscr{F}(\Src f)},z\in\atoms^{\mathscr{F}(\Dst g)},\\
\exists y\in\atoms^{\mathscr{F}(\Dst f)}:(x\times^{\mathsf{FCD}}y\in\atoms
f\land y\times^{\mathsf{FCD}}z\in\atoms g)
\end{array}}.
\end{multline*}
\end{prop}
\begin{proof}
Using the theorem \ref{fcd-atom-middle},
\[
x\times^{\mathsf{FCD}}z\nasymp g\circ f\Leftrightarrow x\suprel{g\circ
f}z\Leftrightarrow\exists y\in\atoms^{\mathscr{F}(\Dst
f)}:(x\times^{\mathsf{FCD}}y\nasymp f\land y\times^{\mathsf{FCD}}z\nasymp g).
\]
\end{proof}
\begin{cor}
$g\circ f=\bigsqcup\setcond{G\circ F}{F\in\atoms f,G\in\atoms g}$
for every composable funcoids $f$, $g$.\end{cor}
\begin{thm}
Let $f$ be a funcoid.
\begin{enumerate}
\item \label{ffilt-r}$\mathcal{X}\suprel f\mathcal{Y}\Leftrightarrow\exists
F\in\atoms f:\mathcal{X}\suprel F\mathcal{Y}$
for every $\mathcal{X}\in\mathscr{F}(\Src f)$, $\mathcal{Y}\in\mathscr{F}(\Dst
f)$;
\item \label{ffilt-f}$\supfun f\mathcal{X}=\bigsqcup_{F\in\atoms f}\supfun
F\mathcal{X}$
for every $\mathcal{X}\in\mathscr{F}(\Src f)$.
\end{enumerate}
\end{thm}
\begin{proof}
~
\begin{widedisorder}
\item [{\ref{ffilt-r}}] ~
\begin{align*}
\exists F\in\atoms f:\mathcal{X}\suprel F\mathcal{Y} & \Leftrightarrow\\
\exists a\in\atoms^{\mathscr{F}(\Src f)},b\in\atoms^{\mathscr{F}(\Dst
f)}:(a\times^{\mathsf{FCD}}b\nasymp
f\land\mathcal{X}\suprel{a\times^{\mathsf{FCD}}b}\mathcal{Y}) &
\Leftrightarrow\\
\exists a\in\atoms^{\mathscr{F}(\Src f)},b\in\atoms^{\mathscr{F}(\Dst
f)}:(a\times^{\mathsf{FCD}}b\nasymp f\land
a\times^{\mathsf{FCD}}b\nasymp\mathcal{X}\times^{\mathsf{FCD}}\mathcal{Y}) &
\Leftrightarrow\\
\exists F\in\atoms f:(F\nasymp f\land
F\nasymp\mathcal{X}\times^{\mathsf{FCD}}\mathcal{Y}) & \Leftrightarrow\\
f\nasymp\mathcal{X}\times^{\mathsf{FCD}}\mathcal{Y} & \Leftrightarrow\\
\mathcal{X}\suprel f\mathcal{Y}.
\end{align*}

\item [{\ref{ffilt-f}}] Let $\mathcal{Y}\in\mathscr{F}(\Dst f)$. Suppose
$\mathcal{Y}\nasymp\supfun f\mathcal{X}$. Then $\mathcal{X}\suprel
f\mathcal{Y}$;
$\exists F\in\atoms f:\mathcal{X}\suprel F\mathcal{Y}$; $\exists F\in\atoms
f:\mathcal{Y}\nasymp\supfun F\mathcal{X}$;
$\mathcal{Y}\nasymp\bigsqcup_{F\in\atoms f}\supfun F\mathcal{X}$.
So $\supfun f\mathcal{X}\sqsubseteq\bigsqcup_{F\in\atoms f}\supfun
F\mathcal{X}$.
The contrary $\supfun f\mathcal{X}\sqsupseteq\bigsqcup_{F\in\atoms f}\supfun
F\mathcal{X}$
is obvious.
\end{widedisorder}
\end{proof}

\section{Complete funcoids}
\begin{defn}
\index{funcoid!co-complete}I will call \emph{co-complete} such a
funcoid $f$ that $\rsupfun fX$ is a principal filter for every
$X\in\mathscr{T}(\Src f)$.\end{defn}
\begin{obvious}
Funcoid $f$ is co-complete iff $\supfun f\mathcal{X}\in\mathfrak{P}(\Dst f)$
for every $\mathcal{X}\in\mathfrak{P}(\Src f)$.\end{obvious}
\begin{defn}
\index{generalized closure}I will call \emph{generalized closure}
such a function $\alpha\in(\mathscr{T}B)^{\mathscr{T}A}$ (for some
sets $A$, $B$) that
\begin{enumerate}
\item $\alpha\bot=\bot$;
\item $\forall I,J\in\mathscr{T}A:\alpha(I\sqcup J)=\alpha I\sqcup\alpha J$.
\end{enumerate}
\end{defn}
\begin{obvious}
A funcoid $f$ is co-complete iff $\rsupfun f=\mathord{\uparrow}\circ\alpha$
for a generalized closure $\alpha$.\end{obvious}
\begin{rem}
Thus funcoids can be considered as a generalization of generalized
closures. A topological space in Kuratowski sense is the same as reflexive
and transitive generalized closure. So topological spaces can be considered
as a special case of funcoids.\end{rem}
\begin{defn}
\index{funcoid!complete}I will call a \emph{complete funcoid} a funcoid
whose reverse is co-complete.\end{defn}
\begin{thm}
The following conditions are equivalent for every funcoid $f$:
\begin{enumerate}
\item \label{cfcd:main}funcoid $f$ is complete;
\item \label{cfcd:r-filt}$\forall S\in\subsets\mathscr{F}(\Src
f),J\in\mathscr{T}(\Dst f):\left(\bigsqcup S\suprel
fJ\Leftrightarrow\exists\mathcal{I}\in S:\mathcal{I}\suprel fJ\right)$;
\item \label{cfcd:r-set}$\forall S\in\subsets\mathscr{T}(\Src f),J\in\mathscr{T}(\Dst
f):\left(\bigsqcup S\rsuprel fJ\Leftrightarrow\exists I\in S:I\rsuprel
fJ\right)$;
\item \label{cfcd:f-filt}$\forall S\in\subsets\mathscr{F}(\Src f):\supfun
f\bigsqcup S=\bigsqcup\rsupfun{\supfun f}S$;
\item \label{cfcd:f-set}$\forall S\in\subsets\mathscr{T}(\Src f):\rsupfun
f\bigsqcup S=\bigsqcup\rsupfun{\rsupfun f}S$;
\item \label{cfcd:sing}$\forall A\in\mathscr{T}(\Src f):\rsupfun
fA=\bigsqcup_{a\in\atoms A}\rsupfun fa$.
\end{enumerate}
\end{thm}
\begin{proof}
~
\begin{description}
\item [{\ref{cfcd:r-set}$\Rightarrow$\ref{cfcd:main}}] For every
$S\in\subsets\mathscr{T}(\Src f)$,
$J\in\mathscr{T}(\Dst f)$
\[
\bigsqcup S\sqcap\rsupfun{f^{-1}}J\ne\bot\Leftrightarrow\exists I\in
S:I\sqcap\rsupfun{f^{-1}}J\ne\bot,
\]
consequently by theorem~\ref{crit1} we have that $\rsupfun{f^{-1}}J$
is a principal filter.
\item [{\ref{cfcd:main}$\Rightarrow$\ref{cfcd:r-filt}}] For every
$S\in\subsets\mathscr{F}(\Src f)$,
$J\in\mathscr{T}(\Dst f)$ we have that $\rsupfun{f^{-1}}J$ is a
principal filter, consequently
\[
\bigsqcup S\sqcap\rsupfun{f^{-1}}J\ne\bot \Leftrightarrow\exists\mathcal{I}\in
S:\mathcal{I}\sqcap\rsupfun{f^{-1}}J\ne\bot.
\]
From this follows \ref{cfcd:r-filt}.
\item [{\ref{cfcd:sing}$\Rightarrow$\ref{cfcd:f-set}}] ~
\begin{align*}
\rsupfun f\bigsqcup S & =\\
\bigsqcup_{a\in\atoms\bigsqcup S}\rsupfun fa & =\\
\bigsqcup\bigcup_{A\in S}\setcond{\rsupfun fa}{a\in\atoms A} & =\\
\bigsqcup_{A\in S}\bigsqcup_{a\in\atoms A}\rsupfun fa & =\\
\bigsqcup_{A\in S}\rsupfun fA & =\\
\bigsqcup\rsupfun{\rsupfun f}S.
\end{align*}

\item [{\ref{cfcd:r-filt}$\Rightarrow$\ref{cfcd:f-filt}}] Using
theorem~\ref{crit1},
\begin{align*}
J\nasymp\supfun f\bigsqcup S & \Leftrightarrow\\
\bigsqcup S\suprel fJ & \Leftrightarrow\\
\exists\mathcal{I}\in S:\mathcal{I}\suprel fJ & \Leftrightarrow\\
\exists\mathcal{I}\in S:J\nasymp\supfun f\mathcal{I} & \Leftrightarrow\\
J\nasymp\bigsqcup\rsupfun{\supfun f}S.
\end{align*}

\item
[{\ref{cfcd:r-filt}$\Rightarrow$\ref{cfcd:r-set},~\ref{cfcd:f-filt}
$\Rightarrow$\ref{cfcd:f-set},~\ref{cfcd:f-set}$\Rightarrow$\ref{cfcd:r-set},
~\ref{cfcd:f-set}$\Rightarrow$\ref{cfcd:sing}}] Obvious.
\end{description}
\end{proof}
The following proposition shows that complete funcoids are a direct
generalization of pretopological spaces.
\begin{prop}
To specify a complete funcoid $f$ it is enough to specify $\rsupfun f$
on one-element sets, values of $\rsupfun f$ on one element sets can
be specified arbitrarily.\end{prop}
\begin{proof}
From the above theorem is clear that knowing $\rsupfun f$ on one-element
sets $\rsupfun f$ can be found on every set and then the value of
$\supfun f$ can be inferred for every filter.

Choosing arbitrarily the values of $\rsupfun f$ on one-element sets
we can define a complete funcoid the following way: $\rsupfun
fX=\bigsqcup_{\alpha\in\atoms X}\rsupfun f\alpha$
for every $X\in\mathscr{T}(\Src f)$. Obviously it is really a complete
funcoid.\end{proof}
\begin{thm}
A funcoid is principal iff it is both complete and co-complete.\end{thm}
\begin{proof}
~
\begin{description}
\item [{$\Rightarrow$}] Obvious.
\item [{$\Leftarrow$}] Let $f$ be both a complete and co-complete funcoid.
Consider the relation $g$ defined by that $\uparrow\rsupfun g\alpha=\rsupfun
f\alpha$
for one-element sets~$\alpha$ ($g$ is correctly defined because
$f$ corresponds to a generalized closure). Because $f$ is a complete
funcoid $f$ is the funcoid corresponding to $g$.
\end{description}
\end{proof}
\begin{thm}
\label{fcd-join-compl}If $R\in\subsets\mathsf{FCD}(A,B)$ is a set
of (co-)complete funcoids then $\bigsqcup R$ is a (co-)complete funcoid
(for every sets $A$ and $B$).\end{thm}
\begin{proof}
It is enough to prove for co-complete funcoids. Let
$R\in\subsets\mathsf{FCD}(A,B)$
be a set of co-complete funcoids. Then for every $X\in\mathscr{T}(\Src f)$
\[
\rsupfun{\bigsqcup R}X=\bigsqcup_{f\in R}\rsupfun fX
\]
is a principal filter (used theorem \ref{fcd-join-sets}).\end{proof}
\begin{cor}
\label{fcd-compl-join}If $R$ is a set of binary relations between
sets $A$ and $B$ then
$\bigsqcup\rsupfun{\uparrow^{\mathsf{FCD}(A,B)}}R=\uparrow^{\mathsf{FCD}(A,B)}
\bigcup R$.\end{cor}
\begin{proof}
From two last theorems.\end{proof}
\begin{lem}
\label{fcd-rep}Every funcoid is representable as meet (on the lattice
of funcoids) of binary relations of the form
$X\times Y\sqcup\overline{X}\times\top^{\mathscr{T}(B)}$ (where $X$, $Y$ are typed sets).\end{lem}
\begin{proof}
Let $f\in\mathsf{FCD}(A,B)$, $X\in\mathscr{T}A$, $Y\in\up\supfun fX$,
$g(X,Y)\eqdef X\times Y\sqcup\overline{X}\times\top^{\mathscr{T}(B)}$.
Then $g(X,Y)=X\times^{\mathsf{FCD}}Y\sqcup\overline{X}\times^{\mathsf{FCD}}\top^{\mathscr{F}(B)}$.
For every $K\in\mathscr{T}A$
\begin{multline*}
\rsupfun{g(X,Y)}K=\rsupfun{X\times^{\mathsf{FCD}}Y}K\sqcup\rsupfun{\overline{X}
\times^{\mathsf{FCD}}\top^{\mathscr{F}(B)}}K=\\
\left(\begin{cases}
\bot^{\mathscr{F}(B)} & \text{if }K=\bot^{\mathscr{T}A}\\
Y & \text{if }\bot^{\mathscr{T}A}\ne K\sqsubseteq X\\
\top^{\mathscr{F}(B)} & \text{if }K\nsqsubseteq X
\end{cases}\right)\sqsupseteq\rsupfun fK;
\end{multline*}
so $g(X,Y)\sqsupseteq f$. For every $X\in\mathscr{T}A$
\[
\bigsqcap_{Y\in\up\rsupfun fX}\rsupfun{g(X,Y)}X=\bigsqcap^{\mathscr{F}}_{Y\in\up\rsupfun
fX}Y=\rsupfun fX;
\]
consequently
\[
\rsupfun{\bigsqcap\setcond{g(X,Y)}{X\in\mathscr{T}A,Y\in\up\rsupfun
fX}}X\sqsubseteq\rsupfun fX
\]
that is
\[
\bigsqcap\setcond{g(X,Y)}{X\in\mathscr{T}A,Y\in\up\rsupfun fX}\sqsubseteq f
\]
 and finally
\[
f=\bigsqcap\setcond{g(X,Y)}{X\in\mathscr{T}A,Y\in\up\rsupfun fX}.
\]
\end{proof}
\begin{cor}
\label{fcd-filtered}Filtrators of funcoids are filtered.
\end{cor}

\begin{thm}\label{metcomp-thm}
~
\begin{enumerate}
\item \label{metcomp}$g$ is metacomplete if $g$ is a complete funcoid.
\item \label{cometcomp}$g$ is co-metacomplete if $g$ is a co-complete
funcoid.
\end{enumerate}
\end{thm}
\begin{proof}
~
\begin{widedisorder}
\item [{\ref{metcomp}}] Let $R$ be a set of funcoids from a set $A$ to a set
$B$ and $g$ be a funcoid from $B$ to some $C$. Then
\begin{align*}
\rsupfun{g\circ\bigsqcup R}X & =\\
\supfun g\rsupfun{\bigsqcup R}X & =\\
\supfun g\bigsqcup_{f\in R}\rsupfun fX & =\\
\bigsqcup_{f\in R}\supfun g\rsupfun fX & =\\
\bigsqcup_{f\in R}\rsupfun{g\circ f}X & =\\
\rsupfun{\bigsqcup_{f\in R}(g\circ f)}X & =\\
\rsupfun{\bigsqcup\rsupfun{g\circ}R}X
\end{align*}
for every typed set $X\in\mathscr{T}A$. So $g\circ\bigsqcup
R=\bigsqcup\rsupfun{g\circ}R$.
\item [{\ref{cometcomp}}] By duality.
\end{widedisorder}
\end{proof}
\begin{conjecture}
$g$ is complete if $g$ is a metacomplete funcoid.
\end{conjecture}
I will denote $\mathsf{ComplFCD}$ and $\mathsf{CoComplFCD}$ the
sets of small complete and co-complete funcoids correspondingly.
$\mathsf{ComplFCD}(A,B)$
are complete funcoids from $A$ to $B$ and likewise with
$\mathsf{CoComplFCD}(A,B)$.
\begin{obvious}
$\mathsf{ComplFCD}$ and $\mathsf{CoComplFCD}$ are closed regarding
composition of funcoids.\end{obvious}
\begin{prop}
$\mathsf{ComplFCD}$ and $\mathsf{CoComplFCD}$ (with induced order)
are complete lattices.\end{prop}
\begin{proof}
It follows from theorem \ref{fcd-join-compl}.\end{proof}
\begin{thm}
Atoms of the lattice $\mathsf{ComplFCD}(A,B)$ are exactly funcoidal
products of the form $\uparrow^{A}\{\alpha\}\times^{\mathsf{FCD}}b$
where $\alpha\in A$ and $b$ is an ultrafilter on $B$.\end{thm}
\begin{proof}
First, it's easy to see that $\uparrow^{A}\{\alpha\}\times^{\mathsf{FCD}}b$
are elements of $\mathsf{ComplFCD}(A,B)$. Also $\bot^{\mathsf{FCD}(A,B)}$
is an element of $\mathsf{ComplFCD}(A,B)$.

$\uparrow^{A}\{\alpha\}\times^{\mathsf{FCD}}b$ are atoms of
$\mathsf{ComplFCD}(A,B)$
because they are atoms of $\mathsf{FCD}(A,B)$.

It remains to prove that if $f$ is an atom of $\mathsf{ComplFCD}(A,B)$
then $f=\uparrow^{A}\{\alpha\}\times^{\mathsf{FCD}}b$ for some $\alpha\in A$
and an ultrafilter $b$ on $B$.

Suppose $f\in\mathsf{FCD}(A,B)$ is a non-empty complete funcoid.
Then there exists $\alpha\in A$ such that $\rsupfun
f@\{\alpha\}\ne\bot^{\mathscr{F}(B)}$.
Thus $\uparrow^{A}\{\alpha\}\text{\texttimes}^{\mathsf{FCD}}b\sqsubseteq f$
for some ultrafilter $b$ on $B$. If $f$ is an atom then
$f=\uparrow^{A}\{\alpha\}\text{\texttimes}^{\mathsf{FCD}}b$.\end{proof}
\begin{thm}
\label{complfcd-rep}$G\mapsto\bigsqcup_{\alpha\in
A}(\uparrow^{A}\{\alpha\}\times^{\mathsf{FCD}}G(\alpha))$
is an order isomorphism from the set of functions $G\in\mathscr{F}(B)^{A}$
to the set $\mathsf{ComplFCD}(A,B)$.

The inverse isomorphism is described by the formula $G(\alpha)=\rsupfun
f@\{\alpha\}$
where $f$ is a complete funcoid.\end{thm}
\begin{proof}
$\bigsqcup_{\alpha\in A}(\uparrow^{A}\{\alpha\}\times^{\mathsf{FCD}}G(\alpha))$
is complete because $G(\alpha)=\bigsqcup\atoms G(\alpha)$ and thus
\[
\bigsqcup_{\alpha\in
A}(\uparrow^{A}\{\alpha\}\times^{\mathsf{FCD}}G(\alpha))=\bigsqcup\setcond{
\uparrow^{A}\{\alpha\}\times^{\mathsf{FCD}}b}{\alpha\in A,b\in\atoms G(\alpha)}
\]
is complete. So $G\mapsto\bigsqcup_{\alpha\in
A}(\uparrow^{A}\{\alpha\}\times^{\mathsf{FCD}}G(\alpha))$
is a function from $G\in\mathscr{F}(B)^{A}$ to $\mathsf{ComplFCD}(A,B)$.

Let $f$ be complete. Then take
\[
G(\alpha)=\bigsqcup\setcond{b\in\atoms^{\mathscr{F}(\Dst
f)}}{\uparrow^{A}\{\alpha\}\times^{\mathsf{FCD}}b\sqsubseteq f}
\]
and we have $f=\bigsqcup_{\alpha\in
A}(\uparrow^{A}\{\alpha\}\times^{\mathsf{FCD}}G(\alpha))$
obviously. So $G\mapsto\bigsqcup_{\alpha\in
A}(\uparrow^{A}\{\alpha\}\times^{\mathsf{FCD}}G(\alpha))$
is surjection onto $\mathsf{ComplFCD}(A,B)$.

Let now prove that it is an injection:

Let
\[
f=\bigsqcup_{\alpha\in
A}(\uparrow^{A}\{\alpha\}\times^{\mathsf{FCD}}F(\alpha))=\bigsqcup_{\alpha\in
A}(\uparrow^{A}\{\alpha\}\times^{\mathsf{FCD}}G(\alpha))
\]
for some $F,G\in\mathscr{F}(\Dst f)^{\Src f}$. We need to prove $F=G$.
Let $\beta\in\Src f$.
\[
\rsupfun f@\{\beta\}=\bigsqcup_{\alpha\in
A}\rsupfun{\uparrow^{A}\{\alpha\}\times^{\mathsf{FCD}}F(\alpha)}@\{
\beta\}=F(\beta).
\]
Similarly $\rsupfun f@\{\beta\}=G(\beta)$. So $F(\beta)=G(\beta)$.

We have proved that it is a bijection. To show that it is monotone
is trivial.

Denote $f=\bigsqcup_{\alpha\in
A}(\uparrow^{A}\{\alpha\}\times^{\mathsf{FCD}}G(\alpha))$.
Then
\begin{multline*}
\rsupfun f@\{\alpha'\}=\text{(because
\ensuremath{\uparrow^{A}\{\alpha'\}} is principal)}=\\
\bigsqcup_{\alpha\in
A}\supfun{\uparrow^{A}\{\alpha\}\times^{\mathsf{FCD}}G(\alpha)}@\{
\alpha'\}=\supfun{\uparrow^{A}\{\alpha'\}\times^{\mathsf{FCD}}G(\alpha')}
@\{\alpha'\}=G(\alpha').
\end{multline*}
\end{proof}
\begin{cor}
$G\mapsto\bigsqcup_{\alpha\in
A}(G(\alpha)\times^{\mathsf{FCD}}\uparrow^{A}\{\alpha\})$
is an order isomorphism from the set of functions $G\in\mathscr{F}(B)^{A}$
to the set $\mathsf{CoComplFCD}(A,B)$.

The inverse isomorphism is described by the formula
$G(\alpha)=\rsupfun{f^{-1}}@\{\alpha\}$
where $f$ is a co-complete funcoid.
\end{cor}

\begin{cor}
$\mathsf{ComplFCD}(A,B)$ and $\mathsf{CoComplFCD}(A,B)$ are co-frames.
\end{cor}

\section{Funcoids corresponding to pretopologies}

Let $\Delta$ be a pretopology on a set $U$ and $\cl$ the preclosure
corresponding to it (see theorem \ref{pretop-bij}).

Both induce a funcoid, I will show that these two funcoids are reverse
of each other:
\begin{thm}
Let $f$ be a complete funcoid defined by the formula $\rsupfun
f@\{x\}=\Delta(x)$
for every $x\in U$, let $g$ be a co-complete funcoid defined by
the formula $\rsupfun gX=\uparrow^{U}\cl(\GR X)$ for every $X\in\mathscr{T}U$.
Then $g=f^{-1}$.\end{thm}
\begin{rem}
It is obvious that funcoids $f$ and $g$ exist.\end{rem}
\begin{proof}
For $X,Y\in\mathscr{T}U$ we have
\begin{align*}
X\rsuprel gY & \Leftrightarrow\\
\uparrow Y\nasymp\supfun g\uparrow X & \Leftrightarrow\\
Y\nasymp\cl(\GR X) & \Leftrightarrow\\
\exists y\in Y:\Delta(y)\nasymp\uparrow X & \Leftrightarrow\\
\exists y\in Y:\rsupfun f\uparrow^{U}\{y\}\nasymp\uparrow X & \Leftrightarrow\\
\text{(proposition \ref{b-f-back-distr} and properties of complete funcoids)}\\
\rsupfun fY\nasymp\uparrow X & \Leftrightarrow\\
Y\rsuprel fX.
\end{align*}


So $g=f^{-1}$.
\end{proof}

\section{Completion of funcoids}
\begin{thm}
$\Cor f=\Cor'f$ for an element $f$ of a filtrator of funcoids.\end{thm}
\begin{proof}
By theorem~\ref{cor-eq} and corollary~\ref{fcd-filtered}.\end{proof}
\begin{defn}
\index{completion!of funcoid}\emph{Completion} of a funcoid
$f\in\mathsf{FCD}(A,B)$
is the complete funcoid $\Compl f\in\mathsf{FCD}(A,B)$ defined by
the formula $\rsupfun{\Compl f}@\{\alpha\}=\rsupfun
f@\{\alpha\}$
for $\alpha\in\Src f$.
\end{defn}

\begin{defn}
\index{co-completion!of funcoid}\emph{Co-completion} of a funcoid
$f$ is defined by the formula
\[
\CoCompl f=(\Compl f^{-1})^{-1}.
\]
\end{defn}
\begin{obvious}
$\Compl f\sqsubseteq f$ and $\CoCompl f\sqsubseteq f$.\end{obvious}
\begin{prop}
The filtrator $(\mathsf{FCD}(A,B),\mathsf{ComplFCD}(A,B))$ is
filtered.\end{prop}
\begin{proof}
Because the filtrator of funcoids is filtered.\end{proof}
\begin{thm}
$\Compl f=\Cor^{\mathsf{ComplFCD}(A,B)}f=\Cor'^{\mathsf{ComplFCD}(A,B)}f$
for every funcoid $f\in\mathsf{FCD}(A,B)$.\end{thm}
\begin{proof}
$\Cor^{\mathsf{ComplFCD}(A,B)}f=\Cor'^{\mathsf{ComplFCD}(A,B)}f$
using theorem \ref{cor-eq} since the filtrator
$(\mathsf{FCD}(A,B),\mathsf{ComplFCD}(A,B))$
is filtered.

Let $g\in\up^{\mathsf{ComplFCD}(A,B)}f$. Then
$g\in\mathsf{ComplFCD}(A,B)$ and $g\sqsupseteq f$. Thus $g=\Compl
g\sqsupseteq\Compl f$.

Thus $\forall g\in\up^{\mathsf{ComplFCD}(A,B)}f:g\sqsupseteq\Compl f$.

Let $\forall g\in\up^{\mathsf{ComplFCD}(A,B)}f:h\sqsubseteq g$
for some $h\in\mathsf{ComplFCD}(A,B)$.

Then $h\sqsubseteq\bigsqcap\up^{\mathsf{ComplFCD}(A,B)}f=f$
and consequently $h=\Compl h\sqsubseteq\Compl f$.

Thus
\[
\Compl
f=\bigsqcap^{\mathsf{ComplFCD}(A,B)}\up^{\mathsf{ComplFCD}(A,B)}f=\Cor^{\mathsf{
ComplFCD}(A,B)}f.
\]
\end{proof}
\begin{thm}
$\rsupfun{\CoCompl f}X=\Cor\rsupfun fX$ for every funcoid $f$ and
typed set $X\in\mathscr{T}(\Src f)$.\end{thm}
\begin{proof}
$\CoCompl f\sqsubseteq f$ thus $\rsupfun{\CoCompl f}X\sqsubseteq\rsupfun fX$
but $\rsupfun{\CoCompl f}X$ is a principal filter thus $\rsupfun{\CoCompl
f}X\sqsubseteq\Cor\rsupfun fX$.

Let $\alpha X=\Cor\rsupfun fX$. Then $\alpha\bot^{\mathscr{T}(\Src
f)}=\bot^{\mathscr{F}(\Dst f)}$
and
\begin{multline*}
\alpha(X\sqcup Y)=\Cor\rsupfun f(X\sqcup Y)=\Cor(\rsupfun fX\sqcup\rsupfun
fY)=\\
\Cor\rsupfun fX\sqcup\Cor\rsupfun fY=\alpha X\sqcup\alpha Y
\end{multline*}
(used theorem~\ref{dual-core-join}). Thus $\alpha$ can be continued
till $\supfun g$ for some funcoid $g$. This funcoid is co-complete.

Evidently $g$ is the greatest co-complete element of $\mathsf{FCD}(\Src f,\Dst
f)$
which is lower than $f$.

Thus $g=\CoCompl f$ and $\Cor\rsupfun fX=\alpha X=\rsupfun gX=\rsupfun{\CoCompl
f}X$.\end{proof}
\begin{thm}
$\mathsf{ComplFCD}(A,B)$ is an atomistic lattice.\end{thm}
\begin{proof}
Let $f\in\mathsf{ComplFCD}(A,B)$, $X\in\mathscr{T}(\Src f)$.
\[
\rsupfun fX=\bigsqcup_{x\in\atoms X}\rsupfun fx=\bigsqcup_{x\in\atoms
X}\rsupfun{f|_{x}}x=\bigsqcup_{x\in\atoms X}\rsupfun{f|_{x}}X,
\]
thus $f=\bigsqcup_{x\in\atoms X}(f|_{x})$. It is trivial that every
$f|_{x}$ is a join of atoms of $\mathsf{ComplFCD}(A,B)$.\end{proof}
\begin{thm}
A funcoid is complete iff it is a join (on the lattice $\mathsf{FCD}(A,B)$)
of atomic complete funcoids.\end{thm}
\begin{proof}
It follows from the theorem \ref{fcd-join-compl} and the previous
theorem.\end{proof}
\begin{cor}
$\mathsf{ComplFCD}(A,B)$ is join-closed.\end{cor}
\begin{thm}
$\Compl\bigsqcup R=\bigsqcup\rsupfun{\Compl}R$ for every
$R\in\subsets\mathsf{FCD}(A,B)$
(for every sets $A$, $B$).\end{thm}
\begin{proof}
For every typed set $X$
\begin{align*}
\rsupfun{\Compl\bigsqcup R}X & =\\
\bigsqcup_{x\in\atoms X}\rsupfun{\bigsqcup R}x & =\\
\bigsqcup_{x\in\atoms X}\bigsqcup_{f\in R}\rsupfun fx & =\\
\bigsqcup_{f\in R}\bigsqcup_{x\in\atoms X}\rsupfun fx & =\\
\bigsqcup_{f\in R}\rsupfun{\Compl f}X & =\\
\rsupfun{\bigsqcup\rsupfun{\Compl}R}X.
\end{align*}
\end{proof}
\begin{cor}
$\Compl$ is a lower adjoint.\end{cor}
\begin{conjecture}
$\Compl$ is not an upper adjoint (in general).\end{conjecture}
\begin{prop}
$\Compl f=\bigsqcup_{\alpha\in\Src f}(f|_{\uparrow\{\alpha\}})$
for every funcoid $f$.\end{prop}
\begin{proof}
Let denote $R$ the right part of the equality to prove.

$\rsupfun R@\{\beta\}=\bigsqcup_{\alpha\in\Src
f}\rsupfun{f|_{\uparrow\{\alpha\}}}@\{\beta\}=\rsupfun
f@\{\beta\}$
for every $\beta\in\Src f$ and $R$ is complete as a join of complete
funcoids.

Thus $R$ is the completion of $f$.\end{proof}
\begin{conjecture}
$\Compl f=f\psetminus(\Omega\times^{\mathsf{FCD}}\mho)$.
\end{conjecture}
This conjecture may be proved by considerations similar to these in
the section ``Fr\'echet filter''.
\begin{lem}
Co-completion of a complete funcoid is complete.\end{lem}
\begin{proof}
Let $f$ be a complete funcoid.
\begin{multline*}
\rsupfun{\CoCompl f}X=\Cor\rsupfun fX=\Cor\bigsqcup_{x\in\atoms X}\rsupfun fx=\\
\bigsqcup_{x\in\atoms X}\Cor\rsupfun fx=\bigsqcup_{x\in\atoms
X}\rsupfun{\CoCompl f}x
\end{multline*}
for every set typed $X\in\mathscr{T}(\Src f)$. Thus $\CoCompl f$
is complete.\end{proof}
\begin{thm}
$\Compl\CoCompl f=\CoCompl\Compl f=\Cor f$ for every funcoid $f$.\end{thm}
\begin{proof}
$\Compl\CoCompl f$ is co-complete since (used the lemma) $\CoCompl f$
is co-complete. Thus $\Compl\CoCompl f$ is a principal funcoid. $\CoCompl f$
is the greatest co-complete funcoid under $f$ and $\Compl\CoCompl f$
is the greatest complete funcoid under $\CoCompl f$. So $\Compl\CoCompl f$
is greater than any principal funcoid under $\CoCompl f$ which is
greater than any principal funcoid under $f$. Thus $\Compl\CoCompl f$
is the greatest principal funcoid under $f$. Thus $\Compl\CoCompl f=\Cor f$.
Similarly $\CoCompl\Compl f=\Cor f$.
\end{proof}

\subsection{More on completion of funcoids}
\begin{prop}
For every composable funcoids $f$ and $g$
\begin{enumerate}
\item \label{compl-ge}$\Compl(g\circ f)\sqsupseteq\Compl g\circ\Compl f$;
\item \label{cocompl-ge}$\CoCompl(g\circ f)\sqsupseteq\CoCompl g\circ\CoCompl
f$.
\end{enumerate}
\end{prop}
\begin{proof}
~
\begin{disorder}
\item [{\ref{compl-ge}}] $\Compl g\circ\Compl f=\Compl(\Compl g\circ\Compl
f)\sqsubseteq\Compl(g\circ f)$.
\item [{\ref{cocompl-ge}}] $\CoCompl g\circ\CoCompl f=\CoCompl(\CoCompl
g\circ\CoCompl f)\sqsubseteq\CoCompl(g\circ f)$.
\end{disorder}
\end{proof}
\begin{prop}\label{comp-compl}
For every composable funcoids $f$ and $g$
\begin{enumerate}
\item \label{cocompl-eq}$\CoCompl(g\circ f)=(\CoCompl g)\circ f$ if $f$
is a co-complete funcoid.
\item \label{compl-eq}$\Compl(f\circ g)=f\circ\Compl g$ if $f$ is a complete
funcoid.
\end{enumerate}
\end{prop}
\begin{proof}
~
\begin{widedisorder}
\item [{\ref{cocompl-eq}}] For every $X\in\mathscr{T}(\Src f)$
\begin{align*}
\rsupfun{\CoCompl(g\circ f)}X & =\\
\Cor\rsupfun{g\circ f}X & =\\
\Cor\supfun g\rsupfun fX & =\\
\supfun{\CoCompl g}\rsupfun fX & =\\
\rsupfun{(\CoCompl g)\circ f}X.
\end{align*}

\item [{\ref{compl-eq}}] $(\CoCompl(g\circ f))^{-1}=f^{-1}\circ(\CoCompl
g)^{-1}$;
$\Compl(g\circ f)^{-1}=f^{-1}\circ\Compl g^{-1}$; $\Compl(f^{-1}\circ
g^{-1})=f^{-1}\circ\Compl g^{-1}$.
After variable replacement we get $\Compl(f\circ g)=f\circ\Compl g$
(after the replacement $f$ is a complete funcoid).
\end{widedisorder}
\end{proof}
\begin{cor}
~
For every composable funcoids $f$ and $g$
\begin{enumerate}
\item $\Compl f\circ\Compl g=\Compl(\Compl f\circ g)$.
\item $\CoCompl g\circ\CoCompl f=\CoCompl(g\circ\CoCompl f)$.
\end{enumerate}
\end{cor}
\begin{prop}
For every composable funcoids $f$ and $g$
\begin{enumerate}
\item \label{compl2-eq}$\Compl(g\circ f)=\Compl(g\circ(\Compl f))$;
\item \label{cocompl2-eq}$\CoCompl(g\circ f)=\CoCompl((\CoCompl g)\circ f)$.
\end{enumerate}
\end{prop}
\begin{proof}
~
\begin{widedisorder}
\item [{\ref{compl2-eq}}] ~
\begin{multline*}
\rsupfun{g\circ(\Compl f)}@\{x\}=\supfun g\rsupfun{\Compl
f}@\{x\}=\\
\supfun g\rsupfun f@\{x\}=\rsupfun{g\circ f}@\{x\}.
\end{multline*}



Thus $\Compl(g\circ(\Compl f))=\Compl(g\circ f)$.

\item [{\ref{cocompl2-eq}}] $(\Compl(g\circ(\Compl f))^{-1}=(\Compl(g\circ
f))^{-1}$;
$\CoCompl(g\circ(\Compl f))^{-1}=\CoCompl(g\circ f)^{-1}$; $\CoCompl((\Compl
f)^{-1}\circ g^{-1})=\CoCompl(f^{-1}\circ g^{-1})$;
$\CoCompl((\CoCompl f^{-1})\circ g^{-1})=\CoCompl(f^{-1}\circ g^{-1})$.
After variable replacement $\CoCompl((\CoCompl g)\circ f)=\CoCompl(g\circ f)$.
\end{widedisorder}
\end{proof}
\begin{thm}
The filtrator of funcoids (from a given set~$A$ to a given set~$B$)
is with co-separable core.\end{thm}
\begin{proof}
Let $f,g\in\mathsf{FCD}(A,B)$ and $f\sqcup g=\top$. Then for every
$X\in\mathscr{T}A$ we have
\begin{multline*}
\supfun f^{\ast}X\sqcup\supfun g^{\ast}X=\top\Leftrightarrow\Cor\supfun
f^{\ast}X\sqcup\Cor\supfun g^{\ast}X=\top\Leftrightarrow\\
\rsupfun{\CoCompl f}X\sqcup\rsupfun{\CoCompl g}X=\top.
\end{multline*}
Thus $\rsupfun{\CoCompl f\sqcup\CoCompl g}X=\top$;
\begin{equation}
f\sqcup g=\top\Rightarrow\CoCompl f\sqcup\CoCompl g=\top.\label{fcd-sep-d}
\end{equation}
Applying the dual of the formulas (\ref{fcd-sep-d}) to the formula
(\ref{fcd-sep-d}) we get:
\[
f\sqcup g=\top\Rightarrow\Compl\CoCompl f\sqcup\Compl\CoCompl g=\top
\]
that is $f\sqcup g=\top\Rightarrow\Cor f\sqcup\Cor g=\top$. So
$\mathsf{FCD}(A,B)$
is with co-separable core.\end{proof}
\begin{cor}
The filtrator of complete funcoids is also with co-separable core.
\end{cor}

\section{Monovalued and injective funcoids}

\index{funcoid!monovalued}\index{monovalued!funcoid}Following the
idea of definition of monovalued morphism let's call \emph{monovalued}
such a funcoid $f$ that $f\circ f^{-1}\sqsubseteq\id_{\im f}^{\mathsf{FCD}}$.

\index{funcoid!injective}\index{injective!funcoid}Similarly, I will
call a funcoid injective when $f^{-1}\circ f\sqsubseteq\id_{\dom
f}^{\mathsf{FCD}}$.
\begin{obvious}
A funcoid $f$ is:
\begin{enumerate}
\item monovalued iff $f\circ f^{-1}\sqsubseteq1_{\Dst f}^{\mathsf{FCD}}$;
\item injective iff $f^{-1}\circ f\sqsubseteq1_{\Src f}^{\mathsf{FCD}}$.
\end{enumerate}
\end{obvious}
In other words, a funcoid is monovalued (injective) when it is a monovalued
(injective) morphism of the category of funcoids. Monovaluedness is
dual of injectivity.
\begin{obvious}
~
\begin{enumerate}
\item A morphism $(\mathcal{A},\mathcal{B},f)$ of the category of funcoid
triples is monovalued iff the funcoid $f$ is monovalued.
\item A morphism $(\mathcal{A},\mathcal{B},f)$ of the category of funcoid
triples is injective iff the funcoid $f$ is injective.
\end{enumerate}
\end{obvious}
\begin{thm}
The following statements are equivalent for a funcoid $f$:
\begin{enumerate}
\item \label{fcd-mv}$f$ is monovalued.
\item \label{fcd-mmv}It is metamonovalued.
\item \label{fcd-wmmv}It is weakly metamonovalued.
\item \label{mnv-atom}$\forall a\in\atoms^{\mathscr{F}(\Src f)}:\supfun
fa\in\atoms^{\mathscr{F}(\Dst f)}\cup\{\bot^{\mathscr{F}(\Dst f)}\}$.
\item \label{mnv-flt}$\forall\mathcal{I},\mathcal{J}\in\mathscr{F}(\Dst
f):\supfun{f^{-1}}(\mathcal{I}\sqcap\mathcal{J})=\supfun{f^{-1}}\mathcal{I}
\sqcap\supfun{f^{-1}}\mathcal{J}$.
\item \label{mnv-set}$\forall I,J\in\mathscr{T}(\Dst f):\rsupfun{f^{-1}}(I\sqcap
J)=\rsupfun{f^{-1}}I\sqcap\rsupfun{f^{-1}}J$.
\end{enumerate}
\end{thm}
\begin{proof}
~
\begin{description}
\item [{\ref{mnv-atom}$\Rightarrow$\ref{mnv-flt}}] Let
$a\in\atoms^{\mathscr{F}(\Src f)}$,
$\supfun fa=b$. Then because $b\in\atoms^{\mathscr{F}(\Dst
f)}\cup\{\bot^{\mathscr{F}(\Dst f)}\}$
\begin{gather*}
(\mathcal{I}\sqcap\mathcal{J})\sqcap b\ne\bot \Leftrightarrow
\mathcal{I}\sqcap b\ne\bot\land\mathcal{J}\sqcap b\ne\bot;\\
a\suprel f\mathcal{I}\sqcap\mathcal{J}\Leftrightarrow a\suprel f\mathcal{I}\land
a\suprel f\mathcal{J};\\
\mathcal{I}\sqcap\mathcal{J}\suprel{f^{-1}}a\Leftrightarrow\mathcal{I}\suprel{f^
{-1}}a\land\mathcal{J}\suprel{f^{-1}}a;\\
a\sqcap\supfun{f^{-1}}(\mathcal{I}\sqcap\mathcal{J})\ne\bot
\Leftrightarrow a\sqcap\supfun{f^{-1}}\mathcal{I}\ne\bot
\land a\sqcap\supfun{f^{-1}}\mathcal{J}\ne\bot;\\
\supfun{f^{-1}}(\mathcal{I}\sqcap\mathcal{J})=\supfun{f^{-1}}\mathcal{I}
\sqcap\supfun{f^{-1}}\mathcal{J}.
\end{gather*}

\item [{\ref{mnv-flt}$\Rightarrow$\ref{fcd-mv}}]
$\supfun{f^{-1}}a\sqcap\supfun{f^{-1}}b=\supfun{f^{-1}}(a\sqcap
b)=\supfun{f^{-1}}\bot=\bot$
for every two distinct atomic filter objects $a$ and $b$ on $\Dst f$.
This is equivalent to $\lnot(\supfun{f^{-1}}a\suprel fb)$; $b\asymp\supfun
f\supfun{f^{-1}}a$;
$b\asymp\supfun{f\circ f^{-1}}a$; $\lnot(a\suprel{f\circ f^{-1}}b)$.
So $a\suprel{f\circ f^{-1}}b\Rightarrow a=b$ for every ultrafilters
$a$ and $b$. This is possible only when $f\circ f^{-1}\sqsubseteq1_{\Dst
f}^{\mathsf{FCD}}$.
\item [{\ref{mnv-set}$\Rightarrow$\ref{mnv-flt}}] ~
\begin{align*}
\supfun{f^{-1}}(\mathcal{I}\sqcap\mathcal{J}) & =\\
\bigsqcap\rsupfun{\rsupfun f}\up(\mathcal{I}\sqcap\mathcal{J}) & =\\
\bigsqcap\rsupfun{\rsupfun f}\setcond{I\sqcap
J}{I\in\up\mathcal{I},J\in\up\mathcal{J}} & =\\
\bigsqcap\setcond{\rsupfun f(I\sqcap J)}{I\in\up\mathcal{I},J\in\up\mathcal{J}}
& =\\
\bigsqcap\setcond{\rsupfun fI\sqcap\rsupfun
fJ}{I\in\up\mathcal{I},J\in\up\mathcal{J}} & =\\
\bigsqcap\setcond{\rsupfun
fI}{I\in\up\mathcal{I}}\sqcap\bigsqcap\setcond{\rsupfun fJ}{J\in\up\mathcal{J}}
& =\\
\supfun{f^{-1}}\mathcal{I}\sqcap\supfun{f^{-1}}\mathcal{J}.
\end{align*}

\item [{\ref{mnv-flt}$\Rightarrow$\ref{mnv-set}}] Obvious.
\item [{$\lnot$\ref{mnv-atom}$\Rightarrow$$\lnot$\ref{fcd-mv}}] Suppose
$\supfun fa\notin\atoms^{\mathscr{F}(\Dst f)}\cup\{\bot^{\mathscr{F}(\Dst f)}\}$
for some $a\in\atoms^{\mathscr{F}(\Src f)}$. Then there exist two
atomic filters $p$ and $q$ on $\Dst f$ such that $p\ne q$ and
$\supfun fa\sqsupseteq p\land\supfun fa\sqsupseteq q$. Consequently
$p\nasymp\supfun fa$; $a\nasymp\supfun{f^{-1}}p$;
$a\sqsubseteq\supfun{f^{-1}}p$;
$\supfun{f\circ f^{-1}}p=\supfun f\supfun{f^{-1}}p\sqsupseteq\supfun
fa\sqsupseteq q$;
$\supfun{f\circ f^{-1}}p\nsqsubseteq p$ and $\supfun{f\circ
f^{-1}}p\ne\bot^{\mathscr{F}(\Dst f)}$.
So it cannot be $f\circ f^{-1}\sqsubseteq1_{\Dst f}^{\mathsf{FCD}}$.
\item [{\ref{fcd-mmv}$\Rightarrow$\ref{fcd-wmmv}}] Obvious.
\item [{\ref{fcd-mv}$\Rightarrow$\ref{fcd-mmv}}] ~
\[
\supfun{\left(\bigsqcap G\right)\circ f}x=\supfun{\bigsqcap
G}\supfun fx=\bigsqcap_{g\in G}\supfun g\supfun fx=\bigsqcap_{g\in
G}\supfun{g\circ f}x=\supfun{\bigsqcap_{g\in G}(g\circ f)}x
\]
for every atomic filter object $x\in\atoms^{\mathscr{F}(\Src f)}$.
Thus $\left(\bigsqcap G\right)\circ f=\bigsqcap_{g\in G}(g\circ f)$.
\item [{\ref{fcd-wmmv}$\Rightarrow$\ref{fcd-mv}}] Take
$g=a\times^{\mathsf{FCD}}y$
and $h=b\times^{\mathsf{FCD}}y$ for arbitrary atomic filter objects
$a\ne b$ and~$y$. We have $g\sqcap h=\bot$; thus $(g\circ f)\sqcap(h\circ
f)=(g\sqcap h)\circ f=\bot$
and thus impossible $x\suprel fa\land x\suprel fb$ as otherwise $x\suprel{g\circ
f}y$
and $x\suprel{h\circ f}y$ so $x\suprel{(g\circ f)\sqcap(h\circ f)}y$.
Thus $f$ is monovalued.
\end{description}
\end{proof}
\begin{cor}
A binary relation corresponds to a monovalued funcoid iff it is a
function.\end{cor}
\begin{proof}
Because $\forall I,J\in\subsets(\im f):\rsupfun{f^{-1}}(I\sqcap
J)=\rsupfun{f^{-1}}I\sqcap\rsupfun{f^{-1}}J$
is true for a funcoid $f$ corresponding to a binary relation if and
only if it is a function (see proposition~\ref{rel-mono}).\end{proof}
\begin{rem}
This corollary can be reformulated as follows: For binary relations
(principal funcoids) the classic concept of monovaluedness and monovaluedness
in the above defined sense of monovaluedness of a funcoid are the
same.\end{rem}
\begin{thm}
If $f$, $g$ are funcoids, $f\sqsubseteq g$ and $g$ is monovalued
then $g|_{\dom f}=f$.\end{thm}
\begin{proof}
Obviously $g|_{\dom f}\sqsupseteq f$. Suppose for contrary that $g|_{\dom
f}\sqsubset f$.
Then there exists an atom $a\in\atoms\dom f$ such that $\langle g|_{\dom
f}\rangle a\neq\langle f\rangle a$
that is $\supfun ga\sqsubset\supfun fa$ what is impossible.
\end{proof}

\section{Open maps}
\begin{defn}
\index{open map}An \emph{open map} from a topological space to a
topological space is a function which maps open sets into open sets.
\end{defn}
An obvious generalization of this is \emph{open map} $f$ from an
endofuncoid $\mu$ to an endofuncoid $\nu$, which is by definition
a function (or rather a principal, entirely defined, monovalued funcoid)
from $\Ob\mu$ to $\Ob\nu$ such that
\[
\forall x\in\Ob\mu,V\in\rsupfun{\mu}\{x\}:\rsupfun
fV\sqsupseteq\supfun{\nu}\rsupfun f@\{x\}.
\]


This formula is equivalent (exercise!) to
\[
\forall x\in\Ob\mu:\supfun f\rsupfun{\mu}@\{x\}\sqsupseteq\supfun{\nu}\rsupfun
f@\{x\}.
\]


It can be abstracted/simplified further (now for an \emph{arbitrary}
funcoid $f$ from $\Ob\mu$ to $\Ob\nu$):
\[
\Compl(f\circ\mu)\sqsupseteq\Compl(\nu\circ f).
\]

\begin{defn}
\index{funcoid!open}An \emph{open funcoid} from an endofuncoid $\mu$
to an endofuncoid $\nu$ is a funcoid $f$ from $\Ob\mu$ to $\Ob\nu$
such that $\Compl(f\circ\mu)\sqsupseteq\Compl(\nu\circ f)$.\end{defn}
\begin{obvious}
A funcoid $f$ is open iff $f\circ\mu\sqsupseteq\Compl(\nu\circ f)$.\end{obvious}
\begin{thm}
Let $\mu$, $\nu$, $\pi$ be endofuncoids. Let $f$ be an
principal monovalued open funcoid from $\Ob\mu$ to $\Ob\nu$ and $g$ is a open funcoid
from $\Ob\nu$ to $\Ob\pi$. Then $g\circ f$ is an open funcoid from
$\Ob\mu$ to $\Ob\pi$.\end{thm}
\begin{proof}
\begin{align*}
\supfun{g\circ f}\rsupfun{\mu}@\{x\} & = \\
\supfun{g}\supfun{f}\rsupfun{\mu}@\{x\} & \sqsupseteq \\
\supfun{g}\supfun{\nu}\rsupfun{f}@\{x\} & \sqsupseteq \text{ (using that $f$ is monovalued and principal)}\\
\supfun{\pi}\supfun{g}\rsupfun{f}@\{x\} & = \\
\supfun{\pi}\supfun{g\circ f}@\{x\}.
\end{align*}
\end{proof}

\begin{problem}
Devise a pointfree (not using a particular point~$x$) proof of the above theorem.
It should refer to a lemma which may use a particular point, but the proof of the
theorem itself should be without a particular point.
\end{problem}

\section{\texorpdfstring{$T_{0}$-, $T_{1}$-, $T_{2}$-, $T_{3}$-, and $T_{4}$-separable
funcoids}%
{T0-, T1-, T2-, and T3-separable funcoids}}

\index{funcoids!separable}For funcoids it can be generalized $T_{0}$-,
$T_{1}$-, $T_{2}$-, and $T_{3}$- separability. Worthwhile note
that $T_{0}$ and $T_{2}$ separability is defined through $T_{1}$
separability.
\begin{defn}
Let call \emph{$T_{1}$-separable} such endofuncoid $f$ that for
every $\alpha,\beta\in\Ob f$ is true
\[
\alpha\ne\beta\Rightarrow\lnot(@\{\alpha\}\rsuprel f@\{\beta\}).
\]
\end{defn}
\begin{prop}
An endofuncoid $f$ is $T_{1}$-separable iff $\Cor f\sqsubseteq1_{\Ob
f}^{\mathsf{FCD}}$.\end{prop}
\begin{proof}
~
\begin{multline*}
\forall x,y\in\Ob f:(@\{x\}\rsuprel{f}@\{y\}\Rightarrow x=y)\Leftrightarrow\\
\forall x,y\in\Ob f:(@\{x\}\rsuprel{\Cor f}@\{y\}\Rightarrow x=y)\Leftrightarrow\Cor f\sqsubseteq1_{\Ob f}^{\mathsf{FCD}}.
\end{multline*}
\end{proof}

\begin{prop}
An endofuncoid~$f$ is $T_{1}$-separable iff $\Cor\rsupfun{f}\{x\}\sqsubseteq\{x\}$ for every $x\in\Ob f$.
\end{prop}

\begin{proof}
$\Cor \rsupfun{f} \{ x \} \sqsubseteq \{ x \}
\Leftrightarrow \rsupfun{\CoCompl f} \{ x \} \sqsubseteq \{
x \} \Leftrightarrow \Compl \CoCompl f \sqsubseteq
1^{\mathsf{FCD}}_{\Ob f} \Leftrightarrow \Cor f \sqsubseteq
1^{\mathsf{FCD}}_{\Ob f}$.
\end{proof}

\begin{defn}
Let call \emph{$T_{0}$-separable} such funcoid $f\in\mathsf{FCD}(A,A)$
that $f\sqcap f^{-1}$ is $T_{1}$-separable.
\end{defn}

\begin{defn}
Let call \emph{$T_{2}$-separable} such funcoid $f$ that $f^{-1}\circ f$
is $T_{1}$-separable.
\end{defn}
For symmetric transitive funcoids $T_{0}$-, $T_{1}$- and $T_{2}$-separability
are the same (see theorem \ref{sym-trans}).
\begin{obvious}
A funcoid $f$ is $T_{2}$-separable iff $\alpha\ne\beta\Rightarrow\rsupfun
f@\{\alpha\}\nasymp\rsupfun f@\{\beta\}$
for every $\alpha,\beta\in\Src f$.\end{obvious}
\begin{defn}
Funcoid $f$ is \emph{regular} iff for every $C\in\mathscr{T}\Dst f$ and
$p\in\Src f$
\[\supfun{f} \langle f^{- 1} \rangle C \asymp \supfun{f}
@\{ p \} \Leftarrow \uparrow^{\Src f} \{ p \}
\asymp \langle f^{- 1} \rangle C.\]
\end{defn}

\begin{prop}
The following are pairwise equivalent:
\begin{enumerate}
  \item A funcoid $f$ is regular.
  \item $\Compl (f \circ f^{- 1} \circ f) \sqsubseteq \Compl f$.
  \item $\Compl (f \circ f^{- 1} \circ f) \sqsubseteq f$.
\end{enumerate}
\end{prop}

\begin{proof}
Equivalently transform the defining formula for regular funcoids:

$\supfun{f} \langle f^{- 1} \rangle C \asymp \supfun{f}
@\{ p \} \Leftarrow \uparrow^{\Src f} \{ p \}
\asymp \langle f^{- 1} \rangle C$;

$\supfun{f} \langle f^{- 1} \rangle C \nasymp \supfun{f}
@\{ p \} \Rightarrow \uparrow^{\Src f} \{ p \} \nasymp \supfun{f^{-1}}C$;

(by definition of funcoids)

$C \nasymp \supfun{f} \langle f^{- 1} \rangle \supfun{f}
@\{ p \} \Rightarrow C \nasymp \supfun{f}
@\{ p \}$;

$\supfun{f} \langle f^{- 1} \rangle \supfun{f}
@\{ p \} \sqsubseteq \supfun{f}@\{ p \}$;

$\supfun{f \circ f^{- 1} \circ f} @\{ p \}
\sqsubseteq \supfun{f} @\{ p \}$;

$\Compl (f \circ f^{- 1} \circ f) \sqsubseteq \Compl f$;

$\Compl (f \circ f^{- 1} \circ f) \sqsubseteq f$.
\end{proof}

\begin{prop}
If $f$ is complete, regularity of funcoid $f$ is equivalent to $f \circ
\Compl (f^{- 1} \circ f) \sqsubseteq f$.
\end{prop}

\begin{proof}
  By proposition~\ref{comp-compl}.
\end{proof}

\begin{rem}
After seeing how it collapses into algebraic formulas about funcoids, the
definition for a funcoid being regular seems quite arbitrary and sucked out of
the finger (not an example of algebraic elegance). So I present these formulas
only because they coincide with the traditional definition of regular
topological spaces. However this is only my personal opinion and it may be
wrong.
\end{rem}


\begin{defn}
An endofuncoid is $T_{3}$- iff it is both $T_{2}$- and regular.
\end{defn}

A topological space~$S$ is called $T_4$-separable when for any two disjoint closed sets $A,B\subseteq S$
there exist disjoint open sets~$U$,~$V$ containing $A$ and $B$ respectively.

Let $f$ be the complete funcoid corresponding to the topological space.

Since the closed sets are exactly sets of the form $\rsupfun{f^{- 1}} X$ and sets $X$ and $Y$ having non-intersecting open
neighborhood is equivalent to $\rsupfun{f} X \asymp \langle f
\rangle^{\ast} Y$, the above is equivalent to:

$\rsupfun{f^{-1}} A \asymp \rsupfun{f^{-1}} B
\Rightarrow \rsupfun{f} \rsupfun{f^{-1}} A \asymp
\rsupfun{f} \rsupfun{f^{-1}} B$;

$\rsupfun{f} \rsupfun{f^{-1}} A \nasymp \langle f
\rangle^{\ast} \rsupfun{f^{-1}} B \Rightarrow \langle f^{- 1}
\rangle^{\ast} A \nasymp \rsupfun{f^{-1}} B$;

$\rsupfun{f} \rsupfun{f^{-1}} \langle f
\rangle^{\ast} \rsupfun{f^{-1}} A \nasymp B \Rightarrow \langle
f \rangle^{\ast} \rsupfun{f^{-1}} A \nasymp B$;

$\rsupfun{f} \rsupfun{f^{-1}} \langle f
\rangle^{\ast} \rsupfun{f^{-1}} A \sqsubseteq \langle f
\rangle^{\ast} \rsupfun{f^{-1}} A$;

$f \circ f^{- 1} \circ f \circ f^{- 1} \sqsubseteq f \circ f^{- 1}$.

Take the last formula as the definition of $T_4$-funcoid~$f$.

\section{Filters closed regarding a funcoid}
\begin{defn}
\index{filter!closed}Let's call \emph{closed} regarding a funcoid
$f\in\mathsf{FCD}(A,A)$ such filter $\mathcal{A}\in\mathscr{F}(\Src f)$
that $\supfun f\mathcal{A}\sqsubseteq\mathcal{A}$.
\end{defn}
This is a generalization of closedness of a set regarding an unary
operation.
\begin{prop}
If $I$ and $J$ are closed (regarding some funcoid $f$), $S$ is
a set of closed filters on $\Src f$, then
\begin{enumerate}
\item $\mathcal{I}\sqcup\mathcal{J}$ is a closed filter;
\item $\bigsqcap S$ is a closed filter.
\end{enumerate}
\end{prop}
\begin{proof}
Let denote the given funcoid as $f$. $\supfun
f(\mathcal{I}\sqcup\mathcal{J})=\supfun f\mathcal{I}\sqcup\supfun
f\mathcal{J}\sqsubseteq\mathcal{I}\sqcup\mathcal{J}$,
$\supfun f\bigsqcap S\sqsubseteq\bigsqcap\rsupfun{\supfun
f}S\sqsubseteq\bigsqcap S$.
Consequently the filters $\mathcal{I}\sqcup\mathcal{J}$ and $\bigsqcap S$
are closed.\end{proof}
\begin{prop}
If $S$ is a set of filters closed regarding a complete funcoid, then
the filter $\bigsqcup S$ is also closed regarding our funcoid.\end{prop}
\begin{proof}
$\supfun f\bigsqcup S=\bigsqcup\rsupfun{\supfun f}S\sqsubseteq\bigsqcup S$
where $f$ is the given funcoid.
\end{proof}

\section{Proximity spaces}

Fix a set $U$. Let equate typed subsets of $U$ with subsets of~$U$.

We will prove that proximity spaces are essentially the same as reflexive,
symmetric, transitive funcoids.

Our primary interest here is the last axiom (\ref{prox-last}) in
the definition~\ref{prox} of proximity spaces.
\begin{prop}
If $f$ is a transitive, symmetric funcoid, then the last axiom of
proximity holds.\end{prop}
\begin{proof}
~
\begin{multline*}
\neg\left(A\rsuprel fB\right)\Leftrightarrow\neg\left(A\rsuprel{f^{-1}\circ
f}B\right)\Leftrightarrow\rsupfun fB\asymp\rsupfun fA\Leftrightarrow\\
\exists M\in U:M\asymp\rsupfun fA\wedge\overline{M}\asymp\rsupfun fB.
\end{multline*}
\end{proof}
\begin{prop}
For a reflexive funcoid, the last axiom of proximity implies that
it is transitive and symmetric.\end{prop}
\begin{proof}
Let $\neg\left(A\rsuprel fB\right)$ implies $\exists M:M\asymp\rsupfun
fA\wedge\overline{M}\asymp\rsupfun fB$.
Then $\neg\left(A\rsuprel fB\right)$ implies $M\asymp\rsupfun{f}A\land\rsupfun{f}B\sqsubseteq M$,
thus $\rsupfun{f}A\asymp\rsupfun{f}B$; $\neg\left(A\rsuprel{f^{-1}\circ
f}B\right)$
that is $f\sqsupseteq f^{-1}\circ f$ and thus $f=f^{-1}\circ f$.
By theorem \ref{sym-trans} $f$ is transitive and symmetric.\end{proof}
\begin{thm}
Reflexive, symmetric, transitive funcoids endofuncoids on a set~$U$
are essentially the same as proximity spaces on~$U$.\end{thm}
\begin{proof}
Above and theorem~\ref{fcd-as-cont}.\end{proof}
