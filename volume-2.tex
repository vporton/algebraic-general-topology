\documentclass[a4paper,oneside,english,reqno]{amsbook}
\usepackage[T1]{fontenc}
\synctex=-1
% \usepackage{xcolor}
% \usepackage{hyperref}
\usepackage{babel}
\usepackage{textcomp}
\usepackage{mathrsfs}
\usepackage{url}
\usepackage{amstext}
\usepackage{amsmath}
\usepackage{amsthm}
\usepackage{amssymb}
\usepackage{stmaryrd}
\usepackage{agt}
\makeindex
% \usepackage[all]{xy}
% \usepackage[unicode=true,
%  bookmarks=true,bookmarksnumbered=true,bookmarksopen=false,
%  breaklinks=false,pdfborder={0 0 0},backref=false,colorlinks=true]
%  {hyperref}
\hypersetup{pdftitle={Algebraic General Topology. Volume 2 partial draft},
 pdfauthor={Victor Porton},
 pdfsubject={general topology},
 pdfkeywords={algebraic general topology,quasi-uniform spaces,generalizations of proximity spaces,generalizations of nearness spaces,generalizations of uniform spaces}}

\usepackage{xr,refcount}
\externaldocument[book-]{book}
\newcommand{\bookref}[1]{\ref*{book-#1}}

% Continue numbering of book.pdf
\setcounter{thm}{\getrefnumber{book-finalthm}}
\addtocounter{thm}{-1}

\newcommand{\End}{\operatorname{End}}
\newcommand{\cont}{\operatorname{cont}}

\begin{document}

\title{Algebraic General Topology. Volume 2 partial draft}

\author{Victor Porton}

\email{\href{mailto:porton@narod.ru}{porton@narod.ru}}


\urladdr{\href{http://www.mathematics21.org}{http://www.mathematics21.org}}


\date{\today}


\begin{abstract}
Partial rough draft of volume 2 of Algebraic General Topology book.
This volume is meant to contain materials which refer to more advanced
prerequisites than plain ZFC (such as category theory and classical pointfree
topology). This is a \textbf{very} rough draft.
\end{abstract}


\keywords{algebraic general topology, quasi-uniform spaces, generalizations
of proximity spaces, generalizations of nearness spaces, generalizations
of uniform spaces}


\subjclass[2000]{54J05, 54A05, 54D99, 54E05, 54E15, 54E17, 54E99}

\maketitle

\tableofcontents{}

It is a continuation of \cite{volume-1}.

\chapter{Introduction}

I remind some definitions from volume~1~\cite{volume-1}.

I denote a set definition like $\setcond{x\in A}{P(x)}$ instead of customary
$\{ x\in A \mid P(x) \}$ (in order to reduce formulas size).

I denote partial order as~$\sqsubseteq$. I denote lattice operations as
$\bigsqcap$, $\bigsqcup$, $\sqcap$,~$\sqcup$.

The following generalizes monovalued morphisms in category~$\mathbf{Rel}$.

Let $\Hom$-sets be complete lattices.
\begin{defn}
\index{morphism!metamonovalued}A morphism $f$ of a partially ordered
category is \emph{metamonovalued} when $\left(\bigsqcap G\right)\circ f=\bigsqcap_{g\in G}(g\circ f)$
whenever $G$ is a set of morphisms with a suitable domain and image.
\end{defn}

\begin{defn}
\index{morphism!metainjective}A morphism $f$ of a partially ordered
category is \emph{metainjective} when $f\circ\left(\bigsqcap G\right)=\bigsqcap_{g\in G}(f\circ g)$
whenever $G$ is a set of morphisms with a suitable domain and image.\end{defn}
\begin{obvious}
Metamonovaluedness and metainjectivity are dual to each other.\end{obvious}
\begin{defn}
\index{morphism!metacomplete}A morphism $f$ of a partially ordered
category is \emph{metacomplete} when $f\circ\left(\bigsqcup G\right)=\bigsqcup_{g\in G}(f\circ g)$
whenever $G$ is a set of morphisms with a suitable domain and image.
\end{defn}

\begin{defn}
\index{morphism!co-metacomplete}A morphism $f$ of a partially ordered
category is \emph{co-metacomplete} when $\left(\bigsqcup G\right)\circ f=\bigsqcup_{g\in G}(g\circ f)$
whenever $G$ is a set of morphisms with a suitable domain and image.
\end{defn}
Let now $\Hom$-sets be meet-semilattices.
\begin{defn}
\index{morphism!weakly metamonovalued}A morphism $f$ of a partially
ordered category is \emph{weakly metamonovalued} when $(g\sqcap h)\circ f=(g\circ f)\sqcap(h\circ f)$
whenever $g$ and $h$ are morphisms with a suitable domain and image.
\end{defn}

\begin{defn}
\index{morphism!weakly metainjective}A morphism $f$ of a partially
ordered category is \emph{weakly metainjective} when $f\circ(g\sqcap h)=(f\circ g)\sqcap(f\circ h)$
whenever $g$ and $h$ are morphisms with a suitable domain and image.
\end{defn}
Let now $\Hom$-sets be join-semilattices.
\begin{defn}
\index{morphism!weakly metacomplete}A morphism $f$ of a partially
ordered category is \emph{weakly metacomplete} when $f\circ(g\sqcup h)=(f\circ g)\sqcup(f\circ h)$
whenever $g$ and $h$ are morphisms with a suitable domain and image.
\end{defn}

\begin{defn}
\index{morphism!weakly co-metacomplete}A morphism $f$ of a partially
ordered category is \emph{weakly co-metacomplete} when $(g\sqcup h)\circ f=(g\circ f)\sqcup(h\circ f)$
whenever $g$ and $h$ are morphisms with a suitable domain and image.\end{defn}
\begin{obvious}
~
\begin{enumerate}
\item Metamonovalued morphisms are weakly metamonovalued.
\item Metainjective morphisms are weakly metainjective.
\item Metacomplete morphisms are weakly metacomplete.
\item Co-metacomplete morphisms are weakly co-metacomplete.
\end{enumerate}
\end{obvious}

\begin{defn}
\index{morphism!monovalued}For a partially ordered dagger category
I will call \emph{monovalued} morphism such a morphism $f$ that $f\circ f^{\dagger}\sqsubseteq1_{\Dst f}$.
\end{defn}

\begin{defn}
\index{morphism!entirely defined}For a partially ordered dagger category
I will call \emph{entirely defined} morphism such a morphism $f$
that $f^{\dagger}\circ f\sqsupseteq1_{\Src f}$.
\end{defn}

\begin{defn}
\index{morphism!injective}For a partially ordered dagger category
I will call \emph{injective} morphism such a morphism $f$ that $f^{\dagger}\circ f\sqsubseteq1_{\Src f}$.
\end{defn}

\begin{defn}
\index{morphism!surjective}For a partially ordered dagger category
I will call \emph{surjective} morphism such a morphism f that $f\circ f^{\dagger}\sqsupseteq1_{\Dst f}$.\end{defn}
\begin{rem}
It is easy to show that this is a generalization of monovalued, entirely
defined, injective, and surjective functions as morphisms of the category
$\mathbf{Rel}$.\end{rem}
\begin{obvious}
``Injective morphism'' is a dual of ``monovalued morphism'' and
``surjective morphism'' is a dual of ``entirely defined morphism''.\end{obvious}


\chapter{Products in dagger categories with complete ordered Hom-sets}

\fxwarning{This is a rough draft. It is not yet checked for errors.}

\begin{note}
  What I previously denoted $\prod F$ is now denoted $\bigodot^{\text{proj}}_{\sqcap} F$ (and
  likewise for $\mathord{\coprod}$). The other draft chapters referring to
  this chapter may be not yet updated.
\end{note}

\begin{prop}
  ~
  \fxwarning{Should we move this to volume 1?}
  \begin{enumerate}
    \item Every entirely defined monovalued morphism is metamonovalued and metacomplete.
    \item Every surjective injective morphism is metainjective and co-metacomplete.
  \end{enumerate}

\end{prop}

\begin{proof}
Let's prove the first (the second follows from duality):
  
Let $f$ be an entirely defined monovalued morphism.

$\left( \bigsqcap G \right) \circ f \sqsubseteq \bigsqcap_{g \in G} (g \circ
f)$ by monotonicity of composition.

Using the fact that $f$ is monovalued and entirely defined:

$\left( \bigsqcap_{g \in G} (g \circ f) \right) \circ f^{\dagger} \sqsubseteq
\bigsqcap_{g \in G} (g \circ f \circ f^{\dagger}) \sqsubseteq \bigsqcap G$;

$\bigsqcap_{g \in G} (g \circ f) \sqsubseteq \left( \bigsqcap_{g \in G} (g
\circ f) \right) \circ f^{\dagger} \circ f \sqsubseteq \left( \bigsqcap G
\right) \circ f$.

So $\left( \bigsqcap G \right) \circ f = \bigsqcap_{g \in G} (g \circ f)$.

Let $f$ be a entirely defined monovalued morphism.

$f \circ \left( \bigsqcup G \right) \sqsupseteq \bigsqcup_{g \in G} (f \circ
g)$ by monotonicity of composition.

Using the fact that $f$ is entirely defined and monovalued:

$f^{\dagger} \circ \left( \bigsqcup_{g \in G} (f \circ g) \right) \sqsupseteq
\bigsqcup_{g \in G} (f^{\dagger} \circ f \circ g) \sqsupseteq \bigsqcap G$;

$\bigsqcup_{g \in G} (f \circ g) \sqsupseteq f \circ f^{\dagger} \circ
\bigsqcup_{g \in G} (f \circ g) \sqsupseteq f \circ \left( \bigsqcup G
\right)$.

So $f \circ \left( \bigsqcup G \right) = \bigsqcup_{g \in G} (f \circ g)$.
\end{proof}

\section{General product in partially ordered dagger category}

To understand the below better, you can restrict your imagination to the case
when $\mathcal{C}$ is the category $\mathbf{Rel}$.

\subsection{Infimum product}

Let $\mathcal{C}$ be a dagger category, each Hom-set of which is a complete
lattice (having order agreed with the dagger).

We will designate some morphisms as \emph{principal} and require that
principal morphisms are both metacomplete and co-metacomplete. (For a
particular example of the category $\mathbf{Rel}$, all morphisms are
considered principal.)

Let $\prod^{(Q)} X$ be an object for each indexed family $X$ of objects.

Let $\pi$ be a partial function mapping elements $X \in \dom \pi$ (which
consists of small indexed families of objects of $\mathcal{C}$) to indexed
families $\prod^{(Q)} X \rightarrow X_i$ of principal morphisms (called
\emph{projections}) for every $i \in \dom X$.

We will denote particular morphisms as $\pi^X_i$.

\begin{rem}
  In some important examples the function $\pi$ is entire, that is $\dom
  \pi$ is the set of all small indexed families of objects of $\mathcal{C}$.
  However there are also some important examples where it is partial.
\end{rem}

\begin{defn}
  If $\pi$ is defined at $\lambda
  j \in n : \Src F_j$ and $\lambda j \in n : \Dst F_j$, then
  \[ \bigodot^{\text{proj}}_{\sqcap} F = \bigsqcap_{i \in \dom F} ((\pi^{\Dst \circ
   F_{}}_i)^{\dagger} \circ F_i \circ \pi^{\Src \circ F}_i) . \]
\end{defn}

  \[ \prod^{\text{proj}}_{\sqcap} F = \bigsqcap_{i \in \dom F} ((\pi^{\Dst\circ F}_i)^{\dagger} \circ F_i) . \]

\begin{rem}
  \[ (\pi^{\Dst \circ
   F_{}}_i)^{\dagger} \circ F_i \circ \pi^{\Src \circ F}_i \in \Hom \left(
  \prod^{(Q)}_{j \in n} \Src F_j , \prod^{(Q)}_{j \in n} \Dst F_j
  \right) \] are properly defined and have the same sources and destination
  (whenever $i \in \dom F$ is), thus the meet in the formulas is
  properly defined.
\end{rem}

\begin{rem}
  Thus
\begin{multline*}
  F_0 \odot^{\text{proj}}_{\sqcap} F_1 = ((\pi^{(\Dst F_0 , \Dst
  F_1)}_0)^{\dagger} \circ F_0 \circ \pi^{(\Src F_0 , \Src
  F_1)}_0) \sqcap\\ ((\pi^{(\Dst F_0 , \Dst F_1)}_1)^{\dagger}
  \circ F_1 \circ \pi^{(\Src F_0 , \Src F_1)}_1)
\end{multline*}
  that is product is defined by a pure algebraic formula.
\end{rem}

\begin{prop}
  $\bigodot^{\text{proj}}_{\sqcap} F = \max \setcond{ \Phi \in \Hom \left( \prod^{(Q)}_{j \in
  n} \Src F_j , \prod^{(Q)}_{j \in n} \Dst F_j \right)
  }{ \forall i \in n : \Phi \sqsubseteq (\pi^{\lambda
  j \in n : \Dst F_j}_i)^{\dagger} \circ F_i \circ \pi^{\lambda j \in n
  : \Src F_j}_i }$.
\end{prop}

\begin{proof}
  By definition of meet on a complete lattice.
\end{proof}

\begin{cor}
  $\bigodot^{\text{proj}}_{\sqcap} F = \bigsqcup \setcond{ \Phi \in \Hom \left( \prod^{(Q)}_{j
  \in n} \Src F_j , \prod^{(Q)}_{j \in n} \Dst F_j \right)
  }{ \forall i \in n : \Phi \sqsubseteq (\pi^{\lambda
  j \in n : \Dst F_j}_i)^{\dagger} \circ F_i \circ \pi^{\lambda j \in n
  : \Src F_j}_i }$.
\end{cor}

\begin{thm}
  Let $\pi^X_i$ be metamonovalued morphisms. If $S \in \subsets (\Hom
  (A_0 , B_0) \times \Hom (A_1 , B_1))$ for some sets $A_0$, $B_0$,
  $A_1$, $B_1$ then
  \[ \bigsqcap \setcond{ a \times^{(L)} b }{(a , b)\in S } =
     \bigsqcap \dom S \times^{(L)} \bigsqcap \im S. \]
\end{thm}

\begin{proof}
  ~
  \begin{align*}
  \bigsqcap \setcond{ a \times b }{ (a , b) \in S } & = \\
  \bigsqcap \setcond{ ((\pi^{(\Dst a , \Dst
  b)}_0)^{\dagger} \circ a \circ \pi^{(\Src a , \Src b)}_0) \sqcap
  ((\pi^{(\Dst a , \Dst b)}_1)^{\dagger} \circ b \circ
  \pi^{(\Src a , \Src b)}_1) }{ (a , b) \in S } & = \\
  \bigsqcap \setcond{ (\pi^{(\Dst a , \Dst
  b)}_0)^{\dagger} \circ a \circ \pi^{(\Src a , \Src b)}_0
  }{ a \in \dom S } \sqcap
  \bigsqcap
  \setcond{ (\pi^{(\Dst a , \Dst b)}_1)^{\dagger} \circ b \circ
  \pi^{(\Src a , \Src b)}_1 }{ b \in \im S } & = \\
  \left( (\pi^{(\Dst a , \Dst
  b)}_0)^{\dagger} \circ \bigsqcap \setcond{ a }{ a \in
  \dom S } \circ \pi^{(\Src a , \Src b)}_0 \right)
  \sqcap\\ \left( (\pi^{(\Dst a , \Dst b)}_1)^{\dagger} \circ
  \bigsqcap \setcond{ b }{ b \in \im S }
  \circ \pi^{(\Src a , \Src b)}_1 \right) & = \\
  \left(
  (\pi^{(\Dst a , \Dst b)}_0)^{\dagger} \circ \left( \bigsqcap
  \dom S \right) \circ \pi^{(\Src a , \Src b)}_0 \right)
  \sqcap\\ \left( (\pi^{(\Dst a , \Dst b)}_1)^{\dagger} \circ \left(
  \bigsqcap \im S \right) \circ \pi^{(\Src a , \Src b)}_1
  \right) & = \\
  \bigsqcap \dom S \times \bigsqcap \im S.
  \end{align*}
\end{proof}

\begin{cor}
  $(a_0 \times^{(L)} b_0) \sqcap (a_1 \times^{(L)} b_1) = (a_0 \sqcap a_1)
  \times^{(L)} (b_0 \sqcap b_1)$.
\end{cor}

\begin{cor}
  $a_0 \times^{(L)} b_0 \nasymp a_1 \times^{(L)} b_1 \Leftrightarrow a_0
  \nasymp a_1 \wedge b_0 \nasymp b_1$.
\end{cor}

\subsection{Infimum product for endomorphisms}

Let $F$ is an indexed family of endomorphisms of $\mathcal{C}$.

I will denote $\Ob f$ the object (source and destination) of an
endomorphism $f$.

Let also $\pi^X_i$ be a monovalued entirely defined morphism (for each $i \in
\dom F$).

Then $\bigodot^{\text{proj}}_{\sqcap} F = \bigsqcap_{i \in \dom F} ((\pi^{\lambda j \in n :
\Ob F_j}_i)^{\dagger} \circ F_i \circ \pi^{\lambda j \in n : \Ob
F_j}_i)$ (if $\pi$ is defined at $\lambda j \in n : \Ob F_j$).

Abbreviate $\pi_i = \pi^{\lambda j \in n : \Ob F_j}_i$.

So $\bigodot^{\text{proj}}_{\sqcap} F = \bigsqcap_{i \in \dom F} ((\pi_i)^{\dagger} \circ
F_i \circ \pi_i)$.

$\bigodot^{\text{proj}}_{\sqcap} F = \max \setcond{ \Phi \in \End \left( \prod^{(Q)}_{j \in n}
\Ob F_j \right) }{ \forall i \in n : \Phi
\sqsubseteq (\pi_i)^{\dagger} \circ F_i \circ \pi_i }$.

Taking into account that $\pi_i$ is a monovalued entirely defined morphism, we
get:

\begin{obvious}
$\bigodot^{\text{proj}}_{\sqcap} F = \max \setcond{ \Phi \in \End \left( \prod^{(Q)}_{j \in
n} \Ob F_j \right) }{ \forall i \in n : \pi_i
\in \mathrm{C} (\Phi , F_i) }$.
\end{obvious}

\begin{rem}
  The above formula may allow to define the product for non-dagger categories
  (but only for endomorphisms). In this writing I don't introduce a notation
  for this, however.
\end{rem}

\begin{cor}
  $\pi_i \in \mathrm{C} \left( \bigodot^{\text{proj}}_{\sqcap} F , F_i \right)$ for every $i \in
  \dom F$.
\end{cor}

\subsection{Category of continuous morphisms}

Let $\pi_i = \pi_i^X$ (for $i \in \dom F$) be entirely defined
monovalued morphisms (we suppose it is defined at $X$).

Let $\bigotimes$ of an indexed family of morphisms is a morphism; $\pi_i \circ
\bigotimes f = f_i$; $\bigotimes_{i \in n} (\pi_i \circ f) = f$.

\begin{defn}
  The category $\cont (\mathcal{C})$ is defined as follows:
  \begin{itemize}
    \item Objects are endomorphisms of the category $\mathcal{C}$.
    
    \item Morphisms are triples $(f , a , b)$ where $a$ and $b$ are objects
    and $f : \Ob a \rightarrow \Ob b$ is an entirely defined
    monovalue principal morphism of the category $\mathcal{C}$ such that $f
    \in \mathrm{C} (a , b)$ (in other words, $f \circ a \sqsubseteq b \circ
    f$).
    
    \item Composition of morphisms is defined by the formula $(g , b , c)
    \circ (f , a , b) = (g \circ f , a , c)$.
    
    \item Identity morphisms are $(a , a , 1^{\mathcal{C}}_a)$.
  \end{itemize}
\end{defn}

It is really a category:

\begin{proof}
  We need to prove that: composition of morphisms is a morphism, composition
  is associative, and identity morphisms can be canceled on the left and on
  the right.
  
  That composition of morphisms is a morphism by properties of generalized
  continuity.
  
  That composition is associative is obvious.
  
  That identity morphisms can be canceled on the left and on the right is
  obvious.
\end{proof}

\begin{rem}
  The ``physical'' meaning of this category is:
  \begin{itemize}
    \item Objects (endomorphisms of $\mathcal{C}$) are spaces.
    
    \item Morphisms are continuous functions between spaces.
    
    \item $f \circ a \sqsubseteq b \circ f$ intuitively means that $f$
    combined with an infinitely small is less than infinitely small combined
    with $f$ (that is $f$ is continuous).
  \end{itemize}
\end{rem}

\begin{defn}
  $\pi^{\cont (\mathcal{C})}_i = \left( \bigodot^{\text{proj}}_{\sqcap} F , F_i ,
  \pi_i \right)$.
\end{defn}

\begin{prop}
  $\pi_i$ are continuous, that is $\pi^{\cont}
  (\mathcal{C})_i$ are morphisms.
\end{prop}

\begin{proof}
  We need to prove $\pi_i \in \mathrm{C} \left( \bigodot^{\text{proj}}_{\sqcap} F , F_i \right)$
  but that was proved above.
\end{proof}

\begin{lem}
  $f \in \Hom_{\cont (\mathcal{C})} \left( Y ,
  \bigodot^{\text{proj}}_{\sqcap} F \right)$ is continuous iff all $\pi_i \circ f$ are continuous.
\end{lem}

\begin{proof}
  ~
  \begin{description}
    \item[$\Rightarrow$] Let $f \in \Hom_{\cont
    (\mathcal{C})} \left( Y , \bigodot^{\text{proj}}_{\sqcap} F \right)$. Then $f \circ Y
    \sqsubseteq \left( \bigodot^{\text{proj}}_{\sqcap} F \right) \circ f$; $\pi_i \circ f \circ Y
    \sqsubseteq \pi_i \circ \left( \bigodot^{\text{proj}}_{\sqcap} F \right) \circ f$; $\pi_i
    \circ f \circ Y \sqsubseteq \left( \bigodot^{\text{proj}}_{\sqcap} F \right) \circ \pi_i \circ
    f$. Thus $\pi_i \circ f$ is continuous.
    
    \item[$\Leftarrow$] Let all $\pi_i \circ f$ be continuous. Then
    $\pi^{\cont (\mathcal{C})}_i \circ f \in
    \Hom_{\cont (\mathcal{C})} (Y , F_i)$;
    $\pi^{\cont (\mathcal{C})}_i \circ f \circ Y \sqsubseteq
    F_i \circ \pi^{\cont (\mathcal{C})}_i \circ f$. We need
    to prove $Y \sqsubseteq f^{\dagger} \circ \left( \bigodot^{\text{proj}}_{\sqcap} F \right)
    \circ f$ that is
    \[ Y \sqsubseteq f^{\dagger} \circ \bigsqcap_{i \in n} ((\pi_i)^{\dagger}
       \circ F_i \circ \pi_i) \circ f \]
    for what is enough (because $f$ is metamonovalued)
    \[ Y \sqsubseteq \bigsqcap_{i \in n} (f^{\dagger} \circ (\pi_i)^{\dagger}
       \circ F_i \circ \pi_i \circ f) \]
    what follows from $Y \sqsubseteq \bigsqcap_{i \in n} (f^{\dagger} \circ
    (\pi_i)^{\dagger} \circ \pi_i \circ f \circ Y)$ what is obvious.
  \end{description}
\end{proof}

\begin{thm}
  $\bigodot^{\text{proj}}_{\sqcap}$ together with $\bigotimes$ is a (partial) product in the
  category $\cont (\mathcal{C})$.
\end{thm}

\begin{proof}
  Obvious.
  
  Check
  \url{http://math.stackexchange.com/questions/102632/how-to-check-whether-it-is-a-direct-product/102677\#102677}
\end{proof}

\section{On duality}

We will consider duality where both the category $\mathcal{C}$ and orders on
Mor-sets are replaced with their dual. I will denote $A
\xleftrightarrow{\dual} B$ when two formulas $A$ and $B$ are dual with
this duality.

\begin{prop}
  $f \in \mathrm{C} (\mu, \nu) \xleftrightarrow{\dual} f^{\dagger}
  \in \mathrm{C} (\nu^{\dagger} , \mu^{\dagger})$.
\end{prop}

\begin{proof}
  $f \in \mathrm{C} (\mu, \nu) \Leftrightarrow f \circ \mu
  \sqsubseteq \nu \circ f \xleftrightarrow{\dual} \mu^{\dagger}
  \circ f^{\dagger} \sqsupseteq f^{\dagger} \circ^{\dagger} \nu^{- 1}
  \Leftrightarrow f^{\dagger} \in \mathrm{C} (\nu^{\dagger} ,
  \mu^{\dagger})$.
\end{proof}

$f \text{ is entirely defined} \Leftrightarrow f^{\dagger} \circ f \sqsupseteq
1_{\Src f} \xleftrightarrow{\dual} f^{\dagger} \circ f \sqsubseteq
1_{\Src f} \Leftrightarrow f \text{ is injective} \Leftrightarrow
f^{\dagger} \text{ is monovalued}$.

$f \text{ is monovalued} \Leftrightarrow f \circ f^{\dagger} \sqsubseteq
1_{\Dst f} \xleftrightarrow{\dual} f \circ f^{\dagger} \sqsupseteq
1_{\Dst f} \Leftrightarrow f \text{ is surjective} \Leftrightarrow
f^{\dagger} \text{ is entirely defined}$.

\section{General coproduct in partially ordered dagger category}

The below is the dual of the above, proofs are omitted as they are dual.

Let $\iota_i$ \fxnote{What is $\iota$?} are entirely defined monovalued morphisms to an object $Z$.

Let $\iota_i \xleftrightarrow{\dual} \pi_i$ that is $\iota_i =
(\pi_i)^{\dagger}$. We have the above equivalent to $\pi_i$ being monovalued
and entirely defined.

\subsection{Supremum coproduct}

Let $\mathcal{C}$ be a dagger category, each Hom-set of which is a complete
lattice (having order agreed with the dagger).

We will designate some morphisms as \emph{principal} and require that
principal morphisms are both metacomplete and co-metacomplete. (For a
particular example of the category $\mathbf{Rel}$, all morphisms are
considered principal.)

Let $\coprod^{(Q)} X$ be an object for each indexed family $X$ of objects.

Let $\iota$ be a partial function mapping elements $X \in \dom \iota$
(which consists of small indexed families of objects of $\mathcal{C}$) to
indexed families $X_i \rightarrow \coprod^{(Q)} X$ of principal morphisms
(called \emph{injections}) for every $i \in \dom X$.

\begin{defn}
  \emph{Supremum coproduct} $\coprod^{(L)} F$ (such that $\iota$ is defined
  at $\lambda j \in n : \Dst F_j$ and $\lambda j \in n : \Src
  F_j$) is defined by the formula
  \[ \coprod^{(L)} F = \bigsqcup_{i \in \dom F} (\iota^{\lambda j \in n
     : \Src F_j}_i \circ F_i^{\dagger} \circ (\iota^{\lambda j \in n :
     \Dst F_j}_i)^{\dagger}) . \]
\end{defn}

This formula can be (over)simplified to:
\[ \coprod^{(L)} F = \bigsqcup_{i \in \dom F} (\iota^{\Src \circ
   F}_i \circ F_i^{\dagger} \circ (\iota^{\Dst \circ F}_i)^{\dagger}) .
\]

\begin{rem}
  $\iota^{\lambda j \in n : \Src F_j}_i \circ F_i \circ (\iota^{\lambda
  j \in n : \Dst F_j}_i)^{\dagger} \in \Hom \left(
  \coprod^{(Q)}_{j \in n} \Src F_j , \coprod^{(Q)}_{j \in n} \Dst
  F_j \right)$ are properly defined and have the same sources and destination
  (whenever $i \in \dom F$ is), thus the meet in the formulas is
  properly defined.
\end{rem}

\begin{rem}
  Thus
  \begin{multline*}
     F_0 \amalg^{(L)} F_1 = (\iota^{(\Src F_0 , \Src F_1)}_0 \circ
     F_0^{\dagger} \circ (\iota^{(\Dst F_0 , \Dst
     F_1)}_0)^{\dagger}) \sqcup\\ (\iota^{(\Src F_0 , \Src F_1)}_1
     \circ F_1^{\dagger} \circ (\iota^{(\Dst F_0 , \Dst
     F_1)}_1)^{\dagger})
  \end{multline*}
  that is coproduct is defined by a pure algebraic formula.
\end{rem}

\begin{prop}
  $\coprod^{(L)} F = \min \setcond{ \Phi \in \End \left( \coprod^{(Q)}_{j
  \in n} \Ob F_j \right) }{ \forall i \in n :
  \Phi \sqsupseteq \iota^{\lambda j \in n : \Src F_j}_i \circ
  F_i^{\dagger} \circ (\iota^{\lambda j \in n : \Dst F_j}_i)^{\dagger} }$.
\end{prop}

\begin{proof}
  By definition of meet on a complete lattice.
\end{proof}

\begin{cor}
  $\coprod^{(L)} F = \bigsqcap \setcond{ \Phi \in \End \left(
  \coprod^{(Q)}_{j \in n} \Ob F_j \right) }{
  \forall i \in n : \Phi \sqsupseteq \iota^{\lambda j \in n : \Src
  F_j}_i \circ F_i^{\dagger} \circ (\iota^{\lambda j \in n : \Dst
  F_j}_i)^{\dagger} }$.
\end{cor}

\begin{thm}
  Let $\pi^X_i$ be metainjective morphisms. If $S \in \subsets (\Hom
  (A_0 , B_0) \times \Hom (A_1 , B_1))$ for some sets $A_0$, $B_0$,
  $A_1$, $B_1$ then
  \[ \bigsqcup \setcond{ a \times^{(L)} b }{ (a , b)
     \in S } = \bigsqcup \dom S \times^{(L)} \bigsqcup \im
     S. \]
\end{thm}

\begin{cor}
  $(a_0 \amalg^{(L)} b_0) \sqcup (a_1 \amalg^{(L)} b_1) = (a_0 \sqcap a_1)
  \amalg^{(L)} (b_0 \sqcap b_1)$.
\end{cor}

\begin{cor}
  $a_0 \amalg^{(L)} b_0 \equiv a_1 \amalg^{(L)} b_1 \Leftrightarrow a_0 \equiv
  a_1 \wedge b_0 \equiv b_1$.
\end{cor}

\subsection{Supremum coproduct for endomorphisms}

Let $F$ be an indexed family of endomorphisms of $\mathcal{C}$.

I will denote $\Ob f$ the object (source and destination) of an
endomorphism $f$.

Let also $\iota_i$ be a monovalued entirely defined morphism (for each $i \in
\dom F$).

\begin{defn}
  $\coprod^{(L)} F = \bigsqcup_{i \in \dom F} (\iota^{\lambda j \in n :
  \Ob F_j}_i \circ F_i^{\dagger} \circ (\iota^{\lambda j \in n :
  \Ob F_j}_i)^{\dagger})$ (if $\iota$ is defined at $\lambda j \in n :
  \Ob F_j$). (I call it \emph{supremum coproduct}).
\end{defn}

Abbreviate $\iota_i = \iota^{\lambda j \in n : \Ob F_j}_i$.

So $\coprod F = \bigsqcup_{i \in \dom F} (\iota_i \circ F_i^{\dagger}
\circ (\iota_i)^{\dagger})$.

$\coprod F = \min \setcond{ \Phi \in \End \left( \coprod^{(Q)}_{j \in n}
\Ob F_j \right) }{ \forall i \in n : \Phi
\sqsupseteq \iota_i \circ F_i^{\dagger} \circ (\iota_i)^{\dagger} }$.

Taking into account that $\iota_i$ is a monovalued entirely defined morphism,
we get:

\begin{obvious}
$\coprod^{(L)} = \min \setcond{ \Phi \in \End \left( \coprod^{(Q)}_{j \in
n} \Ob F_j \right) }{ \forall i \in n : \iota_i
\in \mathrm{C} (F_i^{\dagger} , \Phi) }$.{\hspace*{\fill}}{\medskip}
\end{obvious}

\begin{cor}
  $\iota_i \in \mathrm{C} \left( F_i , \coprod^{(L)} F \right)$ for every $i
  \in \dom F$.
\end{cor}

\subsection{Category of continuous morphisms}

Let $\iota_i$ (for $i \in \dom F$) be entirely defined monovalued and
metacomplete morphisms.

Let $\bigoplus$ of an indexed family of morphisms is a morphism; $\left(
\bigoplus f \right) \circ \iota_i = f_i$; $\bigoplus_{i \in n} (f \circ
\iota_i) = f$ (a dual of the above).

Let $F_i \in \End \left( \coprod^{(Q)}_{j \in n} \Ob F_j \right)$
for all $i \in n$ (where $n$ is some index set) (a self-dual of the above).

\begin{defn}
  $\iota^{\cont (\mathcal{C})}_i = \left( \coprod^{(L)} F ,
  F_i^{\dagger} , \iota_i \right)$.
\end{defn}

\begin{prop}
  $\iota_i$ are continuous, that is $\iota^{\cont
  (\mathcal{C})}_i$ are morphisms.
\end{prop}

\begin{lem}
  $f \in \Hom_{\cont (\mathcal{C})} \left(
  \coprod^{(L)} F , Y \right)$ \fxwarning{What is $Y$?} is continuous iff all $f \circ
  \iota^{\cont (\mathcal{C})}$ are continuous.
\end{lem}

\begin{thm}
  $\coprod^{(L)}$ together with $\bigoplus$ is a (partial) coproduct in the
  category $\cont (\mathcal{C})$.
\end{thm}

\section{Applying this to the theory of funcoids and reloids}

\subsection{Funcoids}

\begin{defn}
  $\mathbf{Fcd} \eqdef \cont \mathsf{FCD}$.
\end{defn}

Let $F$ be a family of endofuncoids.

The cartesian product $\prod^{(Q)} X \eqdef \prod X$.

I define $\pi_i = \pi^X_i \in \mathsf{FCD} \left( \prod X , X_i
\right)$ as the principal funcoid corresponding to the $i$-th projection.
(Here $\pi$ is entirely defined.)

The disjoint union $\coprod^{(Q)} X \eqdef \coprod X$.

I define $\iota_i = \iota^X_i \in \mathsf{FCD} \left( X_i , \coprod X
\right)$ as the principal funcoid corresponding to the $i$-th canonical
injection. (Here $\iota$ is entirely defined.)

Let $\bigotimes$ and $\bigoplus$ be defined in the same way as in category
$\mathbf{Set}$.

\begin{obvious}
$\pi_i \circ \bigotimes f = f_i$; $\bigotimes_{i \in n} (\pi_i \circ f) =
f$.
\end{obvious}

\begin{obvious}
$\left( \bigoplus f \right) \circ \iota_i = f_i$; $\bigoplus_{i \in n} (f
\circ \iota_i) = f$.
\end{obvious}

It is easy to show that $\pi_i$ is entirely defined monovalued, and $\iota_i$
is metacomplete and co-metacomplete.

Thus we are under conditions for both canonical products and canonical
coproducts and thus both $\bigodot^{\text{proj}}_{\sqcap} F$ and $\coprod^{(L)} F$ are defined.

\subsection{Reloids}

\begin{defn}
  $\mathbf{Rld} \eqdef \cont \mathsf{RLD}$.
\end{defn}

Let $F$ be a family of endoreloids.

The cartesian product $\prod^{(Q)} X \eqdef \prod X$.

I define $\pi_i = \pi^X_i \in \mathsf{RLD} \left( \prod X , X_i
\right)$ as the principal reloid corresponding to the $i$-th projection. (Here
$\pi$ is entirely defined.)

The disjoint union $\coprod^{(Q)} X \eqdef \coprod X$.

I define $\iota_i = \iota^X \in
\mathsf{RLD} \left( X_i , \coprod X \right)$ as the principal reloid
corresponding to the $i$-th canonical injection. (Here $\iota$ is entirely
defined.)

Let $\bigotimes$ and $\bigoplus$ are defined in the same way as in category
$\mathbf{Set}$.

\begin{obvious}
$\pi_i \circ \bigotimes f = f_i$; $\bigotimes_{i \in n} (\pi_i \circ f) =
f$.
\end{obvious}

\begin{obvious}
$\left( \bigoplus f \right) \circ \iota_i = f_i$; $\bigoplus_{i \in n} (f
\circ \iota_i) = f$.
\end{obvious}

It is easy to show that $\pi_i$ is entirely defined monovalued, and $\iota_i$
is metacomplete and co-metacomplete.

Thus we are under conditions for both canonical products and canonical
coproducts and thus both $\bigodot^{\text{proj}}_{\sqcap} F$ and $\coprod^{(L)} F$ are defined.

It is trivial that for uniform spaces infimum product of reloids coincides
with product uniformilty.

\section{Initial and terminal objects}

Initial object of $\mathbf{Fcd}$ is the endofuncoid
$\uparrow^{\mathsf{FCD} (\emptyset , \emptyset)} \emptyset$. It is
initial because it has precisely one morphism $o$ (the empty set considered as
a function) to any object $Y$. $o$ is a morphism because $o \circ
\uparrow^{\mathsf{FCD} (\emptyset , \emptyset)} \emptyset \sqsubseteq Y
\circ o$.

\begin{prop}
  Terminal objects of $\mathbf{Fcd}$ are exactly
  $\uparrow^{\mathscr{F}} \{ \ast \} \times^{\mathsf{FCD}}
  \uparrow^{\mathscr{F}} \{ \ast \} = \uparrow^{\mathsf{FCD}} \{ (\ast
  , \ast) \}$ where $\ast$ is an arbitrary point.
\end{prop}

\begin{proof}
  In order for a function $f : X \rightarrow \uparrow^{\mathsf{FCD}} \{
  (\ast , \ast) \}$ be a morphism, it is required exactly $f \circ X
  \sqsubseteq \uparrow^{\mathsf{FCD}} \{ (\ast , \ast) \} \circ f$
  
  $f \circ X \sqsubseteq (f^{- 1} \circ \uparrow^{\mathsf{FCD}} \{
  (\ast , \ast) \})^{- 1}$; $f \circ X \sqsubseteq (\{ \ast \}
  \times^{\mathsf{FCD}} \langle f^{- 1} \rangle \{ \ast \})^{- 1}$; $f
  \circ X \sqsubseteq \langle f^{- 1} \rangle \{ \ast \}
  \times^{\mathsf{FCD}} \{ \ast \}$ what true exactly when $f$ is a
  constant function with the value $\ast$.
\end{proof}

If $n = \emptyset$ then $Z = \{ \emptyset \}$; $\bigodot^{\text{proj}}_{\sqcap} \emptyset = \max
\mathsf{FCD} (Z , Z) = \uparrow^{\mathscr{F}} \{ \emptyset \}
\times^{\mathsf{FCD}} \uparrow^{\mathscr{F}} \{ \emptyset \} =
\uparrow^{\mathsf{FCD}} \{ (\emptyset , \emptyset) \}$.

\fxnote{Initial and terminal objects of $\mathbf{Rld}$.}

\section{Canonical product and subatomic product}

\fxwarning{Confusion between filters on products and multireloids.}

\begin{prop}
  $\Pr^{\mathsf{RLD}}_i |_{\mathfrak{F} (Z)} = \langle \pi_i \rangle$
  for every index $i$ of a cartesian product $Z$.
\end{prop}

\begin{proof}
  If $\mathcal{X} \in \mathfrak{F} (Z)$ then $(\Pr^{\mathsf{RLD}}_i
  |_{\mathfrak{F} (Z)}) \mathcal{X} = \Pr^{\mathsf{RLD}}_i  \mathcal{X}
  = \bigsqcap^{\mathscr{F}} \rsupfun{\Pr_i} \mathcal{X} =
  \bigsqcap \langle \pi_i \rangle \up \mathcal{X} = \langle \pi_i
  \rangle \mathcal{X}$.
\end{proof}

\begin{prop}
  $\prod^{(A)} F = \bigsqcap_{i \in n} \left( \left( \pi^{\mathsf{FCD}
  \left( \prod_{i \in n} \Dst F \right)}_i \right)^{- 1} \circ F_i \circ
  \pi^{\mathsf{FCD} \left( \prod_{i \in n} \Src F \right)}_i
  \right)$.
\end{prop}

\begin{proof}
  $a \mathrel{\left[ \prod^{(A)} F \right]} b \Leftrightarrow \forall i \in
  \dom F : \Pr^{\mathsf{RLD}}_i a \mathrel{[F_i]}
  \Pr^{\mathsf{RLD}}_i b \Leftrightarrow \forall i \in \dom F :
  \left\langle \left( \pi^{\mathsf{FCD} \left( \prod_{i \in n}
  \Dst F \right)}_i \right)^{- 1} \right\rangle \mathrel{[F_i]}
  \left\langle \pi^{\mathsf{FCD} \left( \prod_{i \in n} \Src F
  \right)}_i \right\rangle \Leftrightarrow \forall i \in \dom F : a
  \mathrel{\left[ \left( \pi^{\mathsf{FCD} \left( \prod_{i \in n}
  \Dst F \right)}_i \right)^{- 1} \circ F_i \circ
  \pi^{\mathsf{FCD} \left( \prod_{i \in n} \Src F \right)}_i
  \right]} b \Leftrightarrow a \mathrel{\left[ \bigsqcap_{i \in n} \left(
  \left( \pi^{\mathsf{FCD} \left( \prod_{i \in n} \Dst F
  \right)}_i \right)^{- 1} \circ F_i \circ \pi^{\mathsf{FCD} \left(
  \prod_{i \in n} \Src F \right)}_i \right) \right]} b$ for ultrafilters
  $a$ and $b$.
\end{proof}

\begin{cor}
  $\bigodot^{\text{proj}}_{\sqcap} F = \prod^{(A)} F$ is $F$ is a small indexed family of
  funcoids.
\end{cor}

\section{Further plans}

Does the formula ${\bigodot^{\text{proj}}_{\sqcap}}_{i \in n} (g_i \circ f_i) = \bigodot^{\text{proj}}_{\sqcap} g \circ
\bigodot^{\text{proj}}_{\sqcap} f$ hold?

Coordinate-wise continuity.

\section{Cartesian closedness}

We are not only to prove (or maybe disprove) that our categories are cartesian closed, but also to find (if any) explicit formulas for exponential transpose and evaluation.

''Definition'' A category is //cartesian closed// iff:
\begin{enumerate}
\item It has finite products.
\item For each objects $A$, $B$ is given an object $\operatorname{MOR} ( A , B)$ (//exponentiation//) and a morphism $\varepsilon^{\mathbf{Dig}}_{A, B} : \operatorname{MOR} ( A , B) \times A \rightarrow B$.
\item For each morphism $f : Z \times A \rightarrow B$ there is given a morphism (//exponential transpose//) $\sim f : Z \rightarrow \operatorname{MOR} ( A , B)$.
\item $\varepsilon_{B,C} \circ ( \sim f \times 1_A) = f$ for $f : A \rightarrow B \times C$.
\item $\sim ( \varepsilon_{B,C} \circ ( g \times 1_A)) = g$ for $g : A \rightarrow \operatorname{MOR} ( B , C)$.
\end{enumerate}

We will also denote $f\mapsto (-f)$ the reverse of the bijection $f\mapsto (\sim f)$.

Our purpose is to prove (or disprove) that categories $\mathbf{Dig}$, $\mathbf{Fcd}$, and $\mathbf{Rld}$ are cartesian closed. Note that they have finite (and even infinite) products is already proved.

Alternative way to prove:
you can prove that the functor $-\times B$ is left adjoint to the exponentiation $-^B$ where the counit is given by the evaluation map.

\subsection{Definitions}

Categories $\mathbf{Dig}$, $\mathbf{Fcd}$, and $\mathbf{Rld}$ are respectively categories of:
\begin{enumerate}
\item discretely continuous maps between digraphs;
\item (proximally) continuous maps between endofuncoids;
\item (uniformly) continuous maps between endoreloids.
\end{enumerate}

''Definition'' //Digraph// is an endomorphism of the category $\mathbf{Rel}$.

For a digraph $A$ we denote $\operatorname{Ob} A$ the set of vertexes or $A$ and $\operatorname{GR} A$ the set of edges or $A$.

''Definition'' Category $\mathbf{Dig}$ of digraphs is the category whose objects are digraphs and morphisms are discretely continuous maps between digraphs. That is morphisms from a digraph $\mu$ to a digraph $\nu$ are functions (or more precisely morphisms of $\mathbf{Set}$) $f$ such that $f \circ \mu \sqsubseteq \nu \circ f$ (or equivalently $\mu \sqsubseteq f^{- 1} \circ \nu \circ f$ or equivalently $f \circ \mu \circ f^{- 1} \sqsubseteq \nu$).

''Remark'' Category of digraphs is sometimes defined in an other (non equivalent) way, allowing multiple edges between two given vertices.

\subsection{Conjectures}

\begin{conjecture}
  The categories $\mathbf{Fcd}$ and $\mathbf{Rld}$ are
  cartesian closed (actually two conjectures).
\end{conjecture}

\url{http://mathoverflow.net/questions/141615/how-to-prove-that-there-are-no-exponential-object-in-a-category}
suggests to investigate colimits to prove that there are no exponential
object.

Our purpose is to prove (or disprove) that categories $\mathbf{Dig}$, $\mathbf{Fcd}$, and $\mathbf{Rld}$ are cartesian closed. Note that they have finite (and even infinite) products is already proved.

Alternative way to prove:
you can prove that the functor $-\times B$ is left adjoint to the exponentiation $-^B$ where the counit is given by the evaluation map.

See \url{http://www.springer.com/us/book/9780387977102} for another way to prove Cartesian closedness.

\subsection{Category Dig is cartesian closed}

Category of digraphs is the simplest of our three categories and it is easy to demonstrate that it is cartesian closed. I demonstrate cartesian closedness of $\mathbf{Dig}$ mainly with the purpose to show a pattern similarly to which we may probably demonstrate our two other categories are cartesian closed.

Let $G$ and $H$ be graphs:
\begin{itemize}
\item $\operatorname{Ob} \operatorname{MOR} ( G , H) = ( \operatorname{Ob} H)^{\operatorname{Ob} G}$;
\item $( f , g) \in \operatorname{GR} \operatorname{MOR} ( G , H) \Leftrightarrow \forall ( v , w) \in \operatorname{GR} G : ( f ( v) , g ( w)) \in \operatorname{GR} H$ for every $f, g \in \operatorname{Ob} \operatorname{MOR} ( G , H) = ( \operatorname{Ob} H)^{\operatorname{Ob} G}$;
\end{itemize}

$\operatorname{GR} 1_{\operatorname{MOR} ( B , C)} = \operatorname{id}_{\operatorname{Ob} \operatorname{MOR} ( B , C)} = \operatorname{id}_{( \operatorname{Ob} H)^{\operatorname{Ob} G}}$

Equivalently

$( f , g) \in \operatorname{GR} \operatorname{MOR} ( G , H) \Leftrightarrow \forall ( v , w) \in \operatorname{GR} G : g \circ \{ ( v , w) \} \circ f^{- 1} \subseteq \operatorname{GR} H$

$( f , g) \in \operatorname{GR} \operatorname{MOR} ( G , H) \Leftrightarrow g \circ ( \operatorname{GR} G) \circ f^{- 1} \subseteq \operatorname{GR} H$

$( f , g) \in \operatorname{GR} \operatorname{MOR} ( G , H) \Leftrightarrow \langle f \times^{( C)} g \rangle \operatorname{GR} G \subseteq \operatorname{GR} H$

The transposition (the isomorphism) is uncurrying.

$\sim f = \lambda a \in Z \lambda y \in A : f ( a , y)$ that is $( \sim f) ( a) ( y) = f ( a , y)$.

$( - f) ( a , y) = f ( a) ( y)$

If $f : A \times B \rightarrow C$ then $\sim f : A \rightarrow \operatorname{MOR} ( B , C)$

''Proposition'' Transposition and its inverse are morphisms of $\mathbf{Dig}$.

''Proof'' It follows from the equivalence $\sim f : A \rightarrow \operatorname{MOR} ( B , C) \Leftrightarrow \forall x, y : ( x A y \Rightarrow ( \sim f) x ( \operatorname{MOR} ( B , C))  ( \sim f) y) \Leftrightarrow \\ \forall x, y : ( x A y \Rightarrow \forall ( v , w) \in B : ( ( \sim f) x v , ( \sim f) y w) \in C) \Leftrightarrow \\ \forall x, y, v, w : ( x A y \wedge v B w \Rightarrow ( ( \sim f) x v , ( \sim f) y w) \in C) \Leftrightarrow \\ \forall x, y, v, w : ( ( x , v)  ( A \times B)  ( y , w) \Rightarrow ( f ( x , v) , f ( y , w)) \in C) \Leftrightarrow f : A \times B \rightarrow C$.

Evaluation $\varepsilon : \operatorname{MOR} ( G , H) \times G \rightarrow H$ is defined by the formula:

Then evaluation is $\varepsilon_{B, C} = - ( 1_{\operatorname{MOR} ( B , C)})$.

So $\varepsilon_{B, C} ( p , q) = ( - ( 1_{\operatorname{MOR} ( B , C)})) ( p , q) = ( 1_{\operatorname{MOR} ( B , C)}) ( p) ( q) = p ( q)$.

''Proposition'' Evaluation is a morphism of $\mathbf{Dig}$.

''Proof'' Because $\varepsilon_{B, C} ( p , q) = - ( 1_{\operatorname{MOR} ( B , C)})$.

It remains to prove:
* $\varepsilon_{B, C} \circ ( \sim f \times 1_{A}) = f$ for $f : A \rightarrow B \times C$;
* $\sim ( \varepsilon_{B, C} \circ ( g \times 1_{A})) = g$ for $g : A \rightarrow \operatorname{MOR} ( B , C)$.

''Proof'' $\varepsilon_{B, C} ( \sim f \times 1_{A}) ( a , p) = \varepsilon_{B, C} ( ( \sim f) a , p) = ( \sim f) a p = f ( a , p)$. So $\varepsilon_{B, C} \circ ( \sim f \times 1_{A}) = f$.

  $\sim ( \varepsilon_{B, C} \circ ( g \times 1_{A})) ( p) ( q) = ( \varepsilon_{B, C} \circ ( g \times 1_{A})) ( p , q) = \varepsilon_{B, C} ( g \times 1_{A}) ( p , q) = \varepsilon_{B, C} ( g p , q) = g ( p) ( q)$. So $\sim ( \varepsilon_{B, C} \circ ( g \times 1_{A})) = g$.

\subsection{By analogy with the proof that Dig is cartesian closed}

The most obvious way for proof attempt that $\mathbf{Fcd}$ is cartesian closed is an analogy with the proof that
$\mathbf{Dig}$ is cartesian closed.

Consider the long formula above. The proof would arise if we replace $x$ and $y$ in this formula with filters and operations and relations on set element with operations and relations on filters.

This proof could be simplified in either of two ways:
\begin{itemize}
\item replace $x$ and $y$ with ultrafilters, see [[Proof for Fcd using ultrafilters]];
\item replace $x$ and $y$ with sets (principal filter), see [[Proof for Fcd using sets]].
\end{itemize}

This is not quite easy however, because we need to calculate uncurrying for a entirely defined monovalued principal funcoid (what is essentially the same as a function of a $\mathbf{Set}$-morphisms) taking either ultrafilters or principal filters as arguments. Such (generalized) uncurrying is not quite easy.

To sum what we need to prove:
\begin{itemize}
\item Transposition is a morphism.
\item Evaluation is a morphism.
\item $\varepsilon_{B,C} \circ ( \sim f \times 1_A) = f$ for $f : A \rightarrow B \times C$.
\item $\sim ( \varepsilon_{B,C} \circ ( g \times 1_A)) = g$ for $g : A \rightarrow \operatorname{MOR} ( B , C)$.
\end{itemize}

\subsection{Attempt to describe exponentials in Fcd}

\begin{itemize}
\item Exponential object $\operatorname{HOM}(A,B)$ is the following endofuncoid:
\item\begin{itemize}
\item Object $\operatorname{Ob}\operatorname{HOM}(A,B) = (\operatorname{Ob} B)^{\operatorname{Ob} A}$;
\item Graph is $\operatorname{GR} \operatorname{HOM} ( A , B) = \uparrow^{\mathsf{FCD}} \setcond{ ( f , g) }{ f, g \in \Hom_{\mathbf{Set}} (\operatorname{Ob} A,\operatorname{Ob} B) \wedge \uparrow^{\mathsf{FCD}}g \circ A \circ \uparrow^{\mathsf{FCD}}f^{- 1} \sqsubseteq B }$.
\end{itemize}
\item Transposition is uncurrying.
\item Evaluation is $\varepsilon_{A, B} x = \langle \operatorname{Pr}^{(A)}_0 x \rangle \operatorname{Pr}^{(A)}_1 x$.
\end{itemize}

We need to prove that the above defined are really an exponential and an evaluation.

Possible ways to prove that $\mathbf{Fcd}$ is cartesian closed follow:

NEW IDEA: Prove $\GR\operatorname{HOM}(A,B) =
\uparrow^{\mathsf{FCD}}\setcond{(\Pr^{(A)}_0 p,\Pr^{(A)}_1 p)}{
p\in??\land\supfun{p}A\sqsubseteq B}$ (what's about other
kinds of projections?)

\subsection{Proof for Fcd using sets}

Currying for sets is $\langle f \rangle ( X \times Y) = \bigcup \langle \langle \sim f \rangle X
\rangle Y$ (as it's easy to prove). This simple formula gives hope, but...

It does not work with sets because an analogy for sets of the last equality of the above mentioned long formula would be:

$\forall X, Y, V, W \in \mathscr{P} \operatorname{Ob} A : \left( X \times V \mathrel{[
A \times B]^{\ast}} Y \times W \Rightarrow \langle f \rangle ( X \times V)
\mathrel{[ C]^{\ast}} \langle f \rangle ( Y \times W) \right) \Rightarrow \\ f : A
\times B \rightarrow C$

but this implication seems false.

The most obvious way for proof attempt that $\mathbf{Fcd}$ is cartesian closed is an analogy with the proof that Dig is cartesian closed.

Use the exponential object, transposition, and evaluation as defined in [[this page|Is category Fcd cartesian closed?]]

\subsection{Reducing to the fact that Dig is cartesian closed}
It is probably a simpler way to prove that $\mathbf{Fcd}$ is cartesian closed by embedding it into $\mathbf{Dig}$ (which is [[already known to be cartesian closed|Category Dig is cartesian closed]]).

$\mathbf{Fcd}$ can be embedded into $\mathbf{Dig}$ by the formulas:
\begin{itemize}
\item $A \mapsto \langle A \rangle$;
\item $f \mapsto \langle f \rangle$.
\end{itemize}

That this really maps a morphism of $\mathbf{Fcd}$ into a morphism of $\mathbf{Dig}$ follows from the fact that $\langle g\circ f\rangle = \langle g\rangle\circ\langle f\rangle$.

Obviously this embedding (denote it $T$) is an injective (both on objects and morphisms) functor.

We will define:
\begin{itemize}
\item $\varepsilon^{\mathbf{Fcd}}_{A, B} = T^{-1} \varepsilon^{\mathbf{Dig}}_{T A, T B}$;
\item $\sim^{\mathbf{Fcd}} f = T^{-1} \sim^{\mathbf{Dig}} T f$.
\end{itemize}

Due to functoriality and injectivity of $T$ it is enough to prove that above defined $\varepsilon^{\mathbf{Fcd}}_{A, B}$ and $\sim^{\mathbf{Fcd}} f$ exist and are morphisms of $\mathbf{Fcd}$.

$\varepsilon^{\mathbf{Dig}}_{T A, T B} \ne T\varepsilon^{\mathbf{Fcd}}_{A, B}$ because $\varepsilon^{\mathbf{Dig}}_{T A, T B}$ accepts ordered pairs as the argument and $T \varepsilon^{\mathbf{Fcd}}_{A, B}$ accepts sets as the argument. So this is a dead end. Can the proof idea be salvaged?

\section{Is category Rld cartesian closed?}

We may attempt to prove that $\mathbf{Rld}$ is cartesian closed by embedding it into supposedly cartesian closed category $\mathbf{Fcd}$ by the function $\rho$:

$\langle \rho f \rangle x = f \circ x \quad \text{and} \quad \langle \rho f^{- 1} \rangle y = f^{- 1} \circ y$.

TODO: More to write on this topic.

\chapter{Equalizers and co-Equalizers in Certain Categories}

It is a rough draft. Errors are possible.

\fxwarning{Change notation $\prod$ $\rightarrow$ $\prod^{(L)}$.}

\section{Categories with embeddings}

\begin{note}
  This section in not used below, it is just to feed your intuition.
\end{note}

The following generalizes the well known concept of embedding function $A
\hookrightarrow B$ for from a set $A$ to a set $B$ where $A \subseteq B$.

I will set that the unique morphism from an object $A$ to an object $B$ of a
thin category is equal to the pair $(A ; B)$.

\begin{defn}
  A \emph{category with embeddings of objects} is a dagger category with a
  preorder of the set of objects together with a functor $\hookrightarrow$ (we
  will denote applying this functor to the object $(A ; B)$ as $A
  \hookrightarrow B$.) such that:
  \begin{itemize}
    \item $\hookrightarrow$ is an identity on objects.
    
    \item Every $A \hookrightarrow B$ is a monomorphism.
    
    \item $(A \hookrightarrow B)^{\dagger} \circ (A \hookrightarrow B) = 1_A$.
  \end{itemize}
\end{defn}

\begin{obvious}
$A\hookrightarrow B$ is defined when $(A ; B)$ is a morphism of the preorder
that is when $A \sqsubseteq B$.
\end{obvious}

\begin{obvious}
$A \hookrightarrow B : A \rightarrow B$ when $A \sqsubseteq B$.
\end{obvious}

\begin{prop}
  $A \hookrightarrow A = 1_A$.
\end{prop}

\begin{proof}
  Because $(A ; A)$ is an identity morphism and $\hookrightarrow$ preserves
  identities.
\end{proof}

\begin{prop}
  $(B \hookrightarrow C) \circ (A \hookrightarrow B) = A \hookrightarrow C$
  whenever $A \sqsubseteq B \sqsubseteq C$.
\end{prop}

\begin{proof}
  $(B \hookrightarrow C) \circ (A \hookrightarrow B) = \hookrightarrow (B ; C)
  \circ \hookrightarrow (A ; B) = \hookrightarrow ((B ; C) \circ (A ; B)) =
  \hookrightarrow (A ; C) = A \hookrightarrow C$.
\end{proof}

\section{\texorpdfstring{Categories under $\mathbf{Rel}$}{Categories under Rel}}

\begin{defn}
  The $\mathbf{Rel}$-morphism $A \rightleftarrows B$
  (\emph{restriction-embedding}) is defined by the formula: $A
  \rightleftarrows B = (A ; B ; \id_{A \cap B})$.
\end{defn}

\begin{obvious}
If $A \subseteq B$ then $A \rightleftarrows B$ is an embedding $A \hookrightarrow B
= (A ; B ; \id_A)$.
\end{obvious}

\begin{obvious}
If $A \supseteq B$ then $A \rightleftarrows B = (A ; B ;
\id_B)$.
\end{obvious}

\begin{obvious}
$A \rightleftarrows A = 1^{\mathbf{Rel}}_A$.
\end{obvious}

\begin{obvious}
$(A \rightleftarrows B)^{- 1} = B \rightleftarrows A$.
\end{obvious}

\begin{defn}
\emph{Dagger functor} between two dagger categories is a functor between
these categories, which commutes with the daggers.
\fxwarning{Clearer wording.}
\end{defn}

\begin{defn}
\emph{Category under $\mathbf{Rel}$} is a pair $(C ; \uparrow)$
where $C$ is a category whose objects are small sets and $\uparrow$ is an
identity-on-objects functor $\mathbf{Rel} \rightarrow C$. I call
$\uparrow$ \emph{up-arrow functor}.
\end{defn}

\begin{defn}
  \emph{Dagger category under $\mathbf{Rel}$} is a pair $(C ;
  \uparrow)$ where $C$ is a dagger category whose objects are small sets and
  $\uparrow$ is a dagger identity-on-objects functor $\mathbf{Rel}
  \rightarrow C$.
\end{defn}

\begin{defn}
  $A \rightleftarrows^{\mathcal{C}} B = \uparrow (A \rightleftarrows B)$. In
  other words, $\rightleftarrows^{\mathcal{C}} = \uparrow \circ
  \rightleftarrows$.
\end{defn}

\begin{prop}
  $A \rightleftarrows^{\mathcal{C}} A = 1_{\mathcal{C}}^A$.
\end{prop}

\begin{proof}
  $A \rightleftarrows^{\mathcal{C}} A = \uparrow (A \rightleftarrows A) =
  \uparrow 1_{\mathbf{Rel}} = 1_{\mathcal{C}}^A$.
\end{proof}

\begin{prop}
  If $f : X \rightarrow Y$ is a $\mathbf{Rel}$-morphism and
  $\im f = A \subseteq Y$ then
  \[ (A \rightleftarrows Y) \circ (Y \rightleftarrows A) \circ f = f. \]
\end{prop}

\begin{proof}
  $(A \rightleftarrows Y) \circ (Y \rightleftarrows A) \circ f = \id_A
  \circ f = f$.
\end{proof}

\begin{cor}
  If $f : X \rightarrow Y$ is a morphism of a category under
  $\mathbf{Rel}$ and $\im f = A \subseteq Y$, then
  \[ (A \rightleftarrows^{\mathcal{C}} Y) \circ (Y
     \rightleftarrows^{\mathcal{C}} A) \circ \uparrow f = \uparrow f. \]
\end{cor}

\begin{prop}
  ~  
  \begin{enumerate}
    \item If $A \subseteq B$ then $A \rightleftarrows^{\mathcal{C}} B$ is a
    monomorphism.
    
    \item If $A \supseteq B$ then $A \rightleftarrows^{\mathcal{C}} B$ is a
    epimorphism.
  \end{enumerate}
\end{prop}

\begin{proof}
  We'll prove only the first as the second is dual.
  
  Let $(A \rightleftarrows^{\mathcal{C}} B) \circ f = (A
  \rightleftarrows^{\mathcal{C}} B) \circ g$. Then $(B
  \rightleftarrows^{\mathcal{C}} A) \circ (A \rightleftarrows^{\mathcal{C}} B)
  \circ f = (B \rightleftarrows^{\mathcal{C}} A) \circ (A
  \rightleftarrows^{\mathcal{C}} B) \circ g$; $1^A \circ f = 1^A \circ g$; $f
  = g$.
\end{proof}

\begin{prop}
  $(B \rightleftarrows C) \circ (A \rightleftarrows B) = A \rightleftarrows C$
  iff $B \supseteq A \cap C$ (for every sets $A$, $B$, $C$).
\end{prop}

\begin{proof}
  $(B \rightleftarrows C) \circ (A \rightleftarrows B) = A \rightleftarrows C$
  is equivalent to:
  
  $(B ; C ; \id_{B \cap C}) \circ (A ; B ; \id_{A \cap B}) = (A ;
  C ; \id_{A \cap C})$;
  
  $(A ; C ; \id_{A \cap B \cap C}) = (A ; C ; \id_{A \cap C})$;
  
  $A \cap B \cap C = A \cap C$;
  
  $B \supseteq A \cap C$.
\end{proof}

\begin{cor}
  $(B \rightleftarrows^{\mathcal{C}} C) \circ (A
  \rightleftarrows^{\mathcal{C}} B) = (A \rightleftarrows^{\mathcal{C}} C)$ if
  $B \supseteq A \cap C$ (for every sets $A$, $B$, $C$).
\end{cor}

\begin{defn}
  \emph{Partially ordered dagger category under $\mathbf{Rel}$} is
  a category which is both a partially ordered dagger category and a category
  under $\mathbf{Rel}$ such that $\uparrow \circ f^{- 1} = (\uparrow
  \circ f)^{\dagger}$ and $A \sqsubseteq B \Rightarrow \uparrow A \sqsubseteq
  \uparrow B$.
\end{defn}

\begin{prop}
  $(A \rightleftarrows^{\mathcal{C}} B)^{\dagger} = B
  \rightleftarrows^{\mathcal{C}} A$ for a dagger category under
  $\mathbf{Rel}$.
\end{prop}

\begin{proof}
  $(A \rightleftarrows^{\mathcal{C}} B)^{\dagger} = (\uparrow (A
  \rightleftarrows B))^{\dagger} = \uparrow (A \rightleftarrows B)^{- 1} =
  \uparrow (B \rightleftarrows A) = B \rightleftarrows^{\mathcal{C}} A$.
\end{proof}

\begin{prop}
  For a partially ordered dagger category $\mathcal{C}$ under
  $\mathbf{Rel}$ we have $A \rightleftarrows^{\mathcal{C}} B$ is:
  \begin{enumerate}
    \item monovalued;
    
    \item injective;
    
    \item entirely defined if $A \subseteq B$;
    
    \item surjective if $B \subseteq A$.
  \end{enumerate}
\end{prop}

\begin{proof}
  ~
  \begin{enumerate}
    \item $(A \rightleftarrows B) \circ (B \rightleftarrows A) \sqsubseteq
    1^{\mathbf{Rel}}_B$; $(A \rightleftarrows B) \circ (A
    \rightleftarrows B)^{- 1} \sqsubseteq 1^{\mathbf{Rel}}_B$; $(A
    \rightleftarrows^{\mathcal{C}} B) \circ (A \rightleftarrows^{\mathcal{C}}
    B)^{\dagger} \sqsubseteq 1^{\mathcal{C}}_B$.
    
    \item $(B \rightleftarrows A) \circ (A \rightleftarrows B) \sqsubseteq
    1^{\mathbf{Rel}}_A$; $(A \rightleftarrows B)^{- 1} \circ (A
    \rightleftarrows B) \sqsubseteq 1^{\mathbf{Rel}}_A$; $(A
    \rightleftarrows^{\mathcal{C}} B)^{\dagger} \circ (A
    \rightleftarrows^{\mathcal{C}} B) \sqsubseteq 1^{\mathcal{C}}_A$.
    
    \item $(B \rightleftarrows A) \circ (A \rightleftarrows B) \sqsupseteq
    1^{\mathbf{Rel}}_A$; $(A \rightleftarrows B)^{- 1} \circ (A
    \rightleftarrows B) \sqsupseteq 1^{\mathbf{Rel}}_A$; $(A
    \rightleftarrows^{\mathcal{C}} B)^{\dagger} \circ (A
    \rightleftarrows^{\mathcal{C}} B) \sqsupseteq 1^{\mathcal{C}}_A$.
    
    \item $(A \rightleftarrows B) \circ (B \rightleftarrows A) \sqsupseteq
    1^{\mathbf{Rel}}_A$; $(A \rightleftarrows B) \circ (A
    \rightleftarrows B)^{- 1} \sqsupseteq 1^{\mathbf{Rel}}_A$; $(A
    \rightleftarrows^{\mathcal{C}} B) \circ (A \rightleftarrows^{\mathcal{C}}
    B)^{\dagger} \sqsupseteq 1^{\mathcal{C}}_A$.
  \end{enumerate}
\end{proof}

\section{Rectangular embedding-restriction}

\begin{defn}
  $\iota_{B_0, B_1} f = (A_1 \rightleftarrows^{\mathcal{C}} B_1) \circ f \circ
  (B_0 \rightleftarrows^{\mathcal{C}} A_0)$ for $f \in
  \Hom_{\mathcal{C}} (A_0 ; A_1)$.
\end{defn}

For brevity $\iota_B f = \iota_{B, B} f$.

\begin{prop}
  $\iota_{\Src f, \Dst f} f = f$.
\end{prop}

\begin{proof}
  $\iota_{\Src f, \Dst f} f = (\Dst f
  \rightleftarrows^{\mathcal{C}} \Dst f) \circ f \circ (\Src f
  \rightleftarrows^{\mathcal{C}} \Src f) = 1_{\mathcal{C}}^{\Dst
  f} \circ f \circ 1_{\mathcal{C}}^{\Src f} = f$.
\end{proof}

\begin{prop}
  The function $\iota_{B_0, B_1} |_{f \in \Hom_{\mathcal{C}} (A_0 ;
  A_1)}$ is injective, if $A_0 \subseteq B_0 \wedge A_1 \subseteq B_1$.
\end{prop}

\begin{proof}
  Because $A_1 \rightleftarrows^{\mathcal{C}} B_1$ is a monomorphism and $A_0
  \rightleftarrows^{\mathcal{C}} B_0$ is an epimorphism.
\end{proof}

\begin{prop}
  $\iota_{C_0, C_1} \iota_{B_0, B_1} f = \iota_{C_0, C_1} f$ for $B_0
  \supseteq A_0 \cap C_0$, $B_1 \supseteq A_1 \cap C_1$ and $f : A_0
  \rightarrow A_1$.
\end{prop}

\begin{proof}
  $\iota_{C_0, C_1} \iota_{B_0, B_1} f = (B_1 \rightleftarrows^{\mathcal{C}}
  C_1) \circ (A_1 \rightleftarrows^{\mathcal{C}} B_1) \circ f \circ (B_0
  \rightleftarrows^{\mathcal{C}} A_0) \circ (C_0
  \rightleftarrows^{\mathcal{C}} B_0) = (A_1 \rightleftarrows^{\mathcal{C}}
  C_1) \circ f \circ (C_0 \rightleftarrows^{\mathcal{C}} A_0) = \iota_{C_0,
  C_1} f$.
\end{proof}

\begin{prop}
  Let $f : A_0 \rightarrow A_1$ and $g : A_1 \rightarrow A_2$ and $A_1
  \subseteq B_1$. Then $\iota_{B_0, B_2} (g \circ f) = \iota_{B_1, B_1} g
  \circ \iota_{B_0, B_1} f$.
\end{prop}

\begin{proof}
  $\iota_{B_0, B_2} (g \circ f) = (A_2 \rightleftarrows^{\mathcal{C}} B_2)
  \circ (g \circ f) \circ (B_0 \rightleftarrows^{\mathcal{C}} A_0) = (A_2
  \rightleftarrows^{\mathcal{C}} B_2) \circ g \circ \id_{A_1} \circ f
  \circ (B_0 \rightleftarrows^{\mathcal{C}} A_0) = (A_2
  \rightleftarrows^{\mathcal{C}} B_2) \circ g \circ (B_1 \rightleftarrows A_1)
  \circ (A_1 \rightleftarrows B_1) \circ f \circ (B_0
  \rightleftarrows^{\mathcal{C}} A_0) = \iota_{B_1, B_1} g \circ \iota_{B_0,
  B_1} f$.
\end{proof}

\section{\texorpdfstring{Examples of partially ordered dagger categories under
$\mathbf{Rel}$}{Examples of partially ordered dagger categories under Rel}}

\subsection{Generalized rebase of filters}

In \cite{volume-1} I defined \emph{rebase} $\mathcal{A} \div A$ for a
set-theoretic filter $\mathcal{A}$ and a set $X$ such that $\exists X \in
\mathcal{A} : X \subseteq A$.

Now define a generalized rebase for every set-theoretic filter $\mathcal{A}$
and every set $A$:

\begin{defn}
  $\mathcal{A} \div A = \bigsqcap \setcond{ \uparrow^A  (X \cap A) }
  { X \in \mathcal{A} }$.
\end{defn}

\begin{prop}
  These two definitions coincide.
\end{prop}

\begin{proof}
  It is proved in
{\cite{volume-1}} $\setcond{ X \in \subsets A }
{\exists Y \in \mathcal{A} : Y\cap A \subseteq X }$ is a filter.

If $P \in \setcond{ X \in \subsets A }{ \exists Y
\in \mathcal{A} : Y\cap A \subseteq X }$ then $P \in \subsets A$ and $Y\cap A
\subseteq P$ for some $Y \in \mathcal{A}$. Thus $P \supseteq Y \cap A \in
\bigsqcap \setcond{ \uparrow^A  (Y \cap A) }{ Y \in \mathcal{A} }$.

If $P \in \bigsqcap \setcond{ \uparrow^A  (X \cap A) }{
X \in \mathcal{A} }$ then by properties of generalized filter bases,
there exists $X \in \mathcal{A}$ such that $P \supseteq X \cap A$. Also $P \in
\subsets A$. Thus $P \in \setcond{ X \in \subsets A
}{ \exists Y \in \mathcal{A} : Y\cap A \subseteq X }$.

\fxwarning{Clear this proof: wording, consistent use of letters.}
\end{proof}

\begin{prop}
  $(\mathcal{X} \div A) \div B = \mathcal{X} \div B$ if $B \subseteq A$.
\end{prop}

\begin{proof}
  $(\mathcal{X} \div A) \div B = \bigsqcap \setcond{ \uparrow^B  (Y \cap B)
  }{ Y \in \bigsqcap \setcond{ \uparrow^A  (X \cap A)}
  {X \in \mathcal{X} } \mathcal{} } =
  \bigsqcap \setcond{ \uparrow^B  (X \cap A) }{ X \in
  \mathcal{X} } \sqcap \uparrow^B B = \bigsqcap \setcond{ \uparrow^B  (X
  \cap A \cap B) }{ X \in \mathcal{X} } =
  \mathcal{X} \div (A \cap B) = \mathcal{X} \div B$.
\end{proof}

\subsection{\texorpdfstring{Category $\mathbf{Rel}$}{Category Rel}}

Category $\mathbf{Rel}$ with the identity up-arrow functor to itself
and ``reverse relation'' as the dagger is an obvious example of a partially
ordered dagger category under $\mathbf{Rel}$.

\begin{prop}
  $\iota_{A, B} f = (A ; B ; \GR f \cap (A \times B))$.
\end{prop}

\begin{proof}
  $\iota_{A, B} f = (\Dst f \rightleftarrows B) \circ f \circ (A
  \rightleftarrows \Src f) = (A ; B ; \GR f \cap (A \times B))$.
\end{proof}

\subsection{\texorpdfstring{Category $\mathsf{FCD}$}{Category FCD}}

Category $\mathsf{FCD}$ with the up-arrow functor
$\uparrow^{\mathsf{FCD}}$ and ``reverse funcoid'' as the dagger is a
partially ordered dagger category under $\mathbf{Rel}$.

\begin{prop}
  $A \rightleftarrows^{\mathsf{FCD}} B = (A ; B ; \lambda \mathcal{X}
  \in \mathfrak{F} (A) : \mathcal{X} \div B ; \lambda \mathcal{Y} \in
  \mathfrak{F} (B) : \mathcal{Y} \div A)$ for objects $A \subseteq B$ of
  $\mathsf{FCD}$.
\end{prop}

\begin{proof}
  $\langle A \rightleftarrows^{\mathsf{FCD}} B \rangle \mathcal{X} =
  \bigsqcap \left\{ \langle A \rightleftarrows^{\mathsf{FCD}} B
  \rangle^{\ast} X \hspace{1em} | \hspace{1em} X \in \mathcal{X} \right\} =
  \bigsqcap \left\{ \uparrow^B  \langle A \rightleftarrows B \rangle X
  \hspace{1em} | \hspace{1em} X \in \mathcal{X} \right\} = \bigsqcap \left\{
  \uparrow^B  (X \cap A \cap B) \hspace{1em} | \hspace{1em} X \in \mathcal{X}
  \right\} = \bigsqcap \left\{ \uparrow^B  (X \cap B) \hspace{1em} |
  \hspace{1em} X \in \mathcal{X} \right\} = \mathcal{X} \div B$.
  
  Rest follows from symmetry.
\end{proof}

\begin{prop}
  ~
  \begin{enumerate}
    \item $\langle A \rightleftarrows^{\mathsf{FCD}} B \rangle^{\ast} X
    = \uparrow^B X$ for every $X \in \subsets A$ if $A \subseteq B$.
    
    \item $\langle (B \rightleftarrows^{\mathsf{FCD}} A) \rangle^{\ast}
    Y = \uparrow^A (Y \cap A)$ for every $Y \in \subsets B$ if $A \subseteq
    B$.
  \end{enumerate}
\end{prop}

\begin{proof}
  By definition of principal funcoid.
\end{proof}

\subsection{\texorpdfstring{Category $\mathsf{RLD}$}{Category RLD}}

Category $\mathsf{RLD}$ with the up-arrow functor
$\uparrow^{\mathsf{RLD}}$ and ``reverse reloid'' as the dagger is a
partially ordered dagger category under $\mathbf{Rel}$.

\begin{obvious}
$A \rightleftarrows^{\mathsf{RLD}} B = \uparrow^{\mathsf{RLD} (A ;
B)} \id_{A \cap B}$.
\end{obvious}

\begin{defn}
  $f \div (A \times B) = (A ; B ; (\GR f) \div (A \times B))$ for every
  reloid $f$.
\end{defn}

\begin{prop}
  $\iota_{A, B} f = f \div (A \times B)$.
\end{prop}

\begin{proof}
  $\iota_{A, B} f = (\Dst f \rightleftarrows^{\mathsf{RLD}} B)
\circ f \circ (A \rightleftarrows^{\mathsf{RLD}} \Src f) =
\bigsqcap \setcond{ \uparrow^{\mathsf{RLD}} ((\Dst f
\rightleftarrows B) \circ F \circ (A \rightleftarrows \Src f))
}{ F \in \GR f } = \bigsqcap \setcond{
\uparrow^{\mathsf{RLD}} (F \cap (A \times B))}
{F \in \GR f } = f \div (A \times B)$.

\fxwarning{Filters on cartesian products vs reloids.}
\end{proof}

\section{Equalizers}

Categories $\cont (\mathcal{C})$ are defined above.

I will denote $W$ the forgetful functor from $\cont
(\mathcal{C})$ to $\mathcal{C}$.

In the definition of the category $\cont (\mathcal{C})$ take
values of $\uparrow$ as principal morphisms. \fxwarning{Wording.}

\begin{lem}
  Let $f : X \rightarrow Y$ be a morphism of the category
  $\cont (\mathcal{C})$ where $\mathcal{C}$ is a concrete
  category (so $W f = \uparrow \varphi$ for a $\mathbf{Rel}$-morphism
  $\varphi$ because $f$ is principal) and $\im \varphi = A \subseteq
  \Ob Y$. Factor it $\varphi = (A \rightleftarrows \Ob Y) \circ u$
  where $u : \Ob X \rightarrow A$ using properties of
  $\mathbf{Set}$. Then $u$ is a morphism of $\cont
  (\mathcal{C})$ (that is a continuous function $X \rightarrow \iota_A Y$).
\end{lem}

\begin{proof}
  $(A \rightleftarrows \Ob Y)^{- 1} \circ \varphi = (A \rightleftarrows
  \Ob Y)^{- 1} \circ (A \rightleftarrows \Ob Y) \circ u$;
  
  $(A \rightleftarrows^{\mathcal{C}} \Ob Y)^{- 1} \circ \uparrow \varphi
  = (A \rightleftarrows^{\mathcal{C}} \Ob Y)^{- 1} \circ (A
  \rightleftarrows^{\mathcal{C}} \Ob Y) \circ \uparrow u$;
  
  $(A \rightleftarrows^{\mathcal{C}} \Ob Y)^{- 1} \circ \uparrow \varphi
  = \uparrow u$;
  
  $X \sqsubseteq (\uparrow u)^{- 1} \circ \pi_A Y \circ \uparrow u
  \Leftrightarrow X \sqsubseteq (\uparrow \varphi)^{- 1} \circ (A
  \rightleftarrows^{\mathcal{C}} \Ob Y) \circ \pi_A Y \circ (A
  \rightleftarrows^{\mathcal{C}} \Ob Y)^{- 1} \circ \uparrow \varphi
  \Leftrightarrow X \sqsubseteq (\uparrow \varphi)^{- 1} \circ (A
  \rightleftarrows^{\mathcal{C}} \Ob Y) \circ (A
  \rightleftarrows^{\mathcal{C}} \Ob Y)^{- 1} \circ Y \circ (A
  \rightleftarrows^{\mathcal{C}} \Ob Y) \circ (A
  \rightleftarrows^{\mathcal{C}} \Ob Y)^{- 1} \circ \uparrow \varphi
  \Leftrightarrow X \sqsubseteq (\uparrow \varphi)^{- 1} \circ Y \circ
  \uparrow \varphi \Leftrightarrow X \sqsubseteq (W f)^{- 1} \circ Y \circ W
  f$ what is true by definition of continuity.
\end{proof}

Equational definition of equalizers:

\url{http://nforum.mathforge.org/comments.php?DiscussionID=5328/}

\begin{thm}
  The following is an equalizer of parallel morphisms $f, g : A \rightarrow B$
  of category $\cont (\mathcal{C})$:
  \begin{itemize}
    \item the object $X = \iota_{\setcond{ x \in \Ob A }{
    f x = g x }} A$;
    
    \item the morphism $\Ob X \rightleftarrows \Ob A$ considered
    as a morphism $X \rightarrow A$.
  \end{itemize}
\end{thm}

\begin{proof}
  Denote $e = \Ob X \rightleftarrows \Ob A$.
  
  Let $f \circ z = g \circ z$ for some morphism $z$.
  
  Let's prove $e \circ u = z$ for some $u : \Src z \rightarrow X$.
  Really, as a morphism of $\mathbf{Set}$ it exists and is unique.
  
  Consider $z$ as as a generalized element.
  
  $f (z) = g (z)$. So $z \in X$ (that is $\Dst z \in X$). Thus $z = e
  \circ u$ for some $u$ (by properties of $\mathbf{Set}$). The
  generalized element $u$ is a $\cont (\mathcal{C})$-morphism
  because of the lemma above. It is unique by properties of
  $\mathbf{Set}$.
\end{proof}

We can (over)simplify the above theorem by the obvious below:

\begin{obvious}
$\setcond{ x \in \Ob A }{ f x = g x } = \dom (f \cap g)$.
\end{obvious}

\section{Co-equalizers}

\url{http://math.stackexchange.com/questions/539717/how-to-construct-co-equalizers-in-mathbftop}	

Let $\sim$ be an equivalence relation. Let's denote $\pi$ its canonical
projection.

\begin{defn}
  $f / \sim = \uparrow \pi \circ f \circ \uparrow \pi^{- 1}$ for every
  morphism $f$.
\end{defn}

\begin{obvious}
$\Ob (f / \sim) = (\Ob f) / r$.
\end{obvious}

\begin{obvious}
$f / \sim = \langle \uparrow^{\mathsf{FCD}} \pi \times^{(C)}
\uparrow^{\mathsf{FCD}} \pi \rangle f$ for every morphism
$f$.
\end{obvious}

To define co-equalizers of morphisms $f$ and $g$ let $\sim$ be is the smallest
equivalence relation such that $f x = g x$.

\begin{lem}
  Let $f : X \rightarrow Y$ be a morphism of the category
  $\cont (\mathcal{C})$ where $\mathcal{C}$ is a concrete
  category (so $W f = \uparrow \varphi$ for a $\mathbf{Rel}$-morphism
  $\varphi$ because $f$ is principal) such that $\varphi$ respects $\sim$.
  Factor it $\varphi = u \circ \pi$ where $u : \Ob (X / \sim)
  \rightarrow \Ob Y$ using properties of $\mathbf{Set}$. Then
  $u$ is a morphism of $\cont (\mathcal{C})$ (that is a
  continuous function $X / \sim \rightarrow Y$).
\end{lem}

\begin{proof}
  $f \circ X \circ f^{- 1} \sqsubseteq Y$; $\uparrow u \circ \uparrow \pi
  \circ X \circ \uparrow \pi^{- 1} \circ \uparrow u^{- 1} \sqsubseteq Y$;
  $\uparrow u \in \mathrm{C} (\uparrow \pi \circ X \circ \uparrow \pi^{- 1} ;
  Y) = \mathrm{C} (X / \sim ; Y)$.
\end{proof}

\begin{thm}
  The following is a co-equalizer of parallel morphisms $f, g : A \rightarrow
  B$ of category $\cont (\mathcal{C})$:
  \begin{itemize}
    \item the object $Y = f / \sim$;
    
    \item the morphism $\pi$ considered as a morphism $B \rightarrow Y$.
  \end{itemize}
\end{thm}

\begin{proof}
  Let $z \circ f = z \circ g$ for some morphism $z$.
  
  Let's prove $u \circ \pi = z$ for some $u : Y \rightarrow \Dst z$.
  Really, as a morphism of $\mathbf{Set}$ it exists and is unique.
  
  $\Src z \in Y$. Thus $z = u \circ \pi$ for some $u$ (by properties of
  $\mathbf{Set}$). The function $u$ is a $\cont
  (\mathcal{C})$-morphism because of the lemma above. It is unique by
  properties of $\mathbf{Set}$ ($\pi$ obviously respects equivalence
  classes).
\end{proof}

\section{Rest}

\begin{thm}
  The categories $\cont (\mathcal{C})$ (for example in
  $\mathbf{Fcd}$ and $\mathbf{Rld}$) are complete.
\end{thm}

\begin{proof}
  They have products and equalizers.
\end{proof}

\begin{thm}
  The categories $\cont (\mathcal{C})$ (for example in
  $\mathbf{Fcd}$ and $\mathbf{Rld}$) are co-complete.
\end{thm}

\begin{proof}
  They have co-products and co-equalizers.
\end{proof}

\begin{defn}
  I call morphisms $f$ and $g$ of a category with embeddings
  \emph{equivalent} ($f \sim g$) when there exist a morphism $p$ such that
  $\Src p \sqsubseteq \Src f$, $\Src p \sqsubseteq
  \Src g$, $\Dst p \sqsubseteq \Dst f$, $\Dst p
  \sqsubseteq \Dst g$ and $\iota_{\Src f, \Dst f} p = f$ and
  $\iota_{\Src g, \Dst g} p = g$.
\end{defn}

\begin{problem}
  Find under which conditions:
  \begin{enumerate}
    \item Equivalence of morphisms is an equivalence relation.
    
    \item Equivalence of morphisms is a congruence for our category.
  \end{enumerate}
\end{problem}
\chapter{Categories of filters}

In~\cite{filt-cat} two categories, whose objects are related with filters on sets, are defined and researched.

Accordingly~\cite{filt-cat} infinite product is defined just in the first (denoted $\mathscr{F}$ there) of these two categories.
So we will for now consider the first category. (Usefulness of the second category for our research is questionable.)

Let $f:A\rightarrow B$ be a function, $\mathcal{A}$ be a filter on~$A$.

\begin{prop}
$\setcond{Y\in\subsets B}{\rsupfun{f^{-1}}Y\in\mathcal{A}}$ is a filter.
\end{prop}

\begin{proof}
That it is an upper set is obvious.

Let $Y_0,Y_1\in\setcond{Y\in\subsets B}{\rsupfun{f^{-1}}Y\in\mathcal{A}}$. Then
$\rsupfun{f^{-1}}Y_0\in\mathcal{A}$ and $\rsupfun{f^{-1}}Y_1\in\mathcal{A}$.
We have
\[ \rsupfun{f^{-1}}(Y_0\cap Y_1) = \rsupfun{f^{-1}}Y_0 \cap \rsupfun{f^{-1}}Y_1 \in \mathcal{A} \]
since $f$ is monovalued.
Thus $Y_0 \cap Y_1\in\setcond{Y\in\subsets B}{\rsupfun{f^{-1}}Y\in\mathcal{A}}$.
\end{proof}

\begin{thm}
\fxwarning{Should be moved above in the book.}
$\setcond{Y\in\subsets B}{\rsupfun{f^{-1}}Y\in\mathcal{A}}$ is equal to the filter generated
by the filter base $\rsupfun{\rsupfun{f}}\mathcal{A}$, for every filter~$\mathcal{A}$.
\end{thm}

\begin{proof}
Denote $\mathcal{B} = \setcond{Y\in\subsets B}{\rsupfun{f^{-1}}Y\in\mathcal{A}}$,
$\mathcal{C} = \rsupfun{\rsupfun{f}}\mathcal{A}$.

Let $Y\in\mathcal{C}$. Then $Y=\rsupfun{f}A$ where $A\in\mathcal{A}$.
Then $\rsupfun{f^{-1}}\rsupfun{f}A\supseteq A$ and so $\rsupfun{f^{-1}}\rsupfun{f}A\in\mathcal{A}$.
This proves $\rsupfun{f}A\in\mathcal{B}$, that is $Y\in\mathcal{B}$.

Let now $Y\in\mathcal{B}$. Then $\rsupfun{f}\rsupfun{f^{-1}}Y\subseteq Y$. Since $\rsupfun{f^{-1}}Y\in\mathcal{A}$,
we have that $Y$ is a supset of some set of the form $\rsupfun{f}A$, so $Y\in\mathcal{C}$.
\end{proof}

\begin{cor}
$\up\supfun{f}\mathcal{A} = \setcond{Y\in\subsets B}{\rsupfun{f^{-1}}Y\in\up\mathcal{A}}$.
\end{cor}

\begin{defn}
The \emph{category of filtered sets} $\mathbf{Filt}$ is the category defined as follows:
\begin{enumerate}
\item Objects are pairs $(A;\mathcal{A})$ where $A$ is a (small) set and $\mathcal{A}$ is a filter on~$A$.
\item Morphisms from $(A;\mathcal{A})$ to $(B;\mathcal{B})$ are functions $f:A\rightarrow B$ such that
$\supfun{f}\mathcal{A} \sqsubseteq \mathcal{B}$.
\item Identities are identity functions.
\end{enumerate}
\end{defn}

To verify that it is a category is straightforward.

It is the same category as $\mathscr{F}$ in \cite{filt-cat}, as follows from an above proposition.

We will prove that starred reloidal product is a categorical product in this category.
First we will prove the special case that binary reloidal product is a categorical product in this category.
\chapter{Power of filters}

\section{Germs of functions}

\begin{defn}
  Functions $f, g \in \mathbf{Rel} (\Ob \mathcal{X} ; B)$
  \emph{are of the same $\mathcal{X}$-germ} for a filter object
  $\mathcal{X}$ iff there exists $X \in \up \mathcal{X}$ such that $f|_X
  = g|_X$.
\end{defn}

\begin{prop}
  Being of the same germ is an equivalence relation.
\end{prop}

\begin{proof}
  
  \begin{description}
    \item[Reflexivity] Take arbitrary $X \in \up \mathcal{X}$.
    
    \item[Symmetry] Obvious.
    
    \item[Transitivity] Let $f|_X = g|_X$ and $g|_Y = h|_Y$. Then $f|_{X \cap
    Y} = h|_{X \cap Y}$.
  \end{description}
\end{proof}

\begin{defn}
  A \emph{germ} is an equivalence class of being the same germ.
\end{defn}

\begin{obvious}
Every germ is a filter on $\mathbf{Set}$.
\end{obvious}

\begin{thm}
  Let $A$, $B$ be sets. Suppose $\card B \neq 1$.
  
  The following are mutually inverse bijections between monovalued reloids $f
  : A \rightarrow B$ with $\dom f = \mathcal{X}$ and $\mathcal{X}$-germs
  $S$ of functions $A \rightarrow B$ for $\mathcal{X} \in \mathscr{F} A$:
  \begin{itemize}
    \item $f \mapsto \up^{\mathbf{Set}} f$;
    
    \item $S \mapsto \bigsqcap^{\mathsf{RLD}} S$.
  \end{itemize}
\end{thm}

\begin{proof}
  We can assume that $A, B \neq \emptyset$ because otherwise the theorem is
  obvious.
  
  First prove that $\up^{\mathbf{Set}} f$ is an
  $\mathcal{X}$-germ. Really, $F \in \up^{\mathbf{Set}} f
  \Leftrightarrow F \sqsupseteq f \Leftrightarrow F|_{\mathcal{X}} = f
  \Leftrightarrow \exists X \in \up \mathcal{X} : F|_X \sqsupseteq f$;
  thus $F, G \in \up^{\mathbf{Set}} f \Rightarrow \exists X \in
  \up \mathcal{X} : F|_X \sqsupseteq f \wedge \exists Y \in \up
  \mathcal{X} : G|_Y \sqsupseteq f \Rightarrow \exists X \in \up
  \mathcal{X} : F|_{X \cap Y} \sqsupseteq f \wedge \exists Y \in \up
  \mathcal{X} : G|_{X \cap Y} \sqsupseteq f \Rightarrow \exists Z \in
  \up \mathcal{X} : (F|_Z \sqsupseteq f \wedge G|_Z \sqsupseteq f)
  \Rightarrow \exists Z \in \up \mathcal{X} : (F \sqcap G) |_Z
  \sqsupseteq f$ and $F \in \up^{\mathbf{Set}} f \wedge \exists
  X \in \up \mathcal{X} : F|_X = G|_X \Rightarrow F \sqsupseteq f \wedge
  F|_{\mathcal{X}} = G|_{\mathcal{X}} \Rightarrow G|_{\mathcal{X}} \sqsupseteq
  f \Rightarrow G \in \up^{\mathbf{Set}} f$. We have proved
  that $\up^{\mathbf{Set}} f$ is an equivalence class of the
  suitable equivalence relation, that is $\up^{\mathbf{Set}} f$
  is an $\mathcal{X}$-germ.
  
  That $\bigsqcap^{\mathsf{RLD}} S$ is a monovalued reloid is obvious.
  We need to prove that $\im \bigsqcap^{\mathsf{RLD}} S =
  \mathcal{X}$.
  
  If $\mathcal{X} = X$ then obviously $S$ has just one element $F$ and
  $\im \bigsqcap^{\mathsf{RLD}} S = \im F = X =
  \mathcal{X}$. Otherwise for every $X \in \up \mathcal{X}$ there are
  elements $F$, $G$ of $S$ such that $\dom (F \sqcap G) \sqsubseteq X$
  (using $\card B > 1$).
  
  By properties of generalized filter bases $X \times \top \sqsupseteq
  \bigsqcap^{\mathsf{RLD}} S \Leftrightarrow \exists F, G \in S : X
  \times \top \sqsupseteq F \sqcap G \Leftrightarrow X \sqsupseteq
  \mathcal{X}$. Thus $\im \bigsqcap^{\mathsf{RLD}} S =
  \mathcal{X}$.
  
  It remains to prove that our correspondences are mutually inverse.
  
  Let $f_0 : A \rightarrow B$ be a monovalued reloid and $\dom f =
  \mathcal{X}$. Let $S = \up^{\mathbf{Set}} f$ and $f_1 =
  \bigsqcap^{\mathsf{RLD}} S$. We need to prove $f_1 = f_0$. Really,
  $f_1 \sqsupseteq f_0$ is obvious and $f_1 = \bigsqcap^{\mathsf{RLD}}
  \up^{\mathbf{Set}} f_0 \sqsubseteq
  \bigsqcap^{\mathsf{RLD}} \up^{\mathbf{Rel}} f_0$.
  
  Let $S_0$ be an $\mathcal{X}$-germ of functions $A \rightarrow B$. Let $f =
  \bigsqcap^{\mathsf{RLD}} S_0$ and $S_1 =
  \up^{\mathbf{Set}} f$. We need to prove $S_1 = S_0$. Really,
  \[ S_1 = \up^{\mathbf{Set}}  \bigsqcap^{\mathsf{RLD}}
     S_0 = \mathbf{Set} \cap \up^{\mathbf{Rel}} 
     \bigsqcap^{\mathsf{RLD}} S_0 = \mathbf{Set} \cap
     \up^{\mathbf{Rel}}  \bigsqcap^{\mathsf{RLD}}
     \up \bigsqcap^{\mathscr{F} (\mathbf{Set})} S_0 = \up
     \bigsqcap^{\mathscr{F} (\mathbf{Set})} S_0 = S_0 . \]
\end{proof}
\section{Power of filters}

Let's define $\mathcal{Y}^{\mathcal{X}}$ for filters~$\mathcal{X}$,~$\mathcal{Y}$:

First define $Y^{\mathcal{X}}$ for a set~$Y$:
\[ Y^{\mathcal{X}} = \setcond{ f \in \mathsf{RLD} (\Ob \mathcal{X}
   ; Y) }{ \dom f = \mathcal{X} \wedge f\text{ is monovalued} } . \]

Now $\mathcal{Y}^{\mathcal{X}} = \bigsqcap^{\mathsf{RLD}}_{Y \in
\up \mathcal{Y}} Y^{\mathcal{X}}$.

TODO: Check $\mathcal{Y}^1 \cong \mathcal{Y}$; $\mathcal{Z}^{\mathcal{X}
\times^{\mathsf{RLD}} \mathcal{Y}} \cong
(\mathcal{Z}^{\mathcal{X}})^{\mathcal{Y}}$; $\mathcal{Z}^{\mathcal{X} \amalg
\mathcal{Y}} \cong \mathcal{Z}^{\mathcal{X}} \times^{\mathsf{RLD}}
\mathcal{Z}^{\mathcal{Y}}$; $\mathcal{Y}^2 \cong \mathcal{Y}\times^{\mathsf{RLD}}\mathcal{Y}$;
$\mathcal{Y}^0 \cong 1$; $\mathcal{Y}^N \cong \prod^{\mathsf{RLD}}_{n\in N}\mathcal{Y}$.
More formulas at \url{https://en.wikipedia.org/wiki/Cartesian_closed_category}.

Isn't it a cartesian closed category?

Andreas Blass says it is not cartesian closed: ``Unfortunately, the two categories of filters in my paper are
not cartesian closed.  This is mentioned in a parenthetical comment
near the bottom of page 141.  The operation of cartesian product with
the cofinite filter on the natural numbers has no right adjoint,
because it does not preserve infinite coproducts.''
\url{http://matwbn.icm.edu.pl/ksiazki/fm/fm94/fm94115.pdf}

But it is probably a braided closed monoidal category?
\chapter{Matters related to tensor product}

These consideration on (possibly infinite) indexed families of
join-semilattices is based on \cite{nforum-todd-tensor} (for the finite case).

Let $\mathfrak{A}$ be an indexed family of join-semilattices with least
elements. Let $T$ also be a join-semilattice.

Let $F (X)$ mean free join-semilattice for a set $X$.

\begin{defn}
$\mathbf{SepJoin}(\prod\mathfrak{A};T)$ is the set of maps from $\prod\mathfrak{A}$
to~$T$, preserving joins in every argument $i\in\dom\mathfrak{A}$.
\end{defn}

\begin{obvious}
The set of free join-semilattices $F (X)$ is order-isomorphic to the set of
subsets $X$ of a ``universal'' set $\mho$.{\hspace*{\fill}}{\medskip}
\end{obvious}

Let $i : \prod \mathfrak{A} \rightarrow F \left( \prod \mathfrak{A} \right)$
be the universal embedding.

Let $\sim$ be defined as the smallest equivalence relation on $F \left( \prod
\mathfrak{A} \right)$ that for every $k \in \dom \mathfrak{A}$, $L \in
\prod_{i \in (\dom \mathfrak{A}) \setminus \{ k \}} \mathfrak{A}_i$:
\begin{enumerate}
  \item $i (L \cup \{ (k ; g \sqcup h) \}) \sim i (L \cup \{ (k ; g) \})
  \sqcup i (L \cup \{ (k ; h) \})$;
  
  \item $\bot \sim i (L \cup \{ (k ; \bot) \})$;
  
  \item $x \sim y \wedge x' \sim y' \Rightarrow x \sqcup x' \sim y \sqcup y'$
  for all $x, y, x', y' \in F \left( \prod \mathfrak{A} \right)$.
\end{enumerate}

\begin{obvious}
Some function $h : X \rightarrow Y$ induces a well defined map $\psi : X / E
\rightarrow Y$ on equivalence classes, if $E \subseteq F$ where $x \mathrel{F}
y \Leftrightarrow h x = h y$.{\hspace*{\fill}}{\medskip}
\end{obvious}

\begin{lem}
  The set of join-homomorphisms $\psi : F \left( \prod \mathfrak{A} \right) / \sim
  \rightarrow T$ is isomorphic to the set of maps $\phi :
  \prod \mathfrak{A} \rightarrow T$ preserving finite joins in separate arguments.
\end{lem}

\begin{proof}
  The quotient map $q : F \left( \prod \mathfrak{A} \right) \rightarrow F
  \left( \prod \mathfrak{A} \right) / \sim$ which takes an element $x$ to its
  equivalence class $[x]$ map is well defined because
  \[ x \sim y \wedge x' \sim y' \Rightarrow x \sqcup x' \sim y \sqcup y' . \]
  The map $q$ \ preserves join. $F \left( \prod \mathfrak{A} \right) / \sim$
  is associative, commutative, and idempotent since it is so on $F \left(
  \prod \mathfrak{A} \right)$ and thus is a join-semilattice.
  
  Let join-preserving map $\psi : F \left( \prod \mathfrak{A} \right) / \sim
  \rightarrow T$. It is easy to show that $\psi \circ q \circ i$ preserves
  joins in separate arguments.
  
  Let now $\phi : \prod \mathfrak{A} \rightarrow T$ preserves joins in
  separate arguments. There is a unique join-preserving map $\tilde{\phi} : F
  \left( \prod \mathfrak{A} \right) \rightarrow T$ such that $\tilde{\phi}
  \circ i = \phi$. We must show that this induces a well-defined
  join-preserving map $\psi : F \left( \prod \mathfrak{A} \right) / \sim
  \rightarrow T$ such that $\psi (q (x)) = \tilde{\phi} (x)$ for all $x \in F
  \left( \prod \mathfrak{A} \right)$ (clearly at most one function $\psi$ can
  satisfy this equation since $q$ is surjective). This will show that $\psi$
  bijectively correspond to $\tilde{\phi}$ and thus bijectively correspond to
  $\phi$. (This will finish the proof as that this bijection is monotone is
  obvious.)
  
  Using the ``obvious'' above, it's enough (taking into account that $\sim$ is
  the minimal equivalence relation subject to the above formulas) to prove
  that:
  \begin{enumerate}
    \item $\tilde{\phi} (i (L \cup \{ (k ; g \sqcup h) \})) = \tilde{\phi} (i
    (L \cup \{ (k ; g) \}) \sqcup i (L \cup \{ (k ; h) \}))$;
    
    \item $\tilde{\phi} (\bot) = \tilde{\phi} (i (L \cup \{ (k ; \bot) \}))$;
    
    \item $\tilde{\phi} (x) = \tilde{\phi} (y) \wedge \tilde{\phi} (x') =
    \tilde{\phi} (y') \Rightarrow \tilde{\phi} (x \sqcup x') = \tilde{\phi} (y
    \sqcup y')$
  \end{enumerate}
  The first easily follows from $\tilde{\phi} \circ i = \phi$ and the fact
  that $\tilde{\phi}$ preserves binary joins.
  
  The second easily follows from $\tilde{\phi} \circ i = \phi$ and that $\phi$
  preserves $\bot$.
  
  The third follows from the fact that $\tilde{\phi}$ preserves joins.
\end{proof}

\begin{cor}
  The poset of prestaroids
  $\mathsf{preStrd} (\mathfrak{A})$ is isomorphic to an ideal
  (on a join-semilattice), provided that $\mathfrak{A}$ is an indexed family
  of join-semilattices.
\end{cor}

\begin{proof}
  $\mathsf{Strd} (\mathfrak{A}) \cong \mathbf{SepJoin} (\mathfrak{A}; 2)
  \cong F \left( \prod \mathfrak{A} \right) / \sim \rightarrow 2 \cong
  \mathfrak{I} \left( F \left( \prod \mathfrak{A} \right) / \sim \right)$.
\end{proof}

\fxwarning{Check below (especially posets vs dual posets) for errors.}

\begin{cor}
$\mathsf{preStrd}$ is a complete lattice.
\end{cor}

\begin{proof}
Corollary~\bookref{filt-is-complete}.
\end{proof}

\begin{cor}
$\mathsf{preStrd}$ is a filtered filtrator.
\end{cor}

\begin{proof}
Theorem~\bookref{semifilt-joinclosed}.
\end{proof}

\fxnote{Try to prove that $\mathsf{preStrd}$ is atomic and moreover atomistic (under certain conditions). Other properties?}
\chapter{Funcoids as closed sets}

\fxnote{\url{https://ncatlab.org/toddtrimble/published/topogeny}
and \url{https://math.stackexchange.com/q/2681502/4876}}

\fxnote{What about the infinite products?}

\begin{thm}
The set of staroids
$\subsets X_1\times\dots\times\subsets X_n\to 2$ is
order isomorphic to co-frame of closed subsets of topological
product $\beta X_1\times\dots\times\beta X_n$.
\end{thm}

\begin{proof}
$\subsets X_1\times\dots\times\subsets X_n\to 2$ is isomorphic to
the frame of ideals
$\mathfrak{J}(\subsets X_1\times\dots\times\subsets X_n)$ what is
dual (check!) to the frame of ideals of the distributive lattice
$\subsets X_1\otimes\dots\otimes\subsets X_n$.
This by ?? is the coproduct $\sum_i \subsets X_i$ in the category
of boolean algebras.
By Stone duality it is isomorphic to the topology of it spectrum
$\beta X_1\times\dots\times\beta X_n$.
\end{proof}

\chapter{Categories related with funcoids}

I consider some categories related with pointfree funcoids.

\section{Draft status}

This is a rough partial draft.

\section{Topic of this article}

In this article are considered some categories related to \emph{pointfree
funcoids}.

\section{Category of continuous morphisms}

I will denote $\Ob f$ the object (source and destination) of an
endomorphism $f$.

\begin{defn}
  Let $C$ is a partially ordered category. The category
  $\cont (C)$ (which I call \emph{the category of
  continuous morphism} over $C$) is:
  \begin{itemize}
    \item Objects are endomorphisms of category $C$.
    
    \item Morphisms are triples $(f , a , b)$ where $a$ and $b$ are objects
    and $f : \Ob a \rightarrow \Ob b$ is a morphism of the
    category $C$ such that $f \circ a \sqsubseteq b \circ f$.
    
    \item Composition of morphisms is defined by the formula $(g , b , c)
    \circ (f , a , b) = (g \circ f , a , c)$.
    
    \item Identity morphisms are $(a , a , 1^C_a)$.
  \end{itemize}
\end{defn}

It is really a category:

\begin{proof}
  We need to prove that: composition of morphisms is a morphism, composition
  is associative, and identity morphisms can be canceled on the left and on
  the right.
  
  That composition of morphisms is a morphism follows from these implications:
  \[ f \circ a \sqsubseteq b \circ f \wedge g \circ b \sqsubseteq c \circ g
     \Rightarrow g \circ f \circ a \sqsubseteq g \circ b \circ f \sqsubseteq c
     \circ g \circ f. \]
  That composition is associative is obvious.
  
  That identity morphisms can be canceled on the left and on the right is
  obvious.
\end{proof}

\begin{rem}
  The ``physical'' meaning of this category is:
  \begin{itemize}
    \item Objects (endomorphisms of $C$) are spaces.
    
    \item Morphisms are continuous functions between spaces.
    
    \item $f \circ a \sqsubseteq b \circ f$ intuitively means that $f$
    combined with an infinitely small is less than infinitely small combined
    with $f$ (that is $f$ is continuous).
  \end{itemize}
\end{rem}

\begin{rem}
  Every $\Hom (\mathfrak{A}, \mathfrak{B})$ of $\mathbf{Pos}$
  is partially ordered by the formula $a \leqslant b \Leftrightarrow \forall x
  \in \mathfrak{A}: a (x) \leqslant b (x)$. So $\cont
  (\mathbf{Pos})$ is defined.
\end{rem}

\begin{defn}
  I call a $\mathbf{Pos}$-morphism \emph{monovalued} when it maps
  atoms to atoms or least element.
\end{defn}

\begin{defn}
  I call a $\mathbf{Pos}$-morphism \emph{entirely defined} when
  its value is non-least on every non-least element.
\end{defn}

\begin{obvious}
A morphism is both monovalued and entirely defined iff it maps atoms into
atoms.
\end{obvious}

\fxnote{Show how it relates with dagger categories.}

\begin{defn}
  $\mathbf{mePos}$ is the subcategory of $\mathbf{Pos}$ with
  only monovalued and entirely defined morphisms.
\end{defn}

\begin{obvious}
This is a well defined category.{\hspace*{\fill}}{\medskip}
\end{obvious}

\begin{defn}
  $\mathbf{mefp} \mathsf{FCD}$ is the subcategory of
  $\mathbf{fp} \mathsf{FCD}$ with only monovalued and entirely
  defined morphisms.
\end{defn}

\begin{rem}
  In the two above definitions different definitions of monovaluedness and
  entire definedness from different articles.
\end{rem}

\section{Definition of the categories}

\begin{defn}
  A \emph{(pointfree) endo-funcoid} is a (pointfree) funcoid with the same
  source and destination (an endomorphism of the category of (pointfree)
  funcoids). I will denote $\Ob f$ the object of an endomorphism $f$.
\end{defn}

\begin{obvious}
The \emph{category of continuous pointfree funcoids} $\cont
(\mathbf{fp} \mathsf{FCD})$ is:
\begin{itemize}
  \item Objects are small pointfree endo-funcoids.
  
  \item Morphisms from an object $a$ to an object $b$ are triples $(f , a ,
  b)$ where $f$ is a pointfree funcoid from $\Ob a$ to $\Ob b$
  such that $f$ is a continuous morphism from $a$ to $b$ (that is $f \circ a
  \sqsubseteq b \circ f$, or equivalently $a \sqsubseteq f^{- 1} \circ b \circ
  f$, or equivalently $f \circ a \circ f^{- 1} \sqsubseteq f$).
  
  \item Composition is the composition of pointfree funcoids.
  
  \item Identity for an object $a$ is $(I^{\mathsf{FCD}}_{\Ob a}
  , a , a)$.
\end{itemize}
\end{obvious}

\section{Isomorphisms}

\begin{thm}
  If $f$ is an isomorphism $a \rightarrow b$ of the category
  $\cont (\mathbf{fp}
  \mathsf{FCD})$, then:
  \begin{enumerate}
    \item $f \circ a = b \circ f$;
    
    \item $a = f^{- 1} \circ b \circ f$;
    
    \item $f \circ a \circ f^{- 1} = b$.
  \end{enumerate}
\end{thm}

\begin{proof}
  Note that $f$ is monovalued and entirely defined.
  
  1. We have $f \circ a \sqsubseteq b \circ f$ and $f^{- 1} \circ b
  \sqsubseteq a \circ f^{- 1}$. Consequently $f^{- 1} \circ f \circ a
  \sqsubseteq f^{- 1} \circ b \circ f$; $a \sqsubseteq f^{- 1} \circ b \circ
  f$; $a \circ f^{- 1} \sqsubseteq f^{- 1} \circ b \circ f \circ f^{- 1}$; $a
  \circ f^{- 1} \sqsubseteq f^{- 1} \circ b$. Similarly $b \circ f \sqsubseteq
  f \circ a$. So $f \circ a = b \circ f$.
  
  2 and 3. Follow from the definition of isomorphism.
\end{proof}

Isomorphisms are meant to preserve structure of objects. I will show that
(under certain conditions) isomorphisms of $\cont
(\mathbf{fp} \mathsf{FCD})$ really preserve
structure of objects.

First we will consider an isomorphism between objects $a$ and $b$ which are
funcoids (not the general case of pointfree funcoids). In this case a map
which preserves structure of objects is a \emph{bijection}. It is really a
bijection as the following theorem says:

\begin{thm}
If $f$ is an isomorphism of the category of funcoids then $f$ is a discrete
funcoid (so, it is essentially a bijection).
\fxnote{Split it into two propositions: about completeness and co-completeness.}
\end{thm}

\begin{proof}
  $\supfun{f}^{\ast} A \sqcap \supfun{f}^{\ast} ((\Src f)
  \setminus A) = 0^{\Dst f}$ because $f$ is monovalued.
  
  $\supfun{f}^{\ast} A \sqcup \supfun{f}^{\ast} ((\Src f)
  \setminus A) = 1^{\Dst f}$.
  
  Therefore $\supfun{f}^{\ast} A$ is a principal filter (theorem 49 in
  {\cite{filters}}). So $f$ is co-complete.
  
  That $f$ is complete follows from symmetry.
\end{proof}

For wider class of pointfree funcoids the concept of bijection does not make
sense. Instead we would want a structure preserving map to be \emph{order
isomorphism}.

Actually, for mapping between $\subsets A$ and $\subsets B$ where $A$
and $B$ are some sets (including the above considered case of funcoids from
$A$ to $B$) bijection and order isomorphism are essentially the same:

\begin{prop}
  Bijections $F$ between sets $A$ and $B$ bijectively correspond to order
  isomorphisms $f$ between $\subsets A$ and $\subsets B$ by the formula
  $f = \supfun{F}$.
\end{prop}

\begin{proof}
  Let $F$ is a bijection. Then $X \subseteq Y \Rightarrow \supfun{F} X
  \subseteq \supfun{F} Y$ and $\langle F^{- 1} \rangle \langle F
  \rangle X = X$ for every sets $X, Y \in \subsets A$. Thus $f = \langle F
  \rangle$ is an order isomorphism.
  
  Let now $f$ is an order isomorphism between $\subsets A$ and $\subsets
  B$. Then $f (\{ x \})$ is a singleton for every $x \in A$. Take $F (x)$ to
  the unique $y$ such that $f (\{ x \}) = \{ y \}$. Obviously $f$ is a
  bijection and $f = \supfun{F}$.
\end{proof}

For arbitrary pointfree funcoids isomorphisms do not necessarily preserve
structure. It holds only for \emph{increasing pointfree funcoids}:

\begin{defn}
  I call a pointfree funcoid $f$ \emph{increasing} iff $\supfun{f}$
  and $\langle f^{- 1} \rangle$ are monotone functions.
\end{defn}

\begin{prop}
  If $f$ is an increasing isomorphism of the category of pointfree funcoids
  then $\supfun{f}$ is an order isomorphism.
\end{prop}

\begin{proof}
  We have: $\supfun{f} \circ \langle f^{- 1} \rangle = \langle f \circ
  f^{- 1} \rangle = \langle \id^{\mathsf{FCD}}_{\mathfrak{B}}
  \rangle = \id_{\mathfrak{B}}$ and $\langle f^{- 1} \rangle \circ
  \supfun{f} = \langle f^{- 1} \circ f \rangle = \langle
  \id^{\mathsf{FCD}}_{\mathfrak{A}} \rangle =
  \id_{\mathfrak{A}}$. Thus $\supfun{f}$ is a bijection.
  
  $\supfun{f}$ is increasing and bijective.
\end{proof}

\begin{rem}
  Non-increasing isomorphisms of the category of pointfree funcoids are
  against sound mind, they don't preserve the structure of the source, that is
  for them $\supfun{f}$ or $\langle f^{- 1} \rangle$ are not order
  isomorphisms.
\end{rem}

\begin{obvious}
Isomorphisms of $\cont (\mathbf{Pos})$ and
$\cont (\mathbf{mePos})$ are order
isomorphisms.
\end{obvious}

\section{Direct products}

\fxerror{Now this section is a complete mess. Clean it up.}

Consider the category $\mathbf{contFcd}$ which is the full
subcategory $\cont (\mathbf{mePos})$ restricted to
objects which are essentially increasing pointfree funcoids.

Let $f_1 : Y \rightarrow X_1$ and $f_2 : Y \rightarrow X_2$ are morphisms of
$\mathbf{contFcd}$.

The product object is $X_1 \times^{(C)} X_2$ (cross composition product of
funcoids used). It is easy to see that $X_1 \times^{(C)} X_2$ is an object of
$\mathbf{contFcd}$ that is an endo-funcoid.

The morphism $f_1 \times^{(D)} f_2 : Y \rightarrow X_1 \times^{(C)} X_2$ is
defined by the formula $(f_1 \times^{(D)} f_2) y = f_1 y
\times^{\mathsf{FCD}} f_2 y$.

$f_1 \times^{(D)} f_2$ is monovalued and entirely defined because so are $f_1$
and $f_2$.
\[ (f_1 \times^{(D 2)} f_2) y = \bigcup \left\{ f_1 Y
   \times^{\mathsf{FCD}} f_2 Y \hspace{1em} | \hspace{1em} Y \in
   \atoms^{\mathfrak{A}} y \right\} . \]

\fxnote{Is $(f_1 \times^{(D 2)} f_2)$ a pointfree funcoid?}

To prove that it is really a morphism we need to show
\[ (f_1 \times^{(D)} f_2) \circ Y \sqsubseteq (X_1 \times^{(C)} X_2) \circ
   (f_1 \times^{(D)} f_2) \]
that is (for every $y$)
\[ (f_1 \times^{(D)} f_2) Y y \sqsubseteq (X_1 \times^{(C)} X_2) (f_1
   \times^{(D)} f_2) y. \]
Really, $(f_1 \times^{(D)} f_2) Y y = f_1 Y y \times^{\mathsf{FCD}} f_2
Y y$;

$(X_1 \times^{(C)} X_2) (f_1 \times^{(D)} f_2) y = (X_1 \times^{(C)} X_2) (f_1
y \times^{\mathsf{FCD}} f_2 y) = X_1 f_1 y \times^{\mathsf{FCD}}
X_2 f_2 y$;

but it is easy to show $f_1 Y y \times^{\mathsf{FCD}} f_2 Y y
\sqsubseteq X_1 f_1 y \times^{\mathsf{FCD}} X_2 f_2 y$.

??

I define ??

\fxnote{Prove that it is a direct product in $\mathbf{contFcd}$.}
\chapter{Product of funcoids over a filter}

The following definition is inspired by the usual definition of Tychonoff
product of topological spaces.

\begin{defn}
  Let $f$ be an indexed family of funcoids. Let $\mathcal{F}$ be a filter on
  $\dom f$.
  \[ a \mathrel{\left[ \prod^{[\mathcal{F}]} f \right]} b \Leftrightarrow
     \exists N \in \mathcal{F} \forall i \in N : \Pr^{\mathsf{RLD}}_i a
     \suprel{f_i} \Pr^{\mathsf{RLD}}_i b \]
  for atomic reloids $a$ and $b$.
\end{defn}

\begin{rem}
  We are especially interested in the special case when $\mathcal{F}$ is the
  cofinite filter. In this case $a \mathrel{\left[ \prod^{[\mathcal{F}]} f
  \right]} b$ is defined by the condition that $\Pr^{\mathsf{RLD}}_i a
  \suprel{f_i} \Pr^{\mathsf{RLD}}_i $ for an infinite number of
  indexes $i$.
\end{rem}

\begin{obvious}
$a \mathrel{\left[ \prod^{\suprel{ \top^{\mathscr{F} (\dom f)} }} f
\right]} b \Leftrightarrow a \mathrel{\left[ \prod^{(A)} f \right]} b$.
\end{obvious}

\begin{prop}
  $\neg \left( \mathcal{X} \suprel{f} \mathcal{Y} \right)$ implies $\neg
  \left( X \suprel{f} Y \right)$ for some $X \in \up \mathcal{X}$, $Y
  \in \up \mathcal{Y}$.
\end{prop}

\begin{proof}
  Suppose $\neg \left( \mathcal{X} \suprel{f} \mathcal{Y} \right)$. Then
  $\mathcal{Y} \asymp \supfun{f} \mathcal{X}$. Thus by separability of
  core for filters $Y \asymp \supfun{f} \mathcal{X}$ for some $Y \in
  \up \mathcal{Y}$, that is $\neg \left( \mathcal{X} \suprel{f} Y
  \right)$. Apply this result twice.
\end{proof}

\begin{lem}
  ~  
  \[ \forall X \in \prod_{i \in D} \up a_i, Y \in \prod_{i \in D}
     \up b_i \exists x \in \prod_{i \in D} \mathrm{atoms} \uparrow X_i,
     y \in \prod_{i \in D} \mathrm{atoms} \uparrow Y_i \exists N \in
     \mathcal{F} \forall j \in N : x_j \mathrel{[f_j]} y_j \]
  implies $\exists N \in \mathcal{F} \forall i \in N : a_i \suprel{f_i}
  b_i$.
\end{lem}

\begin{proof}
  Suppose for the contrary $\neg \left( a_i \suprel{f_i} b_i \right)$ for
  all $i \in N$ where $N \in \mathcal{F}$ (i.e. for an infinite number of
  indexes if $\mathcal{F}$ is the cofinite filter). Then (lemma above) there
  are $X_i \in \up a_i$ and $Y_i \in \up b_i$ such that $\neg
  \left( X_i \mathrel{[f_j]^{\ast}} Y_i \right)$ for $i \in N$. Thus $\neg
  \left( x_i \suprel{f_i} y_i \right)$ for $i \in N$, contrary to the
  condition.
\end{proof}

\begin{prop}
  The funcoid $\prod^{[\mathcal{F}]} f$ exists.
\end{prop}

\begin{proof}
  We need to prove that
  \[ \forall X \in \up a, Y \in \up b \exists x \in \atoms
     \uparrow^{\mathsf{RLD}} X, y \in \atoms
     \uparrow^{\mathsf{RLD}} Y : x \mathrel{\left[ \prod^{(A 2)} f
     \right]} y \]
  implies $a \mathrel{\left[ \prod^{[\mathcal{F}]} f \right]} b$.
  
  Equivalently transforming it: \fxwarning{More detailed proof.}
  
  $\forall X \in \up a, Y \in \up b \exists x \in \atoms
  \uparrow^{\mathsf{RLD}} X, y \in \atoms
  \uparrow^{\mathsf{RLD}} Y \\  
  \exists N \in \mathcal{F} \forall i \in N : \Pr^{\mathsf{RLD}}_i x
  \suprel{f_i} \Pr^{\mathsf{RLD}}_i y$; \\
  $\forall X \in \up a, Y \in \up b \exists x \in \prod_{i \in
  \dom f} \atoms \uparrow^{\mathsf{RLD}} X_i, y \in
  \prod_{i \in \dom f} \atoms \uparrow^{\mathsf{RLD}} Y_i \\  
  \exists N \in \mathcal{F} \forall i \in N : x_i \suprel{f_i} y_i$;

  \[ \forall X \in \prod_{i \in D} \up a_i, Y \in \prod_{i \in D}
     \up b_i \exists x \in \prod_{i \in D} \mathrm{atoms} \uparrow X_i,
     y \in \prod_{i \in D} \mathrm{atoms} \uparrow Y_i \exists N \in
     \mathcal{F} \forall j \in N : x_j \mathrel{[f_j]} y_j \]
  where $D = \dom f$.
  
  Thus by the lemma $\exists N \in \mathcal{F} \forall i \in N : a_i
  \suprel{f_i} b_i$, that is $a \suprel{\prod^{[\mathcal{F}]} f} b$.
\end{proof}

\fxnote{TODO: when $\Pr_j \prod^{[\mathcal{F}]}_{i\in D} a_i = a_j$?}
\chapter{Compact funcoids}

Compact funcoids are defined. Under certain conditions it's proved that the
reloid corresponding to a compact funcoid is the neighborhood of the diagonal
of the product funcoid.

This is a rough partial draft. The proofs are with errors.

\section{The rest}

\begin{defn}
  A funcoid $f$ is \emph{directly compact} iff
  \[ \text{$\forall \mathcal{F} \in \mathfrak{F}: (\supfun{f}
     \mathcal{F} \neq \bot \Rightarrow \Cor \supfun{f} \mathcal{F}
     \neq \bot)$.} \]
\end{defn}

\begin{obvious}
A funcoid $f$ is directly compact iff $\forall a \in \atoms \dom f :
\Cor \supfun{f} a \neq \bot$.
\end{obvious}

\begin{defn}
  A funcoid $f$ is \emph{reversely compact} iff $f^{- 1}$ is directly
  compact.
\end{defn}

\begin{defn}
  A funcoid is \emph{compact} iff it is both directly compact and reversely
  compact.
\end{defn}

\begin{prop}
  $\prod^{\mathsf{RLD}} a = \uparrow^{\mathsf{RLD}} \prod_{i \in
  \dom a} (\uparrow^{\mathsf{RLD}})^{- 1} a_i$ for every indexed
  family $a$ of principal filters.
\end{prop}

\begin{proof}
Because $\prod_{i \in \dom a} (\uparrow^{\mathsf{RLD}})^{- 1} a_i
\in \GR \prod^{\mathsf{RLD}} a$.
\fxwarning{More detailed proof.}
\end{proof}

\begin{lem}
$\prod^{\mathsf{RLD}}_{i \in \dom a} \Cor a_i = \Cor
\prod^{\mathsf{RLD}} a$.
\end{lem}

\begin{proof}
$\Cor \prod^{\mathsf{RLD}} a = \bigsqcap \left\{
\uparrow^{\mathsf{RLD}} \prod A \hspace{1em} | \hspace{1em} A \in
\up a \right\} = \uparrow^{\mathsf{RLD}} \bigcap \left\{ \prod A
\hspace{1em} | \hspace{1em} A \in \up a \right\} =
\uparrow^{\mathsf{RLD}} \bigcap \left\{ \prod A \hspace{1em} |
\hspace{1em} A \in \subsets \prod \mathfrak{U}, \forall i \in \dom a
: A_i \in \up a_i \right\} = \uparrow^{\mathsf{RLD}} \bigcap
\left\{ \prod \bigcap K_i \hspace{1em} | \hspace{1em} K \in \subsets
\subsets \prod \mathfrak{U}, \forall i \in \dom a : K_i \in
\subsets \up a_i \right\} = \uparrow^{\mathsf{RLD}} \bigcap
\left\{ \prod (\uparrow^{\mathsf{RLD}})^{- 1} \Cor a_i
\hspace{1em} | \hspace{1em} i \in \dom a \right\} =
\uparrow^{\mathsf{RLD}} \prod_{i \in \dom
a}^{\mathsf{RLD}} \Cor a_i$.

\fxwarning{Check for little errors.}
\end{proof}

\begin{cor}
  $\prod^{\mathsf{RLD}}_{i \in n} \langle \CoCompl f_i \rangle
  \mathcal{X}_i = \left\langle \CoCompl \prod^{(A)} f \right\rangle
  \prod^{\mathsf{RLD}} \mathcal{X}$ for every $n$-indexed families $f$
  of funcoids and $\mathcal{X}$ of filters on the same set (with $\Src
  f_i = \Base (\mathcal{X}_i)$ for every $i \in n$).
\end{cor}

\begin{proof}
  ~
  \begin{eqnarray*}
    \prod^{\mathsf{RLD}}_{i \in n} \langle \CoCompl f_i \rangle
    \mathcal{X}_i & = & \\
    \prod^{\mathsf{RLD}}_{i \in n} \Cor \langle f_i \rangle
    \mathcal{X}_i & = & \\
    \Cor \prod^{\mathsf{RLD}}_{i \in n} \langle f_i \rangle 
    \mathcal{X}_i & = & \text{(*)}\\
    \Cor \prod^{\mathsf{RLD}}_{i \in n} \langle f_i \rangle
    \Pr^{\mathsf{RLD}}_i \left( \prod^{\mathsf{RLD}} \mathcal{X}
    \right) & = & \\
    \Cor \left\langle \prod^{(A)} f \right\rangle
    \prod^{\mathsf{RLD}} \mathcal{X} & = & \\
    \left\langle \CoCompl \prod^{(A)} f \right\rangle
    \prod^{\mathsf{RLD}} \mathcal{X} . &  & 
  \end{eqnarray*}
  (*) You should verify the special case when $\mathcal{X}_i =
  \bot^{\mathfrak{F}}$ for some $i$.
\end{proof}

\begin{thm}
Let $f$ be an indexed family of funcoids. \fxnote{Reverse theorem (for non-least funcoids).}

\begin{enumerate}
  \item $\prod f$ is directly compact if every $f_i$ is directly compact.
  
  \item $\prod f$ is reversely compact if every $f_i$ is reversely compact.
  
  \item $\prod f$ is compact if every $f_i$ is compact.
\end{enumerate}
\end{thm}

\begin{proof}
  It is enough to prove only the first statement.
  
  Let each $f_i$ is directly compact.
  
  Let $\left\langle \prod f \right\rangle a \neq \bot$. Then $\left\langle \prod
  f \right\rangle a = \left\langle \prod^{(A)} f \right\rangle a =
  \prod^{\mathsf{RLD}}_{i \in \dom f} \langle f_i \rangle
  \Pr^{\mathsf{RLD}}_i a$. Thus every $\langle f_i \rangle
  \Pr^{\mathsf{RLD}}_i a \neq \bot$. Consequently by compactness
  $\Cor \langle f_i \rangle \Pr^{\mathsf{RLD}}_i a \neq \bot$;
  $\prod_{i \in \dom f} \Cor \langle f_i \rangle
  \Pr^{\mathsf{RLD}}_i a \neq \bot$; $\Cor \prod_{i \in \dom
  f} \langle f_i \rangle \Pr^{\mathsf{RLD}}_i a \neq \bot$; $\Cor
  \left\langle \prod f \right\rangle a \neq \bot$.
  
  So $\prod f$ is directly compact.
\end{proof}

\begin{prop}
  The following expressions are pairwise equal:
  \begin{enumerate}
    \item\label{ff-id-s} $\langle f \times^{(A)} f \rangle^{\ast} 1^{\mathsf{RLD}}$;
    
    \item\label{ff-id-at} $\bigsqcup \setcond{ \langle f \times^{(A)} f \rangle p }
    {p \in \atoms 1^{\mathsf{RLD}} }$;
    
    \item\label{ff-id-p} $\bigsqcup \setcond{ \supfun{f} x \times^{\mathsf{RLD}}
    \supfun{f} x }{ x \in \atoms^{\mathscr{F}} }$;
  \end{enumerate}
\end{prop}

\begin{proof}
~
\begin{widedisorder}
\item[\ref{ff-id-s}$\Leftrightarrow$\ref{ff-id-at}] Theorem~\bookref{fcd-atoms}.

\item[\ref{ff-id-at}$\Leftrightarrow$\ref{ff-id-p}]
$\bigsqcup \setcond{ \langle f \times^{(A)} f \rangle p }{p \in \atoms 1^{\mathsf{RLD}} } =
\bigsqcup \setcond{ \supfun{f}\dom p \times^{\mathsf{RLD}} \supfun{f}\im p }{p \in \atoms 1^{\mathsf{RLD}} } =
\bigsqcup \setcond{ \supfun{f} x \times^{\mathsf{RLD}} \supfun{f} x }{ x \in \atoms^{\mathscr{F}} }$.
\end{widedisorder}
\end{proof}

\begin{prop}
  Let $g$ be a reloid and $f = \tofcd g$. Then $\langle f
  \times f \rangle^{\ast} 1^{\mathsf{RLD}} \sqsupseteq g$.
\end{prop}

\begin{proof}
  $\langle f \times f \rangle^{\ast} 1^{\mathsf{RLD}} \nasymp
  \uparrow^{\mathsf{RLD}} Y \Leftrightarrow
  \uparrow^{\mathsf{RLD}} 1^{\mathsf{RLD}} \mathrel{[f \times f]}
  \uparrow^{\mathsf{RLD}} Y \Leftrightarrow
  \uparrow^{\mathsf{FCD}} 1^{\mathsf{RLD}} \mathrel{[f \times^{(C)} f]}
  \uparrow^{\mathsf{FCD}} Y \Leftrightarrow f \circ
  \uparrow^{\mathsf{FCD}} 1^{\mathsf{RLD}} \circ f^{- 1} \nasymp
  \uparrow^{\mathsf{FCD}} Y \Leftrightarrow f \circ f^{- 1} \nasymp
  \uparrow^{\mathsf{FCD}} Y \Leftrightarrow f \nasymp
  \uparrow^{\mathsf{FCD}} Y \Leftrightarrow f \sqcap
  \uparrow^{\mathsf{FCD}} Y \neq \bot \Leftarrow
  \torldin (f \sqcap \uparrow^{\mathsf{FCD}}
  Y) \neq \bot \Leftrightarrow \torldin f \sqcap
  \torldin \uparrow^{\mathsf{FCD}} Y \neq \bot
  \Leftarrow \torldin f \sqcap
  \torldout \uparrow^{\mathsf{FCD}} Y \neq \bot
  \Leftrightarrow \torldin f \sqcap
  \uparrow^{\mathsf{RLD}} Y \neq \bot \Leftrightarrow
  \torldin  \tofcd g \sqcap
  \uparrow^{\mathsf{RLD}} Y \neq \bot \Leftarrow g \sqcap
  \uparrow^{\mathsf{RLD}} Y \neq \bot \Leftrightarrow g \nasymp
  \uparrow^{\mathsf{RLD}} Y$.
\end{proof}

\begin{prop}
  Let $f$ be a funcoid. Then $V \circ M \circ V^{- 1} \in \GR \langle f
  \times f \rangle^{\ast} M$ for every $V \in \GR f$.
\end{prop}

\begin{proof}
  $V \circ M \circ V^{- 1} \in \GR (f \circ \uparrow M \circ f^{- 1}) =
  \GR \langle f \times^{(C)} f \rangle \uparrow M \supseteq \GR
  \langle f \times f \rangle \uparrow M = \GR \langle f \times f
  \rangle^{\ast} M$.
  
  Because
  
  $\uparrow^{\mathsf{FCD}} X \nasymp \langle f \times^{(C)} f \rangle
  \uparrow^{\mathsf{FCD}} M \Leftrightarrow
  \uparrow^{\mathsf{RLD}} X \nasymp \langle f \times f \rangle
  \uparrow^{\mathsf{RLD}} M \Leftrightarrow \tofcd
  (\uparrow^{\mathsf{RLD}} X \sqcap \langle f \times f \rangle
  \uparrow^{\mathsf{RLD}} M) \neq \bot \Rightarrow \tofcd
  \uparrow^{\mathsf{RLD}} X \sqcap \tofcd \langle f
  \times f \rangle \uparrow^{\mathsf{RLD}} M \neq \bot \Leftrightarrow
  \tofcd \uparrow^{\mathsf{RLD}} X \nasymp
  \tofcd \langle f \times f \rangle
  \uparrow^{\mathsf{RLD}} M \Leftrightarrow
  \uparrow^{\mathsf{FCD}} X \nasymp \tofcd \langle f
  \times f \rangle \uparrow^{\mathsf{RLD}} M$;
  
  $\langle f \times^{(C)} f \rangle \uparrow^{\mathsf{FCD}} M
  \sqsubseteq \tofcd \langle f \times f \rangle
  \uparrow^{\mathsf{RLD}} M$
  
  $\GR \langle f \times^{(C)} f \rangle \uparrow^{\mathsf{FCD}} M
  \supseteq \GR \tofcd \langle f \times f \rangle
  \uparrow^{\mathsf{RLD}} M \supseteq \GR \langle f \times f
  \rangle \uparrow^{\mathsf{RLD}} M$
\end{proof}

\begin{prop}
  $\langle f \times f \rangle^{\ast} M \sqsubseteq g \circ
  \uparrow^{\mathsf{RLD}} M \circ g^{- 1}$ whenever
  $\tofcd g = f$ for a reloid $g$.
\end{prop}

\begin{proof}
  For every $V \in \GR g$ we have $V \circ M \circ V^{- 1} \in \GR
  \langle f \times f \rangle^{\ast} M$. Thus $g \circ
  \uparrow^{\mathsf{RLD}} M \circ g^{- 1} = \bigsqcap \left\{ V \circ M
  \circ V^{- 1} \hspace{1em} | \hspace{1em} V \in \GR g \right\}
  \sqsupseteq \bigsqcap \GR \langle f \times f \rangle^{\ast} M =
  \GR \langle f \times f \rangle^{\ast} M$.
\end{proof}

\begin{cor}
  $\langle f \times f \rangle^{\ast} M \sqsubseteq \langle f \times^{(C)} f
  \rangle^{\ast} M$.
\end{cor}

\begin{cor}
  $V \circ V^{- 1} \in \GR \langle f \times f \rangle^{\ast} 1^{\mathsf{RLD}}$; $f
  \circ f^{- 1} \sqsupseteq \langle f \times f \rangle^{\ast} 1^{\mathsf{RLD}}$.
\end{cor}

\begin{proof}
  ??
\end{proof}

\begin{lem}
  $\Cor \langle f \times f \rangle^{\ast} g \sqsubseteq 1^{\mathsf{RLD}}$ if
  $\tofcd g = f$ where $\tofcd g = f$ for a
  $T_1$-separable reloid $g$.
\end{lem}

\begin{proof}
  ??
\end{proof}

\begin{rem}
  I attempted to generalize the below theorem more than the standard general
  topology theorem about correspondence of compact and uniform spaces, but
  haven't really succeeded much, as it appears to be needed that the reloid in
  question is reflexive, symmetric, and transitive, that is just a uniform
  space as in the standard general topology.
\end{rem}

\begin{thm}
  Let $f$ be a $T_1$-separable compact reflexive symmetric funcoid and $g$ be
  a reloid such that
  \begin{enumerate}
    \item $\tofcd g = f$;
    
    \item $g \circ g^{- 1} \sqsubseteq g$.
  \end{enumerate}
  Then $g = \langle f \times^{(A)} f \rangle^{\ast} 1^{\mathsf{RLD}}$.
\end{thm}

\begin{proof}
Prove $\langle f \times^{(A)} f \rangle^{\ast} 1^{\mathsf{RLD}} \sqsubseteq g \circ
g^{- 1} \sqsubseteq g$:
\begin{multline*}
\rsupfun{f \times^{(A)} } 1^{\mathsf{RLD}} = \bigsqcup
\setcond{ \supfun{f} x \times^{\mathsf{RLD}} \supfun{f} x
}{ x \in \atoms^{\mathscr{F}} } = \\
\bigsqcup \setcond{ \torldin (\supfun{f} x
\times^{\mathsf{FCD}} \supfun{f} x) }{
x \in \atoms^{\mathscr{F}} } \sqsubseteq
\torldin  \bigsqcup \setcond{ \supfun{f} x
\times^{\mathsf{FCD}} \supfun{f} x }{ x
\in \atoms^{\mathscr{F}} } = \\
\torldin 
\bigsqcup \setcond{ \langle f \times^{(C)} f \rangle (x
\times^{\mathsf{FCD}} x) }{ x \in
\atoms^{\mathscr{F}} } =
\torldin \bigsqcup \setcond{ f \circ (x \times^{\mathsf{FCD}} x) \circ f^{- 1}
}{ x \in \atoms^{\mathscr{F}} } \sqsubseteq \\
\torldin  (f \circ f^{- 1}) =
\torldin f \circ \torldin
f^{- 1} = g \circ g^{- 1}.
\end{multline*}

It remains to prove $g \sqsubseteq \langle f \times^{(A)} f \rangle^{\ast}
1^{\mathsf{RLD}}$.

\fxwarning{Possible errors.}

Suppose there is $U \in \GR \langle f \times^{(A)} f \rangle^{\ast} 1^{\mathsf{RLD}}$
such that $U \notin \GR g$.

Then $\setcond{ V \setminus U }{ V \in \GR g } = g \setminus U$ would be a proper filter.

Thus by reflexivity $\langle f \times^{(A)} f \rangle^{\ast} (g \setminus U) \neq
\bot$.

By compactness of $f \times^{(A)} f$, $\Cor \langle f \times^{(A)} f \rangle^{\ast}
(g \setminus U) \neq \bot$.

Suppose $\uparrow \{ (x ; x) \} \sqsubseteq \langle f \times^{(A)} f \rangle^{\ast}
(g \setminus U)$; then $g \setminus U \nasymp \langle f^{- 1} \times^{(A)} f^{- 1}
\rangle \{ (x ; x) \}$; $U \sqsubset \langle f^{- 1} \times^{(A)} f^{- 1} \rangle \{
(x ; x) \} \sqsubseteq \langle f^{- 1} \times^{(A)} f^{- 1} \rangle 1^{\mathsf{RLD}}$ what is
impossible.

Thus there exist $x \neq y$ such that $\{ (x ; y) \} \sqsubseteq \Cor
\langle f \times^{(A)} f \rangle^{\ast} (g \setminus U)$. Thus $\{ (x ; y) \}
\sqsubseteq \langle f \times^{(A)} f \rangle^{\ast} g$.

Thus by the lemma $\{ (x ; y) \} \sqsubseteq 1^{\mathsf{RLD}}$ what is impossible. So $U
\in \GR g$.

We have $\GR \langle f \times^{(A)} f \rangle^{\ast} 1^{\mathsf{RLD}} \subseteq
\GR g$; $\langle f \times^{(A)} f \rangle^{\ast} 1^{\mathsf{RLD}} \sqsupseteq g$.
\end{proof}

\begin{cor}
  Let $f$ is a $T_1$-separable (the same as $T_2$ for symmetric transitive)
  compact funcoid and $g$ is a uniform space (reflexive, symmetric, and
  transitive endoreloid) such that $\tofcd g = f$. Then $g =
  \langle f \times f \rangle^{\ast} 1^{\mathsf{RLD}}$.
\end{cor}

An (incomplete) attempt to prove one more theorem follows:

\begin{thm}
  Let $\mu$ and $\nu$ be uniform spaces, $\tofcd
  \mu$ be a compact funcoid. Then a map $f$ is a continuous map from
  $\tofcd \mu$ to $\tofcd \nu$ iff $f$ is
  a (uniformly) continuous map from $\mu$ to $\nu$.
\end{thm}

\begin{proof}
\fxerror{errors in this proof.}

http://math.stackexchange.com/questions/665202/bourbaki-on-the-fact-that-continuous-function-on-a-compact-is-uniformly-continuo/670956?iemail=1\&noredirect=1\#670956

We have $\mu= \langle \tofcd \mu \times
\tofcd \mu \rangle \uparrow^{\mathsf{RLD}} 1^{\mathsf{RLD}}$

$f \in \mathrm{C}_? (\tofcd \mu; \tofcd
\nu)$. Then
\[ f \times f \in \mathrm{C}_? (\tofcd (\mu \times
   \mu) ; \tofcd (\nu \times \nu)) \]
$(f \times f) \circ \tofcd (\mu \times \mu)
\sqsubseteq \tofcd (\nu \times \nu) \circ (f \times f)$

For every $V \in \GR (\nu \times \nu)$ we have $\langle g^{- 1} \rangle
V \in \langle \tofcd (\mu \times \mu) \rangle
\{ y \}$ for some $y$.

$\langle g^{- 1} \rangle V \in \langle \tofcd \mu \times
\tofcd \mu \rangle \uparrow^{\mathsf{RLD}} 1^{\mathsf{RLD}}
= \GR \mu$

$\supfun{g} \langle g^{- 1} \rangle V \sqsubseteq V$

We need to prove $f \in \mathrm{C} (\mu; \nu)$ that is $\forall p \in
\GR \nu \exists q \in \GR \mu: \supfun{f} q
\sqsubseteq p$. But this follows from the above.
\end{proof}

\fxnote{A space is compact if and only if it is both, complete and totally bounded.}

\url{http://math.stackexchange.com/questions/1101995/non-symmetric-version-of-compact-totally-bounded-complete}
\chapter{Pointfree funcoids as a generalization of frames}

I define an injection from the set of frames to the set of pointfree endo-funcoids.

This article is a rough partial draft of a future longer writing.

\section{Definitions}

\subsection{Pointfree funcoid induced by a co-frame}

Let $\mathfrak{L}$ is a co-frame.

We will define pointfree funcoid $\Uparrow \mathfrak{L}$.

Let $\mathcal{B} (\mathfrak{L})$ is a boolean lattice whose co-subframe
$\mathfrak{L}$ is. (That this mapping exists follows from
{\cite{stone-spaces}}, page 53.) There may be probably more than one such
mapping, but we just choose one $\mathcal{B}$ arbitrarily.

Define $\cl (A) = \bigsqcap \left\{ X \in \mathfrak{L} \hspace{1em} |
\hspace{1em} X \sqsupseteq A \right\}$.

Here $\bigsqcap$ can be taken on either $\mathfrak{L}$ or $\mathcal{B}
(\mathfrak{L})$ as they are the same.

\begin{obvious}
  $\cl \in \mathfrak{L}^{\mathcal{B} (\mathfrak{L})}$.
\end{obvious}

$\cl (A \sqcup B) = \bigsqcap \left\{ X \in \mathfrak{L} \hspace{1em} |
\hspace{1em} X \sqsupseteq A \sqcup B \right\} = \bigsqcap \left\{ X \in
\mathfrak{L} \hspace{1em} | \hspace{1em} X \sqsupseteq A, X \sqsupseteq B
\right\} = \bigsqcap \left\{ X_1 \sqcup X_2 \hspace{1em} | \hspace{1em} X_1
\sqsupseteq A, X_2 \sqsupseteq B \right\} = \bigsqcap \left\{ X_1 \hspace{1em}
| \hspace{1em} X_1 \sqsupseteq A \right\} \sqcup \bigsqcap \left\{ X_2
\hspace{1em} | \hspace{1em} X_2 \sqsupseteq B \right\} = \cl A \sqcup
\cl B$.

$\cl 0 = 0$ is obvious.

Hence we are under conditions of the theorem 14.26 in my book.

So there exists a unique pointfree endo-funcoid $\Uparrow \mathfrak{L} \in
\mathsf{FCD} (\mathfrak{F} (\mathcal{B} (\mathfrak{L})) , \mathfrak{F}
(\mathcal{B} (\mathfrak{L})))$ such that
\[ \langle \Uparrow \mathfrak{L} \rangle \mathcal{X} = \bigsqcap^{\mathfrak{F}
   (\mathcal{B} (\mathfrak{L}))} \langle \cl \rangle
   \up^{\text{$(\mathfrak{F} (\mathcal{B} (\mathfrak{L})) , \mathfrak{P}
   (\mathcal{B} (\mathfrak{L})))$}}  \mathcal{X} \]
for every filter $\mathcal{X} \in \mathfrak{F} (\mathcal{B} (\mathfrak{L}))$.

\subsection{Co-frame induced by a pointfree funcoid}

The co-frame $\Downarrow f$ for some pointfree endo-funcoids $f$ will be
defined to be the reverse of $\Uparrow$. See below for exact meaning of being
reverse.

Let restore the co-frame $\mathfrak{L}$ from the pointfree funcoid $\Uparrow
\mathfrak{L}$.

Let poset $\Downarrow f$ for every pointfree funcoid $f$ is defined by the
formula:
\[ \Downarrow f = \left\{ X \in Z (\Ob f) \hspace{1em} | \hspace{1em}
   \supfun{f} X = X \right\} . \]
\begin{rem}
  It seems that $\Downarrow$ is \emph{not} a monovalued function from
  $\mathsf{pFCD}$ to $\Ob (\mathbf{Frm})$.
\end{rem}

\subsection{Isomorphism of co-frames through pointfree funcoids}

\begin{rem}
  $\mathfrak{P} (\mathcal{B} (\mathfrak{L})) = Z (\mathfrak{F} (\mathcal{B}
  (\mathfrak{L})))$ (theorem 4.137 in {\cite{volume-1}}).
\end{rem}

\begin{thm}
  $\mathfrak{L} \mapsto \Downarrow \Uparrow \mathfrak{L}$ (where
  $\mathfrak{L}$ ranges all small frames) is an order isomorphism.
\end{thm}

\begin{proof}
  Let $A' \in \Downarrow \Uparrow \mathfrak{L}$. Then there exists $A \in
  \mathcal{B} (\mathfrak{L})$ such that $A' = \uparrow^{\mathcal{B}
  (\mathfrak{L})} A$.
  
  $\supfun{f} A' = \uparrow^{\mathcal{B} (\mathfrak{L})} \cl A$.
  
  $\supfun{f} A' = A'$ that is $\uparrow^{\mathcal{B} (\mathfrak{L})}
  \cl A = A' = \uparrow^{\mathcal{B} (\mathfrak{L})} A$. So $\cl A
  = A$ and thus $A \in \mathfrak{L}$.
  
  Let now $A \in \mathfrak{L}$. Then take $A' = \uparrow^{\mathcal{B}
  (\mathfrak{L})} A$. We have $\supfun{f} A' = \cl A =
  \uparrow^{\mathcal{B} (\mathfrak{L})} A = A'$. So $A' \in \Downarrow
  \Uparrow \mathfrak{L}$.
  
  We have proved that it is a bijection.
  
  Because $A$ and $A'$ are related by the equation $A' = \uparrow^{\mathcal{B}
  (\mathfrak{L})} A$ it is obvious that this is an order embedding.
\end{proof}

\section{Postface}

Pointfree funcoids are a \textbf{massive} generalization of locales and
frames: They don't only require the lattice of filters to be boolean but these
can be even not lattices of filters at all but just arbitrary posets. I think
a new era in pointfree topology starts.

Much work is yet needed to relate different properties of frames and locales
with corresponding properties of pointfree funcoids.
\chapter{Singularities}

\textbf{Very} rough draft.

\section{Singularities funcoids: some special cases}
We attempt to prove that $\up z$ is closed regarding finite intersections.

For consideration of this, let's consider two special cases (first of which is a specialization of the second).

Let $\mu=\nu$ be the natural proximity on real numbers $\mathbb{R}$.

Let $\Delta$ is the entourage filter of zero.

1. $z=\Delta\times^{\mathsf{FCD}}\Delta$.

2. $z=\nu\circ (\uparrow^{\mathsf{FCD}} f)|_{\Delta}$ for an arbitrary function $f:\mathbb{R}\rightarrow\mathbb{R}$.

(1) is [[also formulated in elementary terms|http://math.stackexchange.com/questions/568513/is-a-set-closed-under-finite-intersections-about-filters]] (without using funcoids).

These two above conjectures are shown to be false by a counter-example in [[this blog post|http://portonmath.wordpress.com/2013/12/18/a-negative-result-on-a-conjecture/]]. It is a discouraging result as it seems from it the plain funcoids can't be used for the multilevel theory of singularities.

\section{Using plain funcoids}
This way if we succeed is the best way to create metasingular numbers because, it (if we succeed) involves just funcoids not some fancy generalization of funcoids.

Approximate definition of "singularity level": //Singularity level// is a transitive, $T_2$-separable endofuncoid.

Now define the funcoid $\nu_{i+1}=\operatorname{SLA}(\nu_i)$:

$\operatorname{Ob}(\nu_{i+1})$ is defined as the set of all generalized limits (having fixed $\mu$, $\nu$, and $G$).

$X \mathrel{[ \nu_{i+1}]^{\ast}} Y \Leftrightarrow \exists z \in \bigcup \operatorname{Ob} \nu \forall K \in \operatorname{up} z \exists x \in \bigcup X, y \in \bigcup Y : x, y\sqsubseteq K$.

The trouble is to prove that the funcoid $\nu_{i+1}$ exists (is really a funcoid).

$\neg(X \mathrel{[ \nu_{i+1}]^{\ast}} \emptyset)$ and $\neg(\emptyset \mathrel{[ \nu_{i+1}]^{\ast}} Y)$ are obvious. We need to prove
$$I\cup J \mathrel{[ \nu_{i+1}]^{\ast}} Y \Leftrightarrow I \mathrel{[ \nu_{i+1}]^{\ast}} Y \vee J \mathrel{[ \nu_{i+1}]^{\ast}} Y$$ and
$$X \mathrel{[ \nu_{i+1}]^{\ast}} I\cup J \Leftrightarrow X \mathrel{[ \nu_{i+1}]^{\ast}} I \vee X \mathrel{[ \nu_{i+1}]^{\ast}} J.$$

Let's attempt to prove the first of the above equations (the second is dual).

$I \cup J \mathrel{[ \operatorname{SLA} ( \nu)]^{\ast}} Y \Leftrightarrow \\ \exists z
\in \bigcup \operatorname{Ob} \nu \forall K \in \operatorname{up} z \exists x \in \bigcup I \cup \bigcup J, y \in \bigcup Y :
x, y \sqsubseteq K \Leftrightarrow \\
\exists z \in \bigcup \operatorname{Ob} \nu \forall K \in \operatorname{up} z : ( \exists x \in \bigcup I \cup
\bigcup J : x \sqsubseteq K \wedge \exists y \in \bigcup Y : y \sqsubseteq K) \Leftrightarrow \\
\exists z \in \bigcup \operatorname{Ob} \nu \forall K \in \operatorname{up} z \exists x \in \bigcup I \cup \bigcup J :
x \sqsubseteq K \wedge \\ \exists z \in \bigcup \operatorname{Ob} \nu \forall K \in \operatorname{up} z
\exists y \in \bigcup Y : y \sqsubseteq K \Leftrightarrow \\
?? \\
\exists z \in \bigcup \operatorname{Ob} \nu : ( \forall K \in \operatorname{up} z \exists x \in \bigcup I : x
\sqsubseteq K \vee \\ \forall K \in \operatorname{up} z \exists x \in \bigcup J : x \sqsubseteq
K) \wedge \exists z \in \bigcup \operatorname{Ob} \nu \forall K \in \operatorname{up} z \exists y \in
\bigcup Y : y \sqsubseteq K \Leftrightarrow \\
( \exists z \in \bigcup \operatorname{Ob} \nu \forall K \in \operatorname{up} z \exists x \in \bigcup I : x
\sqsubseteq K \vee \\ \exists z \in \bigcup \operatorname{Ob} \nu \forall K \in \operatorname{up} z
\exists x \in \bigcup J : x \sqsubseteq K) \wedge \exists z \in \bigcup \operatorname{Ob} \nu \forall
K \in \operatorname{up} z \exists y \in \bigcup Y : y \sqsubseteq K \Leftrightarrow \\
( \exists z \in \bigcup \operatorname{Ob} \nu \forall K \in \operatorname{up} z \exists x \in \bigcup I : x
\sqsubseteq K \wedge \exists z \in \bigcup \operatorname{Ob} \nu \forall K \in \operatorname{up} z
\exists y \in \bigcup Y : y \sqsubseteq K) \vee \\ ( \exists z \in \bigcup \operatorname{Ob} \nu \forall
K \in \operatorname{up} z \exists x \in \bigcup J : x \sqsubseteq K \wedge \exists z \in
\bigcup \operatorname{Ob} \nu \forall K \in \operatorname{up} z \exists y \in \bigcup Y : y \sqsubseteq K)
\Leftrightarrow \\
( \exists z \in \bigcup \operatorname{Ob} \nu : ( \forall K \in \operatorname{up} z \exists x \in \bigcup I :
x \sqsubseteq K \wedge \forall K \in \operatorname{up} z \exists y \in \bigcup Y : y
\sqsubseteq K)) \vee \\ ( \exists z \in \bigcup \operatorname{Ob} \nu : ( \forall K \in \operatorname{up}
z \exists x \in \bigcup J : x \sqsubseteq K \wedge \forall K \in \operatorname{up} z \exists y
\in \bigcup Y : y \sqsubseteq K)) \Leftrightarrow \\
( \exists z \in \bigcup \operatorname{Ob} \nu : ( \forall K \in \operatorname{up} z : ( \exists x \in
\bigcup I : x \sqsubseteq K \wedge \exists y \in \bigcup Y : y \sqsubseteq K))) \vee \\ \exists z
\in \bigcup \operatorname{Ob} \nu \forall K \in \operatorname{up} z : ( \exists x \in \bigcup J : x
\sqsubseteq K \wedge \exists y \in \bigcup Y : y \sqsubseteq K) \Leftrightarrow \\
I \mathrel{[ \operatorname{SLA} ( \nu)]^{\ast}} Y \vee J \mathrel{[ \operatorname{SLA} (
\nu)]^{\ast}} Y$.

To finish the proof we need to fulfill ?? in the above formula. For this it's enough to prove

$\forall K \in \operatorname{up} z \exists x \in \bigcup I\cup \bigcup J : x \sqsubseteq K \Rightarrow \\ \forall K \in \operatorname{up}
z \exists x \in \bigcup I : x \sqsubseteq K \vee \forall K \in \operatorname{up} z \exists x
\in \bigcup J : x \sqsubseteq K$.

If $z=\uparrow Z$ is a principal funcoid, then

$\forall K \in \operatorname{up} z \exists x \in \bigcup I\cup \bigcup J : x \sqsubseteq K \Rightarrow \\ 
\exists x \in \bigcup I\cup \bigcup J : x \sqsubseteq z \Rightarrow \\
\exists x \in \bigcup I : x \sqsubseteq z \vee \exists x \in \bigcup J : x \sqsubseteq z \Rightarrow \\
\forall K \in \operatorname{up}
z \exists x \in \bigcup I : x \sqsubseteq K \vee \forall K \in \operatorname{up} z \exists x
\in \bigcup J : x \sqsubseteq K$.

Following the idea of [[the proof in this math.stackexchange.com question|http://math.stackexchange.com/questions/562908/an-implication-involving-filters\#562974]] it is easy to show that our implication is true if $\operatorname{up} z$ is closed regarding finite meets. See [[this page|Singularities funcoids: some special cases]] for attempts to set it true.
The question is whether our statement holds for non-principal funcoids. Or is there a counterexampe?

\section{Singularities funcoids: special cases proof attempts}
To prove that $\operatorname{GR} ( \Delta \times^{\mathsf{FCD}} \Delta)$ is closed under finite intersections, it's enough to prove that for every $f \in \operatorname{GR} ( \Delta \times^{\mathsf{FCD}} \Delta)$ there is a positive $\varepsilon$ such that $\forall x \in ( - \varepsilon ; \varepsilon) : f x \in \Delta$.

Really, under this assumption:

For $g \in \operatorname{GR} ( \Delta \times^{\mathsf{FCD}} \Delta)$ exists $\zeta > 0$ such that $\forall x \in ( - \zeta ; \zeta) : g x \in \Delta$. Let $\eta = \min \{ \varepsilon, \zeta \}$. So $\forall x \in ( - \eta ; \eta) : ( \langle f \rangle x \in \Delta \wedge \langle g \rangle x \in \Delta)$ and so $\forall x \in ( - \eta ; \eta) : \langle f \cap g \rangle x \in \Delta$ that is $\forall x \in ( - \eta ; \eta) : \langle \uparrow^{\mathsf{FCD}} ( f \cap g) \rangle^{\ast} \{ x \} \sqsupseteq \Delta$ and consequently $f \cap g \in \operatorname{GR} ( \Delta \times^{\mathsf{FCD}} \Delta)$.


TODO: not yet written

% \printindex{}

\bibliographystyle{plain}
\bibliography{refs}

\end{document}
