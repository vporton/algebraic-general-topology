\chapter{Mappings between endofuncoids and topological spaces}

Let's study mappings between topological spaces and endofuncoids.

\begin{defn}
  Let $t$ be a topology.
  \begin{enumerate}
    \item $F^{\fullstar} t = \bigsqcup_{x \in \Ob t} \left( \{ x \} \times
    \bigsqcap^{\mathscr{F}} \setcond{ E \in t }{ x \in
    E } \right)$;
    
    \item $(F_{\fullstar} t) E = \bigcap \setcond{ D \in t }{ E \subseteq D }$.
  \end{enumerate}
\end{defn}

\begin{prop}
  Let $t$ be a topology.
  \begin{enumerate}
    \item $F^{\fullstar} t$ is complete, reflexive, transitive funcoid.
    
    \item $F_{\fullstar} t$ is co-complete, reflexive, transitive funcoid.
    
    \item $F^{\fullstar}$ and $F_{\fullstar}$ are injections.
    
    \item $F_{\fullstar} t = (F^{\fullstar} t)^{- 1}$.
  \end{enumerate}
\end{prop}

\begin{proof}
  By theorem~\bookref{top-two}.
\end{proof}

\begin{defn}
  Let $f$ be an endofuncoid.
  \[ T f = \setcond{ E \in \subsets \Ob f }{
     \forall x \in E : \supfun{f} \{ x \} \sqsubseteq E } . \]
\end{defn}

\begin{prop}
  $T f$ is a topology.
\end{prop}

\begin{proof}
  
  \begin{description}
    \item[Union of open sets is open] $S \subseteq T f \Rightarrow \forall E
    \in S \forall x \in E : \supfun{f} x \sqsubseteq E \Rightarrow
    \forall x \in \bigcup S : \supfun{f} x \sqsubseteq \bigcup S$
    
    \item[Intersection of two open sets is open] Let $X, Y \in T f$. Then
    $\forall x \in X : \supfun{f} x \sqsubseteq X$ and $\forall x \in Y
    : \supfun{f} x \sqsubseteq Y$. So if $x \in X \cap Y$ then $\langle
    f \rangle x \sqsubseteq X$ and $\supfun{f} x \sqsubseteq Y$, so
    $\supfun{f} x \sqsubseteq X \cap Y$. So $X \cap Y \in T f$.
    
    \item[$\Ob f$ is an open set] Obvious.
  \end{description}
\end{proof}

\begin{obvious}
$T f = \setcond{ E \in \subsets \Ob f }{ \langle
\Compl f \rangle E \sqsubseteq E }$.
\end{obvious}

\begin{defn}
Topology $T'f$ for an endofuncoid~$f$ is defined by the formula:
\[
F_{\star} T' f =
1 \sqcup \CoCompl f \sqcup (\CoCompl f)^2 \sqcup \ldots.
\]
\end{defn}

\begin{prop}
There is a unique topology $T'f$ for every endofuncoid~$f$.
\end{prop}

\begin{proof}
We need to prove that
$\rsupfun{1 \sqcup \CoCompl f \sqcup (\CoCompl f)^2 \sqcup \ldots}$ is
a Kuratowski closure.

First, it is a function from principal filters to principal filter.

It remains to show that it is reflexive and transitive.

Reflexivity is obvious. Let us prove transitivity:

\fxerror{$\rsupfun{(\CoCompl f)^n}
(\rsupfun{(\CoCompl f)^m}E\sqcup \rsupfun{(\CoCompl f)^{m+1}}E
\sqcup\dots) =
\rsupfun{(\CoCompl f)^{n+m}}E\sqcup \rsupfun{(\CoCompl f)^{n+m+1}}E$
is not proved.}
$\rsupfun{1 \sqcup \CoCompl f \sqcup (\CoCompl f)^2 \sqcup \ldots}
\rsupfun{1 \sqcup \CoCompl f \sqcup (\CoCompl f)^2 \sqcup \ldots}E =
\rsupfun{1 \sqcup \CoCompl f \sqcup (\CoCompl f)^2 \sqcup \ldots}
(\rsupfun{1}E\sqcup \rsupfun{\CoCompl f}E \sqcup
\rsupfun{(\CoCompl f)^2}E\sqcup \ldots) =
\rsupfun{1}(\rsupfun{1}E\sqcup \rsupfun{\CoCompl f}E \sqcup
\rsupfun{(\CoCompl f)^2}E\sqcup \ldots) \sqcup
\rsupfun{\CoCompl f}(\rsupfun{1}E\sqcup \rsupfun{\CoCompl f}E \sqcup
\rsupfun{(\CoCompl f)^2}E\sqcup \ldots) \sqcup
\rsupfun{(\CoCompl f)^2}(\rsupfun{1}E\sqcup \rsupfun{\CoCompl f}E \sqcup
\rsupfun{(\CoCompl f)^2}E\sqcup \ldots) \sqcup \dots =
(\rsupfun{1}E\sqcup \rsupfun{\CoCompl f}E \sqcup
\rsupfun{(\CoCompl f)^2}E\sqcup \ldots) \sqcup
(\rsupfun{\CoCompl f}E\sqcup \rsupfun{(\CoCompl f)^2}E \sqcup
\rsupfun{(\CoCompl f)^3}E\sqcup \ldots) \sqcup \dots =
\rsupfun{1}E\sqcup \rsupfun{\CoCompl f}E \sqcup
\rsupfun{(\CoCompl f)^2}E\sqcup \ldots =
\rsupfun{1 \sqcup \CoCompl f \sqcup (\CoCompl f)^2 \sqcup \ldots}E$.
\end{proof}

\begin{thm}
  (TODO: Reformulate in terms of $T'$.)
  Let $f$ be an endofuncoid. If every $\{ x \}$ is a closed set in $T f$ then
  \begin{enumerate}
    \item $F^{\fullstar} T f = 1 \sqcup \Compl f \sqcup (\Compl f)^2
    \sqcup \ldots$;
    
    \item $F_{\fullstar} T f = 1 \sqcup \CoCompl f \sqcup (\CoCompl
    f)^2 \sqcup \ldots$
  \end{enumerate}
  (decompose $T$ into two functors: $\Compl$ (not a functor?) and $f
  \mapsto 1 \sqcup f \sqcup f^2 \sqcup \ldots$)
\end{thm}

\begin{proof}
  Due symmetry it's enough to prove only the first.
  
  We will assume without loss of generality (TODO) that $f$ is reflexive.
  
  First, $1 \sqcup \Compl f \sqcup (\Compl f)^2 \sqcup \ldots$ is
  an inverse of a Kuratowski closure: it is complete (as join of complete
  funcoids), reflexive, and transitive ($(1 \sqcup \Compl f \sqcup
  (\Compl f)^2 \sqcup \ldots) \circ (1 \sqcup \Compl f \sqcup
  (\Compl f)^2 \sqcup \ldots) = (1 \sqcup \Compl f \sqcup
  (\Compl f)^2 \sqcup \ldots) \sqcup (\Compl f \sqcup
  (\Compl f)^2 \sqcup (\Compl f)^3 \sqcup \ldots) \sqcup \ldots =
  1 \sqcup \Compl f \sqcup (\Compl f)^2 \sqcup \ldots$).
  
  So there exists topology $t$ such that $F^{\fullstar} t = 1 \sqcup \Compl
  f \sqcup (\Compl f)^2 \sqcup \ldots$
  
  For all $x \in E \in T f$ we have $\langle \Compl f \rangle E
  \sqsubseteq E$, so $\langle (\Compl f)^n \rangle E \sqsubseteq E$ and
  $\langle (\Compl f)^n \rangle \{ x \} \sqsubseteq E$. Thus
  $(\Compl f)^n \sqsubseteq T f$ and so
  \[ 1 \sqcup \Compl f \sqcup (\Compl f)^2 \sqcup \ldots
     \sqsubseteq F^{\fullstar} T f \]
  consequently $t \sqsubseteq T f$.
  
  Suppose
  \[ g \sqsupseteq 1 \wedge g \sqsupseteq \Compl f \wedge g \sqsupseteq
     (\Compl f)^2 \wedge \ldots \]
  It remains to prove $g \sqsupseteq F^{\fullstar} T f$ that is $\langle g
  \rangle^{\ast} \{ x \} \sqsupseteq \langle F^{\fullstar} T f \rangle^{\ast} \{ x
  \}$ that is
  
  $\rsupfun{g} \{ x \} \sqsupseteq \bigsqcap^{\mathscr{F}}
  \setcond{ E \in T f }{ x \in E }$ that is
  \[ \rsupfun{g} \{ x \} \sqsupseteq \bigsqcap^{\mathscr{F}}
     \setcond{ E \in \subsets \Ob f }{ \langle
     \Compl f \rangle E \sqsubseteq E \wedge x \in E } \]
  Let $K \in \up \rsupfun{g} \{ x \}$. Take $E =
  \operatorname{int}_{T f} K$. It remains to prove $x \in \operatorname{int}_{T f} K$.
  
  $\operatorname{int} K = \operatorname{int} K_0 \cup \operatorname{int} K_1 \cup \ldots$ (because it
  is a totally ordered set)
  
  It remains to prove $x \in \overline{\left[ \overline{K} \right]_{T f}}$
  that is $\{ x \} \asymp \left[ \overline{K} \right]_{T f}$ that is $\{ x \}
  \asymp \bigcap \left\{ D \in T f \hspace{1em} | \hspace{1em} \overline{K}
  \subseteq D \right\}$ that is $\exists D \in T f : \left( x \notin D \wedge
  \overline{K} \subseteq D \right)$ that is $\exists D : (x \notin D \wedge
  \langle \Compl f \rangle^{\ast} D \sqsubseteq D \wedge K \cup D =
  \top)$
  
  Take $D = \top \setminus \{ x \}$.
  
  The theorem holds if $\{ x \}$ is a closed set.
\end{proof}

\begin{example}
The condition of $\{x\}$ being closed cannot be removed from
the last theorem.
\end{example}

\begin{proof}
A counterexample is principal funcoid (binary relation)
$f = \{ (0, 0), (1, 1), (0, 1) \}$.

$\supfun{\operatorname{id}_U \cup f \cup f^2 \cup f^3 \cup \ldots} E = E$ for every open set $E$.
So, it can be closure only if all open sets are closed.
Open sets are $\{ \}$, $\{ 1 \}$, $\{ 0, 1 \}$. $\{ 1 \}$ is open but
not closed.
\end{proof}

That $F^{\star}$ and $F_{\star}$ are functors (if we map morphisms to
themselves except of changing the objects) follows from
conjecture~\bookref{top-cont}.

Is $T$ a functor?

Are these functors adjoint?
