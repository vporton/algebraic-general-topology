
\chapter{\label{pf-funcoids}Pointfree funcoids}

This chapter is based on \cite{pointfree}.

This is a routine chapter. There is almost nothing creative here.
I just generalize theorems about funcoids to the maximum extent for
\emph{pointfree funcoids} (defined below) preserving the proof idea.
The main idea behind this chapter is to find weakest theorem conditions
enough for the same theorem statement as for above theorems for funcoids.

For these who know pointfree topology: Pointfree topology notions
of frames and locales is a non-trivial generalization of topological
spaces. Pointfree funcoids are different: I just replace the set of
filters on a set with an arbitrary poset, this readily gives the definition
of \emph{pointfree funcoid}, almost no need of creativity here.

Pointfree funcoids are used in the below definitions of products of
funcoids.


\section{Definition}
\begin{defn}
\index{funcoid!pointfree}\emph{Pointfree funcoid} is a quadruple
$(\mathfrak{A};\mathfrak{B};\alpha;\beta)$ where~$\mathfrak{A}$
and~$\mathfrak{B}$ are posets, $\alpha\in\mathfrak{B}^{\mathfrak{A}}$
and $\beta\in\mathfrak{A}^{\mathfrak{B}}$ such that 
\[
\forall x\in\mathfrak{A},y\in\mathfrak{B}:(y\nasymp\alpha x\Leftrightarrow x\nasymp\beta y).
\]

\end{defn}

\begin{defn}
\index{funcoid!pointfree!source}\index{funcoid!pointfree!destination}The
\emph{source} $\Src(\mathfrak{A};\mathfrak{B};\alpha;\beta)=\mathfrak{A}$
and \emph{destination} $\Dst(\mathfrak{A};\mathfrak{B};\alpha;\beta)=\mathfrak{B}$
for every pointfree funcoid $(\mathfrak{A};\mathfrak{B};\alpha;\beta)$.
\end{defn}
To every funcoid $(A;B;\alpha;\beta)$ corresponds pointfree funcoid
$(\subsets A;\subsets B;\alpha;\beta)$. Thus pointfree funcoids are
a generalization of funcoids.
\begin{defn}
I will denote $\mathsf{pFCD}(\mathfrak{A};\mathfrak{B})$ the set
of pointfree funcoids from $\mathfrak{A}$ to $\mathfrak{B}$ (that
is with source $\mathfrak{A}$ and destination $\mathfrak{B}$), for
every posets $\mathfrak{A}$ and $\mathfrak{B}$.

$\supfun{(\mathfrak{A};\mathfrak{B};\alpha;\beta)}\eqdef\alpha$ for
every pointfree funcoid $(\mathfrak{A};\mathfrak{B};\alpha;\beta)$.
\end{defn}

\begin{defn}
$(\mathfrak{A};\mathfrak{B};\alpha;\beta)^{-1}=(\mathfrak{B};\mathfrak{A};\beta;\alpha)$
for every pointfree funcoid $(\mathfrak{A};\mathfrak{B};\alpha;\beta)$.\end{defn}
\begin{prop}
If $f$ is a pointfree funcoid then $f^{-1}$ is also a pointfree
funcoid.\end{prop}
\begin{proof}
It follows from symmetry in the definition of pointfree funcoid.\end{proof}
\begin{obvious}
$(f^{-1})^{-1}=f$ for every pointfree funcoid $f$.\end{obvious}
\begin{defn}
The relation $\suprel f\in\subsets(\Src f\times\Dst f)$ is defined
by the formula (for every pointfree funcoid $f$ and $x\in\Src f$,
$y\in\Dst f$)
\[
x\suprel fy\eqdef y\nasymp\supfun fx.
\]
\end{defn}
\begin{obvious}
$x\suprel fy\Leftrightarrow y\nasymp\supfun fx\Leftrightarrow x\nasymp\supfun{f^{-1}}y$
for every pointfree funcoid $f$ and $x\in\Src f$, $y\in\Dst f$.
\end{obvious}

\begin{obvious}
$\suprel{f^{-1}}=\suprel f^{-1}$ for every pointfree funcoid $f$.\end{obvious}
\begin{thm}
\label{one-funcoid}Let $\mathfrak{A}$ and $\mathfrak{B}$ be posets.
Then:
\begin{enumerate}
\item \label{pf-fun-one}If $\mathfrak{A}$ is separable, for given value
of $\supfun f$ there exists no more than one $f\in\mathsf{pFCD}(\mathfrak{A};\mathfrak{B})$.
\item \label{pf-rel-one}If $\mathfrak{A}$ and $\mathfrak{B}$ are separable,
for given value of $\suprel f$ there exists no more than one $f\in\mathsf{pFCD}(\mathfrak{A};\mathfrak{B})$.
\end{enumerate}
\end{thm}
\begin{proof}
Let $f,g\in\mathsf{pFCD}(\mathfrak{A};\mathfrak{B})$.
\begin{widedisorder}
\item [{\ref{pf-fun-one}}] Let $\supfun f=\supfun g$. Then for every
$x\in\mathfrak{A}$, $y\in\mathfrak{B}$ we have
\[
x\nasymp\supfun{f^{-1}}y\Leftrightarrow y\nasymp\supfun fx\Leftrightarrow y\nasymp\supfun gx\Leftrightarrow x\nasymp\supfun{g^{-1}}y
\]
and thus by separability of $\mathfrak{A}$ we have $\supfun{f^{-1}}y=\supfun{g^{-1}}y$
that is $\supfun{f^{-1}}=\supfun{g^{-1}}$ and so $f=g$.
\item [{\ref{pf-rel-one}}] Let $\suprel f=\suprel g$. Then for every
$x\in\mathfrak{A}$, $y\in\mathfrak{B}$ we have
\[
x\nasymp\supfun{f^{-1}}y\Leftrightarrow x\suprel fy\Leftrightarrow x\suprel gy\Leftrightarrow x\nasymp\supfun{g^{-1}}y
\]
and thus by separability of $\mathfrak{A}$ we have $\supfun{f^{-1}}y=\supfun{g^{-1}}y$
that is $\supfun{f^{-1}}=\supfun{g^{-1}}$. Similarly we have $\supfun{f}=\supfun{g}$.
Thus $f=g$.
\end{widedisorder}
\end{proof}
\begin{prop}
\label{pfcd-zero}If $\Src f$ and $\Dst f$ have least elements,
then $\supfun f\bot^{\Src f}=\bot^{\Dst f}$ for every pointfree funcoid
$f$.\end{prop}
\begin{proof}
$y\nasymp\supfun f\bot^{\Src f}\Leftrightarrow\bot^{\Src f}\nasymp\supfun{f^{-1}}y\Leftrightarrow0$
for every $y\in\Dst f$. Thus $\supfun f\bot^{\Src f}\asymp\supfun f\bot^{\Src f}$.
So $\supfun f\bot^{\Src f}=\bot^{\Dst f}$.\end{proof}
\begin{prop}
\label{pfcd-mono}If $\Dst f$ is a separable meet-semilattice then
$\supfun f$ is a monotone function (for a pointfree funcoid $f$).\end{prop}
\begin{proof}
~
\begin{align*}
a\sqsubseteq b & \Rightarrow\\
\forall x\in\Dst f:(a\nasymp\supfun{f^{-1}}x\Rightarrow b\nasymp\supfun{f^{-1}}x) & \Rightarrow\\
\forall x\in\Dst f:(x\nasymp\supfun fa\Rightarrow x\nasymp\supfun fb) & \Leftrightarrow\\
\fullstar\supfun fa\subseteq\fullstar\supfun fb & \Rightarrow\\
\supfun fa\sqsubseteq\supfun fb
\end{align*}
(used theorem \ref{msl-sep-conds} and that it is a separable meet-semilattice).\end{proof}
\begin{thm}
\label{pf-dist-func}Let $f$ be a pointfree funcoid from a starrish
join-semilattice $\Src f$ to a separable starrish join-semilattice
$\Dst f$. Then $\supfun f(i\sqcup j)=\supfun fi\sqcup\supfun fj$
for every $i,j\in\Src f$.\end{thm}
\begin{proof}
~
\begin{align*}
\fullstar\supfun f(i\sqcup j) & =\\
\setcond{y\in\Dst f}{y\nasymp\supfun f(i\sqcup j)} & =\\
\setcond{y\in\Dst f}{i\sqcup j\nasymp\supfun{f^{-1}}y} & =\\
\setcond{y\in\Dst f}{i\nasymp\supfun{f^{-1}}y\lor j\nasymp\supfun{f^{-1}}y} & =\\
\setcond{y\in\Dst f}{y\nasymp\supfun fi\lor y\nasymp\supfun fj} & =\\
\setcond{y\in\Dst f}{y\nasymp\supfun fi\sqcup\supfun fj} & =\\
\fullstar(\supfun fi\sqcup\supfun fj).
\end{align*}
Thus $\supfun f(i\sqcup j)=\supfun fi\sqcup\supfun fj$ by separability.\end{proof}
\begin{prop}
\label{pf-join-arg}Let $f$ be a pointfree funcoid. Then:\end{prop}
\begin{enumerate}
\item \label{pf-f-join-y}$k\suprel fi\sqcup j\Leftrightarrow k\suprel fi\lor k\suprel fj$
for every $i,j\in\Dst f$, $k\in\Src f$ if $\Dst f$ is a starrish
join-semilattice.
\item \label{pf-f-join-x}$i\sqcup j\suprel fk\Leftrightarrow i\suprel fk\lor j\suprel fk$
for every $i,j\in\Src f$, $k\in\Dst f$ if $\Src f$ is a starrish
join-semilattice.\end{enumerate}
\begin{proof}
~
\begin{widedisorder}
\item [{\ref{pf-f-join-y}}] $k\suprel fi\sqcup j\Leftrightarrow i\sqcup j\nasymp\supfun fk\Leftrightarrow i\nasymp\supfun fk\lor j\nasymp\supfun fk\Leftrightarrow k\suprel fi\lor k\suprel fj$.
\item [{\ref{pf-f-join-x}}] Similar.
\end{widedisorder}
\end{proof}

\section{Composition of pointfree funcoids}
\begin{defn}
\index{funcoid!pointfree!composition}\emph{Composition} of pointfree
funcoids is defined by the formula 
\[
(\mathfrak{B};\mathfrak{C};\alpha_{2};\beta_{2})\circ(\mathfrak{A};\mathfrak{B};\alpha_{1};\beta_{1})=(\mathfrak{A};\mathfrak{C};\alpha_{2}\circ\alpha_{1};\beta_{1}\circ\beta_{2}).
\]

\end{defn}

\begin{defn}
\index{funcoid!pointfree!composable}I will call funcoids $f$ and
$g$ \emph{composable} when $\Dst f=\Src g$.\end{defn}
\begin{prop}
If $f$, $g$ are composable pointfree funcoids then $g\circ f$ is
pointfree funcoid.\end{prop}
\begin{proof}
Let $f=(\mathfrak{A};\mathfrak{B};\alpha_{1};\beta_{1})$, $g=(\mathfrak{B};\mathfrak{C};\alpha_{2};\beta_{2})$.
For every $x,y\in\mathfrak{A}$ we have 
\[
y\nasymp(\alpha_{2}\circ\alpha_{1})x\Leftrightarrow y\nasymp\alpha_{2}\alpha_{1}x\Leftrightarrow\alpha_{1}x\nasymp\beta_{2}y\Leftrightarrow x\nasymp\beta_{1}\beta_{2}y\Leftrightarrow x\nasymp(\beta_{1}\circ\beta_{2})y.
\]
So $(\mathfrak{A};\mathfrak{C};\alpha_{2}\circ\alpha_{1};\beta_{1}\circ\beta_{2})$
is a pointfree funcoid.\end{proof}
\begin{obvious}
$\langle g\circ f\rangle=\langle g\rangle\circ\langle f\rangle$ for
every composable pointfree funcoids $f$ and $g$.\end{obvious}
\begin{thm}
$(g\circ f)^{-1}=f^{-1}\circ g^{-1}$ for every composable pointfree
funcoids $f$ and $g$.\end{thm}
\begin{proof}
~
\begin{gather*}
\supfun{(g\circ f)^{-1}}=\supfun{f^{-1}}\circ\supfun{g^{-1}}=\supfun{f^{-1}\circ g^{-1}};\\
\supfun{((g\circ f)^{-1})^{-1}}=\supfun{g\circ f}=\supfun{(f^{-1}\circ g^{-1})^{-1}}.
\end{gather*}
\end{proof}
\begin{prop}
$(h\circ g)\circ f=h\circ(g\circ f)$ for every composable pointfree
funcoids $f$, $g$, $h$.\end{prop}
\begin{proof}
$\supfun{(h\circ g)\circ f}=\supfun{h\circ g}\circ\supfun f=\supfun h\circ\supfun g\circ\supfun f=\supfun h\circ\supfun{g\circ f}=\supfun{h\circ(g\circ f)};$

\begin{multline*}
\supfun{((h\circ g)\circ f)^{-1}}=\supfun{f^{-1}\circ(h\circ g)^{-1}}=\supfun{f^{-1}\circ g^{-1}\circ h^{-1}}=\\
\supfun{(g\circ f)^{-1}\circ h^{-1}}=\supfun{(h\circ(g\circ f))^{-1}}.
\end{multline*}
\end{proof}
\begin{xca}
Generalize section~\ref{fcd-rel-another} for pointfree funcoids.
\end{xca}

\section{Pointfree funcoid as continuation}
\begin{prop}
Let $f$ be a pointfree funcoid. Then for every $x\in\Src f$, $y\in\Dst f$
we have
\begin{enumerate}
\item If $(\Src f;\mathfrak{Z})$ is a filtrator with separable core then
$x\suprel fy\Leftrightarrow\forall X\in\up^{\mathfrak{Z}}x:X\suprel fy$.
\item If $(\Dst f;\mathfrak{Z})$ is a filtrator with separable core then
$x\suprel fy\Leftrightarrow\forall Y\in\up^{\mathfrak{Z}}y:x\suprel fY$.
\end{enumerate}
\end{prop}
\begin{proof}
We will prove only the second because the first is similar.
\[
x\suprel fy\Leftrightarrow y\nasymp^{\Dst f}\supfun fx\Leftrightarrow\forall Y\in\up^{\mathfrak{Z}}y:Y\nasymp\supfun fx\Leftrightarrow\forall Y\in\up^{\mathfrak{Z}}y:x\suprel fY.
\]
\end{proof}
\begin{cor}
\label{pf-relatom-both}Let $f$ be a pointfree funcoid and $(\Src f;\mathfrak{Z}_{0})$,
$(\Dst f;\mathfrak{Z}_{1})$ be filtrators with separable core. Then
\[
x\suprel fy\Leftrightarrow\forall X\in\up^{\mathfrak{Z}_{0}}x,Y\in\up^{\mathfrak{Z}_{1}}y:X\suprel fY.
\]
\end{cor}
\begin{proof}
Apply the proposition twice.\end{proof}
\begin{thm}
\label{pf-supfun-up}Let $f$ be a pointfree funcoid. Let $(\Src f;\mathfrak{Z}_{0})$
be a binarily meet-closed filtrator with separable core which is a
meet-semilattice and $\forall x\in\Src f:\up^{\mathfrak{Z}_{0}}x\neq\emptyset$
and $(\Dst f;\mathfrak{Z}_{1})$ be a primary filtrator over a boolean
lattice.
\[
\supfun fx=\bigsqcap^{\Dst f}\rsupfun{\supfun f}\up^{\mathfrak{Z}_{0}}x.
\]
\end{thm}
\begin{proof}
By the previous proposition for every $y\in\Dst f$:
\[
y\nasymp^{\Dst f}\supfun fx\Leftrightarrow x\suprel fy\Leftrightarrow\forall X\in\up^{\mathfrak{Z}_{0}}x:X\suprel fy\Leftrightarrow\forall X\in\up^{\mathfrak{Z}_{0}}x:y\nasymp^{\Dst f}\supfun fX.
\]


Let's denote $W=\setcond{y\sqcap^{\Dst f}\rsupfun fX}{X\in\up^{\mathfrak{Z}_{0}}x}$.
We will prove that $W$ is a generalized filter base over $\mathfrak{Z}_{1}$.
To prove this enough to show that $V=\setcond{\rsupfun fX}{X\in\up^{\mathfrak{Z}_{0}}x}$
is a generalized filter base.

Let $\mathcal{P},\mathcal{Q}\in V$. Then $\mathcal{P}=\rsupfun fA$,
$\mathcal{Q}=\rsupfun fB$ where $A,B\in\up^{\mathfrak{Z}_{0}}x$;
$A\sqcap^{\mathfrak{Z}_{0}}B\in\up^{\mathfrak{Z}_{0}}x$
(used the fact that it is a binarily meet-closed and theorem \ref{up-filt-crit})
and $\mathcal{R}\sqsubseteq\mathcal{P}\sqcap^{\Dst f}\mathcal{Q}$
for $\mathcal{R}=\langle f\rangle(A\sqcap^{\mathfrak{Z}_{0}}B)\in V$
because $\Dst f$ is separable by obvious \ref{filt-is-sep}. So $V$
is a generalized filter base and thus $W$ is a generalized filter
base.

$\bot^{\Dst f}\notin W\Leftrightarrow\bot^{\Dst f}\notin\bigsqcap^{\Dst f}W$
by theorem \ref{genbase-main}. That is
\[
\forall X\in\up^{\mathfrak{Z}_{0}}x:y\sqcap^{\Dst f}\rsupfun fX\ne\bot^{\Dst f}\Leftrightarrow y\sqcap^{\Dst f}\bigsqcap^{\Dst f}\rsupfun{\supfun f}\up^{\mathfrak{Z}_{0}}x\ne\bot^{\Dst f}.
\]
Comparing with the above, 
\[
y\sqcap^{\Dst f}\supfun fx\ne\bot^{\Dst f}\Leftrightarrow y\sqcap^{\Dst f}\bigsqcap^{\Dst f}\rsupfun{\supfun f}\up^{\mathfrak{Z}_{0}}x\ne\bot^{\Dst f}.
\]
So $\supfun fx=\bigsqcap^{\Dst f}\rsupfun{\supfun f}\up^{\mathfrak{Z}_{0}}x$
because $\Dst f$ is separable (obvious \ref{filt-is-sep} and the
fact that $\mathfrak{Z}_{1}$ is a boolean lattice).\end{proof}
\begin{thm}
\label{pf-cont}Let $(\mathfrak{A};\mathfrak{Z}_{0})$ and $(\mathfrak{B};\mathfrak{Z}_{1})$
be primary filtrators over boolean lattices.
\begin{enumerate}
\item \label{pf-cont-f}A function $\alpha\in\mathfrak{B}^{\mathfrak{Z}_{0}}$
conforming to the formulas (for every $I,J\in\mathfrak{Z}_{0}$)
\[
\alpha\bot^{\mathfrak{Z}_{0}}=\bot^{\mathfrak{B}},\quad\alpha(I\sqcup J)=\alpha I\sqcup\alpha J
\]
can be continued to the function $\supfun f$ for a unique $f\in\mathsf{pFCD}(\mathfrak{A};\mathfrak{B})$;
\begin{equation}
\supfun f\mathcal{X}=\bigsqcap^{\mathfrak{B}}\rsupfun{\alpha}\up^{\mathfrak{Z}_{0}}\mathcal{X}\label{pf-alpha-filter}
\end{equation}
for every $\mathcal{X}\in\mathfrak{A}$.
\item \label{pf-cont-r}A relation $\delta\in\subsets(\mathfrak{Z}_{0}\times\mathfrak{Z}_{1})$
conforming to the formulas (for every $I,J,K\in\mathfrak{Z}_{0}$
and $I',J',K'\in\mathfrak{Z}_{1}$)
\begin{equation}
\begin{aligned}\lnot(\bot^{\mathfrak{Z}_{0}}\mathrel\delta I') & ,\quad I\sqcup J\mathrel\delta K'\Leftrightarrow I\mathrel\delta K'\lor J\mathrel\delta K',\\
\lnot(I\mathrel\delta\bot^{\mathfrak{Z}_{1}}) & ,\quad K\mathrel\delta I'\sqcup J'\Leftrightarrow K\mathrel\delta I'\lor K\mathrel\delta J'
\end{aligned}
\label{pf-delta-props}
\end{equation}
can be continued to the relation $\suprel f$ for a unique $f\in\mathsf{pFCD}(\mathfrak{A};\mathfrak{B})$;
\begin{equation}
\mathcal{X}\suprel f\mathcal{Y}\Leftrightarrow\forall X\in\up^{\mathfrak{Z}_{0}}\mathcal{X},Y\in\up^{\mathfrak{Z}_{1}}\mathcal{Y}:X\mathrel\delta Y\label{pf-suprel-delta}
\end{equation}
for every $\mathcal{X}\in\mathfrak{A}$, $\mathcal{Y}\in\mathfrak{B}$.
\end{enumerate}
\end{thm}
\begin{proof}
Existence of no more than one such pointfree funcoids and formulas
(\ref{pf-alpha-filter}) and (\ref{pf-suprel-delta}) follow from
two previous theorems.
\begin{widedisorder}
\item [{\ref{pf-cont-r}}] $\setcond{Y\in\mathfrak{Z}_{1}}{X\mathrel\delta Y}$
is obviously a free star for every $X\in\mathfrak{Z}_{0}$. By properties
of filters on boolean lattices, there exist a unique filter $\alpha X$
such that $\corestar(\alpha X)=\setcond{Y\in\mathfrak{Z}_{1}}{X\mathrel\delta Y}$
for every $X\in\mathfrak{Z}_{0}$. Thus $\alpha\in\mathfrak{B}^{\mathfrak{Z}_{0}}$.
Similarly it can be defined $\beta\in\mathfrak{A}^{\mathfrak{Z}_{1}}$
by the formula $\corestar(\beta Y)=\setcond{X\in\mathfrak{Z}_{0}}{X\mathrel\delta Y}$.
Let's continue the functions $\alpha$ and $\beta$ to $\alpha'\in\mathfrak{B}^{\mathfrak{A}}$
and $\beta'\in\mathfrak{A}^{\mathfrak{B}}$ by the formulas
\[
\alpha'\mathcal{X}=\bigsqcap^{\mathfrak{B}}\rsupfun{\alpha}\up^{\mathfrak{Z}_{0}}\mathcal{X}\quad\text{and}\quad\beta'\mathcal{Y}=\bigsqcap^{\mathfrak{A}}\rsupfun{\beta}\up^{\mathfrak{Z}_{1}}\mathcal{Y}
\]
and $\delta$ to $\delta'\in\subsets(\mathfrak{A}\times\mathfrak{B})$
by the formula
\[
\mathcal{X}\mathrel{\delta'}\mathcal{Y}\Leftrightarrow\forall X\in\up^{\mathfrak{Z}_{0}}\mathcal{X},Y\in\up^{\mathfrak{Z}_{1}}\mathcal{Y}:\mathcal{X}\mathrel\delta\mathcal{Y}.
\]
$\mathcal{Y}\sqcap\alpha'\mathcal{X}\ne\bot^{\mathfrak{B}}\Leftrightarrow\mathcal{Y}\sqcap\bigsqcap\rsupfun{\alpha}\up^{\mathfrak{Z}_{0}}\mathcal{X}\ne\bot^{\mathfrak{B}}\Leftrightarrow\bigsqcap\rsupfun{\mathcal{Y}\sqcap}\rsupfun{\alpha}\up^{\mathfrak{Z}_{0}}\mathcal{X}\ne\bot^{\mathfrak{B}}$.
Let's prove that
\[
W=\rsupfun{\mathcal{Y}\sqcap}\rsupfun{\alpha}\up^{\mathfrak{Z}_{0}}\mathcal{X}
\]
is a generalized filter base: To prove it is enough to show that $\rsupfun{\alpha}\up^{\mathfrak{Z}_{0}}\mathcal{X}$
is a generalized filter base.


If $\mathcal{A},\mathcal{B}\in\rsupfun{\alpha}\up^{\mathfrak{Z}_{0}}\mathcal{X}$
then exist $X_{1},X_{2}\in\up^{\mathfrak{Z}_{0}}\mathcal{X}$ such
that $\mathcal{A}=\alpha X_{1}$ and $\mathcal{B}=\alpha X_{2}$.
Then $\alpha(X_{1}\sqcap^{\mathfrak{Z}_{0}}X_{2})\in\rsupfun{\alpha}\up^{\mathfrak{Z}_{0}}\mathcal{X}$.
So $\rsupfun{\alpha}\up^{\mathfrak{Z}_{0}}\mathcal{X}$ is a generalized
filter base and thus $W$ is a generalized filter base.


By properties of generalized filter bases, $\bigsqcap\rsupfun{\mathcal{Y}\sqcap}\rsupfun{\alpha}\up^{\mathfrak{Z}_{0}}\mathcal{X}\ne\bot^{\mathfrak{B}}$
is equivalent to
\[
\forall X\in\up^{\mathfrak{Z}_{0}}\mathcal{X}:\mathcal{Y}\sqcap\alpha X\ne\bot^{\mathfrak{B}},
\]
what is equivalent to
\begin{align*}
\forall X\in\up^{\mathfrak{Z}_{0}}\mathcal{X},Y\in\up^{\mathfrak{Z}_{1}}\mathcal{Y}:Y\sqcap^{\mathfrak{B}}\alpha X\ne\bot^{\mathfrak{B}} & \Leftrightarrow\\
\forall X\in\up^{\mathfrak{Z}_{0}}\mathcal{X},Y\in\up^{\mathfrak{Z}_{1}}\mathcal{Y}:Y\in\corestar(\alpha X) & \Leftrightarrow\\
\forall X\in\up^{\mathfrak{Z}_{0}}\mathcal{X},Y\in\up^{\mathfrak{Z}_{1}}\mathcal{Y}:X\mathrel\delta Y.
\end{align*}
Combining the equivalencies we get $\mathcal{Y}\sqcap\alpha'\mathcal{X}\ne\bot^{\mathfrak{B}}\Leftrightarrow\mathcal{X}\mathrel{\delta'}\mathcal{Y}$.
Analogously $\mathcal{X}\sqcap\beta'\mathcal{Y}\ne\bot^{\mathfrak{A}}\Leftrightarrow\mathcal{X}\mathrel{\delta'}\mathcal{Y}$.
So $\mathcal{Y}\sqcap\alpha'\mathcal{X}\ne\bot^{\mathfrak{B}}\Leftrightarrow\mathcal{X}\sqcap\beta'\mathcal{Y}\ne\bot^{\mathfrak{A}}$,
that is $(\mathfrak{A};\mathfrak{B};\alpha';\beta')$ is a pointfree
funcoid. From the formula $\mathcal{Y}\sqcap\alpha'\mathcal{X}\ne\bot^{\mathfrak{B}}\Leftrightarrow\mathcal{X}\mathrel{\delta'}\mathcal{Y}$
it follows that $\suprel{(\mathfrak{A};\mathfrak{B};\alpha';\beta')}$
is a continuation of $\delta$.

\item [{\ref{pf-cont-f}}] Let define the relation $\delta\in\subsets(\mathfrak{Z}_{0}\times\mathfrak{Z}_{1})$
by the formula $X\mathrel{\delta}Y\Leftrightarrow Y\sqcap^{\mathfrak{B}}\alpha X\neq\bot^{\mathfrak{B}}$.


That $\neg(\bot^{\mathfrak{Z}_{0}}\mathrel{\delta}I')$ and $\neg(I\mathrel{\delta}\bot^{\mathfrak{Z}_{1}})$
is obvious. We have


\begin{align*}
K\mathrel{\delta}I'\sqcup^{\mathfrak{Z}_{1}}J' & \Leftrightarrow\\
(I'\sqcup^{\mathfrak{Z}_{1}}J')\sqcap^{\mathfrak{B}}\alpha K\neq\bot^{\mathfrak{B}} & \Leftrightarrow\\
(I'\sqcup^{\mathfrak{B}}J')\sqcap\alpha K\neq\bot^{\mathfrak{B}} & \Leftrightarrow\\
(I'\sqcap^{\mathfrak{B}}\alpha K)\sqcup(J'\sqcap^{\mathfrak{B}}\alpha K)\neq\bot^{\mathfrak{B}} & \Leftrightarrow\\
I'\sqcap^{\mathfrak{B}}\alpha K\neq\bot^{\mathfrak{B}}\vee J'\sqcap^{\mathfrak{B}}\alpha K\neq\bot^{\mathfrak{B}} & \Leftrightarrow\\
K\mathrel{\delta}I'\vee K\mathrel{\delta}J'
\end{align*}
and
\begin{align*}
I\sqcup^{\mathfrak{Z}_{0}}J\mathrel{\delta}K' & \Leftrightarrow\\
K'\sqcap^{\mathfrak{B}}\alpha(I\sqcup^{\mathfrak{Z}_{0}}J)\neq\bot^{\mathfrak{B}} & \Leftrightarrow\\
K'\sqcap^{\mathfrak{B}}(\alpha I\sqcup\alpha J)\neq\bot^{\mathfrak{B}} & \Leftrightarrow\\
(K'\sqcap^{\mathfrak{B}}\alpha I)\sqcup(K'\sqcap^{\mathfrak{B}}\alpha J)\neq\bot^{\mathfrak{B}} & \Leftrightarrow\\
K'\sqcap^{\mathfrak{B}}\alpha I\neq0^{\mathfrak{B}}\vee K'\sqcap^{\mathfrak{B}}\alpha J\neq\bot^{\mathfrak{B}} & \Leftrightarrow\\
I\mathrel{\delta}K'\vee J\mathrel{\delta}K'.
\end{align*}



That is the formulas (\ref{pf-delta-props}) are true.


Accordingly the above $\delta$ can be continued to the relation $\suprel f$
for some $f\in\mathsf{pFCD}(\mathfrak{A};\mathfrak{B})$.


$\forall X\in\mathfrak{Z}_{0},Y\in\mathfrak{Z}_{1}:(Y\sqcap^{\mathfrak{B}}\supfun fX\neq\bot^{\mathfrak{B}}\Leftrightarrow X\suprel fY\Leftrightarrow Y\sqcap^{\mathfrak{B}}\alpha X\neq\bot^{\mathfrak{B}})$,
consequently $\forall X\in\mathfrak{Z}_{0}:\alpha X=\supfun fX$ because
our filtrator is with separable core. So $\supfun f$ is a continuation
of $\alpha$.

\end{widedisorder}
\end{proof}
\begin{thm}
Let $(\mathfrak{A};\mathfrak{Z}_{0})$ and $(\mathfrak{B};\mathfrak{Z}_{1})$
be primary filtrators over boolean lattices. If $\alpha\in\mathfrak{B}^{\mathfrak{Z}_{0}}$,
$\beta\in\mathfrak{A}^{\mathfrak{Z}_{1}}$ are functions such that
$Y\nasymp\alpha X\Leftrightarrow X\nasymp\beta Y$ for every $X\in\mathfrak{Z}_{0}$,
$Y\in\mathfrak{Z}_{1}$, then there exists exactly one pointfree funcoid $f:\mathfrak{A}\rightarrow\mathfrak{B}$
such that $\supfun{f}|_{\mathfrak{Z}_0}=\alpha$, $\supfun{f^{-1}}|_{\mathfrak{Z}_1}=\beta$.\end{thm}
\begin{proof}
Prove $\alpha(I\sqcup J)=\alpha I\sqcup\alpha J$. Really, $Y\nasymp\alpha(I\sqcup J)\Leftrightarrow I\sqcup J\nasymp\beta Y\Leftrightarrow I\nasymp\beta Y\vee J\nasymp\beta Y\Leftrightarrow Y\nasymp\alpha I\vee Y\nasymp\alpha J\Leftrightarrow Y\nasymp\alpha I\sqcup\alpha J$.
So $\alpha(I\sqcup J)=\alpha I\sqcup\alpha J$ by star-separability.
Similarly $\beta(I\sqcup J)=\beta I\sqcup\beta J$.

Thus by the theorem above there exists a pointfree funcoid $f$ such that
$\supfun{f}|_{\mathfrak{Z}_0}=\alpha$, $\supfun{f^{-1}}|_{\mathfrak{Z}_1}=\beta$.

That this pointfree funcoid is unique, follows from the above.\end{proof}
\begin{prop}
Let $(\Src f;\mathfrak{Z}_{0})$ be a primary filtrator over a bounded
distributive lattice and $(\Dst f;\mathfrak{Z}_{1})$ be a primary
filtrator over boolean lattice. If $S$ is a generalized filter base
on $\Src f$ then $\supfun f\bigsqcap^{\Src f}S=\bigsqcap^{\Dst f}\rsupfun{\supfun f}S$
for every pointfree funcoid $f$.\end{prop}
\begin{proof}
First the meets $\bigsqcap^{\Src f}S$ and $\bigsqcap^{\Dst f}\rsupfun{\supfun f}S$
exist by corollary \ref{filt-is-complete}.

$(\Src f;\mathfrak{Z}_{0})$ is a binarily meet-closed filtrator by
corollary~\ref{f-meet-closed} and with separable core by theorem
\ref{when-sep-core}; thus we can apply theorem \ref{pf-supfun-up}
($\up x\neq\emptyset$ is obvious).

$\supfun f\bigsqcap^{\Src f}S\sqsubseteq\supfun fX$ for every $X\in S$
because $\Dst f$ is separable by obvious \ref{filt-is-sep} and thus
$\supfun f\bigsqcap^{\Src f}S\sqsubseteq\bigsqcap^{\Dst f}\rsupfun{\supfun f}S$.

Taking into account properties of generalized filter bases:
\begin{align*}
\supfun f\bigsqcap^{\Src f}S & =\\
\bigsqcap^{\Dst f}\rsupfun{\supfun f}\up\bigsqcap S & =\\
\bigsqcap^{\Dst f}\rsupfun{\supfun f}\setcond X{\exists\mathcal{P}\in S:X\in\up\mathcal{P}} & =\\
\bigsqcap^{\Dst f}\setcond{\rsupfun fX}{\exists\mathcal{P}\in S:X\in\up\mathcal{P}} & \sqsupseteq\text{ (because \ensuremath{\Dst f} is a separable poset)}\\
\bigsqcap^{\Dst f}\setcond{\supfun f\mathcal{P}}{\mathcal{P}\in S} & =\\
\bigsqcap^{\Dst f}\rsupfun{\supfun f}S.
\end{align*}
\end{proof}
\begin{prop}
$\mathcal{X}\suprel f\bigsqcap S\Leftrightarrow\exists\mathcal{Y}\in S:\mathcal{X}\suprel f\mathcal{Y}$
if $f$ is a pointfree funcoid, $\Dst f$ is a meet-semilattice with
least element and $S$ is a generalized filter base on $\Dst f$.\end{prop}
\begin{proof}
~
\begin{multline*}
\mathcal{X}\suprel f\bigsqcap S\Leftrightarrow\bigsqcap S\sqcap\supfun f\mathcal{X}\neq\bot\Leftrightarrow\bigsqcap\langle\langle f\rangle\mathcal{X}\sqcap\rangle^{\ast}S\neq\bot\Leftrightarrow\\
\text{(by properties of generalized filter bases)}\Leftrightarrow\\
\exists\mathcal{Y}\in\langle\supfun f\mathcal{X}\sqcap\rangle^{\ast}S:\mathcal{Y}\neq\bot\Leftrightarrow\exists\mathcal{Y}\in S:\langle f\rangle\mathcal{X}\sqcap\mathcal{Y}\neq\bot\Leftrightarrow\exists\mathcal{Y}\in S:\mathcal{X}\suprel f\mathcal{Y}.
\end{multline*}
\end{proof}
\begin{thm}
\label{pfcd-as-func}A function $\varphi:\mathfrak{A}\rightarrow\mathfrak{B}$,
where $(\mathfrak{A};\mathfrak{Z}_{0})$ and~$(\mathfrak{B};\mathfrak{Z}_{1})$
are primary filtrators over boolean lattices, preserves finite joins
(including nullary joins) and filtered meets iff there exists a pointfree
funcoid $f$ such that $\supfun f=\varphi$.\end{thm}
\begin{proof}
Backward implication follows from above.

Let $\psi=\varphi|_{\mathfrak{Z}_{0}}$. Then $\psi$ preserves bottom
element and binary joins. Thus there exists a funcoid $f$ such that
$\rsupfun f=\psi$.

It remains to prove that $\supfun f=\varphi$.

Really, $\supfun f\mathcal{X}=\bigsqcap\rsupfun{\rsupfun f}\up\mathcal{X}=\bigsqcap\rsupfun{\psi}\up\mathcal{X}=\bigsqcap\rsupfun{\varphi}\up\mathcal{X}=\varphi\bigsqcap\mathcal{X}=\varphi\mathcal{X}$
for every $\mathcal{X}\in\mathscr{F}(\Src f)$.\end{proof}
\begin{cor}
Funcoids $f$ from $A$ to $B$ bijectively correspond by the formula
$\langle f\rangle=\varphi$ to functions $\varphi:\mathscr{F}(A)\rightarrow\mathscr{F}(B)$
preserving finite joins and filtered meets.\end{cor}
\begin{thm}
The set of pointfree funcoids between sets of filters on boolean
lattices is a co-frame.\end{thm}
\begin{proof}
Theorems \ref{pfcd-as-func} and \ref{frame-main}.
\end{proof}

\section{The order of pointfree funcoids}
\begin{defn}
\index{funcoid!pointfree!order}The order of pointfree funcoids $\mathsf{pFCD}(\mathfrak{A};\mathfrak{B})$
is defined by the formula:
\[
f\sqsubseteq g\Leftrightarrow\forall x\in\mathfrak{A}:\supfun fx\sqsubseteq\supfun gx\land\forall y\in\mathfrak{B}:\supfun{f^{-1}}y\sqsubseteq\supfun{g^{-1}}y.
\]
\end{defn}
\begin{prop}
It is really a partial order on the set $\mathsf{pFCD}(\mathfrak{A};\mathfrak{B})$.\end{prop}
\begin{proof}
~
\begin{description}
\item [{Reflexivity}] Obvious.
\item [{Transitivity}] It follows from transitivity of the order relations
on $\mathfrak{A}$ and~$\mathfrak{B}$.
\item [{Antisymmetry}] It follows from antisymmetry of the order relations
on $\mathfrak{A}$ and~$\mathfrak{B}$.
\end{description}
\end{proof}
\begin{rem}
It is enough to define order of pointfree funcoids on every set $\mathsf{pFCD}(\mathfrak{A};\mathfrak{B})$
where $\mathfrak{A}$ and $\mathfrak{B}$ are posets. We do not need
to compare pointfree funcoids with different sources or destinations.\end{rem}
\begin{obvious}
$f\sqsubseteq g\Rightarrow\suprel f\subseteq\suprel g$ for every
$f,g\in\mathsf{pFCD}(\mathfrak{A};\mathfrak{B})$ for every posets
$\mathfrak{A}$ and~$\mathfrak{B}$.\end{obvious}
\begin{thm}
\label{pf-fcd-poset}If $\mathfrak{A}$ and $\mathfrak{B}$ are separable
posets then $f\sqsubseteq g\Leftrightarrow\suprel f\subseteq\suprel g$.\end{thm}
\begin{proof}
From the theorem \ref{one-funcoid}.\end{proof}
\begin{prop}
If $\mathfrak{A}$ and $\mathfrak{B}$ have least elements, then $\mathsf{pFCD}(\mathfrak{A};\mathfrak{B})$
has least element.\end{prop}
\begin{proof}
It is $(\mathfrak{A};\mathfrak{B};\mathfrak{A}\times\{\bot^{\mathfrak{B}}\};\mathfrak{B}\times\{\bot^{\mathfrak{A}}\})$.\end{proof}
\begin{thm}
\label{pf-join-core}Let $(\mathfrak{A};\mathfrak{Z}_{0})$ and $(\mathfrak{B};\mathfrak{Z}_{1})$
be primary filtrators over boolean lattices. Then for $R\in\subsets\mathsf{pFCD}(\mathfrak{A};\mathfrak{B})$
and $X\in\mathfrak{Z}_{0}$, $Y\in\mathfrak{Z}_{1}$ we have:
\begin{enumerate}
\item \label{pf-join-r}$X\suprel{\bigsqcup R}Y\Leftrightarrow\exists f\in R:X\suprel fY$;
\item \label{pf-join-f}$\supfun{\bigsqcup R}X=\bigsqcup_{f\in R}\supfun fX$.
\end{enumerate}
\end{thm}
\begin{proof}
~
\begin{widedisorder}
\item [{\ref{pf-join-f}}] $\alpha X\eqdef\bigsqcup_{f\in R}\supfun fX$
(by corollary \ref{filt-is-complete} all joins on $\mathfrak{B}$
exist). We have $\alpha\bot^{\mathfrak{A}}=\bot^{\mathfrak{B}}$;
\begin{align*}
\alpha(I\sqcup^{\mathfrak{Z}_{0}}J) & =\\
\bigsqcup\setcond{\supfun f(I\sqcup^{\mathfrak{Z}_{0}}J)}{f\in R} & =\\
\bigsqcup\setcond{\supfun f(I\sqcup^{\mathfrak{A}}J)}{f\in R} & =\\
\bigsqcup\setcond{\supfun fI\sqcup^{\mathfrak{B}}\supfun fJ}{f\in R} & =\\
\bigsqcup\setcond{\supfun fI}{f\in R}\sqcup^{\mathfrak{B}}\bigsqcup\setcond{\supfun fJ}{f\in R} & =\\
\alpha I\sqcup^{\mathfrak{B}}\alpha J
\end{align*}
(used theorem \ref{pf-dist-func}). By theorem \ref{pf-cont} the
function $\alpha$ can be continued to $\supfun h$ for an $h\in\mathsf{pFCD}(\mathfrak{A};\mathfrak{B})$.
Obviously
\begin{equation}
\forall f\in R:h\sqsupseteq f.\label{pf-fcd-bigcup-least}
\end{equation}
And $h$ is the least element of $\mathsf{pFCD}(\mathfrak{A};\mathfrak{B})$
for which the condition (\ref{pf-fcd-bigcup-least}) holds. So $h=\bigsqcup R$.

\begin{widedisorder}
\item [{\ref{pf-join-r}}] ~
\begin{align*}
X\suprel{\bigsqcup R}Y & \Leftrightarrow\\
Y\sqcap^{\mathfrak{B}}\supfun{\bigsqcup R}X\ne\bot^{\mathfrak{B}} & \Leftrightarrow\\
Y\sqcap^{\mathfrak{B}}\bigsqcup\setcond{\supfun fX}{f\in R}\ne\bot^{\mathfrak{B}} & \Leftrightarrow\\
\exists f\in R:Y\sqcap^{\mathfrak{B}}\supfun fX\ne\bot^{\mathfrak{B}} & \Leftrightarrow\\
\exists f\in R:X\suprel fY
\end{align*}
(used theorem \ref{b-f-back-distr}).
\end{widedisorder}
\end{widedisorder}
\end{proof}
\begin{cor}
\label{pf-fcd-compl}If $(\mathfrak{A};\mathfrak{Z}_{0})$ and $(\mathfrak{B};\mathfrak{Z}_{1})$
are primary filtrators over boolean lattices then $\mathsf{pFCD}(\mathfrak{A};\mathfrak{B})$
is a complete lattice.\end{cor}
\begin{proof}
Apply \cite{pm:complete-lattice-criteria}.\end{proof}
\begin{thm}
\label{pf-fin-join}Let $\mathfrak{A}$ and $\mathfrak{B}$ be starrish
join-semilattices. Then for $f,g\in\mathsf{pFCD}(\mathfrak{A};\mathfrak{B})$:
\begin{enumerate}
\item \label{pf-fin-j-f}$\supfun{f\sqcup g}x=\supfun fx\sqcup\supfun gx$
for every $x\in\mathfrak{A}$;
\item \label{pf-fin-j-r}$\suprel{f\sqcup g}=\suprel f\cup\suprel g$.
\end{enumerate}
\end{thm}
\begin{proof}
~\end{proof}
\begin{widedisorder}
\item [{\ref{pf-fin-j-f}}] Let $\alpha\mathcal{X}\eqdef\supfun fx\sqcup\supfun gx$;
$\beta\mathcal{Y}\eqdef\supfun{f^{-1}}y\sqcup\supfun{g^{-1}}y$ for
every $x\in\mathfrak{A}$, $y\in\mathfrak{B}$. Then
\begin{align*}
y\nasymp^{\mathfrak{B}}\alpha x & \Leftrightarrow\\
y\nasymp\supfun fx\lor y\nasymp\supfun gx & \Leftrightarrow\\
x\nasymp\supfun{f^{-1}}y\lor x\nasymp\supfun{g^{-1}}y & \Leftrightarrow\\
x\nasymp\supfun{f^{-1}}y\sqcup\supfun{g^{-1}}y & \Leftrightarrow\\
x\nasymp\beta y.
\end{align*}
So $h=(\mathfrak{A};\mathfrak{B};\alpha;\beta)$ is a pointfree funcoid.
Obviously $h\sqsupseteq f$ and $h\sqsupseteq g$. If $p\sqsupseteq f$
and $p\sqsupseteq g$ for some $p\in\mathsf{pFCD}(\mathfrak{A};\mathfrak{B})$
then $\supfun px\sqsupseteq\supfun fx\sqcup\supfun gx=\supfun hx$
and $\supfun{p^{-1}}y\sqsupseteq\supfun{f^{-1}}y\sqcup\supfun{g^{-1}}y=\supfun{h^{-1}}y$
that is $p\sqsupseteq h$. So $f\sqcup g=h$.
\item [{\ref{pf-fin-j-r}}] ~
\begin{align*}
x\suprel{f\sqcup g}y & \Leftrightarrow\\
y\nasymp\supfun{f\sqcup g}x & \Leftrightarrow\\
y\nasymp\supfun fx\sqcup\supfun gx & \Leftrightarrow\\
y\nasymp\supfun fx\lor y\nasymp\supfun gx & \Leftrightarrow\\
x\suprel fy\lor x\suprel gy
\end{align*}
for every $x\in\mathfrak{A}$, $y\in\mathfrak{B}$.\end{widedisorder}

\section{Domain and range of a pointfree funcoid}
\begin{defn}
\index{funcoid!pointfree!identity}Let $\mathfrak{A}$ be a poset.
The \emph{identity pointfree funcoid} $1_{\mathfrak{A}}^{\mathsf{pFCD}}=(\mathfrak{A};\mathfrak{A};\id_{\mathfrak{A}};\id_{\mathfrak{A}})$.
\end{defn}
It is trivial that identity funcoid is really a pointfree funcoid.

Let now $\mathfrak{A}$ be a meet-semilattice.
\begin{defn}
\index{funcoid!pointfree!restricted identity}Let $a\in\mathfrak{A}$.
The \emph{restricted identity pointfree funcoid} $\id_{a}^{\mathsf{pFCD}(\mathfrak{A})}=(\mathfrak{A};\mathfrak{A};a\sqcap^{\mathfrak{A}};a\sqcap^{\mathfrak{A}})$.\end{defn}
\begin{prop}
The restricted pointfree funcoid is a pointfree funcoid.\end{prop}
\begin{proof}
We need to prove that $(a\sqcap^{\mathfrak{A}}x)\nasymp^{\mathfrak{A}}y\Leftrightarrow(a\sqcap^{\mathfrak{A}}y)\nasymp^{\mathfrak{A}}x$
what is obvious.\end{proof}
\begin{obvious}
$(\id_{a}^{\mathsf{pFCD}(\mathfrak{A})})^{-1}=\id_{a}^{\mathsf{pFCD}(\mathfrak{A})}$.
\end{obvious}

\begin{obvious}
$x\suprel{\id_{a}^{\mathsf{pFCD}(\mathfrak{A})}}y\Leftrightarrow a\nasymp^{\mathfrak{A}}x\sqcap^{\mathfrak{A}}y$
for every $x,y\in\mathfrak{A}$.\end{obvious}
\begin{defn}
\index{funcoid!pointfree!restricting}I will define \emph{restricting}
of a pointfree funcoid $f$ to an element $a\in\Src f$ by the formula
$f|_{a}\eqdef f\circ\id_{a}^{\mathsf{pFCD}(\Src f)}$.
\end{defn}

\begin{defn}
\index{funcoid!pointfree!image}Let $f$ be a pointfree funcoid whose
source is a set with greatest element. \emph{Image} of $f$ will be
defined by the formula $\im f=\supfun f\top$.\end{defn}
\begin{prop}
$\im f\sqsupseteq fx$ for every $x\in\Src f$ whenever $\Dst f$
is a separable meet-semilattice with greatest element.\end{prop}
\begin{proof}
$\supfun f\top$ is greater than every $\supfun fx$ (where $x\in\Src f$)
by proposition~\ref{pfcd-mono}.\end{proof}
\begin{defn}
\index{funcoid!pointfree!domain}\emph{Domain} of a pointfree funcoid
$f$ is defined by the formula $\dom f=\im f^{-1}$.\end{defn}
\begin{prop}
$\supfun f\dom f=\im f$ if $f$ is a pointfree funcoid and both $\Src f$
and $\Dst f$ have greatest element and both $\Src f$
and $\Dst f$ are separable posets.\end{prop}
\begin{proof}
For every $y\in\Dst f$
\begin{multline*}
y\nasymp\supfun f\dom f\Leftrightarrow\dom f\nasymp\supfun{f^{-1}}y\Leftrightarrow\supfun{f^{-1}}\top\nasymp\supfun{f^{-1}}y\Leftrightarrow
\text{(by separability of $\Src f$)}\\
\supfun{f^{-1}}y\ne\bot\Leftrightarrow\top\nasymp\supfun{f^{-1}}y\Leftrightarrow y\nasymp\supfun f\top\Leftrightarrow y\nasymp\im f.
\end{multline*}
So $\supfun f\dom f=\im f$ by separability of $\Dst f$.\end{proof}
\begin{prop}
$\supfun fx=\supfun f(x\sqcap\dom f)$
for every $x\in\Src f$ for a pointfree funcoid $f$ whose source
is a bounded separable meet-semilattice and destination is a bounded separable poset.\end{prop}
\begin{proof}
For every $y\in\Dst f$ we have
\begin{multline*}
y\nasymp\supfun f(x\sqcap\dom f)\Leftrightarrow x\sqcap\dom f\sqcap\supfun{f^{-1}}y\ne\bot^{\Src f}\Leftrightarrow\\
x\sqcap\im f^{-1}\sqcap\supfun{f^{-1}}y\ne\bot^{\Src f}\Leftrightarrow x\sqcap\supfun{f^{-1}}y\ne\bot^{\Src f}\Leftrightarrow y\nasymp\supfun fx.
\end{multline*}
Thus $\supfun fx=\supfun f(x\sqcap\dom f)$ by separability of $\Dst f$.\end{proof}
\begin{prop}
$x\nasymp\dom f\Leftrightarrow(\supfun fx\text{ is not least})$ for
every pointfree funcoid $f$ and $x\in\Src f$ if $\Dst f$ has greatest
element $\top$ and $\Src f$ is a separable poset.\end{prop}
\begin{proof}
$x\nasymp\dom f\Leftrightarrow x\nasymp\supfun{f^{-1}}\top^{\Dst f}\Leftrightarrow\top^{\Dst f}\nasymp\supfun fx\Leftrightarrow(\supfun fx\text{ is not least})$.\end{proof}
\begin{prop}
$\dom f=\bigsqcup\setcond{a\in\atoms^{\Src f}}{\supfun fa\ne\bot^{\Dst f}}$
for every pointfree funcoid $f$ whose destination is a bounded separable meet-semilattice
and source is an atomistic poset.\end{prop}
\begin{proof}
For every $a\in\atoms^{\Src f}$ we have 
\begin{multline*}
a\nasymp\dom f\Leftrightarrow a\nasymp\supfun{f^{-1}}\top^{\Dst f}\Leftrightarrow\top^{\Dst f}\nasymp\supfun fa\Leftrightarrow\\
\text{(because $\Dst f$ is a separable meet-semilattice)}\Leftrightarrow\supfun fa\ne\bot^{\Dst f}.
\end{multline*}
So $\dom f=\bigsqcup\setcond{a\in\atoms^{\Src f}}{a\nasymp\dom f}=\bigsqcup\setcond{a\in\atoms^{\Src f}}{\supfun fa\ne\bot^{\Dst f}}$.\end{proof}
\begin{prop}
$\dom(f|_{a})=a\sqcap\dom f$ for every pointfree funcoid $f$ and
$a\in\Src f$ where $\Src f$ is a meet-semilattice and
$\Dst f$ has greatest element.\end{prop}
\begin{proof}
~
\begin{multline*}
\dom(f|_{a})=\im(\id_{a}^{\mathsf{pFCD}(\Src f)}\circ f^{-1})=\\
\supfun{\id_{a}^{\mathsf{pFCD}(\Src f)}}\supfun{f^{-1}}\top^{\Dst f}=a\sqcap\supfun{f^{-1}}\top^{\Dst f}=a\sqcap\dom f.
\end{multline*}
\end{proof}
\begin{prop}
For every composable pointfree funcoids $f$ and $g$
\begin{enumerate}
\item \label{pf-im-ge-dom}If $\im f\sqsupseteq\dom g$ then $\im(g\circ f)=\im g$, provided that
the posets $\Src f$ and $\Dst f=\Src g$ have greatest elements and $\Dst f$
and $\Dst g$ are separable.
\item \label{pf-im-le-dom}If $\im f\sqsubseteq\dom g$ then $\dom(g\circ f)=\dom g$, provided that
the posets $\Dst g$ and $\Dst f=\Src g$ have greatest elements and $\Src g$
and $\Src f$ are separable.
\end{enumerate}
\end{prop}
\begin{proof}
~
\begin{widedisorder}
\item [{\ref{pf-im-ge-dom}}] $\im(g\circ f)=\supfun{g\circ f}\top^{\Src f}=\supfun g\supfun f\top^{\Src f}=\supfun g\im f=\supfun g\dom g=\supfun g\top^{\Src g}=\im g$.
\item [{\ref{pf-im-le-dom}}] $\dom(g\circ f)=\im(f^{-1}\circ g^{-1})$
what by the proved is equal to $\im f^{-1}$ that is $\dom f$.
\end{widedisorder}
\end{proof}

\section{Category of pointfree funcoids}

\index{category!of pointfree funcoids}I will define the category
$\mathsf{pFCD}$ of pointfree funcoids:
\begin{itemize}
\item The class of objects are small posets.
\item The set of morphisms from $\mathfrak{A}$ to $\mathfrak{B}$ is $\mathsf{pFCD}(\mathfrak{A};\mathfrak{B})$.
\item The composition is the composition of pointfree funcoids.
\item Identity morphism for an object $\mathfrak{A}$ is $(\mathfrak{A};\mathfrak{A};\id_{\mathfrak{A}};\id_{\mathfrak{A}})$.
\end{itemize}
To prove that it is really a category is trivial.

\index{category!of pointfree funcoid triples}The \emph{category of
pointfree funcoid quintuples} is defined as follows:
\begin{itemize}
\item Objects are pairs $(\mathfrak{A};\mathcal{A})$ where $\mathfrak{A}$
is a small meet-semilattice with least element and $\mathcal{A}\in\mathfrak{A}$.
\item The morphisms from an object $(\mathfrak{A};\mathcal{A})$ to an object
$(\mathfrak{B};\mathcal{B})$ are tuples $(\mathfrak{A};\mathfrak{B};\mathcal{A};\mathcal{B};f)$
where $f\in\mathsf{pFCD}(\mathfrak{A};\mathfrak{B})$ and $f\sqsubseteq\mathcal{A}\times^{\mathsf{FCD}}\mathcal{B}$.
(Note: least elements are required for $\mathcal{A}\times^{\mathsf{FCD}}\mathcal{B}$
to be properly defined.)
\item The composition is defined by the formula $(\mathcal{B};\mathcal{C};g)\circ(\mathcal{A};\mathcal{B};f)=(\mathcal{A};\mathcal{C};g\circ f)$.
\item Identity morphism for an object $(\mathfrak{A};\mathcal{A})$ is $\id_{\mathcal{A}}^{\mathsf{pFCD}(\mathfrak{A})}$.
(Note: this is defined only for meet-semilattices.)
\end{itemize}
To prove that it is really a category is trivial.


\section{Specifying funcoids by functions or relations on atomic filters}
\begin{thm}
\label{spfn-atoms}Let $\mathfrak{A}$ be an atomic poset and $(\mathfrak{B};\mathfrak{Z}_{1})$
is a primary filtrator over a boolean lattice. Then for every $f\in\mathsf{pFCD}(\mathfrak{A};\mathfrak{B})$
and $\mathcal{X}\in\mathfrak{A}$ we have
\[
\supfun f\mathcal{X}=\bigsqcup^{\mathfrak{B}}\rsupfun{\supfun f}\atoms^{\mathfrak{A}}\mathcal{X}.
\]
\end{thm}
\begin{proof}
For every $Y\in\mathfrak{Z}_{1}$ we have
\begin{multline*}
Y\nasymp^{\mathfrak{B}}\supfun f\mathcal{X}\Leftrightarrow\mathcal{X}\nasymp^{\mathfrak{A}}\supfun{f^{-1}}Y\Leftrightarrow\\
\exists x\in\atoms^{\mathfrak{A}}\mathcal{X}:x\nasymp^{\mathfrak{A}}\supfun{f^{-1}}Y\Leftrightarrow\exists x\in\atoms^{\mathfrak{A}}\mathcal{X}:Y\nasymp^{\mathfrak{B}}\supfun fx.
\end{multline*}
Thus $\corestar\supfun f\mathcal{X}=\bigcup\rsupfun{\corestar}\rsupfun{\supfun f}\atoms^{\mathfrak{A}}\mathcal{X}=\corestar\bigsqcup^{\mathfrak{B}}\rsupfun{\supfun f}\atoms^{\mathfrak{A}}\mathcal{X}$
(used corollary~\ref{d-f-join}). Consequently $\supfun f\mathcal{X}=\bigsqcup^{\mathfrak{B}}\rsupfun{\supfun f}\atoms^{\mathfrak{A}}\mathcal{X}$
by the corollary \ref{d-inj}.\end{proof}
\begin{prop}
Let $f$ be a pointfree funcoid. Then for every $\mathcal{X}\in\Src f$
and $\mathcal{Y}\in\Dst f$
\begin{enumerate}
\item $\mathcal{X}\suprel f\mathcal{Y}\Leftrightarrow\exists x\in\atoms\mathcal{X}:x\suprel f\mathcal{Y}$
if $\Src f$ is an atomic poset.
\item $\mathcal{X}\suprel f\mathcal{Y}\Leftrightarrow\exists y\in\atoms\mathcal{Y}:\mathcal{X}\suprel fy$
if $\Dst f$ is an atomic poset.
\end{enumerate}
\end{prop}
\begin{proof}
I will prove only the second as the first is similar.

If $\mathcal{X}\suprel f\mathcal{Y}$, then $\mathcal{Y}\nasymp\supfun f\mathcal{X}$,
consequently exists $y\in\atoms\mathcal{Y}$ such that $y\nasymp\supfun f\mathcal{X}$,
$\mathcal{X}\suprel fy$. The reverse is obvious.\end{proof}
\begin{cor}
\label{pf-suprel-atoms}If $f$ is a pointfree funcoid with both source
and destination being atomic posets, then for every $\mathcal{X}\in\Src f$
and $\mathcal{Y}\in\Dst f$
\[
\mathcal{X}\suprel f\mathcal{Y}\Leftrightarrow\exists x\in\atoms\mathcal{X},y\in\atoms\mathcal{Y}:x\suprel fy.
\]
\end{cor}
\begin{proof}
Apply the theorem twice.\end{proof}
\begin{cor}
If $\mathfrak{A}$ is a separable atomic poset and $\mathfrak{B}$
is a separable poset then $f\in\mathsf{pFCD}(\mathfrak{A};\mathfrak{B})$
is determined by the values of $\supfun fX$ for $X\in\atoms^{\mathfrak{A}}$.\end{cor}
\begin{proof}
~
\[
y\nasymp\supfun fx\Leftrightarrow x\nasymp\supfun{f^{-1}}y\Leftrightarrow\exists X\in\atoms x:X\nasymp\supfun{f^{-1}}y\Leftrightarrow\exists X\in\atoms x:y\nasymp\supfun fX.
\]
Thus by separability of $\mathfrak{B}$ we have $\supfun f$ is determined
by $\supfun fX$ for $X\in\atoms x$.

By separability of $\mathfrak{A}$ we infer that $f$ can be restored
from $\supfun f$ (theorem \ref{one-funcoid}).\end{proof}
\begin{thm}
\label{pf-atom-cont}Let $(\mathfrak{A};\mathfrak{Z}_{0})$ and $(\mathfrak{B};\mathfrak{Z}_{1})$
be primary filtrators over boolean lattices.
\begin{enumerate}
\item \label{pf-at-f}A function $\alpha\in\mathfrak{B}^{\atoms^{\mathfrak{A}}}$
such that (for every $a\in\atoms^{\mathfrak{A}}$)
\begin{equation}
\alpha a\sqsubseteq\bigsqcap\rsupfun{\bigsqcup\circ\rsupfun{\alpha}\circ\atoms^{\mathfrak{A}}}\up^{\mathfrak{Z}_{0}}a\label{pf-alpha-a-greater}
\end{equation}
can be continued to the function $\supfun f$ for a unique $f\in\mathsf{pFCD}(\mathfrak{A};\mathfrak{B})$;
\begin{equation}
\supfun f\mathcal{X}=\bigsqcup\rsupfun{\alpha}\atoms^{\mathfrak{A}}\mathcal{X}\label{pf-atoms-fun-cont}
\end{equation}
for every $\mathcal{X}\in\mathfrak{A}$.
\item \label{pf-at-r}A relation $\delta\in\subsets(\atoms^{\mathfrak{A}}\times\atoms^{\mathfrak{B}})$
such that (for every $a\in\atoms^{\mathfrak{A}}$, $b\in\atoms^{\mathfrak{B}}$)
\begin{equation}
\forall X\in\up^{\mathfrak{Z}_{0}}a,Y\in\up^{\mathfrak{Z}_{1}}b\exists x\in\atoms^{\mathfrak{A}}X,y\in\atoms^{\mathfrak{B}}Y:x\mathrel{\delta}y\Rightarrow a\mathrel{\delta}b\label{pf-atoms-delta}
\end{equation}
can be continued to the relation $\suprel f$ for a unique $f\in\mathsf{pFCD}(\mathfrak{A};\mathfrak{B})$;
\begin{equation}
\mathcal{X}\suprel f\mathcal{Y}\Leftrightarrow\exists x\in\atoms\mathcal{X},y\in\atoms\mathcal{Y}:x\mathrel{\delta}y\label{pf-atoms-rel-cont}
\end{equation}
for every $\mathcal{X}\in\mathfrak{A}$, $\mathcal{Y}\in\mathfrak{B}$.
\end{enumerate}
\end{thm}
\begin{proof}
Existence of no more than one such funcoids and formulas (\ref{pf-atoms-fun-cont})
and (\ref{pf-atoms-rel-cont}) follow from the theorem \ref{pf-fcd-atom-args}
and corollary \ref{pf-relatom-both} and the fact that our filtrators
are with separable core.
\begin{widedisorder}
\item [{\ref{pf-at-f}}] Consider the function $\alpha'\in\mathfrak{B}^{\mathfrak{Z}_{0}}$
defined by the formula (for every $X\in\mathfrak{Z}_{0}$) 
\[
\alpha'X=\bigsqcup\rsupfun{\alpha}\atoms^{\mathfrak{A}}X.
\]
Obviously $\alpha'\bot^{\mathfrak{Z}_{0}}=\bot^{\mathfrak{B}}$. For every
$I,J\in\mathfrak{Z}_{0}$
\begin{align*}
\alpha'(I\sqcup J) & =\\
\bigsqcup\rsupfun{\alpha}\atoms^{\mathfrak{A}}(I\sqcup J) & =\\
\bigsqcup\rsupfun{\alpha}(\atoms^{\mathfrak{A}}I\cup\atoms^{\mathfrak{A}}J) & =\\
\bigsqcup(\rsupfun{\alpha}\atoms^{\mathfrak{A}}I\cup\rsupfun{\alpha}\atoms^{\mathfrak{A}}J) & =\\
\bigsqcup\rsupfun{\alpha}\atoms^{\mathfrak{A}}I\sqcup\bigsqcup\rsupfun{\alpha}\atoms^{\mathfrak{A}}J & =\\
\alpha'I\sqcup\alpha'J.
\end{align*}



Let continue $\alpha'$ till a pointfree funcoid $f$ (by the theorem
\ref{pf-cont}): $\supfun f\mathcal{X}=\bigsqcap\rsupfun{\alpha'}\up^{\mathfrak{Z}_{0}}\mathcal{X}$.


Let's prove the reverse of (\ref{pf-alpha-a-greater}):
\begin{align*}
\bigsqcap\rsupfun{\bigsqcup\circ\rsupfun{\alpha}\circ\atoms^{\mathfrak{A}}}\up^{\mathfrak{Z}_{0}}a & =\\
\bigsqcap\rsupfun{\bigsqcup\circ\rsupfun{\alpha}}\rsupfun{\atoms^{\mathfrak{A}}}\up^{\mathfrak{Z}_{0}}a & \sqsubseteq\\
\bigsqcap\rsupfun{\bigsqcup\circ\rsupfun{\alpha}}\rsupfun{\atoms^{\mathfrak{A}}}\{\{a\}\} & =\\
\bigsqcap\left\{ \left(\bigsqcup\circ\rsupfun{\alpha}\right)\{a\}\right\}  & =\\
\bigsqcap\left\{ \bigsqcup\rsupfun{\alpha}\{a\}\right\}  & =\\
\bigsqcap\left\{ \bigsqcup\{\alpha a\}\right\}  & =\\
\bigsqcap\{\alpha a\} & =\alpha a.
\end{align*}



Finally,
\[
\alpha a=\bigsqcap\rsupfun{\bigsqcup\circ\rsupfun{\alpha}\circ\atoms^{\mathfrak{A}}}\up^{\mathfrak{Z}_{0}}a=\bigsqcap\rsupfun{\alpha'}\up^{\mathfrak{Z}_{0}}a=\supfun fa,
\]
so $\supfun f$ is a continuation of $\alpha$.

\item [{\ref{pf-at-r}}] Consider the relation $\delta'\in\subsets(\mathfrak{Z}_{0}\times\mathfrak{Z}_{1})$
defined by the formula (for every $X\in\mathfrak{Z}_{0}$, $Y\in\mathfrak{Z}_{1}$)
\[
X\mathrel{\delta'}Y\Leftrightarrow\exists x\in\atoms^{\mathfrak{A}}X,y\in\atoms^{\mathfrak{B}}Y:x\mathrel{\delta}y.
\]
Obviously $\neg(X\mathrel{\delta'}\bot^{\mathfrak{Z}_{1}})$ and $\neg(\bot^{\mathfrak{Z}_{0}}\mathrel{\delta'}Y)$.
\begin{align*}
I\sqcup J\mathrel{\delta'}Y & \Leftrightarrow\\
\exists x\in\atoms^{\mathfrak{A}}(I\sqcup J),y\in\atoms^{\mathfrak{B}}Y:x\mathrel{\delta}y & \Leftrightarrow\\
\exists x\in\atoms^{\mathfrak{A}}I\cup\atoms^{\mathfrak{A}}J,y\in\atoms^{\mathfrak{B}}Y:x\mathrel{\delta}y & \Leftrightarrow\\
\exists x\in\atoms^{\mathfrak{A}}I,y\in\atoms^{\mathfrak{B}}Y:x\mathrel{\delta}y\lor\exists x\in\atoms^{\mathfrak{A}}J,y\in\atoms^{\mathfrak{B}}Y:x\mathrel{\delta}y & \Leftrightarrow\\
I\mathrel{\delta'}Y\lor J\mathrel{\delta'}Y;
\end{align*}
similarly $X\mathrel{\delta'}I\sqcup J\Leftrightarrow X\mathrel{\delta'}I\lor X\mathrel{\delta'}J$.
Let's continue $\delta'$ till a funcoid $f$ (by the theorem \ref{pf-cont}):
\[
\mathcal{X}\suprel f\mathcal{Y}\Leftrightarrow\forall X\in\up^{\mathfrak{Z}_{0}}\mathcal{X},Y\in\up^{\mathfrak{Z}_{1}}\mathcal{Y}:X\mathrel{\delta'}Y.
\]



The reverse of (\ref{pf-atoms-delta}) implication is trivial, so
\[
\forall X\in\up^{\mathfrak{Z}_{0}}a,Y\in\up^{\mathfrak{Z}_{1}}b\exists x\in\atoms^{\mathfrak{A}}X,y\in\atoms^{\mathfrak{B}}Y:x\mathrel{\delta}y\Leftrightarrow a\mathrel{\delta}b.
\]



\begin{align*}
\forall X\in\up^{\mathfrak{Z}_{0}}a,Y\in\up^{\mathfrak{Z}_{1}}b\exists x\in\atoms^{\mathfrak{A}}X,y\in\atoms^{\mathfrak{B}}Y:x\mathrel{\delta}y & \Leftrightarrow\\
\forall X\in\up^{\mathfrak{Z}_{0}}a,Y\in\up^{\mathfrak{Z}_{1}}b:X\mathrel{\delta'}Y & \Leftrightarrow\\
a\suprel fb.
\end{align*}
So $a\mathrel\delta b\Leftrightarrow a\suprel fb$, that is $\suprel f$
is a continuation of $\delta$.

\end{widedisorder}
\end{proof}
\begin{thm}
\label{pf-meet-atom}Let $(\mathfrak{A};\mathfrak{Z}_{0})$ and $(\mathfrak{B};\mathfrak{Z}_{1})$
be primary filtrators over boolean lattices. If $R\in\subsets\mathsf{pFCD}(\mathfrak{A};\mathfrak{B})$
and $x\in\atoms^{\mathfrak{A}}$, $y\in\atoms^{\mathfrak{B}}$, then
\begin{enumerate}
\item \label{pf-meet-at-f}$\supfun{\bigsqcap R}x=\bigsqcap_{f\in R}\supfun fx$;
\item \label{pf-meet-at-r}$x\suprel{\bigsqcap R}y\Leftrightarrow\forall f\in R:x\suprel fy$.
\end{enumerate}
\end{thm}
\begin{proof}
~
\begin{widedisorder}
\item [{\ref{pf-meet-at-r}}] Let denote $x\mathrel\delta y\Leftrightarrow\forall f\in R:x\suprel fy$.
\begin{gather*}
\forall X\in\up^{\mathfrak{Z}_{0}}a,Y\in\up^{\mathfrak{Z}_{1}}b\exists x\in\atoms^{\mathfrak{A}}X,y\in\atoms^{\mathfrak{B}}Y:x\mathrel\delta y\Rightarrow\\
\forall f\in R,X\in\up^{\mathfrak{Z}_{0}}a,Y\in\up^{\mathfrak{Z}_{1}}b\exists x\in\atoms^{\mathfrak{A}}X,y\in\atoms^{\mathfrak{B}}Y:x\suprel fy\Rightarrow\\
\forall f\in R,X\in\up^{\mathfrak{Z}_{0}}a,Y\in\up^{\mathfrak{Z}_{1}}b:X\suprel fY\Rightarrow\\
\forall f\in R:a\suprel fb\Leftrightarrow\\
a\mathrel\delta b.
\end{gather*}
So by the theorem \ref{pf-atom-cont}, $\delta$ can be continued
till $\suprel p$ for some $p\in\mathsf{pFCD}(\mathfrak{A};\mathfrak{B})$.


For every $q\in\mathsf{pFCD}(\mathfrak{A};\mathfrak{B})$ such that
$\forall f\in R:q\sqsubseteq f$ we have $x\suprel qy\Rightarrow\forall f\in R:x\suprel fy\Leftrightarrow x\mathrel{\delta}y\Leftrightarrow x\suprel py$,
so $q\sqsubseteq p$. Consequently $p=\bigsqcap R$.


From this $x\suprel{\bigsqcap R}y\Leftrightarrow\forall f\in R:x\suprel fy$.

\item [{\ref{pf-meet-at-f}}] From the former 
\begin{multline*}
y\in\atoms^{\mathfrak{B}}\supfun{\bigsqcap R}x\Leftrightarrow y\sqcap\supfun{\bigsqcap R}x\ne\bot^{\mathfrak{B}}\Leftrightarrow\forall f\in R:y\sqcap\supfun fx\ne\bot^{\mathfrak{B}}\Leftrightarrow\\
y\in\bigcap\rsupfun{\atoms^{\mathfrak{B}}}\setcond{\supfun fx}{f\in R}\Leftrightarrow y\in\atoms\bigsqcap\setcond{\supfun fx}{f\in R}
\end{multline*}
for every $y\in\atoms^{\mathfrak{B}}$.


$\mathfrak{B}$ is atomically separable by the corollary \ref{f-atom-sep}.
Thus $\supfun{\bigsqcap R}x=\bigsqcap_{f\in R}\supfun fx$.

\end{widedisorder}
\end{proof}

\section{More on composition of pointfree funcoids}
\begin{prop}
$\mathord{\suprel{g\circ f}}=\mathord{\suprel g}\circ\supfun f=\supfun{g^{-1}}^{-1}\circ\mathord{\suprel f}$
for every composable pointfree funcoids $f$ and $g$.\end{prop}
\begin{proof}
For every $x\in\mathfrak{A}$, $y\in\mathfrak{B}$
\[
x\suprel{g\circ f}y\Leftrightarrow y\nasymp\supfun{g\circ f}x\Leftrightarrow y\nasymp\supfun g\supfun fx\Leftrightarrow\supfun fx\suprel gy\Leftrightarrow x\mathrel{(\mathord{\suprel g}\circ\supfun f)}y.
\]
Thus $\mathord{\suprel{g\circ f}}=\mathord{\suprel g}\circ\supfun f$.
\[
\mathord{\suprel{g\circ f}}=\mathord{\suprel{(f^{-1}\circ g^{-1})^{-1}}}=\mathord{\suprel{f^{-1}\circ g^{-1}}}^{-1}=(\mathord{\suprel{f^{-1}}}\circ\supfun{g^{-1}})^{-1}=\supfun{g^{-1}}^{-1}\circ\mathord{\suprel f}.
\]
\end{proof}
\begin{thm}
\label{pf-fcd-atom-middle}Let $f$ and $g$ be pointfree funcoids
and $\mathfrak{A}=\Dst f=\Src g$ is an atomic poset. Then for every
$\mathcal{X}\in\Src f$ and $\mathcal{Z}\in\Dst g$
\[
\mathcal{X}\suprel{g\circ f}\mathcal{Z}\Leftrightarrow\exists y\in\atoms^{\mathfrak{A}}:(\mathcal{X}\suprel fy\land y\suprel g\mathcal{Z}).
\]
\end{thm}
\begin{proof}
~
\begin{align*}
\exists y\in\atoms^{\mathfrak{A}}:(\mathcal{X}\suprel fy\land y\suprel g\mathcal{Z}) & \Leftrightarrow\\
\exists y\in\atoms^{\mathfrak{A}}:(\mathcal{Z}\nasymp\supfun gy\land y\nasymp\supfun f\mathcal{X}) & \Leftrightarrow\\
\exists y\in\atoms^{\mathfrak{A}}:(y\nasymp\supfun{g^{-1}}\mathcal{Z}\land y\nasymp\supfun f\mathcal{X}) & \Leftrightarrow\\
\supfun{g^{-1}}\mathcal{Z}\nasymp\supfun f\mathcal{X} & \Leftrightarrow\\
\mathcal{X}\suprel{g\circ f}\mathcal{Z}.
\end{align*}
\end{proof}
\begin{thm}
Let $\mathfrak{A}$, $\mathfrak{B}$, $\mathfrak{C}$ be separable
starrish join-semilattices and $\mathfrak{B}$ is atomic. Then:
\begin{enumerate}
\item $f\circ(g\sqcup h)=f\circ g\sqcup f\circ h$ for $g,h\in\mathsf{pFCD}(\mathfrak{A};\mathfrak{B})$
and $f\in\mathsf{pFCD}(\mathfrak{B};\mathfrak{C})$.
\item $(g\sqcup h)\circ f=g\circ f\sqcup h\circ f$ for $f\in\mathsf{pFCD}(\mathfrak{A};\mathfrak{B})$
and $g,h\in\mathsf{pFCD}(\mathfrak{B};\mathfrak{C})$.
\end{enumerate}
\end{thm}
\begin{proof}
I will prove only the first equality because the other is analogous.

We can apply theorem \ref{pf-fin-join}.

For every $\mathcal{X}\in\mathfrak{A}$, $\mathcal{Y}\in\mathfrak{C}$
\begin{align*}
\mathcal{X}\suprel{f\circ(g\sqcup h)}\mathcal{Z} & \Leftrightarrow\\
\exists y\in\atoms^{\mathfrak{B}}:(\mathcal{X}\suprel{g\sqcup h}y\land y\suprel f\mathcal{Z}) & \Leftrightarrow\\
\exists y\in\atoms^{\mathfrak{B}}:((\mathcal{X}\suprel gy\lor\mathcal{X}\suprel hy)\land y\suprel f\mathcal{Z}) & \Leftrightarrow\\
\exists y\in\atoms^{\mathfrak{B}}:((\mathcal{X}\suprel gy\land y\suprel f\mathcal{Z})\lor(\mathcal{X}\suprel hy\land y\suprel f\mathcal{Z})) & \Leftrightarrow\\
\exists y\in\atoms^{\mathfrak{B}}:(\mathcal{X}\suprel gy\land y\suprel f\mathcal{Z})\lor\exists y\in\atoms^{\mathfrak{B}}:(\mathcal{X}\suprel hy\land y\suprel f\mathcal{Z}) & \Leftrightarrow\\
\mathcal{X}\suprel{f\circ g}\mathcal{Z}\lor\mathcal{X}\suprel{f\circ h}\mathcal{Z} & \Leftrightarrow\\
\mathcal{X}\suprel{f\circ g\sqcup f\circ h}\mathcal{Z}.
\end{align*}
Thus $f\circ(g\sqcup h)=f\circ g\sqcup f\circ h$ by theorem \ref{one-funcoid}.\end{proof}
\begin{thm}
\label{qi-bool}Let $\mathfrak{A}$, $\mathfrak{B}$, $\mathfrak{C}$
be posets of filters over some boolean lattices, $f\in\mathsf{pFCD}(\mathfrak{A};\mathfrak{B})$,
$g\in\mathsf{pFCD}(\mathfrak{B};\mathfrak{C})$, $h\in\mathsf{pFCD}(\mathfrak{A};\mathfrak{C})$.
Then 
\[
g\circ f\nasymp h\Leftrightarrow g\nasymp h\circ f^{-1}.
\]
\end{thm}
\begin{proof}
~
\begin{align*}
g\circ f\nasymp h & \Leftrightarrow\\
\exists a\in\atoms^{\mathfrak{A}},c\in\atoms^{\mathfrak{C}}:a\suprel{(g\circ f)\sqcap h}c & \Leftrightarrow\\
\exists a\in\atoms^{\mathfrak{A}},c\in\atoms^{\mathfrak{C}}:(a\suprel{g\circ f}c\land a\suprel hc) & \Leftrightarrow\\
\exists a\in\atoms^{\mathfrak{A}},b\in\atoms^{\mathfrak{B}},c\in\atoms^{\mathfrak{C}}:(a\suprel fb\land b\suprel gc\land a\suprel hc) & \Leftrightarrow\\
\exists a\in\atoms^{\mathfrak{A}},c\in\atoms^{\mathfrak{C}}:(b\suprel gc\land b\suprel{h\circ f^{-1}}c) & \Leftrightarrow\\
\exists a\in\atoms^{\mathfrak{A}},c\in\atoms^{\mathfrak{C}}:b\suprel{g\sqcap(h\circ f^{-1})}c & \Leftrightarrow\\
g\nasymp h\circ f^{-1}.
\end{align*}

\end{proof}

\section{Funcoidal product of elements}
\begin{defn}
\index{product!funcoidal}\emph{Funcoidal product} $\mathcal{A}\times^{\mathsf{FCD}}\mathcal{B}$
where $\mathcal{A}\in\mathfrak{A}$, $\mathcal{B}\in\mathfrak{B}$
and $\mathfrak{A}$ and $\mathfrak{B}$ are posets with least elements
is a pointfree funcoid such that for every $\mathcal{X}\in\mathfrak{A}$,
$\mathcal{Y}\in\mathfrak{B}$ 
\[
\supfun{\mathcal{A}\times^{\mathsf{FCD}}\mathcal{B}}\mathcal{X}=\left\{ \begin{array}{ll}
\mathcal{B} & \text{if }\mathcal{X}\nasymp\mathcal{A};\\
\bot^{\mathfrak{B}} & \text{if }\mathcal{X}\asymp\mathcal{A};
\end{array}\right.\hspace{1em}\text{and}\hspace{1em}\supfun{(\mathcal{A}\times^{\mathsf{FCD}}\mathcal{B})^{-1}}\mathcal{Y}=\left\{ \begin{array}{ll}
\mathcal{A} & \text{if }\mathcal{Y}\nasymp\mathcal{B};\\
\bot^{\mathfrak{A}} & \text{if }\mathcal{Y}\asymp\mathcal{B}.
\end{array}\right.
\]
\end{defn}
\begin{prop}
$\mathcal{A}\times^{\mathsf{FCD}}\mathcal{B}$ is really a pointfree
funcoid and 
\[
\mathcal{X}\suprel{\mathcal{A}\times^{\mathsf{FCD}}\mathcal{B}}\mathcal{Y}\Leftrightarrow\mathcal{X}\nasymp\mathcal{A}\wedge\mathcal{Y}\nasymp\mathcal{B}.
\]
\end{prop}
\begin{proof}
Obvious.\end{proof}
\begin{prop}
Let $\mathfrak{A}$ and $\mathfrak{B}$ be posets with least elements,
$f\in\mathsf{pFCD}(\mathfrak{A};\mathfrak{B})$, $\mathcal{A}\in\mathfrak{A}$,
$\mathcal{B}\in\mathfrak{B}$. Then
\[
f\sqsubseteq\mathcal{A}\times^{\mathsf{FCD}}\mathcal{B}\Rightarrow\dom f\sqsubseteq\mathcal{A}\wedge\im f\sqsubseteq\mathcal{B}.
\]
\end{prop}
\begin{proof}
If $f\sqsubseteq\mathcal{A}\times^{\mathsf{FCD}}\mathcal{B}$ then
$\dom f\sqsubseteq\dom(\mathcal{A}\times^{\mathsf{FCD}}\mathcal{B})\sqsubseteq\mathcal{A}$,
$\im f\sqsubseteq\im(\mathcal{A}\times^{\mathsf{FCD}}\mathcal{B})\sqsubseteq\mathcal{B}$.\end{proof}
\begin{thm}
Let $\mathfrak{A}$ and $\mathfrak{B}$ be separable bounded posets,
$f\in\mathsf{pFCD}(\mathfrak{A};\mathfrak{B})$, $\mathcal{A}\in\mathfrak{A}$,
$\mathcal{B}\in\mathfrak{B}$. Then
\[
f\sqsubseteq\mathcal{A}\times^{\mathsf{FCD}}\mathcal{B}\Leftrightarrow\dom f\sqsubseteq\mathcal{A}\wedge\im f\sqsubseteq\mathcal{B}.
\]
\end{thm}
\begin{proof}
One direction is the proposition above. The other:

If $\dom f\sqsubseteq\mathcal{A}\wedge\im f\sqsubseteq\mathcal{B}$
then $\mathcal{X}\suprel f\mathcal{Y}\Rightarrow\mathcal{Y}\nasymp\supfun f\mathcal{X}\Rightarrow\mathcal{Y}\nasymp\mathcal{B}$
and similarly $\mathcal{X}\suprel f\mathcal{Y}\Rightarrow\mathcal{X}\nasymp\mathcal{A}$.

So $\mathord{\suprel f}\subseteq\mathord{\suprel{\mathcal{A}\times^{\mathsf{FCD}}\mathcal{B}}}$
and thus using separability $f\sqsubseteq\mathcal{A}\times^{\mathsf{FCD}}\mathcal{B}$.\end{proof}
\begin{thm}
Let $\mathfrak{A}$, $\mathfrak{B}$ be sets of filters over boolean
lattices. For every $f\in\mathsf{pFCD}(\mathfrak{A};\mathfrak{B})$
and $\mathcal{A}\in\mathfrak{A}$, $\mathcal{B}\in\mathfrak{B}$ 
\[
f\sqcap(\mathcal{A}\times^{\mathsf{FCD}}\mathcal{B})=\id_{\mathcal{B}}^{\mathsf{pFCD}(\mathfrak{B})}\circ f\circ\id_{\mathcal{A}}^{\mathsf{pFCD}(\mathfrak{A})}.
\]
\end{thm}
\begin{proof}
From above $\mathsf{pFCD}(\mathfrak{A};\mathfrak{B})$ is a (complete)
lattice.

$h\eqdef\id_{\mathcal{B}}^{\mathsf{pFCD}(\mathfrak{B})}\circ f\circ\id_{\mathcal{A}}^{\mathsf{pFCD}(\mathfrak{A})}$.
For every $\mathcal{X}\in\mathfrak{A}$
\[
\supfun h\mathcal{X}=\supfun{\id_{\mathcal{B}}^{\mathsf{pFCD}(\mathfrak{B})}}\supfun f\supfun{\id_{\mathcal{A}}^{\mathsf{pFCD}(\mathfrak{A})}}\mathcal{X}=\mathcal{B}\sqcap\supfun f(\mathcal{A}\sqcap\mathcal{X})
\]
and
\[
\supfun{h^{-1}}\mathcal{X}=\supfun{\id_{\mathcal{A}}^{\mathsf{pFCD}(\mathfrak{A})}}\supfun{f^{-1}}\supfun{\id_{\mathcal{B}}^{\mathsf{pFCD}(\mathfrak{B})}}\mathcal{X}=\mathcal{A}\sqcap\supfun{f^{-1}}(\mathcal{B}\sqcap\mathcal{X}).
\]
From this, as easy to show, $h\sqsubseteq f$ and $h\sqsubseteq\mathcal{A}\times^{\mathsf{FCD}}\mathcal{B}$.
If $g\sqsubseteq f\wedge g\sqsubseteq\mathcal{A}\times^{\mathsf{FCD}}\mathcal{B}$
for a $g\in\mathsf{pFCD}(\mathfrak{A};\mathfrak{B})$ then $\dom g\sqsubseteq\mathcal{A}$,
$\im g\sqsubseteq\mathcal{B}$,
\[
\supfun g\mathcal{X}=\mathcal{B}\sqcap\supfun g(\mathcal{A}\sqcap\mathcal{X})\sqsubseteq\mathcal{B}\sqcap\supfun f(\mathcal{A}\sqcap\mathcal{X})=\supfun{\id_{\mathcal{B}}^{\mathsf{pFCD}(\mathfrak{B})}}\supfun f\supfun{\id_{\mathcal{A}}^{\mathsf{pFCD}(\mathfrak{A})}}\mathcal{X}=\supfun h\mathcal{X},
\]
and similarly $\supfun{g^{-1}}\mathcal{Y}\sqsubseteq\supfun{h^{-1}}\mathcal{Y}$.

$g\sqsubseteq h$. So $h=f\sqcap(\mathcal{A}\times^{\mathsf{FCD}}\mathcal{B})$.\end{proof}
\begin{cor}
Let $\mathfrak{A}$, $\mathfrak{B}$ be sets of filters over boolean
lattices. For every $f\in\mathsf{pFCD}(\mathfrak{A};\mathfrak{B})$
and $\mathcal{A}\in\mathfrak{A}$ we have $f|_{\mathcal{A}}=f\sqcap(\mathcal{A}\times^{\mathsf{FCD}}\top^{\mathfrak{B}})$.\end{cor}
\begin{proof}
$\text{\ensuremath{f\sqcap(\mathcal{A}\times^{\mathsf{FCD}}\top^{\mathfrak{B}})}}=\id_{\top^{\mathfrak{B}}}^{\mathsf{pFCD}(\mathfrak{B})}\circ f\circ\id_{\mathcal{A}}^{\mathsf{pFCD}(\mathfrak{A})}=f\circ\id_{\mathcal{A}}^{\mathsf{pFCD}(\mathfrak{A})}=f|_{\mathcal{A}}$.\end{proof}
\begin{cor}
\label{pf-intr-prod}Let $\mathfrak{A}$, $\mathfrak{B}$ be sets
of filters over boolean lattices. For every $f\in\mathsf{pFCD}(\mathfrak{A};\mathfrak{B})$
and $\mathcal{A}\in\mathfrak{A}$, $\mathcal{B}\in\mathfrak{B}$ we
have
\[
f\nasymp\mathcal{A}\times^{\mathsf{FCD}}\mathcal{B}\Leftrightarrow\mathcal{A}\suprel f\mathcal{B}.
\]
\end{cor}
\begin{proof}
~
\begin{align*}
f\nasymp\mathcal{A}\times^{\mathsf{FCD}}\mathcal{B} & \Leftrightarrow\\
f\sqcap(\mathcal{A}\times^{\mathsf{FCD}}\mathcal{B})\ne\bot^{\mathsf{pFCD}(\mathfrak{A};\mathfrak{B})} & \Leftrightarrow\\
\supfun{f\sqcap(\mathcal{A}\times^{\mathsf{FCD}}\mathcal{B})}\top^{\mathfrak{A}}\ne\bot^{B} & \Leftrightarrow\\
\supfun{\id_{\mathcal{B}}^{\mathsf{pFCD}(\mathfrak{B})}\circ f\circ\id_{\mathcal{A}}^{\mathsf{pFCD}(\mathfrak{A})}}\top^{\mathfrak{A}}\ne\bot^{B} & \Leftrightarrow\\
\supfun{\id_{\mathcal{B}}^{\mathsf{pFCD}(\mathfrak{B})}}\supfun f\supfun{\id_{\mathcal{A}}^{\mathsf{pFCD}(\mathfrak{A})}}\top^{\mathfrak{A}}\ne\bot^{B} & \Leftrightarrow\\
\mathcal{B}\sqcap\supfun f(\mathcal{A}\sqcap\top^{\mathfrak{A}})\ne\bot^{B} & \Leftrightarrow\\
\mathcal{B}\sqcap\supfun f\mathcal{A}\ne\bot^{B} & \Leftrightarrow\\
\mathcal{A}\suprel f\mathcal{B}.
\end{align*}
\end{proof}
\begin{thm}
Let $\mathfrak{A}$, $\mathfrak{B}$ be sets of filters over boolean
lattices. Then the poset $\mathsf{pFCD}(\mathfrak{A};\mathfrak{B})$
is separable.\end{thm}
\begin{proof}
Let $f,g\in\mathsf{pFCD}(\mathfrak{A};\mathfrak{B})$ and $f\neq g$.
By the theorem \ref{one-funcoid} $\mathord{\suprel f}\neq\mathord{\suprel g}$.
That is there exist $x,y\in\mathfrak{A}$ such that $x\suprel fy\nLeftrightarrow x\suprel gy$
that is $f\sqcap(x\times^{\mathsf{FCD}}y)\neq\bot^{\mathsf{pFCD}(\mathfrak{A};\mathfrak{B})}\nLeftrightarrow g\sqcap(x\times^{\mathsf{FCD}}y)\neq\bot^{\mathsf{pFCD}(\mathfrak{A};\mathfrak{B})}$.
Thus $\mathsf{pFCD}(\mathfrak{A};\mathfrak{B})$ is separable.\end{proof}
\begin{thm}
Let $\mathfrak{A}$ and $\mathfrak{B}$ be posets of filters over
boolean lattices. If $S\in\subsets(\mathfrak{A}\times\mathfrak{B})$
then
\[
\bigsqcap_{(\mathcal{A};\mathcal{B})\in S}(\mathcal{A}\times^{\mathsf{FCD}}\mathcal{B})=\bigsqcap\dom S\times^{\mathsf{FCD}}\bigsqcap\im S.
\]
\end{thm}
\begin{proof}
If $x\in\atoms^{\mathfrak{A}}$ then by the theorem \ref{pf-meet-atom}
\[
\supfun{\bigsqcap_{(\mathcal{A};\mathcal{B})\in S}(\mathcal{A}\times^{\mathsf{FCD}}\mathcal{B})}x=\bigsqcap\setcond{\supfun{\mathcal{A}\times^{\mathsf{FCD}}\mathcal{B}}x}{(\mathcal{A};\mathcal{B})\in S}.
\]
If $x\sqcap\bigsqcap\dom S\ne\bot^{\mathfrak{A}}$ then
\begin{gather*}
\forall(\mathcal{A};\mathcal{B})\in S:(x\sqcap\mathcal{A}\ne\bot^{\mathfrak{A}}\land\supfun{\mathcal{A}\times^{\mathsf{FCD}}\mathcal{B}}x=\mathcal{B});\\
\setcond{\supfun{\mathcal{A}\times^{\mathsf{FCD}}\mathcal{B}}x}{(\mathcal{A};\mathcal{B})\in S}=\im S;
\end{gather*}
if $x\sqcap\bigsqcap\dom S=\bot^{\mathfrak{A}}$ then
\begin{gather*}
\forall(\mathcal{A};\mathcal{B})\in S:(x\sqcap\mathcal{A}=\bot^{\mathfrak{A}}\land\supfun{\mathcal{A}\times^{\mathsf{FCD}}\mathcal{B}}x=\bot^{\mathcal{B}});\\
\setcond{\supfun{\mathcal{A}\times^{\mathsf{FCD}}\mathcal{B}}x}{(\mathcal{A};\mathcal{B})\in S}\ni\bot^{\mathfrak{B}}.
\end{gather*}
So
\[
\supfun{\bigsqcap_{(\mathcal{A};\mathcal{B})\in S}(\mathcal{A}\times^{\mathsf{FCD}}\mathcal{B})}x=\begin{cases}
\bigsqcap\im S & \text{if }x\sqcap\bigsqcap\dom S\ne\bot^{\mathfrak{A}};\\
\bot^{\mathfrak{B}} & \text{if }x\sqcap\bigsqcap\dom S=\bot^{\mathfrak{A}}.
\end{cases}
\]
From this by theorem \ref{pf-atom-cont} the statement of our theorem
follows.\end{proof}
\begin{cor}
Let $\mathfrak{A}$ and $\mathfrak{B}$ be posets of filters over
boolean lattices.

For every $\mathcal{A}_{0},\mathcal{A}_{1}\in\mathfrak{A}$ and $\mathcal{B}_{0},\mathcal{B}_{1}\in\mathfrak{B}$
\[
(\mathcal{A}_{0}\times^{\mathsf{FCD}}\mathcal{B}_{0})\sqcap(\mathcal{A}_{1}\times^{\mathsf{FCD}}\mathcal{B}_{1})=(\mathcal{A}_{0}\sqcap\mathcal{A}_{1})\times^{\mathsf{FCD}}(\mathcal{B}_{0}\sqcap\mathcal{B}_{1}).
\]
\end{cor}
\begin{proof}
$(\mathcal{A}_{0}\times^{\mathsf{FCD}}\mathcal{B}_{0})\sqcap(\mathcal{A}_{1}\times^{\mathsf{FCD}}\mathcal{B}_{1})=\bigsqcap\{\mathcal{A}_{0}\times^{\mathsf{FCD}}\mathcal{B}_{0},\mathcal{A}_{1}\times^{\mathsf{FCD}}\mathcal{B}_{1}\}$
what is by the last theorem equal to $(\mathcal{A}_{0}\sqcap\mathcal{A}_{1})\times^{\mathsf{FCD}}(\mathcal{B}_{0}\sqcap\mathcal{B}_{1})$.\end{proof}
\begin{thm}
Let $(\mathfrak{A};\mathfrak{Z}_{0})$ and $(\mathfrak{B};\mathfrak{Z}_{1})$
be primary filtrators over boolean lattices. If $\mathcal{A}\in\mathfrak{A}$
then $\mathcal{A}\times^{\mathsf{FCD}}$ is a complete homomorphism
of the lattice $\mathfrak{A}$ to a the lattice $\mathsf{pFCD}(\mathfrak{A};\mathfrak{B})$,
if also $\mathcal{A}\neq\bot^{\mathfrak{A}}$ then it is an order
embedding.\end{thm}
\begin{proof}
Let $S\in\mathscr{P}\mathfrak{A}$, $X\in\mathfrak{Z}_{0}$, $x\in\atoms^{\mathfrak{A}}$.
\begin{align*}
\supfun{\bigsqcup\rsupfun{\mathcal{A}\times^{\mathsf{FCD}}}S}X & =\\
\bigsqcup_{\mathcal{B}\in S}\supfun{\mathcal{A}\times^{\mathsf{FCD}}\mathcal{B}}X & =\\
\begin{cases}
\bigsqcup S & \text{if }X\sqcap^{\mathfrak{A}}\mathcal{A}\ne\bot^{\mathfrak{A}}\\
\bot^{\mathfrak{B}} & \text{if }X\sqcap^{\mathfrak{A}}\mathcal{A}=\bot^{\mathfrak{A}}
\end{cases} & =\\
\supfun{\mathcal{A}\times^{\mathsf{FCD}}\bigsqcup S}X.
\end{align*}
Thus $\bigsqcup\rsupfun{\mathcal{A}\times^{\mathsf{FCD}}}S=\mathcal{A}\times^{\mathsf{FCD}}\bigsqcup S$
by theorem \ref{pf-supfun-up}.
\begin{align*}
\supfun{\bigsqcap\rsupfun{\mathcal{A}\times^{\mathsf{FCD}}}S}x & =\\
\bigsqcap_{\mathcal{B}\in S}\supfun{\mathcal{A}\times^{\mathsf{FCD}}\mathcal{B}}x & =\\
\begin{cases}
\bigsqcap S & \text{if }X\sqcap^{\mathfrak{A}}\mathcal{A}\ne\bot^{\mathfrak{A}}\\
\bot^{\mathfrak{B}} & \text{if }X\sqcap^{\mathfrak{A}}\mathcal{A}=\bot^{\mathfrak{A}}
\end{cases} & =\\
\supfun{\mathcal{A}\times^{\mathsf{FCD}}\bigsqcap S}x.
\end{align*}
Thus $\bigsqcap\rsupfun{\mathcal{A}\times^{\mathsf{FCD}}}S=\mathcal{A}\times^{\mathsf{FCD}}\bigsqcap S$
by theorem \ref{spfn-atoms}.

If $\mathcal{A}\neq\bot^{\mathfrak{A}}$ then obviously
$\mathcal{A}\times^{\mathsf{FCD}}\mathcal{X}\sqsubseteq\mathcal{A}\times^{\mathsf{FCD}}\mathcal{Y} \Leftrightarrow
\mathcal{X}\sqsubseteq\mathcal{Y}$, because $\im(\mathcal{A}\times^{\mathsf{FCD}}\mathcal{X})=\mathcal{X}$ and
$\im(\mathcal{A}\times^{\mathsf{FCD}}\mathcal{Y})=\mathcal{Y}$.\end{proof}
\begin{prop}
Let $\mathfrak{A}$ be a meet-semilattice with least element and $\mathfrak{B}$
be a poset with least element. If $a$ is an atom of $\mathfrak{A}$,
$f\in\mathsf{pFCD}(\mathfrak{A};\mathfrak{B})$ then $f|_{a}=a\times^{\mathsf{FCD}}\langle f\rangle a$.\end{prop}
\begin{proof}
Let $\mathcal{X}\in\mathfrak{A}$. 
\[
\mathcal{X}\sqcap a\neq\bot^{\mathfrak{A}}\Rightarrow\supfun{f|_{a}}\mathcal{X}=\supfun fa,\hspace{1em}\mathcal{X}\sqcap a=\bot^{\mathfrak{A}}\Rightarrow\supfun{f|_{a}}\mathcal{X}=\bot^{\mathfrak{B}}.
\]
\end{proof}
\begin{prop}
$f\circ(\mathcal{A}\times^{\mathsf{FCD}}\mathcal{B})=\mathcal{A}\times^{\mathsf{FCD}}\langle f\rangle\mathcal{B}$
for elements $\mathcal{A}\in\mathfrak{A}$ and $B\in\mathfrak{B}$
of some posets $\mathfrak{A}$, $\mathfrak{B}$, $\mathfrak{C}$ with
least elements and $f\in\mathsf{pFCD}(\mathfrak{B};\mathfrak{C})$.\end{prop}
\begin{proof}
Let $\mathcal{X}\in\mathfrak{A}$, $\mathcal{Y}\in\mathfrak{B}$.
\begin{gather*}
\supfun{f\circ(\mathcal{A}\times^{\mathsf{FCD}}\mathcal{B})}\mathcal{X}=\left(\begin{cases}
\supfun f\mathcal{B} & \text{if }\mathcal{X}\nasymp\mathcal{A}\\
\bot & \text{if }\mathcal{X}\asymp\mathcal{A}
\end{cases}\right)=\supfun{\mathcal{A}\times^{\mathsf{FCD}}\langle f\rangle\mathcal{B}}\mathcal{X};\\
\begin{aligned}\supfun{(f\circ(\mathcal{A}\times^{\mathsf{FCD}}\mathcal{B}))^{-1}}\mathcal{Y} & =\\
\supfun{(\mathcal{B}\times^{\mathsf{FCD}}\mathcal{A})\circ f^{-1}}\mathcal{Y} & =\\
\left(\begin{cases}
\mathcal{A} & \text{if }\supfun{f^{-1}}\mathcal{Y}\nasymp\mathcal{B}\\
\bot & \text{if }\supfun{f^{-1}}\mathcal{Y}\asymp\mathcal{B}
\end{cases}\right) & =\\
\left(\begin{cases}
\mathcal{A} & \text{if }\mathcal{Y}\nasymp\supfun f\mathcal{B}\\
\bot & \text{if }\mathcal{Y}\asymp\supfun f\mathcal{B}
\end{cases}\right) & =\\
\supfun{\supfun f\mathcal{B}\times^{\mathsf{FCD}}\mathcal{A}}\mathcal{Y} & =\\
\supfun{(\mathcal{A}\times^{\mathsf{FCD}}\supfun f\mathcal{B})^{-1}}\mathcal{Y}.
\end{aligned}
\end{gather*}

\end{proof}

\section{Atomic pointfree funcoids}
\begin{thm}
\label{pf-atom}Let $\mathfrak{A}$, $\mathfrak{B}$ be sets of filters
over boolean lattices. A $f\in\mathsf{pFCD}(\mathfrak{A};\mathfrak{B})$
is an atom of the poset $\mathsf{pFCD}(\mathfrak{A};\mathfrak{B})$
iff there exist $a\in\atoms^{\mathfrak{A}}$ and $b\in\atoms^{\mathfrak{B}}$
such that $f=a\times^{\mathsf{FCD}}b$.\end{thm}
\begin{proof}
$\mathfrak{A}$ and $\mathfrak{B}$ are atomic by the theorem \ref{filt-atomic}.
\begin{description}
\item [{$\Rightarrow$}] Let $f$ be an atom of the poset $\mathsf{pFCD}(\mathfrak{A};\mathfrak{B})$.
Let's get elements $a\in\atoms\dom f$ and $b\in\atoms\supfun fa$.
Then for every $\mathcal{X}\in\mathfrak{A}$
\[
\mathcal{X}\asymp a\Rightarrow\supfun{a\times^{\mathsf{FCD}}b}\mathcal{X}=\bot^{\mathfrak{B}}\sqsubseteq\supfun f\mathcal{X},\quad\mathcal{X}\nasymp a\Rightarrow\supfun{a\times^{\mathsf{FCD}}b}\mathcal{X}=b\sqsubseteq\supfun f\mathcal{X}.
\]
So $a\times^{\mathsf{FCD}}b\sqsubseteq f$; because $f$ is atomic
we have $f=a\times^{\mathsf{FCD}}b$.
\item [{$\Leftarrow$}] Let $a\in\atoms^{\mathfrak{A}}$, $b\in\atoms^{\mathfrak{B}}$,
$f\in\mathsf{pFCD}(\mathfrak{A};\mathfrak{B})$. If $b\asymp^{\mathfrak{B}}\supfun fa$
then $\lnot(a\suprel fb)$, $f\sqcap(a\times^{\mathsf{FCD}}b)=\bot^{\mathsf{pFCD}(\mathfrak{A};\mathfrak{B})}$
(because $\mathfrak{A}$ and $\mathfrak{B}$ are bounded meet-semilattices);
if $b\sqsubseteq\supfun fa$ then $\forall\mathcal{X}\in\mathfrak{A}:(\mathcal{X}\nasymp a\Rightarrow\supfun f\mathcal{X}\sqsupseteq b)$,
$f\sqsupseteq a\times^{\mathsf{FCD}}b$. Consequently $f\sqcap(a\times^{\mathsf{FCD}}b)=\bot^{\mathsf{pFCD}(\mathfrak{A};\mathfrak{B})}\lor f\sqsupseteq a\times^{\mathsf{FCD}}b$;
that is $a\times^{\mathsf{FCD}}b$ is an atomic pointfree funcoid.
\end{description}
\end{proof}
\begin{thm}
Let $\mathfrak{A}$, $\mathfrak{B}$ be sets of filters over boolean
lattices. Then $\mathsf{pFCD}(\mathfrak{A};\mathfrak{B})$ is atomic.\end{thm}
\begin{proof}
Let $f\in\mathsf{pFCD}(\mathfrak{A};\mathfrak{B})$ and $f\neq\bot^{\mathsf{pFCD}(\mathfrak{A};\mathfrak{B})}$.
Then $\dom f\neq\bot^{\mathfrak{A}}$, thus exists $a\in\atoms\dom f$.
So $\supfun fa\neq\bot^{\mathfrak{B}}$ thus exists $b\in\atoms\supfun fa$.
Finally the atomic pointfree funcoid $a\times^{\mathsf{FCD}}b\sqsubseteq f$.\end{proof}
\begin{thm}
\label{pf-fcd-sep}Let $\mathfrak{A}$, $\mathfrak{B}$ be sets of
filters over boolean lattices. Then the poset $\mathsf{pFCD}(\mathfrak{A};\mathfrak{B})$
is separable.\end{thm}
\begin{proof}
Let $f,g\in\mathsf{pFCD}(\mathfrak{A};\mathfrak{B})$, $f\sqsubset g$.
Then taking into account the theorem~\ref{pf-fcd-atom-args} exists
$a\in\atoms^{\mathfrak{A}}$ such that $\supfun fa\sqsubset\supfun ga$.
By corollary \ref{f-atom-sep} $\mathfrak{B}$ is atomically separable.
So exists $b\in\atoms^{\mathfrak{B}}$ such that $\supfun fa\sqcap b=\bot^{\mathfrak{B}}$
and $b\sqsubseteq\supfun ga$. For every $x\in\atoms^{\mathfrak{A}}$
\begin{gather*}
\supfun fa\sqcap\supfun{a\times^{\mathsf{FCD}}b}a=\supfun fa\sqcap b=\bot^{\mathfrak{B}},\\
x\ne a\Rightarrow\supfun fx\sqcap\supfun{a\times^{\mathsf{FCD}}b}x=\supfun fx\sqcap\bot^{\mathfrak{B}}=\bot^{\mathfrak{B}}.
\end{gather*}
Thus $\supfun fx\sqcap\supfun{a\times^{\mathsf{FCD}}b}x=\bot^{\mathfrak{B}}$
and consequently $f\sqcap(a\times^{\mathsf{FCD}}b)=\bot^{\mathsf{pFCD}(\mathfrak{A};\mathfrak{B})}$.
\begin{gather*}
\supfun{a\times^{\mathsf{FCD}}b}a=b\sqsubseteq\supfun ga,\\
x\ne a\Rightarrow\supfun{a\times^{\mathsf{FCD}}b}x=\bot^{\mathfrak{B}}\sqsubseteq\supfun gx.
\end{gather*}
Thus $\supfun{a\times^{\mathsf{FCD}}b}x\sqsubseteq\supfun gx$ and
consequently $a\times^{\mathsf{FCD}}b\sqsubseteq g$.

So the lattice of pointfree funcoids is separable by the theorem \ref{msl-sep-conds}.\end{proof}
\begin{cor}
\label{pf-fcd-is-sep}Let $\mathfrak{A}$, $\mathfrak{B}$ be sets
of filters over boolean lattices. The poset $\mathsf{pFCD}(\mathfrak{A};\mathfrak{B})$
is:
\begin{enumerate}
\item separable;
\item atomically separable;
\item conforming to Wallman's disjunction property.
\end{enumerate}
\end{cor}
\begin{proof}
By the theorem \ref{sep-conds}.\end{proof}
\begin{rem}
For more ways to characterize (atomic) separability of the lattice
of pointfree funcoids see subsections ``\nameref{sep-and-full}''
and ``\nameref{atm-sep}''.\end{rem}
\begin{cor}
\label{pf-atomistic}Let $(\mathfrak{A};\mathfrak{Z}_{0})$ and $(\mathfrak{B};\mathfrak{Z}_{1})$
be primary filtrators over boolean lattices. The poset $\mathsf{pFCD}(\mathfrak{A};\mathfrak{B})$
is an atomistic lattice.\end{cor}
\begin{proof}
By the corollary \ref{pf-fcd-compl} $\text{\ensuremath{\mathsf{pFCD}(\mathfrak{A};\mathfrak{B})}}$
is a complete lattice. We can use theorem~\ref{amstc-sep}.\end{proof}
\begin{prop}
Let $\mathfrak{A}$, $\mathfrak{B}$ be sets of filters over boolean
lattices.

$\atoms(f\sqcup g)=\atoms f\cup\atoms g$ for every $f,g\in\mathsf{pFCD}(\mathfrak{A};\mathfrak{B})$.\end{prop}
\begin{proof}
~
\begin{multline*}
(a\times^{\mathsf{FCD}}b)\sqcap(f\sqcup g)\neq\emptyset\Leftrightarrow a\suprel{f\sqcup g}b\Leftrightarrow a\suprel fb\vee a\suprel gb\Leftrightarrow\\
(a\times^{\mathsf{\mathsf{pFCD}}}b)\sqcap f\neq\bot^{\mathsf{pFCD}(\mathfrak{A};\mathfrak{B})}\vee(a\times^{\mathsf{FCD}}b)\sqcap g\neq\bot^{\mathsf{pFCD}(\mathfrak{A};\mathfrak{B})}
\end{multline*}
for every $a\in\atoms^{\mathfrak{A}}$ and $b\in\atoms^{\mathfrak{B}}$
(used the corollary \ref{pf-intr-prod} and theorem \ref{pf-fin-join}).\end{proof}
\begin{thm}
Let $(\mathfrak{A};\mathfrak{Z}_{0})$ and $(\mathfrak{B};\mathfrak{Z}_{1})$
be primary filtrators over boolean lattices. Then $\mathsf{pFCD}(\mathfrak{A};\mathfrak{B})$
is a co-frame.\end{thm}
\begin{proof}
Theorems \ref{pfcd-as-func} and \ref{frame-main}.\end{proof}
\begin{cor}
Let $(\mathfrak{A};\mathfrak{Z}_{0})$ and $(\mathfrak{B};\mathfrak{Z}_{1})$
be primary filtrators over boolean lattices. Then $\mathsf{pFCD}(\mathfrak{A};\mathfrak{B})$
is a co-brouwerian lattice.\end{cor}
\begin{prop}
Let $\mathfrak{A}$, $\mathfrak{B}$, $\mathfrak{C}$ be sets of filters
over some boolean lattices and $f\in\mathsf{pFCD}(\mathfrak{A};\mathfrak{B})$,
$g\in\mathsf{pFCD}(\mathfrak{B};\mathfrak{C})$. Let $\mathfrak{B}$
be an atomic poset. Then
\begin{multline*}
\atoms(g\circ f)=\\
\setcond{x\times^{\mathsf{FCD}}z}{x\in\atoms^{\mathfrak{A}},z\in\atoms^{\mathfrak{C}},\exists y\in\atoms^{\mathfrak{B}}:(x\times^{\mathsf{FCD}}y\in\atoms f\land y\times^{\mathsf{FCD}}z\in\atoms g)}.
\end{multline*}
\end{prop}
\begin{proof}
~
\begin{align*}
(x\times^{\mathsf{FCD}}z)\sqcap(g\circ f)\ne\bot^{\mathsf{pFCD}(\mathfrak{A};\mathfrak{C})} & \Leftrightarrow\\
x\suprel{g\circ f}z & \Leftrightarrow\\
\exists y\in\atoms^{\mathfrak{B}}:(x\suprel fy\land y\suprel gz) & \Leftrightarrow\\
\exists y\in\atoms^{\mathfrak{B}}:((x\times^{\mathsf{FCD}}y)\sqcap f\ne\bot^{\mathsf{pFCD}(\mathfrak{A};\mathfrak{B})}\land(y\times^{\mathsf{FCD}}z)\sqcap g\ne\bot^{\mathsf{pFCD}(\mathfrak{B};\mathfrak{C})})
\end{align*}
(were used corollary \ref{pf-intr-prod} and theorem \ref{pf-fcd-atom-middle}).\end{proof}
\begin{thm}
Let $f$ be a pointfree funcoid between sets of filters on boolean
lattices.
\begin{enumerate}
\item \label{pf-r-at}$\mathcal{X}\suprel f\mathcal{Y}\Leftrightarrow\exists F\in\atoms f:\mathcal{X}\suprel F\mathcal{Y}$
for every $\mathcal{X}\in\mathscr{F}(\Src f)$, $\mathcal{Y}\in\mathscr{F}(\Dst f)$;
\item \label{pf-f-at}$\supfun f\mathcal{X}=\bigsqcup_{F\in\atoms f}\supfun F\mathcal{X}$
for every $\mathcal{X}\in\mathscr{F}(\Src f)$.
\end{enumerate}
\end{thm}
\begin{proof}
~
\begin{widedisorder}
\item [{\ref{pf-r-at}}] ~
\begin{align*}
\exists F\in\atoms f:\mathcal{X}\suprel F\mathcal{Y} & \Leftrightarrow\\
\exists a\in\atoms^{\mathscr{F}(\Src f)},b\in\atoms^{\mathscr{F}(\Dst f)}:(a\times^{\mathsf{FCD}}b\nasymp f\land\mathcal{X}\suprel{a\times^{\mathsf{FCD}}b}\mathcal{Y}) & \Leftrightarrow\\
\exists a\in\atoms^{\mathscr{F}(\Src f)},b\in\atoms^{\mathscr{F}(\Dst f)}:(a\times^{\mathsf{FCD}}b\nasymp f\land a\times^{\mathsf{FCD}}b\nasymp\mathcal{X}\times^{\mathsf{FCD}}\mathcal{Y}) & \Leftrightarrow\\
\exists F\in\atoms f:(F\nasymp f\land F\nasymp\mathcal{X}\times^{\mathsf{FCD}}\mathcal{Y}) & \Leftrightarrow\\
f\nasymp\mathcal{X}\times^{\mathsf{FCD}}\mathcal{Y} & \Leftrightarrow\\
\mathcal{X}\suprel f\mathcal{Y}.
\end{align*}

\item [{\ref{pf-f-at}}] Let $\mathcal{Y}\in\mathscr{F}(\Dst f)$. Suppose
$\mathcal{Y}\nasymp\supfun f\mathcal{X}$. Then $\mathcal{X}\suprel f\mathcal{Y}$;
$\exists F\in\atoms f:\mathcal{X}\suprel F\mathcal{Y}$; $\exists F\in\atoms f:\mathcal{Y}\nasymp\supfun F\mathcal{X}$;
$\mathcal{Y}\nasymp\bigsqcup_{F\in\atoms f}\supfun F\mathcal{X}$.
So $\supfun f\mathcal{X}\sqsubseteq\bigsqcup_{F\in\atoms f}\supfun F\mathcal{X}$.
The contrary $\supfun f\mathcal{X}\sqsupseteq\bigsqcup_{F\in\atoms f}\supfun F\mathcal{X}$
is obvious.
\end{widedisorder}
\end{proof}

\section{Complete pointfree funcoids}
\begin{defn}
\index{funcoid!pointfree!complete}Let $\mathfrak{A}$ and $\mathfrak{B}$
be posets. A pointfree funcoid $f\in\mathsf{pFCD}(\mathfrak{A};\mathfrak{B})$
is \emph{complete}, when for every $S\in\mathscr{P}\mathfrak{A}$
whenever both $\bigsqcup S$ and $\bigsqcup\rsupfun{\supfun f}S$
are defined we have
\[
\supfun f\bigsqcup S=\bigsqcup\rsupfun{\supfun f}S.
\]
\end{defn}
\begin{defn}
\index{funcoid!pointfree!co-complete}Let $(\mathfrak{A};\mathfrak{Z}_{0})$
and $(\mathfrak{B};\mathfrak{Z}_{1})$ be filtrators. I will call a \emph{co-complete pointfree funcoid}
a pointfree funcoid $f\in\mathsf{pFCD}(\mathfrak{A};\mathfrak{B})$
such that $\supfun fX\in\mathfrak{Z}_1$ for every $X\in\mathfrak{Z}_0$.\end{defn}
\begin{prop}
Let $(\mathfrak{A};\mathfrak{Z}_{0})$ and $(\mathfrak{B};\mathfrak{Z}_{1})$
be primary filtrators over boolean lattices. Co-complete pointfree
funcoids $\mathsf{pFCD}(\mathfrak{A};\mathfrak{B})$ bijectively correspond
to functions $\mathfrak{Z}_{1}^{\mathfrak{Z}_{0}}$ preserving finite
joins, where the bijection is $f\mapsto\supfun f|_{\mathfrak{Z}_{0}}$.\end{prop}
\begin{proof}
It follows from the theorem \ref{pf-cont}.\end{proof}
\begin{thm}
\label{pf-compl-conds}Let $(\mathfrak{A};\mathfrak{Z}_0)$ be a down-aligned,
with join-closed, binarily meet-closed and separable core which is
a complete boolean lattice.

Let $(\mathfrak{B};\mathfrak{Z}_{1})$ be a star-separable filtrator.

The following conditions are equivalent for every pointfree funcoid
$f\in\mathsf{pFCD}(\mathfrak{A};\mathfrak{B})$:
\begin{enumerate}
\item \label{pf-ax:fcd-full-main}$f^{-1}$ is co-complete;
\item \label{pf-ax:fcd-full-fa-filt}$\forall S\in\mathscr{P}\mathfrak{A},J\in\mathfrak{Z}_{1}:(\bigsqcup^{\mathfrak{A}}S\suprel fJ\Rightarrow\exists\mathcal{I}\in S:\mathcal{I}\suprel fJ)$;
\item \label{pf-ax:fcd-full-fa-set}$\forall S\in\mathscr{P}\mathfrak{Z}_{0},J\in\mathfrak{Z}_{1}:(\bigsqcup^{\mathfrak{Z}_{0}}S\suprel fJ\Rightarrow\exists I\in S:I\suprel fJ)$;
\item \label{pf-ax:fcd-full-eq-filt}$f$ is complete;
\item \label{pf-ax:fcd-full-eq-set}$\forall S\in\mathscr{P}\mathfrak{Z}_{0}:\supfun f\bigsqcup^{\mathfrak{Z}_{0}}S=\bigsqcup^{\mathfrak{B}}\rsupfun{\supfun f}S$.
\end{enumerate}
\end{thm}
\begin{proof}
First note that the theorem \ref{crit1} applies to the filtrator
$(\mathfrak{A};\mathfrak{Z}_{0})$.
\begin{description}
\item [{\ref{pf-ax:fcd-full-fa-set}$\Rightarrow$\ref{pf-ax:fcd-full-main}}] For
every $S\in\subsets\mathfrak{Z}_{0}$, $J\in\mathfrak{Z}_{1}$ 
\begin{equation}
\bigsqcup^{\mathfrak{Z}_{0}}S\sqcap^{\mathfrak{A}}\supfun{f^{-1}}J\neq\bot^{\mathfrak{A}}\Rightarrow\exists I\in S:I\sqcap^{\mathfrak{A}}\supfun{f^{-1}}J\neq\bot^{\mathfrak{A}},\label{eq:fcd-full-ee}
\end{equation}
consequently by the theorem \ref{crit1} we have $\supfun{f^{-1}}J\in\mathfrak{Z}_{0}$.
\item [{\ref{pf-ax:fcd-full-main}$\Rightarrow$\ref{pf-ax:fcd-full-fa-filt}}] For
every $S\in\subsets\mathfrak{A}$, $J\in\mathfrak{Z}_{1}$ we have
$\supfun{f^{-1}}J\in\mathfrak{Z}_{0}$, consequently 
\[
\forall S\in\subsets\mathfrak{A},J\in\mathfrak{Z}_{1}:\left(\bigsqcup^{\mathfrak{A}}S\nasymp\supfun{f^{-1}}J\Rightarrow\exists\mathcal{I}\in S:\mathcal{I}\nasymp\supfun{f^{-1}}J\right).
\]
From this follows \ref{pf-ax:fcd-full-fa-filt}.
\item [{\ref{pf-ax:fcd-full-fa-filt}$\Rightarrow$\ref{pf-ax:fcd-full-eq-filt}}] Let
$\supfun f\bigsqcup^{\mathfrak{Z}_{0}}S$ and $\bigsqcup^{\mathfrak{B}}\rsupfun{\supfun f}S$
be defined. We have $\supfun f\bigsqcup^{\mathfrak{A}}S=\supfun f\bigsqcup^{\mathfrak{Z}_{0}}S$.
\begin{align*}
J\sqcap^{\mathfrak{B}}\supfun f\bigsqcup^{\mathfrak{A}}S\ne\bot^{\mathfrak{B}} & \Leftrightarrow\\
\bigsqcup^{\mathfrak{A}}S\suprel fJ & \Leftrightarrow\\
\exists\mathcal{I}\in S:\mathcal{I}\suprel fJ & \Leftrightarrow\\
\exists\mathcal{I}\in S:J\sqcap^{\mathfrak{B}}\supfun f\mathcal{I}\ne\bot^{\mathfrak{B}} & \Leftrightarrow\\
J\sqcap^{\mathfrak{B}}\bigsqcup^{\mathfrak{B}}\rsupfun{\supfun f}S\ne\bot^{\mathfrak{B}}
\end{align*}
(used theorem \ref{crit1}). Thus $\supfun f\bigsqcup^{\mathfrak{A}}S=\bigsqcup^{\mathfrak{B}}\rsupfun{\supfun f}S$
by star-separability of $(\mathfrak{B};\mathfrak{Z}_{1})$.
\item [{\ref{pf-ax:fcd-full-eq-set}$\Rightarrow$\ref{pf-ax:fcd-full-fa-set}}] Let
$\supfun f\bigsqcup^{\mathfrak{Z}_{0}}S$ be defined. Then $\bigsqcup^{\mathfrak{B}}\rsupfun{\supfun f}S$
is also defined because $\supfun f\bigsqcup^{\mathfrak{Z}_{0}}S=\bigsqcup^{\mathfrak{B}}\rsupfun{\supfun f}S$.
Then
\[
\bigsqcup^{\mathfrak{Z}_{0}}S\suprel fJ\Leftrightarrow J\sqcap^{\mathfrak{B}}\supfun f\bigsqcup^{\mathfrak{Z}_{0}}S\neq\bot^{\mathfrak{B}}\Leftrightarrow J\sqcap^{\mathfrak{B}}\bigsqcup^{\mathfrak{B}}\rsupfun{\supfun f}S\neq\bot^{\mathfrak{B}}
\]
what by theorem \ref{crit1} is equivalent to $\exists I\in S:J\sqcap^{\mathfrak{B}}\supfun fI\neq\bot^{\mathfrak{B}}$
that is $\exists I\in S:I\suprel fJ$.
\item [{\ref{pf-ax:fcd-full-fa-filt}$\Rightarrow$\ref{pf-ax:fcd-full-fa-set},~\ref{pf-ax:fcd-full-eq-filt}$\Rightarrow$\ref{pf-ax:fcd-full-eq-set}}] By
join-closedness of the core of $(\mathfrak{A};\mathfrak{Z}_{0})$.
\end{description}
\end{proof}
\begin{thm}
\label{pf-join-cocompl}Let $(\mathfrak{A};\mathfrak{Z}_{0})$ and
$(\mathfrak{B};\mathfrak{Z}_{1})$ be primary filtrators over boolean
lattices. If $R$ is a set of co-complete pointfree funcoids in $\mathsf{pFCD}(\mathfrak{A};\mathfrak{B})$
then $\bigsqcup R$ is a co-complete pointfree funcoid.\end{thm}
\begin{proof}
Let $R$ be a set of co-complete pointfree funcoids. Then for every
$X\in\mathfrak{Z}_{0}$
\[
\supfun{\bigsqcup R}X=\bigsqcup_{f\in R}^{\mathfrak{B}}\supfun fX=\bigsqcup_{f\in R}^{\mathfrak{Z}_{1}}\supfun fX\in\mathfrak{Z}_{1}
\]
(used the theorem \ref{pf-join-core} and corollary \ref{semifilt-joinclosed}).
\end{proof}
Let $\mathfrak{A}$ and $\mathfrak{B}$ be posets with least elements.
I will denote $\mathsf{ComplpFCD}(\mathfrak{A};\mathfrak{B})$ and
$\mathsf{CoComplpFCD}(\mathfrak{A};\mathfrak{B})$ the sets of complete
and co-complete funcoids correspondingly from a poset $\mathfrak{A}$
to a poset $\mathfrak{B}$.
\begin{prop}
~
\begin{enumerate}
\item \label{pf-compl-comp}Let $f\in\mathsf{ComplpFCD}(\mathfrak{A};\mathfrak{B})$
and $g\in\mathsf{ComplpFCD}(\mathfrak{B};\mathfrak{C})$ where $\mathfrak{A}$
and $\mathfrak{C}$ are posets with least elements and $\mathfrak{B}$
is a complete lattice. Then $g\circ f\in\mathsf{ComplpFCD}(\mathfrak{A};\mathfrak{C})$.
\item \label{pf-cocompl-comp}Let $f\in\mathsf{CoComplpFCD}(\mathfrak{A};\mathfrak{B})$
and $g\in\mathsf{CoComplpFCD}(\mathfrak{B};\mathfrak{C})$ where 
$(\mathfrak{A};\mathfrak{Z}_{0})$, $(\mathfrak{B};\mathfrak{Z}_{1})$,
$(\mathfrak{C};\mathfrak{Z}_{2})$ are filtrators. Then $g\circ f\in\mathsf{CoComplpFCD}(\mathfrak{A};\mathfrak{C})$.
\end{enumerate}
\end{prop}
\begin{proof}
~
\begin{widedisorder}
\item [{\ref{pf-compl-comp}}] Let $\bigsqcup S$ and $\bigsqcup\rsupfun{\supfun{g\circ f}}S$
be defined. Then
\[
\supfun{g\circ f}\bigsqcup S=\supfun g\supfun f\bigsqcup S=\supfun g\bigsqcup\rsupfun{\supfun f}S=\bigsqcup\rsupfun{\supfun g}\rsupfun{\supfun f}S=\bigsqcup\rsupfun{\supfun{g\circ f}}S.
\]

\item [{\ref{pf-cocompl-comp}}] $\supfun{g\circ f}\mathfrak{Z}_{0}=\supfun g\supfun f\mathfrak{Z}_{0}\in\mathfrak{Z}_{2}$
because $\supfun f\mathfrak{Z}_{0}\in\mathfrak{Z}_{1}$.
\end{widedisorder}
\end{proof}
\begin{prop}
Let $(\mathfrak{A};\mathfrak{Z}_{0})$ and $(\mathfrak{B};\mathfrak{Z}_{1})$
be primary filtrators over boolean lattices. Then $\mathsf{CoComplpFCD}(\mathfrak{A};\mathfrak{B})$
(with induced order) is a complete lattice.\end{prop}
\begin{proof}
Follows from the theorem \ref{pf-join-cocompl}.\end{proof}
\begin{thm}
Let $(\mathfrak{A};\mathfrak{Z}_{0})$ and $(\mathfrak{B};\mathfrak{Z}_{1})$
be primary filtrators where $\mathfrak{Z}_{0}$ and $\mathfrak{Z}_{1}$
are boolean lattices. Let $R$ be a set of pointfree funcoids from
$\mathfrak{A}$ to $\mathfrak{B}$.

$g\circ\left(\bigsqcup R\right)=\bigsqcup_{g\in R}(g\circ f)=\bigsqcup\rsupfun{g\circ}R$
if $g$ is a complete pointfree funcoid from~$\mathfrak{B}$.\end{thm}
\begin{proof}
For every $X\in\mathfrak{A}$
\begin{align*}
\supfun{g\circ\left(\bigsqcup R\right)}X & =\\
\supfun g\supfun{\bigsqcup R}X & =\\
\supfun g\bigsqcup_{f\in R}\supfun fX & =\\
\bigsqcup_{f\in R}\supfun g\supfun fX & =\\
\bigsqcup_{f\in R}\supfun{g\circ f}X & =\\
\supfun{\bigsqcup_{f\in R}(g\circ f)}X & =\\
\supfun{\bigsqcup\rsupfun{g\circ}R}X.
\end{align*}
So $g\circ\left(\bigsqcup R\right)=\bigsqcup\rsupfun{g\circ}R$.
\end{proof}

\section{Completion and co-completion}
\begin{defn}
Let $(\mathfrak{A};\mathfrak{Z}_{0})$ and $(\mathfrak{B};\mathfrak{Z}_{1})$
be primary filtrators over boolean lattices and $\mathfrak{Z}_{1}$
is a complete atomistic lattice.

\index{funcoid!pointfree!co-completion}\index{co-completion!funcoid!pointfree}\emph{Co-completion}
of a pointfree funcoid $f\in\mathsf{pFCD}(\mathfrak{A};\mathfrak{B})$
is pointfree funcoid $\CoCompl f$ defined by the formula (for every
$X\in\mathfrak{Z}_{0}$) 
\[
\supfun{\CoCompl f}X=\Cor\supfun fX.
\]
\end{defn}
\begin{prop}
Above defined co-completion always exists.\end{prop}
\begin{proof}
Existence of $\Cor\supfun fX$ follows from completeness of $\mathfrak{Z}_{1}$.

We may apply the theorem \ref{pf-cont} because 
\[
\Cor\supfun f(X\sqcup^{\mathfrak{Z}_{0}}Y)=\Cor(\supfun fX\sqcup^{\mathfrak{B}}\supfun fY)=\Cor\supfun fX\sqcup^{\mathfrak{Z}_{1}}\Cor\supfun fY
\]
by proposition \ref{dual-core-join}.\end{proof}
\begin{obvious}
Co-completion is always co-complete.\end{obvious}
\begin{prop}
For above defined always $\CoCompl f\sqsubseteq f$.\end{prop}
\begin{proof}
By proposition \ref{cor-less}.
\end{proof}

\section{Monovalued and injective pointfree funcoids}
\begin{defn}
Let $\mathfrak{A}$ and $\mathfrak{B}$ be posets. Let $f\in\mathsf{pFCD}(\mathfrak{A};\mathfrak{B})$.

The pointfree funcoid $f$ is:
\begin{itemize}
\item \index{funcoid!pointfree!monovalued}\emph{monovalued} when $f\circ f^{-1}\sqsubseteq1_{\mathfrak{B}}^{\mathsf{pFCD}}$.
\item \index{funcoid!pointfree!injective}\emph{injective} when $f^{-1}\circ f\sqsubseteq1_{\mathfrak{A}}^{\mathsf{pFCD}}$.
\end{itemize}
\end{defn}
Monovaluedness is dual of injectivity.
\begin{prop}
Let $\mathfrak{A}$ and $\mathfrak{B}$ be posets. Let $f\in\mathsf{pFCD}(\mathfrak{A};\mathfrak{B})$.

The pointfree funcoid $f$ is:
\begin{itemize}
\item monovalued iff $f\circ f^{-1}\sqsubseteq\id_{\im f}^{\mathsf{pFCD}(\mathfrak{B})}$,
if $\mathfrak{A}$ has greatest element is defined and $\mathfrak{B}$ is a separable meet-semilattice;
\item injective iff $f^{-1}\circ f\sqsubseteq\id_{\dom f}^{\mathsf{pFCD}(\mathfrak{A})}$,
if $\mathfrak{B}$ has greatest element is defined and $\mathfrak{A}$ is a separable meet-semilattice. 
\end{itemize}
\end{prop}
\begin{proof}
It's enough to prove $f\circ f^{-1}\sqsubseteq1_{\mathfrak{B}}^{\mathsf{pFCD}}\Leftrightarrow f\circ f^{-1}\sqsubseteq\id_{\im f}^{\mathsf{pFCD}(\mathfrak{B})}$.
$\im f$ is defined because $\mathfrak{A}$ has greatest element. $\id_{\im f}^{\mathsf{pFCD}(\mathfrak{B})}$ is defined because $\mathfrak{B}$ is a meet-semilattice.
\begin{description}
\item [{$\Leftarrow$}] Obvious.
\item [{$\Rightarrow$}] Let $f\circ f^{-1}\sqsubseteq1_{\mathfrak{B}}^{\mathsf{pFCD}}$.
Then $\supfun{f\circ f^{-1}}x\sqsubseteq x$; $\supfun{f\circ f^{-1}}x\sqsubseteq\im f$ (proposition~\ref{pfcd-mono}).
Thus $\supfun{f\circ f^{-1}}x\sqsubseteq x\sqcap\im f=\supfun{\id_{\im f}^{\mathsf{pFCD}(\mathfrak{B})}}x$.


$\supfun{(f\circ f^{-1})^{-1}}x\sqsubseteq x$ and $\supfun{(f\circ f^{-1})^{-1}}x\sqsubseteq\supfun{f\circ f^{-1}}x\sqsubseteq\im f$.
Thus $\supfun{(f\circ f^{-1})^{-1}}x\sqsubseteq x\sqcap\im f=\supfun{\id_{\im f}^{\mathsf{pFCD}(\mathfrak{B})}}x$.


Thus $f\circ f^{-1}\sqsubseteq\id_{\im f}^{\mathsf{pFCD}(\mathfrak{B})}$.

\end{description}
\end{proof}
\begin{thm}
Let $\mathfrak{A}$ be an atomistic meet-semilattice with least element,
$\mathfrak{B}$ be an atomistic bounded meet-semilattice. The following
statements are equivalent for every $f\in\mathsf{pFCD}(\mathfrak{A};\mathfrak{B})$:
\begin{enumerate}
\item \label{pf-thm:func-mono-def}$f$ is monovalued.
\item \label{pf-thm:func-mono-atom}$\forall a\in\atoms^{\mathfrak{A}}:\supfun fa\in\atoms^{\mathfrak{B}}\cup\{\bot^{\mathfrak{B}}\}$.
\item \label{pf-thm:func-mono-intr-flt}$\forall i,j\in\mathfrak{B}:\supfun{f^{-1}}(i\sqcap j)=\supfun{f^{-1}}i\sqcap\supfun{f^{-1}}j$.
\end{enumerate}
\end{thm}
\begin{proof}
~
\begin{description}
\item [{\ref{pf-thm:func-mono-atom}$\Rightarrow$\ref{pf-thm:func-mono-intr-flt}}] Let
$a\in\atoms^{\mathfrak{A}}$, $\supfun fa=b$. Then because $b\in\atoms^{\mathfrak{B}}\cup\{\bot^{\mathfrak{B}}\}$
\begin{gather*}
(i\sqcap j)\sqcap b\ne\bot^{\mathfrak{B}}\Leftrightarrow i\sqcap b\ne\bot^{\mathfrak{B}}\land j\sqcap b\ne\bot^{\mathfrak{B}};\\
a\suprel fi\sqcap j\Leftrightarrow a\suprel fi\land a\suprel fj;\\
i\sqcap j\suprel{f^{-1}}a\Leftrightarrow i\suprel{f^{-1}}a\land j\suprel{f^{-1}}a;\\
a\sqcap^{\mathfrak{A}}\supfun{f^{-1}}(i\sqcap j)\ne\bot^{\mathfrak{A}}\Leftrightarrow a\sqcap\supfun{f^{-1}}i\ne\bot^{\mathfrak{A}}\land a\sqcap\supfun{f^{-1}}j\ne\bot^{\mathfrak{A}};\\
a\sqcap^{\mathfrak{A}}\supfun{f^{-1}}(i\sqcap j)\ne\bot^{\mathfrak{A}}\Leftrightarrow a\sqcap\supfun{f^{-1}}i\sqcap\supfun{f^{-1}}j\ne\bot^{\mathfrak{A}};\\
\supfun{f^{-1}}(i\sqcap j)=\supfun{f^{-1}}i\sqcap\supfun{f^{-1}}j.
\end{gather*}

\item [{\ref{pf-thm:func-mono-intr-flt}$\Rightarrow$\ref{pf-thm:func-mono-def}}] $\supfun{f^{-1}}a\sqcap\supfun{f^{-1}}b=\supfun{f^{-1}}(a\sqcap b)=\supfun{f^{-1}}\bot^{\mathfrak{B}}=\bot^{\mathfrak{A}}$
(by proposition \ref{pfcd-zero})
for every two distinct $a,b\in\atoms^{\mathfrak{B}}$. This is equivalent
to $\lnot(\supfun{f^{-1}}a\suprel fb)$; $b\sqcap\supfun f\supfun{f^{-1}}a=\bot^{\mathfrak{B}}$;
$b\sqcap\supfun{f\circ f^{-1}}a=\bot^{\mathfrak{B}}$; $\lnot(a\suprel{f\circ f^{-1}}b)$.
So $a\suprel{f\circ f^{-1}}b\Rightarrow a=b$ for every $a,b\in\atoms^{\mathfrak{B}}$.
This is possible only (corollary \ref{pf-suprel-atoms} and the fact
that $\mathfrak{B}$ is atomic) when $f\circ f^{-1}\sqsubseteq1_{\mathfrak{B}}^{\mathsf{pFCD}}$.
\item [{$\neg$\ref{pf-thm:func-mono-atom}$\Rightarrow\neg$\ref{pf-thm:func-mono-def}}] Suppose
$\supfun fa\notin\atoms^{\mathfrak{B}}\cup\{\bot^{\mathfrak{B}}\}$
for some $a\in\atoms^{\mathfrak{A}}$. Then there exist two atoms
$p\neq q$ such that $\supfun fa\sqsupseteq p\wedge\supfun fa\sqsupseteq q$.
Consequently $p\sqcap\supfun fa\neq0^{\mathfrak{B}}$; $a\sqcap\supfun{f^{-1}}p\ne\bot^{\mathfrak{A}}$;
$a\sqsubseteq\supfun{f^{-1}}p$; $\supfun{f\circ f^{-1}}p=\supfun f\supfun{f^{-1}}p\sqsupseteq\supfun fa\sqsupseteq q$
(by proposition \ref{pfcd-mono} because $\mathfrak{B}$ is separable
by proposition \ref{atom-is-sep});
$\supfun{f\circ f^{-1}}p\nsqsubseteq p$ and $\supfun{f\circ f^{-1}}p\ne\bot^{\mathfrak{B}}$.
So it cannot be $f\circ f^{-1}\sqsubseteq1_{\mathfrak{B}}^{\mathsf{pFCD}}$.
\end{description}
\end{proof}
\begin{thm}
The following is equivalent for primary filtrators $(\mathfrak{A};\mathfrak{Z}_{0})$
and $(\mathfrak{B};\mathfrak{Z}_{1})$ over boolean lattices and pointfree
funcoids $f:\mathfrak{A}\rightarrow\mathfrak{B}$:
\begin{enumerate}
\item \label{pfcd-mv}$f$ is monovalued.
\item \label{pfcd-mmv}It is metamonovalued.
\item \label{pfcd-wmmv}It is weakly metamonovalued.
\end{enumerate}
\end{thm}
\begin{proof}
~
\begin{description}
\item [{\ref{pfcd-mmv}$\Rightarrow$\ref{pfcd-wmmv}}] Obvious.
\item [{\ref{pfcd-mv}$\Rightarrow$\ref{pfcd-mmv}}] ~
\[
\supfun{\left(\bigsqcap G\right)\circ f}x=\supfun{\left(\bigsqcap G\right)}\supfun fx=\bigsqcap_{g\in G}\supfun g\supfun fx=\bigsqcap_{g\in G}\supfun{g\circ f}x=\supfun{\bigsqcap_{g\in G}(g\circ f)}x
\]
for every atomic filter object $x\in\atoms^{\mathfrak{A}}$. Thus
$\left(\bigsqcap G\right)\circ f=\bigsqcap_{g\in G}(g\circ f)$.
\item [{\ref{pfcd-wmmv}$\Rightarrow$\ref{pfcd-mv}}] Take $g=a\times^{\mathsf{FCD}}y$
and $h=b\times^{\mathsf{FCD}}y$ for arbitrary atomic filter objects
$a\ne b$ and~$y$. We have $g\sqcap h=\bot$; thus $(g\circ f)\sqcap(h\circ f)=(g\sqcap h)\circ f=\bot$
and thus impossible $x\suprel fa\land x\suprel fb$ as otherwise $x\suprel{g\circ f}y$
and $x\suprel{h\circ f}y$ so $x\suprel{(g\circ f)\sqcap(h\circ f)}y$.
Thus $f$ is monovalued.
\end{description}
\end{proof}
\begin{thm}
Let $(\mathfrak{A};\mathfrak{Z}_{0})$ and $(\mathfrak{B};\mathfrak{Z}_{1})$
be primary filtrators over boolean lattices. A pointfree funcoid
$f\in\mathsf{pFCD}(\mathfrak{A};\mathfrak{B})$ is monovalued iff
\[
\forall I,J\in\mathfrak{Z}_{1}:\supfun{f^{-1}}(I\sqcap^{\mathfrak{Z}_{1}}J)=\supfun{f^{-1}}I\sqcap\supfun{f^{-1}}J.
\]
\end{thm}
\begin{proof}
$\mathfrak{A}$ and $\mathfrak{B}$ are complete lattices (corollary
\ref{filt-is-complete}).

$(\mathfrak{B};\mathfrak{Z}_{1})$ is a filtrator with separable core
by the theorem \ref{when-sep-core}.

$(\mathfrak{B};\mathfrak{Z}_{1})$ is binarily meet-closed by the
theorem \ref{f-meet-closed}.

$\mathfrak{A}$ and $\mathfrak{B}$ are starrish by corollary \ref{filt-also-distr}.

$\mathfrak{A}$ is separable by proposition~\ref{filt-is-sep}.

We are under conditions of the theorem \ref{pf-supfun-up}.
\begin{description}
\item [{$\Rightarrow$}] Obvious (taking into account that $(\mathfrak{B};\mathfrak{Z}_{1})$
is binarily meet-closed).
\item [{$\Leftarrow$}] ~
\begin{align*}
\supfun{f^{-1}}(\mathcal{I}\sqcap\mathcal{J}) & =\\
\bigsqcap\rsupfun{\supfun{f^{-1}}}\up^{\mathfrak{Z}_{1}}(\mathcal{I}\sqcap\mathcal{J}) & =\\
\bigsqcap\rsupfun{\supfun{f^{-1}}}\setcond{I\sqcap^{\mathfrak{Z}_{1}}J}{I\in\up\mathcal{I},J\in\up\mathcal{J}} & =\\
\bigsqcap\setcond{\supfun{f^{-1}}(I\sqcap^{\mathfrak{Z}_{1}}J)}{I\in\up\mathcal{I},J\in\up\mathcal{J}} & =\\
\bigsqcap\setcond{\supfun{f^{-1}}I\sqcap\supfun{f^{-1}}J}{I\in\up\mathcal{I},J\in\up\mathcal{J}} & =\\
\bigsqcap\setcond{\supfun{f^{-1}}I}{I\in\up\mathcal{I}}\sqcap\bigsqcap\setcond{\supfun{f^{-1}}J}{J\in\up\mathcal{J}} & =\\
\supfun{f^{-1}}\mathcal{I}\sqcap^{\mathfrak{A}}\supfun{f^{-1}}\mathcal{J}
\end{align*}
(used theorem \ref{pf-supfun-up}, theorem \ref{f-inf-meet-form},
theorem \ref{pf-dist-func}).
\end{description}
\end{proof}
\begin{prop}
Let $\mathfrak{A}$ be an atomistic meet-semilattice with least element,
$\mathfrak{B}$ be an atomistic bounded meet-semilattice. Then if
$f$, $g$ are pointfree funcoids from~$\mathfrak{A}$ to~$\mathfrak{B}$,
$f\sqsubseteq g$ and $g$ is monovalued then $g|_{\dom f}=f$.\end{prop}
\begin{proof}
Obviously $g|_{\dom f}\sqsupseteq f$. Suppose for contrary that $g|_{\dom f}\sqsubset f$.
Then there exists an atom $a\in\atoms\dom f$ such that $\langle g|_{\dom f}\rangle a\neq\langle f\rangle a$
that is $\supfun ga\sqsubset\supfun fa$ what is impossible.
\end{proof}

\section{Elements closed regarding a pointfree funcoid}

Let $\mathfrak{A}$ be a poset. Let $f\in\mathsf{pFCD}(\mathfrak{A};\mathfrak{A})$.
\begin{defn}
\index{closed!regarding pointfree funcoid}Let's call \emph{closed}
regarding a pointfree funcoid $f$ such element $a\in\mathfrak{A}$
that $\supfun fa\sqsubseteq a$.\end{defn}
\begin{prop}
If $i$ and $j$ are closed (regarding a pointfree funcoid $f\in\mathsf{pFCD}(\mathfrak{A};\mathfrak{A})$),
$S$ is a set of closed elements (regarding $f$), then
\begin{enumerate}
\item $i\sqcup j$ is a closed element, if $\mathfrak{A}$ is a separable
starrish join-semilattice;
\item $\bigsqcap S$ is a closed element if $\mathfrak{A}$ is a separable
complete lattice.
\end{enumerate}
\end{prop}
\begin{proof}
$\supfun f(i\sqcup j)=\supfun fi\sqcup\supfun fj\sqsubseteq i\sqcup j$
(theorem \ref{pf-dist-func}), $\supfun f\bigsqcap S\sqsubseteq\bigsqcap\rsupfun{\supfun f}S\sqsubseteq\bigsqcap S$
(used separability of $\mathfrak{A}$ twice). Consequently the elements
$i\sqcup j$ and $\bigsqcap S$ are closed.\end{proof}
\begin{prop}
If $S$ is a set of elements closed regarding a complete pointfree
funcoid $f$ with separable destination which is a complete lattice,
then the element $\bigsqcup S$ is also closed regarding our funcoid.\end{prop}
\begin{proof}
$\supfun f\bigsqcup S=\bigsqcup\rsupfun{\supfun f}S\sqsubseteq\bigsqcup S$.
\end{proof}

\section{Connectedness regarding a pointfree funcoid}

Let $\mathfrak{A}$ be a poset with least element. Let $\mu\in\mathsf{pFCD}(\mathfrak{A};\mathfrak{A})$.
\begin{defn}
\index{connected!regarding pointfree funcoid}An element $a\in\mathfrak{A}$
is called \emph{connected} regarding a pointfree funcoid $\mu$ over
$\mathfrak{A}$ when 
\[
\forall x,y\in\mathfrak{A}\setminus\{\bot^{\mathfrak{A}}\}:(x\sqcup y=a\Rightarrow x\suprel{\mu}y).
\]
\end{defn}
\begin{prop}
Let $(\mathfrak{A};\mathfrak{Z})$ be a co-separable filtrator with
join-closed core. An $A\in\mathfrak{Z}$ is connected regarding a
funcoid $\mu$ iff 
\[
\forall X,Y\in\mathfrak{Z}\setminus\{\bot^{\mathfrak{Z}}\}:(X\sqcup^{\mathfrak{Z}}Y=A\Rightarrow X\suprel{\mu}Y).
\]
\end{prop}
\begin{proof}
~
\begin{description}
\item [{$\Rightarrow$}] Obvious.
\item [{$\Leftarrow$}] Follows from co-separability.
\end{description}
\end{proof}
\begin{obvious}
For $\mathfrak{A}$ being a set of filters over a boolean lattice,
an element $a\in\mathfrak{A}$ is connected regarding a pointfree
funcoid $\mu$ iff it is connected regarding the funcoid $\mu\sqcap(a\times^{\mathsf{FCD}}a)$.\end{obvious}
\begin{xca}
Consider above without requirement of existence of least element.
\end{xca}

\section{Binary relations are pointfree funcoids}

Below for simplicity we will equate $\mathscr{T}A$ with $\subsets A$.

\begin{thm}
\label{pf-rel}Pointfree funcoids~$f$ between powerset posets~$\mathscr{T}A$
and~$\mathscr{T}B$ bijectively (moreover this bijection is an order-isomorphism)
correspond to morphisms~$p\in\mathbf{Rel}(A;B)$ by the formulas:
\begin{gather}
\supfun f=\rsupfun p,\quad\supfun{f^{-1}}=\rsupfun{p^{-1}};\label{pf-rel-f}\\
(x;y)\in\GR p\Leftrightarrow y\in\supfun f\{x\}\Leftrightarrow x\in\supfun{f^{-1}}\{y\}.\label{pf-rel-gr}
\end{gather}
\end{thm}
\begin{proof}
Suppose $p\in\mathbf{Rel}(A;B)$ and prove that there is a pointfree
funcoid~$f$ conforming to~(\ref{pf-rel-f}). Really, for every
$X\in\mathscr{T}A$, $Y\in\mathscr{T}B$ 
\begin{multline*}
Y\nasymp\supfun fX\Leftrightarrow Y\nasymp\rsupfun pX\Leftrightarrow Y\nasymp\supfun pX\Leftrightarrow\\
X\nasymp\supfun{p^{-1}}Y\Leftrightarrow X\nasymp\rsupfun{p^{-1}}Y\Leftrightarrow X\nasymp\supfun{f^{-1}}Y.
\end{multline*}


Now suppose $f\in\mathsf{pFCD}(\mathscr{T}A;\mathscr{T}B)$ and prove
that the relation defined by the formula~(\ref{pf-rel-gr}) exists.
To prove it, it's enough to show that $y\in\supfun f\{x\}\Leftrightarrow x\in\supfun{f^{-1}}\{y\}$.
Really, 
\[
y\in\supfun f\{x\}\Leftrightarrow\{y\}\nasymp\supfun f\{x\}\Leftrightarrow\{x\}\nasymp\supfun{f^{-1}}\{y\}\Leftrightarrow x\in\supfun{f^{-1}}\{y\}.
\]


It remains to prove that functions defined by~(\ref{pf-rel-f}) and~(\ref{pf-rel-gr})
are mutually inverse. (That these functions are monotone is obvious.)

Let $p_{0}\in\mathbf{Rel}(A;B)$ and $f\in\mathsf{pFCD}(\mathscr{T}A;\mathscr{T}B)$
corresponds to~$p_{0}$ by the formula~(\ref{pf-rel-f}); let $p_{1}\in\mathbf{Rel}(A;B)$
corresponds to~$f$ by the formula~(\ref{pf-rel-gr}). Then $p_{0}=p_{1}$
because 
\[
(x;y)\in\GR p_{0}\Leftrightarrow y\in\supfun f\{x\}\Leftrightarrow y\in\rsupfun{p_{0}}\{x\}\Leftrightarrow(x;y)\in\GR p_{1}.
\]


Let now $f_{0}\in\mathsf{pFCD}(\mathscr{T}A;\mathscr{T}B)$ and $p\in\mathbf{Rel}(A;B)$
corresponds to~$f_{0}$ by the formula~(\ref{pf-rel-gr}); let $f_{1}\in\mathsf{pFCD}(\mathscr{T}A;\mathscr{T}B)$
corresponds to~$p$ by the formula~(\ref{pf-rel-f}). Then $(x;y)\in\GR p\Leftrightarrow y\in\supfun{f_{0}}\{x\}$
and $\supfun{f_{1}}=\rsupfun p$; thus 
\[
y\in\supfun{f_{1}}\{x\}\Leftrightarrow y\in\rsupfun p\{x\}\Leftrightarrow(x;y)\in\GR p\Leftrightarrow y\in\supfun{f_{0}}\{x\}.
\]
So $\supfun{f_{0}}=\supfun{f_{1}}$. Similarly $\supfun{f_{0}^{-1}}=\supfun{f_{1}^{-1}}$.\end{proof}
\begin{prop}
The bijection defined by the theorem~\ref{pf-rel} preserves composition
and identities, that is is a functor between categories $\mathbf{Rel}$
and $(A;B)\mapsto\mathsf{pFCD}(\mathscr{T}A;\mathscr{T}B)$.\end{prop}
\begin{proof}
Let $\supfun f=\rsupfun p$ and $\supfun g=\rsupfun q$. Then $\supfun{g\circ f}=\supfun g\circ\supfun f=\rsupfun q\circ\rsupfun p=\rsupfun{q\circ p}$.
Likewise $\supfun{(g\circ f)^{-1}}=\rsupfun{(q\circ p)^{-1}}$. So
it preserves composition.

Let $p=1_{\mathbf{Rel}}^{A}$ for some set~$A$. Then $\supfun f=\rsupfun p=\rsupfun{1_{\mathbf{Rel}}^{A}}=\id_{\subsets A}$
and likewise $\supfun{f^{-1}}=\id_{\subsets A}$, that is $f$ is
an identity pointfree funcoid. So it preserves identities.\end{proof}
\begin{prop}
The bijection defined by the theorem~\ref{pf-rel} preserves reversal.\end{prop}
\begin{proof}
$\supfun{f^{-1}}=\rsupfun{p^{-1}}$.\end{proof}
\begin{prop}
The bijection defined by the theorem~\ref{pf-rel} preserves monovaluedness
and injectivity.\end{prop}
\begin{proof}
Because it is a functor which preserves reversal.\end{proof}
\begin{prop}
The bijection defined by the theorem~\ref{pf-rel} preserves domain
an image.\end{prop}
\begin{proof}
$\im f=\supfun f\top=\rsupfun p\top=\im p$, likewise for domain.\end{proof}
\begin{prop}
The bijection defined by the theorem~\ref{pf-rel} maps cartesian
products to corresponding funcoidal products.\end{prop}
\begin{proof}
$\supfun{A\times^{\mathsf{FCD}}B}X=\begin{cases}
B & \text{if }X\nasymp A\\
\bot & \text{if }X\asymp A
\end{cases}=\rsupfun{A\times B}X$. Likewise $\supfun{(A\times^{\mathsf{FCD}}B)^{-1}}X=\rsupfun{(A\times B)^{-1}}X$.\end{proof}

