\chapter{Manifolds and surfaces}

\section{Sides of a surface}

\begin{defn}
Let $\mu$ be an endofuncoid on a set~$U$.
\emph{Surface side} of a set~$T\subseteq\Ob\mu$ is a connected component
(regarding $\mu$) of the filter $(\rsupfun{\mu}T)\setminus T$.
\fxnote{$\mu$ is used twice in this definition. We may generalize
for two different funcoids instead.}
\end{defn}

\begin{xca}
Prove that open disk (in a usual 3-dimensional space) has two surface sides
and closed disk has one surface side.
\end{xca}

\section{Special points}

\begin{defn}
\emph{Cardinality special point} is  a limit point of a subset of~$T$ such
that all elements~$x$ of the subset have the same cardinality of connected
components of $(\rsupfun{\mu}\{x\})\setminus T$ but the the limit point has
a different cardinality.
\end{defn}

We can replace ``the same cardinality of connected components'' with
``filters being isomorphic'' and get what I call
\emph{isomorphism special points}. Correction: we do not need requirement
that other filters in vicinity of a special point are isomorphic
(neither we need to require that their cardinalities are the same).

\fxnote{Try to replace isomorphism~$f$ with continuous function~$f$.}

\begin{xca}
Excluding special points (either cardinality or isomorphism) from closed disk
produces open disk.
\end{xca}

Now define \emph{shift special points}.

Let $I$ be an interval on~$\mathbb{R}$ (containing zero?)

A point~$a$ is \emph{shift special} if there exists a transformation
(that is a continuous function $f:I\times\mu\to\mu$ such that:
\begin{enumerate}
  \item $f(0)$ is identity. \fxwarning{Is this condition needed?}
  \item for every sufficiently small~$\epsilon>0$ we have $f(\epsilon,a)\in T$;
  \item there is $\epsilon>0$ such that for every $0<\epsilon'<\epsilon$ we have
    $f(\epsilon')$ being not continuous at~$a$ regarding complete funcoid
    defined by the function $x\mapsto\rsupfun{\mu}\{x\}\setminus T$.
\end{enumerate}

We may consider to additonally require that every~$f(\epsilon)$ is isomorphism
of funcoids.

\begin{example}
$T$~is disk $\setcond{(x,y,0)}{x^2+y^2\leq 1}$. $f$~is the contraction
$(\epsilon,v)\mapsto\frac{1}{1+\epsilon}v$. $a=(1,0,0)$.

In the usual topology~$f$ is continuous. In
$x\mapsto\rsupfun{\mu}\{x\}\setminus T$ we have the function
$\epsilon\mapsto f(\epsilon)$ not continuous at zero.
So~$a$ is a shift special point.
\end{example}

\begin{proof}
$f (0) (v) = v$. Thus $\langle f (0) \rangle (\rsupfun{\mu} \{ a
\} \setminus T) = \rsupfun{\mu} \{ a \} \setminus T$ intersects
the plane $Z = 0$. But $f (0, a)$

??
\end{proof}

\begin{question}
Can we exclude real numbers from the play?
\end{question}

\begin{question}
How cardinality special points and shift special points are related with each
others?
\end{question}

\begin{question}
How the number of surface sides is related with usual surface sides for
manifolds?
\end{question}

Prove that $2$-manifold image which special points removed has the same number
of sides as the defined above.
