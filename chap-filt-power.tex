\chapter{Power of filters}

\section{Germs of functions}

\begin{defn}
  Functions $f, g \in \mathbf{Rel} (\Ob \mathcal{X} ; B)$
  \emph{are of the same $\mathcal{X}$-germ} for a filter object
  $\mathcal{X}$ iff there exists $X \in \up \mathcal{X}$ such that $f|_X
  = g|_X$.
\end{defn}

\begin{prop}
  Being of the same germ is an equivalence relation.
\end{prop}

\begin{proof}
  
  \begin{description}
    \item[Reflexivity] Take arbitrary $X \in \up \mathcal{X}$.
    
    \item[Symmetry] Obvious.
    
    \item[Transitivity] Let $f|_X = g|_X$ and $g|_Y = h|_Y$. Then $f|_{X \cap
    Y} = h|_{X \cap Y}$.
  \end{description}
\end{proof}

\begin{defn}
  A \emph{germ} is an equivalence class of being the same germ.
\end{defn}

\begin{obvious}
Every germ is a filter on $\mathbf{Set}$.
\end{obvious}

\begin{thm}
  Let $A$, $B$ be sets. Suppose $\card B \neq 1$. \fxwarning{It seems that the theorem holds even if $\card B = 1$.}
  
  The following are mutually inverse bijections between monovalued reloids $f
  : A \rightarrow B$ with $\dom f = \mathcal{X}$ and $\mathcal{X}$-germs
  $S$ of functions $A \rightarrow B$ for $\mathcal{X} \in \mathscr{F} A$:
  \begin{itemize}
    \item $f \mapsto \up^{\mathbf{Set}} f$;
    
    \item $S \mapsto \bigsqcap^{\mathsf{RLD}} S$.
  \end{itemize}
\end{thm}

\begin{proof}
  We can assume that $A, B \neq \emptyset$ because otherwise the theorem is
  obvious.
  
  First prove that $\up^{\mathbf{Set}} f$ is an
  $\mathcal{X}$-germ. Really, $F \in \up^{\mathbf{Set}} f
  \Leftrightarrow F \sqsupseteq f \Leftrightarrow F|_{\mathcal{X}} = f
  \Leftrightarrow \exists X \in \up \mathcal{X} : F|_X \sqsupseteq f$;
  thus $F, G \in \up^{\mathbf{Set}} f \Rightarrow \exists X \in
  \up \mathcal{X} : F|_X \sqsupseteq f \wedge \exists Y \in \up
  \mathcal{X} : G|_Y \sqsupseteq f \Rightarrow \exists X \in \up
  \mathcal{X} : F|_{X \cap Y} \sqsupseteq f \wedge \exists Y \in \up
  \mathcal{X} : G|_{X \cap Y} \sqsupseteq f \Rightarrow \exists Z \in
  \up \mathcal{X} : (F|_Z \sqsupseteq f \wedge G|_Z \sqsupseteq f)
  \Rightarrow \exists Z \in \up \mathcal{X} : (F \sqcap G) |_Z
  \sqsupseteq f$ and $F \in \up^{\mathbf{Set}} f \wedge \exists
  X \in \up \mathcal{X} : F|_X = G|_X \Rightarrow F \sqsupseteq f \wedge
  F|_{\mathcal{X}} = G|_{\mathcal{X}} \Rightarrow G|_{\mathcal{X}} \sqsupseteq
  f \Rightarrow G \in \up^{\mathbf{Set}} f$. We have proved
  that $\up^{\mathbf{Set}} f$ is an equivalence class of the
  suitable equivalence relation, that is $\up^{\mathbf{Set}} f$
  is an $\mathcal{X}$-germ.
  
  That $\bigsqcap^{\mathsf{RLD}} S$ is a monovalued reloid is obvious.
  We need to prove that $\im \bigsqcap^{\mathsf{RLD}} S =
  \mathcal{X}$.
  
  If $\mathcal{X} = X$ then obviously $S$ has just one element $F$ and
  $\im \bigsqcap^{\mathsf{RLD}} S = \im F = X =
  \mathcal{X}$. Otherwise for every $X \in \up \mathcal{X}$ there are
  elements $F$, $G$ of $S$ such that $\dom (F \sqcap G) \sqsubseteq X$
  (using $\card B > 1$).
  
  By properties of generalized filter bases $X \times \top \sqsupseteq
  \bigsqcap^{\mathsf{RLD}} S \Leftrightarrow \exists F, G \in S : X
  \times \top \sqsupseteq F \sqcap G \Leftrightarrow X \sqsupseteq
  \mathcal{X}$. Thus $\im \bigsqcap^{\mathsf{RLD}} S =
  \mathcal{X}$.
  
  It remains to prove that our correspondences are mutually inverse.
  
  Let $f_0 : A \rightarrow B$ be a monovalued reloid and $\dom f =
  \mathcal{X}$. Let $S = \up^{\mathbf{Set}} f$ and $f_1 =
  \bigsqcap^{\mathsf{RLD}} S$. We need to prove $f_1 = f_0$. Really,
  $f_1 \sqsupseteq f_0$ is obvious and $f_1 = \bigsqcap^{\mathsf{RLD}}
  \up^{\mathbf{Set}} f_0 \sqsubseteq
  \bigsqcap^{\mathsf{RLD}} \up^{\mathbf{Rel}} f_0$.
  
  Let $S_0$ be an $\mathcal{X}$-germ of functions $A \rightarrow B$. Let $f =
  \bigsqcap^{\mathsf{RLD}} S_0$ and $S_1 =
  \up^{\mathbf{Set}} f$. We need to prove $S_1 = S_0$. Really,
  \[ S_1 = \up^{\mathbf{Set}}  \bigsqcap^{\mathsf{RLD}}
     S_0 = \mathbf{Set} \cap \up^{\mathbf{Rel}} 
     \bigsqcap^{\mathsf{RLD}} S_0 = \mathbf{Set} \cap
     \up^{\mathbf{Rel}}  \bigsqcap^{\mathsf{RLD}}
     \up \bigsqcap^{\mathscr{F} (\mathbf{Set})} S_0 = \up
     \bigsqcap^{\mathscr{F} (\mathbf{Set})} S_0 = S_0 . \]
\end{proof}
\section{Power of filters}

Let's define $\mathcal{Y}^{\mathcal{X}}$ for filters~$\mathcal{X}$,~$\mathcal{Y}$:

First define $Y^{\mathcal{X}}$ for a set~$Y$:
\[ Y^{\mathcal{X}} = \setcond{ f \in \mathsf{RLD} (\Ob \mathcal{X}
   ; Y) }{ \dom f = \mathcal{X} \wedge f\text{ is monovalued} } . \]

Now $\mathcal{Y}^{\mathcal{X}} = \bigsqcap^{\mathsf{RLD}}_{Y \in
\up \mathcal{Y}} Y^{\mathcal{X}}$.

TODO: Check $\mathcal{Y}^1 \cong \mathcal{Y}$; $\mathcal{Z}^{\mathcal{X}
\times^{\mathsf{RLD}} \mathcal{Y}} \cong
(\mathcal{Z}^{\mathcal{X}})^{\mathcal{Y}}$; $\mathcal{Z}^{\mathcal{X} \amalg
\mathcal{Y}} \cong \mathcal{Z}^{\mathcal{X}} \times^{\mathsf{RLD}}
\mathcal{Z}^{\mathcal{Y}}$; $\mathcal{Y}^2 \cong \mathcal{Y}\times^{\mathsf{RLD}}\mathcal{Y}$;
$\mathcal{Y}^0 \cong 1$; $\mathcal{Y}^N \cong \prod^{\mathsf{RLD}}_{n\in N}\mathcal{Y}$.
More formulas at \url{https://en.wikipedia.org/wiki/Cartesian_closed_category}.

Isn't it a cartesian closed category?

Andreas Blass says it is not cartesian closed: ``Unfortunately, the two categories of filters in my paper are
not cartesian closed.  This is mentioned in a parenthetical comment
near the bottom of page 141.  The operation of cartesian product with
the cofinite filter on the natural numbers has no right adjoint,
because it does not preserve infinite coproducts.''
\url{http://matwbn.icm.edu.pl/ksiazki/fm/fm94/fm94115.pdf}

But it is probably a braided closed monoidal category?