\chapter{Generalized cofinite filters}

The following is a straightforward generalization of cofinite filter.

\begin{defn}
  $\Omega_{1 a} = \bigsqcap^{\mathfrak{A}}_{X \in
  \operatorname{coatoms}^{\mathfrak{Z}}} X$; $\Omega_{1 b} =
  \bigsqcap^{\mathfrak{A}}_{X \in \operatorname{coatoms}^{\mathfrak{A}}} X$.
\end{defn}

\begin{prop}
  For primary filtrators $\Omega_{1 a} = \Omega_{1 b}$.
\end{prop}

\begin{proof}
  Proposition 531.
\end{proof}

Thus for primary filtrators I will denote it just $\Omega$.

\begin{prop}
  Let $\mathfrak{Z}$ be a subset of $\subsets U$. Let it be a
  meet-semilattice with greatest element \fxnote{existence of greatest element
  seems unnecessary.} Let also every non-coempty cofinite set lies in
  $\mathfrak{Z}$. Then
  \begin{equation}
    \corestar \Omega = \setcond{ Y \in \mathfrak{Z} }{
    \card \atoms^{\mathfrak{Z}} Y \geq \omega } .
    \label{d-cofin}
  \end{equation}
\end{prop}

\begin{proof}
  $\Omega$ exists by corollary 496.
  
  $Y \in \corestar \Omega \Leftrightarrow Y \nasymp^{\mathfrak{A}} 
  \bigsqcap^{\mathfrak{A}}_{X \in \operatorname{coatoms}^{\mathfrak{Z}}} X
  \Leftrightarrow \text{(by properties of filter bases)} \Leftrightarrow
  \forall S \in \subsets_{\operatorname{fin}} \operatorname{coatoms}^{\mathfrak{Z}} : Y
  \nasymp^{\mathfrak{A}} \bigsqcap^{\mathfrak{A}} S \Leftrightarrow
  \text{(theorem 512)} \Leftrightarrow \forall S \in \subsets_{\operatorname{fin}}
  \operatorname{coatoms}^{\mathfrak{Z}} : Y \nasymp \bigsqcap S \Leftrightarrow
  \forall K \in \subsets_{\operatorname{fin}} U : Y \setminus K \neq \emptyset
  \Leftrightarrow \card Y \geq \omega \Leftrightarrow \card
  \atoms^{\mathfrak{Z}} Y \geq \omega$.
  \fxwarning{Define $\subsets_{\operatorname{fin}}$.}
\end{proof}

\begin{cor}
  Formula (\ref{d-cofin}) holds for both reloids and funcoids.
\end{cor}

\begin{proof}
  For reloiods it's straightforward, for funcoids take that they are
  isomorphic to filters on lattice $\Gamma$.
\end{proof}

\begin{cor}
$\Omega^{\mathsf{FCD}} \ne \bot^{\mathsf{FCD}}$ (for $\mathsf{FCD}(A,B)$ where $A\times B$ is an infinite set).
\end{cor}

\begin{prop}
  $\Cor \Omega_{1 a} = \bot^{\mathfrak{Z}}$ if $\forall \alpha \in
  \atoms^{\mathfrak{Z}} \exists X \in \up \Omega : a
  \nsqsubseteq X$ and $\mathfrak{Z}$ is an atomic poset.
\end{prop}

\begin{proof}
  Suppose $\alpha \in \atoms^{\mathfrak{Z}} \Cor \Omega$. Then
  $\exists X \in \up \Omega : \alpha \nsqsubseteq X$.
  Therefore $\alpha \notin \Cor \Omega$. So $\atoms^{\mathfrak{Z}}
  \Cor \Omega_{1 a} = \emptyset$ and thus by atomicity $\Cor
  \Omega_{1 a} = \bot^{\mathfrak{Z}}$.
\end{proof}

\begin{cor}
  $\Cor \Omega^{\mathsf{FCD}} = \bot$.
\end{cor}

\begin{prop}
  $\Omega_{1 a} = \max \setcond{ \mathcal{X} \in \mathfrak{A} }{
  \Cor \mathcal{X} = \bot^{\mathfrak{Z}} }$ provided
  $\forall \alpha \in \atoms^{\mathfrak{Z}} \exists X \in \up
  \Omega : a \nsqsubseteq X$ and $\mathfrak{Z}$ is an atomic
  poset and $(\mathfrak{A}; \mathfrak{Z})$ is a filtered filtrator.
\end{prop}

\begin{proof}
  Due the last proposition, it is enough to show that $\Cor \mathcal{X}
  = \bot^{\mathfrak{Z}} \Rightarrow \mathcal{X} \sqsubseteq \Omega_{1 a}$ for
  every $\mathcal{X} \in \mathfrak{A}$.
  
  Let $\Cor \mathcal{X} = \bot^{\mathfrak{Z}}$ for some $\mathcal{X} \in
  \mathfrak{A}$. Because of our filtrator being filtered, it's enough to show
  $X \in \up \Omega_{1 a}$. $X = a_0 \sqcap \ldots \sqcap a_n$ for $a_i$
  being coatoms of $\mathfrak{Z}$. $a_i \sqsupseteq \mathcal{X}$ because
  otherwise $a_i \nsqsupseteq \Cor \mathcal{X}$. So $X
  \in \up \Omega_{1 a}$.
\end{proof}

\begin{prop}
  $\up \Omega = \setcond{ \bigsqcap S }{ S \in
  \subsets_{\operatorname{fin}} \operatorname{coatoms}^{\mathfrak{Z}} }$.
\end{prop}

\begin{proof}
  Because $\setcond{ \bigsqcap S }{ S \in
  \subsets_{\operatorname{fin}} \operatorname{coatoms}^{\mathfrak{Z}} }$ is a
  filter.
\end{proof}

\begin{cor}
  $\up \Omega^{\mathsf{FCD}} = \up
  \Omega^{\mathsf{RLD}}$.
\end{cor}

\begin{defn}
$\Omega_{1c} =
\bigsqcup(\atoms^{\mathfrak{A}}\setminus\mathfrak{Z})$.
\end{defn}

\begin{prop}
$\Omega_{1c} = \Omega_{1a}$ provided
  $\forall \alpha \in \atoms^{\mathfrak{Z}} \exists X \in \up
  \Omega : a \nsqsubseteq X$ and $\mathfrak{Z}$ is an atomic
  poset and $(\mathfrak{A}; \mathfrak{Z})$ is a down-aligned
  filtered filtrator and $\mathfrak{Z}$ is atomistic and ??.
\end{prop}

\begin{proof}
For $x\in\atoms^{\mathfrak{A}}\setminus\mathfrak{Z}$ we have
$\Cor x=\bot$ because otherwise $\bot\ne\Cor x\sqsubset x$.
Thus by previous $x\sqsubseteq\Omega_{1a}$ and so
$\bigsqcup(\atoms^{\mathfrak{A}}\setminus\mathfrak{Z})\sqsubseteq\Omega_{1a}$.

If $x\in\atoms\Omega_{1a}$ then $x\notin\mathfrak{Z}$ because otherwise
$\Cor x\ne\bot$. So \[ \Omega_{1a}=\bigsqcup\atoms \Omega_{1a}=
\bigsqcup(\atoms \Omega_{1a}\setminus\mathfrak{Z})\sqsubseteq
\bigsqcup(\atoms^{\mathfrak{A}}\setminus\mathfrak{Z}). \]
\end{proof}

\begin{thm}
$\Cor$ is the lower adjoint of $\Omega_{1?}\sqcup^{\mathfrak{A}}-$
provided that ??. \fxnote{Prove it in two ways (for two definitions
of adjoint elements).}
\end{thm}

\begin{proof} (1) [atomistic]
It's enough to prove that
\[
\mathcal{X}\sqsubseteq\Omega\sqcup^{\mathfrak{A}}\Cor\mathcal{X}\quad\text{and}\quad\Cor(\Omega\sqcup^{\mathfrak{A}}\mathcal{Y})\sqsubseteq\mathcal{Y}.
\]
$\Cor(\Omega\sqcup^{\mathfrak{A}}\mathcal{Y}) =
\text{(theorem~\bookref{dual-core-join})} =
\Cor\Omega\sqcup^{\mathfrak{Z}}\Cor\mathcal{Y} =
\bot^{\mathfrak{Z}}\sqcup^{\mathfrak{Z}}\Cor\mathcal{Y} =
\Cor\mathcal{Y}\sqsubseteq\mathcal{Y}$.

$\Omega\sqcup^{\mathfrak{A}}\Cor\mathcal{X} =
\bigsqcup\atoms(\Omega\sqcup^{\mathfrak{A}}\Cor\mathcal{X}) =
\bigsqcup(\atoms\Omega\cup\Cor\mathcal{X}) =
\bigsqcup\atoms\Omega\sqcup\bigsqcup\Cor\atoms\mathcal{X}) \sqsupseteq
\bigsqcup(\atoms\mathcal{X}\setminus\mathfrak{Z}) \sqcup
\bigsqcup(\atoms\mathcal{X}\cap\mathfrak{Z}) =
\bigsqcup((\atoms\mathcal{X}\setminus\mathfrak{Z}) \cup
(\atoms\mathcal{X}\cap\mathfrak{Z}) =
\bigsqcup\atoms\mathcal{X} = \mathcal{X}$.
\end{proof}

\begin{proof} (2) [atomistic]
We will prove $\Cor\mathcal{X} \sqsubseteq \mathcal{Y} \Leftrightarrow
\mathcal{X} \sqsubseteq \Omega \sqcup \mathcal{Y}$.

By atomisticity it is equivalent to:
$\atoms^{\mathfrak{A}}\Cor\mathcal{X} \subseteq \atoms^{\mathfrak{A}}\mathcal{Y}
\Leftrightarrow
\atoms^{\mathfrak{A}}\mathcal{X} \subseteq \atoms^{\mathfrak{A}}(\Omega \sqcup \mathcal{Y})$;
(theorem~\bookref{dual-core-join})
$\atoms^{\mathfrak{A}}\Cor\mathcal{X} \sqsubseteq \atoms^{\mathfrak{A}}\mathcal{Y}
\Leftrightarrow
\atoms^{\mathfrak{A}}\mathcal{X} \sqsubseteq \atoms^{\mathfrak{A}}\Omega \cup \atoms^{\mathfrak{A}}\mathcal{Y}$;
what by below is equivalent to:
$\atoms^{\mathfrak{Z}}  \mathcal{X} \subseteq
\atoms^{\mathfrak{Z}}  \mathcal{Y} \Leftrightarrow
\atoms^{\mathfrak{A}}  \mathcal{X} \subseteq \atoms^{\mathfrak{A}}
\Omega \cup \atoms^{\mathfrak{A}}  \mathcal{Y}$.

$\Cor \mathcal{X} \sqsubseteq \mathcal{Y} \Leftrightarrow
\atoms^{\mathfrak{A}} \Cor \mathcal{X} \subseteq
\atoms^{\mathfrak{A}}  \mathcal{Y} \Rightarrow
\atoms^{\mathfrak{Z}} \Cor \mathcal{X} \subseteq
\atoms^{\mathfrak{Z}}  \mathcal{Y} \Leftrightarrow
\atoms^{\mathfrak{Z}}  \mathcal{X} \subseteq \atoms^{\mathfrak{Z}}
\mathcal{Y}$;

$\atoms^{\mathfrak{Z}}  \mathcal{X} \subseteq
\atoms^{\mathfrak{Z}}  \mathcal{Y} \Rightarrow
\text{(theorem~\bookref{cor-join-atom})} \Rightarrow
\Cor \mathcal{X}
\sqsubseteq \Cor \mathcal{Y} \Rightarrow \Cor \mathcal{X}
\sqsubseteq \mathcal{Y}$.

Finishing the proof
$\atoms^{\mathfrak{A}} \mathcal{X} \subseteq \atoms^{\mathfrak{A}}
\Omega \cup \atoms^{\mathfrak{A}} \mathcal{Y} \Leftrightarrow
\atoms^{\mathfrak{A}} \mathcal{X} \subseteq
(\atoms^{\mathfrak{A}} \setminus \mathfrak{Z})
\cup \atoms^{\mathfrak{A}} \mathcal{Y} \Leftrightarrow
\atoms^{\mathfrak{Z}} \mathcal{X} \subseteq
\atoms^{\mathfrak{A}} \mathcal{Y} \Leftrightarrow
\atoms^{\mathfrak{Z}} \mathcal{X} \subseteq
\atoms^{\mathfrak{Z}} \mathcal{Y}$.
\end{proof}

\begin{prop}
  ~  
  \begin{enumerate}
    \item $\langle \Omega^{\mathsf{FCD}} \rangle \{ x \} = \Omega^U$;
    
    \item $\langle \Omega^{\mathsf{FCD}} \rangle p = \top$ for every
    nontrivial atomic filter $p$.
  \end{enumerate}
\end{prop}

\begin{proof}
  $\langle \Omega^{\mathsf{FCD}} \rangle \{ x \} =
  \bigsqcap^{\mathfrak{A}}_{y \in U} (U \setminus \{ y \}) = \Omega^U$;
  $\langle \Omega^{\mathsf{FCD}} \rangle p = \bigsqcap^{\mathfrak{A}}_{y
  \in U} \top = \top$.
\end{proof}

\begin{prop}
  $\tofcd \Omega^{\mathsf{RLD}} =
  \Omega^{\mathsf{FCD}}$.
\end{prop}

\begin{proof}
  $\tofcd \Omega^{\mathsf{RLD}} =
  \bigsqcap^{\mathsf{FCD}} \up \Omega^{\mathsf{RLD}} =
  \Omega^{\mathsf{FCD}}$.
\end{proof}

\begin{prop}
  $(\mathsf{RLD})_{\operatorname{out}} \Omega^{\mathsf{FCD}} =
  \Omega^{\mathsf{RLD}}$.
\end{prop}

\begin{proof}
  $(\mathsf{RLD})_{\operatorname{out}} \Omega^{\mathsf{FCD}} =
  \bigsqcap^{\mathsf{RLD}} \up \Omega^{\mathsf{FCD}} =
  \bigsqcap^{\mathsf{RLD}} \up \Omega^{\mathsf{RLD}} =
  \Omega^{\mathsf{RLD}}$.
\end{proof}

\begin{prop}
  $(\mathsf{RLD})_{\operatorname{in}} \Omega^{\mathsf{FCD}} = \Omega^{\mathsf{RLD}}$.
\end{prop}

\begin{proof}
  \begin{multline*}
  (\mathsf{RLD})_{\operatorname{in}} \Omega^{\mathsf{FCD}} = \bigsqcup
  \setcond{ a \times^{\mathsf{RLD}} b }{ a \in
  \atoms^{\mathfrak{A}}, b \in \atoms^{\mathfrak{A}}, a
  \times^{\mathsf{FCD}} b \sqsubseteq \Omega^{\mathsf{FCD}}
  } = \\
  \bigsqcup \setcond{ a \times^{\mathsf{RLD}} b }{
  a \in \atoms^{\mathfrak{A}}, b \in
  \atoms^{\mathfrak{A}}, \text{not $a$ and $b$ both atomic} } = \\
  \bigsqcup \setcond{ \bigsqcup \atoms (a \times^{\mathsf{RLD}} b)
  }{ a \in \atoms^{\mathfrak{A}}, b \in
  \atoms^{\mathfrak{A}}, \text{not $a$ and $b$ both atomic} } = \\
  \bigsqcup \bigcup \setcond{ \atoms (a \times^{\mathsf{RLD}} b) }{
  a \in \atoms^{\mathfrak{A}}, b \in
  \atoms^{\mathfrak{A}}, \text{not $a$ and $b$ both atomic} } = \\
  \bigsqcup \left( \text{nontrivial atomic reloids under $A \times B$} \right) =
  \Omega^{\mathsf{RLD}}.
  \end{multline*}
\end{proof}
