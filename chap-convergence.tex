
\chapter{Convergence of funcoids}


\section{Convergence}

The following generalizes the well-known notion of a filter convergent
to a point or to a set:
\begin{defn}
\index{converges!regarding funcoid}A filter $\mathcal{F}\in\mathscr{F}(\Dst\mu)$
\emph{converges} to a filter $\mathcal{A}\in\mathscr{F}(\Src\mu)$
regarding a funcoid $\mu$ ($\mathcal{F}\overset{\mu}{\rightarrow}\mathcal{A}$)
iff $\mathcal{F}\sqsubseteq\supfun{\mu}\mathcal{A}$.
\end{defn}

\begin{defn}
\index{converges!regarding funcoid}A funcoid $f$ \emph{converges}
to a filter $\mathcal{A}\in\mathscr{F}(\Src\mu)$ regarding a funcoid
$\mu$ where $\Dst f=\Dst\mu$ (denoted $f\overset{\mu}{\rightarrow}\mathcal{A}$)
iff $\im f\sqsubseteq\supfun{\mu}\mathcal{A}$ that is iff $\im f\overset{\mu}{\rightarrow}\mathcal{A}$.
\end{defn}

\begin{defn}
\index{converges!regarding funcoid}A funcoid $f$ \emph{converges}
to a filter $\mathcal{A}\in\mathscr{F}(\Src\mu)$ on a filter $\mathcal{B}\in\mathscr{F}(\Src f)$
regarding a funcoid $\mu$ where $\Dst f=\Dst\mu$ iff $f|_{\mathcal{B}}\overset{\mu}{\rightarrow}\mathcal{A}$.\end{defn}
\begin{rem}
We can define also convergence for a reloid $f$: $f\overset{\mu}{\rightarrow}\mathcal{A}\Leftrightarrow\im f\sqsubseteq\supfun{\mu}\mathcal{A}$
or what is the same $f\overset{\mu}{\rightarrow}\mathcal{A}\Leftrightarrow\tofcd f\overset{\mu}{\rightarrow}\mathcal{A}$.\end{rem}
\begin{thm}
Let $f$, $g$ be funcoids, $\mu$, $\nu$ be endofuncoids, $\Dst f=\Src g=\Ob\mu$,
$\Dst g=\Ob\nu$, $\mathcal{A}\in\mathscr{F}(\Ob\mu)$. If $f\overset{\mu}{\rightarrow}\mathcal{A}$,
\[
g|_{\supfun{\mu}\mathcal{A}}\in\continuous(\mu\sqcap(\supfun{\mu}\mathcal{A}\times^{\mathsf{FCD}}\supfun{\mu}\mathcal{A});\nu),
\]
and $\supfun{\mu}\mathcal{A}\sqsupseteq\mathcal{A}$, then $g\circ f\overset{\nu}{\rightarrow}\supfun g\mathcal{A}$.\end{thm}
\begin{proof}
~
\begin{gather*}
\im f\sqsubseteq\supfun{\mu}\mathcal{A};\\
\supfun g\im f\sqsubseteq\supfun g\supfun{\mu}\mathcal{A};\\
\im(g\circ f)\sqsubseteq\supfun{g|_{\supfun{\mu}\mathcal{A}}}\supfun{\mu}\mathcal{A};\\
\im(g\circ f)\sqsubseteq\supfun{g|_{\supfun{\mu}\mathcal{A}}}\supfun{\mu\sqcap(\supfun{\mu}\mathcal{A}\times^{\mathsf{FCD}}\supfun{\mu}\mathcal{A})}\mathcal{A};\\
\im(g\circ f)\sqsubseteq\supfun{g|_{\supfun{\mu}\mathcal{A}}\circ(\mu\sqcap(\supfun{\mu}\mathcal{A}\times^{\mathsf{FCD}}\supfun{\mu}\mathcal{A}))}\mathcal{A};\\
\im(g\circ f)\sqsubseteq\supfun{\nu\circ g|_{\supfun{\mu}\mathcal{A}}}\mathcal{A};\\
\im(g\circ f)\sqsubseteq\supfun{\nu\circ g}\mathcal{A};\\
g\circ f\overset{\nu}{\rightarrow}\supfun g\mathcal{A}.
\end{gather*}
\end{proof}
\begin{cor}
Let $f$, $g$ be funcoids, $\mu$, $\nu$ be endofuncoids, $\Dst f=\Src g=\Ob\mu$,
$\Dst g=\Ob\nu$, $\mathcal{A}\in\mathscr{F}(\Ob\mu)$. If $f\overset{\mu}{\rightarrow}\mathcal{A}$,
$g\in\continuous(\mu;\nu)$, and $\supfun{\mu}\mathcal{A}\sqsupseteq\mathcal{A}$
then $g\circ f\overset{\nu}{\rightarrow}\supfun g\mathcal{A}$.\end{cor}
\begin{proof}
From the last theorem and theorem \ref{rect-cont}.
\end{proof}

\section{Relationships between convergence and continuity}
\begin{lem}
Let $\mu$, $\nu$ be endofuncoids, $f\in\mathsf{FCD}(\Ob\mu;\Ob\nu)$,
$\mathcal{A}\in\mathscr{F}(\Ob\mu)$, $\Src f=\Ob\mu$, $\Dst f=\Ob\nu$.
If $f\in\continuous(\mu|_{\mathcal{A}};\nu)$ then
\[
f|_{\supfun{\mu}\mathcal{A}}\overset{\nu}{\rightarrow}\supfun f\mathcal{A}\Leftrightarrow\supfun{f\circ\mu|_{\mathcal{A}}}\mathcal{A}\sqsubseteq\supfun{\nu\circ f}\mathcal{A}.
\]
\end{lem}
\begin{proof}
~
\begin{multline*}
f|_{\supfun{\mu}\mathcal{A}}\overset{\nu}{\rightarrow}\supfun f\mathcal{A}\Leftrightarrow\im f|_{\supfun{\mu}\mathcal{A}}\sqsubseteq\supfun{\nu}\supfun f\mathcal{A}\Leftrightarrow\\
\supfun f\supfun{\mu}\mathcal{A}\sqsubseteq\supfun{\nu}\supfun f\mathcal{A}\Leftrightarrow\supfun{f\circ\mu}\mathcal{A}\sqsubseteq\supfun{\nu\circ f}\mathcal{A}\Leftrightarrow\supfun{f\circ\mu|_{\mathcal{A}}}\mathcal{A}\sqsubseteq\supfun{\nu\circ f}\mathcal{A}.
\end{multline*}
\end{proof}
\begin{thm}
Let $\mu$, $\nu$ be endofuncoids, $f\in\mathsf{FCD}(\Ob\mu;\Ob\nu)$,
$\mathcal{A}\in\mathscr{F}(\Ob\mu)$, $\Src f=\Ob\mu$, $\Dst f=\Ob\nu$.
If $f\in\continuous(\mu|_{\mathcal{A}};\nu)$ then $f|_{\supfun{\mu}\mathcal{A}}\overset{\nu}{\rightarrow}\supfun f\mathcal{A}$.\end{thm}
\begin{proof}
~
\begin{multline*}
f|_{\supfun{\mu}\mathcal{A}}\overset{\nu}{\rightarrow}\supfun f\mathcal{A}\Leftrightarrow\text{(by the lemma)}\Leftrightarrow\supfun{f\circ\mu|_{\mathcal{A}}}\mathcal{A}\sqsubseteq\supfun{\nu\circ f}\mathcal{A}\Leftarrow\\
f\circ\mu|_{\mathcal{A}}\sqsubseteq\nu\circ f\Leftrightarrow f\in\continuous(\mu|_{\mathcal{A}};\nu).
\end{multline*}
\end{proof}
\begin{cor}
Let $\mu$, $\nu$ be endofuncoids, $f\in\mathsf{FCD}(\Ob\mu;\Ob\nu)$,
$\mathcal{A}\in\mathscr{F}(\Ob\mu)$, $\Src f=\Ob\mu$, $\Dst f=\Ob\nu$.
If $f\in\continuous(\mu;\nu)$ then $f|_{\supfun{\mu}\mathcal{A}}\overset{\nu}{\rightarrow}\supfun f\mathcal{A}$.\end{cor}
\begin{thm}
Let $\mu$, $\nu$ be endofuncoids, $f\in\mathsf{FCD}(\Ob\mu;\Ob\nu)$,
$\mathcal{A}\in\mathscr{F}(\Ob\mu)$ be an ultrafilter, $\Src f=\Ob\mu$,
$\Dst f=\Ob\nu$. $f\in\continuous(\mu|_{\mathcal{A}};\nu)$ iff $f|_{\supfun{\mu}\mathcal{A}}\overset{\nu}{\rightarrow}\supfun f\mathcal{A}$.\end{thm}
\begin{proof}
~
\begin{multline*}
f|_{\supfun{\mu}\mathcal{A}}\overset{\nu}{\rightarrow}\supfun f\mathcal{A}\Leftrightarrow\text{(by the lemma)}\Leftrightarrow\supfun{f\circ\mu|_{\mathcal{A}}}\mathcal{A}\sqsubseteq\supfun{\nu\circ f}\mathcal{A}\Leftrightarrow\\
\text{(used the fact that \ensuremath{\mathcal{A}} is an ultrafilter)}\\
f\circ\mu|_{\mathcal{A}}\sqsubseteq\nu\circ f|_{\mathcal{A}}\Leftrightarrow f\circ\mu|_{\mathcal{A}}\sqsubseteq\nu\circ f\Leftrightarrow f\in\continuous(\mu|_{\mathcal{A}};\nu).
\end{multline*}

\end{proof}

\section{Limit}
\begin{defn}
\index{limit}$\lim^{\mu}f=a$ iff $f\overset{\mu}{\rightarrow}\uparrow^{\Src\mu}\{a\}$
for a $T_{2}$-separable funcoid $\mu$ and a non-empty funcoid $f$
such that $\Dst f=\Dst\mu$.
\end{defn}
It is defined correctly, that is $f$ has no more than one limit.
\begin{proof}
Let $\lim^{\mu}f=a$ and $\lim^{\mu}f=b$. Then $\im f\sqsubseteq\rsupfun{\mu}@\{a\}$
and $\im f\sqsubseteq\rsupfun{\mu}@\{b\}$.

Because $f\ne\bot^{\mathsf{FCD}(\Src f;\Dst f)}$ we have $\im f\ne\bot^{\mathscr{F}(\Dst f)}$;
$\rsupfun{\mu}@\{a\}\sqcap\rsupfun{\mu}@\{b\}\ne\bot^{\mathscr{F}(\Dst f)}$;
$\uparrow^{\Src\mu}\{b\}\sqcap\supfun{\mu^{-1}}\rsupfun{\mu}@\{a\}\ne\bot^{\mathscr{F}(\Src\mu)}$;
$\uparrow^{\Src\mu}\{b\}\sqcap\supfun{\mu^{-1}\circ\mu}@\{a\}\ne\bot^{\mathscr{F}(\Src\mu)}$;
$@\{a\}\suprel{\mu^{-1}\circ\mu}@\{b\}$.
Because $\mu$ is $T_{2}$-separable we have $a=b$.\end{proof}
\begin{defn}
$\lim_{\mathcal{B}}^{\mu}f=\lim^{\mu}(f|_{\mathcal{B}})$.\end{defn}
\begin{rem}
We can also in an obvious way define limit of a reloid.
\end{rem}

\section{Generalized limit}

For further consideration on using the below defined generalized limit
see this wiki:

\href{http://portonmath.tiddlyspace.com}{http://portonmath.tiddlyspace.com}


\subsection{\index{limit!generalized}Definition}

Let $\mu$ and $\nu$ be endofuncoids. Let $G$ be a transitive permutation
group on $\Ob\mu$.

For an element $r\in G$ we will denote $\uparrow r=\uparrow^{\mathsf{FCD}(\Ob\mu;\Ob\mu)}r$.

We require that $\mu$ and every $r\in G$ commute, that is
\[
\mu\circ\uparrow r=\uparrow r\circ\mu.
\]


We require for every $y\in\Ob\nu$ 
\begin{equation}
\nu\sqsupseteq\rsupfun{\nu}@\{y\}\times^{\mathsf{FCD}}\rsupfun{\nu}@\{y\}.\label{lim-squares}
\end{equation}

\begin{prop}
Formula (\ref{lim-squares}) follows from $\nu\sqsupseteq\nu\circ\nu^{-1}$.\end{prop}
\begin{proof}
Let $\nu\sqsupseteq\nu\circ\nu^{-1}$. Then
\begin{align*}
\rsupfun{\nu}@\{y\}\times^{\mathsf{FCD}}\rsupfun{\nu}@\{y\} & =\\
\supfun{\nu}@\{y\}\times^{\mathsf{FCD}}\supfun{\nu}@\{y\} & =\\
\nu\circ(\uparrow^{\Ob\nu}\{y\}\times^{\mathsf{FCD}}\uparrow^{\Ob\nu}\{y\})\circ\nu^{-1} & =\\
\nu\circ\uparrow^{\mathsf{FCD}(\Ob\nu;\Ob\nu)}(\{y\}\times\{y\})\circ\nu^{-1} & \sqsubseteq\\
\nu\circ1_{\Ob\nu}^{\mathsf{FCD}}\circ\nu^{-1} & =\\
\nu\circ\nu^{-1} & \sqsubseteq\nu.
\end{align*}
\end{proof}
\begin{rem}
The formula (\ref{lim-squares}) usually works if $\nu$ is a proximity.
It does not work if $\mu$ is a pretopology or preclosure.
\end{rem}
We are going to consider (generalized) limits of arbitrary functions
acting from $\Ob\mu$ to $\Ob\nu$. (The functions in consideration
are not required to be continuous.)
\begin{rem}
Most typically $G$ is the group of translations of some topological
vector space.
\end{rem}
\emph{Generalized limit} is defined by the following formula:
\begin{defn}
\index{limit!generalized}$\xlim f\eqdef\setcond{\nu\circ f\circ\uparrow r}{r\in G}$
for any funcoid $f$.\end{defn}
\begin{rem}
Generalized limit technically is a set of funcoids.
\end{rem}
We will assume that $\dom f\sqsupseteq\rsupfun{\mu}@\{x\}$.
\begin{defn}
$\xlim_{x}f=\xlim f|_{\rsupfun{\mu}@\{x\}}$.\end{defn}
\begin{obvious}
$\xlim_{x}f=\setcond{\nu\circ f|_{\rsupfun{\mu}@\{x\}}\circ\uparrow r}{r\in G}$.\end{obvious}
\begin{rem}
$\xlim_{x}f$ is the same for funcoids $\mu$ and $\Compl\mu$.
\end{rem}
The function $\tau$ will define an injection from the set of points
of the space $\nu$ (``numbers'', ``points'', or ``vectors'')
to the set of all (generalized) limits (i.e. values which $\xlim_{x}f$
may take).
\begin{defn}
$\tau(y)\eqdef\setcond{\rsupfun{\mu}@\{x\}\times^{\mathsf{FCD}}\rsupfun{\nu}@\{y\}}{x\in D}$.\end{defn}
\begin{prop}
$\tau(y)=\setcond{(\rsupfun{\mu}@\{x\}\times^{\mathsf{FCD}}\rsupfun{\nu}@\{y\})\circ\uparrow r}{r\in G}$
for every (fixed) $x\in D$.\end{prop}
\begin{proof}
~
\begin{align*}
(\rsupfun{\mu}@\{x\}\times^{\mathsf{FCD}}\rsupfun{\nu}@\{y\})\circ\uparrow r & =\\
\supfun{\uparrow r^{-1}}\rsupfun{\mu}@\{x\}\times^{\mathsf{FCD}}\rsupfun{\nu}@\{y\} & =\\
\supfun{\mu}\rsupfun{\uparrow r^{-1}}@\{x\}\times^{\mathsf{FCD}}\rsupfun{\nu}@\{y\} & =\\
\rsupfun{\mu}@\{r^{-1}x\}\times^{\mathsf{FCD}}\rsupfun{\nu}@\{y\} & \in \setcond{\rsupfun{\mu}@\{x\}\times^{\mathsf{FCD}}\rsupfun{\nu}@\{y\}}{x\in D}.
\end{align*}


Reversely $\rsupfun{\mu}@\{x\}\times^{\mathsf{FCD}}\rsupfun{\nu}@\{y\}=(\rsupfun{\mu}@\{x\}\times^{\mathsf{FCD}}\rsupfun{\nu}@\{y\})\circ\uparrow e$
where $e$ is the identify element of $G$.\end{proof}
\begin{prop}
$\tau(y)=\xlim(\rsupfun{\mu}@\{x\}\times^{\mathsf{FCD}}\uparrow^{\Base(\Ob\nu)}\{y\})$
(for every $x$). Informally: Every $\tau(y)$ is a generalized limit
of a constant funcoid.\end{prop}
\begin{proof}
~
\begin{align*}
\xlim(\rsupfun{\mu}@\{x\}\times^{\mathsf{FCD}}\uparrow^{\Base(\Ob\nu)}\{y\}) & =\\
\setcond{\nu\circ(\rsupfun{\mu}@\{x\}\times^{\mathsf{FCD}}\uparrow^{\Base(\Ob\nu)}\{y\})\circ\uparrow r}{r\in G} & =\\
\setcond{(\rsupfun{\mu}@\{x\}\times^{\mathsf{FCD}}\rsupfun{\nu}@\{y\})\circ\uparrow r}{r\in G} & =\tau(y).
\end{align*}
\end{proof}
\begin{thm}
If $f|_{\rsupfun{\mu}@\{x\}}\in\continuous(\mu;\nu)$
and $\rsupfun{\mu}@\{x\}\sqsupseteq\uparrow^{\Ob\mu}\{x\}$
then $\xlim_{x}f=\tau(fx)$.\end{thm}
\begin{proof}
$f|_{\rsupfun{\mu}@\{x\}}\circ\mu\sqsubseteq\nu\circ f|_{\rsupfun{\mu}@\{x\}}\sqsubseteq\nu\circ f$;
thus $\langle f\rangle\rsupfun{\mu}@\{x\}\sqsubseteq\supfun{\nu}\rsupfun{f}@\{x\}$;
consequently we have
\begin{gather*}
\nu\sqsupseteq\supfun{\nu}\rsupfun{f}@\{x\}\times^{\mathsf{FCD}}\supfun{\nu}\rsupfun{f}@\{x\}\sqsupseteq\supfun f\rsupfun{\mu}@\{x\}\times^{\mathsf{FCD}}\supfun{\nu}\rsupfun{f}@\{x\}.\\
\begin{aligned}\nu\circ f|_{\rsupfun{\mu}@\{x\}} & \sqsupseteq\\
(\supfun f\rsupfun{\mu}@\{x\}\times^{\mathsf{FCD}}\supfun{\nu}\rsupfun{f}@\{x\})\circ f|_{\rsupfun{\mu}@\{x\}} & =\\
(f|_{\rsupfun{\mu}\{x\}})^{-1}\supfun f\rsupfun{\mu}@\{x\}\times^{\mathsf{FCD}}\supfun{\nu}\rsupfun{f}@\{x\} & \sqsupseteq\\
\supfun{\id_{\dom f|_{\rsupfun{\mu}\{x\}}}^{\mathsf{FCD}}}\rsupfun{\mu}@\{x\}\times^{\mathsf{FCD}}\supfun{\nu}\rsupfun{f}@\{x\} & \sqsupseteq\\
\dom f|_{\rsupfun{\mu}@\{x\}}\times^{\mathsf{FCD}}\supfun{\nu}\rsupfun{f}@\{x\} & =\\
\rsupfun{\mu}@\{x\}\times^{\mathsf{FCD}}\supfun{\nu}\rsupfun{f}@\{x\}.
\end{aligned}
\end{gather*}


$\im(\nu\circ f|_{\rsupfun{\mu}@\{x\}})=\supfun{\nu}\rsupfun{f}@\{x\}$;
\begin{align*}
\nu\circ f|_{\rsupfun{\mu}@\{x\}} & \sqsubseteq\\
\rsupfun{\mu}@\{x\}\times^{\mathsf{FCD}}\im(\nu\circ f|_{\rsupfun{\mu}@\{x\}}) & =\\
\rsupfun{\mu}@\{x\}\times^{\mathsf{FCD}}\supfun{\nu}\rsupfun{f}@\{x\}.
\end{align*}
So $\nu\circ f|_{\rsupfun{\mu}@\{x\}}=\rsupfun{\mu}@\{x\}\times^{\mathsf{FCD}}\supfun{\nu}\rsupfun{f}@\{x\}$.

Thus $\xlim_{x}f=\setcond{(\rsupfun{\mu}@\{x\}\times^{\mathsf{FCD}}\supfun{\nu}\rsupfun{f}@\{x\})\circ\uparrow r}{r\in G}=\tau(fx)$.\end{proof}
\begin{rem}
Without the requirement of $\rsupfun{\mu}@\{x\}\sqsupseteq\uparrow^{\Ob\mu}\{x\}$
the last theorem would not work in the case of removable singularity.\end{rem}
\begin{thm}
Let $\nu\sqsubseteq\nu\circ\nu$. If $f|_{\rsupfun{\mu}@\{x\}}\overset{\nu}{\rightarrow}\uparrow^{\Ob\mu}\{y\}$
then $\xlim_{x}f=\tau(y)$.\end{thm}
\begin{proof}
$\im f|_{\rsupfun{\mu}@\{x\}}\sqsubseteq\rsupfun{\nu}@\{y\}$;
$\supfun f\rsupfun{\mu}@\{x\}\sqsubseteq\rsupfun{\nu}@\{y\}$;
\begin{align*}
\nu\circ f|_{\rsupfun{\mu}@\{x\}} & \sqsupseteq\\
(\rsupfun{\nu}@\{y\}\times^{\mathsf{FCD}}\rsupfun{\nu}@\{y\})\circ f|_{\rsupfun{\mu}@\{x\}} & =\\
\supfun{(f|_{\rsupfun{\mu}@\{x\}})^{-1}}\rsupfun{\nu}@\{y\}\times^{\mathsf{FCD}}\rsupfun{\nu}@\{y\} & =\\
\supfun{\id_{\rsupfun{\mu}\{x\}}^{\mathsf{FCD}}\circ f^{-1}}\rsupfun{\nu}@\{y\}\times^{\mathsf{FCD}}\rsupfun{\nu}@\{y\} & \sqsupseteq\\
\supfun{\id_{\rsupfun{\mu}\{x\}}^{\mathsf{FCD}}\circ f^{-1}}\supfun f\rsupfun{\mu}@\{x\}\times^{\mathsf{FCD}}\rsupfun{\nu}@\{y\} & =\\
\supfun{\id_{\rsupfun{\mu}\{x\}}^{\mathsf{FCD}}}\supfun{f^{-1}\circ f}\rsupfun{\mu}@\{x\}\times^{\mathsf{FCD}}\rsupfun{\nu}@\{y\} & \sqsupseteq\\
\supfun{\id_{\rsupfun{\mu}\{x\}}^{\mathsf{FCD}}}\supfun{\id_{\rsupfun{\mu}\{x\}}^{\mathsf{FCD}}}\rsupfun{\mu}@\{x\}\times^{\mathsf{FCD}}\rsupfun{\nu}@\{y\} & =\\
\rsupfun{\mu}@\{x\}\times^{\mathsf{FCD}}\rsupfun{\nu}@\{y\}.
\end{align*}


On the other hand, $f|_{\rsupfun{\mu}@\{x\}}\sqsubseteq\rsupfun{\mu}@\{x\}\times^{\mathsf{FCD}}\rsupfun{\nu}@\{y\}$;

$\nu\circ f|_{\rsupfun{\mu}@\{x\}}\sqsubseteq\rsupfun{\mu}@\{x\}\times^{\mathsf{FCD}}\supfun{\nu}\rsupfun{\nu}@\{y\}\sqsubseteq\rsupfun{\mu}@\{x\}\times^{\mathsf{FCD}}\rsupfun{\nu}@\{y\}$.

So $\nu\circ f|_{\rsupfun{\mu}@\{x\}}=\rsupfun{\mu}@\{x\}\times^{\mathsf{FCD}}\rsupfun{\nu}@\{y\}$.

$\xlim_{x}f=\setcond{\nu\circ f|_{\rsupfun{\mu}@\{x\}}\circ\uparrow r}{r\in G}=\setcond{(\rsupfun{\mu}@\{x\}\times^{\mathsf{FCD}}\rsupfun{\nu}@\{y\})\circ\uparrow r}{r\in G}=\tau(y)$.\end{proof}
\begin{cor}
If $\lim_{\rsupfun{\mu}@\{x\}}^{\nu}f=y$ then $\xlim_{x}f=\tau(y)$.
\end{cor}
We have injective $\tau$ if $\rsupfun{\nu}@\{y_{1}\}\sqcap\rsupfun{\nu}@\{y_{2}\}=\bot^{\mathscr{F}(\Ob\mu)}$
for every distinct $y_{1},y_{2}\in\Ob\nu$ that is if $\nu$ is $T_{2}$-separable.
