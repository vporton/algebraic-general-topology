\documentclass[english,reqno,12pt]{amsbook}
\usepackage[paperwidth=5.06in, paperheight=7.81in]{geometry}
\usepackage[T1]{fontenc}
\synctex=-1
% \usepackage{xcolor}
% \usepackage{hyperref}
\usepackage{babel}
\usepackage{textcomp}
\usepackage{mathrsfs}
\usepackage{url}
\usepackage{amstext}
\usepackage{amsmath}
\usepackage{amsthm}
\usepackage{amssymb}
\usepackage{stmaryrd}
\usepackage{agt}
\makeindex
% \usepackage[all]{xy}
% \usepackage[unicode=true,
%  bookmarks=true,bookmarksnumbered=true,bookmarksopen=false,
%  breaklinks=false,pdfborder={0 0 0},backref=false,colorlinks=true]
%  {hyperref}
\hypersetup{pdftitle={Algebraic General Topology. Book 3: Algebra. Edition 2},
 pdfauthor={Victor Porton},
 pdfsubject={general topology, algebra},
 pdfkeywords={algebraic general topology,quasi-uniform spaces,generalizations of proximity spaces,generalizations of nearness spaces,generalizations of uniform spaces,ordered semigroups,ordered monoids,abstract algebra,universal algebra}}

\usepackage{xr,refcount}
\externaldocument[book-]{volume-1}
\newcommand{\bookref}[1]{\ref*{book-#1}}

% Continue numbering of book.pdf
\setcounter{thm}{\getrefnumber{book-finalthm}}
\addtocounter{thm}{-1}

\begin{document}

\title{Algebraic General Topology.\\Book 3: Algebra.\\Edition 2}

\author{Victor Porton}

\email{\href{mailto:porton@narod.ru}{porton@narod.ru}}

\urladdr{\href{http://www.mathematics21.org}{http://www.mathematics21.org}}

\date{\today}

\begin{abstract}
I define \emph{space} as an element of an ordered semigroup action, that is a semigroup action conforming to a partial order. Topological spaces, uniform spaces, proximity spaces, directed graphs, metric spaces, etc.\ all are spaces. It can be further generalized to ordered precategory actions (that I call \emph{interspaces}). I build basic general topology (continuity, limit, openness, closedness, hausdorffness, etc.) in an arbitrary space. Now general topology is an algebraic theory.

Was a spell laid onto Earth mathematicians not to find the most important structure in general topology until 2019?
\end{abstract}


\keywords{algebraic general topology, quasi-uniform spaces, generalizations of proximity spaces, generalizations of nearness spaces, generalizations
of uniform spaces, ordered semigroups, ordered monoids, abstract algebra, universal algebra}


\subjclass[2000]{06F05, 06F99, 08A99, 20M30, 20M99, 54J05, 54A05, 54D99, 54E05, 54E15, 54E17, 54E99}

\maketitle

\tableofcontents{}

This is a draft.

It is a continuation of \cite{volume-1}.

You can read this text without any knowledge of algebraic general topology~(\cite{volume-1}). But to have some examples of how to apply this theory, you need to know what funcoids and reloids are and how funcoids are related with topological spaces.

\chapter{Ordered Semigroups}

I will show that most of the topology can be formulated in an ordered semigroup or an ordered monoid.

For simplicity of notation I speak about semigroups, but all this can be easily generalized to precategories.

\begin{defn}
\emph{Ordered semigroup} (or \emph{semigroup}) is a set together with binary operation~$\circ$ and binary relation~$\leq$ on it, conforming both to semigroup axioms and partial order axioms and:
\[ x_0\leq x_1\land y_0\leq y_1\Rightarrow y_0\circ x_0\leq y_1\circ x_1. \]
\end{defn}

In this book I will call elements of an ordered semigroup \emph{spaces}, because they generalize such things as topological spaces, proximity spaces, uniform spaces, (directed) graphs.

Note that we consider functions as spaces, too.

\begin{defn}
\emph{Ordered monoid} (or \emph{pomonoid}) is an ordered semigroup that is a monoid.
\end{defn}

\begin{defn}
\emph{Curried ordered semigroup action} on a poset~$\mathfrak{A}$ for an ordered semigroup~$S$ is a ?? function $f$ such that
\begin{enumerate}
\item $(f(b\circ a))x = (Tb)(Ta)x$ for all $a,b\in S$, $x\in\mathfrak{A}$;
\item $x\leq y\Rightarrow(Ta)x\leq (Ta)y$ for all $a\in S$, $x,y\in\mathfrak{A}$;
\item $a\leq b\Rightarrow(Ta)x\leq (Tb)x$ for all $a,b\in S$, $x\in\mathfrak{A}$.
\end{enumerate}
\end{defn}

\begin{rem}
Google search for ``"ordered semigroup action"'' showed nothing. Was a spell laid onto Earth mathematicians not to find the most important structure in general topology?
\end{rem}

Each curried ordered semigroup action induces \emph{the curried ordered semigroup action of space actions}, whose elements are the same a of the original one, spaces are the curried actions of actions of the original semigroup, semigroup operation is function composition, and order of spaces is the product order. PROOF.

\emph{Funcoids} described in the first volume of this book form an ordered semigroup with curried action $\supfun{}$. \emph{Reloids} form an ordered semigroup with curried action $a\mapsto\supfun{\torldin a}$.
As we know from the first volume, funcoids are a generalization of topological spaces, proximity spaces, and directed graphs (``discrete spaces''), reloids is a generalization of uniform spaces and directed graphs. Funcoid is determined by its action. So most of the customary general topology can be described in terms of ordered semigroup actions!

This book is mainly about this topic: describing general topology in terms of ordered semigroup actions. Above are the new axioms for general topology. No topological spaces here.

We will need also \emph{semigroups with involution}.

\begin{defn}
\emph{Semigroup with involution} is a semigroup together with the operation $a\mapsto a^{\dagger}$ (\emph{involution}) such that:
\begin{enumerate}
\item $a^{\dagger\dagger} = a$;
\item $(b\circ a)^{\dagger} = a^{\dagger}\circ b^{\dagger}$.
\end{enumerate}
\end{defn}

For an \emph{ordered semigroup with involution} we will additionally require $a\leq b\Rightarrow a^{\dagger}\leq b^{\dagger}$ (and consequently $a\leq b\Leftrightarrow a^{\dagger}\leq b^{\dagger}$).

Now we have a formalism to describe many topological properties (following the ideas of first volume):

Continuity is described by the formulas $f\circ a\leq a\circ f$, $f\circ a\circ f^{\dagger}\leq a$, $a\leq f^{\dagger}\circ a\circ f$.

Limit is described?? by the formula $(Tf)x\leq(Tg)y$.

Generalized limit of an arbitrary (discontionuous) space is described by the formula $\xlim f=\setcond{\nu\circ f\circ r}{r\in G}$.

\emph{Neighborhood} of element~$x$ is such a $y$ that $(Ta)x\leq y$. \emph{Interior} of~$x$ (if it exists) if the join of all $y$ such that $x$ is a neighborhood of~$x$.

An element~$x$ is closed regarding~$a$ iff $(Ta)x\leq x$. $x$~is open iff $x$ is closed regarding $Ta^{\dagger}$.

To define compactness we additionally need the structure of filtrator $(\mathfrak{A},\mathfrak{Z})$ on our poset. Then it is space~$a$ is \emph{directly compact} iff
\[\forall x\in\mathfrak{A}:(x\text{ is non-least}\Rightarrow\Cor(Ta)x\text{ is non-least}); \]
$a$~is \emph{reversely compact} iff $a^{\dagger}$ is directly compact; $a$~is \emph{compact} iff it is both directly and reversely compact.

It seem we cannot define \emph{total boundness} purely in terms of ordered semigroups, because it is a property of reloids and reloid is not determined by its action.

Every ordered semigroup action~$T$ defines a relation~$R$: $x(Ra)y\Leftrightarrow y\nasymp(Ta)x$.

\begin{thm}
$(Rb)\circ(Ra) = R(b\circ a)$. \fxerror{Wrong.}
\end{thm}

If $Ra^{\dagger}=(Ra)^{-1}$ for every~$a$, we call the action~$T$ on an involutive semigroup \emph{inter\-sec\-tion-sym\-met\-ric}. In this case our action defines a pointfree funcoid.

A space is connected iff $x\equiv y\Rightarrow x(Ra)y$. Define ``other'' connectedness also through series.

Further axioms:

$(Tf)(x\sqcup y)=(Tf)x\sqcup(Tf)y$

$(T(f\sqcup g))x=(Tf)x\sqcup(Tg)x$

or in more general form for arbitrary posets:

??

Product of actions is ??.

We can define open and closed functions.

\begin{proof}
??
\end{proof}

\emph{Restricted identity transformation} $\id_p$ is the (generally, partially defined) transformation $x\mapsto x\sqcap p$.

\begin{obvious}
$\id_q\circ\id_p = \id_{p\sqcap q}$.
\end{obvious}

\begin{prop}
$p\ne q\Rightarrow\id_p\ne\id_q$.
\end{prop}

\begin{proof}
$\id_p p=p \ne q = \id_q q$.
\end{proof}

\emph{Ordered semigroup action with identities} is an ordered semigroup~$S$ action~$T$ together with a function $p\mapsto\id_p\in S$ such that $T\id_p=\id_p$. (I abuse the notation $\id_p$ for both spaces and for transformations; this won't lead to inconsistencies, because as proved above this mapping is injective on restricted identities.)

\begin{obvious}
For every ordered semigroup action with identities, the identity transformations are entirely defined.
\end{obvious}

From injectivity it follows $\id_{p\sqcap q} = \id_p\circ\id_q$.

\emph{Restriction} of a space~$a$ to element~$x$ is $a|_x=a\circ\id_x$.

\emph{Binary product} in an ordered semigroup action having a greatest element~$\top$ is defined as $p\times q=\id_q\circ\top\circ\id_p$.

\begin{thm}
\[ x(R(p\times q))y\Leftrightarrow x\nasymp p\land y\nasymp q. \]
\end{thm}

\begin{proof}
??
\end{proof}

REMOVE: Ordered semigroup action with a product is such an ordered semigroup action~$T$ together with a binary operation~$\times$ on elements that 

\begin{thm}
$p_0\times q_0\nasymp p_1\times q_1 \Leftrightarrow p_0\nasymp p_1\land q_0\nasymp q_1$.
\end{thm}

We can define space~$a$ \emph{square restricted} to an element~$p$ as $a\sqcap(p\times p)$.

\emph{Hausdorff} space~$a$ can be defined as having $x\overline{Ra}y$ whenever~$x$ and $y$ are closed and $x\asymp y$.

Pointfree funcoids (and consequently funcoids) are an ordered semigroup action. Reloids are also an ordered semigroup action.

Ordered semigroup action is \emph{ordered by elements} when \[ a\leq b \Leftarrow Ta\leq Tb \] that is when \[ a\leq b \Leftarrow \forall x:(Ta)x\leq(Tb)x. \]

A space~$a$ is \emph{complete} when $(Ta)\bigsqcup S=\bigsqcup\rsupfun{Ta}S$ whenever both $\bigsqcup S$ and $\bigsqcup\rsupfun{Ta}S$ are defined.

\begin{defn}
\emph{Completion} of a space is its core part on the filtrator of spaces and complete spaces.
\end{defn}

\begin{note}
Apparently, not every space has a completion.
\end{note}

\begin{defn}
\emph{Kuratowski space} is a complete idempotent ($a\circ a=a$) space.
\end{defn}

Kuratowski spaces are a generalization of topological spaces.

Metric spaces are also spaces (PROOF!): Define the action for a metric space as the action of its induced proximity and composition of metrics $\rho$, $\sigma$ by the formula: \[ (\sigma\circ\rho)(x,z) = \inf_{y\in\mho}(\rho(x,y)+\sigma(y,z)). \]

By the way, what is common in uniform spaces and metric spaces? Both can be considered as a monovalued reloid (for uniform spaces taking boolean values, for metric spaces entirely defined and taking real values). ??

What is then a ball on a uniform space?

Isn't proximity also a XXX-metric, with points being sets or filters?


% \printindex{}

\bibliographystyle{plain}
\bibliography{refs}

\end{document}
