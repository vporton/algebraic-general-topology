
\chapter{\label{chap-filt}Filters and filtrators}

This chapter is based on my article \cite{filters}.

This chapter is grouped in the following way:
\begin{itemize}
\item First it goes a short introduction in pedagogical order (first less
general stuff and examples, last the most general stuff):

\begin{itemize}
\item filters on a set;
\item filters on a meet-semilattice;
\item filters on a poset.
\end{itemize}
\item Then it goes the formal part.
\end{itemize}

\section{Implication tuples}
\begin{defn}
An \emph{implications tuple} is a tuple $(P_{1},\ldots,P_{n})$ such
that $P_{1}\Rightarrow\ldots\Rightarrow P_{n}$.\end{defn}
\begin{obvious}
$(P_{1},\ldots,P_{n})$ is an implications tuple iff $P_{i}\Rightarrow P_{j}$
for every $i<j$ (where $i,j\in\{1,\ldots,n\}$).
\end{obvious}
The following is an example of a theorem using an implication tuple:
\begin{example}
The following is an implications tuple:
\begin{enumerate}
\item $A$.
\item $B$.
\item $C$.
\end{enumerate}
\end{example}
This example means just that $A\Rightarrow B\Rightarrow C$.

I prefer here a verbal description instead of symbolic implications
$A\Rightarrow B\Rightarrow C$, because $A$, $B$, $C$ may be long
English phrases and they may not fit into the formula layout.

The main (intuitive) idea of the theorem is expressed by the implication
$P_{1}\Rightarrow P_{n}$, the rest implications ($P_{2}\Rightarrow P_{n}$,
$P_{3}\Rightarrow P_{n}$, ...) are purely technical, as they express
generalizations of the main idea.

For uniformity theorems in the section about filters and filtrators
start with the same $P_{1}$: ``$(\mathfrak{A};\mathfrak{Z})$ is
a powerset filtrator.'' (defined below) That means that the main
idea of the theorem is about powerset filtrators, the rest implications
(like $P_{2}\Rightarrow P_{n}$, $P_{3}\Rightarrow P_{n}$, ...) are
just technical generalizations.


\section{Introduction to filters and filtrators}


\subsection{Filters on a set}

We sometimes want to define something resembling an infinitely small
(or infinitely big) set, for example the infinitely small interval
near $0$ on the real line. Of course there is no such set, just like
as there is no natural number which is the difference $2-3$. To overcome
this shortcoming we introduce whole numbers, and $2-3$ becomes well
defined. In the same way to consider things which are like infinitely
small (or infinitely big) sets we introduce \emph{filters}.

An example of a filter is the infinitely small interval near $0$
on the real line. To come to infinitely small, we consider all intervals
$(-\epsilon;\epsilon)$ for all $\epsilon>0$. This filter consists
of all intervals $(-\epsilon;\epsilon)$ for all $\epsilon>0$ and
also all subsets of $\mathbb{R}$ containing such intervals as subsets.
Informally speaking, this is the greatest filter contained in every
interval $(-\epsilon;\epsilon)$ for all $\epsilon>0$.
\begin{defn}
\index{filter!on a set}A filter on a set $\mho$ is a $\mathcal{F}\in\subsets\subsets\mho$
such that:
\begin{enumerate}
\item $\forall A,B\in\mathcal{F}:A\cap B\in\mathcal{F}$;
\item $\forall A,B\in\subsets\mho:(A\in\mathcal{F}\land B\supseteq A\Rightarrow B\in\mathcal{F})$.
\end{enumerate}
\end{defn}
\begin{xca}
Verify that the above introduced infinitely small interval near $0$
on the real line is a filter on $\mathbb{R}$.
\end{xca}

\begin{xca}
Describe ``the neighborhood of positive infinity'' filter on $\mathbb{R}$.\end{xca}
\begin{defn}
\index{filter!proper}A filter not containing empty set is called
a \emph{proper filter}.\end{defn}
\begin{obvious}
The non-proper filter is $\subsets\mho$.\end{obvious}
\begin{rem}
Some other authors require that all filters are proper. This is a
stupid idea and we allow non-proper filters, in the same way as we
allow to use the number~$0$.
\end{rem}

\subsection{Intro to filters on a meet-semilattice}

A trivial generalization of the above:
\begin{defn}
\index{filter!on a meet-semilattice}A filter on a meet-semilattice
$\mathfrak{Z}$ is a $\mathcal{F}\in\subsets\mathfrak{Z}$ such that:
\begin{enumerate}
\item $\forall A,B\in\mathcal{F}:A\sqcap B\in\mathcal{F}$;
\item $\forall A,B\in\mathfrak{Z}:(A\in\mathcal{F}\land B\sqsupseteq A\Rightarrow B\in\mathcal{F})$.
\end{enumerate}
\end{defn}

\subsection{Intro to filters on a poset}
\begin{defn}
\index{filter!on a poset}A filter on a poset~$\mathfrak{Z}$ is
a $\mathcal{F}\in\subsets\mathfrak{Z}$ such that:
\begin{enumerate}
\item $\forall A,B\in\mathcal{F}\exists C\in\mathcal{F}:C\sqsubseteq A,B$;
\item $\forall A,B\in\mathfrak{Z}:(A\in\mathcal{F}\land B\sqsupseteq A\Rightarrow B\in\mathcal{F})$.
\end{enumerate}
\end{defn}
It is easy to show (and there is a proof of it somewhere below) that
this coincides with the above definition in the case if $\mathfrak{Z}$
is a meet-semilattice.


\section{Filters on a poset}


\subsection{Filters on posets}

Let $\mathfrak{Z}$ be a poset.
\begin{defn}
\index{filter base}\emph{Filter base} is a nonempty subset $F$ of
$\mathfrak{Z}$ such that
\[
\forall X,Y\in F\exists Z\in F:(Z\sqsubseteq X\land Z\sqsubseteq Y).
\]
\end{defn}
\begin{obvious}
A nonempty chain is a filter base.\end{obvious}
\begin{defn}
\index{filter!on poset}\emph{Filter} is a subset of $\mathfrak{Z}$
which is both a filter base and an upper set.
\end{defn}
I will denote the set of filters (for a given or implied poset $\mathfrak{Z}$)
as $\mathfrak{F}$ and call $\mathfrak{F}$ the set of filters over
the poset $\mathfrak{Z}$.
\begin{prop}
If $\top$ is the maximal element of $\mathfrak{Z}$ then $\top\in F$
for every filter~$F$.\end{prop}
\begin{proof}
If $\top\notin F$ then $\forall K\in\mathfrak{Z}:K\notin F$ and
so $F$ is empty what is impossible.\end{proof}
\begin{prop}
Let $S$ be a filter base on a poset. If $A_{0},\ldots,A_{n}\in S$
($n\in\mathbb{N}$), then 
\[
\exists C\in S:(C\sqsubseteq A_{0}\land\ldots\land C\sqsubseteq A_{n}).
\]
\end{prop}
\begin{proof}
It can be easily proved by induction.\end{proof}
\begin{defn}
A function~$f$ from a poset~$\mathfrak{A}$ to a poset~$\mathfrak{B}$
\emph{preserves filtered meets} iff whenever $\bigsqcap S$ is defined
for a filter base~$S$ on~$\mathfrak{A}$ we have $f\bigsqcap S=\bigsqcap\rsupfun fS$.
\end{defn}

\subsection{Filters on meet-semilattices}
\begin{thm}
\label{filt-eq-char}\index{filter!on meet-semilattice}If $\mathfrak{Z}$
is a meet-semilattice and $F$ is a nonempty subset of $\mathfrak{Z}$
then the following conditions are equivalent:
\begin{enumerate}
\item \label{fmslat-filt}$F$ is a filter.
\item \label{fmslat-two}$\forall X,Y\in F:X\sqcap Y\in F$ and $F$ is
an upper set.
\item \label{fmslat-one}$\forall X,Y\in\mathfrak{Z}:(X,Y\in F\Leftrightarrow X\sqcap Y\in F)$.
\end{enumerate}
\end{thm}
\begin{proof}
~
\begin{description}
\item [{\ref{fmslat-filt}$\Rightarrow$\ref{fmslat-two}}] Let $F$ be
a filter. Then $F$ is an upper set. If $X,Y\in F$ then $Z\sqsubseteq X\land Z\sqsubseteq Y$
for some $Z\in F$. Because $F$ is an upper set and $Z\sqsubseteq X\sqcap Y$
then $X\sqcap Y\in F$.
\item [{\ref{fmslat-two}$\Rightarrow$\ref{fmslat-filt}}] Let $\forall X,Y\in F:X\sqcap Y\in F$
and $F$ be an upper set. We need to prove that $F$ is a filter base.
But it is obvious taking $Z=X\sqcap Y$ (we have also taken into account
that $F\ne\emptyset$).
\item [{\ref{fmslat-two}$\Rightarrow$\ref{fmslat-one}}] Let $\forall X,Y\in F:X\sqcap Y\in F$
and $F$ be an upper set. Then
\[
\forall X,Y\in\mathfrak{Z}:(X,Y\in F\Rightarrow X\sqcap Y\in F).
\]



Let $X\sqcap Y\in F$; then $X,Y\in F$ because $F$ is an upper set.

\item [{\ref{fmslat-one}$\Rightarrow$\ref{fmslat-two}}] Let
\[
\forall X,Y\in\mathfrak{Z}:(X,Y\in F\Leftrightarrow X\sqcap Y\in F).
\]



Then $\forall X,Y\in F:X\sqcap Y\in F$. Let $X\in F$ and $X\sqsubseteq Y\in\mathfrak{Z}$.
Then $X\sqcap Y=X\in F$. Consequently $X,Y\in F$. So $F$ is an
upper set.

\end{description}
\end{proof}
\begin{prop}
Let $S$ be a filter base on a meet-semilattice. If $A_{0},\ldots,A_{n}\in S$
($n\in\mathbb{N}$), then
\[
\exists C\in S:C\sqsubseteq A_{0}\sqcap\dots\sqcap A_{n}.
\]
\end{prop}
\begin{proof}
It can be easily proved by induction.\end{proof}
\begin{prop}
If $\mathfrak{Z}$ is a meet-semilattice and $S$ is a filter base
on it, $A\in\mathfrak{Z}$, then $\rsupfun{A\sqcap}S$ is also a filter
base.\end{prop}
\begin{proof}
$\rsupfun{A\sqcap}S\ne\emptyset$ because $S\ne\emptyset$.

Let $X,Y\in\rsupfun{A\sqcap}S$. Then $X=A\sqcap X'$ and $Y=A\sqcap Y'$
where $X',Y'\in S$. There exists $Z'\in S$ such that $Z'\sqsubseteq X'\sqcap Y'$
. So $X\sqcap Y=A\sqcap X'\sqcap Y'\sqsupseteq A\sqcap Z'\in\rsupfun{A\sqcap}S$.
\end{proof}

\subsection{Order of filters. Principal filters}

I will make the set of filters $\mathfrak{F}$ into a poset by the
order defined by the formula: $a\sqsubseteq b\Leftrightarrow a\supseteq b$.
\begin{defn}
\index{filter!principal}The principal filter corresponding to an
element $a\in\mathfrak{Z}$ is
\[
\uparrow a=\setcond{x\in\mathfrak{Z}}{x\sqsupseteq a}.
\]


Elements of $\mathfrak{P}=\rsupfun{\uparrow}\mathfrak{Z}$ are called
\emph{principal filters}.\end{defn}
\begin{obvious}
Principal filters are filters.
\end{obvious}

\begin{obvious}
$\uparrow$ is an order embedding from $\mathfrak{Z}$ to $\mathfrak{F}$.\end{obvious}
\begin{cor}
$\uparrow$ is an order isomorphism between $\mathfrak{Z}$ and $\mathfrak{P}$.
\end{cor}
We will equate principal filters with corresponding elements of the
base poset (in the same way as we equate for example nonnegative whole
numbers and natural numbers).
\begin{prop}
$\uparrow K\sqsupseteq\mathcal{A}\Leftrightarrow K\in\mathcal{A}$.\end{prop}
\begin{proof}
$\uparrow K\sqsupseteq\mathcal{A}\Leftrightarrow\uparrow K\subseteq\mathcal{A}\Leftrightarrow K\in\mathcal{A}$.
\end{proof}

\section{Filters on a Set}

Consider filters on the poset $\mathfrak{Z}=\subsets\mathfrak{U}$
(where $\mathfrak{U}$ is some fixed set) with the order $A\sqsubseteq B\Leftrightarrow A\subseteq B$
(for $A,B\in\subsets\mathfrak{A}$).

In fact, it is a complete atomistic boolean lattice with $\bigsqcap S=\bigcap S$,
$\bigsqcup S=\bigcup S$, $\overline{A}=\mathfrak{U}\setminus A$
for every $S\in\subsets\subsets\mathfrak{U}$ and $A\in\subsets\mathfrak{U}$,
atoms being one-element sets.
\begin{defn}
\index{filter!on set}\index{filter!on powerset}I will call a filter
on the lattice of all subsets of a given set $\mathfrak{U}$ as a
\emph{filter on set}.
\end{defn}

\begin{defn}
I will denote the set on which a filter $\mathcal{F}$ is defined
as $\Base(\mathcal{F})$.\end{defn}
\begin{obvious}
$\Base(\mathcal{F})=\bigcup\mathcal{F}$.\end{obvious}
\begin{prop}
The following are equivalent for a non-empty set $F\in\subsets\subsets\mathfrak{U}$:
\begin{enumerate}
\item $F$ is a filter.
\item $\forall X,Y\in F:X\cap Y\in F$ and $F$ is an upper set.
\item $\forall X,Y\in\subsets\mathfrak{U}:(X,Y\in F\Leftrightarrow X\cap Y\in F)$.
\end{enumerate}
\end{prop}
\begin{proof}
By theorem \ref{filt-eq-char}.\end{proof}
\begin{obvious}
The minimal filter on $\subsets\mathfrak{U}$ is $\subsets\mathfrak{U}$.
\end{obvious}

\begin{obvious}
The maximal filter on $\subsets\mathfrak{U}$ is $\{\mathfrak{U}\}$.
\end{obvious}
I will denote $\uparrow A=\uparrow^{\mathfrak{U}}A=\uparrow^{\subsets\mathfrak{U}}A$.
(The distinction between conflicting notations $\uparrow^{\mathfrak{U}}A$
and $\uparrow^{\subsets\mathfrak{U}}A$ will be clear from the context.)
\begin{prop}
Every filter on a finite set is principal.\end{prop}
\begin{proof}
Let $\mathcal{F}$ be a filter on a finite set. Then obviously $\mathcal{F}=\bigsqcap^{\mathfrak{Z}}\up\mathcal{F}$
and thus $\mathcal{F}$ is principal.
\end{proof}

\section{Filtrators}

$(\mathfrak{F};\mathfrak{P})$ is a poset and its subset (with induced
order on the subset). I call pairs of a poset and its subset like
this \emph{filtrators}.
\begin{defn}
\index{filtrator}I will call a \emph{filtrator} a pair $(\mathfrak{A};\mathfrak{Z})$
of a poset $\mathfrak{A}$ and its subset $\mathfrak{Z}\subseteq\mathfrak{A}$.
I call $\mathfrak{A}$ the \emph{base} of the filtrator and $\mathfrak{Z}$
the \emph{core} of the filtrator. I will also say that $(\mathfrak{A};\mathfrak{Z})$
is a filtrator \emph{over} poset $\mathfrak{Z}$.

I will denote $\base(\mathfrak{A};\mathfrak{Z})=\mathfrak{A}$, $\core(\mathfrak{A};\mathfrak{Z})=\mathfrak{Z}$
for a filtrator $(\mathfrak{A};\mathfrak{Z})$.
\end{defn}
When speaking about filters, we will imply that we consider the filtrator
$(\mathfrak{F};\mathfrak{P})$ or what is the same (as we equate principal
filters with base elements) the filtrator $(\mathfrak{F};\mathfrak{Z})$.


\begin{defn}
\index{filtrator!lattice}I will call a \emph{lattice filtrator} a
pair $(\mathfrak{A};\mathfrak{Z})$ of a lattice $\mathfrak{A}$ and
its subset $\mathfrak{Z}\subseteq\mathfrak{A}$.
\end{defn}

\begin{defn}
\index{filtrator!complete lattice}I will call a \emph{complete lattice
filtrator} a pair $(\mathfrak{A};\mathfrak{Z})$ of a complete lattice
$\mathfrak{A}$ and its subset $\mathfrak{Z}\subseteq\mathfrak{A}$.
\end{defn}

\begin{defn}
\index{filtrator!central}I will call a \emph{central filtrator} a
filtrator $(\mathfrak{A};Z(\mathfrak{A}))$ where $Z(\mathfrak{A})$
is the center of a bounded lattice $\mathfrak{A}$.
\end{defn}

\begin{defn}
\index{element!of filtrator}I will call \emph{element} of a filtrator
an element of its base.
\end{defn}

\begin{defn}
$\up^{\mathfrak{Z}}a=\up a=\setcond{c\in\mathfrak{Z}}{c\sqsupseteq a}$
for an element $a$ of a filtrator.
\end{defn}

\begin{defn}
$\down^{\mathfrak{Z}}a=\down a=\setcond{c\in\mathfrak{Z}}{c\sqsubseteq a}$
for an element $a$ of a filtrator.\end{defn}
\begin{obvious}
``$\up$'' and ``$\down$'' are dual.
\end{obvious}
Our main purpose here is knowing properties of the core of a filtrator
to infer properties of the base of the filtrator, specifically properties
of $\up a$ for every element~$a$.
\begin{defn}
\index{filtrator!with join-closed core}I call a filtrator \emph{with
join-closed core} such a filtrator $(\mathfrak{A};\mathfrak{Z})$
that $\bigsqcup^{\mathfrak{Z}}S=\bigsqcup^{\mathfrak{A}}S$ whenever
$\bigsqcup^{\mathfrak{Z}}S$ exists for $S\in\subsets\mathfrak{Z}$.
\end{defn}

\begin{defn}
\index{filtrator!with meet-closed core}I call a filtrator \emph{with
meet-closed core} such a filtrator $(\mathfrak{A};\mathfrak{Z})$
that $\bigsqcap^{\mathfrak{Z}}S=\bigsqcap^{\mathfrak{A}}S$ whenever
$\bigsqcap^{\mathfrak{Z}}S$ exists for $S\in\subsets\mathfrak{Z}$.
\end{defn}

\begin{defn}
\index{filtrator!with binarily join-closed core}I call a filtrator
with \emph{binarily join-closed core} such a filtrator $(\mathfrak{A};\mathfrak{Z})$
that $a\sqcup^{\mathfrak{Z}}b=a\sqcup^{\mathfrak{A}}b$ whenever $a\sqcup^{\mathfrak{Z}}b$
exists for $a,b\in\mathfrak{Z}$.
\end{defn}

\begin{defn}
\index{filtrator!with binarily meet-closed core}I call a filtrator
with \emph{binarily meet-closed core} such a filtrator $(\mathfrak{A};\mathfrak{Z})$
that $a\sqcap{}^{\mathfrak{Z}}b=a\sqcap^{\mathfrak{A}}b$ whenever
$a\sqcap^{\mathfrak{Z}}b$ exists for $a,b\in\mathfrak{Z}$.
\end{defn}

\begin{defn}
\index{filtrator!prefiltered}\emph{Prefiltered filtrator} is a filtrator
$(\mathfrak{A};\mathfrak{Z})$ such that ``$\up$'' is injective.
\end{defn}

\begin{defn}
\index{filtrator!filtered}\emph{Filtered filtrator} is a filtrator
$(\mathfrak{A};\mathfrak{Z})$ such that
\[
\forall a,b\in\mathfrak{A}:(\up a\supseteq\up b\Rightarrow a\sqsubseteq b).
\]
\end{defn}
\begin{thm}
A filtrator $(\mathfrak{A};\mathfrak{Z})$ is filtered iff $\forall a\in\mathfrak{A}:a=\bigsqcap^{\mathfrak{A}}\up a$.\end{thm}
\begin{proof}
~
\begin{description}
\item [{$\Leftarrow$}] $\up a\supseteq\up b\Rightarrow\bigsqcap^{\mathfrak{A}}\up a\sqsubseteq\bigsqcap^{\mathfrak{A}}\up b\Rightarrow a\sqsubseteq b$.
\item [{$\Rightarrow$}] $a=\bigsqcap^{\mathfrak{A}}\up a$ is equivalent
to $a$ is a greatest lower bound of $\up a$. That is the implication
that $b$ is lower bound of $\up a$ implies $a\sqsupseteq b$.


$b$ is lower bound of $\up a$ implies $\up b\supseteq\up a$. So
as it is filtered $a\sqsupseteq b$.

\end{description}
\end{proof}
\begin{obvious}
Every filtered filtrator is prefiltered.
\end{obvious}

\begin{obvious}
\label{up-straight}``$\up$'' is a straight map from $\mathfrak{A}$
to the dual of the poset $\subsets\mathfrak{Z}$ if $(\mathfrak{A};\mathfrak{Z})$
is a filtered filtrator.\end{obvious}
\begin{defn}
An \emph{isomorphism} between filtrators $(\mathfrak{A}_{0};\mathfrak{Z}_{0})$
and $(\mathfrak{A}_{1};\mathfrak{Z}_{1})$ is an isomorphism between
posets~$\mathfrak{A}_{0}$ and~$\mathfrak{A}_{1}$ such that it
maps $\mathfrak{Z}_{0}$ into $\mathfrak{Z}_{1}$.\end{defn}
\begin{obvious}
Isomorphism isomorphically maps the order on $\mathfrak{Z}_{0}$ into
order on~$\mathfrak{Z}_{1}$.\end{obvious}
\begin{defn}
Two filtrators are \emph{isomorphic} when there exists an isomorphism
between them.
\end{defn}

\begin{defn}
I will call \emph{primary filtrator} a filtrator isomorphic to the
filtrator consisting of the set of filters on a poset and the set
of principal filters on this poset.\end{defn}
\begin{obvious}
The order on a primary filtrator is defined by the formula $a\sqsubseteq b\Leftrightarrow\up a\supseteq\up b$.\end{obvious}
\begin{defn}
\index{filtrator!powerset}I will call a primary filtrator over a
poset isomorphic to a powerset as \emph{powerset filtrator}.\end{defn}
\begin{obvious}
$\up\mathcal{F}$ is a filter for every element~$\mathcal{F}$ of
a primary filtrator. Reversely, there exists a filter $\mathcal{F}$
if $\up\mathcal{F}$ is a filter.\end{obvious}
\begin{thm}
\label{thm1:prim-exists}For every poset~$\mathfrak{Z}$ there exists
a poset $\mathfrak{A}\supseteq\mathfrak{Z}$ such that $(\mathfrak{A};\mathfrak{Z})$
is a primary filtrator.\end{thm}
\begin{proof}
See appendix~\ref{app:prim-exists}.
\end{proof}

\subsection{Filtrators with Separable Core}
\begin{defn}
\index{filtrator!with separable core}Let $(\mathfrak{A};\mathfrak{Z})$
be a filtrator. It is a \emph{filtrator with separable core} when
\[
\forall x,y\in\mathfrak{A}:(x\asymp^{\mathfrak{A}}y\Rightarrow\exists X\in\up x:X\asymp^{\mathfrak{A}}y).
\]
\end{defn}
\begin{prop}
Let $(\mathfrak{A};\mathfrak{Z})$ be a filtrator. It is a \emph{filtrator
with separable core} iff
\[
\forall x,y\in\mathfrak{A}:(x\asymp^{\mathfrak{A}}y\Rightarrow\exists X\in\up x,Y\in\up y:X\asymp^{\mathfrak{A}}Y).
\]
\end{prop}
\begin{proof}
~
\begin{description}
\item [{$\Rightarrow$}] Apply the definition twice.
\item [{$\Leftarrow$}] Obvious.
\end{description}
\end{proof}
\begin{defn}
\index{filtrator!with co-separable core}Let $(\mathfrak{A};\mathfrak{Z})$
be a filtrator. It is a \emph{filtrator with co-separable core} when
\[
\forall x,y\in\mathfrak{A}:(x\equiv^{\mathfrak{A}}y\Rightarrow\exists X\in\down x:X\equiv^{\mathfrak{A}}y).
\]
\end{defn}
\begin{obvious}
Co-separability is the dual of separability.\end{obvious}
\begin{defn}
\index{filtrator!with co-separable core}Let $(\mathfrak{A};\mathfrak{Z})$
be a filtrator. It is a \emph{filtrator with co-separable core} when
\[
\forall x,y\in\mathfrak{A}:(x\equiv^{\mathfrak{A}}y\Rightarrow\exists X\in\down x,Y\in\down y:X\equiv^{\mathfrak{A}}Y).
\]
\end{defn}
\begin{proof}
By duality.
\end{proof}

\section{Alternative primary filtrators}


\subsection{Lemmas}
\begin{lem}
A set $F$ is a lower set iff $\overline{F}$ is an upper set.\end{lem}
\begin{proof}
$X\in\overline{F}\wedge Z\sqsupseteq X\Rightarrow Z\in\overline{F}$
is equivalent to $Z\in F\Rightarrow X\in F\vee Z\nsqsupseteq X$ is
equivalent $Z\in F\Rightarrow(Z\sqsupseteq X\Rightarrow X\in F)$
is equivalent $Z\in F\wedge X\sqsubseteq Z\Rightarrow X\in F$.\end{proof}
\begin{prop}
Let $\mathfrak{Z}$ be a poset with least element~$\bot$. Then for
upper set~$F$ we have $F\ne\subsets\mathfrak{Z}\Leftrightarrow\bot\notin F$.\end{prop}
\begin{proof}
~
\begin{description}
\item [{$\Rightarrow$}] If $\bot\in F$ then $F=\subsets\mathfrak{Z}$
because $F$ is an upper set.
\item [{$\Leftarrow$}] Obvious.
\end{description}
\end{proof}

\subsection{Informal introduction}

We have already defined filters on a poset. Now we will define three
other sets which are order-isomorphic to the set of filters on a poset:
ideals ($\mathfrak{I}$), free stars ($\mathfrak{S}$), and mixers
($\mathfrak{M}$).

These four kinds of objects are related through commutative diagrams.
First we will paint an informal commutative diagram (it makes no formal
sense because it is not pointed the poset for which the filters are
defined):\[
\begin{tikzcd}
  \mathfrak{F}
	\arrow[d, leftrightarrow, "\lnot"]
    \arrow[r, leftrightarrow, "\rsupfun{\dual}"]
    & \mathfrak{I} \arrow[d, leftrightarrow, "\lnot"] \\
  \mathfrak{M}
	\arrow[r, leftrightarrow, "\rsupfun{\dual}"]
    & \mathfrak{S}
\end{tikzcd}
\]

Then we can define ideals, free stars, and mixers as sets following
certain formulas. You can check that the intuition behind these formulas
follows the above commutative diagram. (That is transforming these
formulas by the course of the above diagram, you get formulas of the
other objects in this list.)

After this, we will paint some formal commutative diagrams similar
to the above diagram but with particular posets at which filters,
ideals, free stars, and mixers are defined.


\subsection{Definitions of ideals, free stars, and mixers}

\emph{Filters} and \emph{ideals} are well known concepts. The terms
\emph{free stars} and \emph{mixers} are my new terminology.

Recall that \emph{filters} are nonempty sets $F$ with $A,B\in F\Leftrightarrow\exists Z\in F:(Z\sqsubseteq A\wedge Z\sqsubseteq B)$
(for every $A,B\in\mathfrak{Z}$).
\begin{defn}
\emph{Ideals} are nonempty sets $F$ with $A,B\in F\Leftrightarrow\exists Z\in F:(Z\sqsupseteq A\wedge Z\sqsupseteq B)$
(for every $A,B\in\mathfrak{Z}$).
\end{defn}

\begin{defn}
\emph{Free stars} are sets $F$ not equal to $\subsets\mathfrak{A}$
with $A,B\in\overline{F}\Leftrightarrow\exists Z\in\overline{F}:(Z\sqsupseteq A\wedge Z\sqsupseteq B)$
(for every $A,B\in\mathfrak{Z}$).
\end{defn}

\begin{defn}
\emph{Mixers} are sets $F$ not equal to $\subsets\mathfrak{A}$ with
$A,B\in\overline{F}\Leftrightarrow\exists Z\in\overline{F}:(Z\sqsubseteq A\wedge Z\sqsubseteq B)$
(for every $A,B\in\mathfrak{Z}$).
\end{defn}
By duality and and an above theorem about filters, we have:
\begin{prop}
~
\begin{itemize}
\item Filters are nonempty upper sets $F$ with $A,B\in F\Rightarrow\exists Z\in F:(Z\sqsubseteq A\wedge Z\sqsubseteq B)$
(for every $A,B\in\mathfrak{Z}$).
\item Ideals are nonempty lower sets $F$ with $A,B\in F\Rightarrow\exists Z\in F:(Z\sqsupseteq A\wedge Z\sqsupseteq B)$
(for every $A,B\in\mathfrak{Z}$).
\item Free stars are upper sets $F$ not equal to $\subsets\mathfrak{A}$
with $A,B\in\overline{F}\Rightarrow\exists Z\in\overline{F}:(Z\sqsupseteq A\wedge Z\sqsupseteq B)$
(for every $A,B\in\mathfrak{Z}$).
\item Mixers are lower sets $F$ not equal to $\subsets\mathfrak{A}$ with
$A,B\in\overline{F}\Rightarrow\exists Z\in\overline{F}:(Z\sqsubseteq A\wedge Z\sqsubseteq B)$
(for every $A,B\in\mathfrak{Z}$).
\end{itemize}
\end{prop}

\begin{prop}
The following are equivalent:
\begin{enumerate}
\item \label{free-alt-star}$F$ is a free star.
\item \label{free-alt-eq}$\forall Z\in\mathfrak{A}:(Z\sqsupseteq A\wedge Z\sqsupseteq B\Rightarrow Z\in F)\Leftrightarrow A\in F\vee B\in F$
for every $A,B\in\mathfrak{A}$ and $F\neq\subsets\mathfrak{A}$.
\item \label{free-alt-impl}$\forall Z\in\mathfrak{A}:(Z\sqsupseteq A\wedge Z\sqsupseteq B\Rightarrow Z\in F)\Rightarrow A\in F\vee B\in F$
for every $A,B\in\mathfrak{A}$ and $F$ is an upper set and $F\neq\subsets\mathfrak{A}$.
\end{enumerate}
\end{prop}
\begin{proof}
~
\begin{description}
\item [{\ref{free-alt-star}$\Rightarrow$\ref{free-alt-eq}}] The following
is a chain of equivalencies: 
\begin{gather*}
\exists Z\in\overline{F}:(Z\sqsupseteq A\wedge Z\sqsupseteq B)\Leftrightarrow A\notin F\wedge B\notin F;\\
\forall Z\in\overline{F}:\neg(Z\sqsupseteq A\wedge Z\sqsupseteq B)\Leftrightarrow A\in F\vee B\in F;\\
\forall Z\in\mathfrak{A}:(Z\notin F\Rightarrow\neg(Z\sqsupseteq A\wedge Z\sqsupseteq B))\Leftrightarrow A\in F\vee B\in F;\\
\forall Z\in\mathfrak{A}:(Z\sqsupseteq A\wedge Z\sqsupseteq B\Rightarrow Z\in F)\Leftrightarrow A\in F\vee B\in F.
\end{gather*}

\item [{\ref{free-alt-eq}$\Rightarrow$\ref{free-alt-impl}}] Obvious.
\item [{\ref{free-alt-impl}$\Rightarrow$\ref{free-alt-eq}}] We need
to prove that $F$ is an upper set. let $A\in F$ and $A\sqsubseteq B\in\mathfrak{A}$.
Then $A\in F\lor B\in F$ and thus $\forall Z\in\mathfrak{A}:(Z\sqsupseteq A\wedge Z\sqsupseteq B\Rightarrow Z\in F)$
that is $\forall Z\in\mathfrak{A}:(Z\sqsupseteq B\Rightarrow Z\in F)$
and so $B\in F$.
\end{description}
\end{proof}
\begin{cor}
The following are equivalent:
\begin{enumerate}
\item $F$ is a mixer.
\item $\forall Z\in\mathfrak{A}:(Z\sqsubseteq A\wedge Z\sqsubseteq B\Rightarrow Z\in F)\Leftrightarrow A\in F\vee B\in F$
for every $A,B\in\mathfrak{A}$ and $F\neq\subsets\mathfrak{A}$.
\item $\forall Z\in\mathfrak{A}:(Z\sqsubseteq A\wedge Z\sqsubseteq B\Rightarrow Z\in F)\Rightarrow A\in F\vee B\in F$
for every $A,B\in\mathfrak{A}$ and $F$ is an lower set and $F\neq\subsets\mathfrak{A}$.
\end{enumerate}
\end{cor}
\begin{obvious}
~
\begin{enumerate}
\item A free star cannot contain the least element of the poset.
\item A mixer cannot contain the greatest element of the poset.
\end{enumerate}
\end{obvious}

\subsection{Filters, ideals, free stars, and mixers on semilattices}
\begin{prop}
~
\begin{itemize}
\item Free stars are sets $F$ not equal to $\subsets\mathfrak{A}$ with
$A\in F\lor B\in F\Leftrightarrow\lnot\exists Z\in\overline{F}:(Z\sqsupseteq A\wedge Z\sqsupseteq B)$
(for every $A,B\in\mathfrak{Z}$).
\item Free stars are upper sets $F$ not equal to $\subsets\mathfrak{A}$
with $A\in F\lor B\in F\Leftarrow\lnot\exists Z\in\overline{F}:(Z\sqsupseteq A\wedge Z\sqsupseteq B)$
(for every $A,B\in\mathfrak{Z}$).
\item Mixers are sets $F$ not equal to $\subsets\mathfrak{A}$ with $A\in F\lor B\in F\Leftrightarrow\lnot\exists Z\in\overline{F}:(Z\sqsubseteq A\wedge Z\sqsubseteq B)$
(for every $A,B\in\mathfrak{Z}$).
\item Mixers are lower sets $F$ not equal to $\subsets\mathfrak{A}$ with
$A\in F\lor B\in F\Leftarrow\lnot\exists Z\in\overline{F}:(Z\sqsubseteq A\wedge Z\sqsubseteq B)$
(for every $A,B\in\mathfrak{Z}$).
\end{itemize}
\end{prop}
\begin{proof}
By duality.
\end{proof}
By duality and and an above theorem about filters, we have:
\begin{prop}
~
\begin{itemize}
\item Filters are nonempty sets $F$ with $A\sqcap B\in F\Leftrightarrow A\in F\land B\in F$
(for every $A,B\in\mathfrak{Z}$), whenever $\mathfrak{Z}$ is a meet-semilattice.
\item Ideals are nonempty sets $F$ with $A\sqcup B\in F\Leftrightarrow A\in F\land B\in F$
(for every $A,B\in\mathfrak{Z}$), whenever $\mathfrak{Z}$ is a join-semilattice.
\item Free stars are sets $F$ not equal to $\subsets\mathfrak{A}$ with
$A\sqcup B\in F\Leftrightarrow A\in F\lor B\in F$ (for every $A,B\in\mathfrak{Z}$),
whenever $\mathfrak{Z}$ is a join-semilattice.
\item Mixers are sets $F$ not equal to $\subsets\mathfrak{A}$ with $A\sqcap B\in F\Leftrightarrow A\in F\lor B\in F$
(for every $A,B\in\mathfrak{Z}$), whenever $\mathfrak{Z}$ is a meet-semilattice.
\end{itemize}
\end{prop}
By duality and and an above theorem about filters, we have:
\begin{prop}
~
\begin{itemize}
\item Filters are nonempty upper sets $F$ with $A\sqcap B\in F\Leftarrow A\in F\land B\in F$
(for every $A,B\in\mathfrak{Z}$), whenever $\mathfrak{Z}$ is a meet-semilattice.
\item Ideals are nonempty lower sets $F$ with $A\sqcup B\in F\Leftarrow A\in F\land B\in F$
(for every $A,B\in\mathfrak{Z}$), whenever $\mathfrak{Z}$ is a join-semilattice.
\item Free stars are upper sets $F$ not equal to $\subsets\mathfrak{A}$
with $A\sqcup B\in F\Rightarrow A\in F\lor B\in F$ (for every $A,B\in\mathfrak{Z}$),
whenever $\mathfrak{Z}$ is a join-semilattice.
\item Mixers are lower sets $F$ not equal to $\subsets\mathfrak{A}$ with
$A\sqcap B\in F\Rightarrow A\in F\lor B\in F$ (for every $A,B\in\mathfrak{Z}$),
whenever $\mathfrak{Z}$ is a meet-semilattice.
\end{itemize}
\end{prop}

\subsection{The general diagram}

Let $\mathfrak{A}$ and $\mathfrak{B}$ be two posets connected by
an order reversing isomorphism $\theta:\mathfrak{A}\rightarrow\mathfrak{B}$.
We have commutative diagram on the figure~\ref{theta-sets} in the
category $\mathbf{Set}$:

\begin{figure}[ht]
\caption{\label{theta-sets}}


\begin{tikzcd}[row sep=1cm, column sep=2cm]
  \subsets \mathfrak{A}
	\arrow[d, leftrightarrow, "\lnot"]
    \arrow[r, shift left, rightarrow, "\rsupfun{\theta}"]
    & \subsets \mathfrak{B} \arrow[l, shift left, rightarrow, "\rsupfun{\theta^{-1}}"] \arrow[d, leftrightarrow, "\lnot"] \\
  \subsets \mathfrak{A}
	\arrow[r, shift left, rightarrow, "\rsupfun{\theta}"]
    & \subsets \mathfrak{B} \arrow[l, shift left, rightarrow, "\rsupfun{\theta^{-1}}"]
\end{tikzcd}
\end{figure}

\begin{thm}
This diagram is commutative, every arrow of this diagram is an isomorphism,
every cycle in this diagrams is an identity (therefore ``parallel''
arrows are mutually inverse).\end{thm}
\begin{proof}
That every arrow is an isomorphism is obvious.

Show that $\rsupfun{\theta}\lnot X=\lnot\rsupfun{\theta}X$ for every
set $X\in\subsets\mathfrak{A}$.

Really, 
\begin{multline*}
p\in\rsupfun{\theta}\lnot X\Leftrightarrow\exists q\in\lnot X:p=\theta q\Leftrightarrow\exists q\in\lnot X:\theta^{-1}p=q\Leftrightarrow\theta^{-1}p\in\lnot X\Leftrightarrow\\
\nexists q\in X:q=\theta^{-1}p\Leftrightarrow\nexists q\in X:\theta q=p\Leftrightarrow p\notin\rsupfun{\theta}X\Leftrightarrow p\in\lnot\rsupfun{\theta}X.
\end{multline*}


Thus the theorem follows from lemma~\ref{four-loop-lem}.
\end{proof}
This diagram can be restricted to filters, ideals, free stars, and
mixers, see figure~\ref{theta-flt}:

\begin{figure}[ht]
\caption{\label{theta-flt}}


\begin{tikzcd}[row sep=1cm, column sep=2cm]
  \mathfrak{F}(\mathfrak{A})
	\arrow[d, leftrightarrow, "\lnot"]
    \arrow[r, shift left, rightarrow, "\rsupfun{\theta}"]
    & \mathfrak{I}(\mathfrak{B}) \arrow[l, shift left, rightarrow, "\rsupfun{\theta^{-1}}"] \arrow[d, leftrightarrow, "\lnot"] \\
  \mathfrak{M}(\mathfrak{A})
	\arrow[r, shift left, rightarrow, "\rsupfun{\theta}"]
    & \mathfrak{S}(\mathfrak{B}) \arrow[l, shift left, rightarrow, "\rsupfun{\theta^{-1}}"]
\end{tikzcd}
\end{figure}

\begin{thm}
It is a restriction of the above diagram. Every arrow of this diagram
is an isomorphism, every cycle in these diagrams is an identity. (To
prove that, is an easy application of duality and the above lemma.)
\end{thm}

\subsection{Special diagrams}

Here are two important special cases of the above diagram:\begin{equation}\label{two-diags}\begin{tikzcd}
  \mathfrak{F}(\mathfrak{A})
	\arrow[d, leftrightarrow, "\lnot"]
    \arrow[r, leftrightarrow, "\rsupfun{\dual}"]
    & \mathfrak{I}(\dual\mathfrak{A}) \arrow[d, leftrightarrow, "\lnot"] \\
  \mathfrak{M}(\mathfrak{A})
	\arrow[r, leftrightarrow, "\rsupfun{\dual}"]
    & \mathfrak{S}(\dual\mathfrak{A})
\end{tikzcd}
\quad\text{and}\quad
\begin{tikzcd}
  \mathfrak{F}(\mathfrak{A})
	\arrow[d, leftrightarrow, "\lnot"]
    \arrow[r, leftrightarrow, "\rsupfun{\lnot}"]
    & \mathfrak{I}(\mathfrak{A}) \arrow[d, leftrightarrow, "\lnot"] \\
  \mathfrak{M}(\mathfrak{A})
	\arrow[r, leftrightarrow, "\rsupfun{\lnot}"]
    & \mathfrak{S}(\mathfrak{A})
\end{tikzcd}
\end{equation}(the second diagram is defined for a boolean lattice~$\mathfrak{A}$).


\subsection{Order of ideals, free stars, mixers}

Define order of ideals, free stars, mixers in such a way that the
above diagrams isomorphically preserve order of filters:
\begin{itemize}
\item $A\sqsubseteq B\Leftrightarrow A\supseteq B$ for filters and ideals;
\item $A\sqsubseteq B\Leftrightarrow A\subseteq B$ for free stars and mixers.
\end{itemize}

\subsection{Principal ideals, free stars, mixers}
\begin{defn}
\emph{Principal} ideal generated by an element~$a$ of poset~$\mathfrak{A}$
is $\downarrow a=\setcond{x\in\mathfrak{A}}{x\sqsubseteq a}$.
\end{defn}

\begin{defn}
An ideal is \emph{principal} iff it is generated by some poset element.
\end{defn}

\begin{defn}
The \emph{filtrator of ideals} on a given poset is the pair consisting
of the set of ideals and the set of principal ideals.

The above poset isomorphism maps principal filters into principal
ideals and thus is an isomorphism between the filtrator of filters
on a poset and the filtrator of ideals on the dual poset.\end{defn}
\begin{xca}
Define principal free stars and mixers, filtrators of free stars and
mixers and isomorphisms of these with the filtrator of filters (these
isomorphisms exist because the posets of free stars and mixers are
isomorphic to the poset of filters).\end{xca}
\begin{obvious}
The following filtrators are primary:
\begin{itemize}
\item filtrators of filters;
\item filtrators of ideals;
\item filtrators of free stars;
\item filtrators of mixers.
\end{itemize}
\end{obvious}

\subsubsection{Principal free stars}
\begin{prop}
An upper set $F\in\subsets\mathfrak{Z}$ is a principal filter iff
$\exists Z\in F\forall P\in F:Z\sqsubseteq P$.\end{prop}
\begin{proof}
~
\begin{description}
\item [{$\Rightarrow$}] Obvious.
\item [{$\Leftarrow$}] Let $Z\in F$ and $\forall P\in F:Z\sqsubseteq P$.
$F$ is nonempty because $Z\in F$. It remains to prove that $Z\sqsubseteq P\Leftrightarrow P\in F$.
The reverse implication follows from $\forall P\in F:Z\sqsubseteq P$.
The direct implication follows from that $F$ is an upper set.
\end{description}
\end{proof}
\begin{lem}
If $S\in\subsets\mathfrak{Z}$ is not the complement of empty set and for every $T\in\subsets\mathfrak{A}$
\[
\forall Z\in\mathfrak{A}:(\forall X\in T:Z\sqsupseteq X\Rightarrow Z\in S)\Leftrightarrow T\cap S\neq\emptyset,
\]
then $S$ is a free star.\end{lem}
\begin{proof}
Take $T=\{A,B\}$. Then $\forall Z\in\mathfrak{A}:(Z\sqsupseteq A\wedge Z\sqsupseteq B\Rightarrow Z\in S)\Leftrightarrow A\in S\vee B\in S$.
So $S$ is a free star.\end{proof}
\begin{prop}
A set $S\in\subsets\mathfrak{Z}$ is a principal free star iff $S$
is not the complement of empty set and for every $T\in\subsets\mathfrak{A}$
\[
\forall Z\in\mathfrak{A}:(\forall X\in T:Z\sqsupseteq X\Rightarrow Z\in S)\Leftrightarrow T\cap S\neq\emptyset.
\]
\end{prop}
\begin{proof}
Let $S=\overline{\langle\dual\rangle^{\ast}F}$. We need to prove
that $F$ is a principal filter. Really, we have the following chain
of equivalencies:

$\forall Z\in\mathfrak{A}:(\forall X\in T:Z\sqsupseteq X\Rightarrow Z\in S)\Leftrightarrow T\cap S\neq\emptyset$;

$\forall Z\in\mathfrak{A}:(\forall X\in T:Z\sqsupseteq X\Rightarrow Z\notin\langle\dual\rangle^{\ast}F)\Leftrightarrow T\cap\overline{\langle\dual\rangle^{\ast}F}\neq\emptyset$;

$\forall Z\in\dual\mathfrak{A}:(\forall X\in T:Z\sqsubseteq X\Rightarrow Z\notin F)\Leftrightarrow T\cap\overline{F}\neq\emptyset$;

$\forall Z\in\dual\mathfrak{A}:(\forall X\in T:Z\sqsubseteq X\Rightarrow Z\notin F)\Leftrightarrow T\nsubseteq F$;

$T\subseteq F\Leftrightarrow\neg\forall Z\in\dual\mathfrak{A}:(Z\in F\Rightarrow\neg\forall X\in T:Z\sqsubseteq X)$;

$T\subseteq F\Leftrightarrow\neg\forall Z\in\dual\mathfrak{A}:(Z\notin F\vee\neg\forall X\in T:Z\sqsubseteq X)$;

$T\subseteq F\Leftrightarrow\exists Z\in\dual\mathfrak{A}:(Z\in F\wedge\forall X\in T:Z\sqsubseteq X)$;

$T\subseteq F\Leftrightarrow\exists Z\in F\forall X\in T:Z\sqsubseteq X$;

$\exists Z\in F\forall X\in F:Z\sqsubseteq X$ that is $F$ is a principal
filter ($S$ is an upper set because by the lemma it is a free star;
thus $F$ is also an upper set).\end{proof}
\begin{prop}
$S\in\subsets\mathfrak{A}$ where $\mathfrak{A}$ is a poset is a
principal free star iff all the following:
\begin{enumerate}
\item \label{princ-fs-least}The least element (if it exists) is not in
$S$.
\item \label{princ-fs-form}$\forall Z\in\mathfrak{A}:(\forall X\in T:Z\sqsupseteq X\Rightarrow Z\in S)\Rightarrow T\cap S\neq\emptyset$
for every $T\in\subsets\mathfrak{A}$.
\item $S$ is an upper set.
\end{enumerate}
\end{prop}
\begin{proof}
~
\begin{description}
\item [{$\Rightarrow$}] \ref{princ-fs-least} and \ref{princ-fs-form}
are obvious. $S$ is an upper set because $S$ is a free star.
\item [{$\Leftarrow$}] We need to prove that 
\[
\forall Z\in\mathfrak{A}:(\forall X\in T:Z\sqsupseteq X\Rightarrow Z\in S)\Leftarrow T\cap S\neq\emptyset.
\]
Let $X'\in T\cap S$. Then $\forall X\in T:Z\sqsupseteq X\Rightarrow Z\sqsupseteq X'\Rightarrow Z\in S$
because $S$ is an upper set.
\end{description}
\end{proof}
\begin{prop}
Let $S$ be a complete lattice. $S\in\subsets\mathfrak{A}$ is a principal
free star iff all the following:
\begin{enumerate}
\item The least element (if it exists) is not in $S$.
\item $\bigsqcup T\in S\Rightarrow T\cap S\neq\emptyset$ for every $T\in\subsets\mathfrak{A}$.
\item $S$ is an upper set.
\end{enumerate}
\end{prop}
\begin{proof}
~
\begin{description}
\item [{$\Rightarrow$}] We need to prove only $\bigsqcup T\in S\Rightarrow T\cap S\neq\emptyset$.
Let $\bigsqcup T\in S$. Because $S$ is an upper set, we have $\forall X\in T:Z\sqsupseteq X\Rightarrow Z\sqsupseteq\bigsqcup T\Rightarrow Z\in S$
from which we conclude $T\cap S\neq\emptyset$.
\item [{$\Leftarrow$}] We need to prove only $\forall Z\in\mathfrak{A}:(\forall X\in T:Z\sqsupseteq X\Rightarrow Z\in S)\Rightarrow T\cap S\neq\emptyset$.


Really, if $\forall Z\in\mathfrak{A}:(\forall X\in T:Z\sqsupseteq X\Rightarrow Z\in S)$
then $\bigsqcup T\in S$ and thus $\bigsqcup T\in S\Rightarrow T\cap S\neq\emptyset$.

\end{description}
\end{proof}
\begin{prop}
Let $\mathfrak{A}$ be a complete lattice. $S\in\subsets\mathfrak{A}$
is a principal free star iff the least element (if it exists) is not
in $S$ and for every $T\in\subsets\mathfrak{A}$ 
\[
\bigsqcup T\in S\Leftrightarrow T\cap S\neq\emptyset.
\]
\end{prop}
\begin{proof}
~
\begin{description}
\item [{$\Rightarrow$}] We need to prove only $\bigsqcup T\in S\Leftarrow T\cap S\neq\emptyset$
what follows from that $S$ is an upper set.
\item [{$\Leftarrow$}] We need to prove only that $S$ is an upper set.
To prove this we can use the fact that $S$ is a free star.
\end{description}
\end{proof}
\begin{xca}
Write down similar formulas for mixers.
\end{xca}

\subsection{Starrish posets}
\begin{defn}
\index{starrish}\index{poset!starrish}I will call a poset \emph{starrish}
when the full star $\fullstar a$ is a free star for every element
$a$ of this poset.\end{defn}
\begin{prop}
Every distributive lattice is starrish.\end{prop}
\begin{proof}
Let $\mathfrak{A}$ be a distributive lattice, $a\in\mathfrak{A}$.
Obviously $\bot\notin\fullstar a$ (if $\bot$ exists); obviously
$\fullstar a$ is an upper set. If $x\sqcup y\in\fullstar a$, then
$(x\sqcup y)\sqcap a$ is non-least that is $(x\sqcap a)\sqcup(y\sqcap a)$
is non-least what is equivalent to $x\sqcap a$ or $y\sqcap a$ being
non-least that is $x\in\fullstar a\lor y\in\fullstar a$.\end{proof}
\begin{thm}
\label{atoms-join}If $\mathfrak{A}$ is a starrish join-semilattice
lattice then
\[
\atoms(a\sqcup b)=\atoms a\cup\atoms b
\]
for every $a,b\in\mathfrak{A}$.\end{thm}
\begin{proof}
For every atom $c$ we have:
\begin{align*}
c\in\atoms(a\sqcup b) & \Leftrightarrow\\
c\nasymp a\sqcup b & \Leftrightarrow\\
a\sqcup b\in\fullstar c & \Leftrightarrow\\
a\in\fullstar c\lor b\in\fullstar c & \Leftrightarrow\\
c\nasymp a\lor c\nasymp b & \Leftrightarrow\\
c\in\atoms a\lor c\in\atoms b.
\end{align*}

\end{proof}

\subsubsection{Completely starrish posets}
\begin{defn}
\index{completely starrish}I will call a poset \emph{completely starrish}
when the full star $\star a$ is a principal free star for every element
$a$ of this poset.\end{defn}
\begin{obvious}
Every completely starrish poset is starrish.\end{obvious}
\begin{prop}
Every complete join infinite distributive lattice is completely starrish.\end{prop}
\begin{proof}
Let $\mathfrak{A}$ be a join infinite distributive lattice, $a\in\mathfrak{A}$.
Obviously $\bot\notin\fullstar a$ (if $\bot$ exists); obviously
$\fullstar a$ is an upper set. If $\bigsqcup T\in\fullstar a$, then
$\left(\bigsqcup T\right)\sqcap a$ is non-least that is $\bigsqcup\rsupfun{a\sqcap}T$
is non-least what is equivalent to $a\sqcap x$ being non-least for
some $x\in T$ that is $x\in\fullstar a$.\end{proof}
\begin{thm}
If $\mathfrak{A}$ is a completely starrish complete lattice lattice
then 
\[
\atoms\bigsqcup T=\bigcup\rsupfun{\atoms}T.
\]
for every $T\in\subsets\mathfrak{A}$.\end{thm}
\begin{proof}
For every atom $c$ we have: 
\begin{multline*}
c\in\atoms\bigsqcup T\Leftrightarrow c\nasymp\bigsqcup T\Leftrightarrow\bigsqcup T\in\fullstar c\Leftrightarrow\exists X\in T:X\in\fullstar c\Leftrightarrow\\
\exists X\in T:X\nasymp c\Leftrightarrow\exists X\in T:c\in\atoms X\Leftrightarrow c\in\bigcup\rsupfun{\atoms}T.
\end{multline*}

\end{proof}

\section{Basic properties of filters}
\begin{prop}
$\up\mathcal{A}=\mathcal{A}$ for every filter $\mathcal{A}$ (provided
that we equate elements of the base poset~$\mathfrak{Z}$ with corresponding
principal filters.\end{prop}
\begin{proof}
$A\in\up\mathcal{A}\Leftrightarrow A\sqsupseteq\mathcal{A}\Leftrightarrow\uparrow A\sqsupseteq\mathcal{A}\Leftrightarrow\uparrow A\subseteq\mathcal{A}\Leftrightarrow A\in\mathcal{A}$.
\end{proof}

\subsection{Minimal and maximal filters}
\begin{obvious}
The following is an implications tuple:
\begin{enumerate}
\item $(\mathfrak{A};\mathfrak{Z})$ is a powerset filtrator.
\item $(\mathfrak{A};\mathfrak{Z})$ is a primary filtrator.
\item $\bot^{\mathfrak{A}}$ (equal to the principal filter for the least
element of $\mathfrak{Z}$ if it exists) defined by the formula $\up\bot^{\mathfrak{A}}=\mathfrak{Z}$
is the least element of~$\mathfrak{A}$.
\end{enumerate}
\end{obvious}
\begin{prop}
The following is an implications tuple:
\begin{enumerate}
\item $(\mathfrak{A};\mathfrak{Z})$ is a powerset filtrator.
\item $(\mathfrak{A};\mathfrak{Z})$ is a primary filtrator with greatest
element.
\item $\top^{\mathfrak{A}}$ defined by the formula $\up\top^{\mathfrak{A}}=\{\top^{\mathfrak{Z}}\}$
is the greatest element of~$\mathfrak{A}$.
\end{enumerate}
\end{prop}
\begin{proof}
Take into account that filters are nonempty.
\end{proof}

\subsection{Alignment}
\begin{defn}
\index{filtrator!down-aligned}I call \emph{down-aligned} filtrator
such a filtrator $(\mathfrak{A};\mathfrak{Z})$ that $\mathfrak{A}$
and $\mathfrak{Z}$ have common least element. (Let's denote it $\bot$.)
\end{defn}

\begin{defn}
\index{filtrator!up-aligned}I call \emph{up-aligned} filtrator such
a filtrator $(\mathfrak{A};\mathfrak{Z})$ that $\mathfrak{A}$ and
$\mathfrak{Z}$ have common greatest element. (Let's denote it $\top$.)\end{defn}
\begin{obvious}
\label{filt-aligned}~
\begin{enumerate}
\item If $\mathfrak{Z}$ has least element, the primary filtrator is down-aligned.
\item If $\mathfrak{Z}$ has greatest element, the primary filtrator is
up-aligned.
\end{enumerate}
\end{obvious}
\begin{cor}
Every powerset filtrator is both up and down-aligned.
\end{cor}
We can also define (without requirement of having least and greatest
elements, but coinciding with the above definitions if least/greatest
elements are present):
\begin{defn}
\index{filtrator!weakly down-aligned}I call \emph{weakly down-aligned}
filtrator such a filtrator $(\mathfrak{A};\mathfrak{Z})$ that $\forall a\in\mathfrak{A}\exists b\in\mathfrak{Z}:b\sqsubseteq a$.
\end{defn}

\begin{defn}
\index{filtrator!weakly up-aligned}I call \emph{weakly up-aligned}
filtrator such a filtrator $(\mathfrak{A};\mathfrak{Z})$ that $\forall a\in\mathfrak{A}\exists b\in\mathfrak{Z}:b\sqsupseteq a$.
\end{defn}
However in this edition of this book I do not use the concepts of
weakly down-aligned and weakly up-aligned filtrators for simplicity
of proofs. Maybe I will add it in a future edition.


\section{More advanced properties of filters}


\subsection{Formulas for Meets and Joins of Filters}
\begin{lem}
\label{embed-lemma}If $f$ is an order embedding from a poset $\mathfrak{A}$
to a complete lattice $\mathfrak{B}$ and $S\in\subsets\mathfrak{A}$
and there exists such $\mathcal{F}\in\mathfrak{A}$ that $f\mathcal{F}=\bigsqcup^{\mathfrak{B}}\rsupfun fS$,
then $\bigsqcup^{\mathfrak{A}}S$ exists and $f\bigsqcup^{\mathfrak{A}}S=\bigsqcup^{\mathfrak{B}}\rsupfun fS$.\end{lem}
\begin{proof}
$f$ is an order isomorphism from $\mathfrak{A}$ to $\mathfrak{B}|_{\rsupfun f\mathfrak{A}}$.
$f\mathcal{F}\in\mathfrak{B}|_{\rsupfun f\mathfrak{A}}$.

Consequently, $\bigsqcup^{\mathfrak{B}}\rsupfun fS\in\mathfrak{B}|_{\rsupfun f\mathfrak{A}}$
and $\bigsqcup^{\mathfrak{B}|_{\rsupfun f\mathfrak{A}}}\rsupfun fS=\bigsqcup^{\mathfrak{B}}\rsupfun fS$.

$f\bigsqcup^{\mathfrak{A}}S=\bigsqcup^{\mathfrak{B}|_{\rsupfun f\mathfrak{A}}}\rsupfun fS$
because $f$ is an order isomorphism.

Combining, $f\bigsqcup^{\mathfrak{A}}S=\bigsqcup^{\mathfrak{B}}\rsupfun fS$.\end{proof}
\begin{cor}
If $\mathfrak{B}$ is a complete lattice and $\mathfrak{A}$ is its
subset and $S\in\subsets\mathfrak{A}$ and $\bigsqcup^{\mathfrak{B}}S\in\mathfrak{A}$,
then $\bigsqcup^{\mathfrak{A}}S$
exists and $\bigsqcup^{\mathfrak{A}}S=\bigsqcup^{\mathfrak{B}}S$.\end{cor}

\begin{xca}
The below theorem does not work for $S=\emptyset$. Formulate the general case.
\end{xca}

\begin{thm}
~
\begin{enumerate}
\item \label{inf-lat-filt}If $\mathfrak{Z}$ is a meet-semilattice with
greatest element, then $\bigsqcup^{\mathfrak{F}(\mathfrak{Z})}S$
exists and $\bigsqcup^{\mathfrak{F}(\mathfrak{Z})}=\bigcap S$ for
every $S\in\subsets\mathfrak{F}(\mathfrak{Z})\setminus\{\emptyset\}$.
\item \label{inf-lat-ideal}If $\mathfrak{Z}$ is a join-semilattice with
least element, then $\bigsqcap^{\mathfrak{I}(\mathfrak{Z})}S$ exists
and $\bigsqcap^{\mathfrak{I}(\mathfrak{Z})}=\bigcap S$ for every
$S\in\subsets\mathfrak{I}(\mathfrak{Z})\setminus\{\emptyset\}$.
\item \label{inf-lat-fs}If $\mathfrak{Z}$ is a join-semilattice with least
element, then $\bigsqcup^{\mathfrak{S}(\mathfrak{Z})}S$ exists and
$\bigsqcup^{\mathfrak{S}(\mathfrak{Z})}=\bigcup S$ for every $S\in\subsets\mathfrak{S}(\mathfrak{Z})$.
\item \label{inf-lat-mix}If $\mathfrak{Z}$ is a meet-semilattice with
greatest element, then $\bigsqcap^{\mathfrak{M}(\mathfrak{Z})}S$
exists and $\bigsqcap^{\mathfrak{M}(\mathfrak{Z})}=\bigcup S$ for
every $S\in\subsets\mathfrak{M}(\mathfrak{Z})$.
\end{enumerate}
\end{thm}
\begin{proof}
~
\begin{widedisorder}
\item [{\ref{inf-lat-filt}}] Taking into account the lemma, it is enough
to prove that $\bigcap S$ is a filter. $\bigcap S$ is nonempty because
$\top\in\bigcap S$. For every $A,B\in\mathfrak{Z}$ we have:
\[
A,B\in\bigcap S\Leftrightarrow\forall P\in S:A,B\in P\Leftrightarrow\forall P\in S:A\sqcap B\in P\Leftrightarrow A\sqcap B\in\bigcap S.
\]
So $\bigcap S$ is a filter.
\item [{\ref{inf-lat-ideal}}] By duality.
\item [{\ref{inf-lat-fs}}] Taking into account the lemma, it is enough
to prove that $\bigcup S$ is a free star. $\bigcup S$ is not the
complement of empty set because $\bot\notin\bigcup S$. For every
$A,B\in\mathfrak{Z}$ we have:
\begin{multline*}
A\in\bigcup S\lor B\in\bigcup S\Leftrightarrow\exists P\in S:(A\in P\lor B\in P)\Leftrightarrow\\
\exists P\in S:A\sqcup B\in P\Leftrightarrow A\sqcup B\in\bigcup S.
\end{multline*}

\item [{\ref{inf-lat-mix}}] By duality.
\end{widedisorder}
\end{proof}
\begin{cor}
\label{f-join-form}The following is an implications tuple:
\begin{enumerate}
\item \label{f-join-form-p}$(\mathfrak{A};\mathfrak{Z})$ is a powerset
filtrator.
\item \label{f-join-form-fltr}$(\mathfrak{A};\mathfrak{Z})$ is a primary
filtrator over a meet-semilattice with greatest element~$\top$.
\item \label{f-join-form-conc}$\bigsqcup^{\mathfrak{A}}S$ exists and $\up\bigsqcup^{\mathfrak{A}}S=\bigcap\rsupfun{\up}S$
for every $S\in\subsets\mathfrak{A}\setminus\{\emptyset\}$.
\end{enumerate}
\end{cor}
\begin{proof}
~
\begin{description}
\item [{\ref{f-join-form-p}$\Rightarrow$\ref{f-join-form-fltr}}] Obvious.
\item [{\ref{f-join-form-fltr}$\Rightarrow$\ref{f-join-form-conc}}] By
the theorem.
\end{description}
\end{proof}
\begin{cor}
\label{filt-is-complete}The following is an implications tuple:
\begin{enumerate}
\item $(\mathfrak{A};\mathfrak{Z})$ is a powerset filtrator.
\item $(\mathfrak{A};\mathfrak{Z})$ is a primary filtrator over a meet-semilattice
with greatest element~$\top$.
\item $\mathfrak{A}$ is a complete lattice.
\end{enumerate}
\end{cor}

\begin{cor}
The following is an implications tuple:
\begin{enumerate}
\item $(\mathfrak{A};\mathfrak{Z})$ is a powerset filtrator.
\item $(\mathfrak{A};\mathfrak{Z})$ is a primary filtrator over a meet-semilattice
with greatest element~$\top$.
\item $\up(\mathcal{A}\sqcup^{\mathfrak{A}}\mathcal{B})=\up\mathcal{A}\cap\up\mathcal{B}$
for every $\mathcal{A},\mathcal{B}\in\mathfrak{A}$.
\end{enumerate}
\end{cor}
We will denote meets and joins on the lattice of filters just as~$\sqcap$
and~$\sqcup$.
\begin{cor}
If $\mathfrak{Z}$ is a join-semilattice with least element then $\mathfrak{A}$ is a
join-semilattice and for any $\mathcal{A},\mathcal{B}\in\mathfrak{A}$
\[
\up(\mathcal{A}\sqcup^{\mathfrak{A}}\mathcal{B})=\up\mathcal{A}\cap\up\mathcal{B}.
\]
\end{cor}
\begin{thm}
Let $\mathfrak{Z}$ be a distributive lattice. Then
\begin{enumerate}
\item $\bigsqcap^{\mathfrak{F}(\mathfrak{Z})}S=\setcond{K_{0}\sqcap^{\mathfrak{Z}}\dots\sqcap^{\mathfrak{Z}}K_{n}}{K_{i}\in\bigcup S\text{ where }i=0,\dots,n\text{ for }n\in\mathbb{N}}$
for $S\in\subsets\mathfrak{F}(\mathfrak{Z})\setminus\{\emptyset\}$;
\item $\bigsqcup^{\mathfrak{I}(\mathfrak{Z})}S=\setcond{K_{0}\sqcup^{\mathfrak{Z}}\dots\sqcup^{\mathfrak{Z}}K_{n}}{K_{i}\in\bigcup S\text{ where }i=0,\dots,n\text{ for }n\in\mathbb{N}}$
for $S\in\subsets\mathfrak{I}(\mathfrak{Z})\setminus\{\emptyset\}$.
\end{enumerate}
\end{thm}
\begin{proof}
We will prove only the first, as the second is dual.

Let's denote the right part of the equality to be proven as $R$.
First we will prove that $R$ is a filter. $R$ is nonempty because
$S$ is nonempty.

Let $A,B\in R$. Then $A=X_{0}\sqcap^{\mathfrak{Z}}\dots\sqcap^{\mathfrak{Z}}X_{k}$,
$B=Y_{0}\sqcap^{\mathfrak{Z}}\dots\sqcap^{\mathfrak{Z}}Y_{l}$ where
$X_{i},Y_{j}\in\bigcup S$. So
\[
A\sqcap^{\mathfrak{Z}}B=X_{0}\sqcap^{\mathfrak{Z}}\dots\sqcap^{\mathfrak{Z}}X_{k}\sqcap^{\mathfrak{Z}}Y_{0}\sqcap^{\mathfrak{Z}}\dots\sqcap^{\mathfrak{Z}}Y_{l}\in R.
\]



Let filter $C\sqsupseteq A\in R$. Consequently (distributivity used)
\[
C=C\sqcup^{\mathfrak{Z}}A=(C\sqcup^{\mathfrak{Z}}X_{0})\sqcap^{\mathfrak{Z}}\dots\sqcap^{\mathfrak{Z}}(C\sqcup^{\mathfrak{Z}}X_{k}).
\]



$X_{i}\in P_{i}$ for some $P_{i}\in S$; $C\sqcup^{\mathfrak{Z}}X_{i}\in P_{i}$;
$C\sqcup^{\mathfrak{Z}}X_{i}\in\bigcup S$; consequently $C\in R$.


We have proved that that $R$ is a filter base and an upper set. So
$R$ is a filter.


Let $\mathcal{A}\in S$. Then $\mathcal{A}\subseteq\bigcup S$;
\[
R\supseteq\setcond{K_{0}\sqcap^{\mathfrak{Z}}\dots\sqcap^{\mathfrak{Z}}K_{n}}{K_{i}\in\mathcal{A}\text{ where }i=0,\dots,n\text{ for }n\in\mathbb{N}}=\mathcal{A}.
\]



Consequently $\mathcal{A}\sqsupseteq R$.


Let now $\mathcal{B}\in\mathfrak{A}$ and $\forall\mathcal{A}\in S:\mathcal{A}\sqsupseteq\mathcal{B}$.
Then $\forall\mathcal{A}\in S:\mathcal{A}\subseteq\mathcal{B}$; $\mathcal{B}\supseteq\bigcup S$.
Thus $\mathcal{B}\supseteq T$ for every finite set $T\subseteq\bigcup S$.
Consequently $\up\mathcal{B}\ni\bigsqcap^{\mathfrak{Z}}T$. Thus $\mathcal{B}\supseteq R$;
$\mathcal{B}\sqsubseteq R$.


Comparing we get $\bigsqcap^{\mathfrak{F}(\mathfrak{Z})}S=R$.

\end{proof}
\begin{cor}
\label{f-inf-meet-form}The following is an implications tuple:
\begin{enumerate}
\item \label{inf-meet-form-p}$(\mathfrak{A};\mathfrak{Z})$ is a powerset
filtrator.
\item \label{inf-meet-form-fltr}$(\mathfrak{A};\mathfrak{Z})$ is a primary
filtrator over a distributive lattice.
\item \label{inf-meet-form-conc}$\up\bigsqcap^{\mathfrak{A}}S=\setcond{K_{0}\sqcap^{\mathfrak{Z}}\dots\sqcap^{\mathfrak{Z}}K_{n}}{K_{i}\in\bigcup\rsupfun{\up}S\text{ where }i=0,\dots,n\text{ for }n\in\mathbb{N}}$
for $S\in\subsets\mathfrak{A}\setminus\{\emptyset\}$.
\end{enumerate}
\end{cor}
\begin{proof}
~
\begin{description}
\item [{\ref{inf-meet-form-p}$\Rightarrow$\ref{inf-meet-form-fltr}}] Obvious.
\item [{\ref{inf-meet-form-fltr}$\Rightarrow$\ref{inf-meet-form-conc}}] By
the theorem.
\end{description}
\end{proof}
\begin{thm}
Let $\mathfrak{Z}$ be a distributive lattice. Then:
\begin{enumerate}
\item $\mathcal{F}_{0}\sqcap^{\mathfrak{F}(\mathfrak{Z})}\dots\sqcap^{\mathfrak{F}(\mathfrak{Z})}\mathcal{F}_{m}=\setcond{K_{0}\sqcap^{\mathfrak{Z}}\dots\sqcap^{\mathfrak{Z}}K_{m}}{K_{i}\in\mathcal{F}_{i}\text{ where }i=0,\dots,m}$
for any $\mathcal{F}_{0},\dots,\mathcal{F}_{m}\in\mathfrak{F}(\mathfrak{Z})$;
\item $\mathcal{F}_{0}\sqcup^{\mathfrak{I}(\mathfrak{Z})}\dots\sqcup^{\mathfrak{I}(\mathfrak{Z})}\mathcal{F}_{m}=\setcond{K_{0}\sqcup^{\mathfrak{Z}}\dots\sqcup^{\mathfrak{Z}}K_{m}}{K_{i}\in\mathcal{F}_{i}\text{ where }i=0,\dots,m}$
for any $\mathcal{F}_{0},\dots,\mathcal{F}_{m}\in\mathfrak{I}(\mathfrak{Z})$.
\end{enumerate}
\end{thm}
\begin{proof}
We will prove only the first as the second is dual.

Let's denote the right part of the equality to be proven as $R$.
First we will prove that $R$ is a filter. Obviously $R$ is nonempty.

Let $A,B\in R$. Then $A=X_{0}\sqcap^{\mathfrak{Z}}\dots\sqcap^{\mathfrak{Z}}X_{m}$,
$B=Y_{0}\sqcap^{\mathfrak{Z}}\dots\sqcap^{\mathfrak{Z}}Y_{m}$ where
$X_{i},Y_{i}\in\mathcal{F}_{i}$.
\[
A\sqcap^{\mathfrak{Z}}B=(X_{0}\sqcap^{\mathfrak{Z}}Y_{0})\sqcap^{\mathfrak{Z}}\dots\sqcap^{\mathfrak{Z}}(X_{m}\sqcap^{\mathfrak{Z}}Y_{m}),
\]


consequently $A\sqcap^{\mathfrak{Z}}B\in R$.

Let filter $C\sqsupseteq A\in R$
\[
C=A\sqcup^{\mathfrak{Z}}C=(X_{0}\sqcup^{\mathfrak{Z}}C)\sqcap^{\mathfrak{Z}}\dots\sqcap^{\mathfrak{Z}}(X_{m}\sqcup^{\mathfrak{Z}}C)\in R.
\]


So $R$ is a filter.

Let $P_{i}\in\mathcal{F}_{i}$. Then $P_{i}\in R$ because $P_{i}=(P_{i}\sqcup^{\mathfrak{Z}}P_{0})\sqcap^{\mathfrak{Z}}\dots\sqcap^{\mathfrak{Z}}(P_{i}\sqcup^{\mathfrak{Z}}P_{m})$.
So $\mathcal{F}_{i}\subseteq R$; $\mathcal{F}_{i}\sqsupseteq R$.

Let now $\mathcal{B}\in\mathfrak{A}$ and $\forall i\in\{0,\dots,m\}:\mathcal{F}_{i}\sqsupseteq\mathcal{B}$.
Then $\forall i\in\{0,\dots,m\}:\mathcal{F}_{i}\subseteq\mathcal{B}$.

Let $L_{i}\in\mathcal{B}$ for every $L_{i}\in\mathcal{F}_{i}$. $L_{0}\sqcap^{\mathfrak{Z}}\dots\sqcap^{\mathfrak{Z}}L_{m}\in\mathcal{B}$.
So $\mathcal{B}\supseteq R$; $\mathcal{B}\sqsubseteq R$.

So $\mathcal{F}_{0}\sqcap^{\mathfrak{F}(\mathfrak{Z})}\dots\sqcap^{\mathfrak{F}(\mathfrak{Z})}\mathcal{F}_{m}=R$.\end{proof}
\begin{cor}
\label{f-fin-filt-meet}The following is an implications tuple:
\begin{enumerate}
\item \label{f-fin-filt-meet-p}$(\mathfrak{A};\mathfrak{Z})$ is a powerset
filtrator.
\item \label{fin-filt-meet-fltr}$(\mathfrak{A};\mathfrak{Z})$ is a primary
filtrator over a distributive lattice.
\item \label{fin-filt-meet-conc}$\up(\mathcal{F}_{0}\sqcap^{\mathfrak{A}}\dots\sqcap^{\mathfrak{A}}\mathcal{F}_{m})=\setcond{K_{0}\sqcap^{\mathfrak{Z}}\dots\sqcap^{\mathfrak{Z}}K_{m}}{K_{i}\in\up\mathcal{F}_{i}\text{ where }i=0,\dots,m}$
for any $\mathcal{F}_{0},\dots,\mathcal{F}_{m}\in\mathfrak{A}$.
\end{enumerate}
\end{cor}
\begin{proof}
~
\begin{description}
\item [{\ref{f-fin-filt-meet-p}$\Rightarrow$\ref{fin-filt-meet-fltr}}] Obvious.
\item [{\ref{fin-filt-meet-fltr}$\Rightarrow$\ref{fin-filt-meet-conc}}] By
the theorem.
\end{description}
\end{proof}
More general case of semilattices follows:
\begin{thm}
~
\begin{enumerate}
\item $\bigsqcap^{\mathfrak{F}(\mathfrak{Z})}S=\bigcup\setcond{\uparrow(K_{0}\sqcap^{\mathfrak{Z}}\dots\sqcap^{\mathfrak{Z}}K_{n})}{K_{i}\in\bigcup S\text{ where }i=0,\dots,n\text{ for }n\in\mathbb{N}}$
for $S\in\subsets\mathfrak{F}(\mathfrak{Z})\setminus\{\emptyset\}$
if $\mathfrak{Z}$ is a meet-semilattice;
\item $\bigsqcup^{\mathfrak{I}(\mathfrak{Z})}S=\bigcup\setcond{\uparrow(K_{0}\sqcup^{\mathfrak{Z}}\dots\sqcup^{\mathfrak{Z}}K_{n})}{K_{i}\in\bigcup S\text{ where }i=0,\dots,n\text{ for }n\in\mathbb{N}}$
for $S\in\subsets\mathfrak{I}(\mathfrak{Z})\setminus\{\emptyset\}$
if $\mathfrak{Z}$ is a join-semilattice.
\end{enumerate}
\end{thm}
\begin{proof}
We will prove only the first as the second is dual.

It follows from the fact that 
\[
\bigsqcap^{\mathfrak{F}(\mathfrak{Z})}S=\bigsqcap^{\mathfrak{F}(\mathfrak{Z})}\setcond{K_{0}\sqcap^{\mathfrak{Z}}\dots\sqcap^{\mathfrak{Z}}K_{n}}{K_{i}\in\bigcup S\text{ where }i=0,\dots,n\text{ for }n\in\mathbb{N}}
\]
and that $\setcond{K_{0}\sqcap^{\mathfrak{Z}}\dots\sqcap^{\mathfrak{Z}}K_{n}}{K_{i}\in\bigcup S\text{ where }i=0,\dots,n\text{ for }n\in\mathbb{N}}$
is a filter base.\end{proof}
\begin{cor}
\label{meet-filtx}The following is an implications tuple:
\begin{enumerate}
\item $(\mathfrak{A};\mathfrak{Z})$ is a powerset filtrator.
\item $(\mathfrak{A};\mathfrak{Z})$ is a primary filtrator over a meet-semilattice.
\item $\up\bigsqcap S=\bigcup\setcond{\up(K_{0}\sqcap^{\mathfrak{Z}}\dots\sqcap^{\mathfrak{Z}}K_{n})}{K_{i}\in\bigcup\rsupfun{\up}S\text{ where }i=0,\dots,n\text{ for }n\in\mathbb{N}}$
for every $S\in\subsets\mathfrak{A}\setminus\{\emptyset\}$.
\end{enumerate}
\end{cor}
\begin{thm}
~
\begin{enumerate}
\item $\mathcal{F}_{0}\sqcap^{\mathfrak{F}(\mathfrak{Z})}\dots\sqcap^{\mathfrak{F}(\mathfrak{Z})}\mathcal{F}_{m}=\bigcup\setcond{\uparrow(K_{0}\sqcap^{\mathfrak{Z}}\dots\sqcap^{\mathfrak{Z}}K_{m})}{K_{i}\in\mathcal{F}_{i}\text{ where }i=0,\dots,m}$
for $S\in\subsets\mathfrak{F}(\mathfrak{Z})\setminus\{\emptyset\}$
if $\mathfrak{Z}$ is a meet-semilattice;
\item $\mathcal{F}_{0}\sqcup^{\mathfrak{I}(\mathfrak{Z})}\dots\sqcup^{\mathfrak{I}(\mathfrak{Z})}\mathcal{F}_{m}=\bigcup\setcond{\uparrow(K_{0}\sqcup^{\mathfrak{Z}}\dots\sqcup^{\mathfrak{Z}}K_{m})}{K_{i}\in\mathcal{F}_{i}\text{ where }i=0,\dots,m}$
for $S\in\subsets\mathfrak{I}(\mathfrak{Z})\setminus\{\emptyset\}$
if $\mathfrak{Z}$ is a join-semilattice.
\end{enumerate}
\end{thm}
\begin{proof}
We will prove only the first as the second is dual.

It follows from the fact that 
\[
\mathcal{F}_{0}\sqcap^{\mathfrak{F}(\mathfrak{Z})}\dots\sqcap^{\mathfrak{F}(\mathfrak{Z})}\mathcal{F}_{m}=
\bigsqcap^{\mathfrak{F}(\mathfrak{Z})}\setcond{K_{0}\sqcap^{\mathfrak{Z}}\dots\sqcap^{\mathfrak{Z}}K_{m}}{K_{i}\in\mathcal{F}_{i}\text{ where }i=0,\dots,m}
\]
and that $\setcond{K_{0}\sqcap^{\mathfrak{Z}}\dots\sqcap^{\mathfrak{Z}}K_{m}}{K_{i}\in\mathcal{F}_{i}\text{ where }i=0,\dots,m}$
is a filter base.\end{proof}
\begin{cor}
$\up(\mathcal{F}_{0}\sqcap^{\mathfrak{F}(\mathfrak{Z})}\dots\sqcap^{\mathfrak{F}(\mathfrak{Z})}\mathcal{F}_{m})=\bigcup\setcond{\up(K_{0}\sqcap^{\mathfrak{Z}}\dots\sqcap^{\mathfrak{Z}}K_{m})}{K_{i}\in\mathcal{F}_{i}\text{ where }i=0,\dots,m}$.
\end{cor}

\subsection{Distributivity of the Lattice of Filters}
\begin{thm}
\label{f-inf-assc}The following is an implications tuple:
\begin{enumerate}
\item \label{inf-assc-p}$(\mathfrak{A};\mathfrak{Z})$ is a powerset filtrator.
\item \label{inf-assc-prim}$(\mathfrak{A};\mathfrak{Z})$ is a primary
filtrator over a distributive lattice with greatest element.
\item \label{inf-assc-conc}$\mathcal{A}\sqcup^{\mathfrak{A}}\bigsqcap^{\mathfrak{A}}S=\bigsqcap^{\mathfrak{A}}\rsupfun{\mathcal{A}\sqcup^{\mathfrak{A}}}S$
for $S\in\subsets\mathfrak{A}$ and $\mathcal{A}\in\mathfrak{A}$.
\end{enumerate}
\end{thm}
\begin{proof}
~
\begin{description}
\item [{\ref{inf-assc-p}$\Rightarrow$\ref{inf-assc-prim}}] Obvious.
\item [{\ref{inf-assc-prim}$\Rightarrow$\ref{inf-assc-conc}}] Taking
into account the previous section, we have:
\begin{align*}
\up\left(\mathcal{A}\sqcup^{\mathfrak{A}}\bigsqcap^{\mathfrak{A}}S\right) & =\\
\up\mathcal{A}\cap\up\bigsqcap^{\mathfrak{A}}S & =\\
\up\mathcal{A}\cap\setcond{K_{0}\sqcap^{\mathfrak{Z}}\dots\sqcap^{\mathfrak{Z}}K_{n}}{K_{i}\in\bigcup\rsupfun{\up}S\text{ where }i=0,\dots,n\text{ for }n\in\mathbb{N}} & =\\
\setcond{K_{0}\sqcap^{\mathfrak{Z}}\dots\sqcap^{\mathfrak{Z}}K_{n}}{K_{0}\sqcap^{\mathfrak{Z}}\dots\sqcap^{\mathfrak{Z}}K_{n}\in\up\mathcal{A},K_{i}\in\bigcup\rsupfun{\up}S\text{ where }i=0,\dots,n\text{ for }n\in\mathbb{N}} & =\\
\setcond{K_{0}\sqcap^{\mathfrak{Z}}\dots\sqcap^{\mathfrak{Z}}K_{n}}{K_{i}\in\up\mathcal{A},K_{i}\in\bigcup\rsupfun{\up}S\text{ where }i=0,\dots,n\text{ for }n\in\mathbb{N}} & =\\
\setcond{K_{0}\sqcap^{\mathfrak{Z}}\dots\sqcap^{\mathfrak{Z}}K_{n}}{K_{i}\in\up\mathcal{A}\cap\bigcup\rsupfun{\up}S\text{ where }i=0,\dots,n\text{ for }n\in\mathbb{N}} & =\\
\setcond{K_{0}\sqcap^{\mathfrak{Z}}\dots\sqcap^{\mathfrak{Z}}K_{n}}{K_{i}\in\bigcup\rsupfun{\up\mathcal{A}\cap}\rsupfun{\up}S\text{ where }i=0,\dots,n\text{ for }n\in\mathbb{N}} & =\\
\setcond{K_{0}\sqcap^{\mathfrak{Z}}\dots\sqcap^{\mathfrak{Z}}K_{n}}{K_{i}\in\bigcup\setcond{\up\mathcal{A}\cap\up\mathcal{X}}{\mathcal{X}\in S}\text{ where }i=0,\dots,n\text{ for }n\in\mathbb{N}} & =\\
\setcond{K_{0}\sqcap^{\mathfrak{Z}}\dots\sqcap^{\mathfrak{Z}}K_{n}}{K_{i}\in\bigcup\setcond{\up(\mathcal{A}\sqcup^{\mathfrak{A}}\mathcal{X})}{\mathcal{X}\in S}\text{ where }i=0,\dots,n\text{ for }n\in\mathbb{N}} & =\\
\up\bigsqcap^{\mathfrak{A}}\setcond{\mathcal{A}\sqcup^{\mathfrak{A}}\mathcal{X}}{\mathcal{X}\in S} & =\\
\up\bigsqcap^{\mathfrak{A}}\rsupfun{\mathcal{A}\sqcup^{\mathfrak{A}}}S.
\end{align*}

\end{description}
\end{proof}
\begin{cor}
\label{filt-also-distr}The following is an implications tuple:
\begin{enumerate}
\item $(\mathfrak{A};\mathfrak{Z})$ is a powerset filtrator.
\item $(\mathfrak{A};\mathfrak{Z})$ is a primary filtrator over a distributive
lattice with greatest element.
\item $\mathfrak{A}$ is a distributive and co-brouwerian lattice.
\end{enumerate}
\end{cor}

\begin{cor}\label{filt-co-frame}
The following is an implications tuple:
\begin{enumerate}
\item $(\mathfrak{A};\mathfrak{Z})$ is a powerset filtrator.
\item $(\mathfrak{A};\mathfrak{Z})$ is a primary filtrator over a distributive
lattice with greatest element.
\item $\mathfrak{A}$ is a co-frame.
\end{enumerate}
\end{cor}

The below theorem uses the notation and results from section~\ref{some-frames}.
\begin{thm}
\label{frame-main}If $\mathfrak{A}$ is a co-frame and $L$ is a
distributive lattice with greatest element, then $\operatorname{Join}(L;\mathfrak{A})$
is also a co-frame.
\end{thm}
\begin{proof}
Let $F=\uparrow\circ\bigsqcap:\Upper(\mathfrak{A})\rightarrow\Upper(\mathfrak{A})$;
$F$ is a co-nucleus by above.

Since $\Upper(\mathfrak{A})\cong\mathbf{Pos}(\mathfrak{A};2)$ by
proposition \ref{down-is-homo}, we may regard $F$ as a co-nucleus
on $\mathbf{Pos}(\mathfrak{A};2)$.

$\operatorname{Join}(L;\mathfrak{A})\cong\operatorname{Join}(L;\Fix(F))$
by corollary \ref{down-meet-co-nucleus}.

$\operatorname{Join}(L;\Fix(F))\cong\Fix(\operatorname{Join}(L;F))$
by lemma \ref{join-fix-inter}.

By corollary \ref{join-map-co-nucleus} the function $\operatorname{Join}(L;F)$
is a co-nucleus on $\operatorname{Join}\left(L;\mathbf{Pos}(\mathfrak{A};2)\right)$.
\begin{eqnarray*}
\operatorname{Join}\left(L;\mathbf{Pos}(\mathfrak{A};2)\right) & \cong & \text{(by lemma \ref{join-pos-interch})}\\
\mathbf{Pos}(\mathfrak{A};\operatorname{Join}(L;2)) & \cong\\
\mathbf{Pos}(\mathfrak{A};\mathfrak{F}(X)).
\end{eqnarray*}
$\mathfrak{F}(X)$ is a co-frame by corollary~\ref{filt-co-frame}. Thus
$\mathbf{Pos}(\mathfrak{A};\mathfrak{F}(X))$ is a co-frame by lemma~\ref{join-pos-interch}.

Thus $\operatorname{Join}(L;\mathfrak{A})$ is isomorphic to a poset
of fixed points of a co-nucleus on the co-frame $\mathbf{Pos}(\mathfrak{A};\mathfrak{F}(X))$.
By lemma \ref{fix-is-co-frame} $\operatorname{Join}(L;\mathfrak{A})$
is also a co-frame.
\end{proof}

\section{Misc filtrator properties}
\begin{thm}
\label{semifilt-joinclosed}The following is an implications tuple:
\begin{enumerate}
\item \label{semifilt-joinclosed-p}$(\mathfrak{A};\mathfrak{Z})$ is a
powerset filtrator.
\item \label{semifilt-joinclosed-f}$(\mathfrak{A};\mathfrak{Z})$ is a
primary filtrator.
\item \label{semifilt-joinclosed-sfit}$(\mathfrak{A};\mathfrak{Z})$ is
a filtered filtrator.
\item \label{semifilt-joinclosed-conc}$(\mathfrak{A};\mathfrak{Z})$ is
a filtrator with join-closed core.
\end{enumerate}
\end{thm}
\begin{proof}
~
\begin{description}
\item [{\ref{semifilt-joinclosed-p}$\Rightarrow$\ref{semifilt-joinclosed-f}}] Obvious.
\item [{\ref{semifilt-joinclosed-f}$\Rightarrow$\ref{semifilt-joinclosed-sfit}}] $\mathcal{A}$
is obviously a lower bound for $\up\mathcal{A}$. Let $\mathcal{B}$
be a lower bound for $\up\mathcal{A}$ that is
\[
\forall K\in\up\mathcal{A}:\mathcal{B}\sqsubseteq\uparrow K
\]
that is $\forall K\in\up\mathcal{A}:K\in\up\mathcal{B}$ that is $\up\mathcal{A}\subseteq\up\mathcal{B}$
that is $\mathcal{B}\sqsubseteq\mathcal{A}$. So $\mathcal{A}$ is
the greatest lower bound for $\up\mathcal{A}$.
\item [{\ref{semifilt-joinclosed-sfit}$\Rightarrow$\ref{semifilt-joinclosed-conc}}] Let
$(\mathfrak{A};\mathfrak{Z})$ be a filtered filtrator. Let $S\in\subsets\mathfrak{Z}$
and $\bigsqcup^{\mathfrak{Z}}S$ be defined. We need to prove $\bigsqcup^{\mathfrak{A}}S=\bigsqcup^{\mathfrak{Z}}S$.
That $\bigsqcup^{\mathfrak{Z}}S$ is an upper bound for $S$ is obvious.
Let $a\in\mathfrak{A}$ be an upper bound for $S$. It's enough to
prove that $\bigsqcup^{\mathfrak{Z}}S\sqsubseteq a$. Really,
\[
c\in\up a\Rightarrow c\sqsupseteq a\Rightarrow\forall x\in S:c\sqsupseteq x\Rightarrow c\sqsupseteq\bigsqcup^{\mathfrak{Z}}S\Rightarrow c\in\up\bigsqcup^{\mathfrak{Z}}S;
\]
so $\up a\subseteq\up\bigsqcup^{\mathfrak{Z}}S$ and thus $a\sqsupseteq\bigsqcup^{\mathfrak{Z}}S$
because it is filtered.
\end{description}
\end{proof}

\section{Characterization of Binarily Meet-Closed Filtrators}
\begin{thm}
\label{up-filt-crit}The following are equivalent for a filtrator
$(\mathfrak{A};\mathfrak{Z})$ whose core is a meet semilattice such
that $\forall a\in\mathfrak{A}:\up a\ne\emptyset$:
\begin{enumerate}
\item \label{mcl-crit}The filtrator is with binarily meet-closed core.
\item \label{mcl-up}$\up a$ is a filter for every $a\in\mathfrak{A}$.
\end{enumerate}
\end{thm}
\begin{proof}
~
\begin{description}
\item [{\ref{mcl-crit}$\Rightarrow$\ref{mcl-up}}] Let $X,Y\in\up a$.
Then $X\sqcap^{\mathfrak{Z}}Y=X\sqcap^{\mathfrak{A}}Y\sqsupseteq a$.
That $\up a$ is an upper set is obvious. So taking into account that
$\up a\ne\emptyset$, $\up a$ is a filter.
\item [{\ref{mcl-up}$\Rightarrow$\ref{mcl-crit}}] It is enough to prove
that $a\sqsubseteq A,B\Rightarrow a\sqsubseteq A\sqcap^{\mathfrak{Z}}B$
for every $A,B\in\mathfrak{A}$. Really:
\[
a\sqsubseteq A,B\Rightarrow A,B\in\up a\Rightarrow A\sqcap^{\mathfrak{Z}}B\in\up a\Rightarrow a\sqsubseteq A\sqcap^{\mathfrak{Z}}B.
\]

\end{description}
\end{proof}
\begin{cor}
\label{f-meet-closed}The following is an implications tuple:
\begin{enumerate}
\item \label{f-meet-closed-p}$(\mathfrak{A};\mathfrak{Z})$ is a powerset
filtrator.
\item \label{f-meet-closed-fltr}$(\mathfrak{A};\mathfrak{Z})$ is a primary
filtrator over a meet semilattice.
\item \label{f-meet-closed-conc}$(\mathfrak{A};\mathfrak{Z})$ is with
binarily meet-closed core.
\end{enumerate}
\end{cor}
\begin{proof}
~
\begin{description}
\item [{\ref{f-meet-closed-p}$\Rightarrow$\ref{f-meet-closed-fltr}}] Obvious.
\item [{\ref{f-meet-closed-fltr}$\Rightarrow$\ref{f-meet-closed-conc}}] From
the theorem.
\end{description}
\end{proof}

\subsection{Separability of Core for Primary Filtrators}
\begin{thm}
\label{when-sep-core}The following is an implications tuple:
\begin{enumerate}
\item \label{when-sep-core-p}$(\mathfrak{A};\mathfrak{Z})$ is a powerset
filtrator.
\item \label{when-sep-core-fltr}$(\mathfrak{A};\mathfrak{Z})$ is a primary
filtrator over a meet semilattice with least element.
\item \label{when-sep-core-conc}$(\mathfrak{A};\mathfrak{Z})$ is with
separable core.
\end{enumerate}
\end{thm}
\begin{proof}
~
\begin{description}
\item [{\ref{when-sep-core-p}$\Rightarrow$\ref{when-sep-core-fltr}}] Obvious.
\item [{\ref{when-sep-core-fltr}$\Rightarrow$\ref{when-sep-core-conc}}] Let
$\mathcal{A}\asymp^{\mathfrak{A}}\mathcal{B}$ where $\mathcal{A},\mathcal{B}\in\mathfrak{A}$.
\[
\up(\mathcal{A}\sqcap^{\mathfrak{A}}\mathcal{B})=\bigcup\setcond{\up(A\sqcap^{\mathfrak{Z}}B)}{A\in\up\mathcal{A},B\in\up\mathcal{B}}.
\]
So
\begin{align*}
\bot\in\up(\mathcal{A}\sqcap^{\mathfrak{A}}\mathcal{B}) & \Leftrightarrow\\
\exists A\in\up\mathcal{A},B\in\up\mathcal{B}:\bot\in\up(A\sqcap^{\mathfrak{Z}}B) & \Leftrightarrow\\
\exists A\in\up\mathcal{A},B\in\up\mathcal{B}:A\sqcap^{\mathfrak{Z}}B=\bot & \Leftrightarrow\\
\exists A\in\up\mathcal{A},B\in\up\mathcal{B}:A\sqcap^{\mathfrak{A}}B=\bot^{\mathfrak{A}}
\end{align*}
(used proposition \ref{f-meet-closed}).
\end{description}
\end{proof}

\section{Core Part}
\begin{defn}
\index{core part}The \emph{core part} of an element $a\in\mathfrak{A}$
is $\Cor a=\bigsqcap^{\mathfrak{Z}}\up a$.
\end{defn}

\begin{defn}
\index{core part!dual}The \emph{dual core part} of an element $a\in\mathfrak{A}$
is $\Cor'a=\bigsqcup^{\mathfrak{Z}}\up a$.\end{defn}
\begin{obvious}
$\Cor'$ is dual of $\Cor$.
\end{obvious}

\begin{obvious}
$\Cor a=\Cor'a=a$ for every element $a$ of the core of a filtrator.\end{obvious}
\begin{thm}
\label{cor-less}The following is an implications tuple:
\begin{enumerate}
\item \label{cor-less-p}$a$ is a filter on a set.
\item \label{cor-less-fcomp}$a$ is a filter on a complete lattice.
\item \label{cor-less-flt}$a$ is an element of a filtered filtrator and
$\Cor a$ exists.
\item \label{cor-less-conc}$\Cor a\sqsubseteq a$ and $\Cor a\in\down a$.
\end{enumerate}
\end{thm}
\begin{proof}
~
\begin{description}
\item [{\ref{cor-less-p}$\Rightarrow$\ref{cor-less-fcomp}}] Obvious.
\item [{\ref{cor-less-fcomp}$\Rightarrow$\ref{cor-less-flt}}] Theorem~\ref{semifilt-joinclosed}.
\item [{\ref{cor-less-flt}$\Rightarrow$\ref{cor-less-conc}}] $\Cor a=\bigsqcap^{\mathfrak{Z}}\up a\sqsubseteq\bigsqcap^{\mathfrak{A}}\up a=a$.
Then obviously $\Cor a\in\down a$.
\end{description}
\end{proof}
\begin{thm}
\label{f-cor-max}The following is an implications tuple:
\begin{enumerate}
\item \label{cor'a-le-p}$a$ is a filter on a set.
\item \label{cor'a-le-fltr}$a$ is a filter on a complete lattice.
\item \label{cor'a-le-a-jce}$a$ is an element $a$ of a filtrator with
join-closed core and $\Cor'a$ exists.
\item \label{cor'a-le-a}$\Cor'a\sqsubseteq a$ and $\Cor'a\in\down a$
and $\Cor'a=\max\down a$.
\end{enumerate}
\end{thm}
\begin{proof}
~
\begin{description}
\item [{\ref{cor'a-le-p}$\Rightarrow$\ref{cor'a-le-fltr}}] Obvious.
\item [{\ref{cor'a-le-fltr}$\Rightarrow$\ref{cor'a-le-a-jce}}] It is
join closed by \ref{semifilt-joinclosed}. $\Cor'a$ exists because
our filtrator is join-closed.
\item [{\ref{cor'a-le-a-jce}$\Rightarrow$\ref{cor'a-le-a}}] $\Cor'a=\bigsqcup^{\mathfrak{Z}}\down a=\bigsqcup^{\mathfrak{A}}\down a\sqsubseteq a$.
Now $\Cor'a\in\down a$ is obvious. Thus $\Cor'a=\max\down a$.
\end{description}
\end{proof}
\begin{prop}
$\Cor'a\sqsubseteq\Cor a$ whenever both $\Cor a$ and $\Cor'a$ exist
for any element $a$ of a filtrator with join-closed core.\end{prop}
\begin{proof}
$\Cor a=\bigsqcap^{\mathfrak{Z}}\up a\sqsupseteq\Cor'a$ because $\forall A\in\up a:\Cor'a\sqsubseteq A$.\end{proof}
\begin{thm}
\label{cor-eq}The following is an implications tuple:
\begin{enumerate}
\item \label{cor-eq-pow}$a$ is a filter on a set.
\item \label{cor-eq-mlat}$a$ is a filter on a complete lattice.
\item \label{cor-eq-filt}$a$ is an element of a filtered filtrator and
both $\Cor a$ and $\Cor'a$ exist.
\item \label{cor-eq-conc}$\Cor'a=\Cor a$.
\end{enumerate}
\end{thm}
\begin{proof}
~
\begin{description}
\item [{\ref{cor-eq-pow}$\Rightarrow$\ref{cor-eq-mlat}}] Obvious.
\item [{\ref{cor-eq-mlat}$\Rightarrow$\ref{cor-eq-filt}}] By theorem~\ref{semifilt-joinclosed}.
\item [{\ref{cor-eq-filt}$\Rightarrow$\ref{cor-eq-conc}}] It is with
join-closed core because it is filtered. So $\Cor'a\sqsubseteq\Cor a$.
$\Cor a\in\down a$. So $\Cor a\sqsubseteq\bigsqcup^{\mathfrak{Z}}\down a=\Cor'a$.
\end{description}
\end{proof}
\begin{cor}
\label{dcor-eq-cor}$\Cor'a=\Cor a=\bigcap a$ for every filter $a$
on a set.
\end{cor}

\section{Intersection and Joining with an Element of the Core}
\begin{lem}\label{int-join-lem}
\label{bool-compl}For a filtrator $(\mathfrak{A};\mathfrak{Z})$
where $\mathfrak{Z}$ is a boolean lattice, for every $B\in\mathfrak{Z}$,
$\mathcal{A}\in\mathfrak{A}$:
\begin{enumerate}
\item $B\asymp^{\mathfrak{A}}\mathcal{A}\Leftrightarrow\overline{B}\sqsupseteq\mathcal{A}$
if it is down-aligned, with binarily meet-closed and separable core;
\item $B\equiv^{\mathfrak{A}}\mathcal{A}\Leftrightarrow\overline{B}\sqsubseteq\mathcal{A}$
if it is up-aligned, with binarily join-closed and co-separable core.
\end{enumerate}
\end{lem}
\begin{proof}
We will prove only the first as the second is dual.
\begin{align*}
B\asymp^{\mathfrak{A}}\mathcal{A} & \Leftrightarrow\\
\exists A\in\up\mathcal{A}:B\asymp^{\mathfrak{A}}A & \Leftrightarrow\\
\exists A\in\up\mathcal{A}:B\sqcap^{\mathfrak{A}}A=\bot & \Leftrightarrow\\
\exists A\in\up\mathcal{A}:B\sqcap^{\mathfrak{Z}}A=\bot & \Leftrightarrow\\
\exists A\in\up\mathcal{A}:\overline{B}\sqsupseteq A & \Leftrightarrow\\
\overline{B}\in\up\mathcal{A} & \Leftrightarrow\\
\overline{B}\sqsupseteq\mathcal{A}.
\end{align*}

\end{proof}

\section{Stars of Elements of Filtrators}
\begin{defn}
\index{core star}\index{star!core}Let $(\mathfrak{A};\mathfrak{Z})$
be a filtrator. \emph{Core star} of an element $a$ of the filtrator
is
\[
\corestar a=\setcond{x\in\mathfrak{Z}}{x\nasymp^{\mathfrak{A}}a}.
\]
\end{defn}
\begin{prop}
$\up a\subseteq\corestar a$ for any non-least element $a$ of a filtrator.\end{prop}
\begin{proof}
For any element $X\in\mathfrak{Z}$
\[
X\in\up a\Rightarrow a\sqsubseteq X\land a\sqsubseteq a\Rightarrow X\nasymp^{\mathfrak{A}}a\Rightarrow X\in\corestar a.
\]
\end{proof}
\begin{thm}
\label{part-is-free}Let $(\mathfrak{A};\mathfrak{Z})$ be a distributive
lattice filtrator with least element and binarily join-closed core
which is a join semilattice. Then $\corestar a$ is a free star for
each $a\in\mathfrak{A}$.\end{thm}
\begin{proof}
For every $A,B\in\mathfrak{Z}$
\begin{align*}
A\sqcup^{\mathfrak{Z}}B\in\corestar a & \Leftrightarrow\\
A\sqcup^{\mathfrak{A}}B\in\corestar a & \Leftrightarrow\\
(A\sqcup^{\mathfrak{A}}B)\sqcap^{\mathfrak{A}}a\ne\bot^{\mathfrak{A}} & \Leftrightarrow\\
(A\sqcap^{\mathfrak{A}}a)\sqcup^{\mathfrak{A}}(B\sqcap^{\mathfrak{A}}a)\ne\bot^{\mathfrak{A}} & \Leftrightarrow\\
A\sqcap^{\mathfrak{A}}a\ne\bot^{\mathfrak{A}}\lor B\sqcap^{\mathfrak{A}}a\ne\bot^{\mathfrak{A}} & \Leftrightarrow\\
A\in\corestar a\lor B\in\corestar a.
\end{align*}


That $\corestar a$ doesn't contain $\bot^{\mathfrak{A}}$ is obvious.\end{proof}
\begin{defn}
\index{filtrator!star-separable}I call a filtrator \emph{star-separable}
when its core is a separation subset of its base.
\end{defn}

\section{Atomic Elements of a Filtrator}

See \cite{primeidealsandfilters,primefilters} for more detailed treatment
of ultrafilters and prime filters.
\begin{thm}
\label{atom-both}The following is an implications tuple:
\begin{enumerate}
\item \label{atom-both-p}$(\mathfrak{A};\mathfrak{Z})$ is a powerset filtrator.
\item \label{atom-both-f}$(\mathfrak{A};\mathfrak{Z})$ is a primary filtrator
over a meet-semilattice with least element.
\item \label{atom-both-mlat}$(\mathfrak{A};\mathfrak{Z})$ is a filtered
down-aligned filtrator with binarily meet-closed core $\mathfrak{Z}$
which is a meet-semilattice.
\item \label{atom-both-conc}$a$ is an atom of $\mathfrak{Z}$ iff $a\in\mathfrak{Z}$
and $a$ is an atom of $\mathfrak{A}$.
\end{enumerate}
\end{thm}
\begin{proof}
~
\begin{description}
\item [{\ref{atom-both-p}$\Rightarrow$\ref{atom-both-f}}] Obvious.
\item [{\ref{atom-both-f}$\Rightarrow$\ref{atom-both-mlat}}] It is filtered
by the theorem~\ref{semifilt-joinclosed}, binarily meet-closed by
corollary~\ref{f-meet-closed}.
\item [{\ref{atom-both-mlat}$\Rightarrow$\ref{atom-both-conc}}] ~

\begin{description}
\item [{$\Leftarrow$}] Obvious.
\item [{$\Rightarrow$}] We need to prove that if $a$ is an atom of $\mathfrak{Z}$
then $a$ is an atom of $\mathfrak{A}$. Suppose the contrary that
$a$ is not an atom of $\mathfrak{A}$. Then there exists $x\in\mathfrak{A}$
such that $\bot\ne x\sqsubset a$. Because ``$\up$'' is a straight
monotone map to the dual of the poset $\subsets\mathfrak{Z}$ (obvious
\ref{up-straight}), $\up a\subset\up x$. So there exists $K\in\up x$
such that $K\notin\up a$. Also $a\in\up x$. We have $K\sqcap^{\mathfrak{Z}}a=K\sqcap^{\mathfrak{A}}a\in\up x$;
$K\sqcap^{\mathfrak{Z}}a\ne\bot$ and $K\sqcap^{\mathfrak{Z}}a\sqsubset a$.
So $a$ is not an atom of $\mathfrak{Z}$.
\end{description}
\end{description}
\end{proof}
\begin{thm}
\label{up-eq-corestar}The following is an implications tuple:
\begin{enumerate}
\item \label{up-eq-corestar-p}$(\mathfrak{A};\mathfrak{Z})$ is a powerset
filtrator.
\item \label{up-eq-corestar-f}$(\mathfrak{A};\mathfrak{Z})$ is a primary
filtrator over a bounded meet-semilattice.
\item \label{up-eq-corestar-mlat}$(\mathfrak{A};\mathfrak{Z})$ is a filtered
down-aligned filtrator and $\mathfrak{A}$ is a meet-semilattice.
\item \label{up-eq-corestar-conc}$a\in\mathfrak{A}$ is an atom of $\mathfrak{A}$
iff $\up a=\corestar a$.
\end{enumerate}
\end{thm}
\begin{proof}
~
\begin{description}
\item [{\ref{up-eq-corestar-p}$\Rightarrow$\ref{up-eq-corestar-f}}] Obvious.
\item [{\ref{up-eq-corestar-f}$\Rightarrow$\ref{up-eq-corestar-mlat}}] It
is filtered by the theorem~\ref{semifilt-joinclosed}, $\mathfrak{A}$
is a meet-semilattice by the corollary~\ref{filt-is-complete}.
\item [{\ref{up-eq-corestar-mlat}$\Rightarrow$\ref{up-eq-corestar-conc}}] ~

\begin{description}
\item [{$\Rightarrow$}] Let $a$ be an atom of $\mathfrak{A}$. $\up a\supseteq\corestar a$
because $a\ne\bot$. $\up a\subseteq\corestar a$ because for any
$K\in\mathfrak{A}$
\[
K\in\up a\Leftrightarrow K\sqsupseteq a\Leftrightarrow K\sqcap^{\mathfrak{A}}a\ne\bot\Leftrightarrow K\in\corestar a.
\]

\item [{$\Leftarrow$}] Let $\up a=\corestar a$. Then $a\ne\bot$. Consequently
for every $x\in\mathfrak{A}$ we have
\begin{align*}
\bot\sqsubset x\sqsubset a & \Rightarrow\\
x\sqcap^{\mathfrak{A}}a\ne\bot & \Rightarrow\\
\forall K\in\up x:K\in\corestar a & \Rightarrow\\
\forall K\in\up x:K\in\up a & \Rightarrow\\
\up x\subseteq\up a & \Rightarrow\\
x\sqsupseteq a.
\end{align*}



So $a$ is an atom of $\mathfrak{A}$.

\end{description}
\end{description}
\end{proof}
\begin{prop}
\label{coat}The following is an implications tuple:
\begin{enumerate}
\item \label{coat-p}$(\mathfrak{A};\mathfrak{Z})$ is a powerset filtrator.
\item \label{coat-fltr}$(\mathfrak{A};\mathfrak{Z})$ is a primary filtrator.
\item \label{coat-conc}Coatoms of $\mathfrak{A}$ are exactly coatoms of
$\mathfrak{Z}$.
\end{enumerate}
\end{prop}
\begin{proof}
~
\begin{description}
\item [{\ref{coat-p}$\Rightarrow$\ref{coat-fltr}}] Obvious.
\item [{\ref{coat-fltr}$\Rightarrow$\ref{coat-conc}}] Suppose $a$ is
a coatom of $\mathfrak{Z}$. Then $a$ is the only non-greatest element
in $\up a$. Suppose $b\sqsupset a$ for some $b\in\mathfrak{A}$.
Then $a$ cannot be in $\up b$ and thus the only possible element
of $\up b$ is the greatest element of $\mathfrak{Z}$ (if it exists)
from what follows $b=\top^{\mathfrak{A}}$. So $a$ is a coatom of
$\mathfrak{A}$.


Suppose now that $a$ is a coatom of $\mathfrak{A}$. To finish the
proof it is enough to show that $a$ is principal. (Then $a$ is non-greatest
and thus is a coatom of $\mathfrak{Z}$.)


Suppose $a$ is non-principal. Then obviously exist two distinct elements
$x$ and $y$ of the core such that $x,y\in\up a$. Thus $a$ is not
an atom of $\mathfrak{A}$.

\end{description}
\end{proof}
\begin{cor}
Atoms of the set of filters on a set~$U$ are exactly sets $U\setminus\{x\}$
where $x\in U$.\end{cor}
\begin{prop}
\label{coat-ic}The following is an implications tuple:
\begin{enumerate}
\item \label{coat-ic-p}$(\mathfrak{A};\mathfrak{Z})$ is a powerset filtrator.
\item \label{coat-ic-fltr}$(\mathfrak{A};\mathfrak{Z})$ is a primary filtrator
over a coatomic poset.
\item \label{coat-ic-conc}$\mathfrak{A}$ is coatomic.
\end{enumerate}
\end{prop}
\begin{proof}
~
\begin{widedisorder}
\item [{\ref{coat-ic-p}$\Rightarrow$\ref{coat-ic-fltr}}] Obvious.
\item [{\ref{coat-ic-fltr}$\Rightarrow$\ref{coat-ic-fltr}}] Suppose
$\mathcal{A}\in\mathfrak{A}$ and $\mathcal{A}\neq\top^{\mathfrak{A}}$.
Then there exists $A\in\up\mathcal{A}$ such that $A$ is not greatest
element of $\mathfrak{Z}$. Consequently there exists a coatom $a\in\mathfrak{Z}$
such that $a\sqsupseteq A$. Thus $a\in\up\mathcal{A}$ and $a$ is
not greatest.
\end{widedisorder}
\end{proof}

\section{Prime Filtrator Elements}
\begin{defn}
\index{prime element}Let $(\mathfrak{A};\mathfrak{Z})$ be a down-aligned
filtrator. \emph{Prime} filtrator elements are such $a\in\mathfrak{A}$
that $\up a$ is a free star.\end{defn}
\begin{prop}
\label{atom-is-prime}The following is an implications tuple:
\begin{enumerate}
\item \label{atom-is-prime-p}$(\mathfrak{A};\mathfrak{Z})$ is a powerset
filtrator.
\item \label{atom-is-prime-f}$(\mathfrak{A};\mathfrak{Z})$ is a primary
filtrator over a bounded distributive lattice.
\item \label{atom-is-prime-jlat}$(\mathfrak{A};\mathfrak{Z})$ is a down-aligned
filtrator with binarily join-closed core, where $\mathfrak{A}$ is
a starrish join-semilattice and $\mathfrak{Z}$ is a join-semilattice.
\item \label{atom-is-prime-conc}Atomic elements of this filtrator are prime.
\end{enumerate}
\end{prop}
\begin{proof}
~
\begin{description}
\item [{\ref{atom-is-prime-p}$\Rightarrow$\ref{atom-is-prime-f}}] Obvious.
\item [{\ref{atom-is-prime-f}$\Rightarrow$\ref{atom-is-prime-jlat}}] $(\mathfrak{A};\mathfrak{Z})$
is with binarily join-closed core by the theorem \ref{semifilt-joinclosed},
$\mathfrak{A}$ is a distributive lattice by theorem \ref{filt-also-distr}.
\item [{\ref{atom-is-prime-jlat}$\Rightarrow$\ref{atom-is-prime-conc}}] Let
$a$ be an atom of the lattice $\mathfrak{A}$. We have for every
$X,Y\in\mathfrak{Z}$
\begin{align*}
X\sqcup^{\mathfrak{Z}}Y\in\up a & \Leftrightarrow\\
X\sqcup^{\mathfrak{A}}Y\in\up a & \Leftrightarrow\\
X\sqcup^{\mathfrak{A}}Y\sqsupseteq a & \Leftrightarrow\\
X\sqcup^{\mathfrak{A}}Y\nasymp^{\mathfrak{A}}a & \Leftrightarrow\\
X\nasymp^{\mathfrak{A}}a\lor Y\nasymp^{\mathfrak{A}}a & \Leftrightarrow\\
X\sqsupseteq a\lor Y\sqsupseteq a & \Leftrightarrow\\
X\in\up a\lor Y\in\up a.
\end{align*}

\end{description}
\end{proof}
The following theorem is essentially borrowed from \cite{stone-spaces}:
\begin{thm}
\label{f-prime-crit}The following is an implications tuple:
\begin{enumerate}
\item \label{prim-crit-p}$(\mathfrak{A};\mathfrak{Z})$ is a powerset filtrator.
\item \label{prim-crit-fltr}$(\mathfrak{A};\mathfrak{Z})$ is a primary
filtrator over a boolean lattice.
\item \label{prim-crit-conc}Let $a\in\mathfrak{A}$. Then the following
are equivalent:

\begin{enumerate}
\item \label{prim-prim}$a$ is prime.
\item \label{a-not-a}For every $A\in\mathfrak{Z}$ exactly one of $\{A,\overline{A}\}$
is in $\up a$.
\item \label{prim-atom}$a$ is an atom of $\mathfrak{A}$.
\end{enumerate}
\end{enumerate}
\end{thm}
\begin{proof}
~
\begin{description}
\item [{\ref{prim-crit-p}$\Rightarrow$\ref{prim-crit-fltr}}] Obvious.
\item [{\ref{prim-crit-fltr}$\Rightarrow$\ref{prim-crit-conc}}] ~

\begin{description}
\item [{\ref{prim-prim}$\Rightarrow$\ref{a-not-a}}] Let $a$ be prime.
Then $A\sqcup^{\mathfrak{Z}}\overline{A}=\top^{\mathfrak{A}}\in\up a$.
Therefore $A\in\up a\lor\overline{A}\in\up a$. But since $A\sqcap^{\mathfrak{Z}}\overline{A}=\bot^{\mathfrak{Z}}$
it is impossible $A\in\up a\land\overline{A}\in\up a$.
\item [{\ref{a-not-a}$\Rightarrow$\ref{prim-atom}}] Obviously $a\ne\bot^{\mathfrak{A}}$.
Let a filter $b\sqsubset a$. So $\up b\supset\up a$. Let $X\in\up(b\setminus a)$.
Then $X\notin\up a$ and thus $\overline{X}\in\up a$ and consequently
$\overline{X}\in\up b$. So $\bot^{\mathfrak{Z}}=X\sqcap^{\mathfrak{Z}}\overline{X}\in\up b$
and thus $b=\bot^{\mathfrak{A}}$. So $a$ is atomic.
\item [{\ref{prim-atom}$\Rightarrow$\ref{prim-prim}}] By the previous
proposition.
\end{description}
\end{description}
\end{proof}

\section{Stars for filters}
\begin{thm}
\label{da-is-free-star}The following is an implications tuple:
\begin{enumerate}
\item \label{da-is-free-star-p}$(\mathfrak{A};\mathfrak{Z})$ is a powerset
filtrator.
\item \label{da-is-free-star-fltr}$(\mathfrak{A};\mathfrak{Z})$ is a primary
filtrator over a bounded distributive lattice.
\item \label{da-is-free-star-conc}$\corestar a$ is a free star for each
$a\in\mathfrak{A}$.
\end{enumerate}
\end{thm}
\begin{proof}
~
\begin{description}
\item [{\ref{da-is-free-star-p}$\Rightarrow$\ref{da-is-free-star-fltr}}] Obvious.
\item [{\ref{da-is-free-star-fltr}$\Rightarrow$\ref{da-is-free-star-conc}}] $\mathfrak{A}$
is a distributive lattice by the corollary \ref{filt-also-distr}.
The filtrator $(\mathfrak{A};\mathfrak{Z})$ is binarily join-closed
by corollary \ref{semifilt-joinclosed}. So we can apply the theorem
\ref{part-is-free}.
\end{description}
\end{proof}

\subsection{Stars of Filters on Boolean Lattices}

In this section we will consider the set of filters $\mathfrak{A}$
on a boolean lattice $\mathfrak{Z}$.
\begin{thm}
\label{f-simpl-star-dual}The following is an implications tuple:
\begin{enumerate}
\item \label{simpl-star-dual-p}$(\mathfrak{A};\mathfrak{Z})$ is a powerset
filtrator.
\item \label{simpl-star-dual-fltr}$(\mathfrak{A};\mathfrak{Z})$ is a primary
filtrator over a boolean lattice.
\item \label{simpl-star-dual-conc}$\corestar\mathcal{A}=\lnot\rsupfun{\lnot}\up\mathcal{A}=\rsupfun{\lnot}\lnot\up\mathcal{A}$
and $\up\mathcal{A}=\lnot\rsupfun{\lnot}\corestar\mathcal{A}=\rsupfun{\lnot}\lnot\corestar\mathcal{A}$.
\end{enumerate}
\end{thm}
\begin{proof}
~
\begin{description}
\item [{\ref{simpl-star-dual-p}$\Rightarrow$\ref{simpl-star-dual-fltr}}] Obvious.
\item [{\ref{simpl-star-dual-fltr}$\Rightarrow$\ref{simpl-star-dual-conc}}] Because
of properties of diagram~(\ref{two-diags}), it is enough to prove
just $\corestar\mathcal{A}=\lnot\rsupfun{\lnot}\up\mathcal{A}$. Really,
$X\in\up\mathcal{A}\Leftrightarrow X\sqsupseteq\mathcal{A}\Leftrightarrow\overline{X}\asymp^{\mathfrak{A}}\mathcal{A}\Leftrightarrow\overline{X}\notin\corestar\mathcal{A}$
for any $X\in\mathfrak{Z}$ (taking into account theorems \ref{up-filt-crit},
\ref{when-sep-core}, \ref{bool-compl}).
\end{description}
\end{proof}
\begin{cor}
\label{d-inj}The following is an implications tuple:
\begin{enumerate}
\item $(\mathfrak{A};\mathfrak{Z})$ is a powerset filtrator.
\item $(\mathfrak{A};\mathfrak{Z})$ is a primary filtrator over a boolean
lattice.
\item $\corestar$ is an order isomorphism from $\mathfrak{A}$ to $\mathfrak{S}(\mathfrak{Z})$.
\end{enumerate}
\end{cor}
\begin{proof}
By properties of the diagram~(\ref{two-diags}).\end{proof}
\begin{cor}
\label{d-f-join}The following is an implications tuple:
\begin{enumerate}
\item \label{d-f-join-p}$(\mathfrak{A};\mathfrak{Z})$ is a powerset filtrator.
\item \label{d-f-join-fltr}$(\mathfrak{A};\mathfrak{Z})$ is a primary
filtrator over a boolean lattice.
\item \label{d-f-join-conc}$\corestar\bigsqcup^{\mathfrak{A}}S=\bigcup\rsupfun{\corestar}S$
for every $S\in\subsets\mathfrak{A}$.
\end{enumerate}
\end{cor}
\begin{proof}
~
\begin{description}
\item [{\ref{d-f-join-p}$\Rightarrow$\ref{d-f-join-fltr}}] Obvious.
\item [{\ref{d-f-join-fltr}$\Rightarrow$\ref{d-f-join-conc}}] $\corestar\bigsqcup^{\mathfrak{A}}S=\bigsqcup^{\mathfrak{S}(\mathfrak{Z})}\rsupfun{\corestar}S=\bigcup\rsupfun{\corestar}S$.
\end{description}
\end{proof}

\section{Generalized Filter Base}
\begin{defn}
\index{filter base!generalized}\emph{Generalized filter base} is
a filter base on the set $\mathfrak{A}$ where $(\mathfrak{A};\mathfrak{Z})$
is a primary filtrator.
\end{defn}

\begin{defn}
If $S$ is a generalized filter base and $\mathcal{A}=\bigsqcap^{\mathfrak{A}}S$
for some $\mathcal{A}\in\mathfrak{A}$, then we call $S$ a generalized
filter base of~$\mathcal{A}$.\end{defn}
\begin{thm}
\label{genbase-main}The following is an implications tuple:
\begin{enumerate}
\item \label{genbase-main-p}$(\mathfrak{A};\mathfrak{Z})$ is a powerset
filtrator.
\item \label{genbase-main-fltr}$(\mathfrak{A};\mathfrak{Z})$ is a primary
filtrator over a meet-semilattice.
\item \label{genbase-main-conc}For a generalized filter base $S$ of $\mathcal{F}\in\mathfrak{A}$
and $K\in\mathfrak{Z}$ we have 
\[
K\in\up\mathcal{F}\Leftrightarrow\exists\mathcal{L}\in S:K\in\up\mathcal{L}.
\]

\end{enumerate}
\end{thm}
\begin{proof}
~
\begin{description}
\item [{\ref{genbase-main-p}$\Rightarrow$\ref{genbase-main-fltr}}] Obvious.
\item [{\ref{genbase-main-fltr}$\Rightarrow$\ref{genbase-main-conc}}] ~

\begin{description}
\item [{$\Leftarrow$}] Because $\mathcal{F}=\bigsqcap^{\mathfrak{A}}S$.
\item [{$\Rightarrow$}] Let $K\in\up\mathcal{F}$. Then (taken into account
corollary~\ref{meet-filtx} and that $S$ is nonempty) there exist
$X_{1},\dots,X_{n}\in\bigcup\rsupfun{\up}S$ such that $K\in\up(X_{1}\sqcap^{\mathfrak{Z}}\dots\sqcap^{\mathfrak{Z}}X_{n})$
that is $K\in\up(\uparrow X_{1}\sqcap^{\mathfrak{Z}}\dots\sqcap^{\mathfrak{Z}}\uparrow X_{n})$.
Consequently (by theorem \ref{up-filt-crit}) $K\in\up(\uparrow X_{1}\sqcap^{\mathfrak{A}}\dots\sqcap^{\mathfrak{A}}\uparrow X_{n})$.
Replacing every $\uparrow X_{i}$ with such $\mathcal{X}_{i}\in S$
that $X_{i}\in\up\mathcal{X}_{i}$ (this is obviously possible to
do), we get a finite set $T_{0}\subseteq S$ such that $K\in\up\bigsqcap^{\mathfrak{A}}T_{0}$.
From this there exists $\mathcal{C}\in S$ such that $\mathcal{C}\sqsubseteq\bigsqcap^{\mathfrak{A}}T_{0}$
and so $K\in\up\mathcal{C}$.
\end{description}
\end{description}
\end{proof}
\begin{cor}
\label{genbase-corr}The following is an implications tuple:
\begin{enumerate}
\item $(\mathfrak{A};\mathfrak{Z})$ is a powerset filtrator.
\item $(\mathfrak{A};\mathfrak{Z})$ is a primary filtrator over a meet-semilattice
with least element.
\item For a generalized filter base $S$ of a $\mathcal{F}\in\mathfrak{A}$
we have
\[
\bot^{\mathfrak{A}}\in S\Leftrightarrow\mathcal{F}=\bot^{\mathfrak{A}}.
\]

\end{enumerate}
\end{cor}
\begin{proof}
Substitute $\bot^{\mathfrak{A}}$ as $K$.\end{proof}
\begin{thm}
\label{genbase-f-closed}The following is an implications tuple:
\begin{enumerate}
\item $(\mathfrak{A};\mathfrak{Z})$ is a powerset filtrator.
\item $(\mathfrak{A};\mathfrak{Z})$ is a primary filtrator over a meet-semilattice
with least element.
\item Let $\mathcal{F}_{0}\sqcap^{\mathfrak{A}}\dots\sqcap^{\mathfrak{A}}\mathcal{F}_{n}\ne\bot^{\mathfrak{A}}$
for every $\mathcal{F}_{0},\dots,\mathcal{F}_{n}\in S$, where $S$
is a nonempty set of elements of $\mathfrak{A}$. Then $\bigsqcap^{\mathfrak{A}}S\ne\bot^{\mathfrak{A}}$.
\end{enumerate}
\end{thm}
\begin{proof}
Consider the set
\[
S'=\setcond{\mathcal{F}_{0}\sqcap^{\mathfrak{A}}\dots\sqcap^{\mathfrak{A}}\mathcal{F}_{n}}{\mathcal{F}_{0},\dots,\mathcal{F}_{n}\in S}.
\]


Obviously $S'$ is nonempty and binarily meet-closed. So $S'$ is
a generalized filter base. Obviously $\bot^{\mathfrak{A}}\notin S$.
So by properties of generalized filter bases $\bigsqcap^{\mathfrak{A}}S'\ne\bot^{\mathfrak{A}}$.
But obviously $\bigsqcap^{\mathfrak{A}}S=\bigsqcap^{\mathfrak{A}}S'$.
So $\bigsqcap^{\mathfrak{A}}S\ne\bot^{\mathfrak{A}}$.\end{proof}
\begin{cor}
\label{princ-fbase}The following is an implications tuple:
\begin{enumerate}
\item \label{princ-fbase-p}$(\mathfrak{A};\mathfrak{Z})$ is a powerset
filtrator.
\item \label{princ-fbase-fltr}$(\mathfrak{A};\mathfrak{Z})$ is a primary
filtrator over a meet-semilattice with least element.
\item \label{princ-fbase-conc}Let $S\in\subsets\mathfrak{Z}$ such that
$S\ne\emptyset$ and $A_{0}\sqcap^{\mathfrak{Z}}\dots\sqcap^{\mathfrak{Z}}A_{n}\ne\bot^{\mathfrak{Z}}$
for every $A_{0},\dots,A_{n}\in S$. Then $\bigsqcap^{\mathfrak{A}}S\ne\bot^{\mathfrak{A}}$.
\end{enumerate}
\end{cor}
\begin{proof}
~
\begin{description}
\item [{\ref{princ-fbase-p}$\Rightarrow$\ref{princ-fbase-fltr}}] Obvious.
\item [{\ref{princ-fbase-fltr}$\Rightarrow$\ref{princ-fbase-conc}}] Because
$(\mathfrak{A};\mathfrak{Z})$ is binarily meet-closed (by the theorem
\ref{up-filt-crit}).
\end{description}
\end{proof}
\begin{thm}
\label{filt-atomic}The following is an implications tuple:
\begin{enumerate}
\item \label{filt-atomic-p}$(\mathfrak{A};\mathfrak{Z})$ is a powerset
filtrator.
\item \label{filt-atomic-fltr}$(\mathfrak{A};\mathfrak{Z})$ is a primary
filtrator over a bounded meet-semilattice.
\item \label{filt-atomic-conc}$\mathfrak{A}$ is an atomic lattice.
\end{enumerate}
\end{thm}
\begin{proof}
~
\begin{description}
\item [{\ref{filt-atomic-p}$\Rightarrow$\ref{filt-atomic-fltr}}] Obvious.
\item [{\ref{filt-atomic-fltr}$\Rightarrow$\ref{filt-atomic-conc}}] Let
$\mathcal{F}\in\mathfrak{A}$. Let choose (by Kuratowski's lemma)
a maximal chain $S$ from $\bot^{\mathfrak{A}}$ to $\mathcal{F}$.
Let $S'=S\setminus\{\bot^{\mathfrak{A}}\}$. $a=\bigsqcap^{\mathfrak{A}}S'\ne\bot^{\mathfrak{A}}$
by properties of generalized filter bases (the corollary \ref{genbase-corr}
which uses the fact that $\mathfrak{Z}$ is a meet-semilattice with
least element). If $a\notin S$ then the chain $S$ can be extended
adding there element $a$ because $\bot^{\mathfrak{A}}\sqsubset a\sqsubseteq\mathcal{X}$
for any $\mathcal{X}\in S'$ what contradicts to maximality of the
chain. So $a\in S$ and consequently $a\in S'$. Obviously $a$ is
the minimal element of $S'$. Consequently (taking into account maximality
of the chain) there is no $\mathcal{Y}\in\mathfrak{A}$ such that
$\bot^{\mathfrak{A}}\sqsubset\mathcal{Y}\sqsubset a$. So $a$ is
an atomic filter. Obviously $a\sqsubseteq\mathcal{F}$.
\end{description}
\end{proof}
\begin{defn}
A complete lattice is \emph{co-compact} iff $\bigsqcap S=\bot$ for
a set $S$ of elements of this lattice implies that there is its finite
subset $T\subseteq S$ such that $\bigsqcap T=\bot$.\end{defn}
\begin{thm}
The following is an implications tuple:
\begin{enumerate}
\item \label{cmpct-p}$(\mathfrak{A};\mathfrak{Z})$ is a powerset filtrator.
\item \label{cmpct-filt}$(\mathfrak{A};\mathfrak{Z})$ is a primary filtrator
over a bounded meet-semilattice.
\item \label{cmpct-conc}$\mathfrak{A}$ is co-compact.
\end{enumerate}
\end{thm}
\begin{proof}
~
\begin{description}
\item [{\ref{cmpct-p}$\Rightarrow$\ref{cmpct-filt}}] Obvious.
\item [{\ref{cmpct-filt}$\Rightarrow$\ref{cmpct-conc}}] Poset~$\mathfrak{A}$
is complete by corollary~\ref{filt-is-complete}.


If $\bot\in\bigsqcap^{\mathfrak{A}}S$ then there are $K_{i}\in\up\bigcup S$
such that $\bot\in\up(K_{0}\sqcap^{\mathfrak{Z}}\ldots\sqcap^{\mathfrak{Z}}K_{n})$
that is $K_{0}\sqcap^{\mathfrak{Z}}\ldots\sqcap^{\mathfrak{Z}}K_{n}=\bot$
from which easily follows $\mathcal{F}_{0}\sqcap^{\mathfrak{A}}\ldots\sqcap^{\mathfrak{A}}\mathcal{F}_{n}=\bot$
for some $\mathcal{F}_{i}\in S$.

\end{description}
\end{proof}

\section{Separability of filters}
\begin{prop}
\label{filt-is-sep}The following is an implications tuple:
\begin{enumerate}
\item \label{filt-is-sep-p}$(\mathfrak{A};\mathfrak{Z})$ is a powerset
filtrator.
\item \label{filt-is-sep-fltr}$(\mathfrak{A};\mathfrak{Z})$ is a primary
filtrator over a boolean lattice.
\item \label{filt-is-sep-conc}$\mathfrak{A}$ is separable.
\end{enumerate}
\end{prop}
\begin{proof}
~
\begin{description}
\item [{\ref{filt-is-sep-p}$\Rightarrow$\ref{filt-is-sep-fltr}}] Obvious.
\item [{\ref{filt-is-sep-fltr}$\Rightarrow$\ref{filt-is-sep-conc}}] By
properties of stars of filters.
\end{description}
\end{proof}
\begin{thm}
\label{filt-atomistic}The following is an implications tuple:
\begin{enumerate}
\item \label{filt-atomistic-p}$(\mathfrak{A};\mathfrak{Z})$ is a powerset
filtrator.
\item \label{filt-atomistic-fltr}$(\mathfrak{A};\mathfrak{Z})$ is a primary
filtrator over a boolean lattice.
\item \label{filt-atomistic-conc}$\mathfrak{A}$ is an atomistic poset.
\end{enumerate}
\end{thm}
\begin{proof}
~
\begin{description}
\item [{\ref{filt-atomistic-p}$\Rightarrow$\ref{filt-atomistic-fltr}}] Obvious.
\item [{\ref{filt-atomistic-fltr}$\Rightarrow$\ref{filt-atomistic-conc}}] Because
(used theorem \ref{atomistic-enough}) $\mathfrak{A}$ is atomic
(theorem \ref{filt-atomic}) and separable.
\end{description}
\end{proof}
\begin{cor}
\label{f-atom-sep}The following is an implications tuple:
\begin{enumerate}
\item $(\mathfrak{A};\mathfrak{Z})$ is a powerset filtrator.
\item $(\mathfrak{A};\mathfrak{Z})$ is a primary filtrator over a boolean
lattice.
\item $\mathfrak{A}$ is atomically separable.
\end{enumerate}
\end{cor}
\begin{proof}
By theorem \ref{atms-is-asep}.
\end{proof}

\section{Some Criteria}
\begin{thm}
\label{crit1}The following is an implications tuple:
\begin{enumerate}
\item \label{crit1-p}$(\mathfrak{A};\mathfrak{Z})$ is a powerset filtrator.
\item \label{crit1-f}$(\mathfrak{A};\mathfrak{Z})$ is a primary filtrator
over a complete boolean lattice.
\item \label{crit1-fltr}$(\mathfrak{A};\mathfrak{Z})$ is a down-aligned,
with join-closed, binarily meet-closed and separable core which is
a complete boolean lattice.
\item \label{crit1-conc}The following conditions are equivalent for any
$\mathcal{F}\in\mathfrak{A}$:

\begin{enumerate}
\item \label{crit1-core}$\mathcal{F}\in\mathfrak{Z}$;
\item \label{crit1-flt}$\forall S\in\subsets\mathfrak{A}:\left(\mathcal{F}\sqcap^{\mathfrak{A}}\bigsqcup^{\mathfrak{A}}S\ne\bot\Rightarrow\exists\mathcal{K}\in S:\mathcal{F}\sqcap^{\mathfrak{A}}\mathcal{K}\ne\bot\right)$;
\item \label{crit1-princ}$\forall S\in\subsets\mathfrak{Z}:\left(\mathcal{F}\sqcap^{\mathfrak{A}}\bigsqcup^{\mathfrak{A}}S\ne\bot\Rightarrow\exists K\in S:\mathcal{F}\sqcap^{\mathfrak{A}}K\ne\bot\right)$.
\end{enumerate}
\end{enumerate}
\end{thm}
\begin{proof}
~
\begin{description}
\item [{\ref{crit1-p}$\Rightarrow$\ref{crit1-f}}] Obvious.
\item [{\ref{crit1-f}$\Rightarrow$\ref{crit1-fltr}}] The filtrator $(\mathfrak{A};\mathfrak{Z})$
is with with join-closed core by theorem \ref{semifilt-joinclosed},
binarily meet-closed core by corollary~\ref{f-meet-closed}, with
separable core by theorem \ref{when-sep-core}.
\item [{\ref{crit1-fltr}$\Rightarrow$\ref{crit1-conc}}] ~

\begin{description}
\item [{\ref{crit1-core}$\Rightarrow$\ref{crit1-flt}}] Let $\mathcal{F}\in\mathfrak{Z}$.
Then (taking into account the lemma~\ref{bool-compl})
\[
\mathcal{F}\sqcap^{\mathfrak{A}}\bigsqcup^{\mathfrak{A}}S\ne\bot\Leftrightarrow\overline{\mathcal{F}}\nsqsupseteq\bigsqcup^{\mathfrak{A}}S\Rightarrow\exists\mathcal{K}\in S:\overline{\mathcal{F}}\nsqsupseteq\mathcal{K}\Leftrightarrow\exists\mathcal{K}\in S:\mathcal{F}\sqcap^{\mathfrak{A}}\mathcal{K}\ne\bot.
\]

\item [{\ref{crit1-flt}$\Rightarrow$\ref{crit1-princ}}] Obvious.
\item [{\ref{crit1-princ}$\Rightarrow$\ref{crit1-core}}] ~
\begin{align*}
\forall S\in\subsets\mathfrak{Z}:\left(\mathcal{F}\sqcap^{\mathfrak{A}}\bigsqcup^{\mathfrak{A}}S\ne\bot\Rightarrow\exists K\in S:\mathcal{F}\sqcap^{\mathfrak{A}}K\ne\bot\right) & \Leftrightarrow\\
\forall S\in\subsets\mathfrak{Z}:\left(\mathcal{F}\nasymp^{\mathfrak{A}}\bigsqcup^{\mathfrak{Z}}S\Rightarrow\exists K\in S:\mathcal{F}\nasymp^{\mathfrak{A}}K\right) & \Leftrightarrow\text{ (lemma \ref{bool-compl})}\\
\forall S\in\subsets\mathfrak{Z}:\left(\overline{\bigsqcup^{\mathfrak{Z}}S}\nsqsupseteq\mathcal{F}\Rightarrow\exists K\in S:\overline{K}\nsqsupseteq\mathcal{F}\right) & \Leftrightarrow\\
\forall S\in\subsets\mathfrak{Z}:\left(\forall K\in S:\overline{K}\sqsupseteq\mathcal{F}\Rightarrow\overline{\bigsqcup^{\mathfrak{Z}}S}\sqsupseteq\mathcal{F}\right) & \Leftrightarrow\\
\forall S\in\subsets\mathfrak{Z}:\left(\forall K\in S:\overline{K}\sqsupseteq\mathcal{F}\Rightarrow\bigsqcap^{\mathfrak{Z}}\rsupfun{\lnot}S\sqsupseteq\mathcal{F}\right) & \Leftrightarrow\\
\forall S\in\subsets\mathfrak{Z}:\left(\forall K\in S:K\sqsupseteq\mathcal{F}\Rightarrow\bigsqcap^{\mathfrak{Z}}S\sqsupseteq\mathcal{F}\right) & \Rightarrow\\
\bigsqcap^{\mathfrak{Z}}\up\mathcal{F}\sqsupseteq\mathcal{F} & \Leftrightarrow\\
\bigsqcap^{\mathfrak{Z}}\up\mathcal{F}\in\up\mathcal{F} & \Rightarrow\\
\mathcal{F}\in\mathfrak{Z}.
\end{align*}

\end{description}
\end{description}
\end{proof}
\begin{rem}
The above theorem strengthens theorem 53 in \cite{filters}. Both
the formulation of the theorem and the proof are considerably simplified.\end{rem}
\begin{defn}
\index{filter base!generated by}Let $S$ be a subset of a meet-semilattice.
The \emph{filter base generated by} $S$ is the set
\[
[S]_{\sqcap}=\setcond{a_{0}\sqcap\dots\sqcap a_{n}}{a_{i}\in S,n=0,1,\dots}.
\]
\end{defn}
\begin{lem}
The set of all finite subsets of an infinite set $A$ has the same
cardinality as $A$.\end{lem}
\begin{proof}
Let denote the number of $n$-element subsets of $A$ as $s_{n}$.
Obviously $s_{n}\le\card A^{n}=\card A$. Then the number $S$ of
all finite subsets of $A$ is equal to
\[
s_{0}+s_{1}+\dots\le\card A+\card A+\dots=\card A.
\]


That $S\ge\card A$ is obvious. So $S=\card A$.\end{proof}
\begin{lem}
A filter base generated by an infinite set has the same cardinality
as that set.\end{lem}
\begin{proof}
From the previous lemma.\end{proof}
\begin{defn}
\index{filter-closed}Let $\mathfrak{A}$ be a complete lattice. A
set $S\in\subsets\mathfrak{A}$ is \emph{filter-closed} when for every
filter base $T\in\subsets S$ we have $\bigsqcap T\in S$.\end{defn}
\begin{thm}
A subset $S$ of a complete lattice is filter-closed iff for every
nonempty chain $T\in\subsets S$ we have $\bigsqcap T\in S$.\end{thm}
\begin{proof}
(proof sketch by \noun{Joel David Hamkins})
\begin{description}
\item [{$\Rightarrow$}] Because every nonempty chain is a filter base.
\item [{$\Leftarrow$}] We will assume that cardinality of a set is an
ordinal defined by von Neumann cardinal assignment (what is a standard
practice in ZFC). Recall that $\alpha<\beta\Leftrightarrow\alpha\in\beta$
for ordinals $\alpha$,~$\beta$.


We will take it as given that for every nonempty chain $T\in\subsets S$
we have $\bigsqcap T\in S$.


We will prove the following statement: If $\card S=n$ then $S$ is
filter closed, for any cardinal~$n$.


Instead we will prove it not only for cardinals but for wider class
of ordinals: If $\card S=n$ then $S$ is filter closed, for any ordinal
$n$.


We will prove it using transfinite induction by~$n$.


For finite $n$ we have $\bigsqcap T\in S$ because $T\subseteq S$
has minimal element.


Let $\card T=n$ be an infinite ordinal.


Let the assumption hold for every $m\in\card T$.


We can assign $T=\setcond{a_{\alpha}}{\alpha\in\card T}$ for some
$a_{\alpha}$ because $\card\card T=\card T$.


Consider $\beta\in\card T$.


Let $P_{\beta}=\setcond{a_{\alpha}}{\alpha\in\beta}$. Let $b_{\beta}=\bigsqcap P_{\beta}$.
Obviously $b_{\beta}=\bigsqcap[P_{\beta}]_{\sqcap}$. We have
\[
\card[P_{\beta}]_{\sqcap}=\card P_{\beta}=\card\beta<\card T
\]
(used the lemma and von Neumann cardinal assignment). By the assumption
of induction $b_{\beta}\in S$.


$\forall\beta\in\card T:P_{\beta}\subseteq T$ and thus $b_{\beta}\sqsupseteq\bigsqcap T$.


It is easy to see that the set $\setcond{P_{\beta}}{\beta\in\card T}$
is a chain. Consequently $\setcond{b_{\beta}}{\beta\in\card T}$ is
a chain.


By the theorem conditions $b=\bigsqcap_{\beta\in\card T}b_{\beta}\in S$
(taken into account that $b_{\beta}\in S$ by the assumption of induction).


Obviously $b\sqsupseteq\bigsqcap T$.


$b\sqsubseteq b_{\beta}$ and so $\forall\beta\in\card T,\alpha\in\beta:b\sqsubseteq a_{\alpha}$.
Let $\alpha\in\card T$. Then (because $\card T$ is a limit ordinal,
see \cite{wiki:limit-ordinal}) there exists $\beta\in\card T$ such
that $\alpha\in\beta\in\card T$. So $b\sqsubseteq a_{\alpha}$ for
every $\alpha\in\card T$. Thus $b\sqsubseteq\bigsqcap T$.


Finally $\bigsqcap T=b\in S$.

\end{description}
\end{proof}

\section{Co-Separability of Core}
\begin{thm}
\label{cosep-crit}The following is an implications tuple.
\begin{enumerate}
\item \label{cosep-crit-p}$(\mathfrak{A};\mathfrak{Z})$ is a powerset
filtrator.
\item \label{cosep-crit-prim}$(\mathfrak{A};\mathfrak{Z})$ is a primary
filtrator over a meet infinite distributive complete lattice.
\item \label{cosep-crit-uff}$(\mathfrak{A};\mathfrak{Z})$ is an up-aligned
filtered filtrator whose core is a meet infinite distributive complete
lattice.
\item \label{cosep-crit-conc}This filtrator is with co-separable core.
\end{enumerate}
\end{thm}
\begin{proof}
~
\begin{description}
\item [{\ref{cosep-crit-p}$\Rightarrow$\ref{cosep-crit-prim}}] Obvious.
\item [{\ref{cosep-crit-prim}$\Rightarrow$\ref{cosep-crit-uff}}] It
is obviously up-aligned, and filtered by theorem~\ref{semifilt-joinclosed}.
\item [{\ref{cosep-crit-uff}$\Rightarrow$\ref{cosep-crit-conc}}] Our
filtrator is with join-closed core (theorem \ref{semifilt-joinclosed}).


Let $a,b\in\mathfrak{A}$. $\Cor a$ and $\Cor b$ exist since $\mathfrak{Z}$
is a complete lattice.


$\Cor a\in\down a$ and $\Cor b\in\down b$ by the theorem~\ref{cor-less}
since our filtrator is filtered. So we have
\begin{align*}
\exists x\in\down a,y\in\down b:x\sqcup^{\mathfrak{A}}y=\top & \Leftarrow\\
\Cor a\sqcup^{\mathfrak{A}}\Cor b=\top & \Leftrightarrow\text{(by finite join-closedness of the core)}\\
\Cor a\sqcup^{\mathfrak{Z}}\Cor b=\top & \Leftrightarrow\\
\bigsqcap^{\mathfrak{Z}}\up a\sqcup^{\mathfrak{Z}}\bigsqcap^{\mathfrak{Z}}\up b=\top & \Leftrightarrow\text{(by infinite distributivity)}\\
\bigsqcap^{\mathfrak{Z}}\setcond{x\sqcup^{\mathfrak{Z}}y}{x\in\up a,y\in\up b}=\top & \Leftarrow\\
\forall x\in\up a,y\in\up b:x\sqcup^{\mathfrak{Z}}y=\top & \Leftrightarrow\text{(by binary join-closedness of the core)}\\
\forall x\in\up a,y\in\up b:x\sqcup^{\mathfrak{A}}y=\top & \Leftarrow\\
a\sqcup^{\mathfrak{A}}b=\top.
\end{align*}


\end{description}
\end{proof}

\section{Complements and Core Parts}
\begin{lem}
If $(\mathfrak{A};\mathfrak{Z})$ is a filtered, up-aligned filtrator
with co-separable core which is a complete lattice, then for any $a,c\in\mathfrak{A}$
\[
c\equiv^{\mathfrak{A}}a\Leftrightarrow c\equiv^{\mathfrak{A}}\Cor a.
\]
\end{lem}
\begin{proof}
~
\begin{description}
\item [{$\Rightarrow$}] If $c\equiv^{\mathfrak{A}}a$ then by co-separability
of the core exists $K\in\down a$ such that $c\equiv^{\mathfrak{A}}K$.
To finish the proof we will show that $K\sqsubseteq\Cor a$. To show
this is enough to show that $\forall X\in\up a:K\sqsubseteq X$ what
is obvious.
\item [{$\Leftarrow$}] $\Cor a\sqsubseteq a$ (by the theorem \ref{cor-less}
using that our filtrator is filtered).
\end{description}
\end{proof}
\begin{thm}
\label{cocompl-cor}If $(\mathfrak{A};\mathfrak{Z})$ is a filtered
up-aligned complete lattice filtrator with co-separable core which
is a complete boolean lattice, then $a^{+}=\overline{\Cor a}$ for
every $a\in\mathfrak{A}$.\end{thm}
\begin{proof}
Our filtrator is with join-closed core (theorem \ref{semifilt-joinclosed}).
\begin{align*}
a^{+} & =\\
\bigsqcap^{\mathfrak{A}}\setcond{c\in\mathfrak{A}}{c\sqcup^{\mathfrak{A}}a=\top^{\mathfrak{A}}} & =\\
\bigsqcap^{\mathfrak{A}}\setcond{c\in\mathfrak{A}}{c\sqcup^{\mathfrak{A}}\Cor a=\top^{\mathfrak{A}}} & =\\
\bigsqcap^{\mathfrak{A}}\setcond{c\in\mathfrak{A}}{c\sqsupseteq\overline{\Cor a}} & =\\
\overline{\Cor a}
\end{align*}
(used the lemma and theorem \ref{bool-compl}).\end{proof}
\begin{cor}
If $(\mathfrak{A};\mathfrak{Z})$ is a filtered up-aligned complete
lattice filtrator with co-separable core which is a complete boolean
lattice, then $a^{+}\in\mathfrak{Z}$ for every $a\in\mathfrak{A}$.\end{cor}
\begin{thm}
\label{compl-and-cor}If $(\mathfrak{A};\mathfrak{Z})$ is a filtered
complete lattice filtrator with down-aligned, binarily meet-closed,
separable core which is a complete boolean lattice, then $a^{\ast}=\overline{\Cor a}=\overline{\Cor'a}$
for every $a\in\mathfrak{A}$.\end{thm}
\begin{proof}
Our filtrator is with join-closed core (theorem \ref{semifilt-joinclosed}).
$a^{\ast}=\bigsqcup^{\mathfrak{A}}\setcond{c\in\mathfrak{A}}{c\sqcap^{\mathfrak{A}}a=\bot^{\mathfrak{A}}}$.
But $c\sqcap^{\mathfrak{A}}a=\bot^{\mathfrak{A}}\Rightarrow\exists C\in\up c:C\sqcap^{\mathfrak{A}}a=\bot^{\mathfrak{A}}$.
So
\begin{align*}
a^{\ast} & =\\
\bigsqcup^{\mathfrak{A}}\setcond{C\in\mathfrak{Z}}{C\sqcap^{\mathfrak{A}}a=\bot^{\mathfrak{A}}} & =\\
\bigsqcup^{\mathfrak{A}}\setcond{C\in\mathfrak{Z}}{a\sqsubseteq\overline{C}} & =\\
\bigsqcup^{\mathfrak{A}}\setcond{\overline{C}}{C\in\mathfrak{Z},a\sqsubseteq C} & =\\
\bigsqcup^{\mathfrak{A}}\setcond{\overline{C}}{C\in\up a} & =\\
\bigsqcup^{\mathfrak{Z}}\setcond{\overline{C}}{C\in\up a} & =\\
\overline{\bigsqcap^{\mathfrak{Z}}\setcond C{C\in\up a}} & =\\
\overline{\bigsqcap^{\mathfrak{Z}}\up a} & =\\
\overline{\Cor a}
\end{align*}
(used theorem \ref{bool-compl}).

$\Cor a=\Cor'a$ by theorem \ref{cor-eq}.\end{proof}
\begin{cor}
If $(\mathfrak{A};\mathfrak{Z})$ is a filtered down-aligned and up-aligned
complete lattice filtrator with binarily meet-closed, separable and
co-separable core which is a complete boolean lattice, then $a^{\ast}=a^{+}\in\mathfrak{Z}$
for every $a\in\mathfrak{A}$.\end{cor}
\begin{proof}
Comparing two last theorems.\end{proof}
\begin{thm}
\label{compl-in-core}The following is an implications tuple:
\begin{enumerate}
\item \label{compl-in-core-f}$(\mathfrak{A};\mathfrak{Z})$ is a primary
filtrator over a complete lattice.
\item \label{compl-in-core-fltr}$(\mathfrak{A};\mathfrak{Z})$ is a complete
lattice filtrator with join-closed separable core which is a complete
lattice.
\item \label{compl-in-core-conc}$a^{\ast}\in\mathfrak{Z}$ for every $a\in\mathfrak{A}$.
\end{enumerate}
\end{thm}
\begin{proof}
~
\begin{description}
\item [{\ref{compl-in-core-f}$\Rightarrow$\ref{compl-in-core-fltr}}] $\mathfrak{A}$
is a complete lattice by \ref{filt-is-complete}. $(\mathfrak{A};\mathfrak{Z})$
is a filtrator with join-closed core by theorem~\ref{semifilt-joinclosed}.
$(\mathfrak{A};\mathfrak{Z})$ is a filtrator with separable core
by theorem \ref{when-sep-core}.
\item [{\ref{compl-in-core-fltr}$\Rightarrow$\ref{compl-in-core-conc}}] $\setcond{c\in\mathfrak{A}}{c\sqcap^{\mathfrak{A}}a=\bot^{\mathfrak{A}}}\supseteq\setcond{A\in\mathfrak{Z}}{A\sqcap^{\mathfrak{A}}a=\bot^{\mathfrak{A}}}$;
consequently $a^{\ast}\sqsupseteq\bigsqcup^{\mathfrak{A}}\setcond{A\in\mathfrak{Z}}{A\sqcap^{\mathfrak{A}}a=\bot^{\mathfrak{A}}}$.


But if $c\in\setcond{c\in\mathfrak{A}}{c\sqcap^{\mathfrak{A}}a=\bot^{\mathfrak{A}}}$
then there exists $A\in\mathfrak{Z}$ such that $A\sqsupseteq c$
and $A\sqcap^{\mathfrak{A}}a=\bot^{\mathfrak{A}}$ that is $A\in\setcond{A\in\mathfrak{Z}}{A\sqcap^{\mathfrak{A}}a=\bot^{\mathfrak{A}}}$.
Consequently $a^{\ast}\sqsubseteq\bigsqcup^{\mathfrak{A}}\setcond{A\in\mathfrak{Z}}{A\sqcap^{\mathfrak{A}}a=\bot^{\mathfrak{A}}}$.


We have $a^{\ast}=\bigsqcup^{\mathfrak{A}}\setcond{A\in\mathfrak{Z}}{A\sqcap^{\mathfrak{A}}a=\bot^{\mathfrak{A}}}=\bigsqcup^{\mathfrak{Z}}\setcond{A\in\mathfrak{Z}}{A\sqcap^{\mathfrak{A}}a=\bot^{\mathfrak{A}}}\in\mathfrak{Z}$.

\end{description}
\end{proof}
\begin{thm}
\label{dual-compl-pseudo}The following is an implications tuple:
\begin{enumerate}
\item \label{dual-compl-pseudo-p}$(\mathfrak{A};\mathfrak{Z})$ is a powerset
filtrator.
\item \label{dual-compl-pseudo-f}$(\mathfrak{A};\mathfrak{Z})$ is a primary
filtrator over a complete boolean lattice.
\item \label{dual-compl-pseudo-fltr}$(\mathfrak{A};\mathfrak{Z})$ is an
up-aligned filtered complete lattice filtrator with co-separable core
which is a complete boolean lattice.
\item \label{dual-compl-pseudo-conc}$a^{+}$ is dual pseudocomplement of
$a$, that is
\[
a^{+}=\min\setcond{c\in\mathfrak{A}}{c\sqcup^{\mathfrak{A}}a=\top^{\mathfrak{A}}}
\]
for every $a\in\mathfrak{A}$.
\end{enumerate}
\end{thm}
\begin{proof}
~
\begin{description}
\item [{\ref{dual-compl-pseudo-p}$\Rightarrow$\ref{dual-compl-pseudo-f}}] Obvious.
\item [{\ref{dual-compl-pseudo-f}$\Rightarrow$\ref{dual-compl-pseudo-fltr}}] $(\mathfrak{A};\mathfrak{Z})$
is filtered by the theorem~\ref{semifilt-joinclosed}. It is with
co-separable core by theorem~\ref{cosep-crit}. $\mathfrak{A}$ is
a complete lattice by corollary \ref{filt-is-complete}.
\item [{\ref{dual-compl-pseudo-fltr}$\Rightarrow$\ref{dual-compl-pseudo-conc}}] Our
filtrator is with join-closed core (theorem \ref{semifilt-joinclosed}).
It's enough to prove that $a^{+}\sqcup^{\mathfrak{A}}a=\top^{\mathfrak{A}}$.
But $a^{+}\sqcup^{\mathfrak{A}}a=\overline{\Cor a}\sqcup^{\mathfrak{A}}a\sqsupseteq\overline{\Cor a}\sqcup^{\mathfrak{A}}\Cor a=\overline{\Cor a}\sqcup^{\mathfrak{Z}}\Cor a=\top^{\mathfrak{A}}$
(used the theorem \ref{cor-less} and the fact that our filtrator
is filtered).
\end{description}
\end{proof}
\begin{prop}
\label{compl-eq-dual}The following is an implications tuple:
\begin{enumerate}
\item \label{compl-eq-dual-p}$(\mathfrak{A};\mathfrak{Z})$ is a powerset
filtrator.
\item \label{compl-eq-dual-fltr}$(\mathfrak{A};\mathfrak{Z})$ is a primary
filtrator over a complete boolean lattice.
\item \label{compl-eq-dual-conc}$a^{\ast}=a^{+}=\overline{\Cor a}=\overline{\Cor' a}\in\mathfrak{Z}$ for
every $a\in\mathfrak{A}$.
\end{enumerate}
\end{prop}
\begin{proof}
~
\begin{description}
\item [{\ref{compl-eq-dual-p}$\Rightarrow$\ref{compl-eq-dual-fltr}}] Obvious.
\item [{\ref{compl-eq-dual-fltr}$\Rightarrow$\ref{compl-eq-dual-conc}}] The
filtrator $(\mathfrak{A};\mathfrak{Z})$ is filtered by the theorem~\ref{semifilt-joinclosed}.
$\mathfrak{A}$ is a complete lattice by corollary \ref{filt-is-complete}.
$(\mathfrak{A};\mathfrak{Z})$ is with co-separable core by theorem
\ref{cosep-crit}. Thus we can apply the theorem \ref{cocompl-cor}.


$(\mathfrak{A};\mathfrak{Z})$ is filtered by the theorem~\ref{semifilt-joinclosed},
binarily meet-closed by proposition \ref{f-meet-closed}, with separable
core by theorem \ref{when-sep-core}. $\mathfrak{A}$ is a complete
lattice by corollary \ref{filt-is-complete}. So we can apply the
theorem \ref{compl-and-cor}.

\end{description}
\end{proof}
\begin{defn}
\index{edge part}The \emph{edge part} of an element $a\in\mathfrak{A}$
is $\Edg a=a\setminus\Cor a$, the \emph{dual edge part} is $\Edg'a=a\setminus\Cor'a$.
\end{defn}
Knowing core part and edge part or dual core part and dual edge part
of an element of a filtrator, the filter can be restored by the formulas:
\[
a=\Cor a\sqcup^{\mathfrak{A}}\Edg a\quad\text{and}\quad a=\Cor'a\sqcup^{\mathfrak{A}}\Edg'a.
\]



\section{Core Part and Atomic Elements}
\begin{prop}
\label{cor-join-atom}The following is an implications tuple:
\begin{enumerate}
\item \label{cor-join-atom-p}$(\mathfrak{A};\mathfrak{Z})$ is a powerset
filtrator.
\item \label{cor-join-atom-f}$(\mathfrak{A};\mathfrak{Z})$ is a primary filtrator
over an atomistic lattice.
\item \label{cor-join-atom-fltr}$(\mathfrak{A};\mathfrak{Z})$ is a filtrator
with join-closed core and $\mathfrak{Z}$ be an atomistic lattice.
\item \label{cor-join-atom-conc}$\Cor'a=\bigsqcup^{\mathfrak{Z}}\setcond x{x\text{ is an atom of }\mathfrak{Z},x\sqsubseteq a}$
for every $a\in\mathfrak{A}$ such that $\Cor'a$ exists.
\end{enumerate}
\end{prop}
\begin{proof}
~
\begin{description}
\item [{\ref{cor-join-atom-p}$\Rightarrow$\ref{cor-join-atom-f}}] Obvious.
\item [{\ref{cor-join-atom-f}$\Rightarrow$\ref{cor-join-atom-fltr}}] $(\mathfrak{A};\mathfrak{Z})$
is with join-closed core by corollary \ref{semifilt-joinclosed}.
\item [{\ref{cor-join-atom-fltr}$\Rightarrow$\ref{cor-join-atom-conc}}] ~
\begin{align*}
\Cor'a & =\\
\bigsqcup^{\mathfrak{Z}}\setcond{A\in\mathfrak{Z}}{A\sqsubseteq a} & =\\
\bigsqcup^{\mathfrak{Z}}\setcond{\bigsqcup^{\mathfrak{Z}}\atoms^{\mathfrak{Z}}A}{A\in\mathfrak{Z},A\sqsubseteq a} & =\\
\bigsqcup^{\mathfrak{Z}}\bigcup\setcond{\atoms^{\mathfrak{Z}}A}{A\in\mathfrak{Z},A\sqsubseteq a} & =\\
\bigsqcup^{\mathfrak{Z}}\setcond x{x\text{ is an atom of }\mathfrak{Z},x\sqsubseteq a}.
\end{align*}

\end{description}
\end{proof}
\begin{cor}
$\Cor a=\uparrow\setcond{p\in\mathfrak{U}}{\uparrow\{p\}\sqsubseteq a}$
and $\bigcap a=\setcond{p\in\mathfrak{U}}{\uparrow\{p\}\sqsubseteq a}$
for every filter $a$ on a set $\mathfrak{U}$.\end{cor}
\begin{proof}
By proposition \ref{dcor-eq-cor}.
\end{proof}

\section{Distributivity of Core Part over Lattice Operations}
\begin{thm}
\label{dual-cor-meet}The following is an implications tuple:
\begin{enumerate}
\item \label{dual-cor-meet-p}$(\mathfrak{A};\mathfrak{Z})$ is a powerset
filtrator.
\item \label{dual-cor-meet-f}$(\mathfrak{A};\mathfrak{Z})$ is a primary
filtrator over a complete lattice.
\item \label{dual-cor-meet-mlat}$(\mathfrak{A};\mathfrak{Z})$ is a join-closed
filtrator and $\mathfrak{A}$ is a meet-semilattice and $\mathfrak{Z}$
is a complete lattice.
\item \label{dual-cor-meet-conc}$\Cor'(a\sqcap^{\mathfrak{A}}b)=\Cor'a\sqcap^{\mathfrak{Z}}\Cor'b$
for every $a,b\in\mathfrak{A}$.
\end{enumerate}
\end{thm}
\begin{proof}
~
\begin{description}
\item [{\ref{dual-cor-meet-p}$\Rightarrow$\ref{dual-cor-meet-f}}] Obvious.
\item [{\ref{dual-cor-meet-f}$\Rightarrow$\ref{dual-cor-meet-mlat}}] $(\mathfrak{A};\mathfrak{Z})$
is with join-closed core by corollary \ref{semifilt-joinclosed}.
$\mathfrak{A}$ is a meet-semilattice by corollary \ref{filt-is-complete}.
\item [{\ref{dual-cor-meet-mlat}$\Rightarrow$\ref{dual-cor-meet-conc}}] From
theorem conditions it follows that $\Cor'(a\sqcap^{\mathfrak{A}}b)$, $\Cor'a$, and $\Cor'b$
exist.


We have $\Cor'p\sqsubseteq p$ for every $p\in\mathfrak{A}$ because
our filtrator is with join-closed core (theorem~\ref{f-cor-max}).


Obviously $\Cor'(a\sqcap^{\mathfrak{A}}b)\sqsubseteq\Cor'a$ and $\Cor'(a\sqcap^{\mathfrak{A}}b)\sqsubseteq\Cor'b$.


If $x\sqsubseteq\Cor'a$ and $x\sqsubseteq\Cor'b$ for some $x\in\mathfrak{Z}$
then $x\sqsubseteq a$ and $x\sqsubseteq b$, thus $x\sqsubseteq a\sqcap^{\mathfrak{A}}b$
and $x\sqsubseteq\Cor'(a\sqcap^{\mathfrak{A}}b)$.

\end{description}
\end{proof}
\begin{thm}
\label{dual-cor-inf-meet}The following is an implications tuple:
\begin{enumerate}
\item \label{cor-inf-meet-p}$(\mathfrak{A};\mathfrak{Z})$ is a powerset
filtrator.
\item \label{cor-inf-meet-f}$(\mathfrak{A};\mathfrak{Z})$ is a primary filtrator
over a complete lattice.
\item \label{cor-inf-meet-fltr}$(\mathfrak{A};\mathfrak{Z})$ is a join-closed
filtrator and both $\mathfrak{A}$ and $\mathfrak{Z}$ are complete
lattices.
\item \label{cor-inf-meet-conc}$\Cor'\bigsqcap^{\mathfrak{A}}S=\bigsqcap^{\mathfrak{Z}}\rsupfun{\Cor'}S$
for every $S\in\subsets\mathfrak{A}$ and $\Cor'\bigsqcap^{\mathfrak{A}}T=\bigsqcap^{\mathfrak{Z}}T$
for every $T\in\subsets\mathfrak{Z}$.
\end{enumerate}
\end{thm}
\begin{proof}
~
\begin{description}
\item [{\ref{cor-inf-meet-p}$\Rightarrow$\ref{cor-inf-meet-f}}] Obvious.
\item [{\ref{cor-inf-meet-f}$\Rightarrow$\ref{cor-inf-meet-fltr}}] It
is with join-closed core by theorem~\ref{semifilt-joinclosed}.
$\mathfrak{A}$ is a complete lattice by corollary \ref{filt-is-complete}.
\item [{\ref{cor-inf-meet-fltr}$\Rightarrow$\ref{cor-inf-meet-conc}}] From
theorem conditions it follows that $\Cor'\bigsqcap^{\mathfrak{A}}S$
exists.


We have $\Cor'p\sqsubseteq p$ for every $p\in\mathfrak{A}$ because
our filtrator is with join-closed core (theorem~\ref{f-cor-max}).


Obviously $\Cor'\bigsqcap^{\mathfrak{A}}S\sqsubseteq\Cor'a$ for every
$a\in S$.


If $x\sqsubseteq\Cor'a$ for every $a\in S$ for some $x\in\mathfrak{Z}$
then $x\sqsubseteq a$, thus $x\sqsubseteq\bigsqcap^{\mathfrak{A}}S$
and $x\sqsubseteq\Cor'\bigsqcap^{\mathfrak{A}}S$.


So $\Cor'\bigsqcap^{\mathfrak{A}}S=\bigsqcap^{\mathfrak{Z}}\rsupfun{\Cor'}S$.
$\Cor'\bigsqcap^{\mathfrak{A}}T=\bigsqcap^{\mathfrak{Z}}T$ trivially
follows from this.

\end{description}
\end{proof}
\begin{thm}
\label{dual-core-join}The following is an implications tuple:
\begin{enumerate}
\item \label{dual-core-join-p}$(\mathfrak{A};\mathfrak{Z})$ is a powerset
filtrator.
\item \label{dual-core-join-f}$(\mathfrak{A};\mathfrak{Z})$ is a primary
filtrator over a complete atomistic distributive lattice.
\item \label{dual-core-join-fltr}$(\mathfrak{A};\mathfrak{Z})$ is a filtered
down-aligned filtrator with binarily meet-closed core $\mathfrak{Z}$
which is a complete atomistic lattice and $\mathfrak{A}$ is a complete
starrish lattice.
\item \label{dual-core-join-conc}$\Cor'(a\sqcup^{\mathfrak{A}}b)=\Cor'a\sqcup^{\mathfrak{Z}}\Cor'b$
for every $a,b\in\mathfrak{A}$.
\end{enumerate}
\end{thm}
\begin{proof}
~
\begin{description}
\item [{\ref{dual-core-join-p}$\Rightarrow$\ref{dual-core-join-f}}] Obvious.
\item [{\ref{dual-core-join-f}$\Rightarrow$\ref{dual-core-join-fltr}}] $(\mathfrak{A};\mathfrak{Z})$
is filtered by theorem~\ref{semifilt-joinclosed}. It is with binarily
meet-close core by corollary~\ref{f-meet-closed}. $\mathfrak{A}$ is starrish
by corollary \ref{filt-also-distr}. $\mathfrak{A}$ is complete by
corollary~\ref{filt-is-complete}.
\item [{\ref{dual-core-join-fltr}$\Rightarrow$\ref{dual-core-join-conc}}] From
theorem conditions it follows that $\Cor'(a\sqcup^{\mathfrak{A}}b)$
exists.


$\Cor'(a\sqcup^{\mathfrak{A}}b)=\bigsqcup^{\mathfrak{Z}}\setcond x{x\text{ is an atom of }\mathfrak{Z},x\sqsubseteq a\sqcup^{\mathfrak{A}}b}$
(used proposition \ref{cor-join-atom}).


By theorem \ref{atom-both} we have
\begin{align*}
\Cor'(a\sqcup^{\mathfrak{A}}b) & =\\
\bigsqcup^{\mathfrak{Z}}((\atoms^{\mathfrak{A}}(a\sqcup^{\mathfrak{A}}b))\cap\mathfrak{Z}) & =\\
\bigsqcup^{\mathfrak{Z}}((\atoms^{\mathfrak{A}}a\cup\atoms^{\mathfrak{A}}b)\cap\mathfrak{Z}) & =\\
\bigsqcup^{\mathfrak{Z}}((\atoms^{\mathfrak{A}}a\cap\mathfrak{Z})\cup(\atoms^{\mathfrak{A}}b\cap\mathfrak{Z})) & =\\
\bigsqcup^{\mathfrak{Z}}(\atoms^{\mathfrak{A}}a\cap\mathfrak{Z})\sqcup^{\mathfrak{Z}}\bigsqcup^{\mathfrak{Z}}(\atoms^{\mathfrak{A}}b\cap\mathfrak{Z})
\end{align*}
(used the theorem \ref{atoms-join}). Again using theorem \ref{atom-both},
we get
\begin{align*}
\Cor'(a\sqcup^{\mathfrak{A}}b) & =\\
\bigsqcup^{\mathfrak{Z}}\setcond x{x\text{ is an atom of }\mathfrak{Z},x\sqsubseteq a}\sqcup^{\mathfrak{Z}}\bigsqcup^{\mathfrak{Z}}\setcond x{x\text{ is an atom of }\mathfrak{Z},x\sqsubseteq b} & =\\
\Cor'a\sqcup^{\mathfrak{Z}}\Cor'b
\end{align*}
(again used proposition \ref{cor-join-atom}).

\end{description}
\end{proof}
\begin{thm}
\label{compl-join}The following is an implications tuple:
\begin{enumerate}
\item \label{compl-join-p}$(\mathfrak{A};\mathfrak{Z})$ is a powerset
filtrator.
\item \label{compl-join-f}$(\mathfrak{A};\mathfrak{Z})$ is a primary filtrator
over a complete atomistic boolean lattice.
\item \label{compl-join-fltr}$(\mathfrak{A};\mathfrak{Z})$ is a filtered
starrish down-aligned complete lattice filtrator with binarily meet-closed,
separable core which is a complete atomistic boolean lattice.
\item \label{compl-join-conc}$(a\sqcup^{\mathfrak{A}}b)^{\ast}=a^{\ast}\sqcap^{\mathfrak{Z}}b^{\ast}$
for every $a,b\in\mathfrak{A}$.
\end{enumerate}
\end{thm}
\begin{proof}
~
\begin{description}
\item [{\ref{compl-join-p}$\Rightarrow$\ref{compl-join-f}}] Obvious.
\item [{\ref{compl-join-f}$\Rightarrow$\ref{compl-join-fltr}}] $(\mathfrak{A};\mathfrak{Z})$
is a filtered (theorem~\ref{semifilt-joinclosed}), distributive
(corollary \ref{filt-also-distr}) complete lattice filtrator (corollary
\ref{filt-is-complete}), with binarily meet-closed core (proposition~\ref{f-meet-closed}),
with separable core (theorem~\ref{when-sep-core}).
\item [{\ref{compl-join-fltr}$\Rightarrow$\ref{compl-join-conc}}] $(a\sqcup^{\mathfrak{A}}b)^{\ast}=\overline{\Cor'(a\sqcup^{\mathfrak{A}}b)}=\overline{\Cor'a\sqcup^{\mathfrak{Z}}\Cor'b}=\overline{\Cor'a}\sqcap^{\mathfrak{Z}}\overline{\Cor'b}=a^{\ast}\sqcap^{\mathfrak{Z}}b^{\ast}$
(used theorem \ref{compl-and-cor}).
\end{description}
\end{proof}

\section{Separability criteria}
\begin{thm}
\label{f-intrs-and-compl}The following is an implications tuple:
\begin{enumerate}
\item \label{intrs-compl-p}$(\mathfrak{A};\mathfrak{Z})$ is a powerset
filtrator.
\item \label{intrs-compl-prim}$(\mathfrak{A};\mathfrak{Z})$ is a primary
filtrator over a boolean lattice.
\item \label{intrs-compl-filt}$(\mathfrak{A};\mathfrak{Z})$ is a down-aligned filtrator
with binarily meet-closed and separable core.
\item \label{intrs-compl-conc}$B\asymp^{\mathfrak{A}}\mathcal{A}\Leftrightarrow\overline{B}\sqsupseteq\mathcal{A}$
for every $B\in\mathfrak{Z}$, $\mathcal{A}\in\mathfrak{A}$.
\end{enumerate}
\end{thm}
\begin{proof}
~
\begin{description}
\item [{\ref{intrs-compl-p}$\Rightarrow$\ref{intrs-compl-prim}}] Obvious.
\item [{\ref{intrs-compl-prim}$\Rightarrow$\ref{intrs-compl-filt}}] Using
obvious \ref{filt-aligned}, corollary~\ref{f-meet-closed}, theorem
\ref{when-sep-core}.
\item [{\ref{intrs-compl-filt}$\Rightarrow$\ref{intrs-compl-conc}}] By
the lemma~\ref{int-join-lem}.
\end{description}
\end{proof}
\begin{thm}
The following is an implications tuple:
\begin{enumerate}
\item \label{join-compl-p}$(\mathfrak{A};\mathfrak{Z})$ is a powerset
filtrator.
\item \label{join-compl-prim}$(\mathfrak{A};\mathfrak{Z})$ is a primary
filtrator over a complete boolean lattice.
\item \label{join-compl-filt}$(\mathfrak{A};\mathfrak{Z})$ is an up-aligned filtrator
with binarily join-closed and co-separable core.
\item \label{join-compl-conc}$B\equiv^{\mathfrak{A}}\mathcal{A}\Leftrightarrow\overline{B}\sqsubseteq\mathcal{A}$
for every $B\in\mathfrak{Z}$, $\mathcal{A}\in\mathfrak{A}$.
\end{enumerate}
\end{thm}
\begin{proof}
~
\begin{description}
\item [{\ref{join-compl-p}$\Rightarrow$\ref{join-compl-prim}}] Obvious.
\item [{\ref{join-compl-prim}$\Rightarrow$\ref{join-compl-filt}}] Using
obvious \ref{filt-aligned}, theorem~\ref{semifilt-joinclosed},
theorem \ref{cosep-crit}.
\item [{\ref{join-compl-filt}$\Rightarrow$\ref{join-compl-conc}}] By
the lemma~\ref{int-join-lem}.
\end{description}
\end{proof}

\section{Filtrators over Boolean Lattices}
\begin{prop}
\label{b-bool-minus}The following is an implications tuple:
\begin{enumerate}
\item \label{bool-minus-p}$(\mathfrak{A};\mathfrak{Z})$ is a powerset
filtrator.
\item \label{bool-minus-f}$(\mathfrak{A};\mathfrak{Z})$ is a primary filtrator
over a boolean lattice.
\item \label{bool-minus-fltr}$(\mathfrak{A};\mathfrak{Z})$ is a down-aligned
and up-aligned binarily meet-closed and binarily join-closed distributive
lattice filtrator and $\mathfrak{Z}$ be a boolean lattice.
\item \label{bool-minus-conc}$a\setminus{}^{\mathfrak{A}}B=a\sqcap^{\mathfrak{A}}\overline{B}$
for every $a\in\mathfrak{A}$, $B\in\mathfrak{Z}$.
\end{enumerate}
\end{prop}
\begin{proof}
~
\begin{description}
\item [{\ref{bool-minus-p}$\Rightarrow$\ref{bool-minus-f}}] Obvious.
\item [{\ref{bool-minus-f}$\Rightarrow$\ref{bool-minus-fltr}}] $\mathfrak{A}$
is a distributive lattice by corollary \ref{filt-also-distr}. Our
filtrator is binarily meet-closed by the corollary~\ref{f-meet-closed}
and with join-closed core by the theorem \ref{semifilt-joinclosed}.
It is also up and down aligned.
\item [{\ref{bool-minus-fltr}$\Rightarrow$\ref{bool-minus-conc}}] ~
\begin{gather*}
(a\sqcap^{\mathfrak{A}}\overline{B})\sqcup^{\mathfrak{A}}B=(a\sqcup^{\mathfrak{A}}B)\sqcap^{\mathfrak{A}}(\overline{B}\sqcup^{\mathfrak{A}}B)=(a\sqcup^{\mathfrak{A}}B)\sqcap^{\mathfrak{A}}(\overline{B}\sqcup^{\mathfrak{Z}}B)=(a\sqcup^{\mathfrak{A}}B)\sqcap^{\mathfrak{A}}\top=a\sqcup^{\mathfrak{A}}B.\\
(a\sqcap^{\mathfrak{A}}\overline{B})\sqcap^{\mathfrak{A}}B=a\sqcap^{\mathfrak{A}}(\overline{B}\sqcap^{\mathfrak{A}}B)=a\sqcap^{\mathfrak{A}}(\overline{B}\sqcap^{\mathfrak{Z}}B)=a\sqcap^{\mathfrak{A}}\bot=\bot.
\end{gather*}
So $a\sqcap^{\mathfrak{A}}\overline{B}$ is the difference of $a$
and $B$.
\end{description}
\end{proof}

\section{Distributivity for an Element of Boolean Core}
\begin{lem}
\label{bb-bool-adj}The following is an implications tuple:
\begin{enumerate}
\item \label{bool-adj-p}$(\mathfrak{A};\mathfrak{Z})$ is a powerset filtrator.
\item \label{bool-adj-f}$(\mathfrak{A};\mathfrak{Z})$ is a primary filtrator
over a boolean lattice.
\item \label{bool-adj-fltr}$(\mathfrak{A};\mathfrak{Z})$ is an up-aligned
binarily join-closed and binarily meet-closed distributive lattice
filtrator over a boolean lattice.
\item \label{bool-adj-conc}$A\sqcap^{\mathfrak{A}}$ is a lower adjoint
of $\overline{A}\sqcup^{\mathfrak{A}}$ for every $A\in\mathfrak{Z}$.
\end{enumerate}
\end{lem}
\begin{proof}
~
\begin{description}
\item [{\ref{bool-adj-p}$\Rightarrow$\ref{bool-adj-f}}] Obvious.
\item [{\ref{bool-adj-f}$\Rightarrow$\ref{bool-adj-fltr}}] It is binarily join closed by theorem~\ref{semifilt-joinclosed}.
It is binarily meet-closed by corollary~\ref{f-meet-closed}. It is distributive by
corollary~\ref{filt-also-distr}.
\item [{\ref{bool-adj-fltr}$\Rightarrow$\ref{bool-adj-conc}}] We will
use the theorem \ref{galois-second}.


That $A\sqcap^{\mathfrak{A}}$ and $\overline{A}\sqcup^{\mathfrak{A}}$
are monotone is obvious.


We need to prove (for every $x,y\in\mathfrak{A}$) that
\[
x\sqsubseteq\overline{A}\sqcup^{\mathfrak{A}}(A\sqcap^{\mathfrak{A}}x)\quad\text{and}\quad A\sqcap^{\mathfrak{A}}(\overline{A}\sqcup^{\mathfrak{A}}y)\sqsubseteq y.
\]



Really,
\[
\overline{A}\sqcup^{\mathfrak{A}}(A\sqcap^{\mathfrak{A}}x)=(\overline{A}\sqcup^{\mathfrak{A}}A)\sqcap^{\mathfrak{A}}(\overline{A}\sqcup^{\mathfrak{A}}x)=(\overline{A}\sqcup^{\mathfrak{Z}}A)\sqcap^{\mathfrak{A}}(\overline{A}\sqcup^{\mathfrak{A}}x)=\top\sqcap^{\mathfrak{A}}(\overline{A}\sqcup^{\mathfrak{A}}x)=\overline{A}\sqcup^{\mathfrak{A}}x\sqsupseteq x
\]



and
\[
A\sqcap^{\mathfrak{A}}(\overline{A}\sqcup^{\mathfrak{A}}y)=(A\sqcap^{\mathfrak{A}}\overline{A})\sqcup^{\mathfrak{A}}(A\sqcap^{\mathfrak{A}}y)=(A\sqcap^{\mathfrak{Z}}\overline{A})\sqcup^{\mathfrak{A}}(A\sqcap^{\mathfrak{A}}y)=\bot\sqcup^{\mathfrak{A}}(A\sqcap^{\mathfrak{A}}y)=A\sqcap^{\mathfrak{A}}y\sqsubseteq y.
\]


\end{description}
\end{proof}
\begin{thm}
\label{b-f-back-distr}The following is an implications tuple:
\begin{enumerate}
\item \label{f-back-distr-p}($\mathfrak{A};\mathfrak{Z})$ is a powerset filtrator.
\item \label{f-back-distr-f}($\mathfrak{A};\mathfrak{Z})$ is a primary filtrator over
a boolean lattice.
\item \label{f-back-distr-fltr}$(\mathfrak{A};\mathfrak{Z})$ is an up-aligned
binarily join-closed and binarily meet-closed distributive lattice
filtrator over a boolean lattice.
\item \label{f-back-distr-conc}$A\sqcap^{\mathfrak{A}}\bigsqcup^{\mathfrak{A}}S=\bigsqcup^{\mathfrak{A}}\rsupfun{A\sqcap^{\mathfrak{A}}}S$
for every $A\in\mathfrak{Z}$ and every set $S\in\subsets\mathfrak{A}$.
\end{enumerate}
\end{thm}
\begin{proof}
~
\begin{description}
\item [{\ref{f-back-distr-p}$\Rightarrow$\ref{f-back-distr-f}}] Obvious.
\item [{\ref{f-back-distr-f}$\Rightarrow$\ref{f-back-distr-fltr}}] It is binarily join closed by theorem~\ref{semifilt-joinclosed}.
It is binarily meet-closed by corollary~\ref{f-meet-closed}. It is distributive by
corollary~\ref{filt-also-distr}.
\item [{\ref{f-back-distr-fltr}$\Rightarrow$\ref{f-back-distr-conc}}] Direct
consequence of the lemma.
\end{description}
\end{proof}

\section{More about the Lattice of Filters}
\begin{defn}
\index{ultrafilter}Atoms of $\mathfrak{F}$ are called \emph{ultrafilters}.
\end{defn}

\begin{defn}
\index{ultrafilter!trivial}Principal ultrafilters are also called
\emph{trivial ultrafilters}.\end{defn}
\begin{thm}
\label{pow-filt-central}The following is an implications tuple:
\begin{enumerate}
\item \label{filt-central-p}$(\mathfrak{A};\mathfrak{Z})$ is a powerset
filtrator.
\item \label{filt-central-fltr}$(\mathfrak{A};\mathfrak{Z})$ is a primary
filtrator over a boolean lattice.
\item \label{filt-central-conc}The filtrator $(\mathfrak{A};\mathfrak{Z})$
is central.
\end{enumerate}
\end{thm}
\begin{proof}
~
\begin{description}
\item [{\ref{filt-central-p}$\Rightarrow$\ref{filt-central-fltr}}] Obvious.
\item [{\ref{filt-central-fltr}$\Rightarrow$\ref{filt-central-conc}}] We
can conclude that $\mathfrak{A}$ is atomically separable (the corollary
\ref{f-atom-sep}), with separable core (the theorem \ref{when-sep-core}),
and with join-closed core (theorem~\ref{semifilt-joinclosed}),
binarily meet-closed by corollary~\ref{f-meet-closed}.


We need to prove $Z(\mathfrak{A})=\mathfrak{Z}$.


Let $\mathcal{X}\in Z(\mathfrak{A})$. Then there exists $\mathcal{Y}\in Z(\mathfrak{A})$
such that $\mathcal{X}\sqcap^{\mathfrak{A}}\mathcal{Y}=\bot^{\mathfrak{A}}$
and $\mathcal{X}\sqcup^{\mathfrak{A}}\mathcal{Y}=\top^{\mathfrak{A}}$.
Consequently there is $X\in\up\mathcal{X}$ such that $X\sqcap^{\mathfrak{A}}\mathcal{Y}=\bot^{\mathfrak{A}}$;
we also have $X\sqcup^{\mathfrak{A}}\mathcal{Y}=\top^{\mathfrak{A}}$.
Suppose $X\sqsupset\mathcal{X}$. Then there exists $a\in\atoms^{\mathfrak{A}}X$
such that $a\notin\atoms^{\mathfrak{A}}\mathcal{X}$. We can conclude
also $a\notin\atoms^{\mathfrak{A}}\mathcal{Y}$ (otherwise $X\sqcap^{\mathfrak{A}}\mathcal{Y}\ne\bot^{\mathfrak{A}}$).
Thus $a\notin\atoms(\mathcal{X}\sqcup^{\mathfrak{A}}\mathcal{Y})$
and consequently $\mathcal{X}\sqcup^{\mathfrak{A}}\mathcal{Y}\ne\top^{\mathfrak{A}}$
what is a contradiction. We have $\mathcal{X}=X\in\mathfrak{Z}$.


Let now $X\in\mathfrak{Z}$. Let $Y=\overline{X}$. We have $X\sqcap^{\mathfrak{Z}}Y=\bot^{\mathfrak{A}}$
and $X\sqcup^{\mathfrak{Z}}Y=\top^{\mathfrak{A}}$. Thus $X\sqcap^{\mathfrak{A}}Y=\bigsqcap^{\mathfrak{Z}}\{X\sqcap^{\mathfrak{Z}}Y\}=\bot^{\mathfrak{A}}$;
$X\sqcap^{\mathfrak{A}}Y=X\sqcap^{\mathfrak{Z}}Y=\top^{\mathfrak{A}}$.
We have shown that $X\in Z(\mathfrak{A})$.

\end{description}
\end{proof}

\section{More Criteria}
\begin{thm}
\label{closed-free-star}The following is an implications tuple:
\begin{enumerate}
\item \label{closed-free-star-p}$(\mathfrak{A};\mathfrak{Z})$ is a powerset
filtrator.
\item \label{closed-free-star-fltr}$(\mathfrak{A};\mathfrak{Z})$ is a
primary filtrator over a boolean lattice.
\item \label{closed-free-star-conc}For every $S\in\subsets\mathfrak{A}$
the condition $\exists\mathcal{F}\in\mathfrak{A}:S=\fullstar\mathcal{F}$
is equivalent to conjunction of the following items:

\begin{enumerate}
\item \label{fs-star}$S$ is a free star on $\mathfrak{A}$;
\item \label{fs-fclos}$S$ is filter closed.
\end{enumerate}
\end{enumerate}
\end{thm}
\begin{proof}
~
\begin{description}
\item [{\ref{closed-free-star-p}$\Rightarrow$\ref{closed-free-star-fltr}}] Obvious.
\item [{\ref{closed-free-star-fltr}$\Rightarrow$\ref{closed-free-star-conc}}] ~

\begin{description}
\item [{$\Rightarrow$}] ~

\begin{widedisorder}
\item [{\ref{fs-star}}] That $\bot^{\mathfrak{A}}\notin\fullstar\mathcal{F}$
is obvious. For every $a,b\in\mathfrak{A}$
\begin{align*}
a\sqcup^{\mathfrak{A}}b\in\fullstar\mathcal{F} & \Leftrightarrow\\
(a\sqcup^{\mathfrak{A}}b)\sqcap^{\mathfrak{A}}\mathcal{F}\ne\bot^{\mathfrak{A}} & \Leftrightarrow\\
(a\sqcap^{\mathfrak{A}}\mathcal{F})\sqcup^{\mathfrak{A}}(b\sqcap^{\mathfrak{A}}\mathcal{F})\ne\bot^{\mathfrak{A}} & \Leftrightarrow\\
a\sqcap^{\mathfrak{A}}\mathcal{F}\ne\bot^{\mathfrak{A}}\lor b\sqcap^{\mathfrak{A}}\mathcal{F}\ne\bot^{\mathfrak{A}} & \Leftrightarrow\\
a\in\fullstar\mathcal{F}\lor\in\fullstar\mathcal{F}
\end{align*}
(taken into account corollary \ref{filt-also-distr}). So $\fullstar\mathcal{F}$
is a free star on $\mathfrak{A}$.
\item [{\ref{fs-fclos}}] We have a filter base $T\subseteq S$ and need to prove that
$\bigsqcap^{\mathfrak{A}}T\sqcap\mathcal{F}\ne\bot^{\mathfrak{A}}$.
Because $\rsupfun{\mathcal{F}\sqcap^{\mathfrak{A}}}T$ is a generalized
filter base, $\bot^{\mathfrak{A}}\in\rsupfun{\mathcal{F}\sqcap^{\mathfrak{A}}}T\Leftrightarrow\bigsqcap^{\mathfrak{A}}\rsupfun{\mathcal{F}\sqcap^{\mathfrak{A}}}T=\bot^{\mathfrak{A}}\Leftrightarrow\bigsqcap^{\mathfrak{A}}T\sqcap^{\mathfrak{A}}\mathcal{F}\ne\bot^{\mathfrak{A}}$.
So it is left to prove $\bot^{\mathfrak{A}}\notin\rsupfun{\mathcal{F}\sqcap^{\mathfrak{A}}}T$
what follows from $T\subseteq S$.
\end{widedisorder}
\item [{$\Leftarrow$}] Let $S$ be a free star on $\mathfrak{A}$. Then
for every $A,B\in\mathfrak{Z}$
\begin{align*}
A,B\in S\cap\mathfrak{Z} & \Leftrightarrow\\
A,B\in S & \Leftrightarrow\\
A\sqcup^{\mathfrak{A}}B\in S & \Leftrightarrow\\
A\sqcup^{\mathfrak{Z}}B\in S & \Leftrightarrow\\
A\sqcup^{\mathfrak{Z}}B\in S\cap\mathfrak{Z}
\end{align*}
(taken into account the theorem \ref{semifilt-joinclosed}). So $S\cap\mathfrak{Z}$
is a free star on $\mathfrak{Z}$.


Thus there exists $\mathcal{F}\in\mathfrak{A}$ such that $\corestar\mathcal{F}=S\cap\mathfrak{Z}$.
We have $\up\mathcal{X}\subseteq S\Leftrightarrow\mathcal{X}\in S$
(because $S$ is filter closed) for every $\mathcal{X}\in\mathfrak{A}$;
then (taking into account properties of generalized filter bases)
\begin{align*}
\mathcal{X}\in S & \Leftrightarrow\\
\up\mathcal{X}\subseteq S & \Leftrightarrow\\
\up\mathcal{X}\subseteq\corestar\mathcal{F} & \Leftrightarrow\\
\forall X\in\up\mathcal{X}:X\sqcap^{\mathfrak{A}}\mathcal{F}\ne\bot^{\mathfrak{A}} & \Leftrightarrow\\
\bot^{\mathfrak{A}}\notin\rsupfun{\mathcal{F}\sqcap^{\mathfrak{A}}}\up\mathcal{X} & \Leftrightarrow\\
\bigsqcap^{\mathfrak{A}}\rsupfun{\mathcal{F}\sqcap^{\mathfrak{A}}}\up\mathcal{X}\ne\bot^{\mathfrak{A}} & \Leftrightarrow\\
\mathcal{F}\sqcap^{\mathfrak{A}}\bigsqcap^{\mathfrak{A}}\up\mathcal{X}\ne\bot^{\mathfrak{A}} & \Leftrightarrow\\
\mathcal{F}\sqcap^{\mathfrak{A}}\mathcal{X}\ne\bot^{\mathfrak{A}} & \Leftrightarrow\\
\mathcal{X}\in\fullstar\mathcal{F}.
\end{align*}


\end{description}
\end{description}
\end{proof}

\section{Filters and a Special Sublattice}

Remind that $Z(X)$ is the center of lattice~$X$ and $Da$~is the lattice $\setcond{x\in\mathfrak{A}}{x\sqsubseteq a}$.

\begin{thm}
\label{core-if-intr}The following is an implications tuple:
\begin{enumerate}
\item \label{core-if-intr-p}$(\mathfrak{A};\mathfrak{Z})$ is a powerset
filtrator.
\item \label{core-if-intr-fltr}$(\mathfrak{A};\mathfrak{Z})$ is a primary
filtrator over a boolean lattice.
\item \label{core-if-intr-conc}Let $\mathcal{A}\in\mathfrak{A}$. Then
for each $\mathcal{X}\in\mathfrak{A}$
\[
\mathcal{X}\in Z(D\mathcal{A})\Leftrightarrow\exists X\in\mathfrak{Z}:\mathcal{X}=X\sqcap^{\mathfrak{A}}\mathcal{A}.
\]

\end{enumerate}
\end{thm}
\begin{proof}
~
\begin{description}
\item [{\ref{core-if-intr-p}$\Rightarrow$\ref{core-if-intr-fltr}}] Obvious.
\item [{\ref{core-if-intr-fltr}$\Rightarrow$\ref{core-if-intr-conc}}] ~

\begin{description}
\item [{$\Leftarrow$}] Let $\mathcal{X}=X\sqcap^{\mathfrak{A}}\mathcal{A}$
where $X\in\mathfrak{Z}$. Let also $\mathcal{Y}=\overline{X}\sqcap^{\mathfrak{A}}\mathcal{A}$.
Then $\mathcal{X}\sqcap^{\mathfrak{A}}\mathcal{Y}=X\sqcap^{\mathfrak{A}}\overline{X}\sqcap^{\mathfrak{A}}\mathcal{A}=(X\sqcap^{\mathfrak{Z}}\overline{X})\sqcap^{\mathfrak{A}}\mathcal{A}=\bot^{\mathfrak{A}}\sqcap^{\mathfrak{A}}\mathcal{A}=\bot^{\mathfrak{A}}$
(used corollary~\ref{f-meet-closed}) and $\mathcal{X}\sqcup^{\mathfrak{A}}\mathcal{Y}=(X\sqcup^{\mathfrak{A}}\overline{X})\sqcap^{\mathfrak{A}}\mathcal{A}=(X\sqcup^{\mathfrak{Z}}\overline{X})\sqcap^{\mathfrak{A}}\mathcal{A}=\top^{\mathfrak{A}}\sqcap^{\mathfrak{A}}\mathcal{A}=\mathcal{A}$
(used the theorems \ref{semifilt-joinclosed} and corollary \ref{filt-also-distr}).
So $\mathcal{X}\in Z(D\mathcal{A})$.
\item [{$\Rightarrow$}] Let $\mathcal{X}\in Z(D\mathcal{A})$. Then there
exists $\mathcal{Y}\in Z(D\mathcal{A})$ such that $\mathcal{X}\sqcap^{\mathfrak{A}}\mathcal{Y}=\bot^{\mathfrak{A}}$
and $\mathcal{X}\sqcup^{\mathfrak{A}}\mathcal{Y}=\mathcal{A}$. Then
(used theorem \ref{when-sep-core}) there exists $X\in\up\mathcal{X}$
such that $X\sqcap^{\mathfrak{A}}\mathcal{Y}=\bot^{\mathfrak{A}}$.
We have
\[
\mathcal{X}=\mathcal{X}\sqcup(X\sqcap^{\mathfrak{A}}\mathcal{Y})=X\sqcap^{\mathfrak{A}}(\mathcal{X}\sqcap^{\mathfrak{A}}\mathcal{Y})=X\sqcap^{\mathfrak{A}}\mathcal{A}.
\]

\end{description}
\end{description}
\end{proof}

\section{Complements and Core Parts}
\begin{prop}
For a primary filtrator over a complete boolean lattice both edge
part and dual edge part are always defined.\end{prop}
\begin{proof}
Core part and dual core part are defined because the core is a complete
lattice. Using the theorem \ref{b-bool-minus}.\end{proof}
\begin{thm}
\label{f-compl-meet}The following is an implications tuple:
\begin{enumerate}
\item \label{compl-meet-p}$(\mathfrak{A};\mathfrak{Z})$ is a powerset
filtrator.
\item \label{compl-meet-fltr}$(\mathfrak{A};\mathfrak{Z})$ is a primary
filtrator over a complete boolean lattice.
\item \label{compl-meet-conc}$(a\sqcap^{\mathfrak{A}}b)^{\ast}=a^{\ast}\sqcup^{\mathfrak{Z}}b^{\ast}$
for every $a,b\in\mathfrak{A}$.
\end{enumerate}
\end{thm}
\begin{proof}
~
\begin{description}
\item [{\ref{compl-meet-p}$\Rightarrow$\ref{compl-meet-fltr}}] Obvious.
\item [{\ref{compl-meet-fltr}$\Rightarrow$\ref{compl-meet-conc}}] $(\mathfrak{A};\mathfrak{Z})$
is a filtered (theorem~\ref{semifilt-joinclosed}) up-aligned complete
lattice filtrator with binarily join-closed (theorem \ref{semifilt-joinclosed})
co-separable core (theorem \ref{cosep-crit}) which is a complete
boolean lattice. Thus by the theorem \ref{cocompl-cor}
\[
(a\sqcap^{\mathfrak{A}}b)^{\ast}=(a\sqcap^{\mathfrak{A}}b)^{+}=\overline{\Cor(a\sqcap^{\mathfrak{A}}b)}=\overline{\Cor a\sqcap^{\mathfrak{Z}}\Cor b}=\overline{\Cor a}\sqcup^{\mathfrak{A}}\overline{\Cor b}=a^{+}\sqcup^{\mathfrak{Z}}b^{+}=a^{\ast}\sqcup^{\mathfrak{Z}}b^{\ast}
\]
(used propositions \ref{compl-eq-dual}, \ref{dual-cor-meet}, theorem~\ref{cor-eq}).
\end{description}
\end{proof}

\section{Complementive Filters and Factoring by a Filter}
\begin{defn}
Let $\mathfrak{A}$ be a meet-semilattice and $\mathcal{A}\in\mathfrak{A}$.
The relation~$\sim$ on~$\mathfrak{A}$ is defined by the formula
\[
\forall X,Y\in\mathfrak{A:}(X\sim Y\Leftrightarrow X\sqcap^{\mathfrak{A}}\mathcal{A}=Y\sqcap^{\mathfrak{A}}\mathcal{A}).
\]
\end{defn}
\begin{prop}
The relation $\sim$ is an equivalence relation.\end{prop}
\begin{proof}
~
\begin{description}
\item [{Reflexivity}] Obvious.
\item [{Symmetry}] Obvious.
\item [{Transitivity}] Obvious.
\end{description}
\end{proof}
\begin{defn}
When $X,Y\in\mathfrak{Z}$ and $\mathcal{A}\in\mathfrak{A}$ we define
$X\sim Y\Leftrightarrow\uparrow X\sim\uparrow Y$.\end{defn}
\begin{thm}
\label{eqrel-princ}The following is an implications tuple:
\begin{enumerate}
\item \label{eqrel-princ-p}$(\mathfrak{A};\mathfrak{Z})$ is a powerset
filtrator.
\item \label{eqrel-princ-fltr}$(\mathfrak{A};\mathfrak{Z})$ is a primary
filtrator over a distributive lattice.
\item \label{eqrel-princ-conc}For every $\mathcal{A}\in\mathfrak{A}$ and
$X,Y\in\mathfrak{Z}$ we have
\[
X\sim Y\Leftrightarrow\exists A\in\mathcal{A}:X\sqcap^{\mathfrak{Z}}A=Y\sqcap^{\mathfrak{Z}}A.
\]

\end{enumerate}
\end{thm}
\begin{proof}
~
\begin{description}
\item [{\ref{eqrel-princ-p}$\Rightarrow$\ref{eqrel-princ-fltr}}] Obvious.
\item [{\ref{eqrel-princ-fltr}$\Rightarrow$\ref{eqrel-princ-conc}}] ~
\begin{align*}
\exists A\in\up\mathcal{A}:X\sqcap^{\mathfrak{Z}}A=Y\sqcap^{\mathfrak{Z}}A & \Leftrightarrow \text{ (theorem~\ref{up-filt-crit})}\\
\exists A\in\up\mathcal{A}:\uparrow X\sqcap^{\mathfrak{A}}\uparrow A=\uparrow Y\sqcap^{\mathfrak{A}}\uparrow A & \Rightarrow\\
\exists A\in\up\mathcal{A}:\uparrow X\sqcap^{\mathfrak{A}}\uparrow A\sqcap^{\mathfrak{A}}\mathcal{A}=\uparrow Y\sqcap^{\mathfrak{A}}\uparrow A\sqcap^{\mathfrak{A}}\mathcal{A} & \Leftrightarrow\\
\exists A\in\up\mathcal{A}:\uparrow X\sqcap^{\mathfrak{A}}\mathcal{A}=\uparrow Y\sqcap^{\mathfrak{A}}\mathcal{A} & \Leftrightarrow\\
\uparrow X\sqcap^{\mathfrak{A}}\mathcal{A}=\uparrow Y\sqcap^{\mathfrak{A}}\mathcal{A} & \Leftrightarrow\\
\uparrow X\sim\uparrow Y & \Leftrightarrow\\
X\sim Y.
\end{align*}



On the other hand,
\begin{align*}
\uparrow X\sqcap^{\mathfrak{A}}\mathcal{A}=\uparrow Y\sqcap^{\mathfrak{A}}\mathcal{A} & \Leftrightarrow\\
\setcond{X\sqcap^{\mathfrak{Z}}A_{0}}{A_{0}\in\mathcal{A}}=\setcond{Y\sqcap^{\mathfrak{Z}}A_{1}}{A_{1}\in\mathcal{A}} & \Rightarrow\\
\exists A_{0},A_{1}\in\up\mathcal{A}:X\sqcap^{\mathfrak{Z}}A_{0}=Y\sqcap^{\mathfrak{Z}}A_{1} & \Rightarrow\\
\exists A_{0},A_{1}\in\up\mathcal{A}:X\sqcap^{\mathfrak{Z}}A_{0}\sqcap^{\mathfrak{Z}}A_{1}=Y\sqcap^{\mathfrak{Z}}A_{0}\sqcap^{\mathfrak{Z}}A_{1} & \Rightarrow\\
\exists A\in\up\mathcal{A}:Y\sqcap^{\mathfrak{Z}}A=X\sqcap^{\mathfrak{Z}}A.
\end{align*}


\end{description}
\end{proof}
\begin{prop}
The relation $\sim$ is a congruence\footnote{See Wikipedia for a definition of congruence.}
for each of the following:
\begin{enumerate}
\item \label{cong-mslat}a meet-semilattice $\mathfrak{A}$;
\item \label{cong-dist}a distributive lattice $\mathfrak{A}$.
\end{enumerate}
\end{prop}
\begin{proof}
Let $a_{0},a_{1},b_{0},b_{1}\in\mathfrak{A}$ and $a_{0}\sim a_{1}$
and $b_{0}\sim b_{1}$.
\begin{disorder}
\item [{\ref{cong-mslat}}] $a_{0}\sqcap b_{0}\sim a_{1}\sqcap b_{1}$because
$(a_{0}\sqcap b_{0})\sqcap\mathcal{A}=a_{0}\sqcap(b_{0}\sqcap\mathcal{A})=a_{0}\sqcap(b_{1}\sqcap\mathcal{A})=b_{1}\sqcap(a_{0}\sqcap\mathcal{A})=b_{1}\sqcap(a_{1}\sqcap\mathcal{A})=(a_{1}\sqcap b_{1})\sqcap\mathcal{A}$.
\item [{\ref{cong-dist}}] Taking the above into account, we need to prove
only $a_{0}\sqcup b_{0}\sim a_{1}\sqcup b_{1}$. We have
\[
(a_{0}\sqcup b_{0})\sqcap\mathcal{A}=(a_{0}\sqcap\mathcal{A})\sqcup(b_{0}\sqcap\mathcal{A})=(a_{1}\sqcap\mathcal{A})\sqcup(b_{1}\sqcap\mathcal{A})=(a_{1}\sqcup b_{1})\sqcap\mathcal{A}.
\]

\end{disorder}
\end{proof}
\begin{defn}
We will denote $A/(\sim)=A/((\sim)\cap A\times A)$ for a set $A$
and an equivalence relation $\sim$ on a set $B\supseteq A$. I will
call $\sim$ a congruence on $A$ when $(\sim)\cap(A\times A)$
is a congruence on $A$.\end{defn}
\begin{thm}
\label{factor-isomor}The following is an implications tuple:
\begin{enumerate}
\item \label{factor-isomor-p}$(\mathfrak{A};\mathfrak{Z})$ is a powerset
filtrator.
\item \label{factor-isomor-fltr}$(\mathfrak{A};\mathfrak{Z})$ is a primary
filtrator over a boolean lattice.
\item \label{factor-isomor-conc}Let $A\in\mathfrak{A}$. Consider the function
$\gamma:Z(D\mathcal{A})\rightarrow\mathfrak{Z}/\mathord\sim$ defined
by the formula (for every $p\in Z(D\mathcal{A})$)
\[
\gamma p=\setcond{X\in\mathfrak{Z}}{X\sqcap^{\mathfrak{A}}\mathcal{A}=p}.
\]



Then:
\begin{enumerate}
\item $\gamma$ is a lattice isomorphism.
\item $\forall Q\in q:\gamma^{-1}q=Q\sqcap^{\mathfrak{A}}\mathcal{A}$ for
every $q\in\mathfrak{Z}/\mathord\sim$.
\end{enumerate}
\end{enumerate}
\end{thm}
\begin{proof}
~
\begin{description}
\item [{\ref{factor-isomor-p}$\Rightarrow$\ref{factor-isomor-fltr}}] Obvious.
\item [{\ref{factor-isomor-fltr}$\Rightarrow$\ref{factor-isomor-conc}}] $\forall p\in Z(D\mathcal{A}):\gamma p\ne\emptyset$
because of theorem \ref{core-if-intr}. Thus it is easy to see that
$\gamma p\in\mathfrak{Z}/\mathord\sim$ and that $\gamma$ is an injection.


Let's prove that $\gamma$ is a lattice homomorphism:


$\gamma(p_{0}\sqcap^{\mathfrak{A}}p_{1})=\setcond{X\in\mathfrak{Z}}{X\sqcap^{\mathfrak{A}}\mathcal{A}=p_{0}\sqcap^{\mathfrak{A}}p_{1}}$;
\begin{align*}
\gamma p_{0}\sqcap^{\mathfrak{Z}/\mathord\sim}\gamma p_{1} & =\\
\setcond{X_{0}\in\mathfrak{Z}}{X_{0}\sqcap^{\mathfrak{A}}\mathcal{A}=p_{0}}\sqcap^{\mathfrak{Z}/\mathord\sim}\setcond{X_{1}\in\mathfrak{Z}}{X_{1}\sqcap^{\mathfrak{A}}\mathcal{A}=p_{1}} & =\\
\setcond{X_{0}\sqcap^{\mathfrak{A}}X_{1}}{X_{0},X_{1}\in\mathfrak{Z},X_{0}\sqcap^{\mathfrak{A}}\mathcal{A}=p_{0}\land X_{1}\sqcap^{\mathfrak{A}}\mathcal{A}=p_{1}} & \subseteq\\
\setcond{X'\in\mathfrak{Z}}{X'\sqcap^{\mathfrak{A}}\mathcal{A}=p_{0}\sqcap^{\mathfrak{A}}p_{1}} & =\\
\gamma(p_{0}\sqcap^{\mathfrak{A}}p_{1}).
\end{align*}



Because $\gamma p_{0}\sqcap^{\mathfrak{Z}/\mathord\sim}\gamma p_{1}$
and $\gamma(p_{0}\sqcap^{\mathfrak{A}}p_{1})$ are equivalence classes,
thus follows $\gamma p_{0}\sqcap^{\mathfrak{Z}/\mathord\sim}\gamma p_{1}=\gamma(p_{0}\sqcap^{\mathfrak{A}}p_{1})$.


To finish the proof it is enough to show that $\forall Q\in q:q=\gamma(Q\sqcap^{\mathfrak{A}}\mathcal{A})$
for every $q\in\mathfrak{Z}/\mathord\sim$. (From this it follows
that $\gamma$ is surjective because $q$ is not empty and thus $\exists Q\in q:q=\gamma(Q\sqcap^{\mathfrak{A}}\mathcal{A})$.)
Really,
\[
\gamma(Q\sqcap^{\mathfrak{A}}\mathcal{A})=\setcond{X\in\mathfrak{Z}}{X\sqcap^{\mathfrak{A}}\mathcal{A}=Q\sqcap^{\mathfrak{A}}\mathcal{A}}=[Q]=q.
\]


\end{description}
\end{proof}
This isomorphism is useful in both directions to reveal properties
of both lattices $Z(D\mathcal{A})$ and $q\in\mathfrak{Z}/\mathord\sim$.
\begin{cor}
\label{frac-is-bool}The following is an implications tuple:
\begin{enumerate}
\item $(\mathfrak{A};\mathfrak{Z})$ is a powerset filtrator.
\item $(\mathfrak{A};\mathfrak{Z})$ is a primary filtrator over a boolean
lattice.
\item $\mathfrak{Z}/\mathord\sim$ is a boolean lattice
\end{enumerate}
\end{cor}
\begin{proof}
Because $Z(D\mathcal{A})$ is a boolean lattice (theorem \ref{centr-bool}).
\end{proof}

\section{Pseudodifference of filters}
\begin{prop}
\label{filt-pseudo}The following is an implications tuple:
\begin{enumerate}
\item \label{filt-pseudo-p}$\mathfrak{A}$ is a lattice of filters on a
set.
\item \label{filt-pseudo-f}$\mathfrak{A}$ is a lattice of filters over
a boolean lattice.
\item \label{filt-pseudo-atcbr}$\mathfrak{A}$ is an atomistic co-brouwerian
lattice.
\item \label{filt-pseudo-conc}For every $a,b\in\mathfrak{A}$ the following
expressions are always equal:

\begin{enumerate}
\item $a\psetminus b=\bigsqcap\setcond{z\in\mathfrak{A}}{a\sqsubseteq b\sqcup z}$
(quasidifference of $a$ and $b$);
\item $a\mathop\#b=\bigsqcup\setcond{z\in\mathfrak{A}}{z\sqsubseteq a\land z\sqcap b=\bot}$
(second quasidifference of $a$ and $b$);
\item $\bigsqcup(\atoms a\setminus\atoms b)$.
\end{enumerate}
\end{enumerate}
\end{prop}
\begin{proof}
~
\begin{description}
\item [{\ref{filt-pseudo-p}$\Rightarrow$\ref{filt-pseudo-f}}] Obvious.
\item [{\ref{filt-pseudo-f}$\Rightarrow$\ref{filt-pseudo-atcbr}}] By
corollary \ref{filt-also-distr} theorem \ref{filt-atomistic}.
\item [{\ref{filt-pseudo-atcbr}$\Rightarrow$\ref{filt-pseudo-conc}}] Theorem
\ref{pdiff-eq1}.\end{description}
\begin{conjecture}
$a\psetminus b=a\mathop\#b$ for arbitrary filters $a$, $b$ on powersets
is not provable in ZF (without axiom of choice).
\end{conjecture}
\end{proof}

\section{Function spaces of posets}
\begin{defn}
Let $\mathfrak{A}_{i}$ be a family of posets indexed by some set
$\dom\mathfrak{A}$. We will define order of families of elements of posets by
the formula 
\[
a\sqsubseteq b\Leftrightarrow\forall i\in\dom\mathfrak{A}:a_{i}\sqsubseteq b_{i}.
\]
\index{function space of posets}\index{product order}I will call
this new poset $\prod\mathfrak{A}$ \emph{the function
space} of posets and the above order \emph{product order}.\end{defn}
\begin{prop}
The function space for posets is also a poset.\end{prop}
\begin{proof}
~
\begin{description}
\item [{Reflexivity}] Obvious.
\item [{Antisymmetry}] Obvious.
\item [{Transitivity}] Obvious.
\end{description}
\end{proof}
\begin{obvious}
$\mathfrak{A}$ has least element iff each $\mathfrak{A}_{i}$ has
a least element. In this case
\[
\bot^{\prod\mathfrak{A}}=\prod_{i\in\dom\mathfrak{A}}\bot^{\mathfrak{A}_{i}}.
\]
\end{obvious}
\begin{prop}
$a\nasymp b\Leftrightarrow\exists i\in\dom\mathfrak{A}:a_{i}\nasymp b_{i}$
for every $a,b\in\prod\mathfrak{A}$ if every $\mathfrak{A}_{i}$
has least element.\end{prop}
\begin{proof}
If $\dom\mathfrak{A}=\emptyset$, then $a=b=\bot$, $a\asymp b$ and
thus the theorem statement holds. Assume $\dom\mathfrak{A}\neq\emptyset$.
\begin{align*}
a\nasymp b & \Leftrightarrow\\
\exists c\in\prod\mathfrak{A}\setminus\{\bot^{\prod\mathfrak{A}}\}:(c\sqsubseteq a\land c\sqsubseteq b) & \Leftrightarrow\\
\exists c\in\prod\mathfrak{A}\setminus\{\bot^{\prod\mathfrak{A}}\}\forall i\in\dom\mathfrak{A}:(c_{i}\sqsubseteq a_{i}\land c_{i}\sqsubseteq b_{i}) & \Leftrightarrow\\
\text{(for the reverse implication take \ensuremath{c_{j}=\bot^{\mathfrak{A}_{j}}} for \ensuremath{i\ne j})}\\
\exists i\in\dom\mathfrak{A},c\in\mathfrak{A}_{i}\setminus\{\bot^{\mathfrak{A}_{i}}\}:(c\sqsubseteq a_{i}\land c\sqsubseteq b_{i}) & \Leftrightarrow\\
\exists i\in\dom\mathfrak{A}:a_{i}\nasymp b_{i}.
\end{align*}
\end{proof}
\begin{prop}
~
\begin{enumerate}
\item If $\mathfrak{A}_{i}$ are join-semilattices then $\mathfrak{A}$
is a join-semilattice and 
\begin{equation}
A\sqcup B=\mylamdba i{\dom\mathfrak{A}}{Ai\sqcup Bi}.\label{func-union}
\end{equation}

\item If $\mathfrak{A}_{i}$ are meet-semilattices then $\mathfrak{A}$
is a meet-semilattice and 
\[
A\sqcap B=\mylamdba i{\dom\mathfrak{A}}{Ai\sqcap Bi}.
\]

\end{enumerate}
\end{prop}
\begin{proof}
It is enough to prove the formula (\ref{func-union}).

It's obvious that $\mylamdba i{\dom\mathfrak{A}}{Ai\sqcup Bi}\sqsupseteq A,B$.

Let $C\sqsupseteq A,B$. Then (for every $i\in\dom\mathfrak{A}$)
$Ci\sqsupseteq Ai$ and $Ci\sqsupseteq Bi$. Thus $Ci\sqsupseteq Ai\sqcup Bi$
that is $C\sqsupseteq\mylamdba i{\dom\mathfrak{A}}{Ai\sqcup Bi}$.\end{proof}
\begin{cor}
If $\mathfrak{A}_{i}$ are lattices then $\prod\mathfrak{A}$ is a lattice.\end{cor}
\begin{obvious}
If $\mathfrak{A}_{i}$ are distributive lattices then $\prod\mathfrak{A}$
is a distributive lattice.\end{obvious}
\begin{prop}
If $\mathfrak{A}_{i}$ are boolean lattices then $\prod\mathfrak{A}$
is a boolean lattice.\end{prop}
\begin{proof}
We need to prove only that every element $a\in\prod\mathfrak{A}$
has a complement. But this complement is evidently $\mylamdba i{\dom a}{\overline{a_{i}}}$.\end{proof}
\begin{prop}
If every $\mathfrak{A}_{i}$ is a poset then for every $S\in\subsets\prod\mathfrak{A}$
\begin{enumerate}
\item $\bigsqcup S=\mylamdba i{\dom\mathfrak{A}}{\bigsqcup_{x\in S}x_{i}}$
whenever every $\bigsqcup_{x\in S}x_{i}$ exists;
\item $\bigsqcap S=\mylamdba i{\dom\mathfrak{A}}{\bigsqcap_{x\in S}x_{i}}$
whenever every $\bigsqcap_{x\in S}x_{i}$ exists.
\end{enumerate}
\end{prop}
\begin{proof}
It's enough to prove the first formula.

$\left(\mylamdba i{\dom\mathfrak{A}}{\bigsqcap_{x\in S}x_{i}}\right)_{i}=\bigsqcup_{x\in S}x_{i}\sqsupseteq x_{i}$
for every $x\in S$ and $i\in\dom\mathfrak{A}$.

Let $y\sqsupseteq x$ for every $x\in S$. Then $y_{i}\sqsupseteq x_{i}$
for every $i\in\dom\mathfrak{A}$ and thus $y_{i}\sqsupseteq\bigsqcup_{x\in S}x_{i}=\left(\mylamdba i{\dom\mathfrak{A}}{\bigsqcup_{x\in S}x_{i}}\right)_{i}$
that is $y\sqsupseteq\mylamdba i{\dom\mathfrak{A}}{\bigsqcup_{x\in S}x_{i}}$.

Thus $\bigsqcup S=\mylamdba i{\dom\mathfrak{A}}{\bigsqcup_{x\in S}x_{i}}$
by the definition of join.\end{proof}
\begin{cor}
\label{prod-join2}If $\mathfrak{A}_{i}$ are posets then for every
$S\in\subsets\prod\mathfrak{A}$
\begin{enumerate}
\item $\bigsqcup S=\mylamdba i{\dom\mathfrak{A}}{\bigsqcup_{x\in S}x_{i}}$
whenever $\bigsqcup S$ exists;
\item $\bigsqcap S=\mylamdba i{\dom\mathfrak{A}}{\bigsqcap_{x\in S}x_{i}}$
whenever $\bigsqcap S$ exists.
\end{enumerate}
\end{cor}
\begin{proof}
It is enough to prove that (for every $i$) $\bigsqcup_{x\in S}x_{i}$
exists whenever $\bigsqcup S$ exists.

Fix $i\in\dom\mathfrak{A}$.

Take $y_{i}=\left(\bigsqcup S\right)_{i}$ and let prove that $y_{i}$
is the least upper bound of $\bigsqcup_{x\in S}x_{i}$.

$y_{i}$ is an upper bound of $\bigsqcup_{x\in S}x_{i}$ because $\bigsqcup S\sqsupseteq x$
and thus $\left(\bigsqcup S\right)_{i}\sqsupseteq x_{i}$ for every
$x\in S$.

Let $x\in S$ and for some $t\in\mathfrak{A}_{i}$
\[
T(t)=\mylamdba j{\dom\mathfrak{A}}{\begin{cases}
t & \text{if }i=j\\
x_{i} & \text{if }i\ne j.
\end{cases}}
\]
Let $t\sqsupseteq x_{i}$. Then $T(t)\sqsupseteq x$ for every $x\in S$.
So $T(t)\sqsupseteq\bigsqcup S$ and consequently $t=T(t)_{i}\sqsupseteq y_{i}$.

So $y_{i}$ is the least upper bound of $\setcond{x_{i}}{x\in S}$.\end{proof}
\begin{cor}
If $\mathfrak{A}_{i}$ are complete lattices then $\mathfrak{A}$
is a complete lattice.\end{cor}
\begin{obvious}
If $\mathfrak{A}_{i}$ are complete (co-)brouwerian lattices then
$\mathfrak{A}$ is a (co-)brouwerian lattice.\end{obvious}
\begin{prop}
If each $\mathfrak{A}_{i}$ is a separable poset with least element
(for some index set $n$) then $\prod\mathfrak{A}$ is a separable
poset.\end{prop}
\begin{proof}
Let $a\neq b$. Then $\exists i\in\dom\mathfrak{A}:a_{i}\neq b_{i}$.
So $\exists x\in\mathfrak{A}_{i}:(x\nasymp a_{i}\wedge x\asymp b_{i})$
(or vice versa).

Take $y=(((\dom\mathfrak{A})\setminus\{i\})\times\{\bot\})\cup\{(i;x)\}$.
Then $y\nasymp a$ and $y\asymp b$.\end{proof}
\begin{obvious}
If every $\mathfrak{A}_{i}$ is a poset with least element $\bot_{i}$,
then the set of atoms of $\prod\mathfrak{A}$ is
\[
\setcond{(\{k\}\times\atoms^{\mathfrak{A}_{k}})\cup(\mylamdba i{(\dom\mathfrak{A})\setminus\{k\}}{\bot_{i}})}{k\in\dom\mathfrak{A}}.
\]
\end{obvious}
\begin{prop}
If every $\mathfrak{A}_{i}$ is an atomistic poset with least element
$\bot_{i}$, then $\prod\mathfrak{A}$ is an atomistic poset.\end{prop}
\begin{proof}
$x_{i}=\bigsqcup\atoms x_{i}$ for every $x_{i}\in\mathfrak{A}_{i}$.
Thus
\[
x=\mylamdba i{\dom x}{x_{i}}=\mylamdba i{\dom x}{\bigsqcup\atoms x_{i}=\bigsqcup_{i\in\dom x}\mylamdba j{\dom x}{\begin{cases}
x_{i} & \text{if }j=i\\
\bot_{i} & \text{if }j\ne i.
\end{cases}}}
\]
 Take join two times.\end{proof}
\begin{cor}
If $\mathfrak{A}_{i}$ are atomistic posets with least elements, then
$\prod\mathfrak{A}$ is atomically separable.\end{cor}
\begin{proof}
Proposition \ref{atms-is-asep}.\end{proof}
\begin{prop}
Let $(\mathfrak{A}_{i\in n};\mathfrak{Z}_{i\in n})$ be a family of
filtrators. Then $\left(\prod\mathfrak{A};\prod\mathfrak{Z}\right)$
is a filtrator.\end{prop}
\begin{proof}
We need to prove that $\prod\mathfrak{Z}$ is a sub-poset of $\prod\mathfrak{A}$.
First $\prod\mathfrak{Z}\subseteq\prod\mathfrak{A}$ because $\mathfrak{Z}_{i}\subseteq\mathfrak{A}_{i}$
for each $i\in n$.

Let $A,B\in\prod\mathfrak{Z}$ and $A\sqsubseteq^{\prod\mathfrak{Z}}B$.
Then $\forall i\in n:A_{i}\sqsubseteq^{\mathfrak{Z}_{i}}B_{i}$; consequently
$\forall i\in n:A_{i}\sqsubseteq^{\mathfrak{A}_{i}}B_{i}$ that is
$A\sqsubseteq^{\prod\mathfrak{A}}B$.\end{proof}
\begin{prop}
Let $(\mathfrak{A}_{i\in n};\mathfrak{Z}_{i\in n})$ be a family of
filtrators.
\begin{enumerate}
\item The filtrator $\left(\prod\mathfrak{A};\prod\mathfrak{Z}\right)$
is (binarily) join-closed if every $(\mathfrak{A}_{i};\mathfrak{Z}_{i})$
is (binarily) join-closed.
\item The filtrator $\left(\prod\mathfrak{A};\prod\mathfrak{Z}\right)$
is (binarily) meet-closed if every $(\mathfrak{A}_{i};\mathfrak{Z}_{i})$
is (binarily) meet-closed.
\end{enumerate}
\end{prop}
\begin{proof}
Let every $(\mathfrak{A}_{i};\mathfrak{Z}_{i})$ be binarily join-closed.
Let $A,B\in\prod\mathfrak{Z}$ and $A\sqcup^{\prod\mathfrak{Z}}B$
exist. Then (by corollary \ref{prod-join2}) 
\[
A\sqcup^{\prod\mathfrak{Z}}B=\mylamdba in{A_{i}\sqcup^{\mathfrak{Z}_{i}}B_{i}=}\mylamdba in{A_{i}\sqcup^{\mathfrak{A}_{i}}B_{i}=}A\sqcup^{\prod\mathfrak{A}}B.
\]


Let now every $(\mathfrak{A}_{i};\mathfrak{Z}_{i})$ be join-closed.
Let $S\in\subsets\prod\mathfrak{Z}$ and $\bigsqcup^{\prod\mathfrak{Z}}S$
exist. Then (by corollary \ref{prod-join2}) 
\[
\bigsqcup^{\prod\mathfrak{Z}}S=\mylamdba i{\dom\mathfrak{A}}{\bigsqcup^{\mathfrak{Z}_{i}}\setcond{x_{i}}{x\in S}}=\mylamdba i{\dom\mathfrak{A}}{\bigsqcup^{\mathfrak{A}_{i}}\setcond{x_{i}}{x\in S}}=\bigsqcup^{\prod\mathfrak{A}}S.
\]


The rest follows from symmetry.\end{proof}
\begin{prop}
If each $(\mathfrak{A}_{i};\mathfrak{Z}_{i})$ where $i\in n$ (for
some index set $n$) is a down-aligned filtrator with separable core
then $\left(\prod\mathfrak{A};\prod\mathfrak{Z}\right)$ is with separable
core.\end{prop}
\begin{proof}
Let $a\neq b$. Then $\exists i\in n:a_{i}\neq b_{i}$. So $\exists x\in\mathfrak{Z}_{i}:(x\nasymp a_{i}\wedge x\asymp b_{i})$
(or vice versa).

Take $y=((n\setminus\{i\})\times\{\bot\})\cup\{(i;x)\}$. Then we have
$y\nasymp a$ and $y\asymp b$ and $y\in\mathfrak{Z}$.\end{proof}
\begin{prop}
Let every $\mathfrak{A}_{i}$ be a bounded lattice. Every $(\mathfrak{A}_{i};\mathfrak{Z}_{i})$
is a central filtrator iff $\left(\prod\mathfrak{A};\prod\mathfrak{Z}\right)$
is a central filtrator.\end{prop}
\begin{proof}
~
\begin{align*}
x\in Z\left(\prod\mathfrak{A}\right) & \Leftrightarrow\\
\exists y\in\prod\mathfrak{A}:(x\sqcap y=\bot^{\prod\mathfrak{A}}\land x\sqcup y=\top^{\prod\mathfrak{A}}) & \Leftrightarrow\\
\exists y\in\prod\mathfrak{A}\forall i\in\dom\mathfrak{A}:(x_{i}\sqcap y_{i}=\bot^{\mathfrak{A}_{i}}\land x_{i}\sqcup y_{i}=\top^{\mathfrak{A}_{i}}) & \Leftrightarrow\\
\forall i\in\dom\mathfrak{A}\exists y\in\mathfrak{A}_{i}:(x_{i}\sqcap y=\bot^{\mathfrak{A}_{i}}\land x_{i}\sqcup y=\top^{\mathfrak{A}_{i}}) & \Leftrightarrow\\
\forall i\in\dom\mathfrak{A}:x_{i}\in Z(\mathfrak{A}_{i}).
\end{align*}


So 
\begin{multline*}
Z\left(\prod\mathfrak{A}\right)=\prod\mathfrak{Z}\Leftrightarrow\prod_{i\in\dom\mathfrak{A}}Z(\mathfrak{A}_{i})=\prod\mathfrak{Z}\Leftrightarrow\\
\text{(because every \ensuremath{\mathfrak{Z}_{i}} is nonempty)}\Leftrightarrow\forall i\in\dom\mathfrak{A}:Z(\mathfrak{A}_{i})=\mathfrak{Z}_{i}.
\end{multline*}
\end{proof}
\begin{prop}
For every element $a$ of a product filtrator $\left(\prod\mathfrak{A};\prod\mathfrak{Z}\right)$:
\begin{enumerate}
\item $\up a=\prod_{i\in\dom a}\up a_{i}$;
\item $\down a=\prod_{i\in\dom a}\down a_{i}$.
\end{enumerate}
\end{prop}
\begin{proof}
We will prove only the first as the second is dual.
\begin{multline*}
\up a=\setcond{c\in\prod\mathfrak{Z}}{c\sqsupseteq a}=\setcond{c\in\prod\mathfrak{Z}}{\forall i\in\dom a:c_{i}\sqsupseteq a_{i}}=\\
\setcond{c\in\prod\mathfrak{Z}}{\forall i\in\dom a:c_{i}\in\up a_{i}}=\prod_{i\in\dom a}\up a_{i}.
\end{multline*}
\end{proof}
\begin{prop}
If every $(\mathfrak{A}_{i\in n};\mathfrak{Z}_{i\in n})$ is a prefiltered
filtrator with $\up x\neq\emptyset$ for every $x\in\mathfrak{A}_{i}$
(for every $i\in n$), then $\left(\prod\mathfrak{A};\prod\mathfrak{Z}\right)$
is a prefiltered lattice filtrator.\end{prop}
\begin{proof}
Let $a,b\in\prod\mathfrak{A}$ and $a\neq b$. Then there exists $i\in n$
such that $a_{i}\neq b_{i}$ and so $\up a_{i}\neq\up b_{i}$. Consequently
$\prod_{i\in\dom a}\up a_{i}\neq\prod_{i\in\dom a}\up b_{i}$ (taken
into account that $\up x\neq\emptyset$ for every $x\in\mathfrak{A}_{i}$)
that is $\up a\neq\up b$.\end{proof}
\begin{prop}
Let every $(\mathfrak{A}_{i\in n};\mathfrak{Z}_{i\in n})$ be a filtered
filtrator with $\up x\neq\emptyset$ for every $x\in\mathfrak{A}_{i}$
(for every $i\in n$). Then $\left(\prod\mathfrak{A};\prod\mathfrak{Z}\right)$
is a filtered filtrator.\end{prop}
\begin{proof}
Let every $(\mathfrak{A}_{i};\mathfrak{Z}_{i})$ be a filtered filtrator.
Let $\up a\supseteq\up b$ for some $a,b\in\prod\mathfrak{A}$. Then
$\prod_{i\in\dom a}\up a_{i}\supseteq\prod_{i\in\dom a}\up b_{i}$
and consequently (taking into account that $\up x\neq\emptyset$ for
every $x\in\mathfrak{A}_{i}$) $\up a_{i}\supseteq\up b_{i}$ for
every $i\in n$. Then $\forall i\in n:a_{i}\sqsubseteq b_{i}$ that
is $a\sqsubseteq b$.\end{proof}
\begin{prop}
Let $(\mathfrak{A}_{i};\mathfrak{Z}_{i})$ be filtrators and each
$\mathfrak{Z}_{i}$ be a complete lattice with $\up x\neq\emptyset$
for every $x\in\mathfrak{A}_{i}$ (for every $i\in n$). For $a\in\prod\mathfrak{A}$:
\begin{enumerate}
\item $\Cor a=\mylamdba i{\dom a}{\Cor a_{i}}$;
\item $\Cor'a=\mylamdba i{\dom a}{\Cor'a_{i}}$.
\end{enumerate}
\end{prop}
\begin{proof}
We will prove only the first, because the second is dual.
\begin{align*}
\Cor a & =\\
\bigsqcap^{\prod\mathfrak{Z}}\up a & =\\
\mylamdba i{\dom a}{\bigsqcap^{\mathfrak{Z}_{i}}\setcond{x_{i}}{x\in\up a}} & =\text{(\ensuremath{\up x\ne\emptyset} taken into account)}\\
\mylamdba i{\dom a}{\bigsqcap^{\mathfrak{Z}_{i}}\setcond x{x\in\up a_{i}}} & =\\
\mylamdba i{\dom a}{\bigsqcap^{\mathfrak{Z}_{i}}\up a_{i}} & =\\
\mylamdba i{\dom a}{\Cor a_{i}}.
\end{align*}
\end{proof}
\begin{prop}
If each $(\mathfrak{A}_{i};\mathfrak{Z}_{i})$ is a filtrator with
(co)separable core and each $\mathfrak{A}_{i}$ has a least (greatest)
element, then $\left(\prod\mathfrak{A};\prod\mathfrak{Z}\right)$
is a filtrator with (co)separable core.\end{prop}
\begin{proof}
We will prove only for separable core, as co-separable core is dual.
\begin{align*}
x\asymp^{\prod\mathfrak{A}}y & \Leftrightarrow\\
\text{(used the fact that \ensuremath{\mathfrak{A}_{i}} has a least element)}\\
\forall i\in\dom\mathfrak{A}:x_{i}\asymp^{\mathfrak{A}_{i}}y_{i} & \Rightarrow\\
\forall i\in\dom\mathfrak{A}\exists X\in\up x_{i}:X\asymp^{\mathfrak{A}_{i}}y_{i} & \Leftrightarrow\\
\exists X\in\up x\forall i\in\dom\mathfrak{A}:X_{i}\asymp^{\mathfrak{A}_{i}}y_{i} & \Leftrightarrow\\
\exists X\in\up x:X\asymp^{\prod\mathfrak{A}}y
\end{align*}
for every $x,y\in\prod\mathfrak{A}$.\end{proof}
\begin{obvious}
~
\begin{enumerate}
\item If each $(\mathfrak{A}_{i};\mathfrak{Z}_{i})$ is a down-aligned filtrator,
then $\left(\prod\mathfrak{A};\prod\mathfrak{Z}\right)$ is a down-aligned
filtrator.
\item If each $(\mathfrak{A}_{i};\mathfrak{Z}_{i})$ is an up-aligned filtrator,
then $\left(\prod\mathfrak{A};\prod\mathfrak{Z}\right)$ is an up-aligned
filtrator.
\end{enumerate}
\end{obvious}
\begin{prop}
If every $b_{i}$ is substractive from $a_{i}$ where $a$ and $b$
are $n$-indexed families of elements of distributive lattices with least elements
(where $n$ is an index set), then $a\setminus b=\mylamdba in{a_{i}\setminus b_{i}}$.\end{prop}
\begin{proof}
We need to prove $(\mylamdba in{a_{i}\setminus b_{i}})\sqcap b=\bot$
and $a\sqcup b=b\sqcup(\mylamdba in{a_{i}\setminus b_{i}})$. Really
\begin{gather*}
(\mylamdba in{a_{i}\setminus b_{i}})\sqcap b=\mylamdba in{(a_{i}\setminus b_{i})\sqcap b_{i}}=\bot;\\
b\sqcup(\mylamdba in{a_{i}\setminus b_{i}})=\mylamdba in{b_{i}\sqcup(a_{i}\setminus b_{i})}=\mylamdba in{b_{i}\sqcup a_{i}=a\sqcup b}.
\end{gather*}
\end{proof}
\begin{prop}
If every $\mathfrak{A}_{i}$ is a distributive lattice, then $a\setminus^{\ast}b=\mylamdba i{\dom\mathfrak{A}}{a_{i}\setminus^{\ast}b_{i}}$
for every $a,b\in\prod\mathfrak{A}$ whenever every $a_{i}\setminus^{\ast}b_{i}$
is defined.\end{prop}
\begin{proof}
We need to prove that $\mylamdba i{\dom\mathfrak{A}}{a_{i}\setminus^{\ast}b_{i}}=\bigsqcap\setcond{z\in\prod\mathfrak{A}}{a\sqsubseteq b\sqcup z}$.

To prove it is enough to show $a_{i}\setminus^{\ast}b_{i}=\bigsqcap\setcond{z_{i}}{z\in\prod\mathfrak{A},a\sqsubseteq b\sqcup z}$
that is $a_{i}\setminus^{\ast}b_{i}=\bigsqcap\setcond{z\in\mathfrak{A}_{i}}{a_{i}\sqsubseteq b_{i}\sqcup z}$
(for the reverse implication take $z_{j}=a_{i}$ for $j\neq i$) what
is true by definition.\end{proof}
\begin{prop}
If every $\mathfrak{A}_{i}$ is a distributive lattice with least
element, then $a\mathbin\#b=\lambda i\in\dom\mathfrak{A}:a_{i}\mathbin\#b_{i}$
for every $a,b\in\prod\mathfrak{A}$ whenever every $a_{i}\mathbin\#b_{i}$
is defined.\end{prop}
\begin{proof}
We need to prove that $\lambda i\in\dom\mathfrak{A}:a_{i}\mathbin\#b_{i}=\bigsqcup\setcond{z\in\prod\mathfrak{A}}{z\sqsubseteq a\land z\asymp b}$.

To prove it is enough to show $a_{i}\mathbin\#b_{i}=\bigsqcup\setcond{z_{i}}{z\in\prod\mathfrak{A},z\sqsubseteq a\land z\asymp b}$
that is $a_{i}\mathbin\#b_{i}=\bigsqcup\setcond{z\in\mathfrak{A}_{i}}{z\sqsubseteq a_{i}\land\forall j\in\dom\mathfrak{A}:z_{j}\asymp b_{j}}$
that is $a_{i}\mathbin\#b_{i}=\bigsqcup\setcond{z\in\mathfrak{A}_{i}}{z\sqsubseteq a_{i}\land z\asymp b_{i}}$
(take $z_{j}=\bot$ for $j\ne i$) what is true by definition.\end{proof}
\begin{prop}
Let every $\mathfrak{A}_{i}$ be a poset with least element and $a_{i}^{\ast}$
is defined. Then $a^{\ast}=\mylamdba i{\dom\mathfrak{A}}{a_{i}^{\ast}}$.\end{prop}
\begin{proof}
We need to prove that $\mylamdba i{\dom\mathfrak{A}}{a_{i}^{\ast}}=\bigsqcup\setcond{c\in\mathfrak{A}}{c\asymp a}$.
To prove this it is enough to show that $a_{i}^{\ast}=\bigsqcup\setcond{c_{i}}{c\in\prod\mathfrak{A},c\asymp a}$
that is $a_{i}^{\ast}=\bigsqcup\setcond{c_{i}}{c\in\prod\mathfrak{A},\forall j\in\dom\mathfrak{A}:c_{j}\asymp a_{j}}$
that is $a_{i}^{\ast}=\bigsqcup\setcond{c_{i}}{c\in\prod\mathfrak{A},c_{i}\asymp a_{i}}$
(take $c_{j}=\bot$ for $j\neq i$) that is $a_{i}^{\ast}=\bigsqcup\setcond{c\in\mathfrak{A}_{i}}{c\asymp a_{i}}$
what is true by definition.\end{proof}
\begin{cor}
Let every $\mathfrak{A}_{i}$ be a poset with greatest element and
$a_{i}^{+}$ is defined. Then $a^{+}=\mylamdba i{\dom\mathfrak{A}}{a_{i}^{+}}$.\end{cor}
\begin{proof}
By duality.
\end{proof}

\section{Filters on a Set}

In this section we will fix a powerset filtrator~$(\mathfrak{A};\mathfrak{Z})=(\mathfrak{A};\subsets\mathfrak{U})$
for some set~$\mathfrak{U}$.

The consideration below is about filters on a set $\mathfrak{U}$,
but this can be generalized for filters on complete atomic boolean
algebras due complete atomic boolean algebras are isomorphic to algebras
of sets on some set $\mathfrak{U}$.


\subsection{Fr\'echet Filter}
\begin{defn}
\index{filter!Fr\'echet}\index{filter!cofinite}$\Omega=\setcond{\mathfrak{U}\setminus X}{X\text{ is a finite subset of }\mathfrak{U}}$
is called either \emph{Fr\'echet filter} or \emph{cofinite filter}.
\end{defn}
It is trivial that Fr\'echet filter is a filter.
\begin{prop}
$\Cor\Omega=\bot^{\mathfrak{Z}}$; $\bigcap\Omega=\emptyset$.\end{prop}
\begin{proof}
This can be deduced from the formula $\forall\alpha\in\mathfrak{U}\exists X\in\Omega:\alpha\notin X$.\end{proof}
\begin{thm}
$\max\setcond{\mathcal{X}\in\mathfrak{A}}{\Cor\mathcal{X}=\bot^{\mathfrak{Z}}}=\max\setcond{\mathcal{X}\in\mathfrak{A}}{\bigcap\mathcal{X}=\emptyset}=\Omega$.\end{thm}
\begin{proof}
Due the last proposition, it is enough to show that $\Cor\mathcal{X}=\bot^{\mathfrak{Z}}\Rightarrow\mathcal{X}\sqsubseteq\Omega$
for every filter $\mathcal{X}$.

Let $\Cor\mathcal{X}=\bot^{\mathfrak{Z}}$ for some filter $\mathcal{X}$.
Let $X\in\Omega$. We need to prove that $X\in\mathcal{X}$.

$X=\mathfrak{U}\setminus\{\alpha_{0},\dots,\alpha_{n}\}$. $\mathfrak{U}\setminus\{\alpha_{i}\}\in\mathcal{X}$
because otherwise $\alpha_{i}\in\uparrow^{-1}\Cor\mathcal{X}$. So
$X\in\mathcal{X}$.\end{proof}
\begin{thm}
$\Omega=\bigsqcup^{\mathfrak{A}}\setcond x{x\text{ is a non-trivial ultrafilter}}$.\end{thm}
\begin{proof}
It follows from the facts that $\Cor x=\bot^{\mathfrak{Z}}$ for every
non-trivial ultrafilter $x$, that $\mathfrak{A}$ is an atomistic
lattice, and the previous theorem.\end{proof}
\begin{thm}
$\Cor$ is the lower adjoint of $\Omega\sqcup^{\mathfrak{A}}\mathord-$.\end{thm}
\begin{proof}
Because both $\Cor$ and $\Omega\sqcup^{\mathfrak{A}}\mathord-$ are
monotone, it is enough (theorem \ref{galois-second}) to prove (for
every filters $\mathcal{X}$ and $\mathcal{Y}$)
\[
\mathcal{X}\sqsubseteq\Omega\sqcup^{\mathfrak{A}}\Cor\mathcal{X}\quad\text{and}\quad\Cor(\Omega\sqcup^{\mathfrak{A}}\mathcal{Y})\sqsubseteq\mathcal{Y}.
\]


$\Cor(\Omega\sqcup^{\mathfrak{A}}\mathcal{Y})=\Cor\Omega\sqcup^{\mathfrak{Z}}\Cor\mathcal{Y}=\bot^{\mathfrak{Z}}\sqcup^{\mathfrak{Z}}\Cor\mathcal{Y}=\Cor\mathcal{Y}\sqsubseteq\mathcal{Y}$.
\end{proof}
$\Omega\sqcup^{\mathfrak{A}}\Cor\mathcal{X}\sqsupseteq\Edg\mathcal{X}\sqcup^{\mathfrak{A}}\Cor\mathcal{X}=\mathcal{X}$.
\begin{cor}
$\Cor\mathcal{X}=\mathcal{X}\psetminus\Omega$ for every filter on
a set.\end{cor}
\begin{proof}
By theorem \ref{cobrow-adj}.\end{proof}
\begin{cor}
$\Cor\bigsqcup^{\mathfrak{A}}S=\bigsqcup^{\mathfrak{A}}\rsupfun{\Cor}S$
for any set $S$ of filters on a powerset.
\end{cor}
This corollary can be rewritten in elementary terms and proved elementarily:
\begin{prop}
$\bigcap\bigcap S=\bigcup_{F\in S}\bigcap S$ for a set~$S$ of filters
on some set.\end{prop}
\begin{proof}
(by \noun{Andreas Blass}) The $\supseteq$ direction is rather formal.
Consider any one of the sets being intersected on the left side, i.e.,
any set $X$ that is in all the filters in $S$, and consider any
of the sets being unioned (that's not a word, but you know what I
mean) on the right, i.e., $\bigcap F$ for some $F\in S$. Then, since
$X\in F$, we have $\bigcap F\subseteq X$. Taking the union over
all $F\in S$ (while keeping $X$ fixed), we get that the right side
of your equation is $\subseteq X$. Since that's true for all $X\in\bigcap S$,
we infer that the right side is a subset of the left side. (This argument
seems to work in much greater generality; you just need that the relevant
infima (in place of intersections) exist in your poset.)

For the $\subseteq$ direction, consider any element $x\in\bigcap\bigcap S$,
and suppose, toward a contradiction, that it is not an element of
the union on the right side of your equation. So, for each $F\in S$,
we have $x\notin\bigcap F$, and therefore we can find a set $A_{F}\in F$
with $x\notin A_{F}$. Let $B=\bigcup_{F\in S}A_{F}$ and notice that
$B\in F$ for every $F\in S$ (because $B\supseteq A_{F}$). So $B\in\bigcap S$.
But, by choice of the $A_{F}$'s, we have $x\notin B$, contrary to
the assumption that $x\in\bigcap\bigcap S$.\end{proof}
\begin{prop}
$\corestar\Omega(U)$ is the set of infinite subsets of $U$.\end{prop}
\begin{proof}
$\corestar\Omega(U)=\lnot\rsupfun{\lnot}\Omega(U)$.

$\rsupfun{\lnot}\Omega$ is the set of finite subsets of~$U$. Thus
$\lnot\rsupfun{\lnot}\Omega(U)$ is the set of infinite subsets of
$U$.
\end{proof}

\subsection{Number of Filters on a Set}
\begin{defn}
\index{finite intersection property}A collection $Y$ of sets has
finite intersection property iff intersection of any finite subcollection
of $Y$ is non-empty.
\end{defn}
The following was borrowed from \cite{blassnotesultra}. Thanks to
\noun{Andreas Blass} for email support about his proof.
\begin{lem}
(by \noun{Hausdorff}) For an infinite set $X$ there is a family $\mathcal{F}$
of $2^{\card X}$ many subsets of $X$ such that given any disjoint
finite subfamilies $\mathcal{A}$, $\mathcal{B}$, the intersection
of sets in $\mathcal{A}$ and complements of sets in $\mathcal{B}$
is nonempty.\end{lem}
\begin{proof}
Let
\[
X'=\setcond{(P;Q)}{P\in\subsets X\text{ is finite},Q\in\subsets\subsets P}.
\]


It's easy to show that $\card X'=\card X$. So it is enough to show
this for $X'$ instead of $X$. Let
\[
\mathcal{F}=\setcond{\setcond{(P;Q)\in X'}{Y\cap P\in Q}}{Y\in\subsets X}.
\]


To finish the proof we show that for every disjoint finite $Y_{+}\in\subsets\subsets X$
and finite $Y_{-}\in\subsets\subsets X$ there exist $(P;Q)\in X'$
such that
\[
\forall Y\in Y_{+}:(P;Q)\in\setcond{(P;Q)\in X'}{Y\cap P\in Q}\quad\text{and}\quad\forall Y\in Y_{-}:(P;Q)\notin\setcond{(P;Q)\in X'}{Y\cap P\in Q}
\]


what is equivalent to existence $(P;Q)\in X'$ such that
\[
\forall Y\in Y_{+}:Y\cap P\in Q\quad\text{and}\quad\forall Y\in Y_{-}:Y\cap P\notin Q.
\]


For existence of this $(P;Q)$, it is enough existence of $P$ such
that intersections $Y\cap P$ are different for different $Y\in Y_{+}\cup Y_{-}$.

Really, for each pair of distinct $Y_{0},Y_{1}\in Y_{+}\cup Y_{-}$
choose a point which lies in one of the sets $Y_{0}$, $Y_{1}$ and
not in an other, and call the set of such points $P$. Then $Y\cap P$
are different for different $Y\in Y_{+}\cup Y_{-}$.\end{proof}
\begin{cor}
For an infinite set $X$ there is a family $\mathcal{F}$ of $2^{\card X}$
many subsets of $X$ such that for arbitrary disjoint subfamilies
$\mathcal{A}$ and $\mathcal{B}$ the set $\mathcal{A}\cup\setcond{X\setminus A}{A\in\mathcal{B}}$
has finite intersection property.\end{cor}
\begin{thm}
Let $X$ be a set. The number of ultrafilters on $X$ is $2^{2^{\card X}}$
if $X$ is infinite and $\card X$ if $X$ is finite.\end{thm}
\begin{proof}
The finite case follows from the fact that every ultrafilter on a
finite set is trivial. Let $X$ be infinite. From the lemma, there
exists a family $\mathcal{F}$ of $2^{\card X}$ many subsets of $X$
such that for every $\mathcal{G}\in\subsets\mathcal{F}$ we have $\Phi(\mathcal{F};\mathcal{G})=\bigsqcap^{\mathfrak{A}}\mathcal{G}\sqcap\bigsqcap^{\mathfrak{A}}\setcond{X\setminus A}{A\in\mathcal{F}\setminus\mathcal{G}}\ne\bot^{\mathfrak{A}(X)}$.

This filter contains all sets from $\mathcal{G}$ and does not contain
any sets from $\mathcal{F}\setminus\mathcal{G}$. So for every suitable
pairs $(\mathcal{F}_{0};\mathcal{G}_{0})$ and $(\mathcal{F}_{1};\mathcal{G}_{1})$
there is $A\in\Phi(\mathcal{F}_{0};\mathcal{G}_{0})$ such that $\overline{A}\in\Phi(\mathcal{F}_{1};\mathcal{G}_{1})$.
Consequently all filters $\Phi(\mathcal{F};\mathcal{G})$ are disjoint.
So for every pair $(\mathcal{F};\mathcal{G})$ where $\mathcal{G}\in\subsets\mathcal{F}$
there exist a distinct ultrafilter under $\Phi(\mathcal{F};\mathcal{G})$,
but the number of such pairs $(\mathcal{F};\mathcal{G})$ is $2^{2^{\card X}}$.
Obviously the number of all filters is not above $2^{2^{\card X}}$.\end{proof}
\begin{cor}
The number of filters on $\mathfrak{U}$ is $2^{2^{\card X}}$ if
$\mathfrak{U}$ us infinite and $2^{\card\mathfrak{U}}$ if $\mathfrak{U}$
is finite.\end{cor}
\begin{proof}
The finite case is obvious. The infinite case follows from the theorem
and the fact that filters are collections of sets and there cannot
be more than $2^{2^{\card\mathfrak{U}}}$ collections of sets on $\mathfrak{U}$.
\end{proof}

\section{Some Counter-Examples}
\begin{example}
There exist a bounded distributive lattice which is not lattice with
separable center.\end{example}
\begin{proof}
The lattice with the Hasse diagram\footnote{See Wikipedia for a definition of Hasse diagrams.}
on figure \ref{not-sep-center} is bounded and distributive because
it does not contain ``diamond lattice'' nor ``pentagon lattice''
as a sublattice \cite{wiki:distributive-lattice}. 
\begin{figure}
\caption{\label{not-sep-center}}
\[
\xymatrix{ & 1\\
 & a\ar@{-}[u]\\
x\ar@{-}[ur] &  & y\ar@{-}[ul]\\
 & 0\ar@{-}[ul]\ar@{-}[ur]
}
\]
\end{figure}


It's center is $\{0,1\}$. $x\sqcap y=0$ despite $\up x=\{x,a,1\}$
but $y\sqcap1\ne 0$ consequently the lattice is not with separable
center.\end{proof}

In this section $\mathfrak{A}$ denotes the set of filters on a set.
\begin{example}
There is a separable poset (that is a set with $\fullstar$ being
an injection) such that $\fullstar x\subseteq\fullstar y$ does not
imply $x\sqsubseteq y$ (where $\sqsubseteq$ denotes our order) for
elements $x$, $y$ of this poset.\end{example}
\begin{proof}
(with help of sci.math partakers) Consider a poset with the Hasse
diagram~\ref{sep-counter}.

\begin{figure}[ht]
\caption{\label{sep-counter}}
\[
\xymatrix{a\ar@{-}[d]\ar@{-}[dr] & b\ar@{-}[dl]\ar@{-}[d]\ar@{-}[dr]\\
p & q & r
}
\]
\end{figure}


Then $\fullstar p=\{p,a,b\}$, $\fullstar q=\{q,a,b\}$, $\fullstar r=\{r,b\}$,
$\fullstar a=\{p,q,a,b\}$, $\fullstar b=\{p,q,a,b,r\}$.

Thus $\fullstar x=\fullstar y\Rightarrow x=y$ for any $x$, $y$
in our poset.

$\fullstar a\subseteq\fullstar b$ but not $a\sqsubseteq b$.\end{proof}
\begin{example}
There is a prefiltered filtrator which is not filtered.\end{example}
\begin{proof}
(\noun{Matthias Klupsch}) Take $\mathfrak{A}=\{a,b\}$ with the order
being equality and $\mathfrak{Z}=\{b\}$. Then $\up a=\emptyset\sqsubseteq\{b\}=\up b$,
so $\up$ is injective, hence the filtrator is prefiltered, but because
of $a\nsqsubseteq b$ the filtrator is not filtered.
\end{proof}
For further examples we will use the filter $\Delta$ defined by the
formula
\[
\Delta=\bigsqcap^{\mathfrak{A}}\setcond{(-\epsilon;\epsilon)}{\epsilon\in\mathbb{R},\epsilon>0}
\]


and more general
\[
\Delta+a=\bigsqcap^{\mathfrak{A}}\setcond{(a-\epsilon;a+\epsilon)}{\epsilon\in\mathbb{R},\epsilon>0}.
\]

\begin{example}
There exists $A\in\subsets U$ such that $\bigsqcap^{\mathfrak{A}}A\ne\bigsqcap A$.\end{example}
\begin{proof}
$\bigsqcap^{\mathfrak{A}}\setcond{(-\epsilon;\epsilon)}{\epsilon\in\mathbb{R},\epsilon>0}=\uparrow\{0\}\ne\Delta$.\end{proof}
\begin{example}
There exists a set $U$ and a filter $a$ and a set $S$ of filters
on the set $U$ such that $a\sqcap^{\mathfrak{A}}\bigsqcup^{\mathfrak{A}}S\ne\bigsqcup^{\mathfrak{A}}\rsupfun{a\sqcap^{\mathfrak{A}}}S$.\end{example}
\begin{proof}
Let $a=\Delta$ and $S=\setcond{\uparrow^{\mathbb{R}}(\epsilon;+\infty)}{\epsilon>0}$.
Then $a\sqcap^{\mathfrak{A}}\bigsqcup^{\mathfrak{A}}S=\Delta\sqcap^{\mathfrak{A}}(0;+\infty)\ne\bot^{\mathfrak{A}}$
while $\bigsqcup^{\mathfrak{A}}\rsupfun{a\sqcap^{\mathfrak{A}}}S=\bigsqcup^{\mathfrak{A}}\{\bot^{\mathfrak{A}}\}=\bot^{\mathfrak{A}}$.\end{proof}
\begin{example}
There are tornings which are not weak partitions.\end{example}
\begin{proof}
$\setcond{\Delta+a}{a\in\mathbb{R}}$ is a torning but not weak partition
of the real line.\end{proof}
\begin{lem}
Let $\mathfrak{A}$ be the set of filters on a set $U$. Then $X\sqcap^{\mathfrak{A}}\Omega\sqsubseteq Y\sqcap^{\mathfrak{A}}\Omega$
iff $X\setminus Y$ is a finite set, for every sets $X,Y\in\subsets U$.\end{lem}
\begin{proof}
Let $M$ be the set of finite subsets of $U$.
\begin{align*}
X\sqcap^{\mathfrak{A}}\Omega\sqsubseteq Y\sqcap^{\mathfrak{A}}\Omega & \Leftrightarrow\\
\setcond{X\cap K_{X}}{K_{X}\in\Omega}\supseteq\setcond{Y\cap K_{Y}}{K_{Y}\in\Omega} & \Leftrightarrow\\
\forall K_{Y}\in\Omega\exists K_{X}\in\Omega:Y\cap K_{Y}=X\cap K_{X} & \Leftrightarrow\\
\forall L_{Y}\in M\exists L_{X}\in M:Y\setminus L_{Y}=X\setminus L_{X} & \Leftrightarrow\\
\forall L_{Y}\in M:X\setminus(Y\setminus L_{Y})\in M & \Leftrightarrow\\
X\setminus Y\in M.
\end{align*}
\end{proof}
\begin{example}
There exists a filter $\mathcal{A}$ on a set $U$ such that $(\subsets U)/\mathord\sim$
and $Z(D\mathcal{A})$ are not complete lattices.\end{example}
\begin{proof}
Due to the isomorphism it is enough to prove for $(\subsets U)/\mathord\sim$.

Let take $U=\mathbb{N}$ and $\mathcal{A}=\Omega$ be the Fr\'echet
filter on $\mathbb{N}$.

Partition $\mathbb{N}$ into infinitely many infinite sets $A_{0},A_{1},\ldots$.
To withhold our example we will prove that the set $\{[A_{0}],[A_{1}],\dots\}$
has no supremum in $(\subsets U)/\mathord\sim$.

Let $[X]$ be an upper bound of $[A_{0}],[A_{1}],\ldots$ that is
$\forall i\in\mathbb{N}:X\sqcap^{\mathfrak{A}}\Omega\sqsupseteq A_{i}\sqcap^{\mathfrak{A}}\Omega$
that is $A_{i}\setminus X$ is finite. Consequently $X$ is infinite.
So $X\cap A_{i}\ne\emptyset$.

Choose for every $i\in\mathbb{N}$ some $z_{i}\in X\cap A_{i}$. The
$\{z_{0},z_{1},\dots\}$ is an infinite subset of $X$ (take into
account that $z_{i}\ne z_{j}$ for $i\ne j$). Let $Y=X\setminus\{z_{0},z_{1},\dots\}$.
Then $Y\sqcap^{\mathfrak{A}}\Omega\sqsupseteq A_{i}\sqcap^{\mathfrak{A}}\Omega$
because $A_{i}\setminus Y=A_{i}\setminus(X\setminus\{z_{i}\})=(A_{i}\setminus X)\cup\{z_{i}\}$
which is finite because $A_{i}\setminus X$ is finite. Thus $[Y]$
is an upper bound for $\{[A_{0}],[A_{1}],\dots\}$.

Suppose $Y\sqcap^{\mathfrak{A}}\Omega=X\sqcap^{\mathfrak{A}}\Omega$.
Then $Y\setminus X$ is finite what is not true. So $Y\sqcap^{\mathfrak{A}}\Omega\sqsubset X\sqcap^{\mathfrak{A}}\Omega$
that is $[Y]$ is below $[X]$.
\end{proof}

\subsection{Weak and Strong Partition}
\begin{defn}
\index{independent family}A family $S$ of subsets of a countable
set is \emph{independent} iff the intersection of any binarily many
members of $S$ and the complements of any other binarily many members
of $S$ is infinite.\end{defn}
\begin{lem}
The ``infinite'' at the end of the definition could be equivalently
replaced with ``nonempty'' if we assume that $S$ is infinite.\end{lem}
\begin{proof}
Suppose that some sets from the above definition has a finite intersection
$J$ of cardinality $n$. Then (thanks $S$ is infinite) get one more
set $X\in S$ and we have $J\cap X\ne\emptyset$ and $J\cap(\mathbb{N}\setminus X)\ne\emptyset$.
So $\card(J\cap X)<n$. Repeating this, we prove that for some finite
family of sets we have empty intersection what is a contradiction.\end{proof}
\begin{lem}
There exists an independent family on $\mathbb{N}$ of cardinality
$\mathfrak{c}$.\end{lem}
\begin{proof}
Let $C$ be the set of finite subsets of $\mathbb{Q}$. Since $\card C=\card\mathbb{N}$,
it suffices to find $\mathfrak{c}$ independent subsets of $C$. For
each $r\in\mathbb{R}$ let
\[
E_{r}=\setcond{F\in C}{\card(F\cap(-\infty;r))\text{ is even}}.
\]


All $E_{r_{1}}$ and $E_{r_{2}}$ are distinct for distinct $r_{1},r_{2}\in\mathbb{R}$
since we may consider $F=\{r'\}\in C$ where a rational number $r'$
is between $r_{1}$ and $r_{2}$ and thus $F$ is a member of exactly
one of the sets $E_{r_{1}}$ and $E_{r_{2}}$. Thus $\card\setcond{E_{r}}{r\in\mathbb{R}}=\mathfrak{c}$.

We will show that $\setcond{E_{r}}{r\in\mathbb{R}}$ is independent.
Let $r_{1},\dots,r_{k},s_{1},\ldots,s_{k}$ be distinct reals. It
is enough to show that these have a nonempty intersection, that is
existence of some $F$ such that $F$ belongs to all the $E_{r}$
and none of $E_{s}$.

But this can be easily accomplished taking $F$ having zero or one
element in each of intervals to which $r_{1},\ldots,r_{k},s_{1},\ldots,s_{k}$
split the real line.\end{proof}
\begin{example}
There exists a weak partition of a filter on a set which is not a
strong partition.\end{example}
\begin{proof}
(suggested by \noun{Andreas Blass}) Let $\setcond{X_{r}}{r\in\mathbb{R}}$
an independent family of subsets of $\mathbb{N}$. We can assume $a\ne b\Rightarrow X_{a}\ne X_{b}$
due the above lemma.

Let $\mathcal{F}_{a}$ be a filter generated by $X_{a}$ and the complements
$\mathbb{N}\setminus X_{b}$ for all $b\in\mathbb{R}$, $b\ne a$.
Independence implies that $\mathcal{F}_{a}\ne\bot^{\mathfrak{A}}$
(by properties of filter bases).

Let $S=\setcond{\mathcal{F}_{r}}{r\in\mathbb{R}}$. We will prove
that $S$ is a weak partition but not a strong partition.

Let $a\in\mathbb{R}$. Then $X_{a}\in\mathcal{F}_{a}$ while $\forall b\in\mathbb{R}\setminus\{a\}:\mathbb{N}\setminus X_{a}\in\mathcal{F}_{b}$
and therefore $\mathbb{N}\setminus X_{a}\in\bigsqcup^{\mathfrak{A}}\setcond{\mathcal{F}_{b}}{\mathbb{R}\ni b\ne a}$.
Therefore $\mathcal{F}_{a}\sqcap^{\mathfrak{A}}\bigsqcup^{\mathfrak{A}}\setcond{\mathcal{F}_{b}}{\mathbb{R}\ni b\ne a}=\bot^{\mathfrak{A}}$.
Thus $S$ is a weak partition.

Suppose $S$ is a strong partition. Then for each set $Z\in\subsets\mathbb{R}$
\[
\bigsqcup^{\mathfrak{A}}\setcond{\mathcal{F}_{b}}{b\in Z}\sqcap^{\mathfrak{A}}\bigsqcup^{\mathfrak{A}}\setcond{\mathcal{F}_{b}}{b\in\mathbb{R}\setminus Z}=\bot^{\mathfrak{A}}
\]
what is equivalent to existence of $M(Z)\in\subsets\mathbb{N}$ such
that
\[
M(Z)\in\bigsqcup^{\mathfrak{A}}\setcond{\mathcal{F}_{b}}{b\in Z}\quad\text{and}\quad\mathbb{N}\setminus M(Z)\in\bigsqcup^{\mathfrak{A}}\setcond{\mathcal{F}_{b}}{b\in\mathbb{R}\setminus Z}
\]


that is
\[
\forall b\in Z:M(Z)\in\mathcal{F}_{b}\quad\text{and}\quad\forall b\in\mathbb{R}\setminus Z:\mathbb{N}\setminus M(Z)\in\mathcal{F}_{b}.
\]


Suppose $Z\ne Z'\in\subsets\mathbb{N}$. Without loss of generality
we may assume that some $b\in Z$ but $b\notin Z'$. Then $M(Z)\in\mathcal{F}_{b}$
and $\mathbb{N}\setminus M(Z')\in\mathcal{F}_{b}$. If $M(Z)=M(Z')$
then $\mathcal{F}_{b}=\bot^{\mathfrak{A}}$ what contradicts to the
above.

So $M$ is an injective function from $\subsets\mathbb{R}$ to $\subsets\mathbb{N}$
what is impossible due cardinality issues.\end{proof}
\begin{lem}
(by \noun{Niels Diepeveen}, with help of \noun{Karl Kronenfeld}) Let
$K$ be a collection of free ultrafilters. We have $\bigsqcup K=\Omega$
iff $\exists\mathcal{G}\in\up K:A\in\up\mathcal{G}$ for every infinite
set $A$.\end{lem}
\begin{proof}
~
\begin{description}
\item [{$\Rightarrow$}] Suppose $\bigsqcup K=\Omega$ and let $A$ be
a set such that $\nexists\mathcal{G}\in K:A\in\up\mathcal{G}$. Let's
prove $A$ is finite.


Really, $\forall\mathcal{G}\in K:\mathfrak{U}\setminus A\in\up\mathcal{G}$;
$\mathfrak{U}\setminus A\in\up\Omega$; $A$ is finite.

\item [{$\Leftarrow$}] Let $\exists\mathcal{G}\in K:A\in\up\mathcal{G}$.
Suppose $A$ is a set in $\up\bigsqcup K$.


To finish the proof it's enough to show that $\mathfrak{U}\setminus A$
is finite.


Suppose $\mathfrak{U}\setminus A$ is infinite. Then $\exists\mathcal{G}\in K:\mathfrak{U}\setminus A\in\up\mathcal{G}$;
$\exists\mathcal{G}\in K:A\notin\up\mathcal{G}$; $A\notin\up\bigsqcup K$,
contradiction.

\end{description}
\end{proof}
\begin{lem}
(by \noun{Niels Diepeveen}) If $K$ is a non-empty set of ultrafilters
such that $\bigsqcup K=\Omega$, then for every $\mathcal{G}\in K$
we have $\bigsqcup(K\setminus\{\mathcal{G}\})=\Omega$.\end{lem}
\begin{proof}
$\exists\mathcal{F}\in K:A\in\up\mathcal{F}$ for every infinite set
$A$.

The set $A$ can be partitioned into two infinite sets $A_{1}$, $A_{2}$.

Take $\mathcal{F}_{1},\mathcal{F}_{2}\in K$ such that $A_{1}\in\mathcal{F}_{1}$,
$A_{2}\in\mathcal{F}_{2}$.

$\mathcal{F}_{1}\ne\mathcal{F}_{2}$ because otherwise $A_{1}$ and
$A_{2}$ are not disjoint.

Obviously $A\in\mathcal{F}_{1}$ and $A\in\mathcal{F}_{2}$.

So there exist two different $\mathcal{F}\in K$ such that $A\in\up\mathcal{F}$.
Consequently $\exists\mathcal{F}\in K\setminus\{\mathcal{G}\}:A\in\up\mathcal{F}$
that is $\bigsqcup(K\setminus\{\mathcal{G}\})=\Omega$.\end{proof}
\begin{example}
There exists a filter on a set which cannot be weakly partitioned
into ultrafilters.\end{example}
\begin{proof}
Consider cofinite filter $\Omega$ on any infinite set.

Suppose $K$ is its weak partition into ultrafilters. Then $x\asymp\bigsqcup(K\setminus\{x\})$
for some ultrafilter $x\in K$.

We have $\bigsqcup(K\setminus\{x\})\sqsubset\bigsqcup K$ (otherwise
$x\sqsubseteq\bigsqcup(K\setminus\{x\})$) what is impossible due
the last lemma.\end{proof}
\begin{cor}
There exists a filter on a set which cannot be strongly partitioned
into ultrafilters.
\end{cor}

\section{Open problems about filters}

In this section, I will formulate some conjectures about lattices
of filters on a set. If a conjecture comes true, it may be generalized
for more general lattices (such as, for example, lattices of filters
on arbitrary lattices). I deem that the main challenge is to prove
the special case about lattices of filters on a set, and generalizing
the conjectures is expected to be an easy task.


\subsection{Partitioning}

Consider the complete lattice $[S]$ generated by the set $S$ where
$S$ is a strong partition of some element $a$.
\begin{conjecture}
$[S]=\setcond{\bigsqcup^{\mathfrak{A}}X}{X\in\subsets S}$, where
$[S]$ is the complete lattice generated by a weak partition $S$
of filter on a set.
\end{conjecture}

\begin{conjecture}
$[S]=\setcond{\bigsqcup^{\mathfrak{A}}X}{X\in\subsets S}$, where
$[S]$ is the complete lattice generated by a strong partition $S$
of filter on a set.\end{conjecture}
\begin{prop}
Provided that the last conjecture is true, we have that $[S]$ is
a complete atomic boolean lattice with the set of its atoms being
$S$.\end{prop}
\begin{rem}
Consequently $[S]$ is atomistic, completely distributive and isomorphic
to a power set algebra (see \cite{wiki:bool-canon}).\end{rem}
\begin{proof}
Completeness of $[S]$ is obvious. Let $A\in[S]$. Then there exists
$X\in\subsets S$ such that $A=\bigsqcup^{\mathfrak{A}}X$. Let $B=\bigsqcup^{\mathfrak{A}}(S\setminus X)$.
Then $B\in[S]$ and $A\sqcap^{\mathfrak{A}}B=\bot^{\mathfrak{A}}$.
$A\sqcup^{\mathfrak{A}}B=\bigsqcup^{\mathfrak{A}}S$ is the greatest
element of $[S]$. So we have proved that $[S]$ is a boolean lattice.

Now let prove that $[S]$ is atomic with the set of atoms being $S$.
Let $z\in S$ and $A\in[S]$. If $A\ne z$ then either $A=\bot^{\mathfrak{A}}$
or $x\in X$ where $A=\bigsqcup^{\mathfrak{A}}X$, $X\in\subsets S$
and $x\ne z$. Because $S$ is a partition, $\bigsqcup^{\mathfrak{A}}(X\setminus\{z\})\sqcap^{\mathfrak{A}}z=\bot^{\mathfrak{A}}$
and $\bigsqcup^{\mathfrak{A}}(X\setminus\{z\})\ne\bot^{\mathfrak{A}}$.
So $A=\bigsqcup^{\mathfrak{A}}X=\bigsqcup^{\mathfrak{A}}(X\setminus\{z\})\sqcup^{\mathfrak{A}}z\nsqsubseteq z$.

Finally we will prove that elements of $[S]\setminus S$ are not atoms.
Let $A\in[S]\setminus S$ and $A\ne\bot$. Then $A\sqsupseteq x\sqcup^{\mathfrak{A}}y$
where $x,y\in S$ and $x\ne y$. If $A$ is an atom then $A=x=y$
what is impossible.\end{proof}
\begin{prop}
The conjecture about the value of $[S]$ is equivalent to closedness
of $\setcond{\bigsqcup^{\mathfrak{A}}X}{X\in\subsets S}$ under arbitrary
meets and joins.\end{prop}
\begin{proof}
If $\setcond{\bigsqcup^{\mathfrak{A}}X}{X\in\subsets S}=[S]$ then
trivially $\setcond{\bigsqcup^{\mathfrak{A}}X}{X\in\subsets S}$ is
closed under arbitrary meets and joins.

If $\setcond{\bigsqcup^{\mathfrak{A}}X}{X\in\subsets S}$ is closed
under arbitrary meets and joins, then it is the complete lattice generated
by the set $S$ because it cannot be smaller than the set of all suprema
of subsets of $S$.
\end{proof}
That $\setcond{\bigsqcup^{\mathfrak{A}}X}{X\in\subsets S}$ is closed
under arbitrary joins is trivial. I have not succeeded to prove that
it is closed under arbitrary meets, but have proved a weaker statement
that is is closed under binary meets:
\begin{prop}
$\setcond{\bigsqcup^{\mathfrak{A}}X}{X\in\subsets S}$ is closed under
binary meets.\end{prop}
\begin{proof}
Let $R=\setcond{\bigsqcup^{\mathfrak{A}}X}{X\in\subsets S}$. Then
for every $X,Y\in\subsets S$
\begin{align*}
\bigsqcup^{\mathfrak{A}}X\sqcap^{\mathfrak{A}}\bigsqcup^{\mathfrak{A}}Y & =\\
\bigsqcup^{\mathfrak{A}}((X\cap Y)\cup(X\setminus Y))\sqcap^{\mathfrak{A}}\bigsqcup^{\mathfrak{A}}Y & =\\
\left(\bigsqcup^{\mathfrak{A}}(X\cap Y)\sqcup^{\mathfrak{A}}\bigsqcup^{\mathfrak{A}}(X\setminus Y)\right)\sqcap^{\mathfrak{A}}\bigsqcup^{\mathfrak{A}}Y & =\\
\left(\bigsqcup^{\mathfrak{A}}(X\cap Y)\sqcap^{\mathfrak{A}}\bigsqcup^{\mathfrak{A}}Y\right)\sqcup^{\mathfrak{A}}\left(\bigsqcup^{\mathfrak{A}}(X\setminus Y)\sqcap^{\mathfrak{A}}\bigsqcup^{\mathfrak{A}}Y\right) & =\\
\left(\bigsqcup^{\mathfrak{A}}(X\cap Y)\sqcap^{\mathfrak{A}}\bigsqcup^{\mathfrak{A}}Y\right)\sqcup^{\mathfrak{A}}\bot^{\mathfrak{A}} & =\\
\bigsqcup^{\mathfrak{A}}(X\cap Y)\sqcap^{\mathfrak{A}}\bigsqcup^{\mathfrak{A}}Y.
\end{align*}


Applying the formula $\bigsqcup^{\mathfrak{A}}X\sqcap^{\mathfrak{A}}\bigsqcup^{\mathfrak{A}}Y=\bigsqcup^{\mathfrak{A}}(X\cap Y)\sqcap^{\mathfrak{A}}\bigsqcup^{\mathfrak{A}}Y$
twice we get
\begin{align*}
\bigsqcup^{\mathfrak{A}}X\sqcap^{\mathfrak{A}}\bigsqcup^{\mathfrak{A}}Y & =\\
\bigsqcup^{\mathfrak{A}}(X\cap Y)\sqcap^{\mathfrak{A}}\bigsqcup^{\mathfrak{A}}(Y\cap(X\cap Y)) & =\\
\bigsqcup^{\mathfrak{A}}(X\cap Y)\sqcap^{\mathfrak{A}}\bigsqcup^{\mathfrak{A}}(X\cap Y) & =\\
\bigsqcup^{\mathfrak{A}}(X\cap Y).
\end{align*}


But for any $A,B\in R$ there exist $X,Y\in\subsets S$ such that
$A=\bigsqcup^{\mathfrak{A}}X$, $B=\bigsqcup^{\mathfrak{A}}Y$. So
$A\sqcap^{\mathfrak{A}}B=\bigsqcup^{\mathfrak{A}}X\sqcap\bigsqcup^{\mathfrak{A}}Y=\bigsqcup^{\mathfrak{A}}(X\cap Y)\in R$.
\end{proof}

\subsection{Quasidifference}
\begin{conjecture}
$a\psetminus b=\bigsqcup_{B\in\up b}(a\sqcap^{\mathfrak{A}}(U\setminus B))$
for all $a,b\in\mathfrak{A}$ for each lattice $\mathfrak{A}$ of
filters on a set $U$.
\end{conjecture}

\subsection{Non-Formal Problems}

Find a common generalization of two theorems:
\begin{enumerate}
\item If $\mathfrak{Z}$ is a meet-semilattice with greatest element then
for any $\mathcal{A},\mathcal{B}\in\mathfrak{A}$
\[
\up(\mathcal{A}\sqcup^{\mathfrak{A}}\mathcal{B})=\up\mathcal{A}\cap\up\mathcal{B}.
\]

\item If $\mathfrak{Z}$ is a join-semilattice then $\mathfrak{A}$ is a
join-semilattice then and for any $\mathcal{A},\mathcal{B}\in\mathfrak{A}$
\[
\up(\mathcal{A}\sqcup^{\mathfrak{A}}\mathcal{B})=\up\mathcal{A}\cap\up\mathcal{B}.
\]

\end{enumerate}
Under which conditions $a\psetminus b$ and $a\mathop\#b$ are complementive
to $a$?

Generalize straight maps for arbitrary posets.


\section{Further notation}

Below to define funcoids and reloids we need a fixed powerset filtrator.

Let $(\mathscr{F}A;\mathscr{T}A)$ be an arbitrary but fixed powerset
filtrator. This filtrator exists by the theorem~\ref{thm1:prim-exists}.

\index{filter object}I will call elements of $\mathscr{F}$ \emph{filter
objects}.

For brevity we will denote lattice operations on $\mathscr{F}A$ without
indexes (for example, take $\bigsqcap S=\bigsqcap^{\mathscr{F}A}S$
for $S\in\subsets\mathscr{F}A$).

Note that above we also took operations on $\mathscr{T}A$ without
indexes (for example, take $\bigsqcap S=\bigsqcap^{\mathscr{T}A}S$
for $S\in\subsets\mathscr{T}A$).

Because we identify $\mathscr{T}A$ with principal elements of $\mathscr{F}A$,
the notation like $\bigsqcap S$ for $S\in\subsets\mathscr{T}A$ would
be inconsistent (it can mean both $\bigsqcap^{\mathscr{T}A}S$ or
$\bigsqcap^{\mathscr{F}A}S$). We explicitly state that $\bigsqcap S$
in this case does \emph{not} mean $\bigsqcap^{\mathscr{F}A}S$.

For $\mathcal{X}\in\mathscr{F}$ we will denote $\GR\mathcal{X}$
the corresponding filter on $\subsets A$. It is a convenient notation
to describe relations between filters and sets, consider for example
the formula: $\{x\}\subseteq\bigcap\GR\mathcal{X}$.

We will denote lattice operations without pointing a specific set
like $\bigsqcap^{\mathscr{F}}S=\bigsqcap^{\mathscr{F}(A)}S$ for a
set $S\in\subsets\mathscr{F}(A)$.
