\chapter{More on generalized limit}

\begin{defn}
I will call a permutation group \emph{fixed point free} when
every element of it except of identity has no fixed points.
\end{defn}

\begin{defn}
A funcoid~$f$ is \emph{Kolmogorov} when
$\rsupfun{f}\{x\}\ne\rsupfun{f}\{y\}$ for every distinct points
$x,y\in\dom f$.
\end{defn}

\section{Hausdorff funcoids}

\begin{defn}
\emph{Limit}~$\lim\mathcal{F}=x$ of a filter~$\mathcal{F}$
regarding funcoid~$f$ is such a point that $\rsupfun{f}\{x\}\sqsupseteq\mathcal{F}$.
\end{defn}

\begin{defn}
\emph{Hausdorff} funcoid is such a funcoid that every proper
filter on its image has at most one limit.
\end{defn}

\begin{prop}
The following are pairwise equivalent for every funcoid~$f$:
\begin{enumerate}
\item\label{hd:eq-d} $f$~is Hausdorff.
\item\label{hd:eq-op}
$x\ne y\Rightarrow\rsupfun{f}\{x\}\asymp\rsupfun{f}\{y\}$.
\end{enumerate}
\end{prop}

\begin{proof}
~
\begin{description}
\item[\ref{hd:eq-d}$\Rightarrow$\ref{hd:eq-op}]
If~\ref{hd:eq-op} does not hold,
then there exist distinct points~$x$ and~$y$ such that
$\rsupfun{f}\{x\}\nasymp\rsupfun{f}\{y\}$.
So~$x$ and~$y$ are both limit points of
$\rsupfun{f}\{x\}\sqcap\rsupfun{f}\{y\}$, and thus~$f$ is not
Hausdorff.
\item[\ref{hd:eq-op}$\Rightarrow$\ref{hd:eq-d}]
Suppose~$\mathcal{F}$ is proper.
\[ \rsupfun{f}\{x\}\sqsupseteq\mathcal{F}\land
\rsupfun{f}\{y\}\sqsupseteq\mathcal{F}\Rightarrow
\rsupfun{f}\{x\}\nasymp\rsupfun{f}\{y\}\Rightarrow x=y. \]
\end{description}
\end{proof}

\begin{cor}
Every entirely defined Hausdorff funcoid is Kolmogorov.
\end{cor}

\begin{rem}
It is enough to be ``almost entirely defined'' (having nonempty
value everywhere except of one point).
\end{rem}

\begin{obvious}
For a complete funcoid induced by a topological space this
coincides with the traditional definition of a Hausdorff
topological space.
\end{obvious}

\section{Restoring functions from limit}

Consider alternative definition of generalized limit:
\[ \xlim f = \lambda r\in G: \nu\circ f\circ\uparrow r. \]

Or:
\[ \xlim_a f = \setcond{(\rsupfun{r^{-1}}a, \nu\circ f\circ\uparrow r)}{r\in G} \]
(note this requires explicit filter in the definition of generalized limit).

Operations on the set of generalized limits can be defined (twice) pointwise. \fxwarning{First define operations on
funcoids.}

\begin{prop}
The above defined $\xlim_{\rsupfun{\mu}\{x\}} f$ is a monovalued
function if~$\mu$ is Kolmogorov and~$G$ is fixed point free.
\end{prop}

\begin{proof}
We need to prove $\supfun{r^{-1}}\rsupfun{\mu}\{x\}\ne
\supfun{s^{-1}}\rsupfun{\mu}\{x\}$ for $r,s\in G$, $r\ne s$.
Really, by definition of generalized limit, they commute, so our
formula is equivalent to
$\rsupfun{\mu}\rsupfun{r^{-1}}\{x\}\ne
\rsupfun{\mu}\rsupfun{s^{-1}}\{x\}$;
$\rsupfun{\mu}\rsupfun{r^{-1}\circ s}\rsupfun{s^{-1}}\{x\}\ne
\rsupfun{\mu}\rsupfun{s^{-1}}\{x\}$.
But $r^{-1}\circ s\ne e$, so because it is fixed point free,
$\rsupfun{r^{-1}\circ s}\rsupfun{s^{-1}}\{x\}\ne
\rsupfun{s^{-1}}\{x\}$ and thus by kolmogorovness, we have the
thesis.
\end{proof}

\begin{lem}
Let~$\mu$ and~$\nu$ be Hausdorff funcoids. If function~$f$ is defined at point~$x$, then
\[ fx=
\lim\rsupfun{(\xlim_{\rsupfun{\mu}\{x\}}f)\rsupfun{\mu}\{x\}}\{x\}
\]
\end{lem}

\begin{rem}
The right part is correctly defined because
$\xlim_a f$ is monovalued.
\end{rem}

\begin{proof}
$\lim\rsupfun{(\xlim_{\rsupfun{\mu}\{x\}}f)\rsupfun{\mu}\{x\}}\{x\}=
\lim\rsupfun{\nu\circ f}\{x\}=\lim\rsupfun{\nu}fx=fx$.
\end{proof}

\begin{cor}
Let~$\mu$ and~$\nu$ be Hausdorff funcoids.
Then function~$f$ can be restored from values of $\xlim_{\rsupfun{\mu}\{x\}}f$.
\end{cor}