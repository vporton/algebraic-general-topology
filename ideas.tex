\documentclass{amsart}

\usepackage{agt}
  
\begin{document}

This document contains a list of short ideas of future research in Algebraic
General Topology.

I have created branch \texttt{devel} in \href{https://bitbucket.org/portonv/algebraic-general-topology}{the \LaTeX repository} for the book
to add new ``draft'' features there. The \texttt{devel} branch isn't distributed by me in PDF format, but you can download and compile it yourself.

This research plan is not formal and may contain vague statements.

Should clearly denote $\mathsf{pFCD}(\mathfrak{A};\mathfrak{Z})$ or $\mathsf{pFCD}(\mathfrak{A})$.

\section{Category theory}

I have defined $\mathsf{RLD}\sharp$ to describe $\Hom$-sets of the category or reloids but without source and destination and without composition.
$\mathsf{RLD}$ should be replaced with $\mathsf{RLD}\sharp$ where possible, in order to make the theorems throughout the book a little more general.
Also introduce similar features like $\Gamma\sharp$ and $\mathfrak{F}\Gamma\sharp$ (the last notation may need to be changed).

Misc properties of continuous functions between endofuncoids and endoreloids.

\url{http://nforum.ncatlab.org/discussion/6765/please-help-with-a-proof-that-a-category-is-monoidal/} proves that
finitary staroids are isomorphic to an ideal on a poset (for semilattices only).

\url{http://matwbn.icm.edu.pl/ksiazki/fm/fm94/fm94115.pdf} defines two categories with objects being filters. Another article on the same topic:\\
\url{https://eudml.org/doc/16352} (Koubek, V\'aclav, and Reiterman, Jan. "On the category of filters.")

\fxnote{\url{https://en.wikipedia.org/wiki/Cauchy_space} says ``The category of Cauchy spaces and Cauchy continuous maps is cartesian closed.'' Generalize.
\url{http://www.sciencedirect.com/science/article/pii/0166864187900988}}

\section{Compact funcoids}

Generalize the theorem that compact topology corresponds to only one uniformity.

For compact funcoids the Cantor's theorem that a function continuous on a compact is uniformly continuous.

Every closed subset of a compact space is compact. A compact subset of a Hausdorff space is closed. 17.5 theorem in Willard.

17.6 theorem in Willard.

17.7 theorem in Willard: The continuous image of a compact space is compact.

17.10 Theorem in Willard: A compact Hausdorff space $X$ is a $T_4$-space. Also 17.11 Corollary, 17.13, 17.14 theorem.

"Locally compact" for funcoids. See also 18 "Locally compact spaces: in Willard.

Compactification.

\section{Misc}

Are filters on all Heyting or all co-Heyting lattices star-separable?
\url{http://math.stackexchange.com/q/1326266/4876}

Define generalized pointfree reloids as filters on systems of sides.

\href{https://www.math.ksu.edu/~strecker/primer.ps}{Galois connections primer} -- study to ensure that we considered all Galois connections properties.

\href{https://en.wikipedia.org/wiki/Germ_(mathematics)}{Germs} seems to be equivalent to monovalued reloids.

$\mathcal{A} = \min\setcond{X}{\forall K\in\corestar\mathcal{A}:K\nasymp K}$, so we can restore $\mathcal{A}$ from~$\corestar\mathcal{A}$.

Boolean funcoid is a join-semilattice morphism from a boolean lattice to a boolean lattice. Generalize for pointfree funcoids.

Another way to define pointfree reloid as filters on Galois connections between two posets.

$L \in \GR \prod^{\mathsf{Strd} \ast} A \Leftrightarrow \forall
\text{finite } M \subseteq \dom A \forall i \in M : A_i \nasymp L_i$?

Star-composition with identity staroids?

Does upgrading/downgrading of the ideal which represents a prestaroid coincide with upgrading/downgrading of the prestaroid?

It seems that equivalence of filters on different bases can be generalized:
filters~$\mathcal{A}\in\mathfrak{A}$ and~$\mathcal{B}\in\mathfrak{B}$ are \emph{equivalent} iff
there exists an $X\in\mathfrak{A}\cap\mathfrak{B}$ which is greater than both~$\mathcal{A}$ and~$\mathcal{B}$.
This however works only in the case if order of the orders~$\mathfrak{A}$ and~$\mathfrak{B}$ agree,
that is if then are both a suborders of a greater fixed order.

Under which conditions a function spaces of posets is strongly separable?

Generalize both funcoids and reloids as filters on a superset of the lattice~$\Gamma$ (see ``Funcoids are filters'' chapter).

When the set of filters closed regarding a funcoid is a (co-)frame?

If a formula $F(x_0,\dots,x_n)$ holds for every poset $\mathfrak{A}_i$ then it also holds for product order $\prod\mathfrak{A}$.
(What about infinite formulas like complete lattice joins and meets?)
Moreover $F(x_0,\dots,x_n) = \mylambda{i}{n}{F(x_{0,i},\dots,x_{n,i})}$ (confused logical forms and functions).
It looks like a promising approach, but how to define it exactly? For example, $F$ may be a form always true for boolean
lattices or for Heyting lattices, or whatsoever. How one theorem can encompass all kinds of lattices and posets?
We may attempt to restrict to (partial) functions determined by order.
(This is not enough, because we can define an operation restricting $\setminus$ defined only for posets
of cardinality above or below some cardinal~$\kappa$. For such restricted $\setminus$ the above formula does not work.)
See also \url{https://portonmath.wordpress.com/2016/01/12/a-conjecture-about-product-order-and-logic/}.
It seems that \noun{Todd Trimble} shows a general category-theoretic way to describe this:
\url{https://nforum.ncatlab.org/discussion/6887/operations-on-product-order/}.

Get results from \url{http://ncatlab.org/toddtrimble/published/topogeny}.

What about distributivity of quasicomplements over meets and joins for the filtrator of funcoids? Seems like nontrivial conjectures.

Conjecture: Each filtered filtrator is isomorphic to a primary filtrator. (If it holds, then primary and filtered filtrators are the same!)

Add analog of the last item of the theorem about co-complete funcoids for pointfree funcoids.

Generalize theorems about $\mathsf{RLD}(A;B)$ as $\mathscr{F}(A\times B)$ in order to clean up the notation
(for example in the chapter ``Funcoids are filters'').

Define reloids as a filtrator whose core is an ordered semigroup.
This way reloids can be described in several isomorphic ways (just like primary filtrators are both filtrators of filters, of ideals, etc.)
Is it enough to describe all properties of reloids? Well, it is not a semigroup, it is a precategory.
It seems that we also need functions $\dom$ and $\im$ into partially ordered sets and ``reversion'' (dagger).

\url{http://mathoverflow.net/a/191381/4086} says that $n$-staroids can be identified with certain ideals!

To relax theorem conditions and definition, we can define \emph{protofuncoids} as arbitrary pairs $(\alpha;\beta)$ of functions
between two posets. For protofuncoids composition and reverse are defined.

Add examples of funcoids to demonstrate their power:
$D\sqcup T$ ($D$ is a digraph $T$ is a topological space),
$T\sqcap\setcond{(x;y)}{y\ge x}$ as ``one-side topology'' and also a circle made from its $\pi$-length segment.

Say explicitly that pseudodifference is a special case of difference.

For pointfree funcoids, if $f:\mathfrak{A}\rightarrow\mathfrak{B}$ exists,
then existence of least element of~$\mathfrak{A}$ is equivalent to existence of least element of~$\mathfrak{B}$:
$y \nasymp \supfun{f} \bot^{\mathfrak{A}} \Leftrightarrow
\bot^{\mathfrak{A}} \nasymp \langle f^{- 1} \rangle y \Leftrightarrow 0$. Thus
$\supfun{f} y \asymp \supfun{f} y$ and so $\supfun{f} y =
\bot^{\mathfrak{B}}$.
Can a similar statement be made that $\mathfrak{A}$ being join-semilattice implies $\mathfrak{B}$ being join-semilattice
(at least for separable posets)? If yes, this could allow to shorten some theorem conditions.
It seems we can produce a counter-example for non-separable posets by replacing an element with another element with the same full star.

Develop Todd Trimble's idea to represent funcoids as a relation~$\xi$ further:
Define funcoid as a function from sets to sets of sets
$\xi(A\sqcup B) = \xi A\cap\xi B$ and $\xi\bot=\emptyset$.

Denote the set of least elements as $\operatorname{Least}$. (It is either an
one-element set or empty set.)

Show that cross-composition product is a special case of infimum product.

Analog of order topology for funcoids/reloids.

If a morphism converges to a value on two sets, it converges to the value also on their union.

A set is connected if every function from it to a discrete space is constant. Can this be generalized for generalized connectedness and generalized continuity? I have no idea how to relate these two concepts in general.

Develop theory of \emph{funcoidal groups} by analogy with topological groups.
Attempt to use this theory to solve this open problem:\\
\url{http://garden.irmacs.sfu.ca/?q=op/is_every_regular_paratopological_group_tychonoff}
Is it useful as topological group determines not only a topology but even a uniformity?

A space $\mu$ is $T_2$- iff the diagonal $\Delta$ is closed in $\mu\times\mu$.

The $\beta$-th projection map is not only continuous but also open (Willard, theorem 8.6).

$T_x$-separation axioms for products of spaces.

Willard 13.13 and its important corollary 13.14.

Willard 15.10.

About real-valued functions on endofuncoids: Urysohn's Lemma (and consequences: Tietze's extension theorem) for funcoids.

About product of reloids:\\
\url{http://portonmath.wordpress.com/2012/05/23/unfirunded-questions/}

Generalized Fr\'echet filter on a poset $\mathfrak{A}$ is a filter $\Omega$ such that
\[ \corestar\Omega = \setcond{x\in\mathfrak{A}}{\atoms x\text{ is infinite}}. \]
Research their properties (first, whether they exists for every poset).

Manifolds.

"Micronization" of partial orders:\\
\url{https://portonmath.wordpress.com/2015/05/08/order-top/}

\url{http://www.sciencedirect.com/science/article/pii/0304397585900623}\\
(free download, also Google for "pre-adjunction", also "semi" instead of "pre") Are $\tofcd$ and $\torldin$ adjunct?

Check how \href{http://ncatlab.org/nlab/show/multicategory}{multicategories}
are related with categories with star-morphisms.

At \url{https://en.wikipedia.org/wiki/Semilattice} they are defined distributive
semilattices. A join-semilattice is distributive if and only if the lattice of its ideals (under inclusion) is distributive.

The article \url{http://arxiv.org/abs/1410.1504} has solved ``Every paratopological group is Tychonoff'' conjecture positively.
Rewrite this article in terms of funcoids and reloids (especially with the algebraic formulas characterizing regular funcoids).

Generalize interior in topological spaces as the \emph{interior funcoid} of a complete funcoid~$f$, defined as a pointfree funcoid
$f^\circ: \mathscr{F}\dual\Src f \rightarrow \mathscr{F}\dual\Dst f$ conforming to the formula:
$\rsupfun{f^{\circ}} (I \sqcap J) = \overline{\rsupfun{f} \overline{I \sqcap J}} = \overline{\rsupfun{f} ( \overline{I} \sqcup \overline{J} )}$.
However composition of an interior funcoid with a funcoid is neither a funcoid nor an interior funcoid.
It can be generalized using pseudocomplement.

\end{document}
