
\chapter{Funcoids}

In this chapter (and several following chapters) the word \emph{filter}
will refer to a filter (or equivalently any filter object) on a set
(rather than a filter on an arbitrary poset).


\section{\index{funcoid}Informal introduction into funcoids}

Funcoids are a generalization of proximity spaces and a generalization
of pretopological spaces. Also funcoids are a generalization of binary
relations.

That funcoids are a common generalization of ``spaces'' (proximity
spaces, (pre)topological spaces) and binary relations (including monovalued
functions) makes them smart for describing properties of functions
in regard of spaces. For example the statement ``$f$ is a continuous
function from a space $\mu$ to a space $\nu$'' can be described
in terms of funcoids as the formula $f\circ\mu\sqsubseteq\nu\circ f$
(see below for details).

Most naturally funcoids appear as a generalization of proximity spaces.

Let $\delta$ be a proximity. We will extend the relation~$\delta$
from sets to filters by the formula:
\[
\mathcal{A}\mathrel\delta'\mathcal{B}\Leftrightarrow\forall 
A\in\up\mathcal{A},B\in\up\mathcal{B}:A\mathrel\delta B.
\]


Then (as it will be proved below) there exist two functions 
$\alpha,\beta\in\mathscr{F}^{\mathscr{F}}$
such that
\[
\mathcal{A}\mathrel\delta'\mathcal{B}\Leftrightarrow\mathcal{B}
\sqcap\alpha\mathcal{A}\ne\bot^{\mathscr{F}}\Leftrightarrow\mathcal{A}
\sqcap\beta\mathcal{B}\ne\bot^{\mathscr{F}}.
\]


The pair $(\alpha;\beta)$ is called \emph{funcoid} when 
$\mathcal{B}\sqcap\alpha\mathcal{A}\ne\bot^{\mathscr{F}}\Leftrightarrow\mathcal{
A}\sqcap\beta\mathcal{B}\ne\bot^{\mathscr{F}}$.
So funcoids are a generalization of proximity spaces.

Funcoids consist of two components the first $\alpha$ and the second
$\beta$. The first component of a funcoid $f$ is denoted as $\supfun f$
and the second component is denoted as $\supfun{f^{-1}}$. (The similarity
of this notation with the notation for the image of a set under a
function is not a coincidence, we will see that in the case of principal
funcoids (see below) these coincide.)

One of the most important properties of a funcoid is that it is uniquely
determined by just one of its components. That is a funcoid $f$ is
uniquely determined by the function $\supfun f$. Moreover a funcoid
$f$ is uniquely determined by values of $\supfun f$ on principal
filters.

Next we will consider some examples of funcoids determined by specified
values of the first component on sets.

Funcoids as a generalization of pretopological spaces: Let $\alpha$
be a pretopological space that is a map \emph{$\alpha\in\mathscr{F}^{\mho}$}
for some set $\mho$. Then we define $\alpha'X=\bigsqcup_{x\in X}\alpha x$
for every set $X\in\subsets\mho$. We will prove that there exists
a unique funcoid $f$ such that $\alpha'=\supfun f|_{\mathfrak{P}}\circ\uparrow$
where $\mathfrak{P}$ is the set of principal filters on $\mho$.
So funcoids are a generalization of pretopological spaces. Funcoids
are also a generalization of preclosure operators: For every preclosure
operator $p$ on a set $\mho$ it exists a unique funcoid $f$ such
that $\supfun f|_{\mathfrak{P}}\circ\uparrow=\uparrow\circ p$.

For every binary relation $p$ on a set $\mho$ there exists unique
funcoid $f$ such that
\[
\forall X\in\subsets\mho:\supfun f\uparrow X=\uparrow\rsupfun pX
\]
(where $\rsupfun p$ is defined in the introduction), recall that
a funcoid is uniquely determined by the values of its first component
on sets. I will call such funcoids \emph{principal}. So funcoids are
a generalization of binary relations.

Composition of binary relations (i.e. of principal funcoids) complies
with the formulas:
\[
\rsupfun{g\circ f}=\rsupfun g\circ\rsupfun f\quad\text{and}\quad\rsupfun{(g\circ 
f)^{-1}}=\rsupfun{f^{-1}}\circ\rsupfun{g^{-1}}.
\]
By similar formulas we can define composition of every two funcoids.
Funcoids with this composition form a category (\emph{the category
of funcoids}).

Also funcoids can be reversed (like reversal of $X$ and $Y$ in a
binary relation) by the formula $(\alpha;\beta)^{-1}=(\beta;\alpha)$.
In the particular case if $\mu$ is a proximity we have $\mu^{-1}=\mu$
because proximities are symmetric.

Funcoids behave similarly to (multivalued) functions but acting on
filters instead of acting on sets. Below there will be defined domain
and image of a funcoid (the domain and the image of a funcoid are
filters).


\section{Basic definitions}
\begin{defn}
\index{funcoid}Let us call a \emph{funcoid} from a set $A$ to a
set $B$ a quadruple $(A;B;\alpha;\beta)$ where 
$\alpha\in\mathscr{F}(B)^{\mathscr{F}(A)}$,
$\alpha\in\mathscr{F}(A)^{\mathscr{F}(B)}$ such that
\[
\forall\mathcal{X}\in\mathscr{F}(A),\mathcal{Y}\in\mathscr{F}(B):(\mathcal{Y}
\nasymp\alpha\mathcal{X}\Leftrightarrow\mathcal{X}\nasymp\beta\mathcal{Y}).
\]

\end{defn}

\begin{defn}
\index{funcoid!source}\index{funcoid!destination}\emph{Source} and
\emph{destination} of every funcoid $(A;B;\alpha;\beta)$ are defined
as:
\[
\Src(A;B;\alpha;\beta)=A\quad\text{and}\quad\Dst(A;B;\alpha;\beta)=B.
\]

\end{defn}
I will denote $\mathsf{FCD}(A;B)$ the set of funcoids from $A$ to
$B$.

I will denote $\mathsf{FCD}$ the set of all funcoids (for small sets).
\begin{defn}
\index{endo-funcoid}I will call an \emph{endofuncoid} a funcoid whose source is the same as it's destination.
\end{defn}

\begin{defn}
$\supfun{(A;B;\alpha;\beta)}\eqdef\alpha$ for a funcoid $(A;B;\alpha;\beta)$.
\end{defn}

\begin{defn}
\index{funcoid!reverse}The \emph{reverse} funcoid 
$(A;B;\alpha;\beta)^{-1}=(B;A;\beta;\alpha)$
for a funcoid $(A;B;\alpha;\beta)$.\end{defn}
\begin{note}
The reverse funcoid is \emph{not} an inverse in the sense of group
theory or category theory.\end{note}
\begin{prop}
If $f$ is a funcoid then $f^{-1}$ is also a funcoid.\end{prop}
\begin{proof}
It follows from symmetry in the definition of funcoid.\end{proof}
\begin{obvious}
$(f^{-1})^{-1}=f$ for a funcoid $f$.\end{obvious}
\begin{defn}
The relation $\mathord{\suprel f}\in\subsets(\mathscr{F}(\Src 
f)\times\mathscr{F}(\Dst f))$
is defined (for every funcoid $f$ and $\mathcal{X}\in\mathscr{F}(\Src f)$,
$\mathcal{Y}\in\mathscr{F}(\Dst f)$ by the formula $\mathcal{X}\suprel 
f\mathcal{Y}\Leftrightarrow\mathcal{Y}\nasymp\supfun f\mathcal{X}$.\end{defn}
\begin{obvious}
$\mathcal{X}\suprel f\mathcal{Y}\Leftrightarrow\mathcal{Y}\nasymp\supfun 
f\mathcal{X}\Leftrightarrow\mathcal{X}\nasymp\supfun{f^{-1}}\mathcal{Y}$
for every funcoid $f$ and $\mathcal{X}\in\mathscr{F}(\Src f)$, 
$\mathcal{Y}\in\mathscr{F}(\Dst f)$.
\end{obvious}

\begin{obvious}
$\suprel{f^{-1}}=\suprel f^{-1}$ for a funcoid $f$.\end{obvious}
\begin{thm}
Let $A$, $B$ be sets.
\begin{enumerate}
\item For given value of $\supfun f\in\mathscr{F}(B)^{\mathscr{F}(A)}$
there exists no more than one funcoid $f\in\mathsf{FCD}(A;B)$.
\item For given value of $\mathord{\suprel 
f}\in\subsets(\mathscr{F}(A)\times\mathscr{F}(B))$
there exists no more than one funcoid $f\in\mathsf{FCD}(A;B)$.
\end{enumerate}
\end{thm}
\begin{proof}
Let $f,g\in\mathsf{FCD}(A;B)$.

Obviously, $\supfun f=\supfun g\Rightarrow\suprel f=\suprel g$ and
$\supfun{f^{-1}}=\supfun{g^{-1}}\Rightarrow\suprel f=\suprel g$.
So it's enough to prove that $\suprel f=\suprel g\Rightarrow\supfun f=\supfun 
g$.

Provided that $\suprel f=\suprel g$ we have $\mathcal{Y}\nasymp\supfun 
f\mathcal{X}\Leftrightarrow\mathcal{X}\suprel 
f\mathcal{Y}\Leftrightarrow\mathcal{X}\suprel 
g\mathcal{Y}\Leftrightarrow\mathcal{Y}\nasymp\supfun g\mathcal{X}$
and consequently $\supfun f\mathcal{X}=\supfun g\mathcal{X}$ for
every $\mathcal{X}\in\mathscr{F}(A)$, $\mathcal{Y}\in\mathscr{F}(B)$
because a set of filters is separable, thus $\supfun f=\supfun g$.\end{proof}
\begin{prop}
$\supfun f\bot^{\mathscr{F}(\Src f)}=\bot^{\mathscr{F}(\Dst f)}$
for every funcoid $f$.\end{prop}
\begin{proof}
$\mathcal{Y}\nasymp\supfun f\bot^{\mathscr{F}(\Src 
f)}\Leftrightarrow\bot^{\mathscr{F}(\Src 
f)}\nasymp\supfun{f^{-1}}\mathcal{Y}\Leftrightarrow0\Leftrightarrow\mathcal{Y}
\nasymp\bot^{\mathscr{F}(\Dst f)}$.
Thus $\supfun f\bot^{\mathscr{F}(\Src f)}=\bot^{\mathscr{F}(\Dst f)}$
by separability of filters.\end{proof}
\begin{prop}
$\supfun f(\mathcal{I}\sqcup\mathcal{J})=\supfun f\mathcal{I}\sqcup\supfun 
f\mathcal{J}$
for every funcoid $f$ and $\mathcal{I},\mathcal{J}\in\mathscr{F}(\Src 
f)$.\end{prop}
\begin{proof}
~
\begin{align*}
\fullstar\supfun f(\mathcal{I}\sqcup\mathcal{J}) & =\\
\setcond{\mathcal{Y}\in\mathscr{F}}{\mathcal{Y}\nasymp\supfun 
f(\mathcal{I}\sqcup\mathcal{J})} & =\\
\setcond{\mathcal{Y}\in\mathscr{F}}{\mathcal{I}\sqcup\mathcal{J}\nasymp\supfun{
f^{-1}}\mathcal{Y}} & =\\
\setcond{\mathcal{Y}\in\mathscr{F}}{\mathcal{I}\nasymp\supfun{f^{-1}}\mathcal{Y}
\lor\mathcal{J}\nasymp\supfun{f^{-1}}\mathcal{Y}} & =\\
\setcond{\mathcal{Y}\in\mathscr{F}}{\mathcal{Y}\nasymp\supfun 
f\mathcal{I}\lor\mathcal{Y}\nasymp\supfun f\mathcal{J}} & =\\
\setcond{\mathcal{Y}\in\mathscr{F}}{\mathcal{Y}\nasymp\supfun 
f\mathcal{I}\sqcup\supfun f\mathcal{J}} & =\\
\fullstar(\supfun f\mathcal{I}\sqcup\supfun f\mathcal{J}).
\end{align*}


Thus $\supfun f(\mathcal{I}\sqcup\mathcal{J})=\supfun f\mathcal{I}\sqcup\supfun 
f\mathcal{J}$
because $\mathscr{F}(\Dst f)$ is separable.\end{proof}
\begin{prop}
For every $f\in\mathsf{FCD}(A;B)$ for every sets $A$ and $B$ we
have:
\begin{enumerate}
\item \label{fcd-f-d1}$\mathcal{K}\suprel 
f\mathcal{I}\sqcup\mathcal{J}\Leftrightarrow\mathcal{K}\suprel 
f\mathcal{I}\lor\mathcal{K}\suprel f\mathcal{J}$
for every $\mathcal{I},\mathcal{J}\in\mathscr{F}(B)$, 
$\mathcal{K}\in\mathscr{F}(A)$.
\item \label{fcd-f-d2}$\mathcal{I}\sqcup\mathcal{J}\suprel 
f\mathcal{K}\Leftrightarrow\mathcal{I}\suprel f\mathcal{K}\lor\mathcal{J}\suprel 
f\mathcal{K}$
for every $\mathcal{I},\mathcal{J}\in\mathscr{F}(A)$, 
$\mathcal{K}\in\mathscr{F}(B)$.
\end{enumerate}
\end{prop}
\begin{proof}
~
\begin{disorder}
\item [{\ref{fcd-f-d1}}] ~
\begin{align*}
\mathcal{K}\suprel f\mathcal{I}\sqcup\mathcal{J} & \Leftrightarrow\\
(\mathcal{I}\sqcup\mathcal{J})\sqcap\supfun f\mathcal{K}\ne\bot^{\mathscr{F}(B)} 
& \Leftrightarrow\\
\mathcal{I}\sqcap\supfun 
f\mathcal{K}\ne\bot^{\mathscr{F}(B)}\lor\mathcal{J}\sqcap\supfun 
f\mathcal{K}\ne\bot^{\mathscr{F}(B)} & \Leftrightarrow\\
\mathcal{K}\suprel f\mathcal{I}\lor\mathcal{K}\suprel f\mathcal{J}.
\end{align*}

\item [{\ref{fcd-f-d2}}] Similar.
\end{disorder}
\end{proof}

\subsection{Composition of funcoids}
\begin{defn}
\index{composable!funcoids}\index{funcoids!composable}Funcoids $f$
and $g$ are \emph{composable} when $\Dst f=\Src g$.
\end{defn}

\begin{defn}
\index{composition!funcoids}\index{funcoids!composition}\emph{Composition}
of composable funcoids is defined by the formula
\[
(B;C;\alpha_{2};\beta_{2})\circ(A;B;\alpha_{1};\beta_{1})=(A;C;\alpha_{2}
\circ\alpha_{1};\beta_{1}\circ\beta_{2}).
\]
\end{defn}
\begin{prop}
If $f$, $g$ are composable funcoids then $g\circ f$ is a funcoid.\end{prop}
\begin{proof}
Let $f=(A;B;\alpha_{1};\beta_{1})$, $g=(B;C;\alpha_{2};\beta_{2})$.
For every $\mathcal{X}\in\mathscr{F}(A)$, $\mathcal{Y}\in\mathscr{F}(C)$
we have
\[
\mathcal{Y}\nasymp(\alpha_{2}\circ\alpha_{1})\mathcal{X}\Leftrightarrow\mathcal{
Y}\nasymp\alpha_{2}\alpha_{1}\mathcal{X}\Leftrightarrow\alpha_{1}\mathcal{X}
\nasymp\beta_{2}\mathcal{Y}\Leftrightarrow\mathcal{X}\nasymp\beta_{1}\beta_{2}
\mathcal{Y}\Leftrightarrow\mathcal{X}\nasymp(\beta_{1}\circ\beta_{2})\mathcal{Y}
.
\]


So $(A;C;\alpha_{2}\circ\alpha_{1};\beta_{1}\circ\beta_{2})$ is a
funcoid.\end{proof}
\begin{obvious}
$\supfun{g\circ f}=\supfun g\circ\supfun f$ for every composable
funcoids $f$ and $g$.\end{obvious}
\begin{prop}
$(h\circ g)\circ f=h\circ(g\circ f)$ for every composable funcoids
$f$, $g$, $h$.\end{prop}
\begin{proof}
~
\begin{align*}
\supfun{(h\circ g)\circ f} & =\\
\supfun{h\circ g}\circ\supfun f & =\\
(\supfun h\circ\supfun g)\circ\supfun f & =\\
\supfun h\circ(\supfun g\circ\supfun f) & =\\
\supfun h\circ\supfun{g\circ f} & =\\
\supfun{h\circ(g\circ f)}.
\end{align*}
\end{proof}
\begin{thm}
$(g\circ f)^{-1}=f^{-1}\circ g^{-1}$ for every composable funcoids
$f$ and $g$.\end{thm}
\begin{proof}
$\supfun{(g\circ 
f)^{-1}}=\supfun{f^{-1}}\circ\supfun{g^{-1}}=\supfun{f^{-1}\circ g^{-1}}$.
\end{proof}

\section{Funcoid as continuation}

Let $f$ be a funcoid.
\begin{defn}
$\rsupfun f$ is the function $\mathscr{T}(\Src f)\rightarrow\mathscr{F}(\Dst f)$
defined by the formula
\[
\rsupfun fX=\supfun f\uparrow X.
\]

\end{defn}

\begin{defn}
$\rsuprel f$ is the relation between $\mathscr{T}(\Src f)$ and $\mathscr{T}(\Dst 
f)$
defined by the formula
\[
X\rsuprel fY\Leftrightarrow\uparrow X\suprel f\uparrow Y.
\]
\end{defn}
\begin{obvious}
~
\begin{enumerate}
\item $\rsupfun f=\supfun f\circ\uparrow$;
\item $\mathord{\rsuprel f}=\uparrow^{-1}\circ\mathord{\suprel f}\circ\uparrow$.
\end{enumerate}
\end{obvious}

\begin{obvious}
$\supfun g\rsupfun fX=\rsupfun{g\circ f}X$ for every $X\in\mathscr{T}(\Src 
f)$.\end{obvious}
\begin{thm}
For every funcoid $f$ and $\mathcal{X}\in\mathscr{F}(\Src f)$, 
$\mathcal{Y}\in\mathscr{F}(\Dst f)$
\begin{enumerate}
\item \label{f-filt-x}$\supfun f\mathcal{X}=\bigsqcap\rsupfun{\rsupfun 
f}\up\mathcal{X}$;
\item \label{frel-filt}$\mathcal{X}\suprel f\mathcal{Y}\Leftrightarrow\forall 
X\in\up\mathcal{X},Y\in\up\mathcal{Y}:X\rsuprel fY$.
\end{enumerate}
\end{thm}
\begin{proof}
~
\begin{widedisorder}
\item [{\ref{frel-filt}}] ~ 
\begin{align*}
\mathcal{X}\suprel f\mathcal{Y} & \Leftrightarrow\\
\mathcal{Y}\sqcap\supfun f\mathcal{X}\ne\bot^{\mathscr{F}(\Dst f)} & 
\Leftrightarrow\\
\forall Y\in\up\mathcal{Y}:\uparrow^{\Dst f}Y\sqcap\supfun 
f\mathcal{X}\ne\bot^{\mathscr{F}(\Dst f)} & \Leftrightarrow\\
\forall Y\in\up\mathcal{Y}:\mathcal{X}\suprel f\uparrow Y.
\end{align*}



Analogously $\mathcal{X}\suprel f\mathcal{Y}\Leftrightarrow\forall 
X\in\up\mathcal{X}:\uparrow X\suprel f\mathcal{Y}$.
Combining these two equivalences we get
\[
\mathcal{X}\suprel f\mathcal{Y}\Leftrightarrow\forall 
X\in\up\mathcal{X},Y\in\up\mathcal{Y}:\uparrow X\suprel f\uparrow 
Y\Leftrightarrow\forall X\in\up\mathcal{X},Y\in\up\mathcal{Y}:X\rsuprel fY.
\]


\item [{\ref{f-filt-x}}] ~
\begin{align*}
\mathcal{Y}\sqcap\supfun f\mathcal{X}\ne\bot^{\mathscr{F}(\Dst f)} & 
\Leftrightarrow\\
\mathcal{X}\suprel f\mathcal{Y} & \Leftrightarrow\\
\forall X\in\up\mathcal{X}:\uparrow X\suprel f\mathcal{Y} & \Leftrightarrow\\
\forall X\in\up\mathcal{X}:\mathcal{Y}\sqcap\rsupfun fX\ne\bot^{\mathscr{F}(\Dst 
f)}.
\end{align*}



Let's denote $W=\setcond{\mathcal{Y}\sqcap\rsupfun fX}{X\in\up\mathcal{X}}$.
We will prove that $W$ is a generalized filter base. To prove this
it is enough to show that $V=\setcond{\rsupfun fX}{X\in\up\mathcal{X}}$
is a generalized filter base.


Let $\mathcal{P},\mathcal{Q}\in V$. Then $\mathcal{P}=\rsupfun fA$,
$\mathcal{Q}=\rsupfun fB$ where $A,B\in\up\mathcal{X}$; $A\sqcap 
B\in\up\mathcal{X}$
and $\mathcal{R}\sqsubseteq\mathcal{P}\sqcap\mathcal{Q}$ for 
$\mathcal{R}=\rsupfun f(A\sqcap B)\in V$.
So $V$ is a generalized filter base and thus $W$ is a generalized
filter base.


$\bot^{\mathscr{F}(\Dst f)}\notin W\Leftrightarrow\bigsqcap 
W\ne\bot^{\mathscr{F}(\Dst f)}$
by properties of generalized filter bases. That is
\[
\forall X\in\up\mathcal{X}:\mathcal{Y}\sqcap\rsupfun 
fX\neq\bot^{\mathscr{F}(\Dst 
f)}\Leftrightarrow\mathcal{Y}\sqcap\bigsqcap\rsupfun{\rsupfun 
f}\up\mathcal{X}\ne\bot^{\mathscr{F}(\Dst f)}.
\]



Comparing with the above, $\mathcal{Y}\sqcap\supfun 
f\mathcal{X}\ne\bot^{\mathscr{F}(\Dst 
f)}\Leftrightarrow\mathcal{Y}\sqcap\bigsqcap\rsupfun{\rsupfun 
f}\up\mathcal{X}\ne\bot^{\mathscr{F}(\Dst f)}$.
So $\supfun f\mathcal{X}=\bigsqcap\rsupfun{\rsupfun f}\up\mathcal{X}$
because the lattice of filters is separable.

\end{widedisorder}
\end{proof}
\begin{cor}
Let $f$ be a funcoid.
\begin{enumerate}
\item The value of $f$ can be restored from the value of $\rsupfun f$.
\item The value of $f$ can be restored from the value of $\rsuprel f$.
\end{enumerate}
\end{cor}
\begin{prop}
For every $f\in\mathsf{FCD}(A;B)$ we have (for every $I,J\in\mathscr{T}A$)
\[
\rsupfun f\bot^{\mathscr{T}(A)}=\bot^{\mathscr{F}(B)},\quad\rsupfun f(I\sqcup 
J)=\rsupfun fI\sqcup\rsupfun fJ
\]
and
\begin{align*}
\lnot(I & \rsuprel f\bot^{\mathscr{T}(B)}), & I\sqcup J\rsuprel fK & 
\Leftrightarrow I\rsuprel fK\lor J\rsuprel fK & \text{(for every 
\ensuremath{I,J\in\mathscr{T}A}, \ensuremath{K\in\mathscr{T}B})},\\
\lnot(\bot^{\mathscr{T}(A)} & \rsuprel fI), & K\rsuprel fI\sqcup J & 
\Leftrightarrow K\rsuprel fI\lor K\rsuprel fJ & \text{(for every 
\ensuremath{I,J\in\mathscr{T}B}, \ensuremath{K\in\mathscr{T}A})}.
\end{align*}
\end{prop}
\begin{proof}
$\rsupfun f\bot^{\mathscr{T}(A)}=\supfun f\bot^{\mathscr{T}(A)}=\supfun 
f\bot^{\mathscr{F}(A)}=\bot^{\mathscr{F}(B)}$;
\[
\rsupfun f(I\sqcup J)=\supfun f\uparrow(I\sqcup J)=\supfun f\uparrow 
I\sqcup\supfun f\uparrow J=\rsupfun fI\sqcup\rsupfun fJ.
\]


$I\rsuprel 
f\bot^{\mathscr{T}(B)}\Leftrightarrow\bot^{\mathscr{F}(B)}\nasymp\supfun 
f\uparrow I\Leftrightarrow0$;
\begin{align*}
I\sqcup J\rsuprel fK & \Leftrightarrow\\
\uparrow(I\sqcup J)\suprel f\uparrow K & \Leftrightarrow\\
\uparrow K\nasymp\supfun f\uparrow(I\sqcup J) & \Leftrightarrow\\
\uparrow K\nasymp\rsupfun f(I\sqcup J) & \Leftrightarrow\\
\uparrow K\nasymp\rsupfun fI\sqcup\rsupfun fJ & \Leftrightarrow\\
\uparrow K\nasymp\rsupfun fI\lor\uparrow K\nasymp\rsupfun fJ & \Leftrightarrow\\
I\rsuprel fK\lor J\rsuprel fK.
\end{align*}


The rest follows from symmetry.\end{proof}
\begin{thm}
\label{fcd-as-cont}(fundamental theorem of theory of funcoids) Fix sets $A$ and $B$. Let $L_{F}=\mylamdba 
f{\mathsf{FCD}(A;B)}{\rsupfun f}$
and $L_{R}=\mylamdba f{\mathsf{FCD}(A;B)}{\mathord{\rsuprel f}}$.
\begin{enumerate}
\item \label{main-f}$L_{F}$ is a bijection from the set $\mathsf{FCD}(A;B)$
to the set of functions $\alpha\in\mathscr{F}(B)^{\mathscr{T}A}$
that obey the conditions (for every $I,J\in\mathscr{T}A$)
\begin{equation}
\alpha\bot^{\mathscr{T}(A)}=\bot^{\mathscr{F}(B)},\quad\alpha(I\sqcup J)=\alpha 
I\sqcup\alpha J.\label{fchar-alph}
\end{equation}



For such $\alpha$ it holds (for every $\mathcal{X}\in\mathscr{F}(A)$)
\begin{equation}
\supfun{L_{F}^{-1}\alpha}\mathcal{X}=\bigsqcap\rsupfun{\alpha}\up\mathcal{X}
.\label{fchar-alph-c}
\end{equation}


\item \label{main-r}$L_{R}$ is a bijection from the set $\mathsf{FCD}(A;B)$
to the set of binary relations 
$\delta\in\subsets(\mathscr{T}A\times\mathscr{T}B)$
that obey the conditions
\begin{equation}
\begin{aligned}\lnot(I & \mathrel\delta\bot^{\mathscr{T}(B)}), & I\sqcup 
J\mathrel\delta K & \Leftrightarrow I\mathrel\delta K\lor J\mathrel\delta K & 
\text{(for every \ensuremath{I,J\in\mathscr{T}A}, 
\ensuremath{K\in\mathscr{T}B})},\\
\lnot(\bot^{\mathscr{T}(A)} & \mathrel\delta I), & K\mathrel\delta I\sqcup J & 
\Leftrightarrow K\mathrel\delta I\lor K\mathrel\delta J & \text{(for every 
\ensuremath{I,J\in\mathscr{T}B}, \ensuremath{K\in\mathscr{T}A})}.
\end{aligned}
\label{f-char-delt}
\end{equation}



For such $\delta$ it holds (for every $\mathcal{X}\in\mathscr{F}(A)$,
$\mathcal{Y}\in\mathscr{F}(B)$)
\begin{equation}
\mathcal{X}\suprel{L_{R}^{-1}\delta}\mathcal{Y}\Leftrightarrow\forall 
X\in\up\mathcal{X},Y\in\up\mathcal{Y}:X\mathrel\delta Y.\label{f-char-delt-c}
\end{equation}


\end{enumerate}
\end{thm}
\begin{proof}
Injectivity of $L_{F}$ and $L_{R}$, formulas (\ref{fchar-alph-c})
(for $\alpha\in\im L_{F}$) and (\ref{f-char-delt-c}) (for $\delta\in\im L_{R}$),
formulas (\ref{fchar-alph}) and (\ref{f-char-delt}) follow from
two previous theorems. The only thing remaining to prove is that for
every $\alpha$ and $\delta$ that obey the above conditions a corresponding
funcoid $f$ exists.
\begin{widedisorder}
\item [{\ref{main-r}}] Let define $\alpha\in\mathscr{F}(B)^{\mathscr{T}A}$
by the formula $\corestar(\alpha X)=\setcond{Y\in\mathscr{T}B}{X\mathrel\delta 
Y}$
for every $X\in\mathscr{T}A$. (It is obvious that 
$\setcond{Y\in\mathscr{T}B}{X\mathrel\delta Y}$
is a free star.) Analogously it can be defined 
$\beta\in\mathscr{F}(A)^{\mathscr{T}B}$
by the formula $\corestar(\beta Y)=\setcond{X\in\mathscr{T}A}{X\mathrel\delta 
Y}$.
Let's continue $\alpha$ and $\beta$ to 
$\alpha'\in\mathscr{F}(B)^{\mathscr{F}(A)}$
and $\beta'\in\mathscr{F}(A)^{\mathscr{F}(B)}$ by the formulas
\[
\alpha'\mathcal{X}=\bigsqcap\rsupfun{\alpha}\up\mathcal{X}\quad\text{and}
\quad\beta'\mathcal{Y}=\bigsqcap\rsupfun{\beta}\up\mathcal{Y}
\]
and $\delta$ to $\delta'$ by the formula
\[
\mathcal{X}\mathrel{\delta'}\mathcal{Y}\Leftrightarrow\forall 
X\in\up\mathcal{X},Y\in\up\mathcal{Y}:X\mathrel\delta Y.
\]



$\mathcal{Y}\sqcap\alpha'\mathcal{X}\ne\bot^{\mathscr{F}(B)}
\Leftrightarrow\mathcal{Y}\sqcap\bigsqcap\rsupfun{\alpha}\up\mathcal{X}\ne\bot^{
\mathscr{F}(B)}\Leftrightarrow\bigsqcap\rsupfun{\mathcal{Y}\sqcap}\rsupfun{
\alpha}\up\mathcal{X}\ne\bot^{\mathscr{F}(B)}$.
Let's prove that
\[
W=\rsupfun{\mathcal{Y}\sqcap}\rsupfun{\alpha}\up\mathcal{X}
\]
is a generalized filter base: To prove it is enough to show that 
$\rsupfun{\alpha}\up\mathcal{X}$
is a generalized filter base. If 
$\mathcal{A},\mathcal{B}\in\rsupfun{\alpha}\up\mathcal{X}$
then exist $X_{1},X_{2}\in\up\mathcal{X}$ such that $\mathcal{A}=\alpha X_{1}$,
$\mathcal{B}=\alpha X_{2}$.


Then $\alpha(X_{1}\sqcap X_{2})\in\rsupfun{\alpha}\up\mathcal{X}$.
So $\rsupfun{\alpha}\up\mathcal{X}$ is a generalized filter base
and thus $W$ is a generalized filter base.


By properties of generalized filter bases, 
$\bigsqcap\rsupfun{\mathcal{Y}\sqcap}\rsupfun{\alpha}\mathcal{X}\ne\bot^{
\mathscr{F}(B)}$
is equivalent to
\[
\forall X\in\up\mathcal{X}:\mathcal{Y}\sqcap\alpha X\ne\bot^{\mathscr{F}(B)},
\]
what is equivalent to
\begin{align*}
\forall X\in\up\mathcal{X},Y\in\up\mathcal{Y}:\uparrow Y\sqcap\alpha 
X\ne\bot^{\mathscr{F}(B)} & \Leftrightarrow\\
\forall X\in\up\mathcal{X},Y\in\up\mathcal{Y}:Y\in\corestar(\alpha X) & 
\Leftrightarrow\\
\forall X\in\up\mathcal{X},Y\in\up\mathcal{Y}:X\mathrel\delta Y.
\end{align*}



Combining the equivalencies we get 
$\mathcal{Y}\sqcap\alpha'\mathcal{X}\ne\bot^{\mathscr{F}(B)}
\Leftrightarrow\mathcal{X}\mathrel{\delta'}\mathcal{Y}$.
Analogously 
$\mathcal{X}\sqcap\beta'\mathcal{Y}\ne\bot^{\mathscr{F}(A)}
\Leftrightarrow\mathcal{X}\mathrel{\delta'}\mathcal{Y}$.
So 
$\mathcal{Y}\sqcap\alpha'\mathcal{X}\ne\bot^{\mathscr{F}(B)}
\Leftrightarrow\mathcal{X}\sqcap\beta'\mathcal{Y}\ne\bot^{\mathscr{F}(A)}$,
that is $(A;B;\alpha';\beta')$ is a funcoid. From the formula 
$\mathcal{Y}\sqcap\alpha'\mathcal{X}\ne\bot^{\mathscr{F}(B)}
\Leftrightarrow\mathcal{X}\mathrel{\delta'}\mathcal{Y}$
it follows that
\[
X\rsuprel{(A;B;\alpha';\beta')}Y\Leftrightarrow\uparrow Y\sqcap\alpha'\uparrow 
X\ne\bot^{\mathscr{F}(B)}\Leftrightarrow\uparrow X\mathrel{\delta'}\uparrow 
Y\Leftrightarrow X\mathrel\delta Y.
\]


\item [{\ref{main-f}}] Let define the relation 
$\delta\in\subsets(\mathscr{T}A\times\mathscr{T}B)$
by the formula $X\mathrel\delta Y\Leftrightarrow\uparrow Y\sqcap\alpha 
X\ne\bot^{\mathscr{F}(B)}$.


That $\lnot(I\mathrel\delta\bot^{\mathscr{T}B})$ and 
$\lnot(\bot^{\mathscr{T}A}\mathrel\delta I)$
is obvious. We have
\begin{align*}
I\sqcup J\mathrel\delta K & \Leftrightarrow\\
\uparrow K\sqcap\alpha(I\sqcup J)\ne\bot^{\mathscr{F}(B)} & \Leftrightarrow\\
\uparrow K\sqcap(\alpha I\sqcup\alpha J)\ne\bot^{\mathscr{F}(B)} & 
\Leftrightarrow\\
\uparrow K\sqcap\alpha I\ne\bot^{\mathscr{F}(B)}\lor\uparrow K\sqcap\alpha 
J\ne\bot^{\mathscr{F}(B)} & \Leftrightarrow\\
I\mathrel\delta K\lor J\mathrel\delta K
\end{align*}
and
\begin{align*}
K\mathrel\delta I\sqcup J & \Leftrightarrow\\
\uparrow(I\sqcup J)\sqcap\alpha K\ne\bot^{\mathscr{F}(B)} & \Leftrightarrow\\
(\uparrow I\sqcup\uparrow J)\sqcap\alpha K\ne\bot^{\mathscr{F}(B)} & 
\Leftrightarrow\\
\uparrow I\sqcap\alpha K\ne\bot^{\mathscr{F}(B)}\lor\uparrow J\sqcap\alpha 
K\ne\bot^{\mathscr{F}(B)} & \Leftrightarrow\\
K\mathrel\delta I\lor K\mathrel\delta J.
\end{align*}



That is the formulas (\ref{f-char-delt}) are true.


Accordingly to the above there exists a funcoid $f$ such that
\[
\mathcal{X}\suprel f\mathcal{Y}\Leftrightarrow\forall 
X\in\up\mathcal{X},Y\in\up\mathcal{Y}:X\mathrel\delta Y.
\]



For every $X\in\mathscr{T}A$, $Y\in\mathscr{T}B$ we have:
\[
\uparrow Y\sqcap\supfun f\uparrow 
X\ne\bot^{\mathscr{F}(B)}\Leftrightarrow\uparrow X\suprel f\uparrow 
Y\Leftrightarrow X\mathrel\delta Y\Leftrightarrow\uparrow Y\sqcap\alpha 
X\ne\bot^{\mathscr{F}(B)},
\]
consequently $\forall X\in\mathscr{T}A:\alpha X=\supfun f\uparrow X=\rsupfun 
fX$.

\end{widedisorder}
\end{proof}
Note that by the last theorem to every (quasi-)proximity $\delta$ corresponds
a unique funcoid. So funcoids are a generalization of (quasi-)proximity
structures. Reverse funcoids can be considered as a generalization
of conjugate quasi-proximity.
\begin{cor}
If $\alpha\in\mathscr{F}(B)^{\mathscr{T}A}$, 
$\beta\in\mathscr{F}(A)^{\mathscr{T}B}$
are functions such that $Y\nasymp\alpha X\Leftrightarrow X\nasymp\beta Y$
for every $X\in\mathscr{T}A$, $Y\in\mathscr{T}B$, then there exists
exactly one funcoid $f$ such that $\langle f\rangle^{\ast}=\alpha$,
$\langle f^{-1}\rangle^{\ast}=\beta$.\end{cor}
\begin{proof}
Prove $\alpha(I\sqcup J)=\alpha I\sqcup\alpha J$. Really, 
$Y\nasymp\alpha(I\sqcup J)\Leftrightarrow I\sqcup J\nasymp\beta Y\Leftrightarrow 
I\nasymp\beta Y\vee J\nasymp\beta Y\Leftrightarrow Y\nasymp\alpha I\vee 
Y\nasymp\alpha J\Leftrightarrow Y\nasymp\alpha I\sqcup\alpha J$.
So $\alpha(I\sqcup J)=\alpha I\sqcup\alpha J$ by star-separability.
Similarly $\beta(I\sqcup J)=\beta I\sqcup\beta J$.

Thus by the theorem there exists a funcoid $f$ such that $\langle 
f\rangle^{\ast}=\alpha$,
$\langle f^{-1}\rangle^{\ast}=\beta$.

That this funcoid is unique, follows from the above.\end{proof}
\begin{defn}
Any $\mathbf{Rel}$-morphism $F:A\rightarrow B$ corresponds to a
funcoid $\uparrow^{\mathsf{FCD}}F\in\mathsf{FCD}(A;B)$, where by
definition 
$\supfun{\uparrow^{\mathsf{FCD}}F}\mathcal{X}=\bigsqcap^{\mathscr{F}}\rsupfun{
\rsupfun F}\up\mathcal{X}$
for every $\mathcal{X}\in\mathscr{F}(A)$.
\end{defn}
Using the last theorem it is easy to show that this definition is
monovalued and does not contradict to former stuff. (Take 
$\alpha=\uparrow\circ\rsupfun F$.)
\begin{prop}
$\rsupfun{\uparrow^{\mathsf{FCD}}f}X=\rsupfun fX$ for a 
$\mathbf{Rel}$-morphism~$f$
and $X\in\mathscr{T}\Src f$.\end{prop}
\begin{proof}
$\rsupfun{\uparrow^{\mathsf{FCD}}f}X=\min\rsupfun{\uparrow}\rsupfun{\rsupfun 
f}\up X=\uparrow\rsupfun fX=\rsupfun fX$.\end{proof}
\begin{cor}
$\mathord{\rsuprel{\uparrow^{\mathsf{FCD}}f}}=\mathord{\rsuprel f}$
for every $\mathbf{Rel}$-morphism~$f$.\end{cor}
\begin{proof}
$X\rsuprel{\uparrow^{\mathsf{FCD}}f}Y\Leftrightarrow 
Y\nasymp\rsupfun{\uparrow^{\mathsf{FCD}}f}X\Leftrightarrow Y\nasymp\rsupfun 
fX\Leftrightarrow X\rsuprel fY$
for $X\in\mathscr{T}\Src f$, $Y\in\mathscr{T}\Dst f$.\end{proof}
\begin{defn}
$\uparrow^{\mathsf{FCD}(A;B)}f=\uparrow^{\mathsf{FCD}}(A;B;f)$ for
every binary relation~$f$ between sets~$A$ and~$B$.
\end{defn}

\begin{defn}
\index{funcoid!principal}\index{principal!funcoid}Funcoids corresponding
to a binary relation (= multivalued function) are called \emph{principal
funcoids}.\end{defn}
\begin{prop}
$\uparrow^{\mathsf{FCD}}g\circ\uparrow^{\mathsf{FCD}}f=\uparrow^{\mathsf{FCD}}
(g\circ f)$
for composable morphisms~$f$,~$g$ of category~$\mathbf{Rel}$.\end{prop}
\begin{proof}
For every $X\in\mathscr{T}\Src f$
\begin{multline*}
\rsupfun{\uparrow^{\mathsf{FCD}}g\circ\uparrow^{\mathsf{FCD}}f}X=\rsupfun{
\uparrow^{\mathsf{FCD}}g}\rsupfun{\uparrow^{\mathsf{FCD}}f}X=\\
\rsupfun g\rsupfun fX=\rsupfun{g\circ 
f}X=\rsupfun{\uparrow^{\mathsf{FCD}}(g\circ f)}X.
\end{multline*}
 
\end{proof}
We may equate principal funcoids with corresponding binary relations
by the method of appendix~\ref{app:prim-exists}. This is useful
for describing relationships of funcoids and binary relations, such
as for the formulas of continuous functions and continuous funcoids
(see below).

Thus $(\mathsf{FCD}(A;B);\mathbf{Rel}(A;B))$ is a filtrator. I call
it \emph{filtrator of funcoids}.
\begin{thm}
\label{supfun-genbase}If $S$ is a generalized filter base on $\Src f$
then $\supfun f\bigsqcap S=\bigsqcap\rsupfun{\supfun f}S$ for every
funcoid $f$.\end{thm}
\begin{proof}
$\supfun f\bigsqcap S\sqsubseteq\supfun fX$ for every $X\in S$ and
thus $\supfun f\bigsqcap S\sqsubseteq\bigsqcap\rsupfun{\supfun f}S$.

By properties of generalized filter bases:

\begin{align*}
\supfun f\bigsqcap S & =\\
\bigsqcap\rsupfun{\rsupfun f}\up\bigsqcap S & =\\
\bigsqcap\rsupfun{\rsupfun f}\setcond X{\exists\mathcal{P}\in 
S:X\in\up\mathcal{P}} & =\\
\bigsqcap\setcond{\rsupfun fX}{\exists\mathcal{P}\in S:X\in\up\mathcal{P}} & 
\sqsupseteq\\
\bigsqcap_{\mathcal{P}\in S}\supfun f\mathcal{P} & =\\
\bigsqcap\rsupfun{\supfun f}S.
\end{align*}
\end{proof}
\begin{prop}
$\mathcal{X}\suprel f\bigsqcap S\Leftrightarrow\exists\mathcal{Y}\in 
S:\mathcal{X}\suprel f\mathcal{Y}$
if $f$ is a funcoid and $S$ is a generalized filter base on $\Dst f$.\end{prop}
\begin{proof}
~
\begin{multline*}
\mathcal{X}\suprel f\bigsqcap S\Leftrightarrow\bigsqcap S\sqcap\supfun 
f\mathcal{X}\neq\bot\Leftrightarrow\bigsqcap\langle\langle 
f\rangle\mathcal{X}\sqcap\rangle^{\ast}S\neq\bot\Leftrightarrow\\
\text{(by properties of generalized filter bases)}\Leftrightarrow\\
\exists\mathcal{Y}\in\langle\supfun 
f\mathcal{X}\sqcap\rangle^{\ast}S:\mathcal{Y}
\neq\bot\Leftrightarrow\exists\mathcal{Y}\in S:\langle 
f\rangle\mathcal{X}\sqcap\mathcal{Y}\neq\bot\Leftrightarrow\exists\mathcal{Y}\in 
S:\mathcal{X}\suprel f\mathcal{Y}.
\end{multline*}
\end{proof}
\begin{defn}
\index{preserve filtered meets}
A function $f$ between two posets is said to \emph{preserve filtered meets}, when $f\bigsqcap S=\bigsqcap\rsupfun{f}S$
whenever $\bigsqcap S$ is defined for a filter base~$S$ on the first of the two posets.
\end{defn}
\begin{thm}
\label{fcd-as-func}(discovered by \noun{Todd Trimble}) A function
$\varphi:\mathscr{F}(A)\rightarrow\mathscr{F}(B)$ preserves finite
joins (including nullary joins) and filtered meets iff there exists
a funcoid $f$ such that $\supfun f=\varphi$.\end{thm}
\begin{proof}
Backward implication follows from above.

Let $\psi=\varphi|_{\mathscr{T}A}$. Then $\psi$ preserves bottom
element and binary joins. Thus there exists a funcoid $f$ such that
$\rsupfun f=\psi$.

It remains to prove that $\supfun f=\varphi$.

Really, $\supfun f\mathcal{X}=\bigsqcap\rsupfun{\rsupfun 
f}\up\mathcal{X}=\bigsqcap\rsupfun{\psi}\up\mathcal{X}=\bigsqcap\rsupfun{\varphi
}\up\mathcal{X}=\varphi\bigsqcap\up\mathcal{X}=\varphi\mathcal{X}$
for every $\mathcal{X}\in\mathscr{F}(A)$.\end{proof}
\begin{cor}
Funcoids $f$ from $A$ to $B$ bijectively correspond by the formula
$\langle f\rangle=\varphi$ to functions 
$\varphi:\mathscr{F}(A)\rightarrow\mathscr{F}(B)$
preserving finite joins and filtered meets.
\end{cor}

\section{\label{fcd-rel-another}Another way to represent funcoids as binary
relations}

This is based on a\noun{ Todd Trimble}'s idea.
\begin{defn}
The binary relation 
$\xi^{\circledast}\in\subsets(\mathscr{F}(\Src\xi)\times\mathscr{F}(\Dst\xi))$
for a funcoid $\xi$ is defined by the formula 
$\mathcal{A}\mathrel{\xi^{\circledast}}\mathcal{B}\Leftrightarrow\mathcal{B}
\sqsupseteq\langle\xi\rangle\mathcal{A}$.
\end{defn}

\begin{defn}
The binary relation 
$\xi^{\ast}\in\subsets(\mathscr{T}\Src\xi\times\mathscr{T}\Dst\xi)$
for a funcoid $\xi$ is defined by the formula 
\[
A\mathrel{\xi^{\ast}}B\Leftrightarrow B\sqsupseteq\langle\xi\rangle 
A\Leftrightarrow B\in\up\langle\xi\rangle A.
\]
\end{defn}
\begin{prop}
Funcoid $\xi$ can be restored from
\begin{enumerate}
\item the value of $\xi^{\circledast}$;
\item the value of $\xi^{\ast}$.
\end{enumerate}
\end{prop}
\begin{proof}
~
\begin{enumerate}
\item The value of $\supfun{\xi}$ can be restored from $\xi^{\circledast}$.
\item The value of $\rsupfun{\xi}$ can be restored from $\xi^{\ast}$.
\end{enumerate}
\end{proof}
\begin{thm}
Let $\nu$ and $\xi$ be composable funcoids. Then:
\begin{enumerate}
\item 
\label{fcomp-1}$\xi^{\circledast}\circ\nu^{\circledast}=(\xi\circ\nu)^{
\circledast}$;
\item \label{fcomp-2}$\xi^{\ast}\circ\nu^{\ast}=(\xi\circ\nu)^{\ast}$.
\end{enumerate}
\end{thm}
\begin{proof}
~
\begin{widedisorder}
\item [{\ref{fcomp-1}}] 
\begin{multline*}
\mathcal{A}\mathrel{(\xi^{\circledast}\circ\nu^{\circledast})}\mathcal{C}
\Leftrightarrow\exists\mathcal{B}:\left(\mathcal{A}\mathrel{\nu^{\circledast}}
\mathcal{B}\wedge\mathcal{B}\mathrel{\xi^{\circledast}}\mathcal{C}
\right)\Leftrightarrow\\
\exists\mathcal{B}\in\mathscr{F}(\Dst\nu):(\mathcal{B}\sqsupseteq\supfun{\nu}
\mathcal{A}\wedge\mathcal{C}\sqsupseteq\supfun{\xi}\mathcal{B})\Leftrightarrow\\
\mathcal{C}\sqsupseteq\supfun{\xi}\supfun{\nu}\mathcal{A}\Leftrightarrow\mathcal
{C}\sqsupseteq\supfun{\xi\circ\nu}\mathcal{A}\Leftrightarrow\mathcal{A}\mathrel{
(\xi\circ\nu)^{\circledast}}\mathcal{C}.
\end{multline*}

\item [{\ref{fcomp-2}}] ~ 
\begin{multline*}
A\mathrel{(\xi^{\ast}\circ\nu^{\ast})}C\Leftrightarrow\exists 
B:\left(A\mathrel{\nu^{\ast}}B\wedge 
B\mathrel{\xi^{\ast}}C\right)\Leftrightarrow\\
\exists B:(B\in\up\supfun{\nu}A\wedge 
C\in\up\supfun{\xi}B)\Leftrightarrow\exists 
B\in\up\supfun{\nu}A:C\in\up\supfun{\xi}B.
\end{multline*}

\end{widedisorder}
$A\mathrel{(\xi\circ\nu)^{\ast}}C\Leftrightarrow C\in\up\langle\xi\circ\nu\rangle 
B\Leftrightarrow C\in\up\langle\xi\rangle\supfun{\nu}B$.

It remains to prove 
\[
\exists B\in\up\supfun{\nu}A:C\in\up\langle\xi\rangle B\Leftrightarrow 
C\in\up\supfun{\xi}\langle\nu\rangle A.
\]
$\exists B\in\up\supfun{\nu}A:C\in\up\langle\xi\rangle B\Rightarrow 
C\in\up\supfun{\xi}\supfun{\nu}A$
is obvious.

Let $C\in\up\supfun{\xi}\supfun{\nu}A$. Then 
$C\in\up\bigsqcap\rsupfun{\supfun{\xi}}\up\supfun{\nu}A$;
so by properties of generalized filter bases, $\exists 
P\in\rsupfun{\supfun{\xi}}\up\supfun{\nu}A:C\in\up P$;
$\exists B\in\up\supfun{\nu}A:C\in\up\supfun{\xi}B$.
\end{proof}
\begin{rem}
The above theorem is interesting by the fact that composition of funcoids
is represented as relational composition of binary relations.
\end{rem}

\section{Lattices of funcoids}
\begin{defn}
$f\sqsubseteq g\eqdef\mathord{\suprel f}\subseteq\mathord{\suprel g}$ for $f,g\in\mathsf{FCD}(A;B)$
for every sets $A$, $B$.
\end{defn}
Thus every $\mathsf{FCD}(A;B)$ is a poset. (It's taken into account
that $\mathord{\suprel f}\ne\mathord{\suprel g}$ when $f\ne g$.)

\index{filtrator!of funcoids}We will consider filtrators (\emph{filtrators
of funcoids}) whose base is $\mathsf{FCD}(A;B)$ and whose core are
principal funcoids from $A$ to $B$.
\begin{lem}
$\rsupfun fX=\bigsqcap_{F\in\up f}^{\mathscr{F}}\rsupfun FX$ for
every funcoid $f$ and typed set $X\in\mathscr{T}(\Src f)$.\end{lem}
\begin{proof}
Obviously $\rsupfun fX\sqsubseteq\bigsqcap_{F\in\up f}^{\mathscr{F}}\rsupfun 
FX$.

Let $B\in\up\rsupfun fX$. Let $F_{B}=X\times B\sqcup\overline{X}\times\top$.

$\rsupfun{F_{B}}X=B$.

Let $P\in\mathscr{T}(\Src f)$. We have 
\[
\bot^{\mathscr{T}(\Src f)}\ne P\sqsubseteq 
X\Rightarrow\rsupfun{F_{B}}P=B\sqsupseteq\rsupfun fP
\]
and
\[
P\nsqsubseteq 
X\Rightarrow\rsupfun{F_{B}}P=\top^{\mathscr{F}(\Dst f)}\sqsupseteq\rsupfun fP.
\]


Thus $\rsupfun{F_{B}}P\sqsupseteq\rsupfun fP$ for every $P$ and
so $F_{B}\sqsupseteq f$ that
is $F_{B}\in\up f$.

Thus $\forall B\in\up\rsupfun fX:B\in\up\bigsqcap_{F\in\up 
f}^{\mathscr{F}}\rsupfun FX$
because $B\in\up\rsupfun{F_{B}}X$.

So $\bigsqcap_{F\in\up f}\rsupfun FX\sqsubseteq\rsupfun fX$.\end{proof}
\begin{thm}
\label{fcd-up-x}$\supfun f\mathcal{X}=\bigsqcap_{F\in\up f}^{\mathscr{F}}\supfun 
F\mathcal{X}$
for every funcoid $f$ and $\mathcal{X}\in\mathscr{F}(\Src f)$.\end{thm}
\begin{proof}
~
\begin{align*}
\bigsqcap_{F\in\up f}^{\mathscr{F}}\supfun F\mathcal{X} & =\\
\bigsqcap_{F\in\up f}^{\mathscr{F}}\bigsqcap^{\mathscr{F}}\rsupfun{\rsupfun F}\up\mathcal{X} & =\\
\bigsqcap_{F\in\up 
f}^{\mathscr{F}}\bigsqcap_{X\in\up\mathcal{X}}^{\mathscr{F}}\rsupfun FX & =\\
\bigsqcap_{X\in\up\mathcal{X}}^{\mathscr{F}}\bigsqcap_{F\in\up 
f}^{\mathscr{F}}\rsupfun FX & =\\
\bigsqcap_{X\in\up\mathcal{X}}^{\mathscr{F}}\rsupfun fX & =\\
\supfun f\mathcal{X}
\end{align*}
(the lemma used).
\end{proof}
Below it is shown that $\mathsf{FCD}(A;B)$ are complete lattices
for every sets $A$ and $B$. We will apply lattice operations to
subsets of such sets without explicitly mentioning $\mathsf{FCD}(A;B)$.
\begin{thm}
\label{fcd-join-sets}$\mathsf{FCD}(A;B)$ is a complete lattice (for
every sets $A$ and $B$). For every $R\in\subsets\mathsf{FCD}(A;B)$
and $X\in\mathscr{T}A$, $Y\in\mathscr{T}B$
\begin{enumerate}
\item \label{sr-join}$X\rsuprel{\bigsqcup R}Y\Leftrightarrow\exists f\in 
R:X\rsuprel fY$;
\item \label{sf-join}$\rsupfun{\bigsqcup R}X=\bigsqcup_{f\in R}\rsupfun fX$.
\end{enumerate}
\end{thm}
\begin{proof}
Accordingly \cite{pm:complete-lattice-criteria} to prove that it
is a complete lattice it's enough to prove existence of all joins.
\begin{widedisorder}
\item [{\ref{sf-join}}] $\alpha X\eqdef\bigsqcup_{f\in R}\rsupfun fX$.
We have $\alpha\bot^{\mathscr{T}(\Src f)}=\bot^{\mathscr{F}(\Dst f)}$;
\begin{align*}
\alpha(I\sqcup J) & =\\
\bigsqcup_{f\in R}\rsupfun f(I\sqcup J) & =\\
\bigsqcup_{f\in R}(\rsupfun fI\sqcup\rsupfun fJ) & =\\
\bigsqcup_{f\in R}\rsupfun fI\sqcup\bigsqcup\rsupfun fJ & =\\
\alpha I\sqcup\alpha J.
\end{align*}



So $\rsupfun h=\alpha$ for some funcoid $h$. Obviously
\begin{equation}
\forall f\in R:h\sqsupseteq f.\label{min-fcd-join}
\end{equation}



And $h$ is the least funcoid for which holds the condition (\ref{min-fcd-join}).
So $h=\bigsqcup R$.

\item [{\ref{sr-join}}] ~ 
\begin{align*}
X\rsuprel{\bigsqcup R}Y & \Leftrightarrow\\
\uparrow Y\sqcap\rsupfun{\bigsqcup R}X\ne\bot^{\mathscr{F}(\Dst f)} & 
\Leftrightarrow\\
\uparrow Y\sqcap\bigsqcup_{f\in R}\rsupfun fX\ne\bot^{\mathscr{F}(\Dst f)} & 
\Leftrightarrow\\
\exists f\in R:\uparrow Y\sqcap\rsupfun fX\ne\bot^{\mathscr{F}(\Dst f)} & 
\Leftrightarrow\\
\exists f\in R:X\rsuprel fY
\end{align*}
(used proposition~\ref{b-f-back-distr}).
\end{widedisorder}
\end{proof}
In the next theorem, compared to the previous one, the class of infinite
joins is replaced with lesser class of binary joins and simultaneously
class of sets is changed to more wide class of filters.
\begin{thm}
\label{fcd-fin-join}For every $f,g\in\mathsf{FCD}(A;B)$ and 
$\mathcal{X}\in\mathscr{F}(A)$
(for every sets $A$,~$B$)
\begin{enumerate}
\item \label{fjoin-x}$\supfun{f\sqcup g}\mathcal{X}=\supfun 
f\mathcal{X}\sqcup\supfun g\mathcal{X}$;
\item \label{fjoin-r}$\mathord{\suprel{f\sqcup g}}=\mathord{\suprel 
f}\cup\mathord{\suprel g}$.
\end{enumerate}
\end{thm}
\begin{proof}
~
\begin{widedisorder}
\item [{\ref{fjoin-x}}] Let $\alpha\mathcal{X}\eqdef\supfun 
f\mathcal{X}\sqcup\supfun g\mathcal{X}$;
$\beta\mathcal{Y}\eqdef\supfun{f^{-1}}\mathcal{Y}\sqcup\supfun{g^{-1}}\mathcal{Y
}$
for every $\mathcal{X}\in\mathscr{F}(A)$, $\mathcal{Y}\in\mathscr{F}(B)$.
Then
\begin{align*}
\mathcal{Y}\sqcap\alpha\mathcal{X}\ne\bot^{\mathscr{F}(B)} & \Leftrightarrow\\
\mathcal{Y}\sqcap\supfun 
f\mathcal{X}\ne\bot^{\mathscr{F}(B)}\lor\mathcal{Y}\sqcap\supfun 
g\mathcal{X}\ne\bot^{\mathscr{F}(B)} & \Leftrightarrow\\
\mathcal{X}\sqcap\supfun{f^{-1}}\mathcal{Y}\ne\bot^{\mathscr{F}(A)}\lor\mathcal{
X}\sqcap\supfun{g^{-1}}\mathcal{Y}\ne\bot^{\mathscr{F}(A)} & \Leftrightarrow\\
\mathcal{X}\sqcap\beta\mathcal{Y}\ne\bot^{\mathscr{F}(A)}.
\end{align*}
So $h=(A;B;\alpha;\beta)$ is a funcoid. Obviously $h\sqsupseteq f$
and $h\sqsupseteq g$. If $p\sqsupseteq f$ and $p\sqsupseteq g$
for some funcoid $p$ then $\supfun p\mathcal{X}\sqsupseteq\supfun 
f\mathcal{X}\sqcup\supfun g\mathcal{X}=\supfun h\mathcal{X}$
that is $p\sqsupseteq h$. So $f\sqcup g=h$.
\item [{\ref{fjoin-r}}] For every $\mathcal{X}\in\mathscr{F}(A)$, 
$\mathcal{Y}\in\mathscr{F}(B)$
we have 
\begin{align*}
\mathcal{X}\suprel{f\sqcup g}\mathcal{Y} & \Leftrightarrow\\
\mathcal{Y}\sqcap\supfun{f\sqcup g}\mathcal{X}\ne\bot^{\mathscr{F}(B)} & 
\Leftrightarrow\\
\mathcal{Y}\sqcap(\supfun f\mathcal{X}\sqcup\supfun 
g\mathcal{X})\ne\bot^{\mathscr{F}(B)} & \Leftrightarrow\\
\mathcal{Y}\sqcap\supfun 
f\mathcal{X}\ne\bot^{\mathscr{F}(B)}\lor\mathcal{Y}\sqcap\supfun 
g\mathcal{X}\ne\bot^{\mathscr{F}(B)} & \Leftrightarrow\\
\mathcal{X}\suprel f\mathcal{Y}\lor\mathcal{X}\suprel g\mathcal{Y}.
\end{align*}

\end{widedisorder}
\end{proof}

\section{More on composition of funcoids}
\begin{prop}
\label{comp-fcd-r}$\mathord{\suprel{g\circ f}}=\mathord{\suprel g}\circ\supfun 
f=\supfun{g^{-1}}^{-1}\circ\mathord{\suprel f}$
for every composable funcoids $f$ and $g$.\end{prop}
\begin{proof}
For every $\mathcal{X}\in\mathscr{F}(\Src f)$, $\mathcal{Y}\in\mathscr{F}(\Dst 
g)$
we have
\begin{align*}
\mathcal{X}\suprel{g\circ f}\mathcal{Y} & \Leftrightarrow\\
\mathcal{Y}\sqcap\supfun{g\circ f}\mathcal{X}\ne\bot^{\mathscr{F}(\Dst g)} & 
\Leftrightarrow\\
\mathcal{Y}\sqcap\supfun g\supfun f\mathcal{X}\ne\bot^{\mathscr{F}(\Dst g)} & 
\Leftrightarrow\\
\supfun f\mathcal{X}\suprel g\mathcal{Y} & \Leftrightarrow\\
\mathcal{X}\mathrel{(\mathord{\suprel g}\circ\supfun f)}\mathcal{Y}
\end{align*}
and
\begin{align*}
\mathord{\suprel{g\circ f}} & =\\
\mathord{\suprel{(f^{-1}\circ g^{-1})^{-1}}} & =\\
\mathord{\suprel{f^{-1}\circ g^{-1}}}^{-1} & =\\
(\mathord{\suprel{f^{-1}}}\circ\supfun{g^{-1}})^{-1} & =\\
\supfun{g^{-1}}^{-1}\circ\mathord{\suprel f}.
\end{align*}

\end{proof}
The following theorem is a variant for funcoids of the statement (which
defines compositions of relations) that $x\mathrel{(g\circ 
f)}z\Leftrightarrow\exists y:(x\mathrel fy\land y\mathrel gz)$
for every $x$ and $z$ and every binary relations $f$ and $g$.
\begin{thm}
\label{fcd-atom-middle}For every sets $A$, $B$, $C$ and $f\in\mathsf{FCD}(A;B)$,
$g\in\mathsf{FCD}(B;C)$ and $\mathcal{X}\in\mathscr{F}(A)$, 
$\mathcal{Z}\in\mathscr{F}(C)$
\[
\mathcal{X}\suprel{g\circ f}\mathcal{Z}\Leftrightarrow\exists 
y\in\atoms^{\mathscr{F}(B)}:(\mathcal{X}\suprel fy\land y\suprel g\mathcal{Z}).
\]
\end{thm}
\begin{proof}
~
\begin{align*}
\exists y\in\atoms^{\mathscr{F}(B)}:(\mathcal{X}\suprel fy\land y\suprel 
g\mathcal{Z}) & \Leftrightarrow\\
\exists y\in\atoms^{\mathscr{F}(B)}:(\mathcal{Z}\sqcap\supfun 
gy\ne\bot^{\mathscr{F}(C)}\land y\sqcap\supfun 
f\mathcal{X}\ne\bot^{\mathscr{F}(B)}) & \Leftrightarrow\\
\exists y\in\atoms^{\mathscr{F}(B)}:(\mathcal{Z}\sqcap\supfun 
gy\ne\bot^{\mathscr{F}(C)}\land y\sqsubseteq\supfun f\mathcal{X}) & 
\Rightarrow\\
\mathcal{Z}\sqcap\supfun g\supfun f\mathcal{X}\ne\bot^{\mathscr{F}(C)} & 
\Leftrightarrow\\
\mathcal{X}\suprel{g\circ f}\mathcal{Z}.
\end{align*}


Reversely, if $\mathcal{X}\suprel{g\circ f}\mathcal{Z}$ then $\supfun 
f\mathcal{X}\suprel g\mathcal{Z}$,
consequently there exists $y\in\atoms\supfun f\mathcal{X}$ such that
$y\suprel g\mathcal{Z}$; we have $\mathcal{X}\suprel fy$.\end{proof}
\begin{thm}
For every sets $A$, $B$, $C$
\begin{enumerate}
\item $f\circ(g\sqcup h)=f\circ g\sqcup f\circ h$ for $g,h\in\mathsf{FCD}(A;B)$,
$f\in\mathsf{FCD}(B;C)$;
\item $(g\sqcup h)\circ f=g\circ f\sqcup h\circ f$ for 
$g,h\in\mathsf{FCD}(B;C)$,
$f\in\mathsf{FCD}(A;B)$.
\end{enumerate}
\end{thm}
\begin{proof}
I will prove only the first equality because the other is analogous.

For every $\mathcal{X}\in\mathscr{F}(A)$, $\mathcal{Z}\in\mathscr{F}(C)$
\begin{align*}
\mathcal{X}\suprel{f\circ(g\sqcup h)}\mathcal{Z} & \Leftrightarrow\\
\exists y\in\atoms^{\mathscr{F}(B)}:(\mathcal{X}\suprel{g\sqcup h}y\land 
y\suprel f\mathcal{Z}) & \Leftrightarrow\\
\exists y\in\atoms^{\mathscr{F}(B)}:((\mathcal{X}\suprel 
gy\lor\mathcal{X}\suprel hy)\land y\suprel f\mathcal{Z}) & \Leftrightarrow\\
\exists y\in\atoms^{\mathscr{F}(B)}:((\mathcal{X}\suprel gy\land y\suprel 
f\mathcal{Z})\lor(\mathcal{X}\suprel hy\land y\suprel f\mathcal{Z})) & 
\Leftrightarrow\\
\exists y\in\atoms^{\mathscr{F}(B)}:(\mathcal{X}\suprel gy\land y\suprel 
f\mathcal{Z})\lor\exists y\in\atoms^{\mathscr{F}(B)}:(\mathcal{X}\suprel hy\land 
y\suprel f\mathcal{Z}) & \Leftrightarrow\\
\mathcal{X}\suprel{f\circ g}\mathcal{Z}\lor\mathcal{X}\suprel{f\circ 
h}\mathcal{Z} & \Leftrightarrow\\
\mathcal{X}\suprel{f\circ g\sqcup f\circ h}\mathcal{Z}.
\end{align*}

\end{proof}
Another proof of the above theorem (without atomic filters):
\begin{proof}
~
\begin{align*}
\supfun{f\circ(g\sqcup h)}\mathcal{X} & =\\
\supfun f\supfun{g\sqcup h}\mathcal{X} & =\\
\supfun f(\supfun g\mathcal{X}\sqcup\supfun h\mathcal{X}) & =\\
\supfun f\supfun g\mathcal{X}\sqcup\supfun f\supfun h\mathcal{X} & =\\
\supfun{f\circ g}\mathcal{X}\sqcup\supfun{f\circ h}\mathcal{X} & =\\
\supfun{f\circ g\sqcup f\circ h}\mathcal{X}.
\end{align*}

\end{proof}

\section{Domain and range of a funcoid}
\begin{defn}
\index{funcoid!identity}Let $A$ be a set. The \emph{identity funcoid}
$1_{A}^{\mathsf{FCD}}=(A;A;\id_{\mathscr{F}(A)};\id_{\mathscr{F}(A)})$.
\end{defn}
\begin{obvious}
The identity funcoid is a funcoid.\end{obvious}
\begin{prop}
$\mathord{\suprel f}=\mathord{\suprel{1_{\Dst f}}}\circ\supfun f$
for every funcoid $f$.\end{prop}
\begin{proof}
From proposition~\ref{comp-fcd-r}.\end{proof}
\begin{defn}
\index{funcoid!restricted identity}Let $A$ be a set, 
$\mathcal{A}\in\mathscr{F}(A)$.
The \emph{restricted identity funcoid
\[
\id_{\mathcal{A}}^{\mathsf{FCD}}=(A;A;\mathcal{A}\sqcap;\mathcal{A}\sqcap).
\]
}\end{defn}
\begin{prop}
The restricted identity funcoid is a funcoid.\end{prop}
\begin{proof}
We need to prove that 
$(\mathcal{A}\sqcap\mathcal{X})\sqcap\mathcal{Y}\ne\bot^{\mathscr{F}(A)}
\Leftrightarrow(\mathcal{A}\sqcap\mathcal{Y})\sqcap\mathcal{X}\ne\bot^{\mathscr{
F}(A)}$
what is obvious.\end{proof}
\begin{obvious}
~
\begin{enumerate}
\item $(1_{A}^{\mathsf{FCD}})^{-1}=1_{A}^{\mathsf{FCD}}$;
\item 
$(\id_{\mathcal{A}}^{\mathsf{FCD}})^{-1}=\id_{\mathcal{A}}^{\mathsf{FCD}}$.
\end{enumerate}
\end{obvious}

\begin{obvious}
For every $\mathcal{X},\mathcal{Y}\in\mathscr{F}(A)$
\begin{enumerate}
\item 
$\mathcal{X}\suprel{1_{A}^{\mathsf{FCD}}}\mathcal{Y}\Leftrightarrow\mathcal{X}
\sqcap\mathcal{Y}\ne\bot^{\mathscr{F}(A)}$;
\item 
$\mathcal{X}\suprel{\id_{\mathcal{A}}^{\mathsf{FCD}}}\mathcal{Y}
\Leftrightarrow\mathcal{A}\sqcap\mathcal{X}\sqcap\mathcal{Y}\ne\bot^{\mathscr{F}
(A)}$.
\end{enumerate}
\end{obvious}
\begin{defn}
\index{restricting!funcoid}I will define \emph{restricting} of a
funcoid $f$ to a filter $\mathcal{A}\in\mathscr{F}(\Src f)$ by the
formula
\[
f|_{\mathcal{A}}=f\circ\id_{\mathcal{A}}^{\mathsf{FCD}}.
\]

\end{defn}

\begin{defn}
\index{image!of funcoid}\emph{Image} of a funcoid $f$ will be defined
by the formula $\im f=\supfun f\top^{\mathscr{F}(\Src f)}$.

\index{domain!of funcoid}\emph{Domain} of a funcoid $f$ is defined
by the formula $\dom f=\im f^{-1}$.\end{defn}
\begin{obvious}
For every morphism $f\in\mathbf{Rel}(A;B)$ for sets~$A$ and~$B$
\begin{enumerate}
\item $\im\uparrow^{\mathsf{FCD}}f=\uparrow\im f$;
\item $\dom\uparrow^{\mathsf{FCD}}f=\uparrow\dom f$.
\end{enumerate}
\end{obvious}
\begin{prop}
$\supfun f\mathcal{X}=\supfun f(\mathcal{X}\sqcap\dom f)$ for every
funcoid $f$, $\mathcal{X}\in\mathscr{F}(\Src f)$.\end{prop}
\begin{proof}
For every $\mathcal{Y}\in\mathscr{F}(\Dst f)$ we have
\begin{align*}
\mathcal{Y}\sqcap\supfun f(\mathcal{X}\sqcap\dom f)\ne\bot^{\mathscr{F}(\Dst f)} 
& \Leftrightarrow\\
\mathcal{X}\sqcap\dom f\sqcap\supfun{f^{-1}}\mathcal{Y}\ne\bot^{\mathscr{F}(\Src 
f)} & \Leftrightarrow\\
\mathcal{X}\sqcap\im 
f^{-1}\sqcap\supfun{f^{-1}}\mathcal{Y}\ne\bot^{\mathscr{F}(\Src f)} & 
\Leftrightarrow\\
\mathcal{X}\sqcap\supfun{f^{-1}}\mathcal{Y}\ne\bot^{\mathscr{F}(\Src f)} & 
\Leftrightarrow\\
\mathcal{Y}\sqcap\supfun f\mathcal{X}\ne\bot^{\mathscr{F}(\Dst f)}.
\end{align*}


Thus $\supfun f(\mathcal{X}\sqcap\dom f)=\supfun f\mathcal{X}$ because
the lattice of filters is separable.\end{proof}
\begin{prop}
$\supfun f\mathcal{X}=\im(f|_{\mathcal{X}})$ for every funcoid $f$,
$\mathcal{X}\in\mathscr{F}(\Src f)$.\end{prop}
\begin{proof}
~
\begin{align*}
\im(f|_{\mathcal{X}}) & =\\
\supfun{f\circ\id_{\mathcal{X}}^{\mathsf{FCD}}}\top^{\mathscr{F}(\Src f)} & =\\
\supfun f\supfun{\id_{\mathcal{X}}^{\mathsf{FCD}}}\top^{\mathscr{F}(\Src f)} & 
=\\
\supfun f(\mathcal{X}\sqcap\top^{\mathscr{F}(\Src f)}) & =\\
\supfun f\mathcal{X}.
\end{align*}
\end{proof}
\begin{prop}
$\mathcal{X}\sqcap\dom f\ne\bot^{\mathscr{F}(\Src f)}\Leftrightarrow\supfun 
f\mathcal{X}\ne\bot^{\mathscr{F}(\Dst f)}$
for every funcoid $f$ and $\mathcal{X}\in\mathscr{F}(\Src f)$.\end{prop}
\begin{proof}
~
\begin{align*}
\mathcal{X}\sqcap\dom f\ne\bot^{\mathscr{F}(\Src f)} & \Leftrightarrow\\
\mathcal{X}\sqcap\supfun{f^{-1}}\top^{\mathscr{F}(\Dst 
f)}\ne\bot^{\mathscr{F}(\Src f)} & \Leftrightarrow\\
\top^{\mathscr{F}(\Dst f)}\sqcap\supfun f\mathcal{X}\ne\bot^{\mathscr{F}(\Dst 
f)} & \Leftrightarrow\\
\supfun f\mathcal{X}\ne\bot^{\mathscr{F}(\Dst f)}.
\end{align*}
\end{proof}
\begin{cor}
$\dom f=\bigsqcup\setcond{a\in\atoms^{\mathscr{F}(\Src f)}}{\supfun 
fa\ne\bot^{\mathscr{F}(\Dst f)}}$.\end{cor}
\begin{proof}
This follows from the fact that $\mathscr{F}(\Src f)$ is an atomistic
lattice.\end{proof}
\begin{prop}
$\dom(f|_{\mathcal{A}})=\mathcal{A}\sqcap\dom f$ for every funcoid
$f$ and $\mathcal{A}\in\mathscr{F}(\Src f)$.\end{prop}
\begin{proof}
~

\begin{align*}
\dom(f|_{\mathcal{A}}) & =\\
\im(\id_{\mathcal{A}}^{\mathsf{FCD}}\circ f^{-1}) & =\\
\supfun{\id_{\mathcal{A}}^{\mathsf{FCD}}}\supfun{f^{-1}}\top^{\mathscr{F}(\Dst 
f)} & =\\
\mathcal{A}\sqcap\supfun{f^{-1}}\top^{\mathscr{F}(\Dst f)} & =\\
\mathcal{A}\sqcap\dom f.
\end{align*}
\end{proof}
\begin{thm}
$\im f=\bigsqcap^{\mathscr{F}}\rsupfun{\im}\up f$ and $\dom 
f=\bigsqcap^{\mathscr{F}}\rsupfun{\dom}\up f$
for every funcoid $f$.\end{thm}
\begin{proof}
~
\begin{align*}
\im f & =\\
\supfun f\top^{\mathscr{F}(\Src f)} & =\\
\bigsqcap_{F\in\up f}^{\mathscr{F}}\supfun F\top^{\mathscr{F}(\Src f)} & =\\
\bigsqcap_{F\in\up f}^{\mathscr{F}}\im F & =\\
\bigsqcap^{\mathscr{F}}\rsupfun{\im}\up f.
\end{align*}


The second formula follows from symmetry.\end{proof}
\begin{prop}
For every composable funcoids $f$, $g$:
\begin{enumerate}
\item \label{im-gf}If $\im f\sqsupseteq\dom g$ then $\im(g\circ f)=\im g$.
\item \label{dom-gf}If $\im f\sqsubseteq\dom g$ then $\dom(g\circ f)=\dom f$.
\end{enumerate}
\end{prop}
\begin{proof}
~
\begin{widedisorder}
\item [{\ref{im-gf}}] ~
\begin{align*}
\im(g\circ f) & =\\
\supfun{g\circ f}\top^{\mathscr{F}(\Src f)} & =\\
\supfun g\supfun f\top^{\mathscr{F}(\Src f)} & =\\
\supfun g\im f & =\\
\supfun g(\im f\sqcap\dom g) & =\\
\supfun g\dom g & =\\
\supfun g\top^{\mathscr{F}(\Src g)} & =\\
\im g.
\end{align*}

\item [{\ref{dom-gf}}] $\dom(g\circ f)=\im(f^{-1}\circ g^{-1})$ what
by proved above is equal to $\im f^{-1}$ that is $\dom f$.
\end{widedisorder}
\end{proof}
\section{Categories of funcoids}

I will define two categories, the \emph{category of funcoids} and
the \emph{category of funcoid triples}.

\index{category!of funcoids}The \emph{category of funcoids} is defined
as follows:
\begin{itemize}
\item Objects are small sets.
\item The set of morphisms from a set $A$ to a set $B$ is $\mathsf{FCD}(A;B)$.
\item The composition is the composition of funcoids.
\item Identity morphism for a set is the identity funcoid for that set.
\end{itemize}
To show it is really a category is trivial.

\index{category!of funcoid triples}The \emph{category of funcoid
triples} is defined as follows:
\begin{itemize}
\item Objects are filters on small sets.
\item The morphisms from a filter $\mathcal{A}$ to a filter $\mathcal{B}$
are triples $(\mathcal{A};\mathcal{B};f)$ where 
$f\in\mathsf{FCD}(\Base(\mathcal{A});\Base(\mathcal{B}))$
and $\dom f\sqsubseteq\mathcal{A}\land\im f\sqsubseteq\mathcal{B}$.
\item The composition is defined by the formula 
$(\mathcal{B};\mathcal{C};g)\circ(\mathcal{A};\mathcal{B};f)=(\mathcal{A}
;\mathcal{C};g\circ f)$.
\item Identity morphism for a filter $\mathcal{A}$ is 
$\id_{\mathcal{A}}^{\mathsf{FCD}}$.
\end{itemize}
To prove that it is really a category is trivial.
\begin{prop}
$\uparrow^{\mathsf{FCD}}$ is a functor from $\mathbf{Rel}$ to 
$\mathsf{FCD}$.\end{prop}
\begin{proof}
$\uparrow^{\mathsf{FCD}}(g\circ 
f)=\uparrow^{\mathsf{FCD}}g\circ\uparrow^{\mathsf{FCD}}f$
was proved above. 
$\uparrow^{\mathsf{FCD}}1_{A}^{\mathbf{Rel}}=1_{A}^{\mathsf{FCD}}$
is obvious.
\end{proof}

\section{Specifying funcoids by functions or relations on atomic filters}
\begin{thm}
For every funcoid $f$ and 
$\mathcal{X}\in\mathscr{F}(\Src f)$,
$\mathcal{Y}\in\mathscr{F}(\Dst f)$
\begin{enumerate}
\item \label{f-at-f}$\supfun f\mathcal{X}=\bigsqcup\rsupfun{\supfun 
f}\atoms\mathcal{X}$;
\item \label{f-at-r}$\mathcal{X}\suprel f\mathcal{Y}\Leftrightarrow\exists 
x\in\atoms\mathcal{X},y\in\atoms\mathcal{Y}:x\suprel fy$.
\end{enumerate}
\end{thm}
\begin{proof}
~
\begin{widedisorder}
\item [{\ref{f-at-f}}] ~
\begin{align*}
\mathcal{Y}\sqcap\supfun f\mathcal{X}\ne\bot^{\mathscr{F}(\Dst f)} & 
\Leftrightarrow\\
\mathcal{X}\sqcap\supfun{f^{-1}}\mathcal{Y}\ne\bot^{\mathscr{F}(\Src f)} & 
\Leftrightarrow\\
\exists 
x\in\atoms\mathcal{X}:x\sqcap\supfun{f^{-1}}\mathcal{Y}\ne\bot^{\mathscr{F}(\Src 
f)} & \Leftrightarrow\\
\exists x\in\atoms\mathcal{X}:\mathcal{Y}\sqcap\supfun 
fx\ne\bot^{\mathscr{F}(\Dst f)}.
\end{align*}



$\corestar\supfun f\mathcal{X}=\bigsqcup\rsupfun{\corestar}\rsupfun{\supfun 
f}\atoms\mathcal{X}=\corestar\bigsqcup\rsupfun{\supfun f}\atoms\mathcal{X}$.
So $\supfun f\mathcal{X}=\bigsqcup\rsupfun{\supfun f}\atoms\mathcal{X}$
by corollary~\ref{d-inj}.

\item [{\ref{f-at-r}}] If $\mathcal{X}\suprel f\mathcal{Y}$, then 
$\mathcal{Y}\sqcap\supfun f\mathcal{X}\ne\bot^{\mathscr{F}(\Dst f)}$,
consequently there exists $y\in\atoms\mathcal{Y}$ such that $y\sqcap\supfun 
f\mathcal{X}\ne\bot^{\mathscr{F}(\Dst f)}$,
$\mathcal{X}\suprel fy$. Repeating this second time we get that there
exists $x\in\atoms\mathcal{X}$ such that $x\suprel fy$. From this
it follows
\[
\exists x\in\atoms\mathcal{X},y\in\atoms\mathcal{Y}:x\suprel fy.
\]



The reverse is obvious.

\end{widedisorder}
\end{proof}
\begin{cor}
~Let $f$ be a funcoid.
\begin{itemize}
\item The value of $f$ can be restored from the value of $\supfun 
f|_{\atoms^{\mathscr{F}(\Src f)}}$.
\item The value of $f$ can be restored from the value of $\mathord{\suprel 
f}|_{\atoms^{\mathscr{F}(\Src f)}\times\atoms^{\mathscr{F}(\Dst f)}}$.
\end{itemize}
\end{cor}
\begin{thm}
\label{cont-fcd-on-atoms}Let $A$ and $B$ be sets.
\begin{enumerate}
\item \label{at-restr-f}A function 
$\alpha\in\mathscr{F}(B)^{\atoms^{\mathscr{F}(A)}}$
such that (for every $a\in\atoms^{\mathscr{F}(A)}$)
\begin{equation}
\alpha 
a\sqsubseteq\bigsqcap\rsupfun{\bigsqcup\circ\rsupfun{\alpha}
\circ\atoms\circ\uparrow}\up a\label{at-func-cond}
\end{equation}
can be continued to the function $\supfun f$ for a unique 
$f\in\mathsf{FCD}(A;B)$;
\begin{equation}
\supfun 
f\mathcal{X}=\bigsqcup\rsupfun{\alpha}\atoms\mathcal{X}\label{at-func-eq}
\end{equation}
for every $\mathcal{X}\in\mathscr{F}(A)$.

\item \label{at-restr-r}A relation 
$\delta\in\subsets(\atoms^{\mathscr{F}(A)}\times\atoms^{\mathscr{F}(B)})$
such that (for every $a\in\atoms^{\mathscr{F}(A)}$, 
$b\in\atoms^{\mathscr{F}(B)}$)
\begin{equation}
\forall X\in\up a,Y\in\up b\exists x\in\atoms\uparrow X,y\in\atoms\uparrow 
Y:x\mathrel\delta y\Rightarrow a\mathrel\delta b\label{at-rel-cond}
\end{equation}
can be continued to the relation $\suprel f$ for a unique 
$f\in\mathsf{FCD}(A;B)$;
\begin{equation}
\mathcal{X}\suprel f\mathcal{Y}\Leftrightarrow\exists 
x\in\atoms\mathcal{X},y\in\atoms\mathcal{Y}:x\mathrel\delta y\label{at-rel-eq}
\end{equation}
for every $\mathcal{X}\in\mathscr{F}(A)$, $\mathcal{Y}\in\mathscr{F}(B)$.

\end{enumerate}
\end{thm}
\begin{proof}
Existence of no more than one such funcoids and formulas (\ref{at-func-eq})
and (\ref{at-rel-eq}) follow from the previous theorem.
\begin{widedisorder}
\item [{\ref{at-restr-f}}] Consider the function 
$\alpha'\in\mathscr{F}(B)^{\mathscr{T}A}$
defined by the formula (for every $X\in\mathscr{T}A$)
\[
\alpha'X=\bigsqcup\rsupfun{\alpha}\atoms\uparrow X.
\]



Obviously $\alpha'\bot^{\mathscr{T}A}=\bot^{\mathscr{F}(B)}$. For
every $I,J\in\mathscr{T}A$
\begin{align*}
\alpha'(I\sqcup J) & =\\
\bigsqcup\rsupfun{\alpha}\atoms\uparrow(I\sqcup J) & =\\
\bigsqcup\rsupfun{\alpha}(\atoms\uparrow\cup\atoms\uparrow J) & =\\
\bigsqcup(\rsupfun{\alpha}\atoms\uparrow I\cup\rsupfun{\alpha}\atoms\uparrow J) 
& =\\
\bigsqcup\rsupfun{\alpha}\atoms\uparrow 
I\sqcup\bigsqcup\rsupfun{\alpha}\atoms\uparrow J & =\\
\alpha'I\sqcup\alpha'J.
\end{align*}



Let continue $\alpha'$ till a funcoid $f$ (by the theorem \ref{fcd-as-cont}):
$\supfun f\mathcal{X}=\bigsqcap\rsupfun{\alpha'}\up\mathcal{X}$.


Let's prove the reverse of (\ref{at-func-cond}):
\begin{align*}
\bigsqcap\rsupfun{\bigsqcup\circ\rsupfun{\alpha}\circ\atoms\circ\uparrow}\up a & 
=\\
\bigsqcap\rsupfun{\bigsqcup\circ\rsupfun{\alpha}}\rsupfun{\atoms}\rsupfun{
\uparrow}\up a & \sqsubseteq\\
\bigsqcap\rsupfun{\bigsqcup\circ\rsupfun{\alpha}}\{\{a\}\} & =\\
\bigsqcap\left\{ \left(\bigsqcup\circ\rsupfun{\alpha}\right)\{a\}\right\}  & =\\
\bigsqcap\left\{ \bigsqcup\rsupfun{\alpha}\{a\}\right\}  & =\\
\bigsqcap\left\{ \bigsqcup\{\alpha a\}\right\}  & =\\
\bigsqcap\left\{ \{\alpha a\}\right\}  & =\\
\alpha a.
\end{align*}



Finally,
\[
\alpha 
a=\bigsqcap\rsupfun{\bigsqcup\circ\rsupfun{\alpha}\circ\atoms\circ\uparrow}\up 
a=\bigsqcap\rsupfun{\alpha'}\up a=\supfun fa,
\]



so $\supfun f$ is a continuation of $\alpha$.

\item [{\ref{at-restr-r}}] Consider the relation 
$\delta'\in\subsets(\mathscr{T}A\times\mathscr{T}B)$
defined by the formula (for every $X\in\mathscr{T}A$, $Y\in\mathscr{T}B$)
\[
X\mathrel{\delta'}Y\Leftrightarrow\exists x\in\atoms\uparrow 
X,y\in\atoms\uparrow Y:x\mathrel\delta y.
\]



Obviously $\lnot(X\mathrel{\delta'}\bot^{\mathscr{F}(B)})$ and 
$\lnot(\bot^{\mathscr{F}(A)}\mathrel{\delta'}Y)$.


For suitable $I$ and $J$ we have:
\begin{align*}
I\sqcup J\mathrel{\delta'}Y & \Leftrightarrow\\
\exists x\in\atoms\uparrow(I\sqcup J),y\in\atoms\uparrow Y:x\mathrel\delta y & 
\Leftrightarrow\\
\exists x\in\atoms\uparrow I\cup\atoms\uparrow J,y\in\atoms\uparrow 
Y:x\mathrel\delta y & \Leftrightarrow\\
\exists x\in\atoms\uparrow I,y\in\atoms\uparrow Y:x\mathrel\delta y\lor\exists 
x\in\atoms\uparrow J,y\in\atoms\uparrow Y:x\mathrel\delta y & \Leftrightarrow\\
I\mathrel{\delta'}Y\lor J\mathrel{\delta'}Y;
\end{align*}
similarly $X\mathrel{\delta'}I\sqcup J\Leftrightarrow X\mathrel{\delta'}I\lor 
X\mathrel{\delta'}J$
for suitable $I$ and $J$. Let's continue $\delta'$ till a funcoid
$f$ (by the theorem \ref{fcd-as-cont}):
\[
\mathcal{X}\suprel f\mathcal{Y}\Leftrightarrow\forall 
X\in\up\mathcal{X},Y\in\up\mathcal{Y}:X\mathrel{\delta'}Y.
\]



The reverse of (\ref{at-rel-cond}) implication is trivial, so
\[
\forall X\in\up a,Y\in\up b\exists x\in\atoms\uparrow X,y\in\atoms\uparrow 
Y:x\mathrel\delta y\Leftrightarrow a\mathrel\delta b.
\]



Also
\begin{align*}
\forall X\in\up a,Y\in\up b\exists x\in\atoms\uparrow X,y\in\atoms\uparrow 
Y:x\mathrel\delta y & \Leftrightarrow\\
\forall X\in\up a,Y\in\up b:X\mathrel{\delta'}Y & \Leftrightarrow\\
a\suprel fb.
\end{align*}



So $a\mathrel\delta b\Leftrightarrow a\suprel fb$, that is $\suprel f$
is a continuation of $\delta$.

\end{widedisorder}
\end{proof}
One of uses of the previous theorem is the proof of the following
theorem:
\begin{thm}
\label{fcd-intrs-atom}If $A$ and $B$ are sets, $R\in\subsets\mathsf{FCD}(A;B)$,
$x\in\atoms^{\mathscr{F}(A)}$, $y\in\atoms^{\mathscr{F}(B)}$, then
\begin{enumerate}
\item \label{meet-f-at}$\supfun{\bigsqcap R}x=\bigsqcap_{f\in R}\supfun fx$;
\item \label{meet-r-at}$x\suprel{\bigsqcap R}y\Leftrightarrow\forall f\in 
R:x\suprel fy$.
\end{enumerate}
\end{thm}
\begin{proof}
~
\begin{widedisorder}
\item [{\ref{meet-r-at}}] Let denote $x\mathrel\delta y\Leftrightarrow\forall 
f\in R:x\suprel fy$.
For every $a\in\atoms^{\mathscr{F}(A)}$, $b\in\atoms^{\mathscr{F}(B)}$
\begin{align*}
\forall X\in\up a,Y\in\up b\exists x\in\atoms\uparrow X,y\in\atoms\uparrow 
Y:x\mathrel\delta y & \Rightarrow\\
\forall f\in R,X\in\up a,Y\in\up b\exists x\in\atoms\uparrow X,y\in\atoms\uparrow 
Y:x\suprel fy & \Rightarrow\\
\forall f\in R,X\in\up a,Y\in\up b:X\rsuprel fY & \Rightarrow\\
\forall f\in R:a\suprel fb & \Leftrightarrow\\
a\mathrel\delta b.
\end{align*}



So by theorem~\ref{cont-fcd-on-atoms}, $\delta$ can be continued
till $\suprel p$ for some funcoid $p\in\mathsf{FCD}(A;B)$.


For every funcoid $q\in\mathsf{FCD}(A;B)$ such that $\forall f\in R:q\sqsubseteq 
f$
we have 
\[
x\suprel qy\Rightarrow\forall f\in R:x\suprel fy\Leftrightarrow x\mathrel\delta 
y\Leftrightarrow x\suprel py,
\]



so $q\sqsubseteq p$. Consequently $p=\bigsqcap R$.


From this $x\suprel{\bigsqcap R}y\Leftrightarrow\forall f\in R:x\suprel fy$.

\item [{\ref{meet-f-at}}] From the former
\begin{align*}
y\in\atoms\supfun{\bigsqcap R}x & \Leftrightarrow\\
y\sqcap\supfun{\bigsqcap R}x\ne\bot^{\mathscr{F}(B)} & \Leftrightarrow\\
\forall f\in R:y\sqcap\supfun fx\ne\bot^{\mathscr{F}(B)} & \Leftrightarrow\\
y\in\bigsqcap\rsupfun{\atoms}\setcond{\supfun fx}{f\in R} & \Leftrightarrow\\
y\in\atoms\bigsqcap_{f\in R}\supfun fx
\end{align*}
for every $y\in\atoms^{\mathscr{F}(A)}$. From this it follows $\supfun{\bigsqcap 
R}x=\bigsqcap_{f\in R}\supfun fx$.
\end{widedisorder}
\end{proof}
\begin{thm}
$g\circ f=\bigsqcap\setcond{G\circ F}{F\in\up f.G\in\up g}$ for every
composable funcoids~$f$ and~$g$.\end{thm}
\begin{proof}
Let $x\in\atoms^{\mathscr{F}(\Src f)}$. Then
\begin{align*}
\supfun{g\circ f}x & =\\
\supfun g\supfun fx & =\text{ (theorem \ref{fcd-up-x})}\\
\bigsqcap_{G\in\up g}^{\mathscr{F}}\supfun G\supfun fx & =\text{ (theorem 
\ref{fcd-up-x})}\\
\bigsqcap_{G\in\up g}^{\mathscr{F}}\supfun G\bigsqcap_{F\in\up 
f}^{\mathscr{F}}\supfun Fx & =\text{ (theorem \ref{supfun-genbase})}\\
\bigsqcap_{G\in\up g}^{\mathscr{F}}\bigsqcap_{F\in\up f}^{\mathscr{F}}\supfun 
G\supfun Fx & =\\
\bigsqcap\setcond{\supfun G\supfun Fx}{F\in\up f,G\in\up g} & =\\
\bigsqcap\setcond{\supfun{G\circ F}x}{F\in\up f,G\in\up g} & =\text{ (theorem 
\ref{fcd-intrs-atom})}\\
\supfun{\bigsqcap\setcond{G\circ F}{F\in\up f.G\in\up g}}x.
\end{align*}


Thus $g\circ f=\bigsqcap\setcond{G\circ F}{F\in\up f.G\in\up g}$.\end{proof}
\begin{thm}
\label{fcd-cross}Let $A$, $B$, $C$ be sets, $f\in\mathsf{FCD}(A;B)$,
$g\in\mathsf{FCD}(B;C)$, $h\in\mathsf{FCD}(A;C)$. Then
\[
g\circ f\nasymp h\Leftrightarrow g\nasymp h\circ f^{-1}.
\]
\end{thm}
\begin{proof}
~
\begin{align*}
g\circ f\nasymp h & \Leftrightarrow\\
\exists a\in\atoms^{\mathscr{F}(A)},c\in\atoms^{\mathscr{F}(C)}:a\suprel{(g\circ 
f)\sqcap h}c & \Leftrightarrow\\
\exists a\in\atoms^{\mathscr{F}(A)},c\in\atoms^{\mathscr{F}(C)}:(a\suprel{g\circ 
f}c\land a\suprel hc) & \Leftrightarrow\\
\exists 
a\in\atoms^{\mathscr{F}(A)},b\in\atoms^{\mathscr{F}(B)},c\in\atoms^{\mathscr{F}
(C)}:(a\suprel fb\land b\suprel gc\land a\suprel hc) & \Leftrightarrow\\
\exists b\in\atoms^{\mathscr{F}(B)},c\in\atoms^{\mathscr{F}(C)}:(b\suprel 
gc\land b\suprel{h\circ f^{-1}}c) & \Leftrightarrow\\
\exists 
b\in\atoms^{\mathscr{F}(B)},c\in\atoms^{\mathscr{F}(C)}:b\suprel{g\sqcap(h\circ 
f^{-1})}c & \Leftrightarrow\\
g\nasymp h\circ f^{-1}.
\end{align*}

\end{proof}

\section{Funcoidal product of filters}

A generalization of Cartesian product of two sets is funcoidal product
of two filters:
\begin{defn}
\index{product!funcoidal}\emph{Funcoidal product} of filters $\mathcal{A}$
and $\mathcal{B}$ is such a funcoid 
$\mathcal{A}\times^{\mathsf{FCD}}\mathcal{B}\in\mathsf{FCD}(\Base(\mathcal{A}
);\Base(\mathcal{B}))$
that for every $\mathcal{X}\in\Base(\mathcal{A})$, 
$\mathcal{Y}\in\Base(\mathcal{B})$
\[
\mathcal{X}\suprel{\mathcal{A}\times^{\mathsf{FCD}}\mathcal{B}}\mathcal{Y}
\Leftrightarrow\mathcal{X}\nasymp\mathcal{A}\land\mathcal{Y}\nasymp\mathcal{B}.
\]
\end{defn}
\begin{prop}
$\mathcal{A}\times^{\mathsf{FCD}}\mathcal{B}$ is really a funcoid
and
\[
\supfun{\mathcal{A}\times^{\mathsf{FCD}}\mathcal{B}}\mathcal{X}=\begin{cases}
\mathcal{B} & \text{if }\mathcal{X}\nasymp\mathcal{A}\\
\bot^{\mathscr{F}(\Base(\mathcal{B}))} & \text{if }\mathcal{X}\asymp\mathcal{A}.
\end{cases}
\]
\end{prop}
\begin{proof}
Obvious.\end{proof}
\begin{obvious}
~
\begin{itemize}
\item $\uparrow^{\mathsf{FCD}(U;V)}(A\times B)=\uparrow^{U}A\times\uparrow^{V}B$
for sets $A\subseteq U$ and $B\subseteq V$.
\item $\uparrow^{\mathsf{FCD}}(A\times B)=\uparrow A\times\uparrow B$ for
typed sets~$A$ and~$B$.
\end{itemize}
\end{obvious}
\begin{prop}
$f\sqsubseteq\mathcal{A}\times^{\mathsf{FCD}}\mathcal{B}\Leftrightarrow\dom 
f\sqsubseteq\mathcal{A}\land\im f\sqsubseteq\mathcal{B}$
for every $f\in\mathsf{FCD}(A;B)$ and $\mathcal{A}\in\mathscr{F}(A)$,
$\mathcal{B}\in\mathscr{F}(B)$.\end{prop}
\begin{proof}
If $f\sqsubseteq\mathcal{A}\times^{\mathsf{FCD}}\mathcal{B}$ then
$\dom 
f\sqsubseteq\dom(\mathcal{A}\times^{\mathsf{FCD}}\mathcal{B})\sqsubseteq\mathcal
{A}$,
$\im 
f\sqsubseteq\im(\mathcal{A}\times^{\mathsf{FCD}}\mathcal{B})\sqsubseteq\mathcal{
B}$.
If $\dom f\sqsubseteq\mathcal{A}\land\im f\sqsubseteq\mathcal{B}$
then
\[
\forall\mathcal{X}\in\mathscr{F}(A),\mathcal{Y}\in\mathscr{F}(B):(\mathcal{X}
\suprel 
f\mathcal{Y}\Rightarrow\mathcal{X}\sqcap\mathcal{A}\ne\bot^{\mathscr{F}(A)}
\land\mathcal{Y}\sqcap\mathcal{B}\ne\bot^{\mathscr{F}(B)});
\]
consequently $f\sqsubseteq\mathcal{A}\times^{\mathsf{FCD}}\mathcal{B}$.
\end{proof}
The following theorem gives a formula for calculating an important
particular case of a meet on the lattice of funcoids:
\begin{thm}
$f\sqcap(\mathcal{A}\times^{\mathsf{FCD}}\mathcal{B})=\id_{\mathcal{B}}^{\mathsf
{FCD}}\circ f\circ\id_{\mathcal{A}}^{\mathsf{FCD}}$
for every funcoid $f$ and $\mathcal{A}\in\mathscr{F}(\Src f)$, 
$\mathcal{B}\in\mathscr{F}(\Dst f)$.\end{thm}
\begin{proof}
$h\eqdef\id_{\mathcal{B}}^{\mathsf{FCD}}\circ 
f\circ\id_{\mathcal{A}}^{\mathsf{FCD}}$.
For every $\mathcal{X}\in\mathscr{F}(\Src f)$
\[
\supfun h\mathcal{X}=\supfun{\id_{\mathcal{B}}^{\mathsf{FCD}}}\supfun 
f\supfun{\id_{\mathcal{A}}^{\mathsf{FCD}}}\mathcal{X}=\mathcal{B}\sqcap\supfun 
f(\mathcal{A}\sqcap\mathcal{X}).
\]


From this, as easy to show, $h\sqsubseteq f$ and 
$h\sqsubseteq\mathcal{A}\times^{\mathsf{FCD}}\mathcal{B}$.
If $g\sqsubseteq f\land g\sqsubseteq\mathcal{A}\times^{\mathsf{FCD}}\mathcal{B}$
for a $g\in\mathsf{FCD}(\Src f;\Dst f)$ then $\dom g\sqsubseteq\mathcal{A}$,
$\im g\sqsubseteq\mathcal{B}$,
\[
\supfun g\mathcal{X}=\mathcal{B}\sqcap\supfun 
g(\mathcal{A}\sqcap\mathcal{X})\sqsubseteq\mathcal{B}\sqcap\supfun 
f(\mathcal{A}\sqcap\mathcal{X})=\supfun{\id_{\mathcal{B}}^{\mathsf{FCD}}}\supfun 
f\supfun{\id_{\mathcal{A}}^{\mathsf{FCD}}}\mathcal{X}=\supfun h\mathcal{X},
\]
$g\sqsubseteq h$. So 
$h=f\sqcap(\mathcal{A}\times^{\mathsf{FCD}}\mathcal{B})$.\end{proof}
\begin{cor}
$f|_{\mathcal{A}}=f\sqcap(\mathcal{A}\times^{\mathsf{FCD}}\top^{\mathscr{F}(\Dst 
f)})$
for every funcoid $f$ and $\mathcal{A}\in\mathscr{F}(\Src f)$.\end{cor}
\begin{proof}
$f\sqcap(\mathcal{A}\times^{\mathsf{FCD}}\top^{\mathscr{F}(\Dst 
f)})=\id_{\top^{\mathscr{F}(\Dst f)}}^{\mathsf{FCD}}\circ 
f\circ\id_{\mathcal{A}}^{\mathsf{FCD}}=f\circ\id_{\mathcal{A}}^{\mathsf{FCD}}
=f|_{\mathcal{A}}$.\end{proof}
\begin{cor}
$f\nasymp\mathcal{A}\times^{\mathsf{FCD}}\mathcal{B}\Leftrightarrow\mathcal{A}
\suprel f\mathcal{B}$
for every funcoid $f$ and $\mathcal{A}\in\mathscr{F}(\Src f)$, 
$\mathcal{B}\in\mathscr{F}(\Dst f)$.\end{cor}
\begin{proof}
~
\begin{align*}
f\nasymp\mathcal{A}\times^{\mathsf{FCD}}\mathcal{B} & \Leftrightarrow\\
\rsupfun{f\sqcap(\mathcal{A}\times^{\mathsf{FCD}}\mathcal{B})}\top^{\mathscr{T}
(\Src f)}\ne\bot^{\mathscr{F}(\Dst f)} & \Leftrightarrow\\
\rsupfun{\id_{\mathcal{B}}^{\mathsf{FCD}}\circ 
f\circ\id_{\mathcal{A}}^{\mathsf{FCD}}}\top^{\mathscr{T}(\Src 
f)}\ne\bot^{\mathscr{F}(\Dst f)} & \Leftrightarrow\\
\supfun{\id_{\mathcal{B}}^{\mathsf{FCD}}}\supfun 
f\rsupfun{\id_{\mathcal{A}}^{\mathsf{FCD}}}\top^{\mathscr{F}(\Src 
f)}\ne\bot^{\mathscr{F}(\Dst f)} & \Leftrightarrow\\
\mathcal{B}\sqcap\supfun f(\mathcal{A}\sqcap\top^{\mathscr{F}(\Src 
f)})\ne\bot^{\mathscr{F}(\Dst f)} & \Leftrightarrow\\
\mathcal{B}\sqcap\supfun f\mathcal{A}\ne\bot^{\mathscr{F}(\Dst f)} & 
\Leftrightarrow\\
\mathcal{A}\suprel f\mathcal{B}.
\end{align*}
\end{proof}
\begin{cor}
Every filtrator of funcoids is star-separable.\end{cor}
\begin{proof}
The set of funcoidal products of principal filters is a separation
subset of the lattice of funcoids.\end{proof}
\begin{thm}
\label{meet-prod-fcd}Let $A$, $B$ be sets. If 
$S\in\subsets(\mathscr{F}(A)\times\mathscr{F}(B))$
then
\[
\bigsqcap_{(\mathcal{A};\mathcal{B})\in 
S}(\mathcal{A}\times^{\mathsf{FCD}}\mathcal{B})=\bigsqcap\dom 
S\times^{\mathsf{FCD}}\bigsqcap\im S.
\]
\end{thm}
\begin{proof}
If $x\in\atoms^{\mathscr{F}(A)}$ then by theorem \ref{fcd-intrs-atom}
\[
\supfun{\bigsqcap_{(\mathcal{A};\mathcal{B})\in 
S}(\mathcal{A}\times^{\mathsf{FCD}}\mathcal{B})}x=\bigsqcap_{(\mathcal{A}
;\mathcal{B})\in S}\supfun{\mathcal{A}\times^{\mathsf{FCD}}\mathcal{B}}x.
\]


If $x\nasymp\bigsqcap\dom S$ then
\begin{gather*}
\forall(\mathcal{A};\mathcal{B})\in 
S:(x\sqcap\mathcal{A}\ne\bot^{\mathscr{F}(A)}\land\supfun{\mathcal{A}\times^{
\mathsf{FCD}}\mathcal{B}}x=\mathcal{B});\\
\setcond{\supfun{\mathcal{A}\times^{\mathsf{FCD}}\mathcal{B}}x}{(\mathcal{A}
;\mathcal{B})\in S}=\im S;
\end{gather*}


if $x\asymp\bigsqcap\dom S$ then
\begin{gather*}
\exists(\mathcal{A};\mathcal{B})\in 
S:(x\sqcap\mathcal{A}=\bot^{\mathscr{F}(A)}\land\supfun{\mathcal{A}\times^{
\mathsf{FCD}}\mathcal{B}}x=\bot^{\mathscr{F}(B)});\\
\setcond{\supfun{\mathcal{A}\times^{\mathsf{FCD}}\mathcal{B}}x}{(\mathcal{A}
;\mathcal{B})\in S}\ni\bot^{\mathscr{F}(B)}.
\end{gather*}


So
\[
\supfun{\bigsqcap_{(\mathcal{A};\mathcal{B})\in 
S}(\mathcal{A}\times^{\mathsf{FCD}}\mathcal{B})}x=\begin{cases}
\bigsqcap\im S & \text{if }x\nasymp\bigsqcap\dom S\\
\bot^{\mathscr{F}(B)} & \text{if }x\asymp\bigsqcap\dom S.
\end{cases}
\]


From this the statement of the theorem follows.\end{proof}
\begin{cor}
For every $\mathcal{A}_{0},\mathcal{A}_{1}\in\mathscr{F}(A)$, 
$\mathcal{B}_{0},\mathcal{B}_{1}\in\mathscr{F}(B)$
(for every sets $A$,~$B$)
\[
(\mathcal{A}_{0}\times^{\mathsf{FCD}}\mathcal{B}_{0})\sqcap(\mathcal{A}_{1}
\times^{\mathsf{FCD}}\mathcal{B}_{1})=(\mathcal{A}_{0}\sqcap\mathcal{A}_{1}
)\times^{\mathsf{FCD}}(\mathcal{B}_{0}\sqcap\mathcal{B}_{1}).
\]
\end{cor}
\begin{proof}
$(\mathcal{A}_{0}\times^{\mathsf{FCD}}\mathcal{B}_{0})\sqcap(\mathcal{A}_{1}
\times^{\mathsf{FCD}}\mathcal{B}_{1})=\bigsqcap\{\mathcal{A}\times^{\mathsf{FCD
}}\mathcal{B}_{0},\mathcal{A}_{1}\times^{\mathsf{FCD}}\mathcal{B}_{1}\}$
what is by the last theorem equal to 
$(\mathcal{A}_{0}\sqcap\mathcal{A}_{1})\times^{\mathsf{FCD}}(\mathcal{B}_{0}
\sqcap\mathcal{B}_{1})$.\end{proof}
\begin{thm}
If $A$, $B$ are sets and $\mathcal{A}\in\mathscr{F}(A)$ then 
$\mathcal{A}\times^{\mathsf{FCD}}$
is a complete homomorphism from the lattice $\mathscr{F}(B)$ to the
lattice $\mathsf{FCD}(A;B)$, if also $\mathcal{A}\ne\bot^{\mathscr{F}(A)}$
then it is an order embedding.\end{thm}
\begin{proof}
Let $S\in\subsets\mathscr{F}(B)$, $X\in\mathscr{T}A$, 
$x\in\atoms^{\mathscr{F}(A)}$.
\begin{align*}
\rsupfun{\bigsqcup\rsupfun{\mathcal{A}\times^{\mathsf{FCD}}}S}X & =\\
\bigsqcup_{\mathcal{B}\in 
S}\rsupfun{\mathcal{A}\times^{\mathsf{FCD}}\mathcal{B}}X & =\\
\begin{cases}
\bigsqcup S & \text{if }X\in\corestar\mathcal{A}\\
\bot^{\mathscr{F}(B)} & \text{if }X\notin\corestar\mathcal{A}
\end{cases} & =\\
\rsupfun{\mathcal{A}\times^{\mathsf{FCD}}\bigsqcup S}X;\\
\supfun{\bigsqcap\rsupfun{\mathcal{A}\times^{\mathsf{FCD}}}S}x & =\\
\bigsqcap_{\mathcal{B}\in 
S}\supfun{\mathcal{A}\times^{\mathsf{FCD}}\mathcal{B}}x & =\\
\begin{cases}
\bigsqcap S & \text{if }x\nasymp\mathcal{A}\\
\bot^{\mathscr{F}(B)} & \text{if }x\asymp\mathcal{A}.
\end{cases}
\end{align*}


Thus 
$\bigsqcup\rsupfun{\mathcal{A}\times^{\mathsf{FCD}}}S=\mathcal{A}\times^{\mathsf
{FCD}}\bigsqcup S$
and 
$\bigsqcap\rsupfun{\mathcal{A}\times^{\mathsf{FCD}}}S=\mathcal{A}\times^{\mathsf
{FCD}}\bigsqcap S$.

If $\mathcal{A}\ne\bot^{\mathscr{F}(A)}$ then obviously
$\mathcal{A}\times^{\mathsf{FCD}}\mathcal{X}\sqsubseteq\mathcal{A}\times^{\mathsf{FCD}}\mathcal{Y} \Leftrightarrow
\mathcal{X}\sqsubseteq\mathcal{Y}$.
\end{proof}
The following proposition states that cutting a rectangle of atomic
width from a funcoid always produces a rectangular (representable
as a funcoidal product of filters) funcoid (of atomic width).
\begin{prop}
If $f$ is a funcoid and $a$ is an atomic filter on $\Src f$ then
\[
f|_{a}=a\times^{\mathsf{FCD}}\supfun fa.
\]
\end{prop}
\begin{proof}
Let $\mathcal{X}\in\mathscr{F}(\Src f)$.
\[
\mathcal{X}\nasymp a\Rightarrow\supfun{f|_{a}}\mathcal{X}=\supfun 
fa,\quad\mathcal{X}\asymp 
a\Rightarrow\supfun{f|_{a}}\mathcal{X}=\bot^{\mathscr{F}(\Dst f)}.
\]

\end{proof}

\begin{lem}
$\mylamdba{\mathcal{B}}{\mathscr{F}(B)}{\top^{\mathscr{F}}\times^{\mathsf{FCD}}
\mathcal{B}}$
is an upper adjoint of $\mylamdba f{\mathsf{FCD}(A;B)}{\im f}$ (for
every sets $A$, $B$).\end{lem}
\begin{proof}
We need to prove $\im f\sqsubseteq\mathcal{B}\Leftrightarrow 
f\sqsubseteq\top^{\mathscr{F}}\times^{\mathsf{FCD}}\mathcal{B}$
what is obvious.\end{proof}
\begin{cor}
\label{fcd-dom-join}Image and domain of funcoids preserve joins.\end{cor}
\begin{proof}
By properties of Galois connections and duality.\end{proof}
\begin{prop}
$f\sqsubseteq\mathcal{A}\times^{\mathsf{FCD}}\mathcal{B}\Leftrightarrow\dom 
f\sqsubseteq\mathcal{A}\wedge\im f\sqsubseteq\mathcal{B}$
for every funcoid $f$ and filters $\mathcal{A}\in\mathfrak{F}(\Src f)$,
$\mathcal{B}\in\mathfrak{F}(\Dst f)$.\end{prop}
\begin{proof}
$f\sqsubseteq\mathcal{A}\times^{\mathsf{FCD}}\mathcal{B}\Rightarrow\dom 
f\sqsubseteq\mathcal{A}$
because 
$\dom(\mathcal{A}\times^{\mathsf{FCD}}\mathcal{B})\sqsubseteq\mathcal{A}$.

Let now $\dom f\sqsubseteq\mathcal{A}\wedge\im f\sqsubseteq\mathcal{B}$.
Then $\supfun f\mathcal{X}\neq\bot\Rightarrow\mathcal{X}\nasymp\mathcal{A}$
that is $f\sqsubseteq\mathcal{A}\times^{\mathsf{FCD}}\top$. Similarly
$f\sqsubseteq\top\times^{\mathsf{FCD}}\mathcal{B}$. Thus 
$f\sqsubseteq\mathcal{A}\times^{\mathsf{FCD}}\mathcal{B}$.
\end{proof}

\section{Atomic funcoids}
\begin{thm}
An $f\in\mathsf{FCD}(A;B)$ is an atom of the lattice $\mathsf{FCD}(A;B)$
(for some sets $A$, $B$) iff it is a funcoidal product of two atomic
filter objects.\end{thm}
\begin{proof}
~
\begin{description}
\item [{$\Rightarrow$}] Let $f\in\mathsf{FCD}(A;B)$ be an atom of the
lattice $\mathsf{FCD}(A;B)$. Let's get elements $a\in\atoms\dom f$
and $b\in\atoms\supfun fa$. Then for every $\mathcal{X}\in\mathscr{F}(A)$
\[
\mathcal{X}\asymp 
a\Rightarrow\supfun{a\times^{\mathsf{FCD}}b}\mathcal{X}=\bot^{\mathscr{F}(B)}
\sqsubseteq\supfun f\mathcal{X},\quad\mathcal{X}\nasymp 
a\Rightarrow\supfun{a\times^{\mathsf{FCD}}b}\mathcal{X}=b\sqsubseteq\supfun 
f\mathcal{X}.
\]



So $a\times^{\mathsf{FCD}}b\sqsubseteq f$; because $f$ is atomic we
have $f=a\times^{\mathsf{FCD}}b$.

\item [{$\Leftarrow$}] Let $a\in\atoms^{\mathscr{F}(A)}$, 
$b\in\atoms^{\mathscr{F}(B)}$,
$f\in\mathsf{FCD}(A;B)$. If $b\asymp\supfun fa$ then $\lnot(a\suprel fb)$,
$f\asymp a\times^{\mathsf{FCD}}b$; if $b\sqsubseteq\supfun fa$ then
$\forall\mathcal{X}\in\mathscr{F}(A):(\mathcal{X}\nasymp a\Rightarrow\supfun 
f\mathcal{X}\sqsupseteq b)$,
$f\sqsupseteq a\times^{\mathsf{FCD}}b$. Consequently $f\asymp 
a\times^{\mathsf{FCD}}b\lor f\sqsupseteq a\times^{\mathsf{FCD}}b$;
that is $a\times^{\mathsf{FCD}}b$ is an atom.
\end{description}
\end{proof}
\begin{thm}
The lattice $\mathsf{FCD}(A;B)$ is atomic (for every fixed sets $A$, $B$).\end{thm}
\begin{proof}
Let $f$ be a non-empty funcoid from $A$ to $B$. Then $\dom 
f\ne\bot^{\mathscr{F}(A)}$,
thus by theorem~\ref{filt-atomic} there exists $a\in\atoms\dom f$.
So $\supfun fa\ne\bot^{\mathscr{F}(B)}$ thus it exists $b\in\atoms\supfun fa$.
Finally the atomic funcoid $a\times^{\mathsf{FCD}}b\sqsubseteq f$.\end{proof}
\begin{thm}
The lattice $\mathsf{FCD}(A;B)$ is separable (for every fixed sets $A$,
$B$).\end{thm}
\begin{proof}
Let $f,g\in\mathsf{FCD}(A;B)$, $f\sqsubset g$. Then there exists
$a\in\atoms^{\mathscr{F}(A)}$ such that $\supfun fa\sqsubset\supfun ga$.
So because the lattice $\mathscr{F}(B)$ is atomically separable,
there exists $b\in\atoms^{\mathscr{F}(B)}$ such that $\supfun fa\sqcap 
b=\bot^{\mathscr{F}(B)}$
and $b\sqsubseteq\supfun ga$. For every $x\in\atoms^{\mathscr{F}(A)}$
\begin{gather*}
\supfun fa\sqcap\supfun{a\times^{\mathsf{FCD}}b}a=\supfun fa\sqcap 
b=\bot^{\mathscr{F}(B)},\\
x\ne a\Rightarrow\supfun fx\sqcap\supfun{a\times^{\mathsf{FCD}}b}x=\supfun 
fx\sqcap\bot^{\mathscr{F}(B)}=\bot^{\mathscr{F}(B)}.
\end{gather*}


Thus $\supfun fx\sqcap\supfun{a\times^{\mathsf{FCD}}b}x=\bot^{\mathscr{F}(B)}$
and consequently $f\asymp a\times^{\mathsf{FCD}}b$.
\begin{gather*}
\supfun{a\times^{\mathsf{FCD}}b}a=b\sqsubseteq\supfun ga,\\
x\ne 
a\Rightarrow\supfun{a\times^{\mathsf{FCD}}b}x=\bot^{\mathscr{F}(B)}
\sqsubseteq\supfun gx.
\end{gather*}


Thus $\supfun{a\times^{\mathsf{FCD}}b}x\sqsubseteq\supfun gx$ and
consequently $a\times^{\mathsf{FCD}}b\sqsubseteq g$.

So the lattice $\mathsf{FCD}(A;B)$ is separable by theorem 
\ref{msl-sep-conds}.\end{proof}
\begin{cor}
\label{fcd-is-sep}The lattice $\mathsf{FCD}(A;B)$ is:
\begin{enumerate}
\item separable;
\item strongly separable;
\item atomically separable;
\item conforming to Wallman's disjunction property.
\end{enumerate}
\end{cor}
\begin{proof}
By theorem \ref{sep-conds}.\end{proof}
\begin{rem}
For more ways to characterize (atomic) separability of the lattice
of funcoids see subsections ``Separation subsets and full stars''
and ``Atomically separable lattices''.\end{rem}
\begin{cor}
The lattice $\mathsf{FCD}(A;B)$ is an atomistic lattice.\end{cor}
\begin{proof}
By theorem~\ref{amstc-sep}.\end{proof}
\begin{prop}
$\atoms(f\sqcup g)=\atoms f\cup\atoms g$ for every funcoids 
$f,g\in\mathsf{FCD}(A;B)$
(for every sets $A$, $B$).\end{prop}
\begin{proof}
$a\times^{\mathsf{FCD}}b\nasymp f\sqcup g\Leftrightarrow a\suprel{f\sqcup 
g}b\Leftrightarrow a\suprel fb\lor a\suprel gb\Leftrightarrow 
a\times^{\mathsf{FCD}}b\nasymp f\lor a\times^{\mathsf{FCD}}b\nasymp g$
for every atomic filters $a$ and $b$.\end{proof}
\begin{thm}
The set of funcoids between sets~$A$ and~$B$ is a co-frame.\end{thm}
\begin{proof}
Theorems \ref{fcd-as-cont} and \ref{frame-main}.\end{proof}
\begin{rem}
The above proof does not use axiom of choice (unlike the below proof).
\end{rem}
See also an older proof of the set of funcoids being co-brouwerian:
\begin{thm}
For every $f,g,h\in\mathsf{FCD}(A;B)$, $R\in\subsets\mathsf{FCD}(A;B)$
(for every sets $A$ and $B$)
\begin{enumerate}
\item \label{fcd-dist-j}$f\sqcap(g\sqcup h)=(f\sqcap g)\sqcup(f\sqcap h)$;
\item \label{fcd-dist-m}$f\sqcup\bigsqcap R=\bigsqcap\rsupfun{f\sqcup}R$.
\end{enumerate}
\end{thm}
\begin{proof}
We will take into account that the lattice of funcoids is an atomistic
lattice.
\begin{widedisorder}
\item [{\ref{fcd-dist-j}}] ~ 
\begin{align*}
\atoms(f\sqcap(g\sqcup h)) & =\\
\atoms f\cap\atoms(g\sqcup h) & =\\
\atoms f\cap(\atoms g\cup\atoms h) & =\\
(\atoms f\cap\atoms g)\cup(\atoms f\cap\atoms h) & =\\
\atoms(f\sqcap g)\cup\atoms(f\sqcap h) & =\\
\atoms((f\sqcap g)\sqcup(f\sqcap h)).
\end{align*}

\item [{\ref{fcd-dist-m}}] ~
\begin{align*}
\atoms\left(f\sqcup\bigsqcap R\right) & =\\
\atoms f\cup\atoms\bigsqcap R & =\\
\atoms f\cup\bigcap\rsupfun{\atoms}R & =\\
\bigcap\rsupfun{(\atoms f)\cup}\rsupfun{\atoms}R & =\text{ (use the following 
equality)}\\
\bigcap\rsupfun{\atoms}\rsupfun{f\sqcup}R & =\\
\atoms\bigsqcap\rsupfun{f\sqcup}R.\\
\rsupfun{(\atoms f)\cup}\rsupfun{\atoms}R & =\\
\setcond{(\atoms f)\cup A}{A\in\rsupfun{\atoms}R} & =\\
\setcond{(\atoms f)\cup A}{\exists C\in R:A=\atoms C} & =\\
\setcond{(\atoms f)\cup(\atoms C)}{C\in R} & =\\
\setcond{\atoms(f\sqcup C)}{C\in R} & =\\
\setcond{\atoms B}{\exists C\in R:B=f\sqcup C} & =\\
\setcond{\atoms B}{B\in\rsupfun{f\sqcup}C} & =\\
\rsupfun{\atoms}\rsupfun{f\sqcup}R.
\end{align*}

\end{widedisorder}
\end{proof}
The next proposition is one more (among the theorem \ref{fcd-atom-middle})
generalization for funcoids of composition of relations.
\begin{prop}
For every composable funcoids $f$, $g$
\begin{multline*}
\atoms(g\circ f)=\\
\setcond{x\times^{\mathsf{FCD}}z}{\begin{array}{l}
x\in\atoms^{\mathscr{F}(\Src f)},z\in\atoms^{\mathscr{F}(\Dst g)},\\
\exists y\in\atoms^{\mathscr{F}(\Dst f)}:(x\times^{\mathsf{FCD}}y\in\atoms 
f\land y\times^{\mathsf{FCD}}z\in\atoms g)
\end{array}}.
\end{multline*}
\end{prop}
\begin{proof}
Using the theorem \ref{fcd-atom-middle}, 
\[
x\times^{\mathsf{FCD}}z\nasymp g\circ f\Leftrightarrow x\suprel{g\circ 
f}z\Leftrightarrow\exists y\in\atoms^{\mathscr{F}(\Dst 
f)}:(x\times^{\mathsf{FCD}}y\nasymp f\land y\times^{\mathsf{FCD}}z\nasymp g).
\]
\end{proof}
\begin{cor}
$g\circ f=\bigsqcup\setcond{G\circ F}{F\in\atoms f,G\in\atoms g}$
for every composable funcoids $f$, $g$.\end{cor}
\begin{thm}
Let $f$ be a funcoid.
\begin{enumerate}
\item \label{ffilt-r}$\mathcal{X}\suprel f\mathcal{Y}\Leftrightarrow\exists 
F\in\atoms f:\mathcal{X}\suprel F\mathcal{Y}$
for every $\mathcal{X}\in\mathscr{F}(\Src f)$, $\mathcal{Y}\in\mathscr{F}(\Dst 
f)$;
\item \label{ffilt-f}$\supfun f\mathcal{X}=\bigsqcup_{F\in\atoms f}\supfun 
F\mathcal{X}$
for every $\mathcal{X}\in\mathscr{F}(\Src f)$.
\end{enumerate}
\end{thm}
\begin{proof}
~
\begin{widedisorder}
\item [{\ref{ffilt-r}}] ~
\begin{align*}
\exists F\in\atoms f:\mathcal{X}\suprel F\mathcal{Y} & \Leftrightarrow\\
\exists a\in\atoms^{\mathscr{F}(\Src f)},b\in\atoms^{\mathscr{F}(\Dst 
f)}:(a\times^{\mathsf{FCD}}b\nasymp 
f\land\mathcal{X}\suprel{a\times^{\mathsf{FCD}}b}\mathcal{Y}) & 
\Leftrightarrow\\
\exists a\in\atoms^{\mathscr{F}(\Src f)},b\in\atoms^{\mathscr{F}(\Dst 
f)}:(a\times^{\mathsf{FCD}}b\nasymp f\land 
a\times^{\mathsf{FCD}}b\nasymp\mathcal{X}\times^{\mathsf{FCD}}\mathcal{Y}) & 
\Leftrightarrow\\
\exists F\in\atoms f:(F\nasymp f\land 
F\nasymp\mathcal{X}\times^{\mathsf{FCD}}\mathcal{Y}) & \Leftrightarrow\\
f\nasymp\mathcal{X}\times^{\mathsf{FCD}}\mathcal{Y} & \Leftrightarrow\\
\mathcal{X}\suprel f\mathcal{Y}.
\end{align*}

\item [{\ref{ffilt-f}}] Let $\mathcal{Y}\in\mathscr{F}(\Dst f)$. Suppose
$\mathcal{Y}\nasymp\supfun f\mathcal{X}$. Then $\mathcal{X}\suprel 
f\mathcal{Y}$;
$\exists F\in\atoms f:\mathcal{X}\suprel F\mathcal{Y}$; $\exists F\in\atoms 
f:\mathcal{Y}\nasymp\supfun F\mathcal{X}$;
$\mathcal{Y}\nasymp\bigsqcup_{F\in\atoms f}\supfun F\mathcal{X}$.
So $\supfun f\mathcal{X}\sqsubseteq\bigsqcup_{F\in\atoms f}\supfun 
F\mathcal{X}$.
The contrary $\supfun f\mathcal{X}\sqsupseteq\bigsqcup_{F\in\atoms f}\supfun 
F\mathcal{X}$
is obvious.
\end{widedisorder}
\end{proof}

\section{Complete funcoids}
\begin{defn}
\index{funcoid!co-complete}I will call \emph{co-complete} such a
funcoid $f$ that $\rsupfun fX$ is a principal filter for every 
$X\in\mathscr{T}(\Src f)$.\end{defn}
\begin{obvious}
Funcoid $f$ is co-complete iff $\supfun f\mathcal{X}\in\mathfrak{P}(\Dst f)$
for every $\mathcal{X}\in\mathfrak{P}(\Src f)$.\end{obvious}
\begin{defn}
\index{generalized closure}I will call \emph{generalized closure}
such a function $\alpha\in(\mathscr{T}B)^{\mathscr{T}A}$ (for some
sets $A$, $B$) that
\begin{enumerate}
\item $\alpha\bot^{\mathscr{T}A}=\bot^{\mathscr{T}B}$;
\item $\forall I,J\in\mathscr{T}A:\alpha(I\sqcup J)=\alpha I\sqcup\alpha J$.
\end{enumerate}
\end{defn}
\begin{obvious}
A funcoid $f$ is co-complete iff $\rsupfun f=\uparrow\circ\alpha$
for a generalized closure $\alpha$.\end{obvious}
\begin{rem}
Thus funcoids can be considered as a generalization of generalized
closures. A topological space in Kuratowski sense is the same as reflexive
and transitive generalized closure. So topological spaces can be considered
as a special case of funcoids.\end{rem}
\begin{defn}
\index{funcoid!complete}I will call a \emph{complete funcoid} a funcoid
whose reverse is co-complete.\end{defn}
\begin{thm}
The following conditions are equivalent for every funcoid $f$:
\begin{enumerate}
\item \label{cfcd:main}funcoid $f$ is complete;
\item \label{cfcd:r-filt}$\forall S\in\subsets\mathscr{F}(\Src 
f),J\in\mathscr{T}(\Dst f):\left(\bigsqcup S\suprel 
fJ\Leftrightarrow\exists\mathcal{I}\in S:\mathcal{I}\suprel fJ\right)$;
\item \label{cfcd:r-set}$\forall S\in\subsets\mathscr{T}(\Src f),J\in\mathscr{T}(\Dst 
f):\left(\bigsqcup S\rsuprel fJ\Leftrightarrow\exists I\in S:I\rsuprel 
fJ\right)$;
\item \label{cfcd:f-filt}$\forall S\in\subsets\mathscr{F}(\Src f):\supfun 
f\bigsqcup S=\bigsqcup\rsupfun{\supfun f}S$;
\item \label{cfcd:f-set}$\forall S\in\subsets\mathscr{T}(\Src f):\rsupfun 
f\bigsqcup S=\bigsqcup\rsupfun{\rsupfun f}S$;
\item \label{cfcd:sing}$\forall A\in\mathscr{T}(\Src f):\rsupfun 
fA=\bigsqcup_{a\in\atoms A}\rsupfun fa$.
\end{enumerate}
\end{thm}
\begin{proof}
~
\begin{description}
\item [{\ref{cfcd:r-set}$\Rightarrow$\ref{cfcd:main}}] For every 
$S\in\subsets\mathscr{T}(\Src f)$,
$J\in\mathscr{T}(\Dst f)$
\[
\bigsqcup S\sqcap\rsupfun{f^{-1}}J\ne\bot^{\mathscr{F}(\Src 
f)}\Leftrightarrow\exists I\in 
S:I\sqcap\rsupfun{f^{-1}}J\ne\bot^{\mathscr{F}(\Src f)},
\]
consequently by theorem~\ref{crit1} we have that $\rsupfun{f^{-1}}J$
is a principal filter.
\item [{\ref{cfcd:main}$\Rightarrow$\ref{cfcd:r-filt}}] For every 
$S\in\subsets\mathscr{F}(\Src f)$,
$J\in\mathscr{T}(\Dst f)$ we have that $\rsupfun{f^{-1}}J$ is a
principal filter, consequently
\[
\bigsqcup S\sqcap\rsupfun{f^{-1}}J\ne\bot^{\mathscr{F}(\Src 
f)}\Leftrightarrow\exists\mathcal{I}\in 
S:\mathcal{I}\sqcap\rsupfun{f^{-1}}J\ne\bot^{\mathscr{F}(\Src f)}.
\]
From this follows \ref{cfcd:r-filt}.
\item [{\ref{cfcd:sing}$\Rightarrow$\ref{cfcd:f-set}}] ~
\begin{align*}
\rsupfun f\bigsqcup S & =\\
\bigsqcup_{a\in\atoms\bigsqcup S}\rsupfun fa & =\\
\bigsqcup\bigcup_{A\in S}\setcond{\rsupfun fa}{a\in\atoms A} & =\\
\bigsqcup_{A\in S}\bigsqcup_{a\in\atoms A}\rsupfun fa & =\\
\bigsqcup_{A\in S}\rsupfun fA & =\\
\bigsqcup\rsupfun{\rsupfun f}S.
\end{align*}

\item [{\ref{cfcd:r-filt}$\Rightarrow$\ref{cfcd:f-filt}}] Using 
theorem~\ref{crit1},
\begin{align*}
J\nasymp\supfun f\bigsqcup S & \Leftrightarrow\\
\bigsqcup S\suprel fJ & \Leftrightarrow\\
\exists\mathcal{I}\in S:\mathcal{I}\suprel fJ & \Leftrightarrow\\
\exists\mathcal{I}\in S:J\nasymp\supfun f\mathcal{I} & \Leftrightarrow\\
J\nasymp\bigsqcup\rsupfun{\supfun f}S.
\end{align*}

\item 
[{\ref{cfcd:r-filt}$\Rightarrow$\ref{cfcd:r-set},~\ref{cfcd:f-filt}
$\Rightarrow$\ref{cfcd:f-set},~\ref{cfcd:f-set}$\Rightarrow$\ref{cfcd:r-set},
~\ref{cfcd:f-set}$\Rightarrow$\ref{cfcd:sing}}] Obvious.
\end{description}
\end{proof}
The following proposition shows that complete funcoids are a direct
generalization of pretopological spaces.
\begin{prop}
To specify a complete funcoid $f$ it is enough to specify $\rsupfun f$
on one-element sets, values of $\rsupfun f$ on one element sets can
be specified arbitrarily.\end{prop}
\begin{proof}
From the above theorem is clear that knowing $\rsupfun f$ on one-element
sets $\rsupfun f$ can be found on every set and then the value of
$\supfun f$ can be inferred for every filter.

Choosing arbitrarily the values of $\rsupfun f$ on one-element sets
we can define a complete funcoid the following way: $\rsupfun 
fX=\bigsqcup_{\alpha\in\atoms X}\rsupfun f\alpha$
for every $X\in\mathscr{T}(\Src f)$. Obviously it is really a complete
funcoid.\end{proof}
\begin{thm}
A funcoid is principal iff it is both complete and co-complete.\end{thm}
\begin{proof}
~
\begin{description}
\item [{$\Rightarrow$}] Obvious.
\item [{$\Leftarrow$}] Let $f$ be both a complete and co-complete funcoid.
Consider the relation $g$ defined by that $\uparrow\rsupfun g\alpha=\rsupfun 
f\alpha$
for one-element sets~$\alpha$ ($g$ is correctly defined because
$f$ corresponds to a generalized closure). Because $f$ is a complete
funcoid $f$ is the funcoid corresponding to $g$.
\end{description}
\end{proof}
\begin{thm}
\label{fcd-join-compl}If $R\in\subsets\mathsf{FCD}(A;B)$ is a set
of (co-)complete funcoids then $\bigsqcup R$ is a (co-)complete funcoid
(for every sets $A$ and $B$).\end{thm}
\begin{proof}
It is enough to prove for co-complete funcoids. Let 
$R\in\subsets\mathsf{FCD}(A;B)$
be a set of co-complete funcoids. Then for every $X\in\mathscr{T}(\Src f)$
\[
\rsupfun{\bigsqcup R}X=\bigsqcup_{f\in R}\rsupfun fX
\]
is a principal filter (used theorem \ref{fcd-join-sets}).\end{proof}
\begin{cor}
\label{fcd-compl-join}If $R$ is a set of binary relations between
sets $A$ and $B$ then 
$\bigsqcup\rsupfun{\uparrow^{\mathsf{FCD}(A;B)}}R=\uparrow^{\mathsf{FCD}(A;B)}
\bigcup R$.\end{cor}
\begin{proof}
From two last theorems.\end{proof}
\begin{lem}
\label{fcd-rep}Every funcoid is representable as meet (on the lattice
of funcoids) of binary relations of the form 
$X\times Y\sqcup\overline{X}\times\top^{\mathscr{T}(B)}$ (where $X$, $Y$ are typed sets).\end{lem}
\begin{proof}
Let $f\in\mathsf{FCD}(A;B)$, $X\in\mathscr{T}A$, $Y\in\up\supfun fX$,
$g(X;Y)\eqdef X\times Y\sqcup\overline{X}\times\top^{\mathscr{T}(B)}$.
Then $g(X;Y)=X\times^{\mathsf{FCD}}Y\sqcup\overline{X}\times^{\mathsf{FCD}}\top^{\mathscr{F}(B)}$.
For every $K\in\mathscr{T}A$
\begin{multline*}
\rsupfun{g(X;Y)}K=\rsupfun{X\times^{\mathsf{FCD}}Y}K\sqcup\rsupfun{\overline{X}
\times^{\mathsf{FCD}}\top^{\mathscr{F}(B)}}K=\\
\left(\begin{cases}
\bot^{\mathscr{F}(B)} & \text{if }K=\bot^{\mathscr{T}A}\\
Y & \text{if }\bot^{\mathscr{T}A}\ne K\sqsubseteq X\\
\top^{\mathscr{F}(B)} & \text{if }K\nsqsubseteq X
\end{cases}\right)\sqsupseteq\rsupfun fK;
\end{multline*}
so $g(X;Y)\sqsupseteq f$. For every $X\in\mathscr{T}A$
\[
\bigsqcap_{Y\in\up\rsupfun fX}\rsupfun{g(X;Y)}X=\bigsqcap^{\mathscr{F}}_{Y\in\up\rsupfun 
fX}Y=\rsupfun fX;
\]
consequently
\[
\rsupfun{\bigsqcap\setcond{g(X;Y)}{X\in\mathscr{T}A,Y\in\up\rsupfun 
fX}}X\sqsubseteq\rsupfun fX
\]
that is
\[
\bigsqcap\setcond{g(X;Y)}{X\in\mathscr{T}A,Y\in\up\rsupfun fX}\sqsubseteq f
\]
 and finally
\[
f=\bigsqcap\setcond{g(X;Y)}{X\in\mathscr{T}A,Y\in\up\rsupfun fX}.
\]
\end{proof}
\begin{cor}
\label{fcd-filtered}Filtrators of funcoids are filtered.
\end{cor}

\begin{thm}\label{metcomp-thm}
~
\begin{enumerate}
\item \label{metcomp}$g$ is metacomplete if $g$ is a complete funcoid.
\item \label{cometcomp}$g$ is co-metacomplete if $g$ is a co-complete
funcoid.
\end{enumerate}
\end{thm}
\begin{proof}
~
\begin{widedisorder}
\item [{\ref{metcomp}}] Let $R$ be a set of funcoids from a set $A$ to a set
$B$ and $g$ be a funcoid from $B$ to some $C$. Then
\begin{align*}
\rsupfun{g\circ\bigsqcup R}X & =\\
\supfun g\rsupfun{\bigsqcup R}X & =\\
\supfun g\bigsqcup_{f\in R}\rsupfun fX & =\\
\bigsqcup_{f\in R}\supfun g\rsupfun fX & =\\
\bigsqcup_{f\in R}\rsupfun{g\circ f}X & =\\
\rsupfun{\bigsqcup_{f\in R}(g\circ f)}X & =\\
\rsupfun{\bigsqcup\rsupfun{g\circ}R}X
\end{align*}
for every typed set $X\in\mathscr{T}A$. So $g\circ\bigsqcup 
R=\bigsqcup\rsupfun{g\circ}R$.
\item [{\ref{cometcomp}}] By duality.
\end{widedisorder}
\end{proof}
\begin{conjecture}
$g$ is complete if $g$ is a metacomplete funcoid.
\end{conjecture}
I will denote $\mathsf{ComplFCD}$ and $\mathsf{CoComplFCD}$ the
sets of small complete and co-complete funcoids correspondingly. 
$\mathsf{ComplFCD}(A;B)$
are complete funcoids from $A$ to $B$ and likewise with 
$\mathsf{CoComplFCD}(A;B)$.
\begin{obvious}
$\mathsf{ComplFCD}$ and $\mathsf{CoComplFCD}$ are closed regarding
composition of funcoids.\end{obvious}
\begin{prop}
$\mathsf{ComplFCD}$ and $\mathsf{CoComplFCD}$ (with induced order)
are complete lattices.\end{prop}
\begin{proof}
It follows from theorem \ref{fcd-join-compl}.\end{proof}
\begin{thm}
Atoms of the lattice $\mathsf{ComplFCD}(A;B)$ are exactly funcoidal
products of the form $\uparrow^{A}\{\alpha\}\times^{\mathsf{FCD}}b$
where $\alpha\in A$ and $b$ is an ultrafilter on $B$.\end{thm}
\begin{proof}
First, it's easy to see that $\uparrow^{A}\{\alpha\}\times^{\mathsf{FCD}}b$
are elements of $\mathsf{ComplFCD}(A;B)$. Also $\bot^{\mathsf{FCD}(A;B)}$
is an element of $\mathsf{ComplFCD}(A;B)$.

$\uparrow^{A}\{\alpha\}\times^{\mathsf{FCD}}b$ are atoms of 
$\mathsf{ComplFCD}(A;B)$
because they are atoms of $\mathsf{FCD}(A;B)$.

It remains to prove that if $f$ is an atom of $\mathsf{ComplFCD}(A;B)$
then $f=\uparrow^{A}\{\alpha\}\times^{\mathsf{FCD}}b$ for some $\alpha\in A$
and an ultrafilter $b$ on $B$.

Suppose $f\in\mathsf{FCD}(A;B)$ is a non-empty complete funcoid.
Then there exists $\alpha\in A$ such that $\rsupfun 
f@\{\alpha\}\ne\bot^{\mathscr{F}(B)}$.
Thus $\uparrow^{A}\{\alpha\}\text{\texttimes}^{\mathsf{FCD}}b\sqsubseteq f$
for some ultrafilter $b$ on $B$. If $f$ is an atom then 
$f=\uparrow^{A}\{\alpha\}\text{\texttimes}^{\mathsf{FCD}}b$.\end{proof}
\begin{thm}
\label{complfcd-rep}$G\mapsto\bigsqcup_{\alpha\in 
A}(\uparrow^{A}\{\alpha\}\times^{\mathsf{FCD}}G(\alpha))$
is an order isomorphism from the set of functions $G\in\mathscr{F}(B)^{A}$
to the set $\mathsf{ComplFCD}(A;B)$.

The inverse isomorphism is described by the formula $G(\alpha)=\rsupfun 
f@\{\alpha\}$
where $f$ is a complete funcoid.\end{thm}
\begin{proof}
$\bigsqcup_{\alpha\in A}(\uparrow^{A}\{\alpha\}\times^{\mathsf{FCD}}G(\alpha))$
is complete because $G(\alpha)=\bigsqcup\atoms G(\alpha)$ and thus
\[
\bigsqcup_{\alpha\in 
A}(\uparrow^{A}\{\alpha\}\times^{\mathsf{FCD}}G(\alpha))=\bigsqcup\setcond{
\uparrow^{A}\{\alpha\}\times^{\mathsf{FCD}}b}{\alpha\in A,b\in\atoms G(\alpha)}
\]
is complete. So $G\mapsto\bigsqcup_{\alpha\in 
A}(\uparrow^{A}\{\alpha\}\times^{\mathsf{FCD}}G(\alpha))$
is a function from $G\in\mathscr{F}(B)^{A}$ to $\mathsf{ComplFCD}(A;B)$.

Let $f$ be complete. Then take
\[
G(\alpha)=\bigsqcup\setcond{b\in\atoms^{\mathscr{F}(\Dst 
f)}}{\uparrow^{A}\{\alpha\}\times^{\mathsf{FCD}}b\sqsubseteq f}
\]
and we have $f=\bigsqcup_{\alpha\in 
A}(\uparrow^{A}\{\alpha\}\times^{\mathsf{FCD}}G(\alpha))$
obviously. So $G\mapsto\bigsqcup_{\alpha\in 
A}(\uparrow^{A}\{\alpha\}\times^{\mathsf{FCD}}G(\alpha))$
is surjection onto $\mathsf{ComplFCD}(A;B)$.

Let now prove that it is an injection:

Let
\[
f=\bigsqcup_{\alpha\in 
A}(\uparrow^{A}\{\alpha\}\times^{\mathsf{FCD}}F(\alpha))=\bigsqcup_{\alpha\in 
A}(\uparrow^{A}\{\alpha\}\times^{\mathsf{FCD}}G(\alpha))
\]
for some $F,G\in\mathscr{F}(\Dst f)^{\Src f}$. We need to prove $F=G$.
Let $\beta\in\Src f$.
\[
\rsupfun f@\{\beta\}=\bigsqcup_{\alpha\in 
A}\rsupfun{\uparrow^{A}\{\alpha\}\times^{\mathsf{FCD}}F(\alpha)}@\{
\beta\}=F(\beta).
\]
Similarly $\rsupfun f@\{\beta\}=G(\beta)$. So $F(\beta)=G(\beta)$.

We have proved that it is a bijection. To show that it is monotone
is trivial.

Denote $f=\bigsqcup_{\alpha\in 
A}(\uparrow^{A}\{\alpha\}\times^{\mathsf{FCD}}G(\alpha))$.
Then
\begin{multline*}
\rsupfun f@\{\alpha'\}=\text{(because 
\ensuremath{\uparrow^{A}\{\alpha'\}} is principal)}=\\
\bigsqcup_{\alpha\in 
A}\supfun{\uparrow^{A}\{\alpha\}\times^{\mathsf{FCD}}G(\alpha)}@\{
\alpha'\}=\supfun{\uparrow^{A}\{\alpha'\}\times^{\mathsf{FCD}}G(\alpha')}
@\{\alpha'\}=G(\alpha').
\end{multline*}
\end{proof}
\begin{cor}
$G\mapsto\bigsqcup_{\alpha\in 
A}(G(\alpha)\times^{\mathsf{FCD}}\uparrow^{A}\{\alpha\})$
is an order isomorphism from the set of functions $G\in\mathscr{F}(B)^{A}$
to the set $\mathsf{CoComplFCD}(A;B)$.

The inverse isomorphism is described by the formula 
$G(\alpha)=\rsupfun{f^{-1}}@\{\alpha\}$
where $f$ is a co-complete funcoid.
\end{cor}

\begin{cor}
$\mathsf{ComplFCD}(A;B)$ and $\mathsf{CoComplFCD}(A;B)$ are co-frames.
\end{cor}

\section{Funcoids corresponding to pretopologies}

Let $\Delta$ be a pretopology on a set $U$ and $\cl$ the preclosure
corresponding to it (see theorem \ref{pretop-bij}).

Both induce a funcoid, I will show that these two funcoids are reverse
of each other:
\begin{thm}
Let $f$ be a complete funcoid defined by the formula $\rsupfun 
f@\{x\}=\Delta(x)$
for every $x\in U$, let $g$ be a co-complete funcoid defined by
the formula $\rsupfun gX=\uparrow^{U}\cl(\GR X)$ for every $X\in\mathscr{T}U$.
Then $g=f^{-1}$.\end{thm}
\begin{rem}
It is obvious that funcoids $f$ and $g$ exist.\end{rem}
\begin{proof}
For $X,Y\in\mathscr{T}U$ we have
\begin{align*}
X\rsuprel gY & \Leftrightarrow\\
\uparrow Y\nasymp\supfun g\uparrow X & \Leftrightarrow\\
Y\nasymp\cl(\GR X) & \Leftrightarrow\\
\exists y\in Y:\Delta(y)\nasymp\uparrow X & \Leftrightarrow\\
\exists y\in Y:\rsupfun f\uparrow^{U}\{y\}\nasymp\uparrow X & \Leftrightarrow\\
\text{(proposition \ref{b-f-back-distr} and properties of complete funcoids)}\\
\rsupfun fY\nasymp\uparrow X & \Leftrightarrow\\
Y\rsuprel fX.
\end{align*}


So $g=f^{-1}$.
\end{proof}

\section{Completion of funcoids}
\begin{thm}
$\Cor f=\Cor'f$ for an element $f$ of a filtrator of funcoids.\end{thm}
\begin{proof}
By theorem~\ref{cor-eq} and corollary~\ref{fcd-filtered}.\end{proof}
\begin{defn}
\index{completion!of funcoid}\emph{Completion} of a funcoid 
$f\in\mathsf{FCD}(A;B)$
is the complete funcoid $\Compl f\in\mathsf{FCD}(A;B)$ defined by
the formula $\rsupfun{\Compl f}@\{\alpha\}=\rsupfun 
f@\{\alpha\}$
for $\alpha\in\Src f$.
\end{defn}

\begin{defn}
\index{co-completion!of funcoid}\emph{Co-completion} of a funcoid
$f$ is defined by the formula
\[
\CoCompl f=(\Compl f^{-1})^{-1}.
\]
\end{defn}
\begin{obvious}
$\Compl f\sqsubseteq f$ and $\CoCompl f\sqsubseteq f$.\end{obvious}
\begin{prop}
The filtrator $(\mathsf{FCD}(A;B);\mathsf{ComplFCD}(A;B))$ is 
filtered.\end{prop}
\begin{proof}
Because the filtrator of funcoids is filtered.\end{proof}
\begin{thm}
$\Compl f=\Cor^{\mathsf{ComplFCD}(A;B)}f=\Cor'^{\mathsf{ComplFCD}(A;B)}f$
for every funcoid $f\in\mathsf{FCD}(A;B)$.\end{thm}
\begin{proof}
$\Cor^{\mathsf{ComplFCD}(A;B)}f=\Cor'^{\mathsf{ComplFCD}(A;B)}f$
using theorem \ref{cor-eq} since the filtrator 
$(\mathsf{FCD}(A;B);\mathsf{ComplFCD}(A;B))$
is filtered.

Let $g\in\up^{\mathsf{ComplFCD}(A;B)}f$. Then
$g\in\mathsf{ComplFCD}(A;B)$ and $g\sqsupseteq f$. Thus $g=\Compl 
g\sqsupseteq\Compl f$.

Thus $\forall g\in\up^{\mathsf{ComplFCD}(A;B)}f:g\sqsupseteq\Compl f$.

Let $\forall g\in\up^{\mathsf{ComplFCD}(A;B)}f:h\sqsubseteq g$
for some $h\in\mathsf{ComplFCD}(A;B)$.

Then $h\sqsubseteq\bigsqcap\up^{\mathsf{ComplFCD}(A;B)}f=f$
and consequently $h=\Compl h\sqsubseteq\Compl f$.

Thus
\[
\Compl 
f=\bigsqcap^{\mathsf{ComplFCD}(A;B)}\up^{\mathsf{ComplFCD}(A;B)}f=\Cor^{\mathsf{
ComplFCD}(A;B)}f.
\]
\end{proof}
\begin{thm}
$\rsupfun{\CoCompl f}X=\Cor\rsupfun fX$ for every funcoid $f$ and
typed set $X\in\mathscr{T}(\Src f)$.\end{thm}
\begin{proof}
$\CoCompl f\sqsubseteq f$ thus $\rsupfun{\CoCompl f}X\sqsubseteq\rsupfun fX$
but $\rsupfun{\CoCompl f}X$ is a principal filter thus $\rsupfun{\CoCompl 
f}X\sqsubseteq\Cor\rsupfun fX$.

Let $\alpha X=\Cor\rsupfun fX$. Then $\alpha\bot^{\mathscr{T}(\Src 
f)}=\bot^{\mathscr{F}(\Dst f)}$
and
\begin{multline*}
\alpha(X\sqcup Y)=\Cor\rsupfun f(X\sqcup Y)=\Cor(\rsupfun fX\sqcup\rsupfun 
fY)=\\
\Cor\rsupfun fX\sqcup\Cor\rsupfun fY=\alpha X\sqcup\alpha Y
\end{multline*}
(used theorem~\ref{dual-core-join}). Thus $\alpha$ can be continued
till $\supfun g$ for some funcoid $g$. This funcoid is co-complete.

Evidently $g$ is the greatest co-complete element of $\mathsf{FCD}(\Src f;\Dst 
f)$
which is lower than $f$.

Thus $g=\CoCompl f$ and $\Cor\rsupfun fX=\alpha X=\rsupfun gX=\rsupfun{\CoCompl 
f}X$.\end{proof}
\begin{thm}
$\mathsf{ComplFCD}(A;B)$ is an atomistic lattice.\end{thm}
\begin{proof}
Let $f\in\mathsf{ComplFCD}(A;B)$, $X\in\mathscr{T}(\Src f)$.
\[
\rsupfun fX=\bigsqcup_{x\in\atoms X}\rsupfun fx=\bigsqcup_{x\in\atoms 
X}\rsupfun{f|_{x}}x=\bigsqcup_{x\in\atoms X}\rsupfun{f|_{x}}X,
\]
thus $f=\bigsqcup_{x\in\atoms X}(f|_{x})$. It is trivial that every
$f|_{x}$ is a join of atoms of $\mathsf{ComplFCD}(A;B)$.\end{proof}
\begin{thm}
A funcoid is complete iff it is a join (on the lattice $\mathsf{FCD}(A;B)$)
of atomic complete funcoids.\end{thm}
\begin{proof}
It follows from the theorem \ref{fcd-join-compl} and the previous
theorem.\end{proof}
\begin{cor}
$\mathsf{ComplFCD}(A;B)$ is join-closed.\end{cor}
\begin{thm}
$\Compl\bigsqcup R=\bigsqcup\rsupfun{\Compl}R$ for every 
$R\in\subsets\mathsf{FCD}(A;B)$
(for every sets $A$, $B$).\end{thm}
\begin{proof}
For every typed set $X$
\begin{align*}
\rsupfun{\Compl\bigsqcup R}X & =\\
\bigsqcup_{x\in\atoms X}\rsupfun{\bigsqcup R}x & =\\
\bigsqcup_{x\in\atoms X}\bigsqcup_{f\in R}\rsupfun fx & =\\
\bigsqcup_{f\in R}\bigsqcup_{x\in\atoms X}\rsupfun fx & =\\
\bigsqcup_{f\in R}\rsupfun{\Compl f}X & =\\
\rsupfun{\bigsqcup\rsupfun{\Compl}R}X.
\end{align*}
\end{proof}
\begin{cor}
$\Compl$ is a lower adjoint.\end{cor}
\begin{conjecture}
$\Compl$ is not an upper adjoint (in general).\end{conjecture}
\begin{prop}
$\Compl f=\bigsqcup_{\alpha\in\Src f}(f|_{\uparrow\{\alpha\}})$
for every funcoid $f$.\end{prop}
\begin{proof}
Let denote $R$ the right part of the equality to prove.

$\rsupfun R@\{\beta\}=\bigsqcup_{\alpha\in\Src 
f}\rsupfun{f|_{\uparrow\{\alpha\}}}@\{\beta\}=\rsupfun 
f@\{\beta\}$
for every $\beta\in\Src f$ and $R$ is complete as a join of complete
funcoids.

Thus $R$ is the completion of $f$.\end{proof}
\begin{conjecture}
$\Compl f=f\psetminus(\Omega\times^{\mathsf{FCD}}\mho)$.
\end{conjecture}
This conjecture may be proved by considerations similar to these in
the section ``Fr�chet filter''.
\begin{lem}
Co-completion of a complete funcoid is complete.\end{lem}
\begin{proof}
Let $f$ be a complete funcoid.
\begin{multline*}
\rsupfun{\CoCompl f}X=\Cor\rsupfun fX=\Cor\bigsqcup_{x\in\atoms X}\rsupfun fx=\\
\bigsqcup_{x\in\atoms X}\Cor\rsupfun fx=\bigsqcup_{x\in\atoms 
X}\rsupfun{\CoCompl f}x
\end{multline*}
for every set typed $X\in\mathscr{T}(\Src f)$. Thus $\CoCompl f$
is complete.\end{proof}
\begin{thm}
$\Compl\CoCompl f=\CoCompl\Compl f=\Cor f$ for every funcoid $f$.\end{thm}
\begin{proof}
$\Compl\CoCompl f$ is co-complete since (used the lemma) $\CoCompl f$
is co-complete. Thus $\Compl\CoCompl f$ is a principal funcoid. $\CoCompl f$
is the greatest co-complete funcoid under $f$ and $\Compl\CoCompl f$
is the greatest complete funcoid under $\CoCompl f$. So $\Compl\CoCompl f$
is greater than any principal funcoid under $\CoCompl f$ which is
greater than any principal funcoid under $f$. Thus $\Compl\CoCompl f$
is the greatest principal funcoid under $f$. Thus $\Compl\CoCompl f=\Cor f$.
Similarly $\CoCompl\Compl f=\Cor f$.
\end{proof}

\subsection{More on completion of funcoids}
\begin{prop}
For every composable funcoids $f$ and $g$
\begin{enumerate}
\item \label{compl-ge}$\Compl(g\circ f)\sqsupseteq\Compl g\circ\Compl f$;
\item \label{cocompl-ge}$\CoCompl(g\circ f)\sqsupseteq\CoCompl g\circ\CoCompl 
f$.
\end{enumerate}
\end{prop}
\begin{proof}
~
\begin{disorder}
\item [{\ref{compl-ge}}] $\Compl g\circ\Compl f=\Compl(\Compl g\circ\Compl 
f)\sqsubseteq\Compl(g\circ f)$.
\item [{\ref{cocompl-ge}}] $\CoCompl g\circ\CoCompl f=\CoCompl(\CoCompl 
g\circ\CoCompl f)\sqsubseteq\CoCompl(g\circ f)$.
\end{disorder}
\end{proof}
\begin{prop}\label{comp-compl}
For every composable funcoids $f$ and $g$
\begin{enumerate}
\item \label{cocompl-eq}$\CoCompl(g\circ f)=(\CoCompl g)\circ f$ if $f$
is a co-complete funcoid.
\item \label{compl-eq}$\Compl(f\circ g)=f\circ\Compl g$ if $f$ is a complete
funcoid.
\end{enumerate}
\end{prop}
\begin{proof}
~
\begin{widedisorder}
\item [{\ref{cocompl-eq}}] For every $X\in\mathscr{T}(\Src f)$
\begin{align*}
\rsupfun{\CoCompl(g\circ f)}X & =\\
\Cor\rsupfun{g\circ f}X & =\\
\Cor\supfun g\rsupfun fX & =\\
\supfun{\CoCompl g}\rsupfun fX & =\\
\rsupfun{(\CoCompl g)\circ f}X.
\end{align*}

\item [{\ref{compl-eq}}] $(\CoCompl(g\circ f))^{-1}=f^{-1}\circ(\CoCompl 
g)^{-1}$;
$\Compl(g\circ f)^{-1}=f^{-1}\circ\Compl g^{-1}$; $\Compl(f^{-1}\circ 
g^{-1})=f^{-1}\circ\Compl g^{-1}$.
After variable replacement we get $\Compl(f\circ g)=f\circ\Compl g$
(after the replacement $f$ is a complete funcoid).
\end{widedisorder}
\end{proof}
\begin{cor}
~
For every composable funcoids $f$ and $g$
\begin{enumerate}
\item $\Compl f\circ\Compl g=\Compl(\Compl f\circ g)$.
\item $\CoCompl g\circ\CoCompl f=\CoCompl(g\circ\CoCompl f)$.
\end{enumerate}
\end{cor}
\begin{prop}
For every composable funcoids $f$ and $g$
\begin{enumerate}
\item \label{compl2-eq}$\Compl(g\circ f)=\Compl(g\circ(\Compl f))$;
\item \label{cocompl2-eq}$\CoCompl(g\circ f)=\CoCompl((\CoCompl g)\circ f)$.
\end{enumerate}
\end{prop}
\begin{proof}
~
\begin{widedisorder}
\item [{\ref{compl2-eq}}] ~
\begin{multline*}
\rsupfun{g\circ(\Compl f)}@\{x\}=\supfun g\rsupfun{\Compl 
f}@\{x\}=\\
\supfun g\rsupfun f@\{x\}=\rsupfun{g\circ f}@\{x\}.
\end{multline*}



Thus $\Compl(g\circ(\Compl f))=\Compl(g\circ f)$.

\item [{\ref{cocompl2-eq}}] $(\Compl(g\circ(\Compl f))^{-1}=(\Compl(g\circ 
f))^{-1}$;
$\CoCompl(g\circ(\Compl f))^{-1}=\CoCompl(g\circ f)^{-1}$; $\CoCompl((\Compl 
f)^{-1}\circ g^{-1})=\CoCompl(f^{-1}\circ g^{-1})$;
$\CoCompl((\CoCompl f^{-1})\circ g^{-1})=\CoCompl(f^{-1}\circ g^{-1})$.
After variable replacement $\CoCompl((\CoCompl g)\circ f)=\CoCompl(g\circ f)$.
\end{widedisorder}
\end{proof}
\begin{thm}
The filtrator of funcoids (from a given set~$A$ to a given set~$B$)
is with co-separable core.\end{thm}
\begin{proof}
Let $f,g\in\mathsf{FCD}(A;B)$ and $f\sqcup g=\top$. Then for every
$X\in\mathscr{T}A$ we have 
\begin{multline*}
\supfun f^{\ast}X\sqcup\supfun g^{\ast}X=\top\Leftrightarrow\Cor\supfun 
f^{\ast}X\sqcup\Cor\supfun g^{\ast}X=\top\Leftrightarrow\\
\rsupfun{\CoCompl f}X\sqcup\rsupfun{\CoCompl g}X=\top.
\end{multline*}
Thus $\rsupfun{\CoCompl f\sqcup\CoCompl g}X=\top$; 
\begin{equation}
f\sqcup g=\top\Rightarrow\CoCompl f\sqcup\CoCompl g=\top.\label{fcd-sep-d}
\end{equation}
Applying the dual of the formulas (\ref{fcd-sep-d}) to the formula
(\ref{fcd-sep-d}) we get: 
\[
f\sqcup g=\top\Rightarrow\Compl\CoCompl f\sqcup\Compl\CoCompl g=\top
\]
that is $f\sqcup g=\top\Rightarrow\Cor f\sqcup\Cor g=\top$. So 
$\mathsf{FCD}(A;B)$
is with co-separable core.\end{proof}
\begin{cor}
The filtrator of complete funcoids is also with co-separable core.
\end{cor}

\section{Monovalued and injective funcoids}

\index{funcoid!monovalued}\index{monovalued!funcoid}Following the
idea of definition of monovalued morphism let's call \emph{monovalued}
such a funcoid $f$ that $f\circ f^{-1}\sqsubseteq\id_{\im f}^{\mathsf{FCD}}$.

\index{funcoid!injective}\index{injective!funcoid}Similarly, I will
call a funcoid injective when $f^{-1}\circ f\sqsubseteq\id_{\dom 
f}^{\mathsf{FCD}}$.
\begin{obvious}
A funcoid $f$ is:
\begin{enumerate}
\item monovalued iff $f\circ f^{-1}\sqsubseteq1_{\Dst f}^{\mathsf{FCD}}$;
\item injective iff $f^{-1}\circ f\sqsubseteq1_{\Src f}^{\mathsf{FCD}}$.
\end{enumerate}
\end{obvious}
In other words, a funcoid is monovalued (injective) when it is a monovalued
(injective) morphism of the category of funcoids. Monovaluedness is
dual of injectivity.
\begin{obvious}
~
\begin{enumerate}
\item A morphism $(\mathcal{A};\mathcal{B};f)$ of the category of funcoid
triples is monovalued iff the funcoid $f$ is monovalued.
\item A morphism $(\mathcal{A};\mathcal{B};f)$ of the category of funcoid
triples is injective iff the funcoid $f$ is injective.
\end{enumerate}
\end{obvious}
\begin{thm}
The following statements are equivalent for a funcoid $f$:
\begin{enumerate}
\item \label{fcd-mv}$f$ is monovalued.
\item \label{fcd-mmv}It is metamonovalued.
\item \label{fcd-wmmv}It is weakly metamonovalued.
\item \label{mnv-atom}$\forall a\in\atoms^{\mathscr{F}(\Src f)}:\supfun 
fa\in\atoms^{\mathscr{F}(\Dst f)}\cup\{\bot^{\mathscr{F}(\Dst f)}\}$.
\item \label{mnv-flt}$\forall\mathcal{I},\mathcal{J}\in\mathscr{F}(\Dst 
f):\supfun{f^{-1}}(\mathcal{I}\sqcap\mathcal{J})=\supfun{f^{-1}}\mathcal{I}
\sqcap\supfun{f^{-1}}\mathcal{J}$.
\item \label{mnv-set}$\forall I,J\in\mathscr{T}(\Dst f):\rsupfun{f^{-1}}(I\sqcap 
J)=\rsupfun{f^{-1}}I\sqcap\rsupfun{f^{-1}}J$.
\end{enumerate}
\end{thm}
\begin{proof}
~
\begin{description}
\item [{\ref{mnv-atom}$\Rightarrow$\ref{mnv-flt}}] Let 
$a\in\atoms^{\mathscr{F}(\Src f)}$,
$\supfun fa=b$. Then because $b\in\atoms^{\mathscr{F}(\Dst 
f)}\cup\{\bot^{\mathscr{F}(\Dst f)}\}$
\begin{gather*}
(\mathcal{I}\sqcap\mathcal{J})\sqcap b\ne\bot^{\mathscr{F}(\Dst 
f)}\Leftrightarrow\mathcal{I}\sqcap b\ne\bot^{\mathscr{F}(\Dst 
f)}\land\mathcal{J}\sqcap b\ne\bot^{\mathscr{F}(\Dst f)};\\
a\suprel f\mathcal{I}\sqcap\mathcal{J}\Leftrightarrow a\suprel f\mathcal{I}\land 
a\suprel f\mathcal{J};\\
\mathcal{I}\sqcap\mathcal{J}\suprel{f^{-1}}a\Leftrightarrow\mathcal{I}\suprel{f^
{-1}}a\land\mathcal{J}\suprel{f^{-1}}a;\\
a\sqcap\supfun{f^{-1}}(\mathcal{I}\sqcap\mathcal{J})\ne\bot^{\mathscr{F}(\Src 
f)}\Leftrightarrow a\sqcap\supfun{f^{-1}}\mathcal{I}\ne\bot^{\mathscr{F}(\Src 
f)}\land a\sqcap\supfun{f^{-1}}\mathcal{J}\ne\bot^{\mathscr{F}(\Src f)};\\
\supfun{f^{-1}}(\mathcal{I}\sqcap\mathcal{J})=\supfun{f^{-1}}\mathcal{I}
\sqcap\supfun{f^{-1}}\mathcal{J}.
\end{gather*}

\item [{\ref{mnv-flt}$\Rightarrow$\ref{fcd-mv}}] 
$\supfun{f^{-1}}a\sqcap\supfun{f^{-1}}b=\supfun{f^{-1}}(a\sqcap 
b)=\supfun{f^{-1}}\bot^{\mathscr{F}(\Dst f)}=\bot^{\mathscr{F}(\Src f)}$
for every two distinct atomic filter objects $a$ and $b$ on $\Dst f$.
This is equivalent to $\lnot(\supfun{f^{-1}}a\suprel fb)$; $b\asymp\supfun 
f\supfun{f^{-1}}a$;
$b\asymp\supfun{f\circ f^{-1}}a$; $\lnot(a\suprel{f\circ f^{-1}}b)$.
So $a\suprel{f\circ f^{-1}}b\Rightarrow a=b$ for every ultrafilters
$a$ and $b$. This is possible only when $f\circ f^{-1}\sqsubseteq1_{\Dst 
f}^{\mathsf{FCD}}$.
\item [{\ref{mnv-set}$\Rightarrow$\ref{mnv-flt}}] ~
\begin{align*}
\supfun{f^{-1}}(\mathcal{I}\sqcap\mathcal{J}) & =\\
\bigsqcap\rsupfun{\rsupfun f}\up(\mathcal{I}\sqcap\mathcal{J}) & =\\
\bigsqcap\rsupfun{\rsupfun f}\setcond{I\sqcap 
J}{I\in\up\mathcal{I},J\in\up\mathcal{J}} & =\\
\bigsqcap\setcond{\rsupfun f(I\sqcap J)}{I\in\up\mathcal{I},J\in\up\mathcal{J}} 
& =\\
\bigsqcap\setcond{\rsupfun fI\sqcap\rsupfun 
fJ}{I\in\up\mathcal{I},J\in\up\mathcal{J}} & =\\
\bigsqcap\setcond{\rsupfun 
fI}{I\in\up\mathcal{I}}\sqcap\bigsqcap\setcond{\rsupfun fJ}{J\in\up\mathcal{J}} 
& =\\
\supfun{f^{-1}}\mathcal{I}\sqcap\supfun{f^{-1}}\mathcal{J}.
\end{align*}

\item [{\ref{mnv-flt}$\Rightarrow$\ref{mnv-set}}] Obvious.
\item [{$\lnot$\ref{mnv-atom}$\Rightarrow$$\lnot$\ref{fcd-mv}}] Suppose
$\supfun fa\notin\atoms^{\mathscr{F}(\Dst f)}\cup\{\bot^{\mathscr{F}(\Dst f)}\}$
for some $a\in\atoms^{\mathscr{F}(\Src f)}$. Then there exist two
atomic filters $p$ and $q$ on $\Dst f$ such that $p\ne q$ and
$\supfun fa\sqsupseteq p\land\supfun fa\sqsupseteq q$. Consequently
$p\nasymp\supfun fa$; $a\nasymp\supfun{f^{-1}}p$; 
$a\sqsubseteq\supfun{f^{-1}}p$;
$\supfun{f\circ f^{-1}}p=\supfun f\supfun{f^{-1}}p\sqsupseteq\supfun 
fa\sqsupseteq q$;
$\supfun{f\circ f^{-1}}p\nsqsubseteq p$ and $\supfun{f\circ 
f^{-1}}p\ne\bot^{\mathscr{F}(\Dst f)}$.
So it cannot be $f\circ f^{-1}\sqsubseteq1_{\Dst f}^{\mathsf{FCD}}$.
\item [{\ref{fcd-mmv}$\Rightarrow$\ref{fcd-wmmv}}] Obvious.
\item [{\ref{fcd-mv}$\Rightarrow$\ref{fcd-mmv}}] ~
\[
\supfun{\left(\bigsqcap G\right)\circ f}x=\supfun{\bigsqcap 
G}\supfun fx=\bigsqcap_{g\in G}\supfun g\supfun fx=\bigsqcap_{g\in 
G}\supfun{g\circ f}x=\supfun{\bigsqcap_{g\in G}(g\circ f)}x
\]
for every atomic filter object $x\in\atoms^{\mathscr{F}(\Src f)}$.
Thus $\left(\bigsqcap G\right)\circ f=\bigsqcap_{g\in G}(g\circ f)$.
\item [{\ref{fcd-wmmv}$\Rightarrow$\ref{fcd-mv}}] Take 
$g=a\times^{\mathsf{FCD}}y$
and $h=b\times^{\mathsf{FCD}}y$ for arbitrary atomic filter objects
$a\ne b$ and~$y$. We have $g\sqcap h=\bot$; thus $(g\circ f)\sqcap(h\circ 
f)=(g\sqcap h)\circ f=\bot$
and thus impossible $x\suprel fa\land x\suprel fb$ as otherwise $x\suprel{g\circ 
f}y$
and $x\suprel{h\circ f}y$ so $x\suprel{(g\circ f)\sqcap(h\circ f)}y$.
Thus $f$ is monovalued.
\end{description}
\end{proof}
\begin{cor}
A binary relation corresponds to a monovalued funcoid iff it is a
function.\end{cor}
\begin{proof}
Because $\forall I,J\in\subsets(\im f):\rsupfun{f^{-1}}(I\sqcap 
J)=\rsupfun{f^{-1}}I\sqcap\rsupfun{f^{-1}}J$
is true for a funcoid $f$ corresponding to a binary relation if and
only if it is a function (see proposition~\ref{rel-mono}).\end{proof}
\begin{rem}
This corollary can be reformulated as follows: For binary relations
(principal funcoids) the classic concept of monovaluedness and monovaluedness
in the above defined sense of monovaluedness of a funcoid are the
same.\end{rem}
\begin{thm}
If $f$, $g$ are funcoids, $f\sqsubseteq g$ and $g$ is monovalued
then $g|_{\dom f}=f$.\end{thm}
\begin{proof}
Obviously $g|_{\dom f}\sqsupseteq f$. Suppose for contrary that $g|_{\dom 
f}\sqsubset f$.
Then there exists an atom $a\in\atoms\dom f$ such that $\langle g|_{\dom 
f}\rangle a\neq\langle f\rangle a$
that is $\supfun ga\sqsubset\supfun fa$ what is impossible.
\end{proof}

\section{Open maps}
\begin{defn}
\index{open map}An \emph{open map} from a topological space to a
topological space is a function which maps open sets into open sets.
\end{defn}
An obvious generalization of this is \emph{open map} $f$ from an
endofuncoid $\mu$ to an endofuncoid $\nu$, which is by definition
a function (or rather a principal, entirely defined, monovalued funcoid)
from $\Ob\mu$ to $\Ob\nu$ such that
\[
\forall x\in\Ob\mu,V\in\rsupfun{\mu}\{x\}:\rsupfun 
fV\sqsupseteq\supfun{\nu}\rsupfun f@\{x\}.
\]


This formula is equivalent (exercise!) to
\[
\forall x\in\Ob\mu:\supfun f\rsupfun{\mu}@\{x\}\sqsupseteq\supfun{\nu}\rsupfun 
f@\{x\}.
\]


It can be abstracted/simplified further (now for an \emph{arbitrary}
funcoid $f$ from $\Ob\mu$ to $\Ob\nu$):
\[
\Compl(f\circ\mu)\sqsupseteq\Compl(\nu\circ f).
\]

\begin{defn}
\index{funcoid!open}An \emph{open funcoid} from an endofuncoid $\mu$
to an endofuncoid $\nu$ is a funcoid $f$ from $\Ob\mu$ to $\Ob\nu$
such that $\Compl(f\circ\mu)\sqsupseteq\Compl(\nu\circ f)$.\end{defn}
\begin{obvious}
A funcoid $f$ is open iff $f\circ\mu\sqsupseteq\Compl(\nu\circ f)$.\end{obvious}
\begin{thm}
Let $\mu$, $\nu$, $\pi$ be endofuncoids. Let $f$ be an
principal monovalued open funcoid from $\Ob\mu$ to $\Ob\nu$ and $g$ is a open funcoid
from $\Ob\nu$ to $\Ob\pi$. Then $g\circ f$ is an open funcoid from
$\Ob\mu$ to $\Ob\pi$.\end{thm}
\begin{proof}
\begin{align*}
\supfun{g\circ f}\rsupfun{\mu}@\{x\} & = \\
\supfun{g}\supfun{f}\rsupfun{\mu}@\{x\} & \sqsupseteq \\
\supfun{g}\supfun{\nu}\rsupfun{f}@\{x\} & \sqsupseteq \text{ (using that $f$ is monovalued and principal)}\\
\supfun{\pi}\supfun{g}\rsupfun{f}@\{x\} & = \\
\supfun{\pi}\supfun{g\circ f}@\{x\}.
\end{align*}
\end{proof}

\begin{problem}
Devise a pointfree (not using a particular point~$x$) proof of the above theorem.
It should refer to a lemma which may use a particular point, but the proof of the
theorem itself should be without a particular point.
\end{problem}

\section{\texorpdfstring{$T_{0}$-, $T_{1}$-, $T_{2}$-, and $T_{3}$-separable 
funcoids}%
{T0-, T1-, T2-, and T3-separable funcoids}}

\index{funcoids!separable}For funcoids it can be generalized $T_{0}$-,
$T_{1}$-, $T_{2}$-, and $T_{3}$- separability. Worthwhile note
that $T_{0}$ and $T_{2}$ separability is defined through $T_{1}$
separability.
\begin{defn}
Let call \emph{$T_{1}$-separable} such endofuncoid $f$ that for
every $\alpha,\beta\in\Ob f$ is true
\[
\alpha\ne\beta\Rightarrow\lnot(@\{\alpha\}\rsuprel f@\{\beta\}).
\]
\end{defn}
\begin{prop}
An endofuncoid $f$ is $T_{1}$-separable iff $\Cor f\sqsubseteq1_{\Ob 
f}^{\mathsf{FCD}}$.\end{prop}
\begin{proof}
~
\begin{multline*}
\forall x,y\in\Ob f:(@\{x\}\rsuprel{f}@\{y\}\Rightarrow x=y)\Leftrightarrow\\
\forall x,y\in\Ob f:(@\{x\}\rsuprel{\Cor f}@\{y\}\Rightarrow x=y)\Leftrightarrow\Cor f\sqsubseteq1_{\Ob f}^{\mathsf{FCD}}.
\end{multline*}
\end{proof}
\begin{defn}
Let call \emph{$T_{0}$-separable} such funcoid $f\in\mathsf{FCD}(A;A)$
that $f\sqcap f^{-1}$ is $T_{1}$-separable.
\end{defn}

\begin{defn}
Let call \emph{$T_{2}$-separable} such funcoid $f$ that $f^{-1}\circ f$
is $T_{1}$-separable.
\end{defn}
For symmetric transitive funcoids $T_{0}$-, $T_{1}$- and $T_{2}$-separability
are the same (see theorem \ref{sym-trans}).
\begin{obvious}
A funcoid $f$ is $T_{2}$-separable iff $\alpha\ne\beta\Rightarrow\rsupfun 
f@\{\alpha\}\nasymp\rsupfun f@\{\beta\}$
for every $\alpha,\beta\in\Src f$.\end{obvious}
\begin{defn}
Funcoid $f$ is \emph{regular} iff for every $C\in\mathscr{T}\Dst f$ and 
$p\in\Src f$
\[\supfun{f} \langle f^{- 1} \rangle C \asymp \supfun{f}
@\{ p \} \Leftarrow \uparrow^{\Src f} \{ p \}
\asymp \langle f^{- 1} \rangle C.\]
\end{defn}

\begin{prop}
The following are pairwise equivalent:
\begin{enumerate}
  \item A funcoid $f$ is regular.
  \item $\Compl (f \circ f^{- 1} \circ f) \sqsubseteq \Compl f$.
  \item $\Compl (f \circ f^{- 1} \circ f) \sqsubseteq f$.
\end{enumerate}
\end{prop}

\begin{proof}
Equivalently transform the defining formula for regular funcoids:

$\supfun{f} \langle f^{- 1} \rangle C \asymp \supfun{f}
@\{ p \} \Leftarrow \uparrow^{\Src f} \{ p \}
\asymp \langle f^{- 1} \rangle C$;

$\supfun{f} \langle f^{- 1} \rangle C \nasymp \supfun{f}
@\{ p \} \Rightarrow \uparrow^{\Src f} \{ p \} \nasymp \supfun{f^{-1}}C$;

(by definition of funcoids)

$C \nasymp \supfun{f} \langle f^{- 1} \rangle \supfun{f}
@\{ p \} \Rightarrow C \nasymp \supfun{f}
@\{ p \}$;

$\supfun{f} \langle f^{- 1} \rangle \supfun{f}
@\{ p \} \sqsubseteq \supfun{f}@\{ p \}$;

$\supfun{f \circ f^{- 1} \circ f} @\{ p \}
\sqsubseteq \supfun{f} @\{ p \}$;

$\Compl (f \circ f^{- 1} \circ f) \sqsubseteq \Compl f$;

$\Compl (f \circ f^{- 1} \circ f) \sqsubseteq f$.
\end{proof}

\begin{prop}
If $f$ is complete, regularity of funcoid $f$ is equivalent to $f \circ
\Compl (f^{- 1} \circ f) \sqsubseteq f$.
\end{prop}

\begin{proof}
  By proposition~\ref{comp-compl}.
\end{proof}

\begin{rem}
After seeing how it collapses into algebraic formulas about funcoids, the 
definition for a funcoid being regular seems quite arbitrary and sucked out of 
the finger (not an example of algebraic elegance). So I present these formulas 
only because they coincide with the traditional definition of regular 
topological spaces. However this is only my personal opinion and it may be 
wrong.
\end{rem}


\begin{defn}
An endofuncoid is $T_{3}$- iff it is both $T_{2}$- and regular.
\end{defn}

\section{Filters closed regarding a funcoid}
\begin{defn}
\index{filter!closed}Let's call \emph{closed} regarding a funcoid
$f\in\mathsf{FCD}(A;A)$ such filter $\mathcal{A}\in\mathscr{F}(\Src f)$
that $\supfun f\mathcal{A}\sqsubseteq\mathcal{A}$.
\end{defn}
This is a generalization of closedness of a set regarding an unary
operation.
\begin{prop}
If $I$ and $J$ are closed (regarding some funcoid $f$), $S$ is
a set of closed filters on $\Src f$, then
\begin{enumerate}
\item $\mathcal{I}\sqcup\mathcal{J}$ is a closed filter;
\item $\bigsqcap S$ is a closed filter.
\end{enumerate}
\end{prop}
\begin{proof}
Let denote the given funcoid as $f$. $\supfun 
f(\mathcal{I}\sqcup\mathcal{J})=\supfun f\mathcal{I}\sqcup\supfun 
f\mathcal{J}\sqsubseteq\mathcal{I}\sqcup\mathcal{J}$,
$\supfun f\bigsqcap S\sqsubseteq\bigsqcap\rsupfun{\supfun 
f}S\sqsubseteq\bigsqcap S$.
Consequently the filters $\mathcal{I}\sqcup\mathcal{J}$ and $\bigsqcap S$
are closed.\end{proof}
\begin{prop}
If $S$ is a set of filters closed regarding a complete funcoid, then
the filter $\bigsqcup S$ is also closed regarding our funcoid.\end{prop}
\begin{proof}
$\supfun f\bigsqcup S=\bigsqcup\rsupfun{\supfun f}S\sqsubseteq\bigsqcup S$
where $f$ is the given funcoid.
\end{proof}

\section{Proximity spaces}

Fix a set $U$. Let equate typed subsets of $U$ with subsets of~$U$.

We will prove that proximity spaces are essentially the same as reflexive,
symmetric, transitive funcoids.

Our primary interest here is the last axiom (\ref{prox-last}) in
the definition~\ref{prox} of proximity spaces.
\begin{prop}
If $f$ is a transitive, symmetric funcoid, then the last axiom of
proximity holds.\end{prop}
\begin{proof}
~
\begin{multline*}
\neg\left(A\rsuprel fB\right)\Leftrightarrow\neg\left(A\rsuprel{f^{-1}\circ 
f}B\right)\Leftrightarrow\rsupfun fB\asymp\rsupfun fA\Leftrightarrow\\
\exists M\in U:M\asymp\rsupfun fA\wedge\overline{M}\asymp\rsupfun fB.
\end{multline*}
\end{proof}
\begin{prop}
For a reflexive funcoid, the last axiom of proximity implies that
it is transitive and symmetric.\end{prop}
\begin{proof}
Let $\neg\left(A\rsuprel fB\right)$ implies $\exists M:M\asymp\rsupfun 
fA\wedge\overline{M}\asymp\rsupfun fB$.
Then $\neg\left(A\rsuprel fB\right)$ implies $\neg\left(A\rsuprel{f^{-1}\circ 
f}B\right)$
that is $f\sqsupseteq f^{-1}\circ f$ and thus $f=f^{-1}\circ f$.
By theorem \ref{sym-trans} $f$ is transitive and symmetric.\end{proof}
\begin{thm}
Reflexive, symmetric, transitive funcoids endofuncoids on a set~$U$
are essentially the same as proximity spaces on~$U$.\end{thm}
\begin{proof}
Above and the theorem~\ref{fcd-as-cont}.\end{proof}

