\chapter{More on connectedness}

\begin{prop}
The following are pairwise equivalent:
\begin{enumerate}
 \item a topological space on a set~$U$ is connected.
   \fxwarning{definition; can the topological definition be generalized
   to filters?}
 \item $U$ is connected regarding $f\sqcup f^{-1}$ if $f$ is the corresponding
   complete funcoid.
 \item $U$ is connected regarding $f\sqcup f^{-1}$ if $f$ is the corresponding
   closure space.
 \item $U$ is connected regarding $f\circ f^{-1}$ if $f$ is the corresponding
   complete funcoid.
\end{enumerate}
\end{prop}

\begin{proof}
??
\end{proof}

\begin{prop}
  There are filters $\mathcal{A}$, $\mathcal{B}$, such that there are no
  filters $\mathcal{X} \sqsubseteq \mathcal{A}$, $\mathcal{Y} \sqsubseteq
  \mathcal{B}$ such that $\mathcal{X} \sqcup \mathcal{Y} = \mathcal{A} \sqcup
  \mathcal{B}$ and $\mathcal{X} \asymp \mathcal{Y}$.
\end{prop}

\begin{proof}
  \url{https://math.stackexchange.com/questions/2639206}

  (It also follows that sometimes $Z (D a)$ is not a complete lattice, because
  otherwise we could prove this theorem.)
\end{proof}

\begin{prop}
  If $\mathcal{A}$, $\mathcal{B}$ are filters and $\mathcal{A} \sqcup
  \mathcal{B} = U$ is principal filter, then there are sets $X \sqsubseteq
  \mathcal{A}$, $Y \sqsubseteq \mathcal{B}$ such that $X \sqcup Y = U$ and $X
  \asymp Y$.
\end{prop}

\begin{proof}
  Take $X = \Cor \mathcal{A}$ and $Y' = \Cor \mathcal{B}$. Then $X
  \sqcup Y' = U$ because of co-separability of $\mathfrak{F} (U)$. Take $Y = U
  \setminus X$. Then $X \sqcup Y = U$ and $X \asymp Y$.
\end{proof}

\begin{prop}
  A principal filter $A$ is connected regarding endofuncoid $\mu$ iff
  \[ \forall X, Y \in \mathscr{P} (\Ob \mu) \setminus \{ \bot \} :
     \left( X \sqcup Y = A \wedge X \asymp Y \Rightarrow X \mathrel{[\mu]} Y
     \right) . \]
\end{prop}

\begin{proof}
  Easily follows from ??.
\end{proof}

\begin{defn}
  \emph{Connected component} of a filter regarding a funcoid or a reloid
  is a maximal
  connected subfilter of this filter.
\end{defn}

\begin{obvious}
Subfilter of a connected filter is connected.
\end{obvious}

\begin{prop}
  If $U$ is a principal filter, then it is connected regarding $\mu$ iff it is
  connected regarding $S (\mu)$.
  \fxnote{It should be presented as a corollary of a below theorem.}
\end{prop}

\begin{proof}
  If $U$ is connected regarding $\mu$, it is connected regarding $S (\mu)$,
  obviously.

  Suppose $U$ is connected regarding $S (\mu)$. Then for $X, Y \in \mathscr{P}
  (\Ob \mu) \setminus \{ \bot \}$ we have if $X \sqcup Y = U$ and $X
  \asymp Y$, then $X \mathrel{[S (\mu)]} Y$. So $X \times Y \nasymp 1 \sqcup
  \mu \sqcup \mu^2 \sqcup \ldots$ and thus by distributivity for principal
  filter we have $X \times Y \nasymp \mu^n$ for some $n \geq ? ?$ that is $X
  \mathrel{[\mu^n]} Y$ for some $n$ and thus there are atomic filters $p_0,
  \ldots, p_n$ such that $p_0 \in \atoms^{\mathfrak{F}} X$, $p_n \in
  \atoms^{\mathfrak{F}} Y$ and $p_i \mathrel{[\mu]} p_{i + 1}$. Thus
  there is $k$ such that $p_k \mathrel{[\mu]} p_{k + 1}$ and $p_k \in
  \atoms^{\mathfrak{F}} X$, $p_{k + 1} \in \atoms^{\mathfrak{F}}
  Y$. Thus $X \mathrel{[\mu]} Y$. We have $U$ connected regarding $\mu$.
\end{proof}

Also for $S^{\ast}$

\begin{example}
  Connected components may not form a weak partition.
\end{example}

\begin{proof}
  Consider funcoid $1^{\mathsf{FCD}(\mathbb{R})}\sqcup(\Delta \times^{\mathsf{FCD}} \Delta)$ on real line.
  Then connected components are (prove!) non-zero singletons and $\Delta$. It
  is not a weak partition.
\end{proof}

Add more counter-examples (for non-principal filters).

%%%%%%%%%%%%%%%%%%%%%%%%%%%%%%%%%%%%%%%%%%%%%%%%%%%%%%%

\begin{obvious}
Improper filter $\bot^{\mathscr{F}}$ is connected regarding:
\begin{enumerate}
  \item every funcoid;
  \item every reloid.
\end{enumerate}{\hspace*{\fill}}{\medskip}
\end{obvious}

\begin{prop}
  The only filter connected regarding
  \begin{enumerate}
    \item $\bot^{\mathsf{FCD} (A)}$;
    \item $\bot^{\mathsf{RLD} (A)}$
  \end{enumerate}
  is the improper filter $\bot^{\mathscr{F}}$.
\end{prop}

\begin{proof}
  ~
  \begin{enumerate}
    \item Let $\mathcal{A}$ be a filter. Take $\mathcal{X} = \mathcal{Y} =
    \mathcal{A} \in \mathscr{F} (\Ob \mu) \setminus \{ \bot \}$. Then
    $\mathcal{X} \sqcup \mathcal{Y} = \mathcal{A}$ but not $\mathcal{X}
    \mathrel{[\mu]} \mathcal{Y}$.

    \item $S^{\ast}_1 (\bot^{\mathsf{RLD} (A)}) = S_1
    (\bot^{\mathsf{RLD} (A)}) = \bot^{\mathsf{RLD} (A)}$. Thus
    the only connected filter is~$\bot^{\mathscr{F}}$.
  \end{enumerate}
\end{proof}

\begin{prop}
  Connected filters regarding
  \begin{enumerate}
    \item $1^{\mathsf{FCD} (A)}$;

    \item $1^{\mathsf{RLD} (A)}$
  \end{enumerate}
  are exactly ultrafilters and the improper filter.
\end{prop}

\begin{proof}
  1. That ultrafilters are connected follows from the fact that for every
  non-least $\mathcal{X}$, $\mathcal{Y}$ such that $\mathcal{X} \sqcup
  \mathcal{Y} = \mathcal{A}$ we have $\mathcal{X} = \mathcal{Y} = \mathcal{A}$
  and thus $\mathcal{X} \mathrel{[1^{\mathsf{FCD} (A)}]} \mathcal{Y}$.
  So ultrafilters are connected; so is improper filter too, because improper
  filter is always connected.

  It remains to prove that filters containing more than one distinct
  ultrafilter are not connected. Really let distinct ultrafilters $a, b \in
  \atoms \mathcal{A}$. Then not $a \mathrel{[1^{\mathsf{FCD}
  (A)}]} b$. Thus $\mathcal{A}$ is not connected.

  2. A filter $a$ is connected iff $S^{\ast}_1 (1^{\mathsf{RLD} (A)}
  \sqcap (a \times^{\mathsf{RLD}} a)) \sqsupseteq a
  \times^{\mathsf{RLD}} a$ that is iff $S^{\ast}_1
  (\id^{\mathsf{RLD}}_a) \sqsupseteq a
  \times^{\mathsf{RLD}} a$,

  $\bigsqcap_{F \in \up \id^{\mathsf{RLD}}_a} S_1 (F)
  \sqsupseteq a \times^{\mathsf{RLD}} a$ what by properties of
  generalized filter bases is equivalent to $\bigsqcap_{A \in \up a} S_1
  (\id_A) \sqsupseteq a \times^{\mathsf{RLD}} a$; $\bigsqcap_{A
  \in \up a} \id_A \sqsupseteq a \times^{\mathsf{RLD}} a$;
  $\id^{\mathsf{RLD}}_a \sqsupseteq a
  \times^{\mathsf{RLD}} a$. This is true exactly for ultrafilters and
  the improper filter.
\end{proof}

\begin{defn}
  A \emph{path} regarding funcoid $\mu$ is a tuple $p_0, \ldots, p_n$ ($n
  \in \mathbb{N}$) of atomic filters such that $p_i \mathrel{[\mu]} p_{i +
  1}$ for every $i = 0, \ldots, n - 1$.

  The number $n$ is called \emph{path length}.

  A path is \emph{between} atomic filters $a$ and $b$ iff $p_0 = a$ and
  $p_n = b$.
\end{defn}

\begin{example}
  $\mu \sqsupseteq \id^{\mathsf{FCD}}_{\mathcal{A}}$ is not
  necessary for a filter $\mathcal{A}$ to be connected regarding a funcoid
  $\mu$. Moreover $\mu \sqsupseteq 1^{\mathsf{FCD}}$ is not necessary
  for a filter $\top$ to be connected regarding a funcoid $\mu$.
\end{example}

\begin{proof}
  For counterexample take $\mu = \top \setminus 1$.

  $\langle \mu \rangle \{ x \} = \top \setminus \{ x \}$ (thus $\mu
  \mathrel{\nsqsupseteq} 1^{\mathsf{FCD}}$) and $\langle \mu
  \rangle a = \top$ for a nontrivial ultrafilter $a$.

  Let $\mathcal{X}, \mathcal{Y} \in \mathscr{F} (\Ob \mu) \setminus \{
  \bot \}$ and $\mathcal{X} \sqcup \mathcal{Y} = \top$. If $\mathcal{X}$ is a
  trivial ultrafilter then $\langle \mu \rangle \mathcal{X} = \top \setminus
  \{ x \}$ adn thus $\langle \mu \rangle \mathcal{X} \nasymp \mathcal{Y}$,
  otherwise $\langle \mu \rangle \mathcal{X} \nasymp \mathcal{Y}$. So in any
  case $\mathcal{X} \mathrel{[\mu]} \mathcal{Y}$. Funcoid $\mu$ is connected.
\end{proof}

\begin{prop}
  If there is a nonzero-length path regarding $\mu$ in the filter
  $\mathcal{A}$ between any two its atomic subfilters, then it is connected
  regarding $\mu$.
\end{prop}

\begin{proof}
  Let $\mathcal{X} \sqcup \mathcal{Y} = \mathcal{A}$, $\mathcal{X} \neq \bot$,
  $\mathcal{Y} \neq \bot$. Let $p_0, \ldots, p_n$ ($n \geq 1$) be a path in
  $\mathcal{A}$ and $p_0 \in \atoms \mathcal{X}$ and $p_n \in
  \atoms \mathcal{Y}$. Then (take $k = \min \left\{ i \in \{ 0, \ldots,
  n - 1 \} \hspace{1em} | \hspace{1em} p_{i + 1} \in \atoms \mathcal{Y}
  \right\}$) there are $p_k, p_{k + 1}$ such that $p_k \in \atoms
  \mathcal{X}$, $p_{k + 1} \in \atoms \mathcal{Y}$. But $p_k
  \mathrel{[\mu]} p_{k + 1}$ by definition of path. Thus $\mathcal{X}
  \mathrel{[\mu]} \mathcal{Y}$.
\end{proof}

\begin{prop}
  If a filter $\mathcal{A}$ is connected regarding funcoid $\mu$ reflexive on
  $\mathcal{A}$ then it is connected regarding every $\mu^n$ for $n \in
  \mathbb{Z}_+$.
\end{prop}

\begin{proof}
  Let $\mathcal{X} \sqcup \mathcal{Y} = \mathcal{A}$, $\mathcal{X} \neq \bot$,
  $\mathcal{Y} \neq \bot$. We have $\langle \mu \rangle \mathcal{X} \nasymp
  \mathcal{Y}$.

  Then $\langle \mu \rangle \mathcal{X} \mathrel{\nsqsubseteq}
  \mathcal{X}$; therefore by reflexivity $\langle \mu \rangle \mathcal{X}
  \sqsupset \mathcal{X}$. Repeating this step we get $\langle \mu \rangle
  \langle \mu \rangle \mathcal{X} \sqsupset \mathcal{X}$ that is $\langle
  \mu^2 \rangle \mathcal{X} \sqsupset \mathcal{X}$, etc.

  We have $\langle \mu^n \rangle \mathcal{X} \sqsupset \mathcal{X}$ and thus
  $\langle \mu^n \rangle \mathcal{X} \nasymp \mathcal{Y}$ that is $\mathcal{X}
  \mathrel{[\mu^n]} \mathcal{Y}$.
\end{proof}

\begin{example}
  Connected funcoid without a path between given ultrafilters.
\end{example}

\begin{proof}
  Consider $| \mathbb{R} |$. It is connected (prove!) but there is no
  path (prove!) between two distinct singletons.
\end{proof}

\begin{thm}
  If meet of two connected (regarding a funcoid) filters is non-least,
  then their join is connected.
\end{thm}

\begin{proof}
Let $\mathcal{A}$ and $\mathcal{B}$ be intersecting filters, both
connected regarding an endofuncoid~$\mu$. Let $\mathcal{X}
\sqcup \mathcal{Y} = \mathcal{A} \sqcup \mathcal{B}$ for proper filters
$\mathcal{X}$, $\mathcal{Y}$. Then either $\mathcal{X}$ or $\mathcal{Y}$
intersects $\mathcal{A} \sqcap \mathcal{B}$. Without loss of generality assume
$\mathcal{X} \sqcap \mathcal{A} \sqcap \mathcal{B} \neq \bot$.
Also $\mathcal{Y}$ intersects
either $\mathcal{A}$ or $\mathcal{B}$. Without loss of generality assume
$\mathcal{Y} \sqcap \mathcal{A} \neq \bot$.

Note $\mathcal{X} \sqcap \mathcal{A} \neq \bot$.

We have $(\mathcal{X} \sqcap \mathcal{A}) \sqcup (\mathcal{Y} \sqcap
\mathcal{A}) = (\mathcal{X} \sqcup \mathcal{Y}) \sqcap \mathcal{A} =
(\mathcal{A} \sqcup \mathcal{B}) \sqcap \mathcal{A} = \mathcal{A}$. So
$\mathcal{X} \sqcap \mathcal{A} \mathrel{[\mu]} \mathcal{Y} \sqcap
\mathcal{A}$ because $\mathcal{A}$ is connected, consequently $\mathcal{X}
\mathrel{[\mu]} \mathcal{Y}$ that is $\mathcal{A} \sqcup \mathcal{B}$ is
connected.
\end{proof}

\begin{thm}
  If meet of two connected (regarding a reloid) filters is nonempty, then
  their join is connected.
\end{thm}

\begin{proof}
  Let $S^{\ast}_1 (\mu \sqcap (\mathcal{A} \times \mathcal{A})) = \mathcal{A}
  \times \mathcal{A}$; $S^{\ast}_1 (\mu \sqcap (\mathcal{B} \times
  \mathcal{B})) = \mathcal{B} \times \mathcal{B}$ for filters $\mathcal{A}
  \nasymp \mathcal{B}$.

  $S^{\ast}_1 (\mu \sqcap ((\mathcal{A} \sqcup \mathcal{B}) \times
  (\mathcal{A} \sqcup \mathcal{B}))) = S^{\ast}_1 (\mu \sqcap ((\mathcal{A}
  \times \mathcal{A}) \sqcup (\mathcal{B} \times \mathcal{B}) \sqcup
  (\mathcal{A} \times \mathcal{B}) \sqcup (\mathcal{B} \times \mathcal{A})))
  \sqsupseteq S^{\ast}_1 (\mu \sqcap (\mathcal{A} \times \mathcal{A})) \sqcup
  S^{\ast}_1 (\mu \sqcap (\mathcal{B} \times \mathcal{B})) \sqsupseteq
  (\mathcal{A} \times \mathcal{A}) \sqcup (\mathcal{B} \times \mathcal{B})$.

  Let for example $x \in \atoms \mathcal{A}$. Then $\langle S^{\ast}_1
  (\mu \sqcap ((\mathcal{A} \sqcup \mathcal{B}) \times (\mathcal{A} \sqcup
  \mathcal{B}))) \rangle x \sqsupseteq \mathcal{A}$ and (taking into account
  $\mathcal{A} \nasymp \mathcal{B}$):
  \[ \langle \mu \sqcap ((\mathcal{A} \sqcup \mathcal{B}) \times (\mathcal{A}
     \sqcup \mathcal{B})) \rangle \langle S^{\ast}_1 (\mu \sqcap ((\mathcal{A}
     \sqcup \mathcal{B}) \times (\mathcal{A} \sqcup \mathcal{B}))) \rangle x
     \sqsupseteq \mathcal{B} . \]
  Thus $\langle S^{\ast}_1 (\mu \sqcap ((\mathcal{A} \sqcup \mathcal{B})
  \times (\mathcal{A} \sqcup \mathcal{B}))) \rangle x \sqsupseteq \mathcal{A}$
  and $\langle S^{\ast}_1 (\mu \sqcap ((\mathcal{A} \sqcup \mathcal{B}) \times
  (\mathcal{A} \sqcup \mathcal{B}))) \rangle x \sqsupseteq \mathcal{B}$ for
  every ultrafilter $x \in \atoms (\mathcal{A} \sqcup \mathcal{B})$,
  that is $\langle S^{\ast}_1 (\mu \sqcap ((\mathcal{A} \sqcup \mathcal{B})
  \times (\mathcal{A} \sqcup \mathcal{B}))) \rangle x \sqsupseteq \mathcal{A}
  \sqcup \mathcal{B}$. So $S^{\ast}_1 (\mu \sqcap ((\mathcal{A} \sqcup
  \mathcal{B}) \times (\mathcal{A} \sqcup \mathcal{B}))) \sqsupseteq
  \mathcal{A} \sqcup \mathcal{B}$ that is $\mathcal{A} \sqcup \mathcal{B}$ is
  connected.
\end{proof}

\begin{cor}
  Distinct connected components (for both a funcoid or a reloid) don't
  intersect.
\end{cor}

\begin{proof}
  If connected components $\mathcal{A} \neq \mathcal{B}$ intersect, then
  $\mathcal{A} \sqcup \mathcal{B}$ is a connected filter and either
  $\mathcal{A} \sqcup \mathcal{B} \sqsupset \mathcal{A}$ or $\mathcal{A}
  \sqcup \mathcal{B} \sqsupset \mathcal{B}$ what contradicts to the definition
  of connected components.
\end{proof}

If we add the requirement $\mathcal{X} \asymp \mathcal{Y}$ to the definition
of connected regarding funcoid, it is nonequivalent. Proof??: Consider
connectedness of an ultrafilter.

\begin{prop}
  $S (\mu) = S_1 (\mu \sqcup 1)$ if $\mu$ is an endorelation, endofuncoid, or
  endoreloid. \fxerror{for $S^{\ast}$, too.}
\end{prop}

\begin{proof}
  By proved above $(\mu \sqcup 1)^n = 1 \sqcup \mu \sqcup \ldots \sqcup
  \mu^n$.

  Thus $S_1 (\mu \sqcup 1) = (1 \sqcup \mu) \sqcup (1 \sqcup \mu \sqcup \mu^2)
  \sqcup \ldots = 1 \sqcup \mu \sqcup \mu^2 \sqcup \ldots = S (\mu)$.
\end{proof}

\fxnote{also algebraic properties of $S_1$ and $S^{\ast}_1$}

\begin{thm}
  \fxnote{Move this theorem in the book,}
  $\mathcal{X} \mathrel{\left[ \bigsqcap S \right]} \mathcal{Y}
  \Leftrightarrow \forall f \in S : \mathcal{X} \suprel{f} \mathcal{Y}$ if
  $S$ is a generalized filter base.
\end{thm}

\begin{proof}
  $\mathcal{X} \mathrel{\left[ \bigsqcap S \right]} \mathcal{Y}
  \Leftrightarrow (\mathcal{X} \times^{\mathsf{FCD}} \mathcal{Y})
  \sqcap \bigsqcap S \neq \bot \Leftrightarrow \bigsqcap_{f \in S} f \sqcap
  (\mathcal{X} \times^{\mathsf{FCD}} \mathcal{Y}) \neq \bot
  \Leftrightarrow \text{(by properties of generalized filter bases)}
  \Leftrightarrow \forall f \in S : f \sqcap (\mathcal{X}
  \times^{\mathsf{FCD}} \mathcal{Y}) \neq \bot \Leftrightarrow \forall
  f \in S : \mathcal{X} \suprel{f} \mathcal{Y}$.
\end{proof}

\begin{thm}
  The following are pairwise equivalent for a funcoid $\mu$ and filter
  $\mathcal{A}$:
  \begin{enumerate}
    \item $\mathcal{A}$ is connected regarding funcoid $\mu$

    \item $\mathcal{A}$ is connected regarding every funcoid in $\up
    \mu$.

    \item $\mathcal{A}$ is connected regarding every funcoid in
    $\up^{\Gamma} \mu$.
  \end{enumerate}
\end{thm}

\begin{proof}
  TODO: ``Connectedness'' should be moved after ``Funcoids are filters'' to
  use $\Gamma$ in this proof.

  1$\Rightarrow$2$\Rightarrow$3. Obvious.

  3$\Rightarrow$1. Let $\mathcal{X}, \mathcal{Y} \in \mathscr{F} (\Ob
  \mu)$ and $\mathcal{X} \sqcup \mathcal{Y} = \mathcal{A}$. Then $\forall f
  \in \up^{\Gamma} \mu : \mathcal{X} \suprel{f} \mathcal{Y}$.
  Therefore by the theorem ?? $\mathcal{X} \mathrel{\left[ \bigsqcap
  \up^{\Gamma} \mu \right]} \mathcal{Y}$ that is $\mathcal{X}
  \mathrel{[\mu]} \mathcal{Y}$. So $\mathcal{A}$ is connected regarding $\mu$.
\end{proof}

\begin{conjecture}
  For a $\mathbf{Rel}$-morphism $F$ and a filter $\mathcal{A}$ the
  following are pairwise equivalent:
  \begin{enumerate}
    \item $\mathcal{A}$ is connected regarding $\uparrow^{\mathsf{FCD}}
    F$.

    \item $\mathcal{A}$ is connected regarding $\uparrow^{\mathsf{RLD}}
    F$.

    \item there is a $F$-path in $\mathcal{A}$ for every two
    ultrafilters $a,b\in\atoms\mathcal{A}$.
  \end{enumerate}
\end{conjecture}

Proposed counterexample against $\mathcal{A}$ is connected regarding
$f$ iff it is connected regarding $\tofcd f$:
$f = \mathcal{A}\times^{\mathsf{RLD}}_{(F)}\mathcal{A}$.
First calculate
$(\mathcal{B}\times^{\mathsf{RLD}}_{(F)}\mathcal{C})\circ
(\mathcal{A}\times^{\mathsf{RLD}}_{(F)}\mathcal{B})$
(and also for oblique product).
