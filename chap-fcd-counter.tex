
\chapter{Counter-examples about funcoids and reloids}

For further examples we will use the filter defined by the formula
\[
\Delta=\bigsqcap^{\mathscr{F}(\mathbb{R})}\setcond{(-\epsilon;\epsilon)}{\epsilon\in\mathbb{R},\epsilon>0}.
\]
I will denote $\Omega(A)$ the Fr\'echet filter on a set $A$.
\begin{example}
There exist a funcoid $f$ and a set $S$ of funcoids such that $f\sqcap\bigsqcup S\neq\bigsqcup\rsupfun{f\sqcap}S$.\end{example}
\begin{proof}
Let $f=\Delta\times^{\mathsf{FCD}}\uparrow^{\mathscr{F}(\mathbb{R})}\{0\}$
and $S=\setcond{\uparrow^{\mathsf{FCD}(\mathbb{R};\mathbb{R})}((\epsilon;+\infty)\times\{0\})}{\epsilon\in\mathbb{R},\epsilon>0}$.
Then
\begin{multline*}
f\sqcap\bigsqcup S=(\Delta\times^{\mathsf{FCD}}\uparrow^{\mathscr{F}(\mathbb{R})}\{0\})\sqcap\uparrow^{\mathsf{FCD}(\mathbb{R};\mathbb{R})}((0;+\infty)\times\{0\})=\\
(\Delta\sqcap\uparrow^{\mathscr{F}(\mathbb{R})}(0;+\infty))\times^{\mathsf{FCD}}\uparrow^{\mathscr{F}(\mathbb{R})}\{0\}\ne\bot^{\mathsf{FCD}(\mathbb{R};\mathbb{R})}
\end{multline*}
while $\bigsqcup\rsupfun{f\sqcap}S=\bigsqcup\{\bot^{\mathsf{FCD}(\mathbb{R};\mathbb{R})}\}=\bot^{\mathsf{FCD}(\mathbb{R};\mathbb{R})}$.\end{proof}
\begin{example}
There exist a set $R$ of funcoids and a funcoid $f$ such that $f\circ\bigsqcup R\neq\bigsqcup\rsupfun{f\circ}R$.\end{example}
\begin{proof}
Let $f=\Delta\times^{\mathsf{FCD}}\uparrow^{\mathscr{F}(\mathbb{R})}\{0\}$,
$R=\setcond{\uparrow^{\mathbb{R}}\{0\}\times^{\mathsf{FCD}}\uparrow^{\mathbb{R}}(\epsilon;+\infty)}{\epsilon\in\mathbb{R},\epsilon>0}$.

We have $\bigsqcup R=\uparrow^{\mathbb{R}}\{0\}\times^{\mathsf{FCD}}\uparrow^{\mathbb{R}}(0;+\infty)$;
$f\circ\bigsqcup R=\uparrow^{\mathsf{FCD}(\mathbb{R};\mathbb{R})}(\{0\}\times\{0\})\ne\bot^{\mathsf{FCD}(\mathbb{R};\mathbb{R})}$
and $\bigsqcup\rsupfun{f\circ}R=\bigsqcup\{\bot^{\mathsf{FCD}(\mathbb{R};\mathbb{R})}\}=\bot^{\mathsf{FCD}(\mathbb{R};\mathbb{R})}$.\end{proof}
\begin{example}
There exist a set $R$ of reloids and a reloid $f$ such that $f\circ\bigsqcup R\neq\bigsqcup\rsupfun{f\circ}R$.\end{example}
\begin{proof}
Let $f=\Delta\times^{\mathsf{RLD}}\uparrow^{\mathscr{F}(\mathbb{R})}\{0\}$,
$R=\setcond{\uparrow^{\mathbb{R}}\{0\}\times^{\mathsf{RLD}}\uparrow^{\mathbb{R}}(\epsilon;+\infty)}{\epsilon\in\mathbb{R},\epsilon>0}$.

We have $\bigsqcup R=\uparrow^{\mathbb{R}}\{0\}\times^{\mathsf{RLD}}\uparrow^{\mathbb{R}}(0;+\infty)$;
$f\circ\bigsqcup R=\uparrow^{\mathsf{RLD}(\mathbb{R};\mathbb{R})}(\{0\}\times\{0\})\ne\bot^{\mathsf{RLD}(\mathbb{R};\mathbb{R})}$
and $\bigsqcup\rsupfun{f\circ}R=\bigsqcup\{\bot^{\mathsf{RLD}(\mathbb{R};\mathbb{R})}\}=\bot^{\mathsf{RLD}(\mathbb{R};\mathbb{R})}$.\end{proof}
\begin{example}
There exist a set $R$ of funcoids and filters $\mathcal{X}$ and
$\mathcal{Y}$ such that
\begin{enumerate}
\item \label{count-join-rel}$\mathcal{X}\suprel{\bigsqcup R}\mathcal{Y}\land\nexists f\in R:\mathcal{X}\suprel f\mathcal{Y}$;
\item \label{count-join-fun}$\supfun{\bigsqcup R}\mathcal{X}\sqsupset\bigsqcup\setcond{\supfun f\mathcal{X}}{f\in R}$.
\end{enumerate}
\end{example}
\begin{proof}
~
\begin{widedisorder}
\item [{\ref{count-join-rel}}] Take $\mathcal{X}=\Delta$ and $\mathcal{Y}=\top^{\mathscr{F}(\mathbb{R})}$,
$R=\setcond{\uparrow^{\mathsf{FCD}(\mathbb{R};\mathbb{R})}((\epsilon;+\infty)\times\mathbb{R})}{\epsilon\in\mathbb{R},\epsilon>0}$.
Then $\bigsqcup R=\uparrow^{\mathsf{FCD}(\mathbb{R};\mathbb{R})}((0;+\infty)\times\mathbb{R})$.
So $\mathcal{X}\suprel{\bigsqcup R}\mathcal{Y}$ and $\forall f\in R:\lnot(\mathcal{X}\suprel f\mathcal{Y})$.
\item [{\ref{count-join-fun}}] With the same $\mathcal{X}$ and $R$ we
have $\supfun{\bigsqcup R}\mathcal{X}=\top^{\mathscr{F}(\mathbb{R})}$
and $\supfun f\mathcal{X}=\bot^{\mathscr{F}(\mathbb{R})}$ for every
$f\in R$, thus $\bigsqcup\setcond{\supfun f\mathcal{X}}{f\in R}=\bot^{\mathscr{F}(\mathbb{R})}$.
\end{widedisorder}
\end{proof}
\begin{example}
$\bigsqcup_{\mathcal{B}\in T}(\mathcal{A}\times^{\mathsf{RLD}}\mathcal{B})\ne\mathcal{A}\times^{\mathsf{RLD}}\bigsqcup T$
for some filter $\mathcal{A}$ and set of filters $T$ (with a common
base).\end{example}
\begin{proof}
Take $\mathbb{R}_{+}=\setcond{x\in\mathbb{R}}{x>0}$, $\mathcal{A}=\Delta$,
$T=\setcond{\uparrow\{x\}}{x\in\mathbb{R}_{+}}$ where $\mathord\uparrow=\mathord{\uparrow^{\mathbb{R}}}$.

$\bigsqcup T=\uparrow\mathbb{R}_{+}$; $\mathcal{A}\times^{\mathsf{RLD}}\bigsqcup T=\Delta\times^{\mathsf{RLD}}\uparrow\mathbb{R}_{+}$.

$\bigsqcup_{\mathcal{B}\in T}(\mathcal{A}\times^{\mathsf{RLD}}\mathcal{B})=\bigsqcup_{x\in\mathbb{R}_{+}}(\Delta\times^{\mathsf{RLD}}\uparrow\{x\})$.

We'll prove that $\bigsqcup_{x\in\mathbb{R}_{+}}(\Delta\times^{\mathsf{RLD}}\uparrow\{x\})\ne\Delta\times^{\mathsf{RLD}}\uparrow\mathbb{R}_{+}$.

Consider $K=\bigcup_{x\in\mathbb{R}_{+}}(\{x\}\times(-1/x;1/x))$.

$K\in\up(\Delta\times^{\mathsf{RLD}}\uparrow\{x\})$ and thus $K\in\up\bigsqcup_{x\in\mathbb{R}_{+}}(\Delta\times^{\mathsf{RLD}}\uparrow\{x\})$
. But $K\notin\up(\Delta\times^{\mathsf{RLD}}\uparrow\mathbb{R}_{+})$.\end{proof}
\begin{thm}
For a filter $a$ we have $a\times^{\mathsf{RLD}}a\sqsubseteq1_{\Base(a)}^{\mathsf{RLD}}$
only in the case if $a=\bot^{\mathscr{F}(\Base(a))}$ or $a$ is a
trivial ultrafilter.\end{thm}
\begin{proof}
If $a\times^{\mathsf{RLD}}a\sqsubseteq1_{\Base(a)}^{\mathsf{RLD}}$
then there exists $m\in\up(a\times^{\mathsf{RLD}}a)$ such that $m\sqsubseteq1_{\Base(a)}^{\mathbf{Rel}}$.
Consequently there exist $A,B\in\up a$ such that $A\times B\sqsubseteq1_{\Base(a)}^{\mathbf{Rel}}$
what is possible only in the case when $\uparrow A=\uparrow B=a$
is trivial a ultrafilter or the least filter.\end{proof}
\begin{cor}
Reloidal product of a non-trivial atomic filter with itself is non-atomic.\end{cor}
\begin{proof}
Obviously $(a\times^{\mathsf{RLD}}a)\sqcap1_{\Base(a)}^{\mathsf{RLD}}\ne\bot^{\mathsf{RLD}}$
and $(a\times^{\mathsf{RLD}}a)\sqcap1_{\Base(a)}^{\mathsf{RLD}}\sqsubset a\times^{\mathsf{RLD}}a$.\end{proof}
\begin{example}
There exist two atomic reloids whose composition is non-atomic and
non-empty.\end{example}
\begin{proof}
Let $a$ be a non-trivial ultrafilter on $\mathbb{N}$ and $x\in\mathbb{N}$.
Then
\begin{multline*}
(a\times^{\mathsf{RLD}}\uparrow^{\mathbb{N}}\{x\})\circ(\uparrow^{\mathbb{N}}\{x\}\times^{\mathsf{RLD}}a)=\bigsqcap_{A\in a}^{\mathsf{RLD}(\mathbb{N};\mathbb{N})}((A\times\{x\})\circ(\{x\}\times A)=\\
\bigsqcap_{A\in a}^{\mathsf{RLD}(\mathbb{N};\mathbb{N})}(A\times A)=a\times^{\mathsf{RLD}}a
\end{multline*}
is non-atomic despite of $a\times^{\mathsf{RLD}}\uparrow^{\mathbb{N}}\{x\}$
and $\uparrow^{\mathbb{N}}\{x\}\times^{\mathsf{RLD}}a$ are atomic.\end{proof}
\begin{example}
There exists non-monovalued atomic reloid.\end{example}
\begin{proof}
From the previous example it follows that the atomic reloid $\uparrow^{\mathbb{N}}\{x\}\times^{\mathsf{RLD}}a$
is not monovalued.\end{proof}
\begin{example}
Non-convex reloids exist.\end{example}
\begin{proof}
Let $a$ be a non-trivial ultrafilter. Then $\id_{a}^{\mathsf{RLD}}$
is non-convex. This follows from the fact that only reloidal products
which are below $1_{\Base(a)}^{\mathsf{RLD}}$ are reloidal products
of ultrafilters and $\id_{a}^{\mathsf{RLD}}$ is not their join.\end{proof}
\begin{example}
$\torldin f\ne\torldout f$ for a funcoid $f$.\end{example}
\begin{proof}
Let $f=1_{\mathbb{N}}^{\mathsf{FCD}}$. Then $\torldin f=\bigsqcup_{a\in\atoms^{\mathscr{F}(\mathbb{N})}}(a\times^{\mathsf{RLD}}a)$
and $\torldout f=1_{\mathbb{N}}^{\mathsf{RLD}}$. But we have shown
above $a\times^{\mathsf{RLD}}a\nsqsubseteq1_{\mathbb{N}}^{\mathsf{RLD}}$
for non-trivial ultrafilter $a$, and so $\torldin f\nsqsubseteq\torldout f$.\end{proof}
\begin{prop}
\label{fcd-meet-frechet}$1_{\mathfrak{U}}^{\mathsf{FCD}}\sqcap\uparrow^{\mathsf{FCD}(\mathfrak{U};\mathfrak{U})}((\mathfrak{U}\times\mathfrak{U})\setminus\id_{\mathfrak{U}})=\id_{\Omega(\mathfrak{U})}^{\mathsf{FCD}}\ne\bot^{\mathsf{FCD}(\mathfrak{U};\mathfrak{U})}$
for every infinite set $\mathfrak{U}$.\end{prop}
\begin{proof}
Note that $\supfun{\id_{\Omega(\mathfrak{U})}^{\mathsf{FCD}}}\mathcal{X}=\mathcal{X}\sqcap\Omega(\mathfrak{U})$
for every filter $\mathcal{X}$ on $\mathfrak{U}$.

Let $f=1_{\mathfrak{U}}^{\mathsf{FCD}}$, $g=\uparrow^{\mathsf{FCD}(\mathfrak{U};\mathfrak{U})}((\mathfrak{U}\times\mathfrak{U})\setminus\id_{\mathfrak{U}})$.

Let $x$ be a non-trivial ultrafilter on $\mathfrak{U}$. If $X\in\up x$
then $\card X\ge2$ (In fact, $X$ is infinite but we don't need this.)
and consequently $\rsupfun gX=\top^{\mathscr{F}(\mathfrak{U})}$.
Thus $\supfun gx=\top^{\mathscr{F}(\mathfrak{U})}$. Consequently
\[
\supfun{f\sqcap g}x=\supfun fx\sqcap\supfun gx=x\sqcap\top^{\mathscr{F}(\mathfrak{U})}=x.
\]
Also $\supfun{\id_{\Omega(\mathfrak{U})}^{\mathsf{FCD}}}x=x\sqcap\Omega(\mathfrak{U})=x$.

Let now $x$ be a trivial ultrafilter. Then $\supfun fx=x$ and $\supfun gx=\top^{\mathscr{F}(\mathfrak{U})}\setminus x$.
So
\[
\supfun{f\sqcap g}x=\supfun fx\sqcap\supfun gx=x\sqcap(\top^{\mathscr{F}(\mathfrak{U})}\setminus x)=\bot^{\mathscr{F}(\mathfrak{U})}.
\]
Also $\supfun{\id_{\Omega(\mathfrak{U})}^{\mathsf{FCD}}}x=x\sqcap\Omega(\mathfrak{U})=\bot^{\mathscr{F}(\mathfrak{U})}$.

So $\supfun{f\sqcap g}x=\supfun{\id_{\Omega(\mathfrak{U})}^{\mathsf{FCD}}}x$
for every ultrafilter $x$ on $\mathfrak{U}$. Thus $f\sqcap g=\id_{\Omega(\mathfrak{U})}^{\mathsf{FCD}}$.\end{proof}
\begin{example}
There exist binary relations $f$ and $g$ such that $\uparrow^{\mathsf{FCD}(A;B)}f\sqcap\uparrow^{\mathsf{FCD}(A;B)}g\ne\uparrow^{\mathsf{FCD}(A;B)}(f\cap g)$
for some sets $A$, $B$ such that $f,g\subseteq A\times B$.\end{example}
\begin{proof}
From the proposition above.\end{proof}
\begin{example}
There exists a principal funcoid which is not a complemented element
of the lattice of funcoids.\end{example}
\begin{proof}
I will prove that quasi-complement of the funcoid $1_{\mathbb{N}}^{\mathsf{FCD}}$
is not its complement (it is enough by proposition~\ref{compl-is-pseud}). We have:
\begin{align*}
(1_{\mathbb{N}}^{\mathsf{FCD}})^{\ast} & =\\
\bigsqcup\setcond{c\in\mathsf{FCD}(\mathbb{N};\mathbb{N})}{c\asymp1_{\mathbb{N}}^{\mathsf{FCD}}} & \sqsupseteq\\
\bigsqcup\setcond{\uparrow^{\mathbb{N}}\{\alpha\}\times^{\mathsf{FCD}}\uparrow^{\mathbb{N}}\{\beta\}}{\alpha,\beta\in\mathbb{N},\uparrow^{\mathbb{N}}\{\alpha\}\times^{\mathsf{FCD}}\uparrow^{\mathbb{N}}\{\beta\}\asymp1_{\mathbb{N}}^{\mathsf{FCD}}} & =\\
\bigsqcup\setcond{\uparrow^{\mathbb{N}}\{\alpha\}\times^{\mathsf{FCD}}\uparrow^{\mathbb{N}}\{\beta\}}{\alpha,\beta\in\mathbb{N},\alpha\ne\beta} & =\\
\uparrow^{\mathsf{FCD}(\mathbb{N};\mathbb{N})}\bigcup\setcond{\{\alpha\}\times\{\beta\}}{\alpha,\beta\in\mathbb{N},\alpha\ne\beta} & =\\
\uparrow^{\mathsf{FCD}(\mathbb{N};\mathbb{N})}(\mathbb{N}\times\mathbb{N}\setminus\id_{\mathbb{N}})
\end{align*}
(used corollary \ref{fcd-compl-join}). But by proved above $(1_{\mathbb{N}}^{\mathsf{FCD}})^{\ast}\sqcap1_{\mathbb{N}}^{\mathsf{FCD}}\ne\bot^{\mathscr{F}(\mathbb{N})}$.\end{proof}
\begin{example}
There exists a funcoid $h$ such that $\up h$ is not a filter.\end{example}
\begin{proof}
Consider the funcoid $h=\id_{\Omega(\mathbb{N})}^{\mathsf{FCD}}$.
We have (from the proof of proposition \ref{fcd-meet-frechet}) that
$f\in\up h$ and $g\in\up h$, but $f\sqcap g\notin\up h$.\end{proof}
\begin{example}
There exists a funcoid $h\ne\bot^{\mathsf{FCD}(A;B)}$ such that $\torldout h=\bot^{\mathsf{RLD}(A;B)}$.\end{example}
\begin{proof}
Consider $h=\id_{\Omega(\mathbb{N})}^{\mathsf{FCD}}$. By proved above
$h=f\sqcap g$ where $f=1_{\mathbb{N}}^{\mathsf{FCD}}=\uparrow^{\mathsf{FCD}(\mathbb{N};\mathbb{N})}\id_{\mathbb{N}}$,
$g=\uparrow^{\mathsf{FCD}(\mathbb{N};\mathbb{N})}(\mathbb{N}\times\mathbb{N}\setminus\id_{\mathbb{N}})$.

We have $\id_{\mathbb{N}},\mathbb{N}\times\mathbb{N}\setminus\id_{\mathbb{N}}\in\GR h$.

So
\[
\torldout h=\bigsqcap^{\mathsf{RLD}}\up h=\bigsqcap^{\mathsf{RLD}(\mathbb{N};\mathbb{N})}\GR h\sqsubseteq\uparrow^{\mathsf{RLD}(\mathbb{N};\mathbb{N})}(\id_{\mathbb{N}}\cap(\mathbb{N}\times\mathbb{N}\setminus\id_{\mathbb{N}}))=\bot^{\mathsf{RLD}(\mathbb{N};\mathbb{N})};
\]
and thus $\torldout h=\bot^{\mathsf{RLD}(\mathbb{N};\mathbb{N})}$.\end{proof}
\begin{example}
There exists a funcoid $h$ such that $\tofcd\torldout h\ne h$.\end{example}
\begin{proof}
It follows from the previous example.\end{proof}
\begin{example}
$\torldin\tofcd f\ne f$ for some convex reloid $f$.\end{example}
\begin{proof}
Let $f=1_{\mathbb{N}}^{\mathsf{RLD}}$. Then $\tofcd f=1_{\mathbb{N}}^{\mathsf{FCD}}$.
Let $a$ be some non-trivial ultrafilter on $\mathbb{N}$. Then $\torldin\tofcd f\sqsupseteq a\times^{\mathsf{RLD}}a\nsqsubseteq1_{\mathbb{N}}^{\mathsf{RLD}}$
and thus $\torldin\tofcd f\nsqsubseteq f$.\end{proof}
\begin{example}
There exist composable funcoids $f$ and $g$ such that
\[
\torldout(g\circ f)\ne\torldout g\circ\torldout f.
\]
\end{example}
\begin{proof}
$f=\id_{\Omega(\mathbb{N})}^{\mathsf{FCD}}$ and $g=\top^{\mathscr{F}(\mathbb{N})}\times^{\mathsf{FCD}}\uparrow^{\mathbb{N}}\{\alpha\}$
for some $\alpha\in\mathbb{N}$. Then $\torldout f=\bot^{\mathsf{RLD}(\mathbb{N};\mathbb{N})}$
and thus $\torldout g\circ\torldout f=\bot^{\mathsf{RLD}(\mathbb{N};\mathbb{N})}$.

We have $g\circ f=\Omega(\mathbb{N})\times^{\mathsf{FCD}}\uparrow^{\mathbb{N}}\{\alpha\}$.

$\torldout(\Omega(\mathbb{N})\times^{\mathsf{FCD}}\uparrow^{\mathbb{N}}\{\alpha\})=\Omega(\mathbb{N})\times^{\mathsf{RLD}}\uparrow^{\mathbb{N}}\{\alpha\}$
by properties of funcoidal reloids.

% Really:
% \begin{align*}
% \torldout(\Omega(\mathbb{N})\times^{\mathsf{FCD}}\uparrow^{\mathbb{N}}\{\alpha\}) & =\\
% \bigsqcap^{\mathsf{RLD}}\up(\Omega(\mathbb{N})\times^{\mathsf{FCD}}\uparrow^{\mathbb{N}}\{\alpha\}) & =\\
% \bigsqcap_{K\in\up\Omega(\mathbb{N})}^{\mathsf{RLD}(\mathbb{N};\mathbb{N})}(K\times\uparrow^{\mathbb{N}}\{\alpha\});\\
% F\in\up\bigsqcap_{K\in\up\Omega(\mathbb{N})}^{\mathsf{RLD}(\mathbb{N};\mathbb{N})}(K\times\uparrow^{\mathbb{N}}\{\alpha\}) & \Leftrightarrow F\in\up\left(\bigsqcap_{K\in\up\Omega(\mathbb{N})}^{\mathscr{F}}K\times^{\mathsf{RLD}}\uparrow^{\mathbb{N}}\{\alpha\}\right)
% \end{align*}
% for every $F\in\subsets(\mathbb{N}\times\mathbb{N})$. Thus 
% \[
% \bigsqcap_{K\in\up\Omega(\mathbb{N})}^{\mathsf{RLD}}(K\times\uparrow^{\mathbb{N}}\{\alpha\})=\bigsqcap_{K\in\up\Omega(\mathbb{N})}^{\mathscr{F}}K\times^{\mathsf{RLD}}\uparrow^{\mathbb{N}}\{\alpha\}=\Omega(\mathbb{N})\times^{\mathsf{RLD}}\uparrow^{\mathbb{N}}\{\alpha\}.
% \]


% So $\torldout(\Omega(\mathbb{N})\times^{\mathsf{FCD}}\uparrow^{\mathbb{N}}\{\alpha\})=\Omega(\mathbb{N})\times^{\mathsf{RLD}}\uparrow^{\mathbb{N}}\{\alpha\}$.

Thus $\torldout(g\circ f)=\Omega(\mathbb{N})\times^{\mathsf{RLD}}\uparrow^{\mathbb{N}}\{\alpha\}\ne\bot^{\mathsf{RLD}(\mathbb{N};\mathbb{N})}$.\end{proof}
\begin{example}
$\tofcd$ does not preserve binary meets.\end{example}
\begin{proof}
$\tofcd(1_{\mathbb{N}}^{\mathsf{RLD}}\sqcap(\top^{\mathsf{RLD}(\mathbb{N};\mathbb{N})}\setminus1_{\mathbb{N}}^{\mathsf{RLD}}))=\tofcd\bot^{\mathsf{RLD}(\mathbb{N};\mathbb{N})}=\bot^{\mathsf{FCD}(\mathbb{N};\mathbb{N})}$.

On the other hand,
\begin{multline*}
\tofcd1_{\mathbb{N}}^{\mathsf{RLD}}\sqcap\tofcd(\top^{\mathsf{RLD}(\mathbb{N};\mathbb{N})}\setminus1_{\mathbb{N}}^{\mathsf{RLD}})=\\
1_{\mathbb{N}}^{\mathsf{FCD}}\sqcap\uparrow^{\mathsf{FCD}(\mathbb{N};\mathbb{N})}(\mathbb{N}\times\mathbb{N}\setminus\id_{\mathbb{N}})=\id_{\Omega(\mathbb{N})}^{\mathsf{FCD}}\ne\bot^{\mathsf{FCD}(\mathbb{N};\mathbb{N})}
\end{multline*}
(used proposition \ref{fcd-discr}).\end{proof}
\begin{cor}
$\tofcd$ is not an upper adjoint (in general).
\end{cor}
Considering restricting polynomials (considered as reloids) to ultrafilters,
it is simple to prove that each that restriction is injective if not
restricting a constant polynomial. Does this hold in general? No,
see the following example:
\begin{example}
There exists a monovalued reloid with atomic domain which is neither
injective nor constant (that is not a restriction of a constant function).\end{example}
\begin{proof}
(based on \cite{MO44055}) Consider the function $F\in\mathbb{N}^{\mathbb{N}\times\mathbb{N}}$
defined by the formula $(x;y)\mapsto x$.

Let $\omega_{x}$ be a non-trivial ultrafilter on the vertical line
$\{x\}\times\mathbb{N}$ for every $x\in\mathbb{N}$.

Let $T$ be the collection of such sets $Y$ that $Y\cap(\{x\}\times\mathbb{N})\in\omega_{x}$
for all but finitely many vertical lines. Obviously $T$ is a filter.

Let $\omega\in\atoms T$.

For every $x\in\mathbb{N}$ we have some $Y\in T$ for which $(\{x\}\times\mathbb{N})\cap Y=\emptyset$
and thus $\uparrow^{\mathbb{N}\times\mathbb{N}}(\{x\}\times\mathbb{N})\sqcap\omega=\bot^{\mathscr{F}(\mathbb{N}\times\mathbb{N})}$.

Let $g=(\uparrow^{\mathsf{RLD}(\mathbb{N};\mathbb{N})}F)|_{\omega}$.
If $g$ is constant, then there exist a constant function $G\in\up g$
and $F\cap G$ is also constant. Obviously $\dom\uparrow^{\mathsf{RLD}(\mathbb{N}\times\mathbb{N};\mathbb{N})}(F\cap G)\sqsupseteq\omega$.
The function $F\cap G$ cannot be constant because otherwise $\omega\sqsubseteq\dom\uparrow^{\mathsf{RLD}(\mathbb{N}\times\mathbb{N};\mathbb{N})}(F\cap G)\sqsubseteq\uparrow^{\mathbb{N}\times\mathbb{N}}(\{x\}\times\mathbb{N})$
for some $x\in\mathbb{N}$ what is impossible by proved above. So
$g$ is not constant.

Suppose that $g$ is injective. Then there exists an injection $G\in\up g$.
$F\sqcap G\in\up g$ is an injection which depends only on the first argument.
So $\dom(F\sqcap G)$ intersects each vertical line by atmost one element that
is $\overline{\dom(F\sqcap G)}$ intersects every vertical line by the whole
line or the line without one element. Thus $\overline{\dom(F\sqcap G)}\in T\sqsupseteq\omega$
and consequently $\dom(F\sqcap G)\notin\omega$ what is impossible.

Thus $g$ is neither injective nor constant.
\end{proof}

\section{Second product. Oblique product}
\begin{defn}
\index{product!second}$\mathcal{A}\times_{F}^{\mathsf{RLD}}\mathcal{B}=\torldout(\mathcal{A}\times^{\mathsf{FCD}}\mathcal{B})$
for every filters $\mathcal{A}$ and $\mathcal{B}$. I will call it
\emph{second product} of filters $A$ and $\mathcal{B}$.\end{defn}
\begin{rem}
The letter $F$ is the above definition is from the word ``funcoid''.
It signifies that it seems to be impossible to define $\mathcal{A}\times_{F}^{\mathsf{RLD}}\mathcal{B}$
directly without referring to funcoidal product.\end{rem}
\begin{defn}
\index{product!oblique}\emph{Oblique products} of filters $\mathcal{A}$
and $\mathcal{B}$ are defined as
\begin{align*}
\mathcal{A}\ltimes\mathcal{B} & =\bigsqcap\setcond{\uparrow^{\mathsf{RLD}}f}{f\in\mathbf{Rel}(\Base(\mathcal{A});\Base(\mathcal{B})),\forall B\in\up\mathcal{B}:\uparrow^{\mathsf{FCD}}f\sqsupseteq\mathcal{A}\times^{\mathsf{FCD}}\uparrow B};\\
\mathcal{A}\rtimes\mathcal{B} & =\bigsqcap\setcond{\uparrow^{\mathsf{RLD}}f}{f\in\mathbf{Rel}(\Base(\mathcal{A});\Base(\mathcal{B})),\forall A\in\up\mathcal{A}:\uparrow^{\mathsf{FCD}}f\sqsupseteq\uparrow A\times^{\mathsf{FCD}}\mathcal{B}}.
\end{align*}
\end{defn}
\begin{prop}
$\mathcal{A}\times_{F}^{\mathsf{RLD}}\mathcal{B}\sqsubseteq\mathcal{A}\ltimes\mathcal{B}\sqsubseteq\mathcal{A}\times^{\mathsf{RLD}}\mathcal{B}$
for every filters $\mathcal{A}$, $\mathcal{B}$.\end{prop}
\begin{proof}
~
\begin{align*}
\mathcal{A}\ltimes\mathcal{B} & \sqsubseteq\\
\bigsqcap\setcond{\uparrow^{\mathsf{RLD}}f}{f\in\mathbf{Rel}(\Base(\mathcal{A});\Base(\mathcal{B})),\forall A\in\up\mathcal{A},B\in\up\mathcal{B}:\uparrow^{\mathsf{FCD}}f\sqsupseteq\uparrow A\times^{\mathsf{FCD}}\uparrow B} & \sqsubseteq\\
\bigsqcap\setcond{\uparrow A\times^{\mathsf{RLD}}\uparrow B}{A\in\up\mathcal{A},B\in\up\mathcal{B}} & =\\
\mathcal{A}\times^{\mathsf{RLD}}\mathcal{B}.\\
\mathcal{A}\ltimes\mathcal{B} & \sqsupseteq\\
\bigsqcap\setcond{\uparrow^{\mathsf{RLD}}f}{f\in\mathbf{Rel}(\Base(\mathcal{A});\Base(\mathcal{B})),\uparrow^{\mathsf{FCD}}f\sqsupseteq\mathcal{A}\times^{\mathsf{FCD}}\mathcal{B}} & =\\
\bigsqcap\setcond{\uparrow^{\mathsf{RLD}}f}{f\in\up(\mathcal{A}\times^{\mathsf{FCD}}\mathcal{B})} & =\\
\torldout(\mathcal{A}\times^{\mathsf{FCD}}\mathcal{B}) & =\\
\mathcal{A}\times_{F}^{\mathsf{RLD}}\mathcal{B}.
\end{align*}
\end{proof}
\begin{conjecture}
$\mathcal{A}\times_{F}^{\mathsf{RLD}}\mathcal{B}\sqsubset\mathcal{A}\ltimes\mathcal{B}$
for some filters $\mathcal{A}$, $\mathcal{B}$.
\end{conjecture}
A stronger conjecture:
\begin{conjecture}
$\mathcal{A}\times_{F}^{\mathsf{RLD}}\mathcal{B}\sqsubset\mathcal{A}\ltimes\mathcal{B}\sqsubset\mathcal{A}\times^{\mathsf{RLD}}\mathcal{B}$
for some filters $\mathcal{A}$, $\mathcal{B}$. Particularly, is
this formula true for $\mathcal{A}=\mathcal{B}=\Delta\sqcap\uparrow^{\mathbb{R}}(0;+\infty)$?
\end{conjecture}
The above conjecture is similar to Fermat Last Theorem as having no
value by itself but being somehow challenging to prove it (not expected
to be as hard as FLT however).
\begin{example}
$\mathcal{A}\ltimes\mathcal{B}\sqsubset\mathcal{A}\times^{\mathsf{RLD}}\mathcal{B}$
for some filters $\mathcal{A}$, $\mathcal{B}$.\end{example}
\begin{proof}
It's enough to prove $\mathcal{A}\ltimes\mathcal{B}\neq\mathcal{A}\times^{\mathsf{RLD}}\mathcal{B}$.

Let $\Delta_{+}=\Delta\sqcap\uparrow^{\mathbb{R}}(0;+\infty)$. Let
$\mathcal{A}=\mathcal{B}=\Delta_{+}$.

Let $K=(\le)|_{\mathbb{R}\times\mathbb{R}}$.

Obviously $K\notin\up(\mathcal{A}\times^{\mathsf{RLD}}\mathcal{B})$.

$\mathcal{A}\ltimes\mathcal{B}\sqsubseteq\uparrow^{\mathsf{RLD}(\Base(\mathcal{A});\Base(\mathcal{B}))}K$
and thus $K\in\up(\mathcal{A}\ltimes\mathcal{B})$ because 
\[
\uparrow^{\mathsf{FCD}(\Base(\mathcal{A});\Base(\mathcal{B}))}K\sqsupseteq\Delta_{+}\times^{\mathsf{FCD}}\uparrow B=\mathcal{A}\times^{\mathsf{FCD}}\uparrow B
\]
for $B=(0;+\infty)$.

Thus $\mathcal{A}\ltimes\mathcal{B}\neq\mathcal{A}\times^{\mathsf{RLD}}\mathcal{B}$.\end{proof}
\begin{example}
\label{secprod-neq}$\mathcal{A}\times_{F}^{\mathsf{RLD}}\mathcal{B}\sqsubset\mathcal{A}\times^{\mathsf{RLD}}\mathcal{B}$
for some filters $\mathcal{A}$, $\mathcal{B}$.\end{example}
\begin{proof}
This follows from the above example.\end{proof}
\begin{prop}
$(\mathcal{A}\ltimes\mathcal{B})\sqcap(\mathcal{A}\rtimes\mathcal{B})=\mathcal{A}\times_{F}^{\mathsf{RLD}}\mathcal{B}$
for every filters $\mathcal{A}$, $\mathcal{B}$.\end{prop}
\begin{proof}
~
\begin{align*}
(\mathcal{A}\ltimes\mathcal{B})\sqcap(\mathcal{A}\rtimes\mathcal{B}) & \sqsubseteq\\
\bigsqcap\setcond{\uparrow^{\mathsf{RLD}}f}{f\in\mathbf{Rel}(\Base(\mathcal{A});\Base(\mathcal{B})),\uparrow^{\mathsf{FCD}}f\sqsupseteq\mathcal{A}\times^{\mathsf{FCD}}\mathcal{B}} & =\\
\bigsqcap\setcond{\uparrow^{\mathsf{RLD}}f}{f\in\mathbf{Rel}(\Base(\mathcal{A});\Base(\mathcal{B})),\uparrow^{\mathsf{FCD}}f\in\up(\mathcal{A}\times^{\mathsf{FCD}}\mathcal{B})} & =\\
\torldout(\mathcal{A}\times^{\mathsf{FCD}}\mathcal{B}) & =\\
\mathcal{A}\times_{F}^{\mathsf{RLD}}\mathcal{B}.
\end{align*}


To finish the proof we need to show $\mathcal{A}\ltimes\mathcal{B}\sqsupseteq\mathcal{A}\times_{F}^{\mathsf{RLD}}\mathcal{B}$
and $\mathcal{A}\rtimes\mathcal{B}\sqsupseteq\mathcal{A}\times_{F}^{\mathsf{RLD}}\mathcal{B}$.
By symmetry it's enough to show $\mathcal{A}\ltimes\mathcal{B}\sqsupseteq\mathcal{A}\times_{F}^{\mathsf{RLD}}\mathcal{B}$
what is proved above.\end{proof}
\begin{example}
$(\mathcal{A}\ltimes\mathcal{B})\sqcup(\mathcal{A}\rtimes\mathcal{B})\sqsubset\mathcal{A}\times^{\mathsf{RLD}}\mathcal{B}$
for some filters $\mathcal{A}$, $\mathcal{B}$.\end{example}
\begin{proof}
(based on \cite{MO72638}) Let $\mathcal{A}=\mathcal{B}=\Omega(\mathbb{N})$.
It's enough to prove $(\mathcal{A}\ltimes\mathcal{B})\sqcup(\mathcal{A}\rtimes\mathcal{B})\neq\mathcal{A}\times^{\mathsf{RLD}}\mathcal{B}$.

Let $X\in\up\mathcal{A}$, $Y\in\up\mathcal{B}$ that is $X\in\Omega(\mathbb{N})$,
$Y\in\Omega(\mathbb{N})$.

Removing one element $x$ from $X$ produces a set $P$. Removing
one element $y$ from $Y$ produces a set $Q$. Obviously $P\in\Omega(\mathbb{N})$,
$Q\in\Omega(\mathbb{N})$.

Obviously $(P\times\mathbb{N})\cup(\mathbb{N}\times Q)\in\up((\mathcal{A}\ltimes\mathcal{B})\sqcup(\mathcal{A}\rtimes\mathcal{B}))$.

$(P\times\mathbb{N})\cup(\mathbb{N}\times Q)\nsupseteq X\times Y$
because $(x;y)\in X\times Y$ but $(x;y)\notin(P\times\mathbb{N})\cup(\mathbb{N}\times Q)$.

Thus $(P\times\mathbb{N})\cup(\mathbb{N}\times Q)\notin\up(\mathcal{A}\times^{\mathsf{RLD}}\mathcal{B})$
by properties of filter bases.\end{proof}
\begin{example}
$\torldout\tofcd f\ne f$ for some convex reloid $f$.\end{example}
\begin{proof}
Let $f=\mathcal{A}\times^{\mathsf{RLD}}\mathcal{B}$ where $\mathcal{A}$
and $B$ are from example \ref{secprod-neq}.

$\tofcd(\mathcal{A}\times^{\mathsf{RLD}}\mathcal{B})=\mathcal{A}\times^{\mathsf{FCD}}\mathcal{B}$
by proposition \ref{fcd-of-rprod}.

So $\torldout\tofcd(\mathcal{A}\times^{\mathsf{RLD}}\mathcal{B})=\torldout(\mathcal{A}\times^{\mathsf{FCD}}\mathcal{B})=\mathcal{A}\times_{F}^{\mathsf{RLD}}\mathcal{B}\ne\mathcal{A}\times^{\mathsf{RLD}}\mathcal{B}$.\end{proof}

