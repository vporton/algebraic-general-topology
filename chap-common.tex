
\chapter{\label{chap-common}Common knowledge, part 1}

In this chapter we will consider some well known mathematical theories.
If you already know them you may skip reading this chapter (or its
parts).


\section{Order theory}


\subsection{Posets}
\begin{defn}
\index{identity relation}The \emph{identity relation} on a set $A$
is $\id_{A}=\setcond{(a;a)}{a\in A}$.
\end{defn}

\begin{defn}
\index{preorder}A \emph{preorder} on a set $A$ is a binary relation
$\sqsubseteq$ which is:
\begin{itemize}
\item reflexive on $A$ ($\opinparens{\sqsubseteq}\supseteq\id_{A}$);
\item transitive ($\opinparens{\sqsubseteq}\circ\opinparens{\sqsubseteq}\subseteq\opinparens{\sqsubseteq}$).
\end{itemize}
\end{defn}

\begin{defn}
\index{partial order}A \emph{partial order} on a set $A$ is a preorder
on $A$ which is antisymmetric ($\opinparens{\sqsubseteq}\cap\opinparens{\sqsubseteq}\subseteq\id_A$).
\end{defn}
The reverse relation is denoted $\sqsupseteq$.
\begin{defn}
\index{subelement}$a$ is a subelement of $b$ (or what is the same
$a$ is \emph{contained} in $b$ or $b$ \emph{contains} $a$) iff
$a\sqsubseteq b$.\end{defn}
\begin{obvious}
The reverse of a partial order is also a partial order.\end{obvious}
\begin{defn}
\index{set!partially ordered}A poset is a set $A$ together with
a partial order on it is called a \emph{partially ordered set} (\emph{poset}
for short).
\end{defn}

\begin{defn}
\index{partial order!strict}Strict partial order $\sqsubset$ corresponding
to the partial order $\sqsubseteq$ on a set $A$ is defined by the
formula $\opinparens{\sqsubset}=\opinparens{\sqsubseteq}\setminus\id_{A}$.
\end{defn}

\begin{defn}
\index{partial order!restricted}A partial order on a set $A$ \emph{restricted}
to a set $B\subseteq A$ is $\opinparens{\sqsubseteq}\cap(B\times B)$.\end{defn}
\begin{obvious}
A partial order on a set $A$ restricted to a set $B\subseteq A$
is a partial order on $B$.\end{obvious}
\begin{defn}
~
\begin{itemize}
\item \index{least element}The \emph{least} element $\bot$ of a poset
$\mathfrak{A}$ is defined by the formula $\forall a\in\mathfrak{A}:\bot\sqsubseteq a$.
\item \index{greatest element}The \emph{greatest} element $\top$ of a
poset $\mathfrak{A}$ is defined by the formula $\forall a\in\mathfrak{A}:\top\sqsupseteq a$.
\end{itemize}
\end{defn}
\begin{prop}
There exist no more than one least element and no more than one greatest
element (for a given poset).\end{prop}
\begin{proof}
By antisymmetry.\end{proof}
\begin{defn}
\index{dual!order}The \emph{dual} order for $\sqsubseteq$ is $\sqsupseteq$.\end{defn}
\begin{obvious}
Dual of a partial order is a partial order.\end{obvious}
\begin{defn}
\index{dual!poset}The \emph{dual} poset for a poset $(A;\mathord{\sqsubseteq})$
is the poset $(A;\mathord{\sqsupseteq})$.
\end{defn}
I will denote dual of a poset~$\mathfrak{A}$ as $(\dual\mathfrak{A})$
and dual of an element~$a\in\mathfrak{A}$ (that is the same element
in the dual poset) as $(\dual a)$.

\index{duality!partial order}Below we will sometimes use \emph{duality}
that is replacement of the partial order and all related operations
and relations with their duals. In other words, it is enough to prove
a theorem for an order $\sqsubseteq$ and the similar theorem for
$\sqsupseteq$ follows by duality.


\subsubsection{Intersecting and joining elements}

Let $\mathfrak{A}$ be a poset.
\begin{defn}
\index{intersecting elements}Call elements $a$ and $b$ of $\mathfrak{A}$
\emph{intersecting}, denoted $a\nasymp b$, when there exists a non-least
element $c$ such that $c\sqsubseteq a\land c\sqsubseteq b$.
\end{defn}

\begin{defn}
$a\asymp b\eqdef\lnot(a\nasymp b)$.\end{defn}
\begin{obvious}
$a_{0}\nasymp b_{0}\land a_{1}\sqsupseteq a_{0}\land b_{1}\sqsupseteq b_{0}\Rightarrow a_{1}\nasymp b_{1}$.\end{obvious}
\begin{defn}
\index{joining elements}I call elements $a$ and $b$ of $\mathfrak{A}$
\emph{joining} and denote $a\equiv b$ when there is no a non-greatest
element $c$ such that $c\sqsupseteq a\land c\sqsupseteq b$.
\end{defn}

\begin{defn}
$a\nequiv b\eqdef\lnot(a\equiv b)$.\end{defn}
\begin{obvious}
Intersecting is the dual of non-joining.
\end{obvious}

\begin{obvious}
$a_{0}\equiv b_{0}\land a_{1}\sqsupseteq a_{0}\land b_{1}\sqsupseteq b_{0}\Rightarrow a_{1}\equiv b_{1}$.
\end{obvious}

\subsection{Linear order}
\begin{defn}
\index{linearly ordered set}A poset $\mathfrak{A}$ is called \emph{linearly
ordered set} (or what is the same, \emph{totally ordered set}) if
$a\sqsupseteq b\lor b\sqsupseteq a$ for every $a,b\in\mathfrak{A}$.\end{defn}
\begin{example}
The set of real numbers with the customary order is a linearly ordered
set.\end{example}
\begin{defn}
\index{chain}A set $X\in\subsets\mathfrak{A}$ where $\mathfrak{A}$
is a poset is called \emph{chain} if $\mathfrak{A}$ restricted to
$X$ is a total order.
\end{defn}

\subsection{Meets and joins}

Let $\mathfrak{A}$ be a poset.
\begin{defn}
\index{least element}Given a set $X\in\subsets\mathfrak{A}$ the
\emph{least element} (also called \emph{minimum} and denoted $\min X$)
of $X$ us such $a\in X$ that $\forall x\in X:a\sqsubseteq x$.
\end{defn}
Least element does not necessarily exists. But if it exists:
\begin{prop}
For a given $X\in\subsets\mathfrak{A}$ there exist no more than one
least element.\end{prop}
\begin{proof}
It follows from anti-symmetry.
\end{proof}
\emph{Greatest element} is the dual of least element:
\begin{defn}
\index{greatest element}Given a set $X\in\subsets\mathfrak{A}$ the
\emph{greatest element} (also called \emph{maximum} and denoted $\max X$)
of $X$ us such $a\in X$ that $\forall x\in X:a\sqsupseteq x$.\end{defn}
\begin{rem}
Least and greatest elements of a set $X$ is a trivial generalization
of the above defined least and greatest element for the entire poset.\end{rem}
\begin{defn}
~
\begin{itemize}
\item \index{minimal element}A \emph{minimal} element of a set $X\in\subsets\mathfrak{A}$
is such $a\in\mathfrak{A}$ that $\nexists x\in X:(a\sqsupseteq x\land x\neq a)$.
\item \index{maximal element}A \emph{maximal} element of a set $X\in\subsets\mathfrak{A}$
is such $a\in\mathfrak{A}$ that $\nexists x\in X:(a\sqsubseteq x\land x\neq a)$.
\end{itemize}
\end{defn}
\begin{rem}
Minimal element is not the same as minimum, and maximal element is
not the same as maximum.\end{rem}
\begin{obvious}
~
\begin{enumerate}
\item The least element (if it exists) is a minimal element.
\item The greatest element (if it exists) is a maximal element.
\end{enumerate}
\end{obvious}
\begin{xca}
Show that there may be more than one minimal and more than one maximal
element for some poset.\end{xca}
\begin{defn}
\index{bounds!upper}\emph{Upper bounds} of a set $X$ is the set
$\setcond{y\in\mathfrak{A}}{\forall x\in X:y\sqsupseteq x}$.
\end{defn}
The dual notion:
\begin{defn}
\index{bounds!lower}\emph{Lower bounds} of a set $X$ is the set
$\setcond{y\in\mathfrak{A}}{\forall x\in X:y\sqsubseteq x}$.
\end{defn}

\begin{defn}
\index{join}\index{supremum}\emph{Join} $\bigsqcup X$ (also called
\emph{supremum} and denoted ``$\sup X$'') of a set $X$ is the
least element of its upper bounds (if it exists).
\end{defn}

\begin{defn}
\index{meet}\index{infimum}\emph{Meet} $\bigsqcap X$ (also called
\emph{infimum} and denoted ``$\inf X$'') of a set $X$ is the greatest
element of its lower bounds (if it exists).
\end{defn}
We will also denote $\bigsqcup_{i\in X}f(i)=\bigsqcup\setcond{f(i)}{x\in X}$
and $\bigsqcap_{i\in X}f(i)=\bigsqcap\setcond{f(i)}{x\in X}$.

We will write $b=\bigsqcup X$ when $b\in\mathfrak{A}$ is the join
of $X$ or say that $\bigsqcup X$ does not exist if there are no
such $b\in\mathfrak{A}$. (And dually for meets.)
\begin{xca}
Provide an example of $\bigsqcup X\notin X$ for some set $X$ on
some poset.\end{xca}
\begin{prop}
~
\begin{enumerate}
\item If $b$ is the greatest element of $X$ then $\bigsqcup X=b$.
\item If $b$ is the least element of $X$ then $\bigsqcap X=b$.
\end{enumerate}
\end{prop}
\begin{proof}
We will prove only the first as the second is dual.

Let $b$ be the greatest element of $X$. Then upper bounds of $X$
are $\setcond{y\in\mathfrak{A}}{y\sqsupseteq b}$. Obviously $b$
is the least element of this set, that is the join.\end{proof}
\begin{defn}
\index{join!binary}\emph{Binary joins and meets} are defined by the
formulas
\[
x\sqcup y=\bigsqcup\{x,y\}\quad\text{and}\quad x\sqcup y=\bigsqcap\{x,y\}.
\]
\end{defn}
\begin{obvious}
$\sqcup$ and $\sqcap$ are symmetric operations (whenever these are
defined for given $x$ and $y$).\end{obvious}
\begin{thm}
~
\begin{enumerate}
\item If $\bigsqcup X$ exists then $y\sqsupseteq\bigsqcup X\Leftrightarrow\forall x\in X:y\sqsupseteq x$.
\item If $\bigsqcap X$ exists then $y\sqsubseteq\bigsqcap X\Leftrightarrow\forall x\in X:y\sqsubseteq x$.
\end{enumerate}
\end{thm}
\begin{proof}
I will prove only the first as the second follows by duality.
\end{proof}
$y\sqsupseteq\bigsqcup X\Leftrightarrow y\text{ is an upper bound for }X\Leftrightarrow\forall x\in X:y\sqsupseteq x$.
\begin{cor}
~
\begin{enumerate}
\item If $a\sqcup b$ exists then $y\sqsupseteq a\sqcup b\Leftrightarrow y\sqsupseteq a\land y\sqsupseteq b$.
\item If $a\sqcap b$ exists then $y\sqsubseteq a\sqcap b\Leftrightarrow y\sqsubseteq a\land y\sqsubseteq b$.
\end{enumerate}
\end{cor}
I will denote meets and joins for a specific poset $\mathfrak{A}$
as $\bigsqcap^{\mathfrak{A}}$, $\bigsqcup^{\mathfrak{A}}$, $\sqcap^{\mathfrak{A}}$,
$\sqcup^{\mathfrak{A}}$.


\subsection{Semilattices}
\begin{defn}
~
\begin{enumerate}
\item \index{semilattice!join-semilattice}A \emph{join-semilattice} is
a poset $\mathfrak{A}$ such that $a\sqcup b$ is defined for every
$a,b\in\mathfrak{A}$.
\item \index{semilattice!meet-semilattice}A \emph{meet-semilattice} is
a poset $\mathfrak{A}$ such that $a\sqcap b$ is defined for every
$a,b\in\mathfrak{A}$.
\end{enumerate}
\end{defn}
\begin{thm}
~
\begin{enumerate}
\item The operation $\sqcup$ is associative for any join-semilattice.
\item The operation $\sqcap$ is associative for any meet-semilattice.
\end{enumerate}
\end{thm}
\begin{proof}
I will prove only the first as the second follows by duality.

We need to prove $(a\sqcup b)\sqcup c=a\sqcup(b\sqcup c)$ for every
$a,b,c\in\mathfrak{A}$.

Taking into account the definition of join, it is enough to prove
that
\[
x\sqsupseteq(a\sqcup b)\sqcup c\Leftrightarrow x\sqsupseteq a\sqcup(b\sqcup c)
\]
for every $x\in\mathfrak{A}$. Really, this follows from the chain
of equivalences:
\begin{align*}
x\sqsupseteq(a\sqcup b)\sqcup c & \Leftrightarrow\\
x\sqsupseteq a\sqcup b\land x\sqsupseteq c & \Leftrightarrow\\
x\sqsupseteq a\land x\sqsupseteq b\land x\sqsupseteq c & \Leftrightarrow\\
x\sqsupseteq a\land x\sqsupseteq b\sqcup c & \Leftrightarrow\\
x\sqsupseteq a\sqcup(b\sqcup c).
\end{align*}
\end{proof}
\begin{obvious}
$a\nasymp b$ iff $a\sqcap b$ is non-least, for every elements $a$,
$b$ of a meet-semilattice.
\end{obvious}

\begin{obvious}
$a\equiv b$ iff $a\sqcup b$ is the greatest element, for every elements
$a$, $b$ of a join-semilattice.
\end{obvious}

\subsection{Lattices and complete lattices}
\begin{defn}
\index{poset!bounded}A \emph{bounded} poset is a poset having both
least and greatest elements.
\end{defn}

\begin{defn}
\index{lattice}\emph{Lattice} is a poset which is both join-semilattice
and meet-semilattice.
\end{defn}

\begin{defn}
\index{lattice!complete}A \emph{complete lattice} is a poset $\mathfrak{A}$
such that for every $X\in\subsets\mathfrak{A}$ both $\bigsqcup X$
and $\bigsqcap X$ exist.\end{defn}
\begin{obvious}
Every complete lattice is a lattice.\end{obvious}
\begin{prop}
Every complete lattice is a bounded poset.\end{prop}
\begin{proof}
$\bigsqcup\emptyset$ is the least and $\bigsqcap\emptyset$ is the
greatest element.\end{proof}
\begin{thm}
Let $\mathfrak{A}$ be a poset.
\begin{enumerate}
\item If $\bigsqcup X$ is defined for every $X\in\subsets\mathfrak{A}$,
then $\mathfrak{A}$ is a complete lattice.
\item If $\bigsqcap X$ is defined for every $X\in\subsets\mathfrak{A}$,
then $\mathfrak{A}$ is a complete lattice.
\end{enumerate}
\end{thm}
\begin{proof}
See \cite{pm:complete-lattice-criteria} or any lattice theory reference.\end{proof}
\begin{obvious}
If $X\subseteq Y$ for some $X,Y\in\subsets\mathfrak{A}$ where $\mathfrak{A}$
is a complete lattice, then
\begin{enumerate}
\item $\bigsqcup X\sqsubseteq\bigsqcup Y$;
\item $\bigsqcap X\sqsupseteq\bigsqcap Y$.
\end{enumerate}
\end{obvious}
\begin{prop}
If $S\in\subsets\subsets\mathfrak{A}$ then for every complete lattice
$\mathfrak{A}$
\begin{enumerate}
\item $\bigsqcup\bigcup S=\bigsqcup_{X\in S}\bigsqcup X$;
\item $\bigsqcap\bigcup S=\bigsqcap_{X\in S}\bigsqcap X$.
\end{enumerate}
\end{prop}
\begin{proof}
We will prove only the first as the second is dual.

By definition of joins, it is enough to prove $y\sqsupseteq\bigsqcup\bigcup S\Leftrightarrow y\sqsupseteq\bigsqcup_{X\in S}\bigsqcup X$.

Really,
\begin{align*}
y\sqsupseteq\bigsqcup\bigcup S & \Leftrightarrow\\
\forall x\in\bigcup S:y\sqsupseteq x & \Leftrightarrow\\
\forall X\in S\forall x\in X:y\sqsupseteq x & \Leftrightarrow\\
\forall X\in S:y\sqsupseteq\bigsqcup X & \Leftrightarrow\\
y\sqsupseteq\bigsqcup_{X\in S}\bigsqcup X.
\end{align*}
\end{proof}
\begin{defn}
A \emph{sublattice} of a lattice is it subset closed regarding~$\sqcup$
and~$\sqcap$.\end{defn}
\begin{obvious}
Sublattice with induced order is also a lattice.
\end{obvious}

\subsection{Distributivity of lattices}
\begin{defn}
\index{lattice!distributive}A \emph{distributive} lattice is such
lattice $\mathfrak{A}$ that for every $x,y,z\in\mathfrak{A}$
\begin{enumerate}
\item $x\sqcap(y\sqcup z)=(x\sqcap y)\sqcup(x\sqcap z)$;
\item $x\sqcup(y\sqcap z)=(x\sqcup y)\sqcap(x\sqcup z)$.
\end{enumerate}
\end{defn}
\begin{thm}
For a lattice to be distributive it is enough just one of the conditions:
\begin{enumerate}
\item $x\sqcap(y\sqcup z)=(x\sqcap y)\sqcup(x\sqcap z)$;
\item $x\sqcup(y\sqcap z)=(x\sqcup y)\sqcap(x\sqcup z)$.
\end{enumerate}
\end{thm}
\begin{proof}
~

\begin{align*}
(x\sqcup y)\sqcap(x\sqcup z) & =\\
((x\sqcup y)\sqcap x)\sqcup((x\sqcup y)\sqcap z) & =\\
x\sqcup((x\sqcap z)\sqcup(y\sqcap z)) & =\\
(x\sqcup(x\sqcap z))\sqcup(y\sqcap z) & =\\
x\sqcup(y\sqcap z)
\end{align*}


(applied $x\sqcap(y\sqcup z)=(x\sqcap y)\sqcup(x\sqcap z)$ twice).
\end{proof}

\subsection{Difference and complement}
\begin{defn}
\index{difference}\index{substractive}Let $\mathfrak{A}$ be a distributive
lattice with least element $\bot$. The \emph{difference} (denoted
$a\setminus b$) of elements $a$ and $b$ is such $c\in\mathfrak{A}$
that $b\sqcap c=\bot$ and $a\sqcup b=b\sqcup c$. I will call $b$
\emph{substractive} from \textbf{$a$} when $a\setminus b$ exists.\end{defn}
\begin{thm}
If $\mathfrak{A}$ is a distributive lattice with least element $\bot$,
there exists no more than one difference of elements $a$, $b$.\end{thm}
\begin{proof}
Let $c$ and $d$ be both differences $a\setminus b$. Then $b\sqcap c=b\sqcap d=\bot$
and $a\sqcup b=b\sqcup c=b\sqcup d$. So
\[
c=c\sqcap(b\sqcup c)=c\sqcap(b\sqcup d)=(c\sqcap b)\sqcup(c\sqcap d)=\bot\sqcup(c\sqcap d)=c\sqcap d.
\]


Similarly $d=d\sqcap c$. Consequently $c=c\sqcap d=d\sqcap c=d$.\end{proof}
\begin{defn}
\index{complementive}I will call $b$ \emph{complementive} to $a$
iff there exists $c\in\mathfrak{A}$ such that $b\sqcap c=\bot$ and
$b\sqcup c=a$.\end{defn}
\begin{prop}
$b$ is complementive to $a$ iff $b$ is substractive from $a$ and
$b\sqsubseteq a$.\end{prop}
\begin{proof}
~
\begin{description}
\item [{$\Leftarrow$}] Obvious.
\item [{$\Rightarrow$}] We deduce $b\sqsubseteq a$ from $b\sqcup c=a$.
Thus $a\sqcup b=a=b\sqcup c$.
\end{description}
\end{proof}
\begin{prop}
If $b$ is complementive to $a$ then $(a\setminus b)\sqcup b=a$.\end{prop}
\begin{proof}
Because $b\sqsubseteq a$ by the previous proposition.\end{proof}
\begin{defn}
\index{complement}Let $\mathfrak{A}$ be a bounded distributive lattice.
The \emph{complement} (denoted $\bar{a}$) of an element $a\in\mathfrak{A}$
is such $b\in\mathfrak{A}$ that $a\sqcap b=\bot$ and $a\sqcup b=\top$.\end{defn}
\begin{prop}
If $\mathfrak{A}$ is a bounded distributive lattice then $\bar{a}=\top\setminus a$.\end{prop}
\begin{proof}
$b=\bar{a}\Leftrightarrow b\sqcap a=\bot\land b\sqcup a=\top\Leftrightarrow b\sqcap a=\bot\land\top\sqcup a=a\sqcup b\Leftrightarrow b=\top\setminus a$.\end{proof}
\begin{cor}
If $\mathfrak{A}$ is a bounded distributive lattice then exists no
more than one complement of an element $a\in\mathfrak{A}$.\end{cor}
\begin{defn}
\index{complemented!element}An element of bounded distributive lattice
is called \emph{complemented} when its complement exists.
\end{defn}

\begin{defn}
\index{complemented!lattice}A distributive lattice is a \emph{complemented
lattice} iff every its element is complemented.\end{defn}
\begin{prop}
For a distributive lattice $(a\setminus b)\setminus c=a\setminus(b\sqcup c)$
if $a\setminus b$ and $(a\setminus b)\setminus c$ are defined.\end{prop}
\begin{proof}
$((a\setminus b)\setminus c)\sqcap c=\bot$; $((a\setminus b)\setminus c)\sqcup c=(a\setminus b)\sqcup c$;
$(a\setminus b)\sqcap b=\bot$; $(a\setminus b)\sqcup b=a\sqcup b$.

We need to prove $((a\setminus b)\setminus c)\sqcap(b\sqcup c)=\bot$
and $((a\setminus b)\setminus c)\sqcup(b\sqcup c)=a\sqcup(b\sqcup c)$.

In fact,
\begin{align*}
((a\setminus b)\setminus c)\sqcap(b\sqcup c) & =\\
(((a\setminus b)\setminus c)\sqcap b)\sqcup(((a\setminus b)\setminus c)\sqcap c) & =\\
(((a\setminus b)\setminus c)\sqcap b)\sqcup\bot & =\\
((a\setminus b)\setminus c)\sqcap b & \sqsubseteq\\
(a\setminus b)\sqcap b & =\bot,
\end{align*}


so $((a\setminus b)\setminus c)\sqcap(b\sqcup c)=\bot$;
\begin{align*}
((a\setminus b)\setminus c)\sqcup(b\sqcup c) & =\\
(((a\setminus b)\setminus c)\sqcup c)\sqcup b & =\\
(a\setminus b)\sqcup c\sqcup b & =\\
((a\setminus b)\sqcup b)\sqcup c & =\\
a\sqcup b\sqcup c.
\end{align*}

\end{proof}

\subsection{Boolean lattices}
\begin{defn}
\index{lattice!boolean}A \emph{boolean lattice} is a complemented
distributive lattice.
\end{defn}
The most important example of a boolean lattice is $\subsets A$ where
\textbf{$A$} is a set, ordered by set inclusion.
\begin{thm}
\index{Morgan's laws}(\noun{De Morgan}'s laws) For every elements $a$,
$b$ of a boolean lattice
\begin{enumerate}
\item $\overline{a\sqcup b}=\bar{a}\sqcap\bar{b}$;
\item $\overline{a\sqcap b}=\bar{a}\sqcup b$.
\end{enumerate}
\end{thm}
\begin{proof}
We will prove only the first as the second is dual.

It is enough to prove that $a\sqcup b$ is a complement of $\bar{a}\sqcap\bar{b}$.
Really:
\begin{gather*}
(a\sqcup b)\sqcap(\bar{a}\sqcap\bar{b})\sqsubseteq a\sqcap(\bar{a}\sqcap\bar{b})=(a\sqcap\bar{a})\sqcap\bar{b}=\bot\sqcap\bar{b}=\bot;\\
(a\sqcup b)\sqcup(\bar{a}\sqcap\bar{b})=((a\sqcup b)\sqcup\bar{a})\sqcap((a\sqcup b)\sqcup\bar{b})\sqsupseteq(a\sqcup\bar{a})\sqcap(b\sqcup\bar{b})=\top\sqcap\top=\top.
\end{gather*}


Thus $(a\sqcup b)\sqcap(\bar{a}\sqcap\bar{b})=\bot$ and $(a\sqcup b)\sqcup(\bar{a}\sqcap\bar{b})=\top$.\end{proof}
\begin{defn}
\index{join infinite distributive}A complete lattice $\mathfrak{A}$
is \emph{join infinite distributive} when $x\sqcap\bigsqcup S=\bigsqcup\rsupfun{x\sqcap}S$;
\index{meet infinite distributive}a complete lattice $\mathfrak{A}$
is \emph{meet infinite distributive} when $x\sqcup\bigsqcap S=\bigsqcap\rsupfun{x\sqcup}S$
for all $x\in\mathfrak{A}$ and $S\in\subsets\mathfrak{A}$.
\end{defn}

\begin{defn}
\index{infinite distributive}\emph{Infinite distributive complete
lattice} is a complete lattice which is both join infinite distributive
and meet infinite distributive.\end{defn}
\begin{thm}
Every complete boolean lattice is both join infinite distributive
and meet infinite distributive.\end{thm}
\begin{proof}
We will prove only join infinitely distributivity, as the other is
dual.

Let $S$ be a subset of a complete boolean lattice.

$x\sqcap\bigsqcup S\sqsupseteq\bigsqcup\rsupfun{x\sqcap}S$ is obvious.
Now let \textbf{$u$} be any upper bound of $\rsupfun{x\sqcap}S$,
that is $x\sqcap y\sqsubseteq u$ for all $y\in S$. Then
\[
y=y\sqcap(x\sqcup\bar{x})=(y\sqcap x)\sqcup(y\sqcap\bar{x})\sqsubseteq u\sqcup\bar{x},
\]


and so $\bigsqcup S\sqsubseteq u\sqcup\bar{x}$. Thus
\[
x\sqcap\bigsqcup S\sqsubseteq x\sqcap(u\sqcup\bar{x})=(x\sqcap u)\sqcup(x\sqcap\bar{x})=(x\sqcap u)\sqcup\bot=x\sqcap u\sqsubseteq u,
\]


that is $x\sqcap\bigsqcup S$ is the least upper bound of $\rsupfun{x\sqcap}S$.\end{proof}
\begin{thm}
\index{De Morgan's laws!infinite}(infinite \noun{De Morgan}'s laws) For
every subset $S$ of a complete boolean lattice
\begin{enumerate}
\item $\overline{\bigsqcup S}=\bigsqcap_{x\in S}\bar{x}$;
\item $\overline{\bigsqcap S}=\bigsqcup{}_{x\in S}\bar{x}$.
\end{enumerate}
\end{thm}
\begin{proof}
It's enough to prove that $\bigsqcup S$ is a complement of $\bigsqcap_{x\in S}\bar{x}$
(the second follows from duality). Really, using the previous theorem:
\begin{gather*}
\bigsqcup S\sqcup\bigsqcap_{x\in S}\bar{x}=\bigsqcap_{x\in S}\rsupfun{\bigsqcup S\sqcup}\bar{x}=\bigsqcap\setcond{\bigsqcup S\sqcup\bar{x}}{x\in S}\sqsupseteq\bigsqcap\setcond{x\sqcup\bar{x}}{x\in S}=\top;\\
\bigsqcup S\sqcap\bigsqcap_{x\in S}\bar{x}=\bigsqcup_{y\in S}\rsupfun{\bigsqcap_{x\in S}\bar{x}\sqcap}y=\bigsqcup\setcond{\bigsqcap_{x\in S}\bar{x}\sqcap y}{y\in S}\sqsubseteq\bigsqcup\setcond{\bar{y}\sqcap y}{y\in S}=\bot.
\end{gather*}


So $\bigsqcup S\sqcup\bigsqcap_{x\in S}\bar{x}=\top$ and $\bigsqcup S\sqcap\bigsqcap_{x\in S}\bar{x}=\bot$.
\end{proof}

\subsection{Center of a lattice}
\begin{defn}
\index{lattice!center}The \emph{center} $Z(\mathfrak{A})$ of a bounded
distributive lattice $\mathfrak{A}$ is the set of its complemented
elements.\end{defn}
\begin{rem}
For a definition of center of non-distributive lattices see \cite{neutralelements}.
\end{rem}

\begin{rem}
In \cite{ADTCGSBVA} the word center and the notation $Z(\mathfrak{A})$
are used in a different sense.\end{rem}
\begin{defn}
\index{sublattice!closed}A sublattice $K$ of a complete lattice
$L$ is a \emph{closed sublattice} of $L$ if $K$ contains the meet
and the join of any its nonempty subset.\end{defn}
\begin{thm}
Center of an infinitely distributive lattice is its closed sublattice.\end{thm}
\begin{proof}
See \cite{center-inf-distr}.\end{proof}
\begin{rem}
See \cite{center-complete} for a more strong result.\end{rem}
\begin{thm}
The center of a bounded distributive lattice constitutes its sublattice.\end{thm}
\begin{proof}
Let $\mathfrak{A}$ be a bounded distributive lattice and $Z(\mathfrak{A})$
be its center. Let $a,b\in Z(\mathfrak{A})$. Consequently $\bar{a},\bar{b}\in Z(\mathfrak{A})$.
Then $\bar{a}\sqcup\bar{b}$ is the complement of $a\sqcap b$ because
\begin{align*}
(a\sqcap b)\sqcap(\bar{a}\sqcup\bar{b})=(a\sqcap b\sqcap\bar{a})\sqcup(a\sqcap b\sqcap\bar{b})=\bot\sqcup\bot=\bot & \text{\quad and}\\
(a\sqcap b)\sqcup(\bar{a}\sqcup\bar{b})=(a\sqcup\bar{a}\sqcup\bar{b})\sqcap(b\sqcup\bar{a\sqcup\bar{b}})=\top\sqcap\top=\top.
\end{align*}


So $a\sqcap b$ is complemented. Similarly $a\sqcup b$ is complemented.\end{proof}
\begin{thm}
\label{centr-bool}The center of a bounded distributive lattice constitutes
a boolean lattice.\end{thm}
\begin{proof}
Because it is a distributive complemented lattice.
\end{proof}

\subsection{Atoms of posets}
\begin{defn}
\index{atom}An atom of a poset is an element which has no non-least
subelements.\end{defn}
\begin{rem}
This definition is valid even for posets without least element.
\end{rem}
I will denote $\atoms^{\mathfrak{A}}a$ or just $(\atoms a)$ the
set of atoms contained in an element $a$ of a poset $\mathfrak{A}$.
I will denote $\atoms^{\mathfrak{A}}$ the set of all atoms of a poset
$\mathfrak{A}$.
\begin{defn}
\index{atomic}A poset $\mathfrak{A}$ is called \emph{atomic} iff
$\atoms a\neq\emptyset$ for every non-least element $a$ of the poset
$\mathfrak{A}$.
\end{defn}

\begin{defn}
\index{atomistic}\emph{Atomistic poset} is such a poset that $a=\bigsqcup\atoms a$
for every non-least element $a$ of this poset.\end{defn}
\begin{obvious}
Every atomistic poset is atomic.\end{obvious}
\begin{prop}
Let $\mathfrak{A}$ be a poset. If $a$ is an atom of $\mathfrak{A}$
and $B\in\mathfrak{A}$ then 
\[
a\in\atoms B\Leftrightarrow a\sqsubseteq B\Leftrightarrow a\nasymp B.
\]
\end{prop}
\begin{proof}
~
\begin{description}
\item [{$a\in\atoms B\Leftrightarrow a\sqsubseteq B$}] Obvious.
\item [{$a\sqsubseteq B\Rightarrow a\nasymp B$}] $a\sqsubseteq B\Rightarrow a\sqsubseteq a\land a\sqsubseteq B$,
thus $a\nasymp B$ because $a$ is not least.
\item [{$a\sqsubseteq B\Leftarrow a\nasymp B$}] $a\nasymp B$ implies
existence of non-least element $x$ such that $x\sqsubseteq B$ and
$x\sqsubseteq a$. Because $a$ is an atom, we have $x=a$. So $a\sqsubseteq B$.
\end{description}
\end{proof}
\begin{thm}
$\atoms\bigsqcap S=\bigcap\rsupfun{\atoms}S$ whenever $\bigsqcap S$
is defined for every $S\in\subsets\mathfrak{A}$ where $\mathfrak{A}$
is a poset.\end{thm}
\begin{proof}
For any atom
\begin{align*}
c\in\atoms\bigsqcap S & \Leftrightarrow\\
c\sqsubseteq\bigsqcap S & \Leftrightarrow\\
\forall a\in S:c\sqsubseteq a & \Leftrightarrow\\
\forall a\in S:c\in\atoms a & \Leftrightarrow\\
c\in\bigcap\rsupfun{\atoms}S.
\end{align*}
\end{proof}
\begin{cor}
\label{atoms-meet}$\atoms(a\sqcap b)=\atoms a\cap\atoms b$ for an
arbitrary meet-semilattice.\end{cor}
\begin{thm}
A complete boolean lattice is atomic iff it is atomistic.\end{thm}
\begin{proof}
~
\begin{description}
\item [{$\Leftarrow$}] Obvious.
\item [{$\Rightarrow$}] Let $\mathfrak{A}$ be an atomic boolean lattice.
Let $a\in\mathfrak{A}$. Suppose $b=\bigsqcup\atoms a\sqsubset a$.
If $x\in\atoms(a\setminus b)$ then $x\sqsubseteq a\setminus b$ and
so $x\sqsubseteq a$ and hence $x\sqsubseteq b$. But we have $x=x\sqcap b\sqsubseteq(a\setminus b)\sqcap b=\bot$
what contradicts to our supposition.
\end{description}
\end{proof}

\subsection{\index{Kuratowski's lemma}Kuratowski's lemma}
\begin{thm}
(Kuratowski's lemma) Any chain in a poset is contained in a maximal
chain (if we order chains by inclusion).
\end{thm}
I will skip the proof of Kuratowski's lemma as this proof can be found
in any set theory or order theory reference.


\subsection{Homomorphisms of posets and lattices}
\begin{defn}
\index{monotone}\index{order homomorphism}A \emph{monotone} function
(also called \emph{order homomorphism}) from a poset $\mathfrak{A}$
to a poset $\mathfrak{B}$ is such a function $f$ that $x\sqsubseteq y\Rightarrow fx\sqsubseteq fy$
for every $x,y\in\mathfrak{A}$.
\end{defn}

\begin{defn}
\index{antitone}A \emph{antitone} function (also called \emph{antitone
order homomorphism}) from a poset $\mathfrak{A}$ to a poset $\mathfrak{B}$
is such a function $f$ that $x\sqsubseteq y\Rightarrow fx\sqsupseteq fy$
for every $x,y\in\mathfrak{A}$.
\end{defn}

\begin{defn}
\index{order embedding}\emph{Order embedding} is a monotone injective
function whose inverse is also monotone.
\end{defn}

\begin{defn}
\index{antitone order embedding}\emph{Antitone order embedding} is
an antitone injective function whose inverse is also antitone.
\end{defn}

\begin{defn}
\index{order isomorphism}\emph{Order isomorphism} is a surjective
order embedding.
\end{defn}
Order isomorphism preserves properties of posets, such as order, joins
and meets, etc.
\begin{defn}
\index{antitone order isomorphism}\emph{Antitone order isomorphism}
is a surjective antitone order embedding.
\end{defn}

\begin{defn}
~
\begin{enumerate}
\item \index{join semilattice!homomorphism}\emph{Join semilattice homomorphism}
is a function $f$ from a join semilattice $\mathfrak{A}$ to a join
semilattice $\mathfrak{B}$, such that $f(x\sqcup y)=fx\sqcup fy$
for every $x,y\in\mathfrak{A}$.
\item \index{meet semilattice!homomorphism}\emph{Meet semilattice homomorphism}
is a function $f$ from a meet semilattice $\mathfrak{A}$ to a meet
semilattice $\mathfrak{B}$, such that $f(x\sqcap y)=fx\sqcap fy$
for every $x,y\in\mathfrak{A}$.
\end{enumerate}
\end{defn}
\begin{obvious}
~
\begin{enumerate}
\item Join semilattice homomorphisms are monotone.
\item Meet semilattice homomorphisms are monotone.
\end{enumerate}
\end{obvious}
\begin{defn}
\index{lattice!homomorphism}A \emph{lattice homomorphism} is a function
from a lattice to a lattice, which is both join semilattice homomorphism
and meet semilattice homomorphism.
\end{defn}

\begin{defn}
\index{complete lattice!homomorphism}\emph{Complete lattice homomorphism}
from a complete lattice $\mathfrak{A}$ to a complete lattice $\mathfrak{B}$
is a function f from $\mathfrak{A}$ to $\mathfrak{B}$ which preserves
all meets and joins, that is $f\bigsqcup S=\bigsqcup\rsupfun fS$
and $f\bigsqcap S=\bigsqcap\rsupfun fS$ for every $S\in\subsets\mathfrak{A}$.
\end{defn}

\subsection{Galois connections}

See \cite{galois-and-fixed,primer-galois} for more detailed treatment
of Galois connections.
\begin{defn}
\index{Galois!connection}Let $\mathfrak{A}$ and $\mathfrak{B}$
be two posets. A \emph{Galois connection} between $\mathfrak{A}$
and $\mathfrak{B}$ is a pair of functions $f=(f^{\ast};f_{\ast})$
with $f^{\ast}:\mathfrak{A}\rightarrow\mathfrak{B}$ and $f_{\ast}:\mathfrak{B}\rightarrow\mathfrak{A}$
such that:
\[
\forall x\in\mathfrak{A},y\in\mathfrak{B}:(f^{\ast}x\sqsubseteq y\Leftrightarrow x\sqsubseteq f_{\ast}y).
\]


\index{adjoint!lower}\index{adjoint!upper}$f_{\ast}$ is called
\emph{the upper adjoint} of $f^{\ast}$ and $f^{\ast}$ is called
\emph{the lower adjoint} of $f_{\ast}$.\end{defn}
\begin{thm}
\label{galois-second}A pair $(f^{\ast};f_{\ast})$ of functions $f^{\ast}:\mathfrak{A}\rightarrow\mathfrak{B}$
and $f_{\ast}:\mathfrak{B}\rightarrow\mathfrak{A}$ is a Galois connection
iff both of the following: 
\begin{enumerate}
\item \label{galois-second-mono}$f^{\ast}$ and $f_{\ast}$ are monotone.
\item \label{galois-second-ineq}$x\sqsubseteq f_{\ast}f^{\ast}x$ and $f^{\ast}f_{\ast}y\sqsubseteq y$
for every $x\in\mathfrak{A}$ and $y\in\mathfrak{B}$.
\end{enumerate}
\end{thm}
\begin{proof}
~
\begin{description}
\item [{$\Rightarrow$}]~

\begin{disorder}
\item [{\ref{galois-second-ineq}}] $x\sqsubseteq f_{\ast}f^{\ast}x$ since
$f^{\ast}x\sqsubseteq f^{\ast}x$; $f^{\ast}f_{\ast}y\sqsubseteq y$
since $f_{\ast}y\sqsubseteq f_{\ast}y$.
\item [{\ref{galois-second-mono}}] Let $a,b\in\mathfrak{A}$ and $a\sqsubseteq b$.
Then $a\sqsubseteq b\sqsubseteq f_{\ast}f^{\ast}b$. So by definition
$f^{\ast}a\sqsubseteq f^{\ast}b$ that is $f^{\ast}$ is monotone.
Analogously $f_{\ast}$ is monotone.
\end{disorder}
\item [{$\Leftarrow$}] $f^{\ast}x\sqsubseteq y\Rightarrow f_{\ast}f^{\ast}x\sqsubseteq f_{\ast}y\Rightarrow x\sqsubseteq f_{\ast}y$.
The other direction is analogous.
\end{description}
\end{proof}
\begin{thm}
~
\begin{enumerate}
\item \label{galois-three-a}$f^{\ast}\circ f_{\ast}\circ f^{\ast}=f^{\ast}$.
\item \label{galois-three-b}$f_{\ast}\circ f^{\ast}\circ f_{\ast}=f_{\ast}$.
\end{enumerate}
\end{thm}
\begin{proof}
~
\begin{disorder}
\item [{\ref{galois-three-a}}] Let $x\in\mathfrak{A}$. We have $x\sqsubseteq f_{\ast}f^{\ast}x$;
consequently $f^{\ast}x\sqsubseteq f^{\ast}f_{\ast}f^{\ast}x$. On
the other hand, $f^{\ast}f_{\ast}f^{\ast}x\sqsubseteq f^{\ast}x$.
So $f^{\ast}f_{\ast}f^{\ast}x=f^{\ast}x$.
\item [{\ref{galois-three-b}}] Similar.
\end{disorder}
\end{proof}
\begin{defn}
\index{idempotent}A function $f$ is called idempotent iff $f(f(X))=f(X)$
for every argument $X$.\end{defn}
\begin{prop}
$f^{\ast}\circ f_{\ast}$ and $f_{\ast}\circ f^{\ast}$ are idempotent.\end{prop}
\begin{proof}
$f^{\ast}\circ f_{\ast}$ is idempotent because $f^{\ast}f_{\ast}f^{\ast}f_{\ast}y=f^{\ast}f_{\ast}y$.
$f_{\ast}\circ f^{\ast}$ is similar.\end{proof}
\begin{thm}
\label{adj-by-other}Each of two adjoints is uniquely determined by
the other.\end{thm}
\begin{proof}
Let $p$ and $q$ be both upper adjoints of $f$. We have for all
$x\in\mathfrak{A}$ and $y\in\mathfrak{B}$:
\[
x\sqsubseteq p(y)\Leftrightarrow f(x)\sqsubseteq y\Leftrightarrow x\sqsubseteq q(y).
\]


For $x=p(y)$ we obtain $p(y)\sqsubseteq q(y)$ and for $x=q(y)$
we obtain $q(y)\sqsubseteq p(y)$. So $q(y)=p(y)$.\end{proof}
\begin{thm}
Let $f$ be a function from a poset $\mathfrak{A}$ to a poset $\mathfrak{B}$.
\begin{enumerate}
\item Both:

\begin{enumerate}
\item \label{max-gal}If $f$ is monotone and $g(b)=\max\setcond{x\in\mathfrak{A}}{fx\sqsubseteq b}$
is defined for every $b\in\mathfrak{B}$ then $g$ is the upper adjoint
of $f$.
\item \label{gal-max}If $g:\mathfrak{B}\rightarrow\mathfrak{A}$ is the
upper adjoint of $f$ then $g(b)=\max\setcond{x\in\mathfrak{A}}{fx\sqsubseteq b}$
for every $b\in\mathfrak{B}$.
\end{enumerate}
\item Both:

\begin{enumerate}
\item If $f$ is monotone and $g(b)=\min\setcond{x\in\mathfrak{A}}{fx\sqsupseteq b}$
is defined for every $b\in\mathfrak{B}$ then $g$ is the lower adjoint
of $f$.
\item If $g:\mathfrak{B}\rightarrow\mathfrak{A}$ is the lower adjoint of
$f$ then $g(b)=\min\setcond{x\in\mathfrak{A}}{fx\sqsupseteq b}$
for every $b\in\mathfrak{B}$.
\end{enumerate}
\end{enumerate}
\end{thm}
\begin{proof}
We will prove only the first as the second is its dual.
\begin{widedisorder}
\item [{\ref{max-gal}}] Let $g(b)=\max\setcond{x\in\mathfrak{A}}{fx\sqsubseteq b}$
for every $b\in\mathfrak{B}$. Then
\[
x\sqsubseteq gy\Leftrightarrow x\sqsubseteq\max\setcond{x\in\mathfrak{A}}{fx\sqsubseteq y}\Rightarrow fx\sqsubseteq y
\]
(because $f$ is monotone) and
\[
x\sqsubseteq gy\Leftrightarrow x\sqsubseteq\max\setcond{x\in\mathfrak{A}}{fx\sqsubseteq y}\Leftarrow fx\sqsubseteq y.
\]



So $fx\sqsubseteq y\Leftrightarrow x\sqsubseteq gy$ that is $f$
is the lower adjoint of $g$.

\item [{\ref{gal-max}}] We have
\[
g(b)=\max\setcond{x\in\mathfrak{A}}{fx\sqsubseteq b}\Leftrightarrow fgb\sqsubseteq b\land\forall x\in\mathfrak{A}:(fx\sqsubseteq b\Rightarrow x\sqsubseteq gb).
\]
what is true by properties of adjoints.
\end{widedisorder}
\end{proof}
\begin{thm}
Let $f$ be a function from a poset $\mathfrak{A}$ to a poset $\mathfrak{B}$.
\begin{enumerate}
\item \label{all-inf}If $f$ is an upper adjoint, $f$ preserves all existing
infima in $\mathfrak{A}$.
\item \label{inf-then-adj}If $\mathfrak{A}$ is a complete lattice and
$f$ preserves all infima, then $f$ is an upper adjoint of a function
$\mathfrak{B}\rightarrow\mathfrak{A}$.
\item If $f$ is a lower adjoint, $f$ preserves all existing suprema in
$\mathfrak{A}$.
\item If $\mathfrak{A}$ is a complete lattice and $f$ preserves all suprema,
then $f$ is an lower adjoint of a function $\mathfrak{B}\rightarrow\mathfrak{A}$.
\end{enumerate}
\end{thm}
\begin{proof}
We will prove only first two items because the rest items are similar.
\begin{widedisorder}
\item [{\ref{all-inf}}] Let $S\in\subsets\mathfrak{A}$ and $\bigsqcap S$
exists. $f\bigsqcap S$ is a lower bound for $\rsupfun fS$ because
$f$ is order-preserving. If $a$ is a lower bound for $\rsupfun fS$
then $\forall x\in S:a\sqsubseteq fx$ that is $\forall x\in S:ga\sqsubseteq x$
where $g$ is the lower adjoint of $f$. Thus $ga\sqsubseteq\bigsqcap S$
and hence $f\bigsqcap S\sqsupseteq a$. So $f\bigsqcap S$ is the
greatest lower bound for $\rsupfun fS$.
\item [{\ref{inf-then-adj}}] Let $\mathfrak{A}$ be a complete lattice
and $f$ preserves all infima. Let
\[
g(a)=\bigsqcap\setcond{x\in\mathfrak{A}}{fx\sqsupseteq a}.
\]



Since $f$ preserves infima, we have
\[
f(g(a))=\bigsqcap\setcond{f(x)}{x\in\mathfrak{A},fx\sqsupseteq a}\sqsupseteq a.
\]
$g(f(b))=\bigsqcap\setcond{x\in\mathfrak{A}}{fx\sqsupseteq fb}\sqsubseteq b$.


Obviously $f$ is monotone and thus $g$ is also monotone.


So $f$ is the upper adjoint of $g$.

\end{widedisorder}
\end{proof}
\begin{cor}
Let $f$ be a function from a complete lattice $\mathfrak{A}$ to
a poset $\mathfrak{B}$. Then:
\begin{enumerate}
\item $f$ is an upper adjoint of a function $\mathfrak{B}\rightarrow\mathfrak{A}$
iff $f$ preserves all infima in $\mathfrak{A}$.
\item $f$ is a lower adjoint of a function $\mathfrak{B}\rightarrow\mathfrak{A}$
iff $f$ preserves all suprema in $\mathfrak{A}$.
\end{enumerate}
\end{cor}

\subsection{Co-Brouwerian lattices}
\begin{defn}
\index{pseudocomplement}Let $\mathfrak{A}$ be a poset. \emph{Pseudocomplement}
of $a\in\mathfrak{A}$ is
\[
\max\setcond{c\in\mathfrak{A}}{c\asymp a}.
\]


If $z$ is the pseudocomplement of $a$ we will denote $z=a^{\ast}$.
\end{defn}

\begin{defn}
\index{pseudocomplement!dual}Let $\mathfrak{A}$ be a poset. \emph{Dual
pseudocomplement} of $a\in\mathfrak{A}$ is
\[
\min\setcond{c\in\mathfrak{A}}{c\equiv a}.
\]


If $z$ is the dual pseudocomplement of $a$ we will denote $z=a^{+}$.\end{defn}
\begin{prop}
If $a$ is a complemented element of a bounded distributive lattice,
then $\bar{a}$ is both pseudocomplement and dual pseudocomplement
of $a$.\end{prop}
\begin{proof}
Because of duality it is enough to prove that $\bar{a}$ is pseudocomplement
of $a$.

We need to prove $c\asymp a\Rightarrow c\sqsubseteq\bar{a}$ for every
element c of our poset, and $\bar{a}\asymp a$. The second is obvious.
Let's prove $c\asymp a\Rightarrow c\sqsubseteq\bar{a}$.

Really, let $c\asymp a$. Then $c\sqcap a=\bot$; $\bar{a}\sqcup(c\sqcap a)=\bar{a}$;
$(\bar{a}\sqcup c)\sqcap(\bar{a}\sqcup a)=\bar{a}$; $\bar{a}\sqcup c=\bar{a}$;
$c\sqsubseteq\bar{a}$.\end{proof}
\begin{defn}
\index{pseudodifference}Let $\mathfrak{A}$ be a join-semilattice.
Let $a,b\in\mathfrak{A}$. \emph{Pseudodifference} of $a$ and $b$
is
\[
\min\setcond{z\in\mathfrak{A}}{a\sqsubseteq b\sqcup z}.
\]


If $z$ is a pseudodifference of $a$ and $b$ we will denote $z=a\psetminus b$.\end{defn}
\begin{rem}
I do not require that $a^{\ast}$ is undefined if there are no pseudocomplement
of $a$ and likewise for dual pseudocomplement and pseudodifference.
In fact below I will define quasicomplement, dual quasicomplement,
and quasidifference which generalize pseudo-{*} counterparts. I will
denote $a^{\ast}$ the more general case of quasicomplement than of
pseudocomplement, and likewise for other notation.\end{rem}
\begin{obvious}
Dual pseudocomplement is the dual of pseudocomplement.\end{obvious}
\begin{defn}
\index{lattice!co-brouwerian}\emph{Co-brouwerian lattice} is a lattice
for which pseudodifference of any two its elements is defined.\end{defn}
\begin{prop}
Every non-empty co-brouwerian lattice $\mathfrak{A}$ has least element.\end{prop}
\begin{proof}
Let $a$ be an arbitrary lattice element. Then
\[
a\psetminus a=\min\setcond{z\in\mathfrak{A}}{a\sqsubseteq a\sqcup z}=\min\mathfrak{A}.
\]


So $\min\mathfrak{A}$ exists.\end{proof}
\begin{defn}
\index{lattice!co-Heyting}\emph{Co-Heyting lattice} is co-brouwerian
lattice with greatest element.
\end{defn}

\begin{defn}
A \emph{co-frame} is the same as a complete co-brouwerian lattice.\end{defn}
\begin{thm}
\label{cobrow-adj}For a co-brouwerian lattice $a\sqcup\mathord-$
is an upper adjoint of $\mathord-\psetminus a$ for every $a\in\mathfrak{A}$.\end{thm}
\begin{proof}
$g(b)=\min\setcond{x\in\mathfrak{A}}{a\sqcup x\sqsupseteq b}=b\psetminus a$
exists for every $b\in\mathfrak{A}$ and thus is the lower adjoint
of $a\sqcup\mathord-$.\end{proof}
\begin{cor}
\label{adjoint-brow-full}$\forall a,x,y\in\mathfrak{A}:(x\psetminus a\sqsubseteq y\Leftrightarrow x\sqsubseteq a\sqcup y)$
for a co-brouwerian lattice.\end{cor}
\begin{defn}
\index{quasidifference}Let $a,b\in\mathfrak{A}$ where $\mathfrak{A}$
is a complete lattice. \emph{Quasidifference} $a\psetminus b$ is
defined by the formula:
\[
a\psetminus b=\bigsqcap\setcond{z\in\mathfrak{A}}{a\sqsubseteq b\sqcup z}.
\]
\end{defn}
\begin{rem}
A more detailed theory of quasidifference (as well as quasicomplement
and dual quasicomplement) will be considered below.\end{rem}
\begin{lem}
$(a\psetminus b)\sqcup b=a\sqcup b$ for elements $a$, $b$ of a
meet infinite distributive complete lattice.\end{lem}
\begin{proof}
~
\begin{align*}
(a\psetminus b)\sqcup b & =\\
\bigsqcap\setcond{z\in\mathfrak{A}}{a\sqsubseteq b\sqcup z}\sqcup b & =\\
\bigsqcap\setcond{z\sqcup b}{z\in\mathfrak{A},a\sqsubseteq b\sqcup z} & =\\
\bigsqcap\setcond{t\in\mathfrak{A}}{t\sqsupseteq b,a\sqsubseteq t} & =\\
a\sqcup b.
\end{align*}
\end{proof}
\begin{thm}
\label{brow-crit}The following are equivalent for a complete lattice~$\mathfrak{A}$:
\begin{enumerate}
\item \label{cobrow-cnd-dist}$\mathfrak{A}$ is meet infinite distributive.
\item \label{cobrow-cnd-cobrow}$\mathfrak{A}$ is a co-brouwerian lattice.
\item \label{cobrow-cnd-cohey}$\mathfrak{A}$ is a co-Heyting lattice.
\item \label{cobrow-cnd-adj}$a\sqcup\mathord-$ has lower adjoint for every
$a\in\mathfrak{A}$.
\end{enumerate}
\end{thm}
\begin{proof}
~\end{proof}
\begin{description}
\item [{\ref{cobrow-cnd-cobrow}$\Leftrightarrow$\ref{cobrow-cnd-cohey}}] Obvious
(taking into account completeness of $\mathfrak{A}$).
\item [{\ref{cobrow-cnd-adj}$\Rightarrow$\ref{cobrow-cnd-dist}}] Let
$\mathord-\psetminus a$ be the lower adjoint of $a\sqcup\mathord-$.
Let $S\in\subsets\mathfrak{A}$. For every $y\in S$ we have $y\sqsupseteq(a\sqcup y)\psetminus a$
by properties of Galois connections; consequently $y\sqsupseteq\left(\bigsqcap\rsupfun{a\sqcup}S\right)\psetminus a$;
$\bigsqcap S\sqsupseteq\left(\bigsqcap\rsupfun{a\sqcup}S\right)\psetminus a$.
So
\[
a\sqcup\bigsqcap S\sqsupseteq\left(\left(\bigsqcap\rsupfun{a\sqcup}S\right)\psetminus a\right)\sqcup a\sqsupseteq\bigsqcap\rsupfun{a\sqcup}S.
\]



But $a\sqcup\bigsqcap S\sqsubseteq\bigsqcap\rsupfun{a\sqcup}S$ is
obvious.

\item [{\ref{cobrow-cnd-dist}$\Rightarrow$\ref{cobrow-cnd-cobrow}}] Let
$a\psetminus b=\bigsqcap\setcond{z\in\mathfrak{A}}{a\sqsubseteq b\sqcup z}$.
To prove that $\mathfrak{A}$ is a co-brouwerian lattice it is enough
to prove $a\sqsubseteq b\sqcup(a\psetminus b)$. But it follows from
the lemma.
\item [{\ref{cobrow-cnd-cobrow}$\Rightarrow$\ref{cobrow-cnd-adj}}] $a\psetminus b=\min\setcond{z\in\mathfrak{A}}{a\sqsubseteq b\sqcup z}$.
So $a\sqcup\mathord-$ is the upper adjoint of $\mathord-\psetminus a$.
\item [{\ref{cobrow-cnd-dist}$\Rightarrow$\ref{cobrow-cnd-adj}}] Because
$a\sqcup\mathord-$ preserves all meets.\end{description}
\begin{cor}
Co-brouwerian lattices are distributive.
\end{cor}
The following theorem is essentially borrowed from \cite{stone-spaces}:
\begin{thm}
A lattice $\mathfrak{A}$ with least element $\bot$ is co-brouwerian
with pseudodifference $\psetminus$ iff $\psetminus$ is a binary
operation on $\mathfrak{A}$ satisfying the following identities:
\begin{enumerate}
\item \label{cbpd-aa}$a\psetminus a=\bot$;
\item \label{cbpd-aba}$a\sqcup(b\psetminus a)=a\sqcup b$;
\item \label{cbpd-bba}$b\sqcup(b\psetminus a)=b$;
\item \label{cbpd-bca}$(b\sqcup c)\psetminus a=(b\psetminus a)\sqcup(c\psetminus a)$.
\end{enumerate}
\end{thm}
\begin{proof}
~
\begin{description}
\item [{$\Leftarrow$}] We have
\[
c\sqsupseteq b\psetminus a\Rightarrow c\sqcup a\sqsupseteq a\sqcup(b\psetminus a)=a\sqcup b\sqsupseteq b;
\]



$c\sqcup a\sqsupseteq b\Rightarrow c=c\sqcup(c\psetminus a)\sqsupseteq(a\psetminus a)\sqcup(c\psetminus a)=(a\sqcup c)\psetminus a\sqsupseteq b\psetminus a$.


So $c\sqsupseteq b\psetminus a\Leftrightarrow c\sqcup a\sqsupseteq b$
that is $a\sqcup\mathord-$ is an upper adjoint of $\mathord-\psetminus a$.
By a theorem above our lattice is co-brouwerian. By another theorem
above $\psetminus$ is a pseudodifference.

\item [{$\Rightarrow$}] ~

\begin{disorder}
\item [{\ref{cbpd-aa}}] Obvious.
\item [{\ref{cbpd-aba}}] ~
\begin{align*}
a\sqcup(b\psetminus a) & =\\
a\sqcup\bigsqcap\setcond{z\in\mathfrak{A}}{b\sqsubseteq a\sqcup z} & =\\
\bigsqcap\setcond{a\sqcup z}{z\in\mathfrak{A},b\sqsubseteq a\sqcup z} & =\\
a\sqcup b.
\end{align*}

\item [{\ref{cbpd-bba}}] $b\sqcup(b\psetminus a)=b\sqcup\bigsqcap\setcond{z\in\mathfrak{A}}{b\sqsubseteq a\sqcup z}=\bigsqcap\setcond{b\sqcup z}{z\in\mathfrak{A},b\sqsubseteq a\sqcup z}=b$.
\item [{\ref{cbpd-bca}}] Obviously $(b\sqcup c)\psetminus a\sqsupseteq b\psetminus a$
and $(b\sqcup c)\psetminus a\sqsupseteq c\psetminus a$. Thus $(b\sqcup c)\psetminus a\sqsupseteq(b\psetminus a)\sqcup(c\psetminus a)$.
We have
\begin{align*}
(b\psetminus a)\sqcup(c\psetminus a)\sqcup a & =\\
((b\psetminus a)\sqcup a)\sqcup((c\psetminus a)\sqcup a) & =\\
(b\sqcup a)\sqcup(c\sqcup a) & =\\
a\sqcup b\sqcup c & \sqsupseteq\\
b\sqcup c.
\end{align*}



From this by definition of adjoints: $(b\psetminus a)\sqcup(c\psetminus a)\sqsupseteq(b\sqcup c)\psetminus a$.

\end{disorder}
\end{description}
\end{proof}
\begin{thm}
\label{cup-pdiff}$\left(\bigsqcup S\right)\psetminus a=\bigsqcup_{x\in S}(x\psetminus a)$
for all $a\in\mathfrak{A}$ and $S\in\subsets\mathfrak{A}$ where
$\mathfrak{A}$ is a co-brouwerian lattice and $\bigsqcup S$ is defined.\end{thm}
\begin{proof}
Because lower adjoint preserves all suprema.\end{proof}
\begin{thm}
$(a\psetminus b)\psetminus c=a\psetminus(b\sqcup c)$ for elements
$a$, $b$, $c$ of a co-frame.\end{thm}
\begin{proof}
$a\psetminus b=\bigsqcap\setcond{z\in\mathfrak{A}}{a\sqsubseteq b\sqcup z}$.

$(a\psetminus b)\psetminus c=\bigsqcap\setcond{z\in\mathfrak{A}}{a\psetminus b\sqsubseteq c\sqcup z}$.

$a\psetminus(b\sqcup c)=\bigsqcap\setcond{z\in\mathfrak{A}}{a\sqsubseteq b\sqcup c\sqcup z}$.

It is left to prove $a\psetminus b\sqsubseteq c\sqcup z\Leftrightarrow a\sqsubseteq b\sqcup c\sqcup z$.
But this follows from corollary \ref{adjoint-brow-full}.\end{proof}
\begin{cor}
$(((a_{0}\psetminus a_{1})\psetminus\dots)\psetminus a_{n})=a_{0}\psetminus(a_{1}\sqcup\dots\sqcup a_{n})$.\end{cor}
\begin{proof}
By math induction.
\end{proof}

\subsection{Dual pseudocomplement on co-Heyting lattices}
\begin{thm}
For co-Heyting algebras $\top\psetminus b=b^{+}$.\end{thm}
\begin{proof}
~
\[
\top\psetminus b=\min\setcond{z\in\mathfrak{A}}{\top\sqsubseteq b\sqcup z}=\min\setcond{z\in\mathfrak{A}}{\top=b\sqcup z}=\min\setcond{z\in\mathfrak{A}}{b\equiv z}=b^{+}.
\]
\end{proof}
\begin{thm}
$(a\sqcap b)^{+}=a^{+}\sqcup b^{+}$ for every elements $a$, $b$
of a co-Heyting algebra.\end{thm}
\begin{proof}
$a\sqcup(a\sqcap b)^{+}\sqsupseteq(a\sqcap b)\sqcup(a\sqcap b)^{+}\sqsupseteq\top$.
So $a\sqcup(a\sqcap b)^{+}\sqsupseteq\top$; $(a\sqcap b)^{+}\sqsupseteq1\psetminus a=a^{+}$.

We have $(a\sqcap b)^{+}\sqsupseteq a^{+}$. Similarly $(a\sqcap b)^{+}\sqsupseteq b^{+}$.
Thus $(a\sqcap b)^{+}\sqsupseteq a^{+}\sqcup b^{+}$.

On the other hand, $a^{+}\sqcup b^{+}\sqcup(a\sqcap b)=(a^{+}\sqcup b^{+}\sqcup a)\sqcap(a^{+}\sqcup b^{+}\sqcup b)$
. Obviously $a^{+}\sqcup b^{+}\sqcup a=a^{+}\sqcup b^{+}\sqcup b=\top$.
So $a^{+}\sqcup b^{+}\sqcup(a\sqcap b)\sqsupseteq\top$ and thus $a^{+}\sqcup b^{+}\sqsupseteq\top\psetminus(a\sqcap b)=(a\sqcap b)^{+}$.

So $(a\sqcap b)^{+}=a^{+}\sqcup b^{+}$.
\end{proof}

\section{\index{category theory}Intro to category theory}

This is a \emph{very} basic introduction to category theory.
\begin{defn}
\index{multigraph!directed}A \emph{directed multigraph} (also known
as \emph{quiver}) is:
\begin{enumerate}
\item \index{vertex}a set $\mathcal{O}$ (\emph{vertices});
\item \index{edge}a set $\mathcal{M}$ (\emph{edge}s);
\item \index{source}\index{destination}functions $\Src$ and $\Dst$ (\emph{source}
and \emph{destination}) from $\mathcal{M}$ to $\mathcal{O}$.
\end{enumerate}
\end{defn}
\index{object}\index{morphism}Note that in category theory vertices
are called \emph{objects} and edges are called \emph{morphisms}.
\begin{defn}
\index{precategory}A \emph{precategory} is a directed multigraph
together with a partial binary operation $\circ$ on the set $\mathcal{M}$
such that $g\circ f$ is defined iff $\Dst f=\Src g$ (for every morphisms
$f$ and $g$) such that
\begin{enumerate}
\item $\Src(g\circ f)=\Src f$ and $\Dst(g\circ f)=\Dst g$ whenever the
composition $g\circ f$ of morphisms $f$ and $g$ is defined.
\item $(h\circ g)\circ f=h\circ(g\circ f)$ whenever compositions in this
equation are defined.
\end{enumerate}
\end{defn}

\begin{defn}
The set $\Hom(A;B)$ (also denoted as $\Hom_{C}(A;B)$ or just $C(A;B)$,
where $C$ is our category) (morphisms from an object $A$ to an object
$B$) is exactly morphisms which have $A$ as the source and $B$
as the destination.
\end{defn}

\begin{defn}
\index{morphism!identity}\emph{Identity morphism} is such a morphism
$e$ that $e\circ f=f$ and $g\circ e=g$ whenever compositions in
these formulas are defined.
\end{defn}

\begin{defn}
\index{category}A \emph{category} is a precategory with additional
requirement that for every object $X$ there exists identity morphism
$1_{X}$.\end{defn}
\begin{prop}
For every object $X$ there exist no more than one identity morphism.\end{prop}
\begin{proof}
Let $p$ and $q$ be both identity morphisms for a object $X$. Then
$p=p\circ q=q$.\end{proof}
\begin{defn}
\index{isomorphism}\index{inverse}An \emph{isomorphism} is such
a morphism $f$ of a category that there exists a morphism $f^{-1}$
(\emph{inverse} of $f$) such that $f\circ f^{-1}=1_{\Dst f}$ and
$f^{-1}\circ f=1_{\Src f}$.\end{defn}
\begin{prop}
An isomorphism has exactly one inverse.\end{prop}
\begin{proof}
Let $g$ and $h$ be both inverses of $f$. Then $h=h\circ1_{\Dst f}=h\circ f\circ g=1_{\Src f}\circ g=g$.\end{proof}
\begin{defn}
\index{groupoid}A \emph{groupoid} is a category all of whose morphisms
are isomorphisms.
\end{defn}

\begin{defn}
\index{endomorphism}A morphism whose source is the same as destination
is called \emph{endomorphism}.
\end{defn}

\begin{defn}
\index{involution}\index{morphism!involutive}An \emph{involution}
or \emph{involutive morphism} is an endomorphism~$f$ that $f\circ f=1_{\Ob f}$.
In other words, an involution is such a self-inverse (that is conforming
to the formula~$f=f^{-1}$) isomorphism.
\end{defn}

\begin{defn}
\emph{Functor} from category~$C$ to category~$D$ is a mapping~$F$
which associates every object~$X$ of~$C$ with an object~$F(X)$
of~$D$ and every morphism~$f:X\rightarrow Y$ of~$C$ with morphism
$F(f):F(X)\rightarrow F(Y)$ of~$D$, such that:
\begin{enumerate}
\item $F(g\circ f)=F(g)\circ F(f)$ for every composable morphisms $f$,~$g$
of $C$;
\item $F(1_{X}^{C})=1_{X}^{D}$ for every object~$X$ of~$C$.
\end{enumerate}
\end{defn}

\subsection{Some important examples of categories}
\begin{xca}
Prove that the below examples of categories are really categories.\end{xca}
\begin{defn}
The category $\mathbf{Set}$ is:
\begin{itemize}
\item Objects are small sets.
\item Morphisms from an object $A$ to an object $B$ are triples $(A;B;f)$
where $f$ is a function from $A$ to $B$.
\item Composition of morphisms is defined by the formula: $(B;C;g)\circ(A;B;f)=(A;C;g\circ f)$
where $g\circ f$ is function composition.
\end{itemize}
\end{defn}

\begin{defn}
The category $\mathbf{Rel}$ is:
\begin{itemize}
\item Objects are small sets.
\item Morphisms from an object $A$ to an object $B$ are triples $(A;B;f)$
where $f$ is a binary relation between $A$ and $B$.
\item Composition of morphisms is defined by the formula: $(B;C;g)\circ(A;B;f)=(A;C;g\circ f)$
where $g\circ f$ is relation composition.
\end{itemize}
\end{defn}
I will denote $\GR(A;B;f)=f$ for any morphism $(A;B;f)$ of either
$\mathbf{Set}$ or $\mathbf{Rel}$.


\begin{defn}
A \emph{subcategory} of a category~$C$ is a category whose set of
objects is a subset of the set of objects of~$C$ and whose set of
morphisms is a subset of the set of morphisms of~$C$.
\end{defn}

\begin{defn}
\index{subcategory!wide}\emph{Wide subcategory} of a category $(\mathcal{O};\mathcal{M})$
is a category $(\mathcal{O};\mathcal{M}')$ where $\mathcal{M}\subseteq\mathcal{M}'$
and the composition on $(\mathcal{O};\mathcal{M}')$ is a restriction
of composition of $(\mathcal{O};\mathcal{M})$. (Similarly \emph{wide
sub-precategory} can be defined.)
\end{defn}

\subsection{Commutative diagrams}
\begin{defn}
A \emph{finite path in directed multigraph} is a tuple $\llbracket e_{0},\ldots,e_{n}\rrbracket$
of edges (where $i\in\mathbb{N}$) such that $\Dst e_{i}=\Src e_{i-1}$
for every $i=0,\dots,n-1$.
\end{defn}

\begin{defn}
The vertices of a finite path are $\Src e_{0}$, $\Dst e_{0}=\Src e_{1}$,
$\Dst e_{1}=\Src e_{2}$, \ldots{}, $\Dst e_{n}$.
\end{defn}

\begin{defn}
Composition of finite paths $\llbracket e_{0},\ldots,e_{n}\rrbracket$
and $\llbracket e_{k},\ldots,e_{m}\rrbracket$ (where $\Dst e_{n}=\Src e_{k}$)
is the path $\llbracket e_{0},\ldots,e_{n},e_{k},\dots e_{m}\rrbracket$.
(It is a path because $\Dst e_{n}=\Src e_{k}$.)
\end{defn}

\begin{defn}
A \emph{cycle} is a finite path whose first vertex is the same as
the last vertex (in other words $\Dst e_{n}=\Src e_{0}$).
\end{defn}

\begin{defn}
A \emph{diagram} in~$C$ is a directed multigraph, whose vertices
are labeled with objects of~$C$ and whose edges are labeled with
morphisms of~$C$.

I will denote the morphism corresponding to a edge~$e$ as $D(e)$.
\end{defn}

\begin{defn}
A diagram in~$C$ is \emph{commutative} when the composition of morphisms
corresponding to a finite path is always the same for finite paths
from a every vertex~$A$ to a vertex~~$B$ independently of the
path choice.

We will say ``\emph{commutative diagram}'' when commutativity of
a diagram is implied by the context.\end{defn}
\begin{rem}
See \href{https://en.wikipedia.org/wiki/Commutative_diagram}{Wikipedia}
for more on definition and examples of commutative diagrams.
\end{rem}
The following is an example of a commutative diagram in~$\mathbf{Set}$
(because $x+5-3=x+4-2$):

\[
\begin{tikzcd}
\mathbb{N} \arrow[r, "+5"] \arrow[d, "+4"] & \mathbb{N} \arrow[d, "-3"] \\
\mathbb{N} \arrow[r, "-2"] & \mathbb{N}
\end{tikzcd}
\]

We are essentially interested in the special case of commutative diagrams
every morphism of which is an isomorphism. So, the below theorem.
\begin{thm}
\label{rehash-isos}If morphisms corresponding to every edge~$e_{i}$
a path $\llbracket e_{0},\ldots,e_{n}\rrbracket$ are isomorphisms
then the following are equivalent:
\begin{itemize}
\item The morphism induced by $\llbracket e_{0},\ldots,e_{n}\rrbracket$
is identity.
\item The morphism induced by $\llbracket e_{n},e_{0},\ldots,e_{n-1}\rrbracket$
is identity.
\item The morphism induced by $\llbracket e_{n-1},e_{n},e_{0},\ldots,e_{n-2}\rrbracket$
is identity.
\item \ldots{}
\item The morphism induced by $\llbracket e_{1},e_{2},\ldots,e_{n},e_{0}\rrbracket$
is identity.
\end{itemize}
In other words, the cycle being an identity does not depend on the
choice of the start edge in the cycle.\end{thm}
\begin{proof}
Each step in the proof is like:
\begin{multline*}
D(n)\circ\dots\circ D(e_{0})=1_{\Src D(e_{0})}\Leftrightarrow\\
D(n)^{-1}\circ D(n)\circ\dots\circ D(e_{0})\circ D(n)=D(n)^{-1}\circ1_{\Src D(e_{0})}\circ D(n)\Leftrightarrow\\
D(n-1)\circ\dots\circ D(e_{0})\circ D(n)=1_{\Src D(e_{n})}.
\end{multline*}
\end{proof}
\begin{lem}
\label{three-loop-lem}Let $f$, $g$, $h$ be isomorphisms. Let $g\circ f=h^{-1}$.
The diagram at the figure~\ref{three-loop} is commutative, every
cycle in the diagram is an identity.

\begin{figure}[ht]
\caption{\label{three-loop}}


\begin{tikzcd}[row sep=2.5cm, column sep=1.5cm]
& \bullet
\arrow[rd, shift left, "f"]
\arrow[ld, shift left, "h^{-1}"] \\
\bullet
\arrow[ru, shift left, "h"]
\arrow[rr, shift left, "g^{-1}"]
& & \bullet
\arrow[lu, shift left, "f^{-1}"]
\arrow[ll, shift left, "g"]
\end{tikzcd}
\end{figure}
\end{lem}
\begin{proof}
We will prove by induction that every cycle of the length~$N$ in
the diagram is an identity.

For cycles of length~$2$ it holds by definition of isomorphism.

For cycles of length~$3$ it holds by theorem~\ref{rehash-isos}.

Consider a cycle of length above~$3$. It is easy to show that this
cycles contains a sub-cycle of length~$3$ or below. (Consider three
first edges $a\overset{e_{0}}{\rightarrow}b\overset{e_{1}}{\rightarrow}c\overset{e_{2}}{\rightarrow}d$
of the path, by pigeonhole principle we have that there are equal
elements among $a$, $b$, $c$,~$d$.) We can exclude the sub-cycle
because it is identity. Thus we reduce to cycles of lesser length.
Applying math induction, we get that every cycle in the diagram is
an identity.

That the diagram is commutative follows from it (because for paths~$\sigma$,~$\tau$
we have the paths $\sigma\circ\tau^{-1}$ and $\tau\circ\sigma^{-1}$
being identities).\end{proof}
\begin{lem}
\label{four-loop-lem}Let $f$, $g$, $h$, $t$ be isomorphisms.
Let $t\circ h\circ g\circ f=1_{\Src f}$. The diagram at the figure~\ref{four-loop}
is commutative, every cycle in the diagram is an identity.

\begin{figure}[ht]
\caption{\label{four-loop}}


\begin{tikzcd}[row sep=1.5cm, column sep=1.5cm]
\circled{(0;0)} \arrow[r, shift left, "f"] \arrow[d, shift left, "t^{-1}"]
  & \circled{(0;1)} \arrow[d, shift left, "g"] \arrow[l, shift left, "f^{-1}"] \\
\circled{(1;0)} \arrow[u, shift left, "t"] \arrow[r, shift left, "h^{-1}"]
  & \circled{(1;1)} \arrow[l, shift left, "h"] \arrow[u, shift left, "g^{-1}"]
\end{tikzcd}
\end{figure}
\end{lem}
\begin{proof}
Assign to every vertex $(i;j)$ of the diagram morphism $W(i;j)$
defined by the table~\ref{square-table}.

\begin{table}[h]
\caption{\label{square-table}}


\begin{tabular}{|c|c|c|}
\hline 
$i$ & $j$ & $W(i;j)$\tabularnewline
\hline 
\hline 
$0$ & $0$ & $1_{\Src f}$\tabularnewline
\hline 
$0$ & $1$ & $f$\tabularnewline
\hline 
$1$ & $0$ & $g\circ f$\tabularnewline
\hline 
$1$ & $1$ & $t^{-1}$\tabularnewline
\hline 
\end{tabular}
\end{table}


It is easy to verify by induction that the morphism corresponding
every cycle in the diagram starting at the vertex $(0;0)$ and ending
with a vertex $(x;y)$ is $W(x;y)$.

Thus the morphism corresponding to every cycle starting at the vertex
$(0;0)$ is identity.

By symmetry, the morphism corresponding to every cycle is identity.

That the diagram is commutative follows from it (because for paths~$\sigma$,~$\tau$
we have the paths $\sigma\circ\tau^{-1}$ and $\tau\circ\sigma^{-1}$
being identities).
\end{proof}

\section{\index{group theory}Intro to group theory}
\begin{defn}
\index{semigroup}A semigroup is a pair of a set $G$ and an associative
binary operation on $G$.
\end{defn}

\begin{defn}
\index{group}A group is a pair of a set $G$ and a binary operation
$\cdot$ on $G$ such that:
\begin{enumerate}
\item $(h\cdot g)\cdot f=h\cdot(g\cdot f)$ for every $f,g,h\in G$.
\item \index{identity}There exists an element $e$ (\emph{identity}) of
$G$ such that $f\cdot e=e\cdot f=f$ for every $f\in G$.
\item For every element $f$ there exists an element $f^{-1}$ such that
$f\cdot f^{-1}=f^{-1}\cdot f=e$.
\end{enumerate}
\end{defn}
\begin{obvious}
Every group is a semigroup.\end{obvious}
\begin{prop}
In every group there exist exactly one identity element.\end{prop}
\begin{proof}
If $p$ and $q$ are both identities, then $p=p\cdot q=q$.\end{proof}
\begin{prop}
Every group element has exactly one inverse.\end{prop}
\begin{proof}
Let $p$ and $q$ be both inverses of $f\in G$. Then $f\cdot p=p\cdot f=e$
and $f\cdot q=q\cdot f=e$. Then $p=p\cdot e=p\cdot f\cdot q=e\cdot q=q$.\end{proof}
\begin{prop}
$(g\cdot f)^{-1}=f^{-1}\cdot g^{-1}$ for every group elements $f$
and $g$.\end{prop}
\begin{proof}
$(f^{-1}\cdot g^{-1})\cdot(g\cdot f)=f^{-1}\cdot g^{-1}\cdot g\cdot f=f^{-1}\cdot e\cdot f=f^{-1}\cdot f=e$.
Similarly $(g\cdot f)\cdot(f^{-1}\cdot g^{-1})=e$. So $f^{-1}\cdot g^{-1}$
is the inverse of $g\cdot f$.\end{proof}
\begin{defn}
\index{group!permutation}A \emph{permutation group} on a set $D$
is a group whose elements are functions on $D$ and whose composition
is function composition.\end{defn}
\begin{obvious}
Elements of a permutation group are bijections.\end{obvious}
\begin{defn}
\index{group!permutation!transitive}A \emph{transitive} permutation
group on a set $D$ is such a permutation group $G$ on $D$ that
for every $x,y\in D$ there exists $r\in G$ such that $y=r(x)$.
\end{defn}
A groupoid with single (arbitrarily chosen) object corresponds to
every group. The morphisms of this category are elements of the group
and the composition of morphisms is the group operation.
