\chapter{Introduction}

I will show that most of the topology can be formulated in an ordered semigroup (or an ordered monoid).

For simplicity of notation I speak about semigroups, but all this can be easily generalized to precategories.

\chapter{Prerequisites}

You need to know about semigroups, ordered semigroups, semigroup actions, before reading further. If in doubt, consult Wikipedia.

\emph{Filtrators} are pairs of a poset and its subset (in induced order). An important example of filtrator is the set of filters on some poset together with the subset of principal filters. (Note that I order filters \emph{reversely} to the set inclusion relation: So for filters I have $a\leq b \Leftrightarrow a\supseteq b$.)

I will denote meet and join on a poset correspondingly as~$\sqcap$ and~$\sqcup$.

I call two elements~$a$ and~$b$ \emph{intersecting} ($a\nasymp b$) when there is a non-least element~$c$ such that $c\leq a\land c\leq b$. For meet-semigroups with meet operation~$\sqcap$ this condition is equivalent to $a\sqcap b$ being non-least element.

I call two elements~$a$ and~$b$ \emph{joining} ($a\equiv b$) when there is no non-greatest element~$c$ such that $c\geq a\land c\geq b$. For meet-semigroups with meet operation~$\sqcup$ this condition is equivalent to $a\sqcup b$ being the greatest element.

I denote $\rsupfun{f}X = \setcond{fx}{x\in X}$.

\chapter{Basic examples}

A topological space is determined by its closure operator.

Consider the semigroup formed by composing together any finite number of topological closure operators (on some fixed ``universal'' set).

This semigroup can be considered as its own action.

So every topological space is an element of this semigroup that is associated with an action.

The set, on which these actions act, is the set of subsets of our universal set. The set of subsets of a set is a partially ordered set.

So we have topological space defined by actions of an ordered semigroup.

Below I will define a \emph{space} as an \emph{ordered semigroup action}.

This includes topological spaces, uniform spaces, proximity spaces, (directed) graphs, metric spaces, vector spaces, etc.

Moreover we can consider the semigroup of all functions~$\subsets\mho\to\subsets\mho$ for some set~$\mho$ (the set of ``points'' of our space). Above we showed that topological spaces correspond to elements of this semigroup. Functions on~$\mho$ also can be considered as elements of this semigroup (replace every function with its ``image of a set'' function). Then we have an ordered semigroup action containing both topospaces and functions. As it was considered in~\cite{volume-1}, we can describe a function~$f$ being continuous from a space~$\mu$ to a space~$\nu$ by the formula $f\circ\mu\leq\nu\circ f$. See, it's an instance of \emph{algebraic} general topology: a topological concept was described by an algebraic formula, without any quantifiers.

\chapter{Precategories}

\begin{defn}
\index{precategory}A \emph{precategory} is a directed multigraph
together with a partial binary operation $\circ$ on the set $\mathcal{M}$
such that $g\circ f$ is defined iff $\Dst f=\Src g$ (for every morphisms
$f$ and $g$) such that
\begin{enumerate}
\item $\Src(g\circ f)=\Src f$ and $\Dst(g\circ f)=\Dst g$ whenever the
composition $g\circ f$ of morphisms $f$ and $g$ is defined.
\item $(h\circ g)\circ f=h\circ(g\circ f)$ whenever compositions in this
equation are defined.
\end{enumerate}
\end{defn}

\begin{defn}
\index{prefunctor}A \emph{prefunctor} is a pair of a function from the set of objects of one precategory to the set of objects of another precategory and a function from the set of morphisms of one precategory to the set of morphisms of another precategory (these functions are denoted by the same letter such as~$\phi$) conforming to the axioms:
\begin{enumerate}
\item $\phi(f): \phi(\Src f)\to\phi(\Dst f)$ for every morphism~$f$ of the first precategory;
\item $\phi(g\circ f)=\phi(g)\circ\phi(f)$ for every composable morphisms~$f$,~$g$ of the first precategory.
\end{enumerate}
\end{defn}

\begin{note}
A semigroup is essentially a special case of a precategory (with only one object) and semigroup homomorphism is a prefunctor.
\end{note}

\chapter{Ordered semigroups}

\begin{defn}
\emph{Ordered semigroup} (or \emph{posemigroup}) is a set together with binary operation~$\circ$ and binary relation~$\leq$ on it, conforming both to semigroup axioms and partial order axioms and:
\[ x_0\leq x_1\land y_0\leq y_1\Rightarrow y_0\circ x_0\leq y_1\circ x_1. \]
\end{defn}

In this book I will call elements of an ordered semigroup \emph{spaces}, because they generalize such things as topological spaces, (quasi)proximity spaces, (quasi)uniform spaces, (directed) graphs. (quasi)metric spaces.

Note that we consider functions as spaces, too. (See~\cite{volume-1} for an interpretation of both topological spaces and functions as funcoids, a kind of spaces.)

\begin{defn}
\emph{Ordered monoid} (or \emph{pomonoid}) is an ordered semigroup that is a monoid.
\end{defn}

\chapter{Ordered semigroup actions}

\begin{defn}
\emph{Curried ordered semigroup action} on a poset~$\mathfrak{A}$ for an ordered semigroup~$S$ is a function $T:S\to(\mathfrak{A}\to\mathfrak{A})$ such that
\begin{enumerate}
\item $(T(b\circ a))x = (Tb)(Ta)x$ for all $a,b\in S$, $x\in\mathfrak{A}$;
$x,y\in\mathfrak{A}$;
\item $a\leq b\Rightarrow(Ta)x\leq (Tb)x$ for all $a,b\in S$, $x\in\mathfrak{A}$;
\item $x\leq y\Rightarrow(Ta)x\leq (Ta)y$ for all $a\in S$. \end{enumerate}
\end{defn}

\begin{rem}
Google search for ``"ordered semigroup action"'' showed nothing. Was a spell laid onto Earth mathematicians not to find the most important structure in general topology?
\end{rem}

We can order actions componentwise. Then the above axioms simplify to:
\begin{enumerate}
\item $T(b\circ a) = (Tb)\circ(Ta)$ for all $a,b\in S$;
\item $T$ is a (not necessarily strictly) increasing;
\item $Ta$ is a (not necessarily strictly) increasing, for every space~$a$.
\end{enumerate}

\begin{defn}
A \emph{functional curried ordered semigroup action} is such an ordered semigroup action that $Ta=a$ for every space~$a$.
\end{defn}

\begin{thm}
Each curried ordered semigroup action induces a functional curried ordered semigroup action, whose elements are the same a of the original one, spaces are the curried actions of actions of the original semigroup, semigroup operation is function composition, and order of spaces is the product order.
\end{thm}

\begin{proof}
That it's a semigroup is obvious. The partial order is the same as the original. It remains to prove the remaining axioms.

For our semigroup
\[ T(b\circ a) = b\circ a=T(b)\circ T(a). \]

$T$ is increasing because it's the identity function.

$Ta$ is the same as one of the original ordered semigroup action and thus is increasing.
\end{proof}

Note that this ``inducting'' is an ordered semigroup homomorphism.

More generally, we can define a \emph{generalized space} as a homomorphism of ordered semigroups. Most of the below however does use poset elements, so it could not be described in terms of generalized spaces.

Having a curried ordered semigroup action and a homomorphism to its ordered semigroup, we can define in an obvious way a new curried ordered semigroup action. The following is an example of this construction (here $\torldin$ is a homomorphism of ordered semigroups).

\emph{Funcoids} described in the first volume of this book form an ordered semigroup with curried action $\supfun{}$. \emph{Reloids} form an ordered semigroup with curried action $a\mapsto\supfun{\torldin a}$.
As we know from the first volume, funcoids are a generalization of topological spaces, proximity spaces, and directed graphs (``discrete spaces''), reloids is a generalization of uniform spaces and directed graphs. Funcoid is determined by its action. So most of the customary general topology can be described in terms of ordered semigroup actions!

Remember that elements of our semigroup order may be such things as sets or more generally filters, they are not just points. So our topological construction is ``pointfree'' (we consider sets or filters, not points).

This book is mainly about this topic: describing general topology in terms of ordered semigroup actions. Above are the new axioms for general topology. No topological spaces here.

We will need also \emph{semigroups with involution}.

Ordered semigroup action is \emph{ordered by elements} when \[ a\leq b \Leftarrow Ta\leq Tb \] that is when \[ a\leq b \Leftarrow \forall x:(Ta)x\leq(Tb)x. \]

Obviously, in this case~$T$ is an injection. So our curried ordered semigroup action is \emph{essentially functional}.

\chapter{Ordered semigroups with involution}

\begin{defn}
\emph{Semigroup with involution} is a semigroup together with the operation $a\mapsto a^{\dagger}$ (\emph{involution}) such that:
\begin{enumerate}
\item $a^{\dagger\dagger} = a$;
\item $(b\circ a)^{\dagger} = a^{\dagger}\circ b^{\dagger}$.
\end{enumerate}
\end{defn}

For an \emph{ordered semigroup with involution} we will additionally require $a\leq b\Rightarrow a^{\dagger}\leq b^{\dagger}$ (and consequently $a\leq b\Leftrightarrow a^{\dagger}\leq b^{\dagger}$).

\chapter{Topological properties}

Now we have a formalism to describe many topological properties (following the ideas of~\cite{volume-1}):

Continuity is described by the formulas $f\circ a\leq a\circ f$, $f\circ a\circ f^{\dagger}\leq a$, $a\leq f^{\dagger}\circ a\circ f$.

Convergence of a function~$f$ from a space~$\mu$ to a space~$\nu$ at filter~$x$ to a set or filter~$y$ is described by the formula $(Tf)(T\mu)x\leq(T\nu)y$.

Generalized limit of an arbitrary space~$f$ (for example, of an arbitrary (possibly discontinuous) function), see \cite{limit}, is described by the formula \[ \xlim f=\setcond{\nu\circ f\circ r}{r\in G}, \]
where~$G$~is a group of spaces (consider for example the group of all translations of a vector space).

\emph{Neighborhood} of element~$x$ is such a $y$ that $(Ta)x\leq y$. \emph{Interior} of~$x$ (if it exists) if the join of all $y$ such that $x$ is a neighborhood of~$x$.

An element~$x$ is closed regarding~$a$ iff $(Ta)x\leq x$. $x$~is open iff $x$ is closed regarding $Ta^{\dagger}$.

To define compactness we additionally need the structure of filtrator $(\mathfrak{A},\mathfrak{Z})$ on our poset. Then it is space~$a$ is \emph{directly compact} iff
\[\forall x\in\mathfrak{A}:(x\text{ is non-least}\Rightarrow\Cor(Ta)x\text{ is non-least}); \]
$a$~is \emph{reversely compact} iff $a^{\dagger}$ is directly compact; $a$~is \emph{compact} iff it is both directly and reversely compact.

However, we can define compactness without specifying~$\mathfrak{Z}$ as we can take~$\mathfrak{Z}$ to be the \emph{center} (the set of all its complemented elements) of the poset~$\mathfrak{A}$.

It seem we cannot define \emph{total boundness} purely in terms of ordered semigroups, because it is a property of reloids and reloid is not determined by its action.

\chapter{A relation}

Every ordered semigroup action~$T$ defines a relation~$R$: $x(Ra)y\Leftrightarrow y\nasymp(Ta)x$.

If $Ra^{\dagger}=(Ra)^{-1}$ for every~$a$, we call the action~$T$ on an involutive semigroup \emph{inter\-sec\-tion-sym\-met\-ric}. In this case our action defines a pointfree funcoid~(see~\cite{volume-1}).

A space is connected iff $x\equiv y\Rightarrow x(Ra)y$.

\fxnote{Define ``other'' connectedness also through series.}

We can define open and closed functions.

\chapter{Further axioms}

Further possible axioms for an ordered semigroup action with binary joins:

\begin{itemize}
\item $(Tf)(x\sqcup y)=(Tf)x\sqcup(Tf)y$;
\item $(T(f\sqcup g))x=(Tf)x\sqcup(Tg)x$.
\end{itemize}

\fxnote{Need to generalize for a wider class of posets.}

\chapter{Product of ordered semigroup actions}

\fxnote{TODO.}

\chapter{Restricted identity transformations}

\emph{Restricted identity transformation} $\id_p$ is the (generally, partially defined) transformation $x\mapsto x\sqcap p$.

\begin{obvious}
$\id_q\circ\id_p = \id_{p\sqcap q}$.
\end{obvious}

\begin{prop}
$p\ne q\Rightarrow\id_p\ne\id_q$.
\end{prop}

\begin{proof}
$\id_p p=p \ne q = \id_q q$.
\end{proof}

\emph{Ordered semigroup action with identities} is an ordered semigroup~$S$ action~$T$ together with a function $p\mapsto\id_p\in S$ such that $T\id_p=\id_p$. (I abuse the notation $\id_p$ for both spaces and for transformations; this won't lead to inconsistencies, because as proved above this mapping is injective on restricted identities.)

\begin{obvious}
For every ordered semigroup action with identities, the identity transformations are entirely defined.
\end{obvious}

From injectivity it follows $\id_{p\sqcap q} = \id_p\circ\id_q$.

\emph{Restriction} of a space~$a$ to element~$x$ is $a|_x=a\circ\id_x$.

\chapter{Binary product of poset elements}

\begin{defn}
I call an ordered semigroup action \emph{correctly bounded} when the set of spaces is bounded and:
\begin{enumerate}
\item $(T\bot)x = \bot$ for every poset element~$x$;
\item $(T\top)x =
\left\{\begin{array}{ll}\top&\text{if }x\ne\bot,\\\bot&\text{if }x=\bot.\end{array}\right.$
\end{enumerate}
\end{defn}

\emph{Binary product} in an ordered semigroup action having a greatest element~$\top$ is defined as $p\times q=\id_q\circ\top\circ\id_p$.

\begin{thm}
If our action is correctly bounded, then
\[ x(R(p\times q))y\Leftrightarrow x\nasymp p\land y\nasymp q. \]
\end{thm}

\begin{proof}
\begin{multline*}
x(R(p\times q))y \Leftrightarrow
y\nasymp(T(p\times q))x \Leftrightarrow \\
y\nasymp(T(\id_q\circ\top\circ\id_p))x \Leftrightarrow \\
y\nasymp (T\id_q)(T\top)(T\id_p)x \Leftrightarrow \\
y\nasymp q\sqcap(T\top)(p\sqcap x) \Leftrightarrow \\
y\nasymp
\left\{\begin{array}{ll}q&\text{if }x\nasymp p,\\\bot&\text{if }x\asymp p.\end{array}\right. \Leftrightarrow \\
x\nasymp p\land y\nasymp q.
\end{multline*}
\end{proof}

\begin{thm}
$p_0\times q_0\nasymp p_1\times q_1 \Leftrightarrow p_0\nasymp p_1\land q_0\nasymp q_1$. \fxwarning{TODO.}
\end{thm}

\chapter{Separable spaces}

\emph{Hausdorff} space~$a$ can be defined as having $x\overline{Ra}y$ whenever~$x$ and $y$ are closed and $x\asymp y$.

\fxnote{Define other kinds of separability through hausdorffness and completion.}

\chapter{Distributive ordered semigroup actions}

We can define (pointwise) order of curried ordered semigroup actions. For functional curried ordered semigroup actions composition is defined. So we have one more ``level'' of ordered semigroups. By the way, it can be continued indefinitely building new and new levels of such ordered semigroups.

More generally we could consider ordered semigroup homomorphisms. Examples of such homomorphisms are $\supfun{}$, $\tofcd$, $\torldin$.

Pointfree funcoids (and consequently funcoids) are an ordered semigroup action. Reloids are also an ordered semigroup action.

\chapter{Complete spaces and completion of spaces}

A space~$a$ is \emph{complete} when $(Ta)\bigsqcup S=\bigsqcup\rsupfun{Ta}S$ whenever both $\bigsqcup S$ and $\bigsqcup\rsupfun{Ta}S$ are defined.

\begin{defn}
\emph{Completion} of a space is its core part (see~\cite{volume-1} for a definition of core part) on the filtrator of spaces and complete spaces.
\end{defn}

\begin{note}
Apparently, not every space has a completion.
\end{note}

\begin{note}
It is unrelated with Cachy-completion.
\end{note}

\chapter{Kuratowski spaces}

\begin{defn}
\emph{Kuratowski space} is a complete idempotent ($a\circ a=a$) space.
\end{defn}

Kuratowski spaces are a generalization of topological spaces.

\chapter{Metric spaces}

Let's denote $\Delta_a$ the proximity space induced by a metric space (or more generally quasimetric space)~$a$.

\begin{thm}
\[ \supfun{\Delta_a}X = \bigsqcap_{\epsilon>0}\bigcup_{x\in X}B(x,\epsilon) \]
($B(x,\epsilon)$ is the open ball of the radius~$\epsilon$ centered at~$x$).
\end{thm}

\begin{proof}
\begin{multline*}
Y\nasymp\supfun{\Delta_a}X\Leftrightarrow X\suprel{\Delta_a}Y \Leftrightarrow \\ \forall\epsilon>0\exists x\in X,y\in Y:\rho_a(x,y)<\epsilon.
\end{multline*}
\begin{multline*}
Y\nasymp\bigsqcap_{\epsilon>0}\bigcup_{x\in X}B_a(x,\epsilon) \Leftrightarrow \\ \forall\epsilon>0:Y\nasymp\bigcup_{x\in X}B_a(x,\epsilon) \Leftrightarrow \\
\forall\epsilon>0\exists x\in X:Y\nasymp B_a(x,\epsilon) \Leftrightarrow \\
\forall\epsilon>0\exists x\in X,y\in Y:\rho_a(x,y)<\epsilon.
\end{multline*}
\end{proof}


Metric spaces (on some fixed set~$\mho$) are also spaces: Define the order on metric spaces by the formula \[ \rho\leq\sigma \Leftrightarrow \forall x,y:\rho(x,y)\geq\sigma(x,y). \] Define the action for a metric space~$a$ as the action $\supfun{\Delta_a}$ of its induced proximity~$\Delta_a$ (see~\cite{volume-1} for a definition of proximity and more generally funcoid actions~$\supfun{}$) and composition of metrics $\rho$, $\sigma$ by the formula: \[ (\sigma\circ\rho)(x,z) = \inf_{y\in\mho}(\rho(x,y)+\sigma(z,y)), \]
where~$\mho$ is the set of points of our metric space.

\begin{lem}
$\Delta_{b\circ a} = \Delta_b\circ\Delta_a$.
\end{lem}

\begin{proof}
Let~$X$,~$Y$ be arbitrary sets on a metric space.
\begin{multline*}
Z\nasymp\supfun{\Delta_{b\circ a}}X \Leftrightarrow \\
\forall\epsilon>0\exists x\in X,z\in Z:\inf_{y\in\mho}(\rho_a(x,y)+\rho_b(y,z))<\epsilon \Leftrightarrow \\
\forall\epsilon>0\exists x\in X,y\in\mho,z\in Z:\rho_a(x,y)+\rho_b(y,z)<\epsilon \Leftrightarrow \\
\forall\epsilon>0\exists x\in X,y\in\mho,z\in Z:(\rho_a(x,y)<\epsilon\land\rho_b(y,z)<\epsilon)
\end{multline*}
\begin{multline*}
Z\nasymp\supfun{\Delta_b\circ\Delta_a}X \Leftrightarrow
Z\nasymp\supfun{\Delta_b}\supfun{\Delta_a}X\Leftrightarrow\\
\supfun{\Delta_b^{-1}}Z\nasymp\supfun{\Delta_a}X \Leftrightarrow\\
\bigsqcap_{\epsilon>0}\bigcup_{x\in X}B_a(x,\epsilon) \nasymp\bigsqcap_{\epsilon>0}\bigcup_{z\in Z}B_b(z,\epsilon) \Leftrightarrow \\
\forall\epsilon>0:\bigcup_{x\in X}B_a(x,\epsilon) \nasymp\bigcup_{z\in Z}B_b(z,\epsilon) \Leftrightarrow \\
\forall\epsilon>0\exists x\in X,z\in Z:B_a(x,\epsilon)\nasymp B_b(z,\epsilon) \Leftrightarrow \\
\forall\epsilon>0\exists x\in X,z\in Z,y\in\mho:(\rho_a(x,z)<\epsilon\land\rho_b(z,y)<\epsilon).
\end{multline*}

So, $Z\nasymp\supfun{\Delta_{b\circ a}}X \Leftrightarrow Z\nasymp\supfun{\Delta_b\circ\Delta_a}X$.
\end{proof}

Let's prove it's really an ordered semigroup action:

\begin{proof}
~
\begin{itemize}
\item It's a semigroup because
\begin{multline*}
(\tau\circ(\sigma\circ\rho))(x,z) = \\
\inf_{y_0\in\mho}((\sigma\circ\rho)(x,y_0)+\tau(y_0,z)) = \\
\inf_{y_0\in\mho}(\inf_{y_1\in\mho}(\rho(x,y_1)+\sigma(y_1,z))+\tau(y_0,z)) = \\
\inf_{y_0,y_1\in\mho}(\rho(x,y)+\sigma(y,z)+\tau(y,z)).
\end{multline*}
Similarly 
\begin{multline*}
((\tau\circ\sigma)\circ\rho)(x,z) = \\
\inf_{y\in\mho}(\rho(x,y)+\sigma(y,z)+\tau(y,z)).
\end{multline*}
Thus $\tau\circ(\sigma\circ\rho)=(\tau\circ\sigma)\circ\rho$.

\item It is an ordered semigroup, because
$(Ta)x = \supfun{\Delta_a} x \sqsubseteq \supfun{\Delta_a} y = (Ta)y$ 
for filters $x\sqsubseteq y$.

\item
\begin{multline*}
T(b\circ a) = \supfun{\Delta_{b\circ a}} = \\ \supfun{\Delta_b\circ\Delta_a} = \\ \supfun{\Delta_b}\circ\supfun{\Delta_a} = (Tb)\circ(Ta);
\end{multline*}
\item $a\leq b\Rightarrow Ta\leq Tb$ is obvious;
\item $x\leq y\Rightarrow(Ta)x\leq (Ta)y$ for all $a\in S$ is obvious.
\end{itemize}
\end{proof}

\fxnote{The above can be generalized for the values of the metric to be certain ordered additive semigroups instead of nonnegative real numbers.}

\fxnote{Isn't proximity also a XXX-metric, with points being sets or filters?}
