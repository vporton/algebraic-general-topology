
\chapter{Reloids}


\section{Basic definitions}
\begin{defn}
\index{reloid}I call a reloid from a set $A$ to a set $B$ a triple
$(A;B;F)$ where $F\in\mathscr{F}(A\times B)$.
\end{defn}

\begin{defn}
\index{reloid!source}\index{reloid!destination}\emph{Source} and
\emph{destination} of every reloid $(A;B;F)$ are defined as
\[
\Src(A;B;F)=A\quad\text{and}\quad\Dst(A;B;F)=B.
\]

\end{defn}
I will denote $\mathsf{RLD}(A;B)$ the set of reloids from $A$ to
$B$.

I will denote $\mathsf{RLD}$ the set of all reloids (for small sets).

\begin{defn}
\index{endo-reloid}I will call \emph{endoreloids} reloids with the
same source and destination.
\end{defn}

\begin{defn}
~
\begin{itemize}
\item $\uparrow^{\mathsf{RLD}}f=(\Src f;\Dst f;\uparrow^{\Src f\times\Dst f}\GR f)$
for every $\mathbf{Rel}$-morphism $f$.
\item $\uparrow^{\mathsf{RLD}(A;B)}f=\uparrow^{\mathsf{RLD}}(A;B;f)$ for
every binary relation $f\in\subsets(A\times B)$.
\end{itemize}
\end{defn}

\begin{defn}
\index{reloid!principal}I call members of a set $\rsupfun{\uparrow^{\mathsf{RLD}}}\mathbf{Rel}(A;B)$
as \emph{principal} reloids.
\end{defn}

Reloids are a generalization of uniform spaces. Also reloids are generalization
of binary relations.
\begin{defn}
\index{reloid!reverse}The \emph{reverse} reloid of a reloid is defined
by the formula
\[
(A;B;F)^{-1}=\left(B;A;\setcond{K^{-1}}{K\in F}\right).
\]
\end{defn}
\begin{note}
The reverse reloid is \emph{not} an inverse in the sense of group
theory or category theory.
\end{note}
Reverse reloid is a generalization of conjugate quasi-uniformity.
\begin{defn}
Every set $\mathsf{RLD}(A;B)$ is a poset by the formula $f\sqsubseteq g\Leftrightarrow\GR f\sqsubseteq\GR g$.
We will apply lattice operations to subsets of $\mathsf{RLD}(A;B)$
without explicitly mentioning $\mathsf{RLD}(A;B)$.\end{defn}

\emph{Filtrators of reloids} are $(\mathsf{RLD}(A;B);\mbox{\ensuremath{\mathbf{Rel}}}(A;B))$
(for all sets $A$, $B$). Here I equate principal reloids with corresponding $\mathbf{Rel}$-morphisms.

\begin{obvious}
The poset $\mathsf{RLD}(A;B)$ is isomorphic to the poset $\mathscr{F}(A\times B)$
for every sets $A$, $B$.
\end{obvious}

\section{Composition of reloids}
\begin{defn}
\index{composable!reloids}Reloids $f$ and $g$ are \emph{composable}
when $\Dst f=\Src g$.
\end{defn}

\begin{defn}
\index{composition!of reloids}\emph{Composition} of (composable)
reloids is defined by the formula
\[
g\circ f=\bigsqcap^{\mathsf{RLD}}\setcond{G\circ F}{F\in\up f,G\in\up g}.
\]
\end{defn}
\begin{obvious}
Composition of reloids is a reloid.
\end{obvious}

\begin{obvious}
$\uparrow^{\mathsf{RLD}}g\circ\uparrow^{\mathsf{RLD}}f=\uparrow^{\mathsf{RLD}}(g\circ f)$
for composable morphisms~$f$,~$g$ of category~$\mathbf{Rel}$.\end{obvious}
\begin{thm}
$(h\circ g)\circ f=h\circ(g\circ f)$ for every composable reloids
$f$, $g$, $h$.\end{thm}
\begin{proof}
For two nonempty collections~$A$ and~$B$ of sets I will denote
\[
A\sim B\Leftrightarrow\forall K\in A\exists L\in B:L\subseteq K\wedge\forall K\in B\exists L\in A:L\subseteq K.
\]
It is easy to see that $\sim$ is a transitive relation.

I will denote $B\circ A=\setcond{L\circ K}{K\in A,L\in B}$.

Let first prove that for every nonempty collections of relations $A$,
$B$, $C$
\[
A\sim B\Rightarrow A\circ C\sim B\circ C.
\]
Suppose $A\sim B$ and $P\in A\circ C$ that is $K\in A$ and $M\in C$
such that $P=K\circ M$. $\exists K'\in B:K'\subseteq K$ because
$A\sim B$. We have $P'=K'\circ M\in B\circ C$. Obviously $P'\subseteq P$.
So for every $P\in A\circ C$ there exists $P'\in B\circ C$ such
that $P'\subseteq P$; the vice versa is analogous. So $A\circ C\sim B\circ C$.

$\up((h\circ g)\circ f)\sim\up(h\circ g)\circ\up f$, $\up(h\circ g)\sim(\up h)\circ(\up g)$.
By proven above $\up((h\circ g)\circ f)\sim(\up h)\circ(\up g)\circ(\up f)$.

Analogously $\up(h\circ(g\circ f))\sim(\up h)\circ(\up g)\circ(\up f)$.

So $\up(h\circ(g\circ f))\sim\up((h\circ g)\circ f)$ what is possible
only if $\up(h\circ(g\circ f))=\up((h\circ g)\circ f)$. Thus $(h\circ g)\circ f=h\circ(g\circ f)$.\end{proof}
\begin{thm}
\label{rld-prod-ff}For every reloid $f$:
\begin{enumerate}
\item \label{rld-ff}$f\circ f=\bigsqcap^{\mathsf{RLD}}\setcond{F\circ F}{F\in\up f}$
if $\Src f=\Dst f$;
\item \label{rld-f1f}$f^{-1}\circ f=\bigsqcap^{\mathsf{RLD}}\setcond{F^{-1}\circ F}{F\in\up f}$;
\item \label{rld-ff1}$f\circ f^{-1}=\bigsqcap^{\mathsf{RLD}}\setcond{F\circ F^{-1}}{F\in\up f}$.
\end{enumerate}
\end{thm}
\begin{proof}
I will prove only \ref{rld-ff} and \ref{rld-f1f} because \ref{rld-ff1}
is analogous to \ref{rld-f1f}.
\begin{widedisorder}
\item [{\ref{rld-ff}}] It's enough to show that $\forall F,G\in\up f\exists H\in\up f:H\circ H\sqsubseteq G\circ F$.
To prove it take $H=F\sqcap G$.
\item [{\ref{rld-f1f}}] It's enough to show that $\forall F,G\in\up f\exists H\in\up f:H^{-1}\circ H\sqsubseteq G^{-1}\circ F$.
To prove it take $H=F\sqcap G$. Then $H^{-1}\circ H=(F\sqcap G)^{-1}\circ(F\sqcap G)\sqsubseteq G^{-1}\circ F$.
\end{widedisorder}
\end{proof}
\begin{thm}
For every sets $A$, $B$, $C$ if $g,h\in\mathsf{RLD}(A;B)$ then
\begin{enumerate}
\item $f\circ(g\sqcup h)=f\circ g\sqcup f\circ h$ for every $f\in\mathsf{RLD}(B;C)$;
\item $(g\sqcup h)\circ f=g\circ f\sqcup h\circ f$ for every $f\in\mathsf{RLD}(C;A)$.
\end{enumerate}
\end{thm}
\begin{proof}
We'll prove only the first as the second is dual.

By the infinite distributivity law for filters we have
\begin{align*}
f\circ g\sqcup f\circ h & =\\
\bigsqcap^{\mathsf{RLD}}\setcond{F\circ G}{F\in\up f,G\in\up g}\sqcup\bigsqcap^{\mathsf{RLD}}\setcond{F\circ H}{F\in\up f,H\in\up h} & =\\
\bigsqcap^{\mathsf{RLD}}\setcond{(F_{1}\circ G)\sqcup^{\mathsf{RLD}}(F_{2}\circ H)}{F_{1},F_{2}\in\up f,G\in\up g,H\in\up h} & =\\
\bigsqcap^{\mathsf{RLD}}\setcond{(F_{1}\circ G)\sqcup(F_{2}\circ H)}{F_{1},F_{2}\in\up f,G\in\up g,H\in\up h}.
\end{align*}


Obviously
\begin{align*}
\bigsqcap^{\mathsf{RLD}}\setcond{(F_{1}\circ G)\sqcup(F_{2}\circ H)}{F_{1},F_{2}\in\up f,G\in\up g,H\in\up h} & \sqsupseteq\\
\bigsqcap^{\mathsf{RLD}}\setcond{(((F_{1}\sqcap F_{2})\circ G)\sqcup((F_{1}\sqcap F_{2})\circ H))}{F_{1},F_{2}\in\up f,G\in\up g,H\in\up h} & =\\
\bigsqcap^{\mathsf{RLD}}\setcond{(F\circ G)\sqcup(F\circ H)}{F\in\up f,G\in\up g,H\in\up h} & =\\
\bigsqcap^{\mathsf{RLD}}\setcond{F\circ(G\sqcup H)}{F\in\up f,G\in\up g,H\in\up h}.
\end{align*}


Because $G\in\up g\land H\in\up h\Rightarrow G\sqcup H\in\up(g\sqcup h)$
we have
\begin{align*}
\bigsqcap^{\mathsf{RLD}}\setcond{F\circ(G\sqcup H)}{F\in\up f,G\in\up g,H\in\up h} & \sqsupseteq\\
\bigsqcap^{\mathsf{RLD}}\setcond{F\circ K}{F\in\up f,K\in\up(g\sqcup h)} & =\\
f\circ(g\sqcup h).
\end{align*}


Thus we have proved $f\circ g\sqcup f\circ h\sqsupseteq f\circ(g\sqcup h)$.
But obviously $f\circ(g\sqcup h)\sqsupseteq f\circ g$ and $f\circ(g\sqcup h)\sqsupseteq f\circ h$
and so $f\circ(g\sqcup h)\sqsupseteq f\circ g\sqcup f\circ h$. Thus
$f\circ(g\sqcup h)=f\circ g\sqcup f\circ h$.\end{proof}
\begin{thm}
\label{rld-cross}Let $A$, $B$, $C$ be sets, $f\in\mathsf{RLD}(A;B)$,
$g\in\mathsf{RLD}(B;C)$, $h\in\mathsf{RLD}(A;C)$. Then
\[
g\circ f\nasymp h\Leftrightarrow g\nasymp h\circ f^{-1}.
\]
\end{thm}
\begin{proof}
~
\begin{align*}
g\circ f\nasymp h & \Leftrightarrow\\
\bigsqcap^{\mathsf{RLD}}\setcond{G\circ F}{F\in\up f,G\in\up g}\sqcap\bigsqcap^{\mathsf{RLD}}\up h\ne\bot^{\mathsf{RLD}(A;C)} & \Leftrightarrow\\
\bigsqcap^{\mathsf{RLD}}\setcond{(G\circ F)\sqcap^{\mathsf{RLD}} H}{F\in\up f,G\in\up g,H\in\up h}\ne\bot^{\mathsf{RLD}(A;C)} & \Leftrightarrow\\
\bigsqcap^{\mathsf{RLD}}\setcond{(G\circ F)\sqcap H}{F\in\up f,G\in\up g,H\in\up h}\ne\bot^{\mathsf{RLD}(A;C)} & \Leftrightarrow\\
\forall F\in\up f,G\in\up g,H\in\up h:\uparrow^{\mathsf{RLD}}((G\circ F)\sqcap H)\ne\bot^{\mathsf{RLD}(A;C)} & \Leftrightarrow\\
\forall F\in\up f,G\in\up g,H\in\up h:G\circ F\nasymp H
\end{align*}
(used properties of generalized filter bases).

Similarly $g\nasymp h\circ f^{-1}\Leftrightarrow\forall F\in\up f,G\in\up g,H\in\up h:G\nasymp H\circ F^{-1}$.

Thus $g\circ f\nasymp h\Leftrightarrow g\nasymp h\circ f^{-1}$ because
$G\circ F\nasymp H\Leftrightarrow G\nasymp H\circ F^{-1}$ by proposition
\ref{rel-cross}.\end{proof}
\begin{thm}
For every composable reloids $f$ and $g$
\begin{enumerate}
\item $g\circ f=\bigsqcup\setcond{g\circ F}{F\in\atoms f}$.
\item $g\circ f=\bigsqcup\setcond{G\circ f}{G\in\atoms g}$.
\end{enumerate}
\end{thm}
\begin{proof}
We will prove only the first as the second is dual.
\end{proof}
Obviously $\bigsqcup\setcond{g\circ F}{F\in\atoms f}\sqsubseteq g\circ f$. We need to prove $\bigsqcup\setcond{g\circ F}{F\in\atoms f}\sqsupseteq g\circ f$.
Really,
\begin{align*}
\bigsqcup\setcond{g\circ F}{F\in\atoms f}\sqsupseteq g\circ f & \Leftrightarrow\\
\forall x\in\mathsf{RLD}(\Src f;\Dst g):\left(x\nasymp g\circ f\Rightarrow x\nasymp\bigsqcup\setcond{g\circ F}{F\in\atoms f}\right) & \Leftarrow\\
\forall x\in\mathsf{RLD}(\Src f;\Dst g):\left(x\nasymp g\circ f\Rightarrow \exists F\in\atoms f:x\nasymp g\circ F\right) & \Leftrightarrow\\
\forall x\in\mathsf{RLD}(\Src f;\Dst g):(g^{-1}\circ x\nasymp f\Rightarrow \exists F\in\atoms f:g^{-1}\circ x\nasymp F)
\end{align*}
what is obviously true.
\begin{cor}
\label{rld-comp-at}If $f$ and $g$ are composable reloids, then
\[
g\circ f=\bigsqcup\setcond{G\circ F}{F\in\atoms f,G\in\atoms g}.
\]
\end{cor}
\begin{proof}
$g\circ f=\bigsqcup_{F\in\atoms f}(g\circ F)=\bigsqcup_{F\in\atoms f}\bigsqcup_{G\in\atoms g}(G\circ F)=\bigsqcup\setcond{G\circ F}{F\in\atoms f,G\in\atoms g}$.
\end{proof}

\section{Reloidal product of filters}
\begin{defn}
\index{product!reloidal}Reloidal product of filters $\mathcal{A}$
and $\mathcal{B}$ is defined by the formula
\[
\mathcal{A}\times^{\mathsf{RLD}}\mathcal{B}\eqdef\bigsqcap^{\mathsf{RLD}}\setcond{A\times B}{A\in\up\mathcal{A},B\in\up\mathcal{B}}.
\]
\end{defn}
\begin{obvious}
~\end{obvious}
\begin{itemize}
\item $\uparrow^{U}A\times^{\mathsf{RLD}}\uparrow^{V}B=\uparrow^{\mathsf{RLD}(U;V)}(A\times B)$
for every sets $A\subseteq U$, $B\subseteq V$.
\item $\uparrow A\times^{\mathsf{RLD}}\uparrow B=\uparrow^{\mathsf{RLD}}(A\times B)$
for every typed sets~$A$,~$B$.\end{itemize}
\begin{thm}
\label{rld-prod-t-atoms}$\mathcal{A}\times^{\mathsf{RLD}}\mathcal{B}=\bigsqcup\setcond{a\times^{\mathsf{RLD}}b}{a\in\atoms\mathcal{A},b\in\atoms\mathcal{B}}$
for every filters $\mathcal{A}$ and $\mathcal{B}$.\end{thm}
\begin{proof}
Obviously $\mathcal{A}\times^{\mathsf{RLD}}\mathcal{B}\sqsupseteq\bigsqcup\setcond{a\times^{\mathsf{RLD}}b}{a\in\atoms\mathcal{A},b\in\atoms\mathcal{B}}$.

Reversely, let $K\in\up\bigsqcup\setcond{a\times^{\mathsf{RLD}}b}{a\in\atoms\mathcal{A},b\in\atoms\mathcal{B}}$.
Then $K\in\up(a\times^{\mathsf{RLD}}b)$ for every $a\in\atoms\mathcal{A}$,
$b\in\atoms\mathcal{B}$. $K\sqsupseteq X_{a}\times Y_{b}$ for some
$X_{a}\in\up a$, $Y_{b}\in\up b$; 
\[
K\sqsupseteq\bigsqcup\setcond{X_{a}\times Y_{b}}{a\in\atoms\mathcal{A},b\in\atoms\mathcal{B}}=\bigsqcup\setcond{X_{a}}{a\in\atoms\mathcal{A}}\times\bigsqcup\setcond{Y_{b}}{b\in\atoms\mathcal{B}}\sqsupseteq A\times B
\]
 where $A\in\up\mathcal{A}$, $B\in\up\mathcal{B}$; $K\in\up(\mathcal{A}\times^{\mathsf{RLD}}\mathcal{B})$.\end{proof}
\begin{thm}
If $\mathcal{A}_{0},\mathcal{A}_{1}\in\mathscr{F}(A)$, $\mathcal{B}_{0},\mathcal{B}_{1}\in\mathscr{F}(B)$
for some sets $A$, $B$ then
\[
\ensuremath{(\mathcal{A}_{0}\times^{\mathsf{RLD}}\mathcal{B}_{0})\sqcap(\mathcal{A}_{1}\times^{\mathsf{RLD}}\mathcal{B}_{1})=(\mathcal{A}_{0}\sqcap\mathcal{A}_{1})\times^{\mathsf{RLD}}(\mathcal{B}_{0}\sqcap\mathcal{B}_{1}).}
\]
\end{thm}
\begin{proof}
~
\begin{align*}
(\mathcal{A}_{0}\times^{\mathsf{RLD}}\mathcal{B}_{0})\sqcap(\mathcal{A}_{1}\times^{\mathsf{RLD}}\mathcal{B}_{1}) & =\\
\bigsqcap^{\mathsf{RLD}}\setcond{P\sqcap Q}{P\in\up(\mathcal{A}_{0}\times^{\mathsf{RLD}}\mathcal{B}_{0}),Q\in\up(\mathcal{A}_{1}\times^{\mathsf{RLD}}\mathcal{B}_{1})} & =\\
\bigsqcap^{\mathsf{RLD}}\setcond{(A_{0}\times B_{0})\sqcap(A_{1}\times B_{1})}{A_{0}\in\up\mathcal{A}_{0},B_{0}\in\up\mathcal{B}_{0},A_{1}\in\up\mathcal{A}_{1},B_{1}\in\up\mathcal{B}_{1}} & =\\
\bigsqcap^{\mathsf{RLD}}\setcond{(A_{0}\sqcap A_{1})\times(B_{0}\sqcap B_{1})}{A_{0}\in\up\mathcal{A}_{0},B_{0}\in\up\mathcal{B}_{0},A_{1}\in\up\mathcal{A}_{1},B_{1}\in\up\mathcal{B}_{1}} & =\\
\bigsqcap^{\mathsf{RLD}}\setcond{K\times L}{K\in\up(\mathcal{A}_{0}\sqcap\mathcal{A}_{1}),L\in\up(\mathcal{B}_{0}\sqcap\mathcal{B}_{1})} & =\\
(\mathcal{A}_{0}\sqcap\mathcal{A}_{1})\times^{\mathsf{RLD}}(\mathcal{B}_{0}\sqcap\mathcal{B}_{1}).
\end{align*}
\end{proof}
\begin{thm}
\label{meet-rld-prod}If $S\in\subsets(\mathscr{F}(A)\times\mathscr{F}(B))$
for some sets $A$, $B$ then
\[
\bigsqcap\setcond{\mathcal{A}\times^{\mathsf{RLD}}\mathcal{B}}{(\mathcal{A};\mathcal{B})\in S}=\bigsqcap\dom S\times^{\mathsf{RLD}}\bigsqcap\im S.
\]
\end{thm}
\begin{proof}
Let $\mathcal{P}=\bigsqcap\dom S$, $\mathcal{Q}=\bigsqcap\im S$;
$l=\bigsqcap\setcond{\mathcal{A}\times^{\mathsf{RLD}}\mathcal{B}}{(\mathcal{A};\mathcal{B})\in S}$.

$\mathcal{P}\times^{\mathsf{RLD}}\mathcal{Q}\sqsubseteq l$ is obvious.

Let $F\in\up(\mathcal{P}\times^{\mathsf{RLD}}\mathcal{Q})$. Then
there exist $P\in\up\mathcal{P}$ and $Q\in\up\mathcal{Q}$ such that
$F\sqsupseteq P\times Q$.

$P=P_{1}\sqcap\dots\sqcap P_{n}$ where $P_{i}\in\dom S$ and $Q=Q_{1}\sqcap\dots\sqcap Q_{m}$
where $Q_{j}\in\im S$.

$P\times Q=\bigsqcap_{i,j}(P_{i}\times Q_{j})$.

$P_{i}\times Q_{j}\in\up(\mathcal{A}\times^{\mathsf{RLD}}\mathcal{B})$
for some $(\mathcal{A};\mathcal{B})\in S$. $P\times Q=\bigsqcap_{i,j}(P_{i}\times Q_{j})\in\up l$.
So $F\in\up l$.\end{proof}
\begin{cor}
$\bigsqcap\rsupfun{\mathcal{A}\times^{\mathsf{RLD}}}T=\mathcal{A}\times^{\mathsf{RLD}}\bigsqcap T$
if $\mathcal{A}$ is a filter and $T$ is a set of filters with common
base.\end{cor}
\begin{proof}
Take $S=\{\mathcal{A}\}\times T$ where $T$ is a set of filters.

Then $\bigsqcap\setcond{\mathcal{A}\times^{\mathsf{RLD}}\mathcal{B}}{\mathcal{B}\in T}=\mathcal{A}\times^{\mathsf{RLD}}\bigsqcap T$
that is $\bigsqcap\rsupfun{\mathcal{A}\times^{\mathsf{RLD}}}T=\mathcal{A}\times^{\mathsf{RLD}}\bigsqcap T$.\end{proof}
\begin{defn}
\index{reloid!convex}I will call a reloid \emph{convex} iff it is
a join of direct products.
\end{defn}

\section{Restricting reloid to a filter. Domain and image}
\begin{defn}
\index{reloid!identity}\emph{Identity reloid} for a set $A$ is defined
by the formula $1_{A}^{\mathsf{RLD}}=\uparrow^{\mathsf{RLD}(A;A)}\id_{A}$.\end{defn}
\begin{obvious}
$(1_{A}^{\mathsf{RLD}})^{-1}=1_{A}^{\mathsf{RLD}}$.\end{obvious}
\begin{defn}
\index{restricting!reloid}I define \emph{restricting} a reloid $f$
to a filter $\mathcal{A}$ as $f|_{\mathcal{A}}=f\sqcap(\mathcal{A}\times^{\mathsf{RLD}}\top^{\mathscr{F}(\Dst f)})$.
\end{defn}

\begin{defn}
\index{reloid!domain}\index{reloid!image}\emph{Domain} and \emph{image}
of a reloid $f$ are defined as follows:
\[
\dom f=\bigsqcap^{\mathscr{F}}\rsupfun{\dom}\up f;\quad\im f=\bigsqcap^{\mathscr{F}}\rsupfun{\im}\up f.
\]
\end{defn}
\begin{prop}
$f\sqsubseteq\mathcal{A}\times^{\mathsf{RLD}}\mathcal{B}\Leftrightarrow\dom f\sqsubseteq\mathcal{A}\land\im f\sqsubseteq\mathcal{B}$
for every reloid $f$ and filters $\mathcal{A}\in\mathscr{F}(\Src f)$,
$\mathcal{B}\in\mathscr{F}(\Dst f)$.\end{prop}
\begin{proof}
~
\begin{description}
\item [{$\Rightarrow$}] It follows from $\dom(\mathcal{A}\times^{\mathsf{RLD}}\mathcal{B})\sqsubseteq\mathcal{A}\land\im(\mathcal{A}\times^{\mathsf{RLD}}\mathcal{B})\sqsubseteq\mathcal{B}$.
\item [{$\Leftarrow$}] $\dom f\sqsubseteq\mathcal{A}\Leftrightarrow\forall A\in\up\mathcal{A}\exists F\in\up f:\dom F\sqsubseteq A$.
Analogously


$\im f\sqsubseteq\mathcal{B}\Leftrightarrow\forall B\in\up\mathcal{B}\exists G\in\up f:\im G\sqsubseteq B$.


Let $\dom f\sqsubseteq\mathcal{A}\land\im f\sqsubseteq\mathcal{B}$,
$A\in\up\mathcal{A}$, $B\in\up\mathcal{B}$. Then there exist $F,G\in\up f$
such that $\dom F\sqsubseteq A\land\im G\sqsubseteq B$. Consequently
$F\sqcap G\in\up f$, $\dom(F\sqcap G)\sqsubseteq A$, $\im(F\sqcap G)\sqsubseteq B$
that is $F\sqcap G\sqsubseteq A\times B$. So there exists $H\in\up f$
such that $H\sqsubseteq A\times B$ for every $A\in\up\mathcal{A}$,
$B\in\up\mathcal{B}$. So $f\sqsubseteq\mathcal{A}\times^{\mathsf{RLD}}\mathcal{B}$.

\end{description}
\end{proof}
\begin{defn}
\index{restricted identity reloid}I call \emph{restricted identity
reloid} for a filter $\mathcal{A}$ the reloid
\[
\id_{\mathcal{A}}^{\mathsf{RLD}}=(1_{\Base(\mathcal{A})}^{\mathsf{RLD}})|_{\mathcal{A}}.
\]
\end{defn}
\begin{thm}
$\id_{\mathcal{A}}^{\mathsf{RLD}}=\bigsqcap_{A\in\up\mathcal{A}}^{\mathsf{RLD}(\Base(\mathcal{A});\Base(\mathcal{A}))}\id_{A}$
for every filter $\mathcal{A}$.\end{thm}
\begin{proof}
Let $K\in\up\bigsqcap_{A\in\up\mathcal{A}}^{\mathsf{RLD}(\Base(\mathcal{A});\Base(\mathcal{A}))}\id_{A}$,
then there exists $A\in\up\mathcal{A}$ such that $\GR K\supseteq\id_{A}$.
Then
\begin{align*}
\id_{\mathcal{A}}^{\mathsf{RLD}} & \sqsubseteq\\
\uparrow^{\mathsf{RLD}(\Base(\mathcal{A});\Base(\mathcal{A}))}\id_{\Base(\mathcal{A})}\sqcap(\mathcal{A}\times^{\mathsf{RLD}}\top^{\mathscr{F}(\Dst f)}) & \sqsubseteq\\
\uparrow^{\mathsf{RLD}(\Base(\mathcal{A});\Base(\mathcal{A}))}\id_{\Base(\mathcal{A})}\sqcap(A\times^{\mathsf{RLD}}\top^{\mathscr{F}(\Dst f)}) & =\\
\uparrow^{\mathsf{RLD}(\Base(\mathcal{A});\Base(\mathcal{A}))}\id_{\Base(\mathcal{A})}\sqcap\uparrow^{\mathsf{RLD}}(A\times\top^{\mathscr{T}(\Base(\mathcal{A}))}) & =\\
\uparrow^{\mathsf{RLD}(\Base(\mathcal{A});\Base(\mathcal{A}))}(\id_{\Base(\mathcal{A})}\cap\GR(A\times\top^{\mathscr{T}(\Base(\mathcal{A}))})) & =\\
\uparrow^{\mathsf{RLD}(\Base(\mathcal{A});\Base(\mathcal{A}))}\id_{A} & \sqsubseteq K.
\end{align*}
Thus $K\in\up\id_{\mathcal{A}}^{\mathsf{RLD}}$.

Reversely let $K\in\up\id_{\mathcal{A}}^{\mathsf{RLD}}=\up(1_{\Base(\mathcal{A})}^{\mathsf{RLD}}\sqcap(\mathcal{A}\times^{\mathsf{RLD}}\top^{\mathscr{F}(\Dst f)}))$,
then there exists $A\in\up\mathcal{A}$ such that 
\begin{align*}
K\in\up\uparrow^{\mathsf{RLD}(\Base(\mathcal{A});\Base(\mathcal{A}))}(\id_{\Base(\mathcal{A})}\cap\GR(A\times\top^{\mathscr{T}(\Base(\mathcal{A}))})) & =\\
\up\uparrow^{\mathsf{RLD}(\Base(\mathcal{A});\Base(\mathcal{A}))}\id_{A} & \sqsupseteq\\
\up\bigsqcap_{A\in\up\mathcal{A}}^{\mathsf{RLD}(\Base(\mathcal{A});\Base(\mathcal{A}))}\id_{A}.
\end{align*}
\end{proof}
\begin{cor}
$(\id_{\mathcal{A}}^{\mathsf{RLD}})^{-1}=\id_{\mathcal{A}}^{\mathsf{RLD}}$.\end{cor}
\begin{thm}
$f|_{\mathcal{A}}=f\circ\id_{\mathcal{A}}^{\mathsf{RLD}}$ for every
reloid $f$ and $\mathcal{A}\in\mathscr{F}(\Src f)$.\end{thm}
\begin{proof}
We need to prove that 
\[
f\sqcap(\mathcal{A}\times^{\mathsf{RLD}}\top^{\mathscr{F}(\Dst f)})=f\circ\bigsqcap^{\mathsf{RLD}(\Src f;\Src f)}\setcond{\id_{A}}{A\in\up\mathcal{A}}.
\]


We have
\begin{align*}
f\circ\bigsqcap^{\mathsf{RLD}(\Src f;\Src f)}\setcond{\id_{A}}{A\in\up\mathcal{A}} & =\\
\bigsqcap^{\mathsf{RLD}(\Src f;\Src f)}\setcond{\GR(F)\circ\id_{A}}{F\in\up f,A\in\up\mathcal{A}} & =\\
\bigsqcap^{\mathsf{RLD}}\setcond{F|_{A}}{F\in\up f,A\in\up\mathcal{A}} & =\\
\bigsqcap^{\mathsf{RLD}}\setcond{F\sqcap(A\times\top^{\mathscr{T}(\Dst f)})}{F\in\up f,A\in\up\mathcal{A}} & =\\
\bigsqcap^{\mathsf{RLD}}\setcond F{F\in\up f}\sqcap\bigsqcap^{\mathsf{RLD}}\setcond{A\times\top^{\mathscr{T}(\Dst f)}}{A\in\up\mathcal{A}} & =\\
f\sqcap(\mathcal{A}\times^{\mathsf{RLD}}\top^{\mathscr{F}(\Dst f)}).
\end{align*}
\end{proof}
\begin{thm}
$(g\circ f)|_{\mathcal{A}}=g\circ(f|_{\mathcal{A}})$ for every composable
reloids $f$ and $g$ and $\mathcal{A}\in\mathscr{F}(\Src f)$.\end{thm}
\begin{proof}
$(g\circ f)|_{\mathcal{A}}=(g\circ f)\circ\id_{\mathcal{A}}^{\mathsf{\mathsf{RLD}}}=g\circ(f\circ\id_{\mathcal{A}}^{\mathsf{\mathsf{RLD}}})=g\circ(f|_{\mathcal{A}})$.\end{proof}
\begin{thm}
$f\sqcap(\mathcal{A}\times^{\mathsf{RLD}}\mathcal{B})=\id_{\mathcal{B}}^{\mathsf{\mathsf{RLD}}}\circ f\circ\id_{\mathcal{A}}^{\mathsf{\mathsf{RLD}}}$
for every reloid $f$ and $\mathcal{A}\in\mathscr{F}(\Src f)$, $\mathcal{B}\in\mathscr{F}(\Dst f)$.\end{thm}
\begin{proof}
~
\begin{align*}
f\sqcap(\mathcal{A}\times^{\mathsf{RLD}}\mathcal{B}) & =\\
f\sqcap(\mathcal{A}\times^{\mathsf{RLD}}\top^{\mathscr{F}(\Dst f)})\sqcap(\top^{\mathscr{F}(\Src f)}\times^{\mathsf{RLD}}\mathcal{B}) & =\\
f|_{\mathcal{A}}\sqcap(\top^{\mathscr{F}(\Src f)}\times^{\mathsf{RLD}}\mathcal{B}) & =\\
(f\circ\id_{\mathcal{A}}^{\mathsf{\mathsf{RLD}}})\sqcap(\top^{\mathscr{F}(\Src f)}\times^{\mathsf{RLD}}\mathcal{B}) & =\\
((f\circ\id_{\mathcal{A}}^{\mathsf{\mathsf{RLD}}})^{-1}\sqcap(\top^{\mathscr{F}(\Src f)}\times^{\mathsf{RLD}}\mathcal{B})^{-1})^{-1} & =\\
((\id_{\mathcal{A}}^{\mathsf{\mathsf{RLD}}}\circ f^{-1})\sqcap(\mathcal{B}\times^{\mathsf{RLD}}\top^{\mathscr{F}(\Src f)}))^{-1} & =\\
(\id_{\mathcal{A}}^{\mathsf{\mathsf{RLD}}}\circ f^{-1}\circ\id_{\mathcal{B}}^{\mathsf{\mathsf{RLD}}})^{-1} & =\\
\id_{\mathcal{B}}^{\mathsf{\mathsf{RLD}}}\circ f\circ\id_{\mathcal{A}}^{\mathsf{\mathsf{RLD}}}.
\end{align*}
\end{proof}
\begin{thm}
$f|_{\uparrow\{\alpha\}}=\uparrow^{\Src f}\{\alpha\}\times^{\mathsf{RLD}}\im(f|_{\uparrow\{\alpha\}})$
for every reloid $f$ and $\alpha\in\Src f$.\end{thm}
\begin{proof}
First,
\begin{align*}
\im(f|_{\uparrow \{\alpha\}}) & =\\
\bigsqcap^{\mathsf{RLD}}\rsupfun{\im}\up(f|_{\uparrow \{\alpha\}}) & =\\
\bigsqcap^{\mathsf{RLD}}\rsupfun{\im}\up(f\sqcap(\uparrow^{\Src f}\{\alpha\}\times\top^{\mathscr{F}(\Dst f)})) & =\\
\bigsqcap^{\mathsf{RLD}}\setcond{\im(F\cap(\{\alpha\}\times\top^{\mathscr{T}(\Dst f)}))}{F\in\up f} & =\\
\bigsqcap^{\mathsf{RLD}}\setcond{\im(F|_{\uparrow \{\alpha\}})}{F\in\up f}.
\end{align*}


Taking this into account we have:
\begin{align*}
\uparrow^{\Src f}\{\alpha\}\times^{\mathsf{RLD}}\im(f|_{\uparrow \{\alpha\}}) & =\\
\bigsqcap^{\mathsf{RLD}}\setcond{\uparrow^{\Src f}\{\alpha\}\times K}{K\in\im(f|_{\uparrow \{\alpha\}})} & =\\
\bigsqcap^{\mathsf{RLD}}\setcond{\uparrow^{\Src f}\{\alpha\}\times\im(F|_{\uparrow \{\alpha\}})}{F\in\up f} & =\\
\bigsqcap^{\mathsf{RLD}}\setcond{F|_{\uparrow \{\alpha\}}}{F\in\up f} & =\\
\bigsqcap^{\mathsf{RLD}}\setcond{F\sqcap(\uparrow^{\Src f}\{\alpha\}\times\top^{\mathscr{T}(\Dst f)})}{F\in\up f} & =\\
\bigsqcap^{\mathsf{RLD}}\setcond F{F\in\up f}\sqcap\uparrow^{\mathsf{RLD}}(\uparrow^{\Src f}\{\alpha\}\times\top^{\mathscr{T}(\Dst f)}) & =\\
f\sqcap\uparrow^{\mathsf{RLD}}(\uparrow^{\Src f}\{\alpha\}\times\top^{\mathscr{T}(\Dst f)}) & =\\
f|_{\uparrow \{\alpha\}}.
\end{align*}
\end{proof}
\begin{lem}
$\mylamdba{\mathcal{B}}{\mathscr{F}(B)}{\top^{\mathscr{F}}\times^{\mathsf{RLD}}\mathcal{B}}$
is an upper adjoint of $\mylamdba f{\mathsf{RLD}(A;B)}{\im f}$ (for
every sets $A$, $B$).\end{lem}
\begin{proof}
We need to prove $\im f\sqsubseteq\mathcal{B}\Leftrightarrow f\sqsubseteq\top^{\mathscr{F}}\times^{\mathsf{RLD}}\mathcal{B}$
what is obvious.\end{proof}
\begin{cor}
\label{rld-dom-join}Image and domain of reloids preserve joins.\end{cor}
\begin{proof}
By properties of Galois connections and duality.
\end{proof}

\section{Categories of reloids}

I will define two categories, the \emph{category of reloids} and the
\emph{category of reloid triples}.

\index{category!of reloids}The \emph{category of reloids} is defined
as follows:
\begin{itemize}
\item Objects are small sets.
\item The set of morphisms from a set $A$ to a set $B$ is $\mathsf{RLD}(A;B)$.
\item The composition is the composition of reloids.
\item Identity morphism for a set is the identity reloid for that set.
\end{itemize}
To show it is really a category is trivial.

\index{category!of reloid triples}The \emph{category of reloid triples}
is defined as follows:
\begin{itemize}
\item Objects are small sets.
\item The morphisms from a filter $\mathcal{A}$ to a filter $\mathcal{B}$
are triples $(\mathcal{A};\mathcal{B};f)$ where $f\in\mathsf{RLD}(\Base(\mathcal{A});\Base(\mathcal{B}))$
and $\dom f\sqsubseteq\mathcal{A}$, $\im f\sqsubseteq\mathcal{B}$.
\item The composition is defined by the formula $(\mathcal{B};\mathcal{C};g)\circ(\mathcal{A};\mathcal{B};f)=(\mathcal{A};\mathcal{C};g\circ f)$.
\item Identity morphism for a filter $\mathcal{A}$ is $\id_{\mathcal{A}}^{\mathsf{\mathsf{RLD}}}$.
\end{itemize}
To prove that it is really a category is trivial.
\begin{prop}
$\uparrow^{\mathsf{RLD}}$ is a functor from $\mathbf{Rel}$ to $\mathsf{RLD}$.\end{prop}
\begin{proof}
$\uparrow^{\mathsf{RLD}}(g\circ f)=\uparrow^{\mathsf{RLD}}g\circ\uparrow^{\mathsf{RLD}}f$
was proved above. $\uparrow^{\mathsf{RLD}}1_{A}^{\mathbf{Rel}}=1_{A}^{\mathsf{RLD}}$
is by definition.
\end{proof}

\section{Monovalued and injective reloids}

\index{reloid!monovalued}\index{monovalued!reloid}Following the
idea of definition of monovalued morphism let's call \emph{monovalued}
such a reloid $f$ that $f\circ f^{-1}\sqsubseteq\id_{\im f}^{\mathsf{RLD}}$.

\index{reloid!injective}\index{injective!reloid}Similarly, I will
call a reloid \emph{injective} when $f^{-1}\circ f\sqsubseteq\id_{\dom f}^{\mathsf{RLD}}$.
\begin{obvious}
A reloid $f$ is
\begin{itemize}
\item monovalued iff $f\circ f^{-1}\sqsubseteq1_{\Dst f}^{\mathsf{RLD}}$;
\item injective iff $f^{-1}\circ f\sqsubseteq1_{\Src f}^{\mathsf{RLD}}$.
\end{itemize}
\end{obvious}
In other words, a reloid is monovalued (injective) when it is a monovalued
(injective) morphism of the category of reloids.

Monovaluedness is dual of injectivity.
\begin{obvious}
~
\begin{enumerate}
\item A morphism $(\mathcal{A};\mathcal{B};f)$ of the category of reloid
triples is monovalued iff the reloid $f$ is monovalued.
\item A morphism $(\mathcal{A};\mathcal{B};f)$ of the category of reloid
triples is injective iff the reloid $f$ is injective.
\end{enumerate}
\end{obvious}
\begin{thm}
~
\begin{enumerate}
\item \label{rld-mon-gr}A reloid $f$ is a monovalued iff there exists
a $\mathbf{Set}$-morphism (monovalued $\mathbf{Rel}$-morphism) $F\in\up f$.
\item \label{rld-mon-inj}A reloid $f$ is a injective iff there exists
an injective $\mathbf{Rel}$-morphism $F\in\up f$.
\item \label{rld-mon-both}A reloid $f$ is a both monovalued and injective
iff there exists an injection (a monovalued and injective $\mathbf{Rel}$-morphism
= injective $\mathbf{Set}$-morphism) $F\in\up f$.
\end{enumerate}
\end{thm}
\begin{proof}
The reverse implications are obvious. Let's prove the direct implications:
\begin{widedisorder}
\item [{\ref{rld-mon-gr}}] Let $f$ be a monovalued reloid. Then $f\circ f^{-1}\sqsubseteq1_{\Dst f}^{\mathsf{RLD}}$.
So there exists
\[
h\in\up(f\circ f^{-1})=\up\bigsqcap^{\mathsf{RLD}}\setcond{F\circ F^{-1}}{F\in\up f}
\]
such that $\uparrow^{\mathsf{RLD}}h\sqsubseteq1_{\Dst f}^{\mathsf{RLD}}$.
It's simple to show that $\setcond{F\circ F^{-1}}{F\in\up f}$ is
a filter base. Consequently there exists $F\in\up f$ such that $\GR(F\circ F^{-1})\subseteq\id_{\Dst f}$
that is $F$ is monovalued.
\item [{\ref{rld-mon-inj}}] Similar.
\item [{\ref{rld-mon-both}}] Let $f$ be a both monovalued and injective
reloid. Then by proved above there exist $F,G\in\up f$ such that
$F$ is monovalued and $G$ is injective. Thus $F\sqcap G\in\up f$
is both monovalued and injective.
\end{widedisorder}
\end{proof}
\begin{conjecture}
A reloid $f$ is monovalued iff
\[
\forall g\in\mathsf{\mathsf{RLD}}(\Src f;\Dst f):(g\sqsubseteq f\Rightarrow\exists\mathcal{A}\in\mathscr{F}(\Src f):g=f|_{\mathcal{A}}).
\]

\end{conjecture}

\section{Complete reloids and completion of reloids}
\begin{defn}
\index{reloid!complete}A \emph{complete} reloid is a reloid representable
as a join of reloidal products $\uparrow^{A}\{\alpha\}\times^{\mathsf{RLD}}b$
where $\alpha\in A$ and $b$ is an ultrafilter on $B$ for some sets
$A$ and $B$.
\end{defn}

\begin{defn}
\index{reloid!co-complete}A \emph{co-complete} reloid is a reloid
representable as a join of reloidal products $a\times^{\mathsf{RLD}}\uparrow^{A}\{\beta\}$
where $\beta\in B$ and $a$ is an ultrafilter on $A$ for some sets
$A$ and $B$.
\end{defn}
I will denote the sets of complete and co-complete reloids correspondingly
as $\mathsf{ComplRLD}$ and $\mathsf{CoComplRLD}$.
\begin{obvious}
Complete and co-complete are dual.\end{obvious}
\begin{thm}
\label{complrld-rep}$G\mapsto\bigsqcup\setcond{\uparrow^{A}\{\alpha\}\times^{\mathsf{RLD}}G(\alpha)}{\alpha\in A}$
is an order isomorphism from the set of functions $G\in\mathscr{F}(B)^{A}$
to the set $\Compl\mathsf{RLD}(A;B)$.

The inverse isomorphism is described by the formula $G(\alpha)=\im(f|_{\uparrow \{\alpha\}})$
where $f$ is a complete reloid.\end{thm}
\begin{proof}
$\bigsqcup\setcond{\uparrow^{A}\{\alpha\}\times^{\mathsf{RLD}}G(\alpha)}{\alpha\in A}$
is complete because $G(\alpha)=\bigsqcup\atoms G(\alpha)$ and thus
\[
\bigsqcup\setcond{\uparrow^{A}\{\alpha\}\times^{\mathsf{RLD}}G(\alpha)}{\alpha\in A}=\bigsqcup\setcond{\uparrow^{A}\{\alpha\}\times^{\mathsf{RLD}}b}{\alpha\in A,b\in\atoms G(\alpha)}
\]
is complete. So $G\mapsto\bigsqcup\setcond{\uparrow^{A}\{\alpha\}\times^{\mathsf{RLD}}G(\alpha)}{\alpha\in A}$
is a function from $G\in\mathscr{F}(B)^{A}$ to $\mathsf{ComplRLD}(A;B)$.

Let $f$ be complete. Then take
\[
G(\alpha)=\bigsqcup\setcond{b\in\atoms^{\mathscr{F}(\Dst f)}}{\uparrow^{A}\{\alpha\}\times^{\mathsf{RLD}}b\sqsubseteq f}
\]
and we have $f=\bigsqcup\setcond{\uparrow^{A}\{\alpha\}\times^{\mathsf{RLD}}G(\alpha)}{\alpha\in A}$
obviously. So $G\mapsto\bigsqcup\setcond{\uparrow^{A}\{\alpha\}\times^{\mathsf{RLD}}G(\alpha)}{\alpha\in A}$
is surjection to $\mathsf{ComplRLD}(A;B)$.

Let now prove that it is an injection:

Let
\[
f=\bigsqcup\setcond{\uparrow^{A}\{\alpha\}\times^{\mathsf{RLD}}F(\alpha)}{\alpha\in A}=\bigsqcup\setcond{\uparrow^{A}\{\alpha\}\times^{\mathsf{RLD}}G(\alpha)}{\alpha\in A}
\]
for some $F,G\in\mathscr{F}(B)^{A}$. We need to prove $F=G$. Let
$\beta\in\Src f$.
\begin{align*}
f\sqcap(\uparrow^{A}\{\beta\}\times^{\mathsf{RLD}}\top^{\mathscr{F}(B)}) & =\text{ (theorem~\ref{b-f-back-distr})}\\
\bigsqcup\setcond{(\uparrow^{A}\{\alpha\}\times^{\mathsf{RLD}}F(\alpha))\sqcap(\uparrow^{A}\{\beta\}\times^{\mathsf{RLD}}\top^{\mathscr{F}(B)})}{\alpha\in A} & =\\
\uparrow^{A}\{\beta\}\times^{\mathsf{RLD}}F(\beta).
\end{align*}
Similarly $f\sqcap(\uparrow^{A}\{\beta\}\times^{\mathsf{RLD}}\top^{\mathscr{F}(B)})=\uparrow^{A}\{\beta\}\times^{\mathsf{RLD}}G(\beta)$.
Thus $\uparrow^{A}\{\beta\}\times^{\mathsf{RLD}}F(\beta)=\uparrow^{A}\{\beta\}\times^{\mathsf{RLD}}G(\beta)$
and so $F(\beta)=G(\beta)$.

We have proved that it is a bijection. To show that it is monotone
is trivial.

Denote $f=\bigsqcup\setcond{\uparrow^{A}\{\alpha\}\times^{\mathsf{RLD}}G(\alpha)}{\alpha\in A}$.
Then
\begin{multline*}
\im(f|_{\uparrow \{\alpha'\}})=\im(f\sqcap(\uparrow^{A}\{\alpha'\}\times\top^{\mathscr{T}(B)}))=\text{(because \ensuremath{\uparrow^{A}\{\alpha'\}\times\top^{\mathscr{T}(B)}} is principal)}=\\
\im\bigsqcup\setcond{(\uparrow^{A}\{\alpha\}\times^{\mathsf{RLD}}G(\alpha))\sqcap(\uparrow^{A}\{\alpha'\}\times\top^{\mathscr{T}(B)})}{\alpha\in\Src f}=\im(\uparrow^{A}\{\alpha'\}\times^{\mathsf{RLD}}G(\alpha'))=G(\alpha').
\end{multline*}
\end{proof}
\begin{cor}
$G\mapsto\bigsqcup\setcond{G(\alpha)\times^{\mathsf{RLD}}\uparrow^{A}\{\alpha\}}{\alpha\in A}$
is an order isomorphism from the set of functions $G\in\mathscr{F}(B)^{A}$
to the set $\CoCompl\mathsf{RLD}(A;B)$.

The inverse isomorphism is described by the formula $G(\alpha)=\im(f^{-1}|_{\uparrow\{\alpha\}})$
where $f$ is a co-complete reloid.
\end{cor}

\begin{cor}
$\Compl\mathsf{RLD}(A;B)$ and $\Compl\mathsf{FCD}(A;B)$ are a co-frames.\end{cor}
\begin{obvious}
Complete and co-complete reloids are convex.
\end{obvious}

\begin{obvious}
Principal reloids are complete and co-complete.
\end{obvious}

\begin{obvious}
Join (on the lattice of reloids) of complete reloids is complete.\end{obvious}
\begin{thm}
A reloid which is both complete and co-complete is principal.\end{thm}
\begin{proof}
Let $f$ be a complete and co-complete reloid. We have
\[
f=\bigsqcup\setcond{\uparrow^{\Src f}\{\alpha\}\times^{\mathsf{RLD}}G(\alpha)}{\alpha\in\Src f}\quad\text{and}\quad f=\bigsqcup\setcond{H(\beta)\times^{\mathsf{RLD}}\uparrow^{\Dst f}\{\beta\}}{\beta\in\Dst f}
\]
for some functions $G:\Src f\rightarrow\mathscr{F}(\Dst f)$ and $H:\Dst f\rightarrow\mathscr{F}(\Src f)$.
For every $\alpha\in\Src f$ we have
\begin{align*}
G(\alpha) & =\\
\im f|_{\uparrow \{\alpha\}} & =\\
\im(f\sqcap(\uparrow^{\Src f}\{\alpha\}\times^{\mathsf{RLD}}\top^{\mathscr{F}(\Dst f)})) & =\text{ (*)}\\
\im\bigsqcup\setcond{(H(\beta)\times^{\mathsf{RLD}}\uparrow^{\Dst f}\{\beta\})\sqcap(\uparrow^{\Src f}\{\alpha\}\times^{\mathsf{RLD}}\top^{\mathscr{F}(\Dst f)})}{\beta\in\Dst f} & =\\
\im\bigsqcup\setcond{(H(\beta)\sqcap\uparrow^{\Src f}\{\alpha\})\times^{\mathsf{RLD}}\uparrow^{\Dst f}\{\beta\}}{\beta\in\Dst f} & =\\
\im\bigsqcup\setcond{\left(\begin{cases}
\uparrow^{\Src f}\{\alpha\}\times^{\mathsf{RLD}}\uparrow^{\Dst f}\{\beta\} & \text{if }\ensuremath{H(\beta)\nasymp\uparrow^{\Src f}\{\alpha\}}\\
\bot^{\mathsf{RLD}(\Src f;\Dst f)} & \text{if }\ensuremath{H(\beta)\asymp\uparrow^{\Src f}\{\alpha\}}
\end{cases}\right)}{\beta\in\Dst f} & =\\
\im\bigsqcup\setcond{\uparrow^{\Src f}\{\alpha\}\times^{\mathsf{RLD}}\uparrow^{\Dst f}\{\beta\}}{\beta\in\Dst f,H(\beta)\nasymp\uparrow^{\Src f}\{\alpha\}} & =\\
\im\bigsqcup\setcond{\uparrow^{\mathsf{RLD}(\Src f;\Dst f)}\{(\alpha;\beta)\}}{\beta\in\Dst f,H(\beta)\nasymp\uparrow^{\Src f}\{\alpha\}} & =\\
\bigsqcup\setcond{\uparrow^{\Dst f}\{\beta\}}{\beta\in\Dst f,H(\beta)\nasymp\uparrow^{\Src f}\{\alpha\}}
\end{align*}
{*} proposition (\ref{b-f-back-distr}) was used.

Thus $G(\alpha)$ is a principal filter that is $G(\alpha)=\uparrow^{\Dst f}g(\alpha)$
for some $g:\Src f\rightarrow\Dst f$; $\uparrow^{\Src f}\{\alpha\}\times^{\mathsf{RLD}}G(\alpha)=\uparrow^{\mathsf{RLD}(\Src f;\Dst f)}(\{\alpha\}\times g(\alpha))$;
$f$ is principal as a join of principal reloids.\end{proof}
\begin{defn}
\index{completion!of reloid}\index{co-completion!of reloid}\emph{Completion}
and \emph{co-completion} of a reloid $f\in\mathsf{RLD}(A;B)$ are
defined by the formulas:
\[
\Compl f=\Cor^{\mathsf{ComplRLD}(A;B)}f;\quad\CoCompl f=\Cor^{\mathsf{CoComplRLD}(A;B)}f.
\]
\end{defn}
\begin{thm}
Atoms of the lattice $\mathsf{ComplRLD}(A;B)$ are exactly reloidal
products of the form $\uparrow^{A}\{\alpha\}\times^{\mathsf{RLD}}b$
where $\alpha\in A$ and $b$ is an ultrafilter on $B$.\end{thm}
\begin{proof}
First, it's easy to see that $\uparrow^{A}\{\alpha\}\times^{\mathsf{RLD}}b$
are elements of $\mathsf{ComplRLD}(A;B)$. Also $\bot^{\mathsf{RLD}(A;B)}$
is an element of $\mathsf{ComplRLD}(A;B)$.

$\uparrow^{A}\{\alpha\}\times^{\mathsf{RLD}}b$ are atoms of $\mathsf{ComplRLD}(A;B)$
because they are atoms of $\mathsf{RLD}(A;B)$.

It remains to prove that if $f$ is an atom of $\mathsf{ComplRLD}(A;B)$
then $f=\uparrow^{A}\{\alpha\}\times^{\mathsf{RLD}}b$ for some $\alpha\in A$
and an ultrafilter $b$ on $B$.

Suppose $f$ is a non-empty complete reloid. Then $\uparrow^{A}\{\alpha\}\times^{\mathsf{RLD}}b\sqsubseteq f$
for some $\alpha\in A$ and an ultrafilter $b$ on $B$. If $f$ is
an atom then $f=\uparrow^{A}\{\alpha\}\times^{\mathsf{RLD}}b$.\end{proof}
\begin{obvious}
$\mathsf{ComplRLD}(A;B)$ is an atomistic lattice.\end{obvious}
\begin{prop}
$\Compl f=\bigsqcup\setcond{f|_{\uparrow \{\alpha\}}}{\alpha\in\Src f}$
for every reloid~$f$.\end{prop}
\begin{proof}
Let's denote $R$ the right part of the equality to be proven.

That $R$ is a complete reloid follows from the equality
\[
f|_{\uparrow \{\alpha\}}=\uparrow^{\Src f}\{\alpha\}\times^{\mathsf{RLD}}\im(f|_{\uparrow \{\alpha\}}).
\]


The only thing left to prove is that $g\sqsubseteq R$ for every complete
reloid $g$ such that $g\sqsubseteq f$.

Really let $g$ be a complete reloid such that $g\sqsubseteq f$.
Then
\[
g=\bigsqcup\setcond{\uparrow^{\Src f}\{\alpha\}\times^{\mathsf{RLD}}G(\alpha)}{\alpha\in\Src f}
\]
for some function $G:\Src f\rightarrow\mathscr{F}(\Dst f)$.

We have $\uparrow^{\Src f}\{\alpha\}\times^{\mathsf{RLD}}G(\alpha)=g|_{\uparrow^{\Src f}\{\alpha\}}\sqsubseteq f|_{\uparrow \{\alpha\}}$.
Thus $g\sqsubseteq R$.\end{proof}
\begin{conjecture}
$\Compl f\sqcap\Compl g=\Compl(f\sqcap g)$ for every $f,g\in\mathsf{RLD}(A;B)$.\end{conjecture}
\begin{thm}
$\Compl\bigsqcup R=\bigsqcup\rsupfun{\Compl}R$ for every set $R\in\subsets\mathsf{RLD}(A;B)$
for every sets $A$, $B$.\end{thm}
\begin{proof}
~
\begin{align*}
\Compl\bigsqcup R & =\\
\bigsqcup\setcond{\left(\bigsqcup R\right)|_{\uparrow^{A}\{\alpha\}}}{\alpha\in A} & =\text{ (proposition\,\ref{b-f-back-distr})}\\
\bigsqcup\setcond{\bigsqcup\setcond{f|_{\uparrow \{\alpha\}}}{\alpha\in A}}{f\in R} & =\\
\bigsqcup\rsupfun{\Compl}R.
\end{align*}
\end{proof}
\begin{lem}
Completion of a co-complete reloid is principal.\end{lem}
\begin{proof}
Let $f$ be a co-complete reloid. Then there is a function $F:\Dst f\rightarrow\mathscr{F}(\Src f)$
such that
\[
f=\bigsqcup\setcond{F(\alpha)\times^{\mathsf{RLD}}\uparrow^{\Dst f}\{\alpha\}}{\alpha\in\Dst f}.
\]


So
\begin{align*}
\Compl f & =\\
\bigsqcup\setcond{\left(\bigsqcup\setcond{F(\alpha)\times^{\mathsf{RLD}}\uparrow^{\Dst f}\{\alpha\}}{\alpha\in\Dst f}\right)|_{\uparrow\{\beta\}}}{\beta\in\Src f} & =\\
\bigsqcup\setcond{\left(\bigsqcup\setcond{F(\alpha)\times^{\mathsf{RLD}}\uparrow^{\Dst f}\{\alpha\}}{\alpha\in\Dst f}\right)\sqcap(\uparrow^{\Src f}\{\beta\}\times^{\mathsf{RLD}}\top^{\mathscr{F}(\Dst f)})}{\beta\in\Src f} & =\text{ (*)}\\
\bigsqcup\setcond{\bigsqcup\setcond{(F(\alpha)\times^{\mathsf{RLD}}\uparrow^{\Dst f}\{\alpha\})\sqcap(\uparrow^{\Src f}\{\beta\}\times^{\mathsf{RLD}}\top^{\mathscr{F}(\Dst f)})}{\alpha\in\Dst f}}{\beta\in\Src f} & =\\
\bigsqcup\setcond{\bigsqcup\setcond{\uparrow^{\Src f}\{\beta\}\times^{\mathsf{RLD}}\uparrow^{\Dst f}\{\alpha\}}{\alpha\in\Dst f}}{\beta\in\Src f,\uparrow^{\Src f}\{\beta\}\sqsubseteq F(\alpha)}
\end{align*}
{*} proposition \ref{b-f-back-distr}.

Thus $\Compl f$ is principal.\end{proof}
\begin{thm}
$\Compl\CoCompl f=\CoCompl\Compl f=\Cor f$ for every reloid~$f$.\end{thm}
\begin{proof}
We will prove only $\Compl\CoCompl f=\Cor f$. The rest follows from
symmetry.

From the lemma $\Compl\CoCompl f$ is principal. It is obvious $\Compl\CoCompl f\sqsubseteq f$.
So to finish the proof we need to show only that for every principal
reloid $F\sqsubseteq f$ we have $F\sqsubseteq\Compl\CoCompl f$.

Really, obviously $F\sqsubseteq\CoCompl f$ and thus $F=\Compl F\sqsubseteq\Compl\CoCompl f$.\end{proof}
\begin{conjecture}
If $f$ is a complete reloid, then it is metacomplete.
\end{conjecture}

\begin{conjecture}
If $f$ is a metacomplete reloid, then it is complete.
\end{conjecture}

\begin{conjecture}
$\Compl f=f\psetminus(\Omega^{\Src f}\times^{\mathsf{RLD}}\top^{\mathscr{F}(\Dst f)})$
for every reloid~$f$.
\end{conjecture}
By analogy with similar properties of funcoids described above:
\begin{prop}
For composable reloids $f$ and $g$ it holds
\begin{enumerate}
\item $\Compl(g\circ f)\sqsupseteq(\Compl g)\circ(\Compl f)$
\item $\CoCompl(g\circ f)\sqsupseteq(\CoCompl g)\circ(\CoCompl f)$.
\end{enumerate}
\end{prop}
\begin{proof}
~
\begin{enumerate}
\item $(\Compl g)\circ(\Compl f)\sqsubseteq\Compl((\Compl g)\circ(\Compl f))\sqsubseteq\Compl(g\circ f)$.
\item By duality.
\end{enumerate}
\end{proof}
\begin{conjecture}
For composable reloids $f$ and $g$ it holds
\begin{enumerate}
\item $\Compl(g\circ f)=(\Compl g)\circ f$ if $f$ is a co-complete reloid;
\item $\CoCompl(f\circ g)=f\circ\CoCompl g$ if $f$ is a complete reloid;
\item $\CoCompl((\Compl g)\circ f)=\Compl(g\circ(\CoCompl f))=(\Compl g)\circ(\CoCompl f)$;
\item $\Compl(g\circ(\Compl f))=\Compl(g\circ f)$;
\item $\CoCompl((\CoCompl g)\circ f)=\CoCompl(g\circ f)$.
\end{enumerate}
\end{conjecture}

\section{What uniform spaces are}
\begin{prop}
\index{uniformity}\index{space!uniform}Uniform spaces are exactly
reflexive, symmetric, transfinite endoreloids.\end{prop}
\begin{proof}
Easy to prove using theorem~\ref{rld-prod-ff}.\end{proof}

