\chapter{On ``Each regular paratopological group is completely regular'' article}

In this chapter I attempt to rewrite the paper~\cite{2014arXiv1410.1504B} in more general setting of funcoids and reloids.
I attempt to construct a ``royal road'' to finding proofs of statements of this paper and similar ones, what is
important because we lose 60 years waiting for any proof.

\section{Definition of normality}

\fxwarning{This section is in the middle of a rewrite.}

By definition (slightly generalizing the special case if $f$ is a
quasi-uniform space) an endo-reloid $f$ on a set $U$ is \emph{normal} when
\[ \rsupfun{\CoCompl (\mathsf{FCD}) f} A \sqsubseteq \rsupfun{
(\CoCompl (\mathsf{FCD}) f)^{\circ}} \rsupfun{\CoCompl (\mathsf{FCD})
f } \rsupfun{F} A \] for every entourage $F \in
\up f$ of $f$ and every set $A \subseteq U$.

Note that this is \emph{not} the same as customary definition of normal topological spaces.

\begin{thm}
  An endoreloid $f$ is normal iff
  \[ \Compl (\mathsf{FCD}) f^{- 1} \circ \CoCompl (\mathsf{FCD}) f \sqsubseteq
  \CoCompl (\mathsf{FCD}) f \circ (\mathsf{FCD}) f. \]
\end{thm}

\begin{proof}
  Equivalently transforming the criterion of normality (which should hold for
  all $F \in \up f$) using proposition~\ref{get-rid-interior}:

  $\rsupfun{(\CoCompl (\mathsf{FCD}) f)^{- 1}}
  \rsupfun{\CoCompl (\mathsf{FCD}) f} A \sqsubseteq
  \rsupfun{\CoCompl (\mathsf{FCD}) f} \rsupfun{F} A$.

  $\rsupfun{\Compl (\mathsf{FCD}) f^{- 1}}
  \rsupfun{\CoCompl (\mathsf{FCD}) f} A \sqsubseteq
  \rsupfun{\CoCompl (\mathsf{FCD}) f} \rsupfun{F} A$.

  Also note
  
  $\bigsqcap_{F \in \up f} \rsupfun{ \CoCompl (\mathsf{FCD}) f
  } \langle F \rangle^{\ast} A = \text{(because funcoids preserve
  filtered meets)} = \rsupfun{ \CoCompl (\mathsf{FCD}) f
  }  \bigsqcap_{F \in \up f} \rsupfun{F} A =
  \rsupfun{ \CoCompl (\mathsf{FCD}) f }
  \rsupfun{ (\mathsf{FCD}) f } A$.

  Thus the above is equivalent to
  $\rsupfun{\Compl (\mathsf{FCD}) f^{- 1}}
  \rsupfun{\CoCompl (\mathsf{FCD}) f} A \sqsubseteq
  \rsupfun{ \CoCompl (\mathsf{FCD}) f }
  \rsupfun{ (\mathsf{FCD}) f } A$.

  And this is in turn equivalent to
  \[ \Compl (\mathsf{FCD}) f^{- 1} \circ \CoCompl (\mathsf{FCD}) f \sqsubseteq
  \CoCompl (\mathsf{FCD}) f \circ (\mathsf{FCD}) f. \]
\end{proof}
