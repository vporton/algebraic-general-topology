\chapter{On ``Each regular paratopological group is completely regular'' article}

In this chapter I attempt to rewrite the paper~\cite{2014arXiv1410.1504B} in more general setting of funcoids and reloids.
I attempt to construct a ``royal road'' to finding proofs of statements of this paper and similar ones, what is
important because we lose 60 years waiting for any proof.

\section{Definition of normality}

By definition (slightly generalizing the special case if $f$ is a
quasi-uniform space) a pair of an endo-reloid~$f$ and a co-complete funcoid~$g$ (playing role of a generalization of a topological space)
on a set $U$ is \emph{normal} when
\[ \rsupfun{g} A \sqsubseteq \rsupfun{
g^{\circ}} \rsupfun{g } \rsupfun{F} A \] for every entourage $F \in
\up f$ of $f$ and every set $A \subseteq U$.

Note that this is \emph{not} the same as customary definition of normal topological spaces.

\begin{thm}
  An endoreloid $f$ is normal iff
  \[ g^{- 1} \circ g \sqsubseteq
  g \circ (\mathsf{FCD}) f. \]
\end{thm}

\begin{proof}
  Equivalently transforming the criterion of normality (which should hold for
  all $F \in \up f$) using proposition~\ref{get-rid-interior}:

  $\rsupfun{g^{- 1}}
  \rsupfun{g} A \sqsubseteq
  \rsupfun{g} \rsupfun{F} A$.

  Also note
  
  $\bigsqcap^{\mathscr{F}}_{F \in \up f} \rsupfun{ g
  } \langle F \rangle^{\ast} A = \text{(because funcoids preserve
  filtered meets)} = \rsupfun{ g
  }  \bigsqcap^{\mathscr{F}}_{F \in \up f} \rsupfun{F} A =
  \rsupfun{ g }
  \rsupfun{ (\mathsf{FCD}) f } A$.

  Thus the above is equivalent to
  $\rsupfun{g^{-1}}
  \rsupfun{g} A \sqsubseteq
  \rsupfun{ g }
  \rsupfun{ (\mathsf{FCD}) f } A$.

  And this is in turn equivalent to
  \[ g^{-1} \circ g \sqsubseteq
  g \circ (\mathsf{FCD}) f. \]
\end{proof}

\begin{cor}
If $g$ is a symmetric endoreloid and $\tofcd f\sqsupseteq g$, then it is normal.
\fxnote{Not quite the same as \noun{Banakh}'s theorem, because the induced topological space is $\CoCompl\tofcd f$ not $\tofcd f$.}
\end{cor}

\section{Urysohn's lemma and friends}

For a detailed proof of Urysohn's lemma see also:\\
\url{http://homepage.math.uiowa.edu/~jsimon/COURSES/M132Fall07/UrysohnLemma_v5.pdf}\\
\url{https://proofwiki.org/wiki/Urysohn's_Lemma}\\
\url{http://planetmath.org/proofofurysohnslemma}

Below follows an alternative proof of Urysohn lemma. Warning: This proof is conditional,
based on unproved conjecture~\bookref{fcd-comp-ent}.

\begin{lem}
  (assuming conjecture~\bookref{fcd-comp-ent}) For every $U \in \up \mu$ (where $\mu$ is a $T_4$ topological space) such that
  $\neg \left( A \rsuprel{U \circ U^{- 1}} B \right)$ there is $W \in
  \up \mu$ such that $U \circ U^{- 1} \sqsupseteq W \circ W^{- 1}
  \circ W \circ W^{- 1}$. For it holds $\neg \left( A \rsuprel{W \circ W^{-
  1}} B \right)$.
\end{lem}

\begin{proof}
  $U \circ U^{- 1} \in \up (\mu \circ \mu^{- 1}) \subseteq
  \up (\mu \circ \mu^{- 1} \circ \mu \circ
  \mu^{- 1})$ (normality used). Thus by the conjecture there exists $W
  \in \up \mu$ such that $U \circ U^{- 1} \sqsupseteq W \circ W^{-
  1} \circ W \circ W^{- 1}$. $W \circ W^{- 1} \sqsubseteq U \circ U^{- 1}$
  thus $\neg \left( A \rsuprel{W \circ W^{- 1}} B \right)$.
\end{proof}

A modified proof of Urysohn's lemma follows. This proof is in part based on~\cite{2014arXiv1410.1504B}.
(I attempt to find common generalization of Urysohn's lemma and results from~\cite{2014arXiv1410.1504B}).

\begin{thm}
Urysohn's lemma (see Wikipedia) for disjoint closed sets~$A$ and~$B$ and function~$f$ on a topological space~$\mu$
(considered as complete funcoid). \fxwarning{The proof was not thoroughly checked for errors.}
\end{thm}

\begin{proof}
(assuming conjecture~\bookref{fcd-comp-ent}) Because $A$ and $B$ are disjoint closed sets, we
have $\rsupfun{\mu} A \asymp \langle \mu
\rangle^{\ast} B$. Thus by the corollary of the lemma take $S_0 \in \up
\mu$ and $\neg \left( A \rsuprel{S_0 \circ S_0^{- 1}} B
\right)$.

We have $\mu \circ \mu^{- 1} \circ \mu \circ \mu^{- 1}
\sqsubseteq \mu \circ \mu^{- 1}$ that is $\up (\mu
\circ \mu^{- 1} \circ \mu \circ \mu^{- 1}) \supseteq
\up (\mu \circ \mu^{- 1})$.

Let's prove by induction: There is a sequence $S$ of binary relations starting
with $S_0$ such that $\neg \left( A \rsuprel{S_i \circ S_i^{- 1}} B
\right)$ and $S_i \circ S_i^{- 1} \sqsupseteq S_{i + 1} \circ S_{i + 1}^{- 1}
\circ S_{i + 1} \circ S_{i + 1}^{- 1}$. It directly follows from the lemma
(and uses the conjecture).

Denote $U_i = S_{i + 1} \circ S_{i + 1}^{- 1}$. We have $U_i \supseteq U_{i +
1} \circ U_{i + 1}$ and $\neg \left( A \mathrel{[U_i]^{\ast}} B \right)$.

Define fractional degree of $U$: $U^r \eqdef U_1^{r_1} \circ
\ldots \circ U_{l_r}^{r_{l_r}}$ for every $r \in \mathbb{Q}_2$ where $r_1
\ldots r_{l_r}$ is the binary expansion of $r$.
\[ f (z) \eqdef \inf \{ 1 \} \cup \setcond{ q \in
   \mathbb{Q}_2 }{ z \in \rsupfun{U^q}
   A } . \]
$f$ is properly defined because $\{ 1 \} \cup \setcond{ q \in \mathbb{Q}_2
}{ z \in \rsupfun{U^q} A }$ is
nonempty and bounded.

If $z \in A$ then $z \in \rsupfun{U^q} A$ for every $q \in
\mathbb{Q}_2$, thus $f (z) = 0$, because obviously $U^q \sqsupseteq 1$.

If $z \in B$ then $z \notin \rsupfun{U^q} A$ for every $q \in
\mathbb{Q}_2$, thus $f (z) = 1$, because $U^q \sqsubseteq U_0$ (??prove by
induction).

It remains to prove that $f$ is continuous.

Let $D (x) = \{ 1 \} \cup \setcond{ q \in \mathbb{Q}_2 }{
z \in \rsupfun{U^q} A }$.

Claim A: $f (x) > q \Rightarrow x \notin \langle \mu^{- 1}
\rangle^{\ast} \rsupfun{U^q} A$

Claim B: $f (x) < q \Rightarrow x \in \rsupfun{U^q} A$

Proof of claim A: If $f (x) > q$ then then there must be some gap between $q$
and $D (x)$; in particular, there exists some $q'$ such that $q < q' < f (x)$.
But $q' < f (x) \Rightarrow x \notin \rsupfun{U^q} A \Rightarrow x
\notin \langle f^{- 1} \rangle^{\ast} \rsupfun{U^q} A$.

Proof of claim B: If $f (x) < q$ then there exists $q' \in D (x)$ such that $f
(x) < q' < q$, in which case $q \in D (x)$, so $x \in \langle U^q
\rangle^{\ast} A$.

To show that $f$ is continuous, it's enough to prove that preimages of $] a ;
1]$ and $[0 ; a [$ are open.

\fxnote{This argument is limited to topological spaces!}
\fxnote{TODO: For arbitrary pretopologies try to use ``For all neighborhoods $B$ of $f(x)$, $f^(-1)B$ is a neighborhood of $x$.''}
\fxnote{However, for both Urysohn lemma and Taras Banakh's result topological spaces suffice.}

Suppose $f (x) \in] a ; 1]$. Pick some $q$ with $a < q < f (x)$. We claim that
the open set $W = X \setminus \langle f^{- 1} \rangle^{\ast} \langle U^q
\rangle^{\ast} A$ is a neighborhood of $x$ that is mapped by $f$ into $] a ;
1]$. First, by (A), $f (x) > q \Rightarrow x \in W$, so $W$ is a neighborhood
of $x$. If $y$ is any point of $W$, then $f (y)$ must be $\geq q > a$;
otherwise, if $f (y) < q$, then, by (B) $y \in \rsupfun{U^q} A
\subseteq \langle f^{- 1} \rangle^{\ast} \rsupfun{U^q} A$.

Suppose $x \in f^{- 1} [0 ; b [$ that is $f (x) < b$ and pick $q$ such that $f
(x) < q < b$. By (B) $x \in \rsupfun{U^q} A$. We claim that the
neighborhood $\rsupfun{U^q} A$ is mapped by $f$ into $[0 ; b [$.
Suppose $y$ is any point of $\rsupfun{U^q} A$. Then $q \in D
(y)$, so $f (y) \leq q < b$.
\end{proof}