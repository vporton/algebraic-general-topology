\chapter{Compact funcoids}

Compact funcoids are defined. Attempted to prove that under certain conditions the
reloid corresponding to a compact funcoid is the neighborhood of the diagonal
of the product funcoid.

This is a rough partial draft. The proofs are with errors.

\fxnote{The below examples also show that subatomic product does not coincide with Tychonoff product.}

\section{The rest}

\begin{defn}
  A funcoid $f$ is \emph{directly compact} iff
  \[ \forall \mathcal{F} \in \mathfrak{F}: (\supfun{f}
     \mathcal{F} \neq \bot \Rightarrow \Cor \supfun{f} \mathcal{F}
     \neq \bot). \]
\end{defn}

\begin{obvious}
A funcoid $f$ is directly compact iff $\forall a \in \atoms \dom f :
\Cor \supfun{f} a \neq \bot$.
\end{obvious}

\begin{defn}
  A funcoid $f$ is \emph{reversely compact} iff $f^{- 1}$ is directly
  compact.
\end{defn}

\begin{defn}
  A funcoid is \emph{compact} iff it is both directly compact and reversely
  compact.
\end{defn}

\begin{prop}
  $\prod^{\mathsf{RLD}} a = \uparrow^{\mathsf{RLD}} \prod_{i \in
  \dom a} (\uparrow^{\mathsf{RLD}})^{- 1} a_i$ for every indexed
  family $a$ of principal filters.
\end{prop}

\begin{proof}
Because $\prod_{i \in \dom a} (\uparrow^{\mathsf{RLD}})^{- 1} a_i
\in \up \prod^{\mathsf{RLD}} a$.
\fxwarning{More detailed proof.}
\end{proof}

\begin{lem}
$\prod^{\mathsf{RLD}}_{i \in \dom a} \Cor a_i = \Cor
\prod^{\mathsf{RLD}} a$.
\end{lem}

\begin{proof}
$\Cor \prod^{\mathsf{RLD}} a = \bigsqcap \left\{
\uparrow^{\mathsf{RLD}} \prod A \hspace{1em} | \hspace{1em} A \in
\up a \right\} = \uparrow^{\mathsf{RLD}} \bigcap \left\{ \prod A
\hspace{1em} | \hspace{1em} A \in \up a \right\} =
\uparrow^{\mathsf{RLD}} \bigcap \left\{ \prod A \hspace{1em} |
\hspace{1em} A \in \subsets \prod \mathfrak{U}, \forall i \in \dom a
: A_i \in \up a_i \right\} = \uparrow^{\mathsf{RLD}} \bigcap
\left\{ \prod \bigcap K_i \hspace{1em} | \hspace{1em} K \in \subsets
\subsets \prod \mathfrak{U}, \forall i \in \dom a : K_i \in
\subsets \up a_i \right\} = \uparrow^{\mathsf{RLD}} \bigcap
\left\{ \prod (\uparrow^{\mathsf{RLD}})^{- 1} \Cor a_i
\hspace{1em} | \hspace{1em} i \in \dom a \right\} =
\uparrow^{\mathsf{RLD}} \prod_{i \in \dom
a}^{\mathsf{RLD}} \Cor a_i$.

\fxwarning{Check for little errors.}
\end{proof}

\begin{cor}
  $\prod^{\mathsf{RLD}}_{i \in n} \langle \CoCompl f_i \rangle
  \mathcal{X}_i = \left\langle \CoCompl \prod^{(A)} f \right\rangle
  \prod^{\mathsf{RLD}} \mathcal{X}$ for every $n$-indexed families $f$
  of funcoids and $\mathcal{X}$ of filters on the same set (with $\Src
  f_i = \Base (\mathcal{X}_i)$ for every $i \in n$).
\end{cor}

\begin{proof}
  ~
  \begin{eqnarray*}
    \prod^{\mathsf{RLD}}_{i \in n} \langle \CoCompl f_i \rangle
    \mathcal{X}_i & = & \\
    \prod^{\mathsf{RLD}}_{i \in n} \Cor \langle f_i \rangle
    \mathcal{X}_i & = & \\
    \Cor \prod^{\mathsf{RLD}}_{i \in n} \langle f_i \rangle 
    \mathcal{X}_i & = & \text{(*)}\\
    \Cor \prod^{\mathsf{RLD}}_{i \in n} \langle f_i \rangle
    \Pr^{\mathsf{RLD}}_i \left( \prod^{\mathsf{RLD}} \mathcal{X}
    \right) & = & \\
    \Cor \left\langle \prod^{(A)} f \right\rangle
    \prod^{\mathsf{RLD}} \mathcal{X} & = & \\
    \left\langle \CoCompl \prod^{(A)} f \right\rangle
    \prod^{\mathsf{RLD}} \mathcal{X} . &  & 
  \end{eqnarray*}
  (*) You should verify the special case when $\mathcal{X}_i =
  \bot^{\mathfrak{F}}$ for some $i$.
\end{proof}

\begin{thm}
Let $f$ be an indexed family of funcoids. \fxnote{Reverse theorem (for non-least funcoids).}

\begin{enumerate}
  \item $\prod f$ is directly compact if every $f_i$ is directly compact.
  
  \item $\prod f$ is reversely compact if every $f_i$ is reversely compact.
  
  \item $\prod f$ is compact if every $f_i$ is compact.
\end{enumerate}
\end{thm}

\begin{proof}
  It is enough to prove only the first statement.
  
  Let each $f_i$ is directly compact.
  
  Let $\left\langle \prod f \right\rangle a \neq \bot$. Then $\left\langle \prod
  f \right\rangle a = \left\langle \prod^{(A)} f \right\rangle a =
  \prod^{\mathsf{RLD}}_{i \in \dom f} \langle f_i \rangle
  \Pr^{\mathsf{RLD}}_i a$. Thus every $\langle f_i \rangle
  \Pr^{\mathsf{RLD}}_i a \neq \bot$. Consequently by compactness
  $\Cor \langle f_i \rangle \Pr^{\mathsf{RLD}}_i a \neq \bot$;
  $\prod_{i \in \dom f} \Cor \langle f_i \rangle
  \Pr^{\mathsf{RLD}}_i a \neq \bot$; $\Cor \prod_{i \in \dom
  f} \langle f_i \rangle \Pr^{\mathsf{RLD}}_i a \neq \bot$; $\Cor
  \left\langle \prod f \right\rangle a \neq \bot$.
  
  So $\prod f$ is directly compact.
\end{proof}

\begin{prop}
  The following expressions are pairwise equal:
  \begin{enumerate}
    \item\label{ff-id-s} $\langle f \times^{(A)} f \rangle^{\ast} 1^{\mathsf{RLD}}$;
    
    \item\label{ff-id-at} $\bigsqcup \setcond{ \langle f \times^{(A)} f \rangle p }
    {p \in \atoms 1^{\mathsf{RLD}} }$;
    
    \item\label{ff-id-p} $\bigsqcup \setcond{ \supfun{f} x \times^{\mathsf{RLD}}
    \supfun{f} x }{ x \in \atoms^{\mathscr{F}} }$;
  \end{enumerate}
\end{prop}

\begin{proof}
~
\begin{widedisorder}
\item[\ref{ff-id-s}$\Leftrightarrow$\ref{ff-id-at}] Theorem~\bookref{fcd-atoms}.

\item[\ref{ff-id-at}$\Leftrightarrow$\ref{ff-id-p}]
$\bigsqcup \setcond{ \langle f \times^{(A)} f \rangle p }{p \in \atoms 1^{\mathsf{RLD}} } =
\bigsqcup \setcond{ \supfun{f}\dom p \times^{\mathsf{RLD}} \supfun{f}\im p }{p \in \atoms 1^{\mathsf{RLD}} } =
\bigsqcup \setcond{ \supfun{f} x \times^{\mathsf{RLD}} \supfun{f} x }{ x \in \atoms^{\mathscr{F}} }$.
\end{widedisorder}
\end{proof}

\begin{prop}\label{ff-ge-g}
  Let $g$ be a reloid and $f = \tofcd g$ and $f=f\circ f^{-1}$. Then $\langle f
  \times^{(A)} f \rangle^{\ast} 1^{\mathsf{RLD}} \sqsupseteq g$.
\end{prop}

\begin{proof}
  $\langle f \times^{(A)} f \rangle^{\ast} 1^{\mathsf{RLD}} \nasymp
  \uparrow^{\mathsf{RLD}} Y \Leftrightarrow
  \uparrow^{\mathsf{RLD}} 1^{\mathsf{RLD}} \mathrel{[f \times^{(A)} f]}
  \uparrow^{\mathsf{RLD}} Y \Leftrightarrow
  \uparrow^{\mathsf{FCD}} 1^{\mathsf{RLD}} \mathrel{[f \times^{(C)} f]}
  \uparrow^{\mathsf{FCD}} Y \Leftrightarrow f \circ
  \uparrow^{\mathsf{FCD}} 1^{\mathsf{RLD}} \circ f^{- 1} \nasymp
  \uparrow^{\mathsf{FCD}} Y \Leftrightarrow f \circ f^{- 1} \nasymp
  \uparrow^{\mathsf{FCD}} Y \Leftrightarrow f \nasymp
  \uparrow^{\mathsf{FCD}} Y \Leftrightarrow f \sqcap
  \uparrow^{\mathsf{FCD}} Y \neq \bot \Leftarrow
  \torldin (f \sqcap \uparrow^{\mathsf{FCD}}
  Y) \neq \bot \Leftrightarrow \torldin f \sqcap
  \torldin \uparrow^{\mathsf{FCD}} Y \neq \bot
  \Leftarrow \torldin f \sqcap
  \torldout \uparrow^{\mathsf{FCD}} Y \neq \bot
  \Leftrightarrow \torldin f \sqcap
  \uparrow^{\mathsf{RLD}} Y \neq \bot \Leftrightarrow
  \torldin  \tofcd g \sqcap
  \uparrow^{\mathsf{RLD}} Y \neq \bot \Leftarrow g \sqcap
  \uparrow^{\mathsf{RLD}} Y \neq \bot \Leftrightarrow g \nasymp
  \uparrow^{\mathsf{RLD}} Y$.
\end{proof}

\begin{prop}
  Let $g$ be a reloid and $f = \tofcd g$ and $f=f\circ f^{-1}$. Then
  $\rsupfun{f \times^{\mathrm{in}} f} 1^{\mathsf{RLD}} \sqsupseteq g$.
\end{prop}

\begin{proof}
$\rsupfun{f \times^{\mathrm{in}} f} 1^{\mathsf{RLD}} =
\rsupfun{\torldin f\circ^{(C)}\torldin f} 1^{\mathsf{RLD}} =
\torldin f\circ 1^{\mathsf{RLD}} \circ \torldin f^{-1} = 
\torldin f\circ \torldin f^{-1} = 
\torldin (f\circ f^{-1}) = \torldin f = \torldin\tofcd g \sqsupseteq g$.
\end{proof}

\begin{lem}
  $\Cor \langle f \times^{(A)} f \rangle^{\ast} g \sqsubseteq 1^{\mathsf{RLD}}$ if
  $\tofcd g = f$ where $\tofcd g = f$ for a
  $T_1$-separable reloid $g$.
\end{lem}

\begin{proof}
  ??
\end{proof}

\begin{rem}
  I attempted to generalize the below theorem more than the standard general
  topology theorem about correspondence of compact and uniform spaces, but
  haven't really succeeded much, as it appears to be needed that the reloid in
  question is reflexive, symmetric, and transitive, that is just a uniform
  space as in the standard general topology.
\end{rem}

Does the reverse inequality hold, that is $g \sqsupseteq \rsupfun{f \times^{(A)} f} 1^{\mathsf{RLD}}$
(for $f=\tofcd g$ and compact~$f$)?

It doesn't:

\begin{example}
$f=1^{\mathbb{R}|_{[0;1]}}$ (where $\mathbb{R}$ denotes usual proximity on real line),
$g=1^{\mathbb{R}|_{[0;1]}}$ (where $\mathbb{R}$ denotes usual uniformity on real line).
Then $\rsupfun{f \times^{(A)} f} x = \rsupfun{f}\dom x\times^{\mathsf{RLD}}\rsupfun{f}\im x =
\dom x\times^{\mathsf{RLD}}\im x\nsqsubseteq??\nsqsubseteq g$
for every nontrivial ultrafilter~$x\in\atoms^{\mathscr{F}[0;1]}$.
\end{example}

\begin{example}
$f=1^{\mathbb{R}|_{[0;1]}}$ (where $\mathbb{R}$ denotes usual proximity on real line),
$g=1^{\mathbb{R}|_{[0;1]}}$ (where $\mathbb{R}$ denotes usual uniformity on real line).
Then $\rsupfun{f \times^{\mathrm{in}} f} x =
\rsupfun{\torldin f \times^{(C)} \torldin f} x =
\torldin f\circ x\circ\torldin f^{-1} =
??$
for every nontrivial ultrafilter~$x\in\atoms^{\mathscr{F}[0;1]}$.
\end{example}

An (incomplete) attempt to prove one more theorem follows:

\begin{thm}
  Let $\mu$ and $\nu$ be uniform spaces, $\tofcd
  \mu$ be a compact funcoid. Then a map $f$ is a continuous map from
  $\tofcd \mu$ to $\tofcd \nu$ iff $f$ is
  a (uniformly) continuous map from $\mu$ to $\nu$.
\end{thm}

\begin{proof}
\fxerror{errors in this proof.}

http://math.stackexchange.com/questions/665202/bourbaki-on-the-fact-that-continuous-function-on-a-compact-is-uniformly-continuo/670956?iemail=1\&noredirect=1\#670956

We have $\mu= \langle \tofcd \mu \times
\tofcd \mu \rangle \uparrow^{\mathsf{RLD}} 1^{\mathsf{RLD}}$

$f \in \mathrm{C}_? (\tofcd \mu; \tofcd
\nu)$. Then
\[ f \times^{(A)} f \in \mathrm{C}_? (\tofcd (\mu \times^{(A)}
   \mu) ; \tofcd (\nu \times^{(A)} \nu)) \]
$(f \times^{(A)} f) \circ \tofcd (\mu \times^{(A)} \mu)
\sqsubseteq \tofcd (\nu \times^{(A)} \nu) \circ (f \times^{(A)} f)$

For every $V \in \up (\nu \times^{(A)} \nu)$ we have $\langle g^{- 1} \rangle
V \in \langle \tofcd (\mu \times^{(A)} \mu) \rangle
\{ y \}$ for some $y$.

$\langle g^{- 1} \rangle V \in \langle \tofcd \mu \times^{(A)}
\tofcd \mu \rangle \uparrow^{\mathsf{RLD}} 1^{\mathsf{RLD}}
= \up \mu$

$\supfun{g} \langle g^{- 1} \rangle V \sqsubseteq V$

We need to prove $f \in \mathrm{C} (\mu; \nu)$ that is $\forall p \in
\up \nu \exists q \in \up \mu: \supfun{f} q
\sqsubseteq p$. But this follows from the above.
\end{proof}

\fxnote{A space is compact if and only if it is both, complete and totally bounded.}

\url{http://math.stackexchange.com/questions/1101995/non-symmetric-version-of-compact-totally-bounded-complete}
